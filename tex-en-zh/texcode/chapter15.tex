\chapter{本章内容}

\section{本节内容}

\begin{paracol}{2}

History can be understood in terms of what might have been as well as what was. We know of no reason that microparasites equid not have continued to play havoc with human society during the modem-period. For example, it is possible that microbiological barriers to the exercise of power, equivalent to malaria but more virulent, could have halted the Western invasion of the periphery in its tracks. The first intrepid Portuguese adventurers who sailed into African waters could have contracted a deadly retrovirus, a more communicable version of AIDS, that would have stopped the opening of the new trade route to Asia before it even began. Columbus, too, and the first waves of settlers in the New World might have encountered diseases that decimated them in the same way that indigenous local populations were affected by measles and other Western childhood diseases. Yet nothing of the kind happened, a coincidence that underlines the intuition that history has a destiny.

\switchcolumn
历史可以通过可能发生的事情以及发生的事情来理解。我们不知道为什么微生物寄生体不能在现代时期继续对人类社会造成伤害。例如,等同于疟疾但更具致命性的微生物障碍可能会阻止西方侵略边缘。第一个翻越非洲水域的葡萄牙冒险家可能感染了一种致命的反转录病毒,一种更具传染性的艾滋病病毒,这会在开辟前往亚洲的新贸易路线之前就阻止了这一过程。哥伦布和新世界的第一波移民也可能遭遇到像本土的人口一样受麻疹和其他西方儿童疾病影响的疾病。然而,没有发生任何这样的事情,这种巧合强调了历史有一个命运的直觉。

\switchcolumn*
Microbes did far less to impede the consolidation of power in the modem period than to facilitate it. Western troops and colonists at the periphery often found that the technological advantages that allowed them to project power were underscored by microbiological ones. Westerners were armed with unseen biological weapons, their relative immunity to childhood diseases that frequently devastated native peoples. This gave voyagers from the West a distinct advantage that their antagonists from less densely settled regions lacked. As events unfolded, the disease transfer was almost entirely in one direction --from Europe outward. There was no equivalent transfer of disease in the other direction, from the periphery to the core. 

\switchcolumn
微生物并不像现代时期那样阻碍了权力的巩固,而是有利于它。在边缘地区的西方部队和殖民者经常发现,使他们有能力施展影响的技术优势得到了微生物方面的加强。西方人武装了无形的生物武器,在童年时期经常摧毁当地人的疾病上则相对免疫。这给西方的航海者带来了与来自人口稀少地区的对手所缺乏的明显优势。随着事态的发展,疾病传播几乎完全呈单向性——从欧洲向外传播。没有等价的疾病从边缘地区向核心地区传播。

\end{paracol}