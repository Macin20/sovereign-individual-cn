\chapter[人类社会第四个阶段]{2000年转折: \\ 人类社会的第四阶段}

\section[预言]{PREMONITIONS\\ 预言}

\begin{paracol}{2}[]
The coming of the year 2000 has haunted the Western imagination for the past thousand years. Ever since the world failed to end at the turn of the first millennium after Christ, theologians, evangelists, poets, seers, and now, even computer programmers have looked to the end of this decade with an expectation that it would bring something momentous. No less an authority than Isaac Newton speculated that the world would end with the year 2000. Michel de Nostradamus, whose prophecies have been read by every generation since they were first published in 1568, forecast the coming of the Third Antichrist in July 1999. Swiss psychologist Carl Jung, connoisseur of the “collective unconscious,” envisioned the birth of a New Age in 1997. Such forecasts may easily be ridiculed. And so can the sober forecasts of economists, such as Dr. Edward Yardeni of Deutsche Bank Securities, who expects computer malfunctions on the millennial midnight to “disrupt the entire global economy.” But whether you view the Y2K computer problem as groundless hysteria ginned up by computer programmers and Information Technology consultants to stir up business, or as a mysterious instance of technology unfolding in concert with the prophetic imagination, there is no denying that circumstances at the eve of the millennium excite more than the usual morbid doubt about where the world is tending.

\switchcolumn
在过去的一千年里,2000年的到来一直困扰着西方人的想象力。自从世界在基督之后的第一个千年之交未能结束以来,神学家、传道者、诗人、先知,甚至现在,甚至计算机程序员也期待着这个十年的结束,期望它会带来一些重大的东西。不亚于艾萨克·牛顿(Isaac Newton)推测世界将在2000年结束的权威。米歇尔·德·诺查丹玛斯(Michel de Nostradamus)的预言自1568年首次出版以来,每一代人都在阅读,他预言了1999年7月第三敌基督者的到来。瑞士心理学家卡尔·荣格(Carl Jung)是“集体无意识”的鉴赏家,他在1997年设想了新时代的诞生。这样的预测可能很容易被嘲笑。经济学家的清醒预测也是如此,例如德意志银行证券(Deutsche Bank Securities)的爱德华·亚德尼(Edward Yardeni)博士,他预计千禧年午夜的计算机故障将“扰乱整个全球经济”。但是,无论你是否将Y2K计算机问题视为计算机程序员和信息技术顾问为搅动业务而制造的毫无根据的歇斯底里,或者作为技术与预言想象力一起展开的一个神秘例子,不可否认的是,千禧年前夕的情况比通常对世界走向何方的病态怀疑更令人兴奋。
\end{paracol}
\begin{paracol}{2}[]
A sense of disquiet about the future has begun to color the optimism so characteristic of Western societies for the past 250 years. People everywhere are hesitant and worried. You see it in their faces. Hear it in their conversation. See it reflected in polls and registered in the ballot box. Just as an invisible, physical change of ions in the atmosphere signals that a thunderstorm is imminent even before the clouds darken and lightning strikes, so now, in the twilight of the millennium, premonitions of change are in the air. One person after another, each in his own way, senses that time is running out on a dying way of life. As the decade expires, a murderous century expires with it, and also a glorious millennium of human accomplishment. All draw to a close with the year 2000.
\begin{tcolorbox}
For there is nothing covered that shall not be revealed, neither hid that shall not be known. \\ 
\begin{flushright}
——MATTHEW 10.26
\end{flushright}
\end{tcolorbox}
\switchcolumn
对未来的不安感已经开始为过去250年来西方社会特有的乐观情绪着色。世界各地的人们都在犹豫和担心。你可以在他们的脸上看到它。在他们的谈话中听到它。看到它反映在民意调查中并在投票箱中登记。正如大气中无形的物理变化甚至在云层变暗和闪电袭击之前就预示着雷暴即将来临,所以现在,在千年的黄昏,空气中弥漫着变化的预感。一个又一个人,每个人都以自己的方式,感觉到时间在垂死的生活方式上已经不多了。随着十年的过去,一个杀戮的世纪也随之结束,也是人类成就的光荣千年。所有这些都在2000年结束。
\begin{tcolorbox}
因为没有什么是不能透露的,也没有隐藏的,是不应该知道的。
\begin{flushright}
—— 马太福音10.26
\end{flushright}
\end{tcolorbox}
\end{paracol}


\begin{paracol}{2}[]
We believe that the modern phase of Western civilization will end with it. This book tells why. Like many earlier works, it is an attempt to see into a glass darkly, to sketch out the vague shapes and dimensions of a future that is still to be. In that sense, we mean our work to be apocalyptic—in the original meaning of the word. Apokalypsis means “unveiling” in Greek. We believe that a new stage in history—the Information Age—is about to be “unveiled.”
\switchcolumn
我们相信,西方文明的现代阶段将随之结束。这本书讲述了原因。像许多早期的作品一样,它试图黑暗地看到玻璃,勾勒出未来仍然存在的模糊形状和维度。从这个意义上说,我们的意思是我们的工作是世界末日的——在这个词的原始含义中。Apokalypsis在希腊语中的意思是“揭幕”。我们相信,历史的新阶段——信息时代——即将“揭开面纱”。
\end{paracol}

\begin{paracol}{2}[]
\begin{tcolorbox}
We are watching the beginnings of a new logical space, an instantaneous electronic everywhereness, which we may all access, enter into, and experience. We have, in short, the beginnings of a new kind of community. The virtual community becomes the model for a secular Kingdom of Heaven; as Jesus said there were many mansions in his Father’s Kingdom, so there are many virtual communities, each reflecting their own needs and desires.
\begin{flushright}
——MICHAEL GRASSO
\end{flushright}
\end{tcolorbox}
\switchcolumn
\begin{tcolorbox}
我们正在观察一个新的逻辑空间的开始,一个瞬间的电子无处不在,我们都可以访问,进入和体验。简而言之,我们已经有了一种新的社区的开端。虚拟社区成为世俗
天国的典范;正如耶稣所说,在他父的王国里有许多豪宅,所以有许多虚拟社区,每个社区都反映了他们自己的需要和愿望。
\begin{flushright}
—— 迈克尔·格拉索
\end{flushright}
\end{tcolorbox}
\end{paracol}


\section[人类社会的第四阶段]{THE FOURTH STAGE OF HUMAN SOCIETY \\ 人类社会的第四阶段}
\begin{paracol}{2}[]
The theme of this book is the new revolution of power which is liberating individuals at the expense of the twentieth-century nation-state. Innovations that alter the logic of violence in unprecedented ways are transforming the boundaries within which the future must lie. If our deductions are correct, you stand at the threshold of the most sweeping revolution in history. Faster than all but a few now imagine, microprocessing will subvert and destroy the nation-state, creating new forms of social organization in the process. This will be far from an easy transformation.
\switchcolumn
本书的主题是权力的新革命,它以牺牲二十世纪的民族国家为代价解放个人。以前所未有的方式改变暴力逻辑的创新正在改变未来必须存在的界限。如果我们的推论是正确的,那么你就站在了历史上最彻底的革命的门槛上。微处理的速度比现在除了少数人想象的要快,它将颠覆和摧毁民族国家,在此过程中创造新的社会组织形式。这远非易事的转变。
\end{paracol}

\begin{paracol}{2}[]
The challenge it will pose will be all the greater because it will happen with incredible speed compared with anything seen in the past. Through all of human history from its earliest beginnings until now, there have been only three basic stages of economic life: (1) hunting-and-gathering societies; (2) agricultural societies; and (3) industrial societies. Now, looming over the horizon, is something entirely new, the fourth stage of social organization: information societies.
\switchcolumn
它将带来的挑战将更大,因为与过去看到的任何东西相比,它将以令人难以置信的速度发生。纵观人类历史,从最早的开始到现在,经济生活只有三个基本阶
段:(1)狩猎和采集社会;(2)农业社会;(3)工业社会。现在,迫在眉睫的是某种全新的社会组织阶段:信息社会。
\end{paracol}

\begin{paracol}{2}[]
Each of the previous stages of society has corresponded with distinctly different phases in the evolution and control of violence. As we explain in detail, information societies promise to dramatically reduce the returns to violence, in part because they transcend locality. The virtual reality of cyberspace, what novelist William Gibson characterized as a “consensual hallucination,” will be as far beyond the reach of bullies as imagination can take it. In the new
millennium, the advantage of controlling violence on a large scale will be far lower than it has been at any time since before the French Revolution. This will have profound consequences. One of these will be rising crime. When the payoff for organizing violence at a large scale tumbles, the payoff from violence at a smaller scale is likely to jump. Violence will become more random and localized. Organized crime will grow in scope. We explain why.
\switchcolumn
社会以前的每个阶段都与暴力演变和控制的明显不同阶段相对应。正如我们详细 解释的那样,信息社会有望大幅减少暴力的回归,部分原因是它们超越了地方性。小说家威廉·吉布森(William Gibson)称之为“自愿幻觉”的网络空间的虚拟现实将远远超出欺凌者的想象范围。在新的千年里,大规模控制暴力的优势将远远低于法国大革命前的任何时候。这将产生深远的影响。其中之一将是犯罪率上升。当大规模组织暴力的回报下降时,小规模暴力的回报可能会增加。暴力将变得更加随机和局部化。有组织犯罪的范围将会扩大。我们解释原因。
\end{paracol}

\begin{paracol}{2}[]
Another logical implication of falling returns to violence is the eclipse of politics, which is the stage for crime on the largest scale. There is much evidence that adherence to the civic myths of the twentieth-century nation-state is rapidly eroding. The death of Communism is merely the most striking example. As we explore in detail, the collapse of morality and growing corruption among leaders of Western governments are not random developments. They are evidence that the potential of the nation-state is exhausted. Even many of its leaders no longer believe the platitudes they mouth. Nor are they believed by others.
\switchcolumn
暴力回归下降的另一个逻辑含义是政治的黯然失色,这是最大规模犯罪的舞台。有很多证据表明,对二十世纪民族国家的公民神话的坚持正在迅速侵蚀。共产主义的死亡只是最引人注目的例子。正如我们详细探讨的那样,西方政府领导人道德的崩溃和日益严重的腐败并不是随机的发展。它们证明民族国家的潜力已经耗尽。甚至许多领导人也不再相信他们所说的陈词滥调。他们也不被其他人相信。
\end{paracol}

\subsection[历史重演]{History Repeats Itself \\历史重演}
\begin{paracol}{2}[]
This is a situation with striking parallels in the past. Whenever technological change has divorced the old forms from the new moving forces of the economy, moral standards shift, and people begin to treat those in command of the old institutions with growing disdain. This widespread revulsion often comes into evidence well before people develop a new coherent ideology of change. So it was in the late fifteenth century, when the medieval Church was the predominant institution of feudalism. Notwithstanding popular belief in “the sacredness of the sacerdotal office,” both the higher and lower ranks of clergy were held in the utmost contempt—not unlike the popular attitude toward politicians and bureaucrats today.
\switchcolumn
这种情况在过去有着惊人的相似之处。每当技术变革将旧形式与新的经济动力分离时,道德标准就会发生变化,人们开始越来越蔑视那些控制旧机构的人。这种普遍的厌恶往往在人们形成一种新的连贯的变革意识形态之前就已经显现出来了。因此,在十五世纪后期,中世纪教会是封建主义的主要机构。尽管人们普遍相信“神圣职位的神圣性”,但神职人员的上级和下级都受到了极大的蔑视——这与今天对政治家和官僚的普遍态度没有什么不同。
\end{paracol}

\begin{paracol}{2}[]
We believe that much can be learned by analogy between the situation at the end of the fifteenth century, when life had become thoroughly saturated by organized religion, and the situation today, when the world has become saturated with politics. The costs of supporting institutionalized religion at the end of the fifteenth century had reached a historic extreme, much as the costs of supporting government have reached a senile extreme today.
\switchcolumn
我们认为,通过类比15世纪末的生活被有组织的宗教彻底浸透的情况与今天世界 政治饱和的情况,可以学到很多东西。十五世纪末支持制度化宗教的成本已经达到了历史的极端,就像支持政府的成本今天达到了衰老的极端一样。
\end{paracol}

\begin{paracol}{2}[]
We know what happened to organized religion in the wake of the Gunpowder Revolution. Technological developments created strong incentives to downsize religious institutions and lower their costs. A similar technological revolution is destined to downsize radically the nation-state early in the new millennium.
\switchcolumn
我们知道在火药革命之后,有组织的宗教发生了什么。技术发展为缩小宗教机构规模和降低成本创造了强烈的动力。类似的技术革命注定要在新千年初期从根本上缩小民族国家的规模。
\end{paracol}

\begin{paracol}{2}[]
\begin{tcolorbox}
Today, after more than a century of electric technology, we have extended our central nervous system itself in a global embrace, abolishing both space and time as far as our planet is concerned
\begin{flushright}
——MARSHALL McLUHAN, 1964
\end{flushright}
\end{tcolorbox}
\switchcolumn
\begin{tcolorbox}
今天,经过一个多世纪的电气技术,我们已经在全球范围内扩展了我们的中枢神经系统本身,就我们的星球而言,废除了空间和时间
\begin{flushright}
——马歇尔·麦克卢汉,1964
\end{flushright}
\end{tcolorbox}
\end{paracol}


\subsection[信息革命]{The Information Revolution \\信息革命}

\begin{paracol}{2}[]
As the breakdown of large systems accelerates, systematic compulsion will recede as a factor shaping economic life and the distribution of income. Efficiency will become more important than the dictates of power in the organization of social institutions. This means that provinces and even cities that can effectively uphold property rights and provide for the administration of justice, while consuming few resources, will be viable sovereignties in the Information Age, as they generally have not been during the last five centuries. An entirely new realm of economic activity that is not hostage to physical violence will emerge in cyberspace. The most obvious benefits will flow to the “cognitive elite,” who will increasingly operate outside political boundaries. They are already equally at home in Frankfurt, London, New York, Buenos Aires, Los Angeles, Tokyo, and Hong Kong. Incomes will become more unequal
within jurisdictions and more equal between them.
\switchcolumn
随着大型系统的加速崩溃,系统性强迫作为影响经济生活和收入分配的一个因素将会消退。在社会机构的组织中,效率将变得比权力的要求更重要。这意味着,能够有效维护财产权和提供司法行政的省份甚至城市,同时消耗的资源很少,在信息时代将是可行的主权,就像过去五个世纪以来通常没有的那样。一个不受身体暴力影响的经济活动全新领域将在网络空间出现。最明显的好处将流向“认知精英”,他们将越来越多地在政治边界之外运作。他们已经在法兰克福、伦敦、纽约、布宜诺斯艾利斯、洛杉矶、东京和香港同样自在。辖区内的收入将变得更加不平等,它们之间的收入将变得更加平等。
\end{paracol}

\begin{paracol}{2}[]
The Sovereign Individual explores the social and financial consequences of this revolutionary change. Our desire is to help you to take advantage of the opportunities of the new age and avoid being destroyed by its impact. If only half of what we expect to see happens, you face change of a magnitude with few precedents in history.
\switchcolumn
《主权个人》探讨了这一革命性变化的社会和财务后果。我们的愿望是帮助您利用新时代的机遇,避免被其影响所摧毁。如果我们期望看到的只有一半发生,你将面临历史上很少有先例的巨大变化。
\end{paracol}

\begin{paracol}{2}[]
The transformation of the year 2000 will not only revolutionize the character of the world economy, it will do so more rapidly than any previous phase change. Unlike the Agricultural Revolution, the Information Revolution will not take millennia to do its work. Unlike the Industrial Revolution, its impact will not be spread over centuries. The Information Revolution will happen within a lifetime.
\switchcolumn
2000年的转变不仅将彻底改变世界经济的性质,而且将比以往任何阶段性变化都快。与农业革命不同,信息革命不需要几千年才能完成其工作。与工业革命不同,它的影响不会延续几个世纪。信息革命将在有生之年发生。
\end{paracol}


\begin{paracol}{2}[]
What is more, it will happen almost everywhere at once. Technical and economic innovations will no longer be confined to small portions of the globe. The transformation will be all but universal. And it will involve a break with the past so profound that it will almost bring to life the magical domain of the gods as imagined by the early agricultural peoples like the ancient Greeks. To a greater degree than most would now be willing to concede, it will prove difficult or impossible to preserve many contemporary institutions in the new millennium. When information societies take shape they will be as different from industrial societies as the Greece of Aeschylus was from the world of the cave dwellers.
\switchcolumn
更重要的是,它几乎会同时发生在任何地方。技术和经济创新将不再局限于全球的一小部分。这种转变几乎是普遍的。它将涉及与如此深刻的过去决裂,以至于它几乎将像古希腊人这样的早期农业民族所想象的那样,使神的神奇领域栩栩如生。在比大多数人现在愿意承认的更大程度上,在新千年中保留许多当代机构将证明是困难或不可能的。当信息社会形成时,它们将与工业社会不同,就像埃斯库罗斯的希腊与穴居者的世界一样。
\end{paracol}

\section[主权个体的崛起]{PROMETHEUS UNBOUND: THE RISE OF THE SOVEREIGN INDIVIDUAL\\普罗米修斯不受束缚:主权个体的崛起}
\begin{paracol}{2}[]

\begin{tcolorbox}
I know of no more encouraging fact than the unquestionable ability of man to elevate his life by conscious endeavor.
\begin{flushright}
—— HENRY DAVID THOREAU
\end{flushright}
\end{tcolorbox}

The coming transformation is both good news and bad. The good news is that the Information Revolution will liberate individuals as never before. For the first time, those who can educate and motivate themselves will be almost entirely free to invent their own work and realize the full benefits of their own productivity. Genius will be unleashed, freed from both the oppression of government and the drags of racial and ethnic prejudice. In the Information Society, no one who is truly able will be detained by the ill-formed opinions of others. It will not matter what most of the people on earth might think of your race, your looks, your age, your sexual proclivities, or the way you wear your hair. In the cybereconomy, they will never see you. The ugly, the fat, the old, the disabled will vie with the young and beautiful on equal terms in utterly color-blind anonymity on the new frontiers of cyberspace.
\switchcolumn

\begin{tcolorbox}
我知道没有比人类通过有意识的努力来提升自己生活的能力更令人鼓舞的事实了。
\begin{flushright}
—— 亨利·大卫·梭罗
\end{flushright}
\end{tcolorbox}

即将到来的转变既有好消息也有坏消息。好消息是信息革命将像从未有过的那样解放个人。首次,那些能够自我教育和激励自己的人将几乎完全自由地发明自己的工作,并实现自己生产力的全部好处。天才将被释放,摆脱了政府的压迫和种族和民族偏见的限制。在信息社会中,没有真正有才华的人会被他人没见过的肤浅见解所阻拦。你的种族、外貌、年龄、性取向或发型方式将无关紧要。在网络经济中,别人永远看不见你。在新的网络空间上,丑陋、肥胖、老年和残疾的人将与年轻和美丽的人平等竞争,实现完全无色觉偏见的匿名。
\end{paracol}

\subsection{思想成为财富}
\begin{paracol}{2}[]
Merit, wherever it arises, will be rewarded as never before. In an environment where the greatest source of wealth will be the ideas you have in your head rather than physical capital alone, anyone who thinks clearly will potentially be rich. The Information Age will be the age of upward mobility. It will afford far more equal opportunity for the billions of humans in parts of the world that never shared fully in the prosperity of industrial society. The brightest, most successful and ambitious of these will emerge as truly Sovereign Individuals. 
\switchcolumn
无论何时何地,凡是有卓越思想的人都将得到前所未有的奖励。在一个最大的财富资源是你脑中的思想而不仅仅是物质资本的环境中,任何能够清晰思考的人都可能富有。信息时代将是流动性的年代。它将为数十亿生活在未曾共享工业社会繁荣的地区的人提供更多平等机会。这些人中最聪明、最成功和最有抱负的人将成为真正的独立个体。
\switchcolumn*
At first, only a handful will achieve full financial sovereignty. But this does not negate the advantages of financial independence. The fact that not everyone attains an equally vast fortune does not mean that it is futile or meaningless to become rich. There are 25,000 millionaires for every billionaire. If you are a millionaire and not a billionaire, that does not make you poor. Equally, in the future, one of the milestones by which you measure your financial success will be not just now many zeroes you can add to your net worth, but whether you can structure your affairs in a way that enables you to realize full individual autonomy and independence. The more clever you are, the less propulsion you will require to achieve financial escape velocity. Persons of even quite modest means will soar as the gravitational pull of politics on the global economy weakens. Unprecedented financial independence will be a reachable goal in your lifetime or that of your children.
\switchcolumn
一开始,只有少数人能实现完全的财务主权。但这并不否定财务独立的优势。并不是每个人都能获得同样巨大的财富,这并不意味着成为富有是毫无意义的。每个亿万富翁都有25000个百万富翁。如果你是百万富翁而不是亿万富翁,那也不代表你是穷人。同样,在未来,衡量你的财务成功的一个里程碑不仅仅是你的净值上有多少个零,而是你能否以一种可以实现个人完全自治和独立的方式来构建你的事务。越聪明的你,就越不需要推动力来实现财务逃逸速度。即使是非常普通的人也可以在全球政治重力对全球经济的影响减弱之际腾飞。在你或你的子孙一生中,无先例的财务独立将成为一个可以实现的目标。
\switchcolumn*
At the highest plateau of productivity, these Sovereign Individuals will compete and interact on terms that echo the relations among the gods in Greek myth. The elusive Mount Olympus of the next millennium will be in cyberspace -- a realm without physical existence that will nonetheless develop what promises to be the world's largest economy by the second decade of the new millennium. By 2025, the cybereconomy will have many millions of participants. Some of them will be as rich as Bill Gates, worth tens of billions of dollars each. The cyberpoor may be those with an income of less than \$200,000 a year. There will be no cyberwelfare. No cybertaxes and no cybergovernment. The cybereconomy, rather than China, could well be the greatest economic phenomenon of the next thirty years.
\switchcolumn
在生产力的最高高原上,这些主权个体将在类似于希腊神话中神之间的关系的条件下竞争和互动。下一个千禧年的神山将是虚拟空间——一个没有实体存在的领域,但它发展成为新千年二十年代世界上最大的经济体之一。到2025年,虚拟经济将有很多百万参与者。其中一些人将像比尔·盖茨一样富有,价值数百亿美元。虚拟贫穷可能是年收入不到20万美元的人。没有虚拟福利。没有虚拟税收,也没有虚拟政府。虚拟经济可能成为未来30年最伟大的经济现象,而不是中国。
\switchcolumn*
The good news is that politicians will no more be able to dominate, suppress, and regulate the greater part of commerce in this new realm than the legislators of the ancient Greek city-states could have trimmed the beard of Zeus. That is good news for the rich. And even better news for the not so rich. The obstacles and burdens that politics imposes are more obstacles to becoming rich than to being rich. The benefits of declining returns to violence and devolving jurisdictions will create scope for every energetic and ambitious person to benefit from the death of politics. Even the consumers of government services will benefit as entrepreneurs extend the benefits of competition. Heretofore, competition between jurisdictions has usually meant competition by means of violence to enforce the rule of a predominant group. Consequently, much of the ingenuity of interjurisdictional competition was channeled into military endeavor. But the advent of the cybereconomy will bring competition on new terms to provision of sovereignty services. A proliferation of jurisdictions will mean proliferating experimentation in new ways of enforcing contracts and otherwise securing the safety of persons and property. The liberation of a large part of the global economy from political control will oblige whatever remains of government as we have known it to operate on more nearly market terms. Governments will ultimately have little choice but to treat populations in territories they serve more like customers, and less in the way that organized criminals treat the victims of a shakedown racket.
\switchcolumn
好消息是政治家将不再能够在这个新领域中支配、压制和规范大部分商业活动,就像古希腊城邦的立法者不能修剪宙斯的胡须一样。这对富人来说是好消息。对于不那么富裕的人来说更是好消息。政治施加的障碍和负担对于成为富人来说是更多的障碍,而对于已经富足的人来说则更少。暴力收益递减和权力下放的益处将为每个有活力和雄心壮志的人创造发挥,从而从政治的消亡中受益。即使是政府服务的消费者也会受益,因为企业家会扩大竞争的好处。迄今为止,司法管辖区之间的竞争通常意味着通过暴力竞争来强制执行主导群体的规则。因此,许多跨领土竞争的独创性都集中在军事事业上。但是,网络经济的出现将为主权服务的供给带来新的竞争条件。司法管辖区的繁殖将意味着在新的执行合同和保障人身和财产安全的方式方面的多种多样的实验。全球经济的大部分解放出了政治控制,这将迫使我们所知道的政府在更接近市场原则的条件下运作。政府最终将别无选择,只能把他们服务的地区人口视为顾客,而不是像有组织的犯罪分子对待勒索诈骗受害者一样。
\end{paracol}

\subsection{超越政治}
\begin{paracol}{2}[]
What mythology described as the province of the gods will become a viable option for the individual -- a life outside the reach of kings and councils. First in scores, then in hundreds, and ultimately in the millions, individuals will escape the shackles of politics. As they do, they will transform the character of governments, shrinking the realm of compulsion and widening the scope of private control over resources.
\switchcolumn
神话描述的神的领域将成为个人的可行选项 - 生活在国王和议会无法触及的生活之外。从成百上千开始,最终达到数百万,个人将逃脱政治的枷锁。他们这样做,将改变政府的性质,缩小强制的范围,扩大对资源的私人控制范围。
\switchcolumn*
The emergence of the sovereign individual will demonstrate yet again the strange prophetic power of myth. Conceiving little of the laws of nature, the early agricultural peoples imagined that ``powers we should call supernatural'' were widely distributed. These powers were sometimes employed by men, sometimes by ``incarnate human gods'' who looked like men and interacted with them in what Sir James George Frazer described in \emph{The Golden Bough} as ``a great democracy.''
\switchcolumn
个人主权的出现将再次证明神话的奇异预言能力。早期的农业民族很少了解自然法则,他们认为“我们应该称之为超自然的力量”是广泛分布的。这些力量有时被人类利用,有时由“人格化的人类神”利用,他们看起来像人类,并与他们互动在詹姆斯·乔治·弗雷泽在《金枝》中所描述的“一个伟大的民主制”中。
\switchcolumn*
When the ancients imagined the children of Zeus living among them they were inspired by a deep belief in magic. They shared with other primitive agricultural peoples an awe of nature, and a superstitious conviction that nature's works were set in motion by individual volition, by magic. In that sense, there was nothing self-consciously prophetic about their view of nature and their gods. They were far from anticipating microtechnology. They could not have imagined its impact in altering the marginal productivity of individuals thousands of years later. They certainly could not have foreseen how it would shift the balance between power and efficiency and thus revolutionize the way that assets are created and protected. Yet what they imagined as they spun their myths has a strange resonance with the world you are likely to see.
\switchcolumn
当古人想象宙斯的子女与他们一起生活时,他们受到了对魔法的深刻信仰的启发。他们与其他原始农业民族分享对自然的敬畏,以及通过个体意志的魔法来控制自然力量的迷信信念。从这个意义上说,他们对自然和他们的神并没有什么自觉的预言性。他们远未预见到微技术的到来。数千年后,他们也无法想象它对个人边际生产力的改变会对生产资产和保护方式的变革带来多大的影响。然而,当他们编织神话时所想象的,却与你可能看到的世界有着奇怪的共鸣。
\end{paracol}

\subsection{Abracadabra咒语}
\begin{paracol}{2}[]
The "abracadabra" of the magic invocation, for example, bears a curious similarity to the password employed to access a computer. In some respects, high-speed computation has already made it possible to mimic the magic of the genie. Early generations of "digital servants" already obey the commands of those who control the computers in which they are sealed much as genies were sealed in magic lamps. The virtual reality of information technology will widen the realm of human wishes to make almost anything that can be imagined seem real. Telepresence will give living individuals the same capacity to span distance at supernatural speed and monitor events from afar that the Greeks supposed was enjoyed by Hermes and Apollo. The Sovereign Individuals of the Information Age, like the gods of ancient and primitive myths, will in due course enjoy a kind of ``diplomatic immunity'' from most of the political woes that have beset mortal human beings in most times and places.
\switchcolumn
魔法咒语中的“阿布拉卡达布拉”与访问电脑的密码惊人地相似。高速计算在某些方面已经让模仿神灵魔法成为可能。早期的“数字仆人”就像法器中被封印的神灵一样服从主人的命令。信息技术的虚拟现实将扩大人类的愿望范围,使几乎任何想象得到的事情都变得真实。远程存在将赋予生命体在超自然速度下跨越距离和远程监控事件的能力,就像希腊神话中的赫尔墨斯和阿波罗一样。信息时代的主权个体,像古代和原始神话中的神灵一样,最终将享有一种“外交豁免权”,使其免于大多数时代和地方困扰凡人的政治问题。
\switchcolumn*
The new Sovereign Individual will operate like the gods of myth in the same physical environment as the ordinary, subject citizen, but in a separate realm politically. Commanding vastly greater resources and beyond the reach of many forms of compulsion, the Sovereign Individual will redesign governments and reconfigure economies in the new millennium. The full implications of this change are all but unimaginable.
\switchcolumn
新的主权个体将在同一物理环境中与普通公民生活,但在政治上处于单独的领域。拥有非常庞大的资源,超出多种形式约束的范围,主权个体将重新设计政府和经济,进入新的千年。这种变化的全部影响几乎无法想象。
\end{paracol}

\subsection{天才与天惩}
\begin{paracol}{2}[]
For anyone who loves human aspiration and success, the Information Age will provide a bounty. That is surely the best news in many generations. But it is bad news as well. The new organization of society implied by the triumph of individual autonomy and the true equalization of opportunity based upon merit will lead to very great rewards for merit and great individual autonomy. This will leave individuals far more responsible for themselves than they have been accustomed to being during the industrial period. It will also precipitate transition crises, including a possibly severe economic depression that will reduce the unearned advantage in living standards that has been enjoyed by residents of advanced industrial societies throughout the twentieth century.  As we write, the top 15 percent of the world's population have an average per-capita income of \$21,000 annually. The remaining 85 percent of the world have an average income of just \$1,000. That huge, hoarded advantage from the past is bound to dissipate under the new conditions of the Information Age. 
\switchcolumn
对于任何一个追逐理想和成功的人来说,信息时代的回报将无与伦比。这无疑是几代人以来最好的消息,但也是一个坏消息。基于个人自治的新型社会组织,以及建立在能力之上的、真正的机会均等,会使才能出众者,得到超级的回报和个人自主性。但是,个人要对自己担负的责任,也会远远超过他们在工业时期所习惯的。此外,在整个20世纪,先进工业社会的居民,享受了不劳而获的优越生活,这种优势也将被削弱。在我们写这本书的时候(1997年之前),世界上前15\%的人口,人均年收入为21000美元;其余85\%的人,平均年收入只有1000美元。在信息时代的新环境下,过去囤积起来的巨大优势,必将烟消云散。
\switchcolumn*
As it does, the capacity of nation-states to redistribute income on a large scale will collapse. Information technology facilitates dramatically increased competition between jurisdictions. When technology is mobile, and transactions occur in cyberspace, as they increasingly will do, governments will no longer be able to charge more for their services than they are worth to the people who pay for them. Anyone with a portable computer and a satellite link will be able to conduct almost any information business anywhere, and that includes almost the whole of the world's multitrillion-dollar financial transactions. 
\switchcolumn
随着它的消散,民族国家大规模重新分配收入的能力将崩溃。信息技术极大地促进了辖区之间的竞争。信息技术加剧了各管辖区之间的竞争。技术是流动的,交易是在网络空间进行的。任何人只要有一台便携式电脑,和一条卫星网络,就可以在任何地方,从事几乎任何信息业务,包括世界上数以万亿美元的金融交易。
\switchcolumn*
This means that you will no longer be obliged to live in a high-tax jurisdiction in order to earn high income. In the future, when most wealth can be earned anywhere, and even spent anywhere, governments that attempt to charge too much as the price of domicile will merely drive away their best customers. If our reasoning is correct, and we believe it is, the nation-state as we know it will not endure in anything like its present form.
\switchcolumn
这意味着,你不再需要为了高收入,而不得不生活在高税率的国家和地区。在未来,大多数财富可以在任何地方赚取,甚至可以在任何地方消费。到那时,政府试图对它的永久居民收取高额的服务费,只会丢掉它们最好的客户。如果我们的推理是正确的,我们相信它是正确的,那么,大家所知道的民族国家,将不会再以任何类似现在的形式而存在。
\end{paracol}

\section[国家末日]{THE END OF NATIONS\\国家末日}
\begin{paracol}{2}[]
Changes that diminish the power of predominant institutions are both unsettling and dangerous. Just as monarchs, lords, popes, and potentates fought ruthlessly to preserve their accustomed privileges in the early stages of the modem period, so today's governments will employ violence, often of a covert and arbitrary kind, in the attempt to hold back the clock. Weakened by the challenge from technology, the state will treat increasingly autonomous individuals, its former citizens, with the same range of ruthlessness and diplomacy it has heretofore displayed in its dealing with other governments. The advent of this new stage in history was punctuated with a bang on August 20, 1998, when the United States fired about \$200 million worth of Tomahawk BGM-109 sea-launched cruise missiles at targets allegedly associated with an exiled Saudi millionaire, Osama bin Laden. Bin Laden became the first person in history to have his satellite phone targeted for attack by cruise missiles. Simultaneously, the United States destroyed a pharmaceutical plant in Khartoum, Sudan, in Bin Laden's honor. The emergence of Bin Laden as the enemy-in-chief of the United States reflects a momentous change in the nature of warfare. A single individual, albeit one with hundreds of millions of dollars, can now be depicted as a plausible threat to the greatest military power of the Industrial era. In statements reminiscent of propaganda employed during the Cold War about the Soviet Union, the United States president and his national security aides portrayed Bin Laden, a private individual, as a transnational terrorist and leading enemy of the United States.
\switchcolumn
削弱了主导机构权力的变化既令人不安,又危险。正如君主、贵族、教皇和有权势的人在现代时期的早期阶段为了维护惯有特权而进行的残酷斗争一样,今天的政府也会用暴力,通常是隐蔽和任意的,试图阻止时钟倒转。受技术挑战削弱的国家将像以前对待其他政府一样,用同样的无情和外交手段处理日益自治的个人——它的前公民。这个历史新阶段的出现在1998年8月20日响起;当时,美国向据称与被流放的沙特亿万富翁奥萨马·本·拉登有关的目标发射了价值约2亿美元的海基巡航导弹。本·拉登成为历史上第一个被巡航导弹攻击卫星电话的人。与此同时,美国在苏丹的喀土穆摧毁了一家制药厂,以表彰本·拉登。本·拉登成为美国最大军事力量的可信威胁。单个人,虽然拥有数百万美元,现在也可以被描绘为对工业时代最大的军事力量构成可信威胁的人。美国总统和他的国家安全助手发表的声明,类似于冷战期间有关苏联的宣传,将本·拉登描述为跨国恐怖分子和美国的头号敌人。
\switchcolumn*
The same military logic that has seen Osama bin Laden elevated to a position as the chief enemy of the United States will assert itself in governments' internal relations with their subjects. Increasingly harsh techniques of exaction will be a logical corollary of the emergence of a new type of bargaining between governments and individuals. Technology will make individuals more nearly sovereign than ever before. And they will be treated that way. Sometimes violently, as enemies, sometimes as equal parties in negotiation, sometimes as allies. But however ruthlessly governments behave, particularly in the transition period, wedding the IRS with the CIA will avail them little. They will be increasingly required by the press of necessity to bargain with autonomous individuals whose resources will no longer be so easily controlled.
\switchcolumn
相同的军事逻辑已将奥萨马·本·拉登提升为美国的首要敌人,这种逻辑也将在政府与国民的内部关系中得到体现。越来越严厉的敛财手段将成为政府与个人谈判出现的逻辑必然结果。技术将让个人比以往任何时候都更接近主权。他们也将被当作如此对待,有时会被视作敌人,有时会被视作平等的谈判方,有时会被视作盟友。然而,无论政府的行为多么无情,尤其是在过渡期间,将联邦税务局与中央情报局(CIA)捆绑在一起是没有什么用处的。由于自主个体的资源不再轻易被控制,政府将越来越需要与自主个体进行谈判来适应这种变化。
\switchcolumn*
The changes implied by the Information Revolution will not only create a fiscal crisis for governments, they will tend to disintegrate all large structures. Fourteen empires have disappeared already in the twentieth century. The breakdown of empires is part of a process that will dissolve the nation-state itself. Government will have to adapt to the growing autonomy of the individual. Taxing capacity will plunge by 50-70 percent. This will tend to make smaller jurisdictions more successful. The challenge of setting competitive terms to attract able individuals and their capital will be more easily undertaken in enclaves than across continents.
\switchcolumn
信息革命所带来的变化不仅会为政府创造财政危机,还将倾向于分解所有的大型结构。20世纪已经有14个帝国消失了。帝国的崩溃是一个过程的一部分,该过程将消解民族国家本身。政府将不得不适应个人日益增长的自治。税收收入能力将下降50-70\%。这将倾向于使较小的司法管辖区更为成功。面对吸引有才华的个体和他们的资本的竞争性条款的挑战,将更容易在飞地中而不是跨越大陆进行。
\switchcolumn*
We believe that as the modern nation-state decomposes, latter-day barbarians will increasingly come to exercise power behind the scenes. Groups like the Russian mafiya, which picks the bones of the former Soviet Union, other ethnic criminal gangs, nomenklaturas, drug lords, and renegade covert agencies will be laws unto themselves. They already are. Far more than is widely understood, the modern barbarians have already infiltrated the forms of the nation-state without greatly changing its appearances. They are microparasites feeding on a dying system. As violent and unscrupulous as a state at war, these groups employ the techniques of the state on a smaller scale. Their growing influence and power are part of the downsizing of politics. Microprocessing reduces the size that groups must attain in order to be effective in the use and control of violence. As this technological revolution unfolds, predatory violence will be organized more and more outside of central control. Efforts to contain violence will also devolve in ways that depend more upon efficiency than magnitude of power.
\switchcolumn
我们相信,随着现代民族国家的解体,后期野蛮人将越来越多地在幕后行使权力。像俄罗斯黑手党、在前苏联领土上乱捡残羹剩饭的其他族裔犯罪团伙、官僚特权阶层、毒品贩子和叛逆的秘密机构这样的团体将成为自己的法律。他们已经是了。现代野蛮人已经在不大改变国家的形象的情况下,渗透到国家形式之中,远比人们所理解的要多得多。他们是在死亡的系统上寄生的微小寄生虫。这些团体和处于战争状态下的政治机构一样具有暴力和不择手段,他们运用国家的技术进行小规模的实施和控制。他们日益增长的影响力和权力是政治下降的一部分。微处理降低了团体必须达到的规模才能在使用和控制暴力方面发挥有效作用的规模。随着这一技术革命的展开,掠夺性暴力将越来越多地组织在中央控制之外。遏制暴力的努力也将以效率而不是权力大小的方式演化。
\end{paracol}

\subsection{倒退的历史}
\begin{paracol}{2}[]
The process by which the nation-state grew over the past five centuries will be put into reverse by the new logic of the Information Age. Local centers of power will reassert themselves as the state devolves into fragmented, overlapping sovereignties. The growing power of organized crime is merely one reflection of this tendency. Multinational companies are already having to subcontract all but essential work. Some conglomerates, such as AT\&T, Unisys, and ITT, have split themselves into several firms in order to function more profitably. The nation-state will devolve like an unwieldy conglomerate, but probably not before it is forced to do so by financial crises.
\switchcolumn
过去五个世纪民族国家成长的过程将被信息时代的新逻辑逆转。当国家分化成交叉重叠的主权时,地方权力中心将重新确立自己的地位。有组织犯罪的不断壮大仅仅是这种趋势的一个反映。跨国公司已经不得不外包除了必要的工作之外的所有工作。一些企业集团,如AT\&T、Unisys和ITT,已经分裂成几个公司,以更赚钱的方式运作。民族国家将像一个难以管理的企业集团一样分化,但可能不会在金融危机迫使它这样做之前。
\switchcolumn*
Not only is power in the world changing, but the work of the world is changing as well. This means that the way business operates will inevitably change. The "virtual corporation" is evidence of a sweeping transformation in the nature of the firm, facilitated by the drop in information and transaction costs. We explore the implications of the Information Revolution for dissolving corporations and doing away with the "good job". In the Information Age, a "job"  will be a task to do, not a position you "have". Microprocessing has created entirely new horizons of economic activity that transcend territorial boundaries. This transcendence of frontiers and territories is perhaps the most revolutionary development since Adam and Eve straggled out of paradise under the sentence of their Maker: "In the sweat of thy face shalt thou eat bread." As technology revolutionizes the tools we use, it also antiquates our laws, reshapes our morals, and alters our perceptions. This book explains how.
\switchcolumn
不仅世界权力在改变,世界工作也在改变。这意味着商业运作的方式不可避免地会发生变化。“虚拟企业”是企业性质发生巨变的证据,这一变化得益于信息和交易成本的降低。我们探讨了信息革命对于溶解公司和消除“好工作”的影响。在信息时代,“工作”将是一项任务,而不是一个你“拥有”的职位。微处理技术已经创造了超越领土界限的全新经济活动领域。这种超越国界和领土的能力或许是自亚当和夏娃受造之后最具革命性的进展:“你必须流着汗水才能吃到面包。”随着技术革新所带来的工具倒退了我们的法律,重塑了我们的道德,改变了我们的感知。本书解释了这一点。
\switchcolumn*
Microprocessing and rapidly improving communications already make it possible for the individual to choose where to work. Transactions on the Internet or the World Wide Web can be encrypted and will soon be almost impossible for tax collectors to capture. Tax-free money already compounds far faster offshore than onshore funds still subject to the high tax burden imposed by the twentieth-century nation-state. After the tum of the millennium, much of the world's commerce will migrate into the new realm of cyberspace, a region where governments will have no more dominion than they exercise over the bottom of the sea or the outer planets. In cyberspace, the threats of physical violence that have been the alpha and omega of politics since time immemorial will vanish. In cyberspace, the meek and the mighty will meet on equal terms. Cyberspace is the ultimate offshore jurisdiction. An economy with no taxes. Bermuda in the sky with diamonds.
\switchcolumn
微处理技术和迅速改进的通信技术已经使个人有选择工作地点的可能性。在互联网或万维网上进行的交易可以进行加密,并且很快就几乎不可能被税务部门发现。免税的资金已经在海外比在岸上完全受到二十世纪民族国家高税负重担的资金更快地复利增长。在千禧年之后,世界上的大部分商业将迁移到新的网络空间,这是一个政府不再能够支配的领域,就像他们对待海底和外层行星一样。在网络空间,一直是政治的尧舜大禹的物理暴力威胁将消失。在网络空间,弱者和强者将在平等的条件下相遇。网络空间是终极的离岸司法管辖区。一个没有税收的经济。天空中的百慕大岛和钻石。
\switchcolumn*
When this greatest tax haven of them all is fully open for business, all funds will essentially be offshore funds at the discretion of their owner. This will have cascading consequences. The state has grown used to treating its taxpayers as a farmer treats his cows, keeping them in a field to be milked. Soon, the cows will have wings.
\switchcolumn
当这个最大的避税天堂完全开放营业时,所有资金实际上将成为业主自行决定的离岸基金。这将产生连锁反应。国家已经习惯了像农民对待奶牛一样对待纳税人,将他们留在田地里挤奶。很快,奶牛将有翅膀。
\end{paracol}

\subsection{国家的复仇}
\begin{paracol}{2}[]
Like an angry farmer, the state will no doubt take desperate measures at first to tether and hobble its escaping herd. It will employ covert and even violent means to restrict access to liberating technologies. Such expedients will work only temporarily, if at all. The twentieth-century nation-state, with all its pretensions, will starve to death as its tax revenues decline.
\switchcolumn
像一位愤怒的农民一样,国家无疑会首先采取绝望的措施来束缚和牵制其逃离的群体。它将采用隐秘甚至暴力手段来限制对解放性技术的接触。这些方法只能在短时间内起作用,如果有的话。二十世纪的民族国家,带着所有的自负,将因为税收下降而面临资金不足的困境。
\switchcolumn*
When the state finds itself unable to meet its committed expenditure by raising tax revenues, it will resort to other, more desperate measures. Among them is printing money. Governments have grown used to enjoying a monopoly over currency that they could depreciate at will. This arbitrary inflation has been a prominent feature of the monetary policy of all twentieth-century states. Even the best national currency of the postwar period, the German mark, lost 71 percent of its value from January 1, 1949, through the end of June 1995. In the same period, the U.S. dollar lost 84 percent of its value. This inflation had the same effect as a tax on all who hold the currency. As we explore later, inflation as revenue option will be largely foreclosed by the emergence of cybermoney. New technologies will. allow the holders of wealth to bypass the national monopolies that have issued and regulated money in the modern period. Indeed, the credit crises that swept through Asia, Russia, and other emerging economies in 1997 and 1998 attest to the fact that national currencies and national credit ratings are anachronisms inimical to the smooth operation of the global economy. It is precisely the fact that the demands of sovereignty require all transactions within a jurisdiction to be denominated in a national currency that creates the vulnerability to mistakes by central bankers and attacks by speculators which precipitated deflationary crises in one jurisdiction after another. In the Information Age, individuals will be able to use cybercurrencies and thus declare their monetary independence. When individuals can conduct their own monetary policies over the World Wide Web it will matter less or not at all that the state continues to control the industrial-era printing presses. Their importance for controlling the world's wealth will be transcended by mathematical algorithms that have no physical existence. In the new millennium, cybermoney controlled by private markets will supersede fiat money issued by governments. Only the poor will be victims of inflation and ensuing collapses into deflation that are consequences of the artificial leverage which fiat money injects into the economy.

\switchcolumn
当国家发现自己无法通过增加税收来满足支出时,就会采取其他更绝望的措施。其中之一是印钞票。政府已经习惯了享有货币垄断权,可以随意贬值,这种任意通胀一直是所有二十世纪国家的货币政策的一个突出特征。甚至二战后期最好的国家货币德国马克在1949年1月1日至1995年6月底期间的价值下降了71\%。在同一时期,美元贬值了84\%。这种通胀对持有货币的人的影响与税收类似。随着我们之后的探讨,通货膨胀作为一种收入选择在信息时代将被广泛淘汰。新技术将允许财富的持有者绕过现代时期发行和管理货币的国家垄断。事实上,1997年和1998年席卷亚洲、俄罗斯和其他新兴经济体的信贷危机表明,国家货币和国家信用评级是不利于全球经济运作的陈旧思维。正是主权要求在一个管辖范围内的所有交易必须以国家货币计价的事实,才造成了银行家错误和投机者攻击的漏洞,从而引发了一个又一个的通货紧缩危机。在信息时代,个人将能够使用\textbf{网络货币},因此宣布他们的货币独立。当个人能够通过万维网进行自己的货币政策时,国家继续控制工业时代的印刷机就会变得不那么重要,甚至不重要。它们为掌控世界财富的重要性,将被没有物理存在的数学算法所超越。\textbf{在新千年,由私人市场控制的网络货币将取代政府发行的法定货币}。只有穷人将成为通货膨胀和随后崩溃的受害者,这是法定货币注入经济的人工杠杆的后果。

\switchcolumn*
Lacking their accustomed scope to tax and inflate, governments, even in traditionally civil countries, will turn nasty. As income tax becomes uncollectible, older and more arbitrary methods of exaction will resurface. The ultimate form of withholding tax --de facto or even overt hostage-taking -- will be introduced by governments desperate to prevent wealth from escaping beyond their reach. Unlucky individuals will find themselves singled out and held to ransom in an almost medieval fashion. Businesses that offer services that facilitate the realization of autonomy by individuals will be subject to infiltration, sabotage, and disruption. Arbitrary forfeiture of property, already commonplace in the United States, where it occurs five thousand times a week, will become even more pervasive. Governments will violate human rights, censor the free flow of information, sabotage useful technologies, and worse. For the same reasons that the late, departed Soviet Union tried in vain to suppress access to personal computers and Xerox machines, Western governments will seek to suppress the cybereconomy by totalitarian means.

\switchcolumn
在没有了习惯性的征收所得税的情况下,即使在传统上文明的国家,政府也将变得残忍。更老、更武断的征税方式将重新出现。政府迫切希望阻止财富逃离其管辖范围,将引入最终形式的代扣税——实际上甚至是公开的劫持人质。不幸的个人将发现自己被单独挑选并以几乎中世纪的方式被绑架和赎金。为个人实现自治的服务的企业将受到渗透、破坏和破坏。在美国,已经普遍存在的任意没收财产行为,每周发生五千次的情况将变得更为普遍。政府将侵犯人权,审查信息的自由流动,破坏有用的技术等等。由于晚已逝去的苏联试图无效地压制个人电脑和施乐复印机的使用,西方政府也将通过极权主义手段试图抑制网络经济。

\end{paracol}


\section[卢德派回归]{RETURN OF THE LUDDITES\\卢德派的回归}
\begin{paracol}{2}[]
Such methods may prove popular among some population segments. The good news about individual liberation and autonomy will seem to be bad news to many who are frightened by the transition crisis and do not expect to be winners in the new configuration of society. The apparent popularity of the draconian capital controls imposed in 1998 by Malaysian prime minister Mahathir Mohamad in the wake of the Asian meltdown testifies to residual enthusiasm among many for the old-fashioned closed economy dominated by the nation-state. This nostalgia for the past will be fed by resentments inflamed by the inevitable transition crisis. The greatest resentment is likely to be centered among those of middle talent in currently rich countries. They particularly may come to feel that information technology poses a threat to their way of life. The beneficiaries of organized compulsion, including millions receiving income redistributed by governments, may resent the new freedom realized by the Sovereign Individuals. Their upset will illustrate the truism that "where you stand is determined by where you sit."
\switchcolumn
这些方法可能会在某些人群中流行。个人解放和自治的好消息可能会给许多人带来打击,他们对转型危机感到恐惧,并且不期望在社会新形态的赢家中。马来西亚总理马哈蒂尔·莫哈末于1998年在亚洲经济危机后实施的严厉的资本管制政策表明,仍有很多人倾向于传统的以国家为主导的封闭经济。这种对过去的怀旧情结将因转型危机必然带来的愤怒而得到滋养。最大的憎恨可能集中在目前富裕国家中的中等才能人群上。他们特别可能会感到信息技术对他们的生活方式构成威胁。有组织强制的受益者,包括数百万接受政府收入再分配的人,可能会对主权个人实现的新自由感到不满。他们的不满将说明“你所站的位置取决于你的处境”的真理。
\switchcolumn*
It would be misleading, however, to attribute all the bad feelings that will be generated in the coming transition crisis to the bald desire to live at someone else's expense. More will be involved. The very character of human society suggests that there is bound to be a misguided moral dimension to the coming Luddite reaction. Think of it as a bald desire fitted with a moral toupee. We explore the moral and moralistic dimensions of the transition crisis. Self-interested grasping of a conscious kind has far less power to motivate actions than does self-righteous fury. While adherence to the civic myths of the twentieth century is rapidly falling away, they are not without their true believers. Many humans, as the passage quoted from Craig Lambert attests, are belongers, who place importance on being members of a group. The same need to identify that motivates fans of organized sports makes some partisans of nations. Everyone who came of age in the twentieth century has been inculcated in the duties and obligations of the twentieth-century citizen. The residual moral imperatives from industrial society will stimulate at least some neo-Luddite attacks on information technologies.

\switchcolumn
然而,将即将到来的过渡危机中将产生的所有不良情绪归咎于光秃秃地想生活在别人的代价之下是具有误导性的。它将涉及更多方面。人类社会的性质表明,即将到来的勒德派反应必然会有一个错误的道德维度。把它想象成一个带有道德假发的光秃秃的愿望。我们探讨了过渡期的道德和道德主义维度。有意识的自我利益追求远不如自以为是的愤怒来推动行为。虽然对于20世纪公民传统的遵循正在迅速减少,但它们并不是没有真正的信徒。正如从Craig Lambert引用的文章所证明的那样,许多人是“属于者”,他们认为成为团队成员很重要。同样的认同需求使一些国家倾向于某些主义。在20世纪成年的每个人都受到20世纪公民责任和义务的熏陶。工业社会的残留道德义务将激发至少一些基于信息技术的新勒德派攻击。

\switchcolumn*
In this sense, this violence to come will be at least partially an expression of what we call ``moral anachronism,'' the application of moral strictures drawn from one stage of economic life to the circumstances of another. Every stage of society requires its own moral rules to help individuals overcome incentive traps peculiar to the choices they face in that particular way of life. Just as a farming society could not live by the moral rules of a migratory Eskimo band, so the Information Society cannot satisfy moral imperatives that emerged to facilitate the success of a militant 20th century industrial state. We explain why.

\switchcolumn
在这个意义上,未来的暴力至少部分上将是我们所谓的“道德时代落后”的表达,即将道德准则应用于另一种经济生活方式的情况。每个社会阶段都需要其自身的道德准则来帮助个人克服该特定生活方式下他们面临的激励陷阱。正如一个农业社会不能按照一个流浪爱斯基摩人团队的道德规则生活一样,信息社会也无法满足于对二十世纪激进的工业国家成功所产生的道德要求。我们会解释原因。

\switchcolumn*
In the next few years, moral anachronism will be in evidence at the core countries of the West in much the way that it has been witnessed at the periphery over the past five centuries. Western colonists and military expeditions stimulated such crises when they encountered indigenous hunting-and-gathering bands, as well as peoples whose societies were still organized for farming. The introduction of new technologies into anachronistic settings caused confusion and moral crises. The success of Christian missionaries in converting millions of indigenous peoples can be laid in large measure to the local crises caused by the sudden introduction of new power arrangements from the outside. Such encounters recurred over and over, from the sixteenth century through the early decades of the twentieth century. We expect similar clashes early in the new millennium as Information Societies supplant those organized along industrial lines.

\switchcolumn
在接下来的几年中,这种道德时代落后将在西方核心国家的许多领域中得到体现,就像在过去的五个世纪里在边缘地区所见到的一样。当西方殖民者和军事远征队遭遇土著狩猎采集部落以及那些仍以耕种种植为生的人时,就会出现这样的危机。新技术的引入到这些时代落后的环境中会导致混乱和道德危机。基督教传教士成功地转化了数百万土著民族,这在很大程度上归功于由外部引入的新能源系统带来的本地危机。从16世纪到20世纪初,这样的冲突一再发生。我们预计,在信息社会取代沿工业线组织的社会时期早期,将会有类似的冲突出现。

\end{paracol}

\subsection{对强制的怀旧情感}
\begin{paracol}{2}[]
The rise of the Information Society will not be wholly welcomed as a promising new phase of history, even among those who benefit from it most. Everyone will feel some misgivings. And many will despise innovations that undermine the territorial nation-state. It is a fact of human nature that radical change of any kind is almost always seen as a dramatic turn for the worse. Five hundred years ago, the courtiers gathered around the duke of Burgundy would have said that unfolding innovations that undermined feudalism were evil. They thought the world was rapidly spiraling downhill at the very time that later historians saw an explosion of human potential in the Renaissance. Likewise, what may someday be seen as a new Renaissance from the perspective of the next millennium will look frightening to tired twentieth-century eyes.

\switchcolumn
信息社会的崛起并不是所有人都欣然接受的一段有前途的历史阶段,即使是那些从中受益最多的人也会感到一些疑虑。每个人都会感到某些不安。许多人会鄙视破坏领土民族国家的创新。这是人性的一个事实,任何一种激进的变革几乎总被视为一个戏剧性的倒退。五百年前,围绕勃艮第公爵的宫廷人士会说,破坏封建制度的展开中的创新是邪恶的。他们认为世界正在迅速地走下坡路,而后来的历史学家则在文艺复兴时期看到了人类潜力的爆发。同样,从下一个千年的角度来看,有一天可能会被看作是新文艺复兴,但它会让疲惫的二十世纪眼睛感到恐惧。

\switchcolumn*
There is a high probability that some who are offended by the new ways, as well as many who are disadvantaged by them, will react unpleasantly. Their nostalgia for compulsion will probably turn violent. Encounters with these new ``Luddites'' will make the transition to radical new forms of social organization at least a measure of bad news for everyone. Get ready to duck. With the speed of change outracing the moral and economic capacity of many in living generations to adapt, you can expect to see a fierce and indignant resistance to the Information Revolution, notwithstanding its great promise to liberate the future.


\switchcolumn
有很大可能会有一些被新方式冒犯的人,以及许多受其不利影响的人,会做出令人不愉快的反应。他们对强迫的怀旧情结可能会变得暴力。与这些新的“卢德派”相遇将使向激进的新社会组织形式的转变对每个人都至少有一些不好的消息。准备好躲避吧。随着变化的速度超过生活中许多人适应的道德和经济能力,你可以预料到对信息革命的凶猛反抗,尽管它有解放未来的巨大承诺。
\switchcolumn*
You must understand and prepare for such unpleasantness. A series of transition crises lies ahead. Deflationary tribulations, such as the Asian contagion that swept through the Far East to Russia and other emerging economies in 1997 and 1998, will erupt sporadically as the dated national and international institutions left over from the Industrial Era prove inadequate to the challenges of the new, dispersed, transnational economy. The new information and communication technologies are more subversive of the modern state than any political threat to its predominance since Columbus sailed. This is important because those in power have seldom reacted peacefully to developments that undermined their authority. They are not likely to now.

\switchcolumn
你必须理解并为这样的不愉快情况做好准备。一系列的转型危机将接踵而至。通货紧缩的磨难,例如 1997 年和 1998 年席卷远东到俄罗斯和其他新兴经济体的亚洲瘟疫,将会间歇性地爆发,因为那些过时的国内外机构已经证明无法应对新的、分散的跨国经济的挑战。新的信息和通讯技术比哥伦布航海后的任何政治威胁都要更具颠覆性,对于现代国家的主导地位更加具有威胁性。这一点很重要,因为那些在权力中的人很少会对破坏他们权威的发展做出和平反应。他们现在也不太可能这么做。

\switchcolumn*
The clash between the new and the old will shape the early years of the new millennium. We expect it to be a time of great danger and great reward, and a time of much diminished civility in some realms and unprecedented scope in others. Increasingly autonomous individuals and bankrupt, desperate governments will confront one another across a new divide. We expect to see a radical restructuring of the nature of sovereignty and the virtual death of politics before the transition is over. Instead of state domination and control of resources, you are destined to see the privatization of almost all services governments now provide. For inescapable reasons that we explore in this book, information technology will destroy the capacity of the state to charge more for its services than they are worth to you and other people who pay for them.
\switchcolumn
新旧之间的冲突将塑造新千年的早期年份。我们预计这将是一个充满危险和奖励的时代,在某些领域中,文明的减弱将是空前的,而在其他领域中,范围将是前所未有的。越来越自主的个人和破产、绝望的政府将在新的分界线上相互对抗。我们预计,在过渡结束之前,主权的性质将发生根本性的重组,政治几乎完全死亡。与其主导和控制资源,你注定会看到几乎所有政府现在提供的服务的私有化。出于我们在本书中探讨的无法逃避的原因,信息技术将摧毁国家为其服务所收费比其价值和其他为其支付的人们的贡献更高的状态的能力。

\end{paracol}

\subsection{市场赋予的主权}
\begin{paracol}{2}[]
To an extent that few would have imagined only a decade ago, individuals will achieve increasing autonomy over territorial nation-states through market mechanisms. All nation-states face bankruptcy and the rapid erosion of their authority. Mighty as they are, the power they retain is the power to obliterate, not to command. Their intercontinental missiles and aircraft carriers are already artifacts, as imposing and useless as the last warhorse of feudalism.
\switchcolumn
仅仅十年前,大多数人都无法想象,通过市场机制,个人将获得对领土国家越来越多的自治权。所有国家都面临破产和权威的迅速侵蚀。尽管它们强大,但它们所保留的权力只是毁灭而非统治的权力。它们的洲际导弹和航空母舰已经成为历史,就像封建主义时代的最后一匹战马一样具有威严和无用。
\switchcolumn*
Information technology makes possible a dramatic extension of markets by altering the way that assets are created and protected. This is revolutionary. Indeed, it promises to be more revolutionary for industrial society than the advent of gunpowder proved to be for feudal agriculture. The transformation of the year 2000 implies the commercialization of sovereignty and the death of politics, no less than guns implied the demise of oath-based feudalism. Citizenship will go the way of chivalry.

\switchcolumn
信息技术通过改变资产的创造和保护方式,使得市场得以大幅度扩展。这是一场革命。实际上,对于工业社会而言,它的革命性可能比火药对封建农业的影响还要深远。2000年的转型意味着主权的商业化和政治的消亡,正如火器对宣誓效忠的封建制度的终结一样。公民身份将逐渐成为历史。

\switchcolumn*
We believe that the age of individual economic sovereignty is coming. Just as steel mills, telephone companies, mines, and railways that were once "nationalized" have been rapidly privatized throughout the world, you will soon see the ultimate form of privatization --the sweeping denationalization of the individual. The Sovereign Individual of the new millennium will no longer be an asset of the state, a de facto item on the treasury's balance sheet. After the transition of the year 2000, denationalized citizens will no longer be citizens as we know them, but customers.

\switchcolumn
我们相信,个人经济主权时代即将到来。正如曾被国有化的钢铁厂、电话公司、矿山和铁路在全球范围内被迅速私有化一样,你将很快见证终极私有化的形式——\textbf{个人的彻底非国有化}。新千年的主权个体将不再是国家资产,不再是国库资产负债表上的一个实际项目。在2000年过渡后,非国有化公民将不再是我们所知道的公民,而是\textbf{顾客}。
\end{paracol}

\section[带宽胜过边界]{BANDWIDTH TRUMPS BORDERS\\带宽胜过边界}



\section[复兴游行的法律]{REVIVING LAWS OF THE MARCH\\复兴游行的法律}


\section[愿望的虚荣]{THE VANITY OF WISHES\\愿望的虚荣}


\section[大型机和Y2K定时炸弹]{MAINFRAMES AND THE Y2K TIME BOMB\\大型机和Y2K定时炸弹}


\section[Y2K与核武库]{Y2K AND THE NUCLEAR ARSENAL\\Y2K与核武库}



\begin{paracol}{2}[]
\switchcolumn
\end{paracol}

