\chapter[历史上大都市的转型]{历史上的大都市转型}
% 选项为页眉页脚的简写

\section{现代世界的衰落}
\begin{paracol}{2}
In our view, you are witnessing nothing less than the waning of the Modern Age. It is a development driven by a ruthless but hidden logic. More than we commonly understand, more than CNN and the newspapers tell us, the next millennium will no longer be "modern." We say this not to imply that you face a savage or backward future, although that is possible, but to emphasize that the stage of history now opening will be qualitatively different from that into which you were born.

\switchcolumn
在我们看来,你们正在见证的不仅仅是现代时代的衰落,更是一种无情但隐藏的逻辑所驱动的发展。在我们共同理解的范畴之外,甚至超越CNN和报纸带给我们的信息,下一个千年将不再是“现代”的。我们之所以这么说,并不是在暗示你们将面临野蛮或落后的未来,虽然这是有可能的,而是强调历史阶段正在打开,而这将与你们出生时的历史阶段在本质上有所不同。
\switchcolumn*
Something new is coming. Just as farming societies differed in kind from hunting-and-gathering bands, and industrial societies differed radically from feudal or yeoman agricultural systems, so the New World to come will mark a radical departure from anything seen before.

\switchcolumn
新的东西即将到来。正如农业社会与狩猎采集社会之间有着本质区别,工业社会与封建或家族农业系统之间有着根本不同的区别,新世界的到来将标志着与以往任何时代的根本性分别。

\switchcolumn*
In the new millennium, economic and political life will no longer be organized on a gigantic scale under the domination of the nation-state as it was during the modern centuries. The civilization that brought you world war, the assembly line, social security, income tax, deodorant, and the toaster oven is dying. Deodorant and the toaster oven may survive. The others won't. Like an ancient and once mighty man, the nation-state has a future numbered in years and days, and no longer in centuries and decades.

\switchcolumn
在新千年中,经济和政治生活将不再像现代世纪那样以国家为主导而组织在巨大的规模上。那个给你带来世界大战、装配线、社会保障、所得税、除臭剂和烤箱的文明正在消亡。除臭剂和烤箱可能会幸存下来,但其他的则不会。像一个古老而曾经强大的人一样,民族国家的未来将只剩下年和日,而不再是世纪和十年。

\switchcolumn*
Governments have already lost much of their power to regulate and compel. The collapse of Communism marked the end of a long cycle of five centuries during which magnitude of power overwhelmed efficiency in the organization of government. It was a time when the returns to violence were high and rising. They no longer are. A phase transition of world-historic dimensions has already begun. Indeed, the future Gibbon who chronicles the decline and fall of the once-Modern Age in the next millennium may declare that it had already ended by the time you read this book. Looking back, he may say, as we do, that it ended with the fall of the Berlin Wall in 1989. Or with the death of the Soviet Union in 1991. Either date could come to stand as a defining event in the evolution of civilization, the end of what we now know as the Modern Age.

\switchcolumn
政府已经失去了大部分的监管和强制力。共产主义的崩溃标志着五个世纪的长周期结束了,在这期间,政府组织中的权力量表现得比效率还要强大。现在,暴力的回报率已经不再高涨。世界历史上已经开始了历史性的相变。事实上,下一个千年记录现代时代沦落和衰落的吉本可能会宣称,当你读这本书时,它已经结束了。回顾过去,他可能会像我们一样说,它已经在1989年的柏林墙倒塌和1991年苏联的解体时期结束了。这两个时间点都可能成为定义文明演化的关键事件,是我们所知的现代时代的结束。

\switchcolumn*
The fourth stage of human development is coming, and perhaps its least predictable feature is the new name under which it will be known. Call it "Post-Modern." Call it the "Cyber Society" or the "Information Age." Or make up your own name. No one knows what conceptual glue will stick a nickname to the next phase of history.

\switchcolumn
人类发展的第四阶段即将来临,也许它最不可预测的特征是它将被称为新的名字。称之为“后现代”,称之为“网络社会”或“信息时代”。或者想出你自己的名字。没有人知道什么概念性的胶水会将昵称与历史的下一个阶段粘在一起。

\switchcolumn*
We do not even know that the five-hundred-year stretch of history just ending will continue to be thought of as "modern." If future historians know anything about word derivations, it will not be. A more descriptive title might be "The Age of the State" or "The Age of Violence." But such a name would fall outside the temporal spectrum that currently defines the epochs of history. "Modern," according to the \emph{Oxford English Dictionary}, means "pertaining to the present and recent times, as distinguished from the remote past... In historical use commonly applied (in contradiction to ancient and medieval) to the time subsequent to the MIDDLE AGES."

\switchcolumn
我们甚至不知道刚刚结束的五百年历史是否会继续被视为“现代”。如果未来的历史学家知道词语源流,那么不会是这样的。更加描述性的标题可能是“国家时代”或“暴力时代”。但这样的名字将超出当前定义历史时代的时间范围。牛津英语词典中,“现代”意味着“与现在和最近时期有关,与远古不同... 在历史使用中,通常应用于中世纪之后的时间(与古代和中世纪相矛盾)。”

\switchcolumn*
A Western people consciously thought of themselves as "modern" only when they came to understand that the medieval period was over. Before 1500, no one had ever thought of the feudal centuries as a "middle" period in Western civilization. The reason is obvious upon reflection: before an age can reasonably be seen as sandwiched in the "middle" of two other historic epochs, it must have already come to an end. Those living during the feudal centuries could not have imagined themselves as living in a halfway house between antiquity and modern civilization until it dawned on them not just that the medieval period was over, but also that medieval civilization differed dramatically from that of the Dark Ages or antiquity.

\switchcolumn
西方的人只有在意识到中世纪已经结束时,才有意识地认为自己是“现代人”。在1500年之前,没有人认为封建时期是西方文明中的“中间时期”。原因很明显,回想一下就会明白:在一个时代被合理地看作夹在另外两个历史时期之间之前,它必须已经结束了。在封建时期生活的人无法想象自己生活在古代文明和现代文明之间的半路上,直到他们意识到中世纪时期不仅已经结束,而且中世纪文明与黑暗时代或古代文明大相径庭。

\switchcolumn*
Human cultures have blind spots. We have no vocabulary to describeparadigm changes in the largest boundaries of life, especially those happening around us. Notwithstanding the many dramatic changes that have unfolded since the time of Moses, only a few heretics have bothered to think about how the transitions from one phase of civilization to another actually unfold.

\switchcolumn
人类文化有盲点。我们没有词汇来描述生活中最大范围的范式变化,尤其是那些发生在我们身边的变化。尽管自摩西时代以来许多戏剧性的变化已经发生,但只有少数异端分子费心思考过文明从一个阶段过渡到另一个阶段的方式。

\switchcolumn*
How are they triggered? What do they have in common? What patterns can help you tell when they begin and know when they are over? When will Great Britain or the United States come to an end? These are questions for which you would be hard-pressed to find conventional answers.

\switchcolumn
它们是如何被触发的?有什么共同点?有哪些模式可以帮助你告诉它们何时开始知道何时结束?英国或美国什么时候会灭亡?这些问题你很难找到常规答案。

\end{paracol}

\subsection{对前瞻的禁忌}
\begin{paracol}{2}
To see "outside" an existing system is like being a stagehand trying to force a dialogue with a character in a play. It breaches a convention that helps keep the system functioning. Every social order incorporates among its key taboos the notion that people living in it should not think about how it will end and what rules may prevail in the new system that takes its place. Implicitly, whatever system exists is the last or the only system that will ever exist. Not that this is so baldly stated. Few who have ever read a history book would find such an assumption realistic if it was articulated. Nonetheless, that is the convention that rules the world. Every social system, however strongly or weakly it clings to power, pretends that its rules will never be superseded. They are the last word. Or perhaps the only word. Primitives assume that theirs is the only possible way oforganizing life. More economically complicated systems that incorporate a sense of history usually place themselves at its apex. Whether they are Chinese mandarins in the court of the emperor, the Marxist nomenklatura in Stalin's Kremlin, or members of the House of Representatives in Washington, the powers-that-be either imagine no history at all or place themselves at the pinnacle of history, in a superior position compared to everyone who came before, and the vanguard of anything to come.  

\switchcolumn
看到一个现有的系统“之外”的东西,就像是一个舞台工作人员试图强迫与戏剧中的角色对话一样。这违反了保持系统运作的一个惯例。每个社会秩序都包含了其关键禁忌中的概念,即生活在其中的人不应该想到它将如何结束以及新的取代它的系统中会出现哪些规则。隐含在其中的是无论什么系统存在,它都是最后一个或唯一一个存在的系统。并不是说就是这么直白地陈述。几乎所有读过历史书的人都不会认为这种假设是现实的,如果这是清晰表述的话。尽管如此,这种惯例仍然统治着世界。每个社会系统,无论它紧紧地还是弱弱地掌握着权力,都假装它的规则永远不会被取代。它们是最后的话语。或许只有话语。原始的人们认为他们的组织生活方式是唯一的可能。更经济复杂的系统包含了对历史的感知,通常将自己置于历史的顶点。无论是中国皇帝朝廷中的官员,斯大林的克里姆林宫中的马克思主义者,还是华盛顿众议院的议员,那些掌权的人们要么没有想过历史,要么将自己置于历史顶点,相对于过去的每个人,处于更高的位置,并成为未来任何事物的先锋。


\switchcolumn*
This is true for practical reasons. The more apparent it is that a system is nearing an end, the more reluctant people will be to adhere to its laws. Any social organization will therefore tend to discourage or play down analyses that anticipate its demise. This alone helps ensure that history's great transitions are seldom spotted as they happen. If you know nothing else about the future, you can rest assured that dramatic changes will be neither welcomed nor advertised by conventional thinkers.

\switchcolumn
这是因为实际的原因。系统接近结束的情况越明显,人们就越不愿意遵守它的法律。因此,任何社会组织都会倾向于阻止或淡化预测其灭亡的分析。这可以确保历史上的伟大转变很少在其发生时被发现。如果你对未来一无所知,你可以放心,戏剧性的变化既不会受到欢迎,也不会被传统思想家广泛宣传。

\switchcolumn*
You cannot depend upon conventional information sources to give you an objective and timely warning about how the world is changing and why. If you wish to understand the great transition now under way, you have little choice but to figure it out for yourself.  

\switchcolumn
你不能依赖传统的信息来源为你提供关于世界正在如何改变以及为什么改变的客观及时的警告。如果你想了解当前正在进行的伟大转变,你别无选择,只能自己想办法。

\end{paracol}

\subsection{超越表面现象}
\begin{paracol}{2}
This means looking beyond the obvious. The record shows that even transitions that are undeniably real in retrospect may not be acknowledged for decades or even centuries after they happen. Consider the fall of Rome. It was probably the most important historic development in the first millennium of the Christian era. Yet long after Rome's demise, the fiction that it survived was held out to public view, like Lenin's embalmed corpse. No one who depended upon the pretenses of officials for his understanding of the "news" would have learned that Rome had fallen until long after that information ceased to matter.  

\switchcolumn
这意味着超越显而易见的东西。纪录显示,即使是回顾起来无可置疑地真实的转变,可能在几十年甚至几个世纪后才被承认。考虑罗马的陨落。这可能是基督教时代的前一千年中最重要的历史发展。然而,在罗马倒塌之后很长一段时间,仍然存在着它仍然生存的虚构,就像列宁的防腐处理过的尸体一样。任何依赖于官员的假装来了解“新闻”的人,直到那些信息不再重要之后很久才会知道罗马的倒塌。

\switchcolumn*
The reason was not merely the inadequacy of communications in the ancient world. The outcome would have been much the same had CNN miraculously been in business, running its videotape in September 476. That is when the last Roman emperor in the West, Romulus Augustulus, was captured in Ravenna and forcibly retired to a villa in Campania on a pension. Even if Wolfe Blitzer had been there with minicams recording the news in 476, it is unlikely that he or anyone else would have dared to characterize those events as marking the end of the Roman Empire. That, of course, is exactly what latter historians said happened.

\switchcolumn
这不仅仅是古代通讯方式的不足所导致的。即使CNN在9月476日奇迹般地营业,并播放其录像带,结果也会大致相同。那是当时西部最后的罗马皇帝罗慕路斯·奥古斯图卢斯在拉文纳被俘的时候,他被强制性地退休到坎帕尼亚的一座别墅里,并领取抚恤金。即使沃尔夫·布里策尔与短视频记录在476年报道新闻,也不太可能有人敢说这些事件标志着罗马帝国的终结。 当然,这正是后来的历史学家所说的事情。

\switchcolumn*
CNN editors probably would not have approved a headline story saying "Rome fell this evening." The powers-that-be denied that Rome had fallen. Peddlers of "news" seldom are partisans of controversy in ways that would undermine their own profits. They may be partisan. They may even be outrageously so. But they seldom report conclusions that would convince subscribers to cancel their subscriptions and head for the hills. Which is why few would have reported the fall of Rome even if it had been technologically possible. Experts would have come forth to say that it was ridiculous to speak of Rome falling. To have said otherwise would have been bad for business and, perhaps, bad for the health of those doing the reporting. The powers in late-fifth-century Rome were barbarians, and they denied that Rome had fallen.

\switchcolumn
CNN编辑们可能不会批准"今晚罗马倒了"这样的头条新闻。那些当权者否认罗马已经倒下。 "新闻"的推销员很少是争议的党派,这些争议可能会破坏自己的利润。他们可能是党派性的。他们甚至可能非常令人震惊。但是,他们很少报道会让订户取消订阅并逃到山上的结论。因此,即使从技术上可能,也很少有人报道罗马的倒塌。专家会站出来说,说罗马倒塌是荒谬的。否则会对商业不利,甚至可能对报道者的健康有害。五世纪晚期的罗马当权者是野蛮人,他们否认罗马已经倒塌。

\switchcolumn*
But it was not merely a case of authorities' saying, "Don't report this or we will kill you." Part ofthe problem was that Rome was already so degener- ate by the later decades of the fifth century that its "fall" genuinely eluded the notice of most people who lived through it. In fact, it was a generation later before Count Marcellinus first suggested that "The Western Roman Empire perished with this Augustulus." Many more decades passed, perhaps centuries, before there was a common acknowledgment that the Roman Empire in the West no longer existed. Certainly Charlemagne believed that he was a legitimate Roman emperor in the year 800.  

\switchcolumn
但问题不仅仅在于当局说,“不要报道这个,否则我们会杀了你。”问题的一部分在于,到了第五世纪后期,罗马已经如此堕落,以至于“倒下”真正逃脱了大多数生活于其中的人的注意。事实上,在一代人之后,马塞利努斯伯爵首次提出“西方罗马帝国已经随着奥古斯图卢斯的倒塌而灭亡”。在许多年,也许是几个世纪之后,人们才共同承认西方的罗马帝国已不存在。当然,查理曼也认为自己是公元800年的合法罗马皇帝。

\switchcolumn*
The point is not that Charlemagne and all who thought in conventional terms about the Roman Empire after 476 were fools. To the contrary. The characterization of social developments is frequently ambiguous. When the power of predominant institutions is brought into the bargain to reinforce a convenient conclusion, even one based largely on pretense, only someone of strong character and strong opinions would dare contradict it. If you try to put yourself in the position of a Roman of the late fifth century, it is easy to imagine how tempting it would have been to conclude that nothing had changed. That certainly was the optimistic conclusion. To have thought otherwise might have been frightening. And why come to a frightening conclusion when a reassuring one was at hand? 

\switchcolumn
重点不在于查理曼和476年以后所有那些按照传统思考罗马帝国的人都是傻瓜。相反,社会发展的表征经常是模糊的。当主导机构的权力混入其中,加强一个基本上基于虚假的便利结论时,只有一个有坚强意志和强烈观点的人敢于反驳。如果你试图站在五世纪末期罗马人的立场上,很容易想象得出得到一个令人鼓舞的结论——什么都没有改变。这当然是乐观的结论。认为情况不同可能会令人感到恐惧。既然有一个令人放心的结论在手,为什么要得出一个令人恐惧的结论呢?

\switchcolumn*
After all, a case could have been made that business would continue as usual. It had in the past. The Roman army, and particularly the frontier garrisons, had been barbarized for centuries. By the third century, it had become regular practice for the army to proclaim a new emperor. By the fourth century, even officers were Germanized and frequently illiterate. There had been many violent overthrows of emperors before Romulus Augustulus was removed from the throne. His departure might have seemed no different to his contemporaries than many other upheavals in a chaotic time. And he was sent packing with a pension. The very fact that he received a pension, even for a brief period before he was murdered, was a reassurance that the system survived. To an optimist, Odoacer, who deposed Romulus Augustulus, reunified rather than destroyed the empire. A son of Attila's sidekick Edecon, Odoacer was a clever man. He did not proclaim himself emperor. Instead, he convened the Senate and prevailed upon its too-suggestible members that they offer the emperorship and thus sovereignty over the whole empire to Zeno, the Eastern emperor in faraway Byzantium. Odoacer was merely to be Zeno's patricius to govern Italy.  

\switchcolumn
毕竟,可以提出一个观点——生意将会照常进行。这在过去是这样。罗马军队,特别是边境的守卫队,已经成为野蛮化的几个世纪了。到了第三个世纪,军队宣布一位新皇帝已经成为一种常规做法。到了第四个世纪,甚至高级军官也已经野蛮化,并经常不识字。在罗穆卢斯·奥古斯都被废黜之前,已经有许多皇帝被暴力推翻。在这个混乱的时期,他的离去可能对他的同时代人来说并没有什么区别。而他被开除,有养老金安排。即使在他被谋杀之前,他获得的养老金的事实也是一个安全的保障,表明体制存活了下来。对于一个乐观主义者来说,罗马帝国奥多阿塞尔推翻罗穆卢斯·奥古斯都,团结起来,而不是被摧毁了。奥多阿塞尔是阿提拉助手埃代孔的儿子,是一个聪明人。他没有宣称自己是皇帝。相反,他召集了参议院,并说服了过于容易受影响的成员,要把皇帝的地位和整个帝国的主权提供给在遥远的拜占庭的东部皇帝泽诺。奥多阿塞尔只是泽诺的执事,统治意大利。

\switchcolumn*
As Will Durant wrote in \emph{The Story of Civilization}, these changes did not appear to be the "fall of Rome" but merely "negligible shifts on the surface of the national scene." When Rome fell, Odoacer said that Rome endured. He, along with almost everyone else, was keen to pretend that nothing had changed. They knew that "the glory that was Rome" was far better than the barbarism that was taking its place. Even the barbarians thought so. As C. W. Previte-Orton wrote in \emph{The Shorter Cambridge Medieval History}, the end of the fifth century, when "the Emperors had been replaced by barbaric German kings," was a time of "persistent make-believe".  

\switchcolumn
正如威尔·杜兰特在《文明的故事》中写的那样,这些变化似乎并不是“罗马的陨落”,而只是“国家形势表面上微不足道的变化”。当罗马崩溃时,奥多阿塞尔说罗马仍然存在。他和几乎所有人都热衷于假装什么都没有改变。他们知道,“罗马的荣耀”比正在取而代之的野蛮更好。即使是野蛮人也这么认为。正如C·W·普雷维特-奥顿在《剑桥中世纪简史》中所写的,五世纪末,当“以野蛮德意志国王取代皇帝时”,这是一个“持久的虚假”。

\end{paracol}

\subsection{"Persistent Make-Believe"}
\begin{paracol}{2}
This "make-believe" involved the preservation of the facade of the old system, even as its essence was "deformed by barbarism." The old forms of government remained the same when the last emperor was replaced by a barbarian "lieutenant." The Senate still met. "The praetorian prefecture and other high offices continued, and were held by eminent Romans." Consuls were still nominated for a year. "The Roman civil administration survived intact." Indeed, in some ways it remained intact until the birth of feudalism at the end of the tenth century. On public occasions, the old imperial insignia was still employed. Christianity was still the state religion. The barbarians still pretended to owe fealty to the Eastern emperor in Constantinople, and to the traditions of Roman law. In fact, in Durant's words, "in the West the great Empire was no more." 

\switchcolumn
这种“虚假”包括保留旧体制的外貌,而其实质则被“野蛮化”了。最后一位皇帝被野蛮的“副官”代替时,旧的政府形式保持不变。参议院仍然开会。“执政府和其他高职继续存在,并由杰出的罗马人担任。”每年仍然会提名执政官。“罗马的文职行政管理能力仍然完好无损。”事实上,在某些方面,它一直保持完好无损,直到十世纪末封建主义的诞生。在公共场合,仍然使用旧的皇帝标志。基督教仍然是国教。野蛮人仍然假装向位于君士坦丁堡的东部皇帝和罗马法的传统效忠。事实上,在杜兰特的话中,“在西方,伟大的帝国不存在了”。

\end{paracol}

  
\subsection{所以呢?} 
\begin{paracol}{2}
The faraway example of the fall of Rome is relevant for a number of reasons as you contemplate conditions in the world today. Most books about the future are really books about the present. We have sought to remedy that defect by making this book about the future first of all a book about the past. We think that you are likely to draw a better perspective about what the future has in store if we illustrate important megapolitical points about the logic of violence with real examples from the past. History is an amazing teacher. The stories it has to tell are more interesting than any we could make up. And many of the more interesting relate to the fall of Rome. They document important lessons that could be relevant to your future in the Information Age.  

\switchcolumn 
罗马帝国的远古衰亡例子对于你思考现今世界的情况而言是相关的。大多数有关未来的书籍实际上都是有关现在的书籍。我们通过使这本书首先成为一本关于过去的书籍来解决这个问题。我们认为,如果我们通过从过去的真实例子中阐明有关暴力逻辑的重要大政治点,你将更有可能得出有关未来的更好视角。历史是一位惊人的老师。它所讲述的故事比我们能够编造的故事更有趣。其中许多有趣的故事与罗马帝国的衰亡有关。它们记录了可以与信息时代的未来相关的重要经验教训。

\switchcolumn*
First of all, the fall of Rome is one of history's more vivid examples of what happened in a major transition when the scale of government was collapsing. The transitions of the year 1000 also jnvolved the collapse of central authority, and did so in a way that increased the complexity and scope of economic activity. The Gunpowder Revolution at the end of the fifteenth century involved major changes in institutions that tended to raise rather than shrink the scale of governance. Today, for the first time in a thousand years, megapolitical conditions in the West are undermining and destroying governments, corporate conglomerates, labor unions, and many other institutions that operate on a large scale.  
  
\switchcolumn
首先,罗马帝国的衰亡是历史上更生动的例子之一,展示了在大规模政府崩溃时会发生什么。公元1000年的转型同样牵涉到中央权威的崩溃,而且这种崩溃方式会增加经济活动的复杂性和范围。15世纪末的火药革命带来了机构上的重大变化,这些变化倾向于扩大而不是缩小治理的范围。今天,在西方,大政治条件首次在一千年内削弱和摧毁了政府、企业集团、工会以及许多其他进行大规模运作的机构。

\switchcolumn*
Of course, the collapse in the scale of governance at the end of the Roman Empire had very different causes from those prevailing now, at the advent of the Information Age. Part of the reason that Rome fell is simply that it had expanded beyond the scale at which the economies of violence could be maintained. The cost of garrisoning the empire's far-flung borders exceeded the economic advantages that an ancient agricultural economy could support. The burden of taxation and regulation required to finance the military effort rose to exceed the carrying capacity of the economy. Corruption became endemic. A large part .of the effort of military commanders, as historian Ramsay MacMullen has documented, was devoted to pursuit of "illicit profits of their command." This they pursued by shaking down the population, what the fourth-century observer Synesius described as "the peace-time war, one almost worse than the barbarian war and arising from the military's indiscipline and the officers' greed."  
  
\switchcolumn
当然,与现今信息时代的来临时占主导地位的原因有所不同,罗马帝国衰落的原因也是如此。罗马的衰落部分原因在于,其疆界扩张超出了维持暴力经济的规模。驻扎帝国遥远边界的成本超过了古代农业经济能够支持的经济优势。为了筹资军事行动所需的税收和监管负担超过了经济承载能力。腐败变得普遍存在。历史学家Ramsay MacMullen所记录的大部分军事指挥官的努力都是为追求“非法利润而奋斗的”。他们通过向人口敲诈勒索盈利。这就是四世纪观察家Synesius所描述的“和平时期的战争,这几乎比野蛮人战争更糟糕的战争,它源于军队的纪律性和军官的贪婪”。

\switchcolumn*
Another important contributing factor to Rome's collapse was a demographic deficit caused by the Antonine plagues. The collapse of the Roman population in many areas obviously contributed to economic and military weakness. Nothing of that kind has happened today, at least not yet. Taking a longer view, perhaps, the scourge of new "plagues" will compound the challenges oftechnological devolution in the new millennium. The unprecedented bulge in human population in the twentieth century creates a tempting target for rapidly mutating microparasites. Fears about the Ebola virus, or something like it, invading metropolitan populations may be well founded. But this is not the place to consider the coevolution of humans and diseases. As interesting a topic as that is, our argument at this juncture is not about why Rome fell, or even about whether the world today is vulnerable to some of the same influences that contributed to Roman decline. It is about something different -- namely, the way that history's great transformations are perceived, or rather, misperceived as they happen.

\switchcolumn
罗马文明崩溃的另一个重要因素是安敦尼瘟疫导致的人口赤字。许多地区罗马人口的崩溃显然导致了经济和军事实力的削弱。至少目前为止,没有发生类似的事情。从更长远的角度来看,新“瘟疫”的肆虐可能会加剧新千年技术倒退的挑战。 20世纪人类规模前所未有的增长创造了一个诱人的目标,那就是迅速突变的微生物寄生虫。对埃博拉病毒或类似病毒入侵都市人口的担忧可能是合理的。但现在不是考虑人类与疾病共同进化的地方。尽管这是一个有趣的话题,但是我们在这里的论点并不是关于为什么罗马崩溃了,甚至也不涉及今天世界是否容易受到导致罗马衰落的相同影响的问题。它与不同的事情有关 -- 也就是说,历史上伟大转变的感知方式,或者更确切地说,它们发生时的误解。


\switchcolumn*
People are always and everywhere to some degree conservative, with a small "c." That implies a reluctance to think in terms of dissolving venerable social conventions, overturning the accepted institutions, and defying the laws and values from which they drew their bearings. Few are inclined to imagine that apparently minor changes in climate or technology or some other variable can somehow be responsible for severing connections to the world of their fathers. The Romans were reluctant to acknowledge the changes unfolding around them. So are we.

\switchcolumn
人们总是在某种程度上保守,带有一点“c”,这意味着不愿意考虑消解尊敬的社会传统、推翻被接受的制度,以及违反他们获得指引的法律和价值观。几乎没有人愿意想象,气候、技术或其他一些变量的显然微小的变化可能会在某种程度上导致与他们父辈的世界断绝联系。罗马人不愿意承认正在发生的变化。我们也是如此。

\switchcolumn*
Yet recognize it or not, we are living through a change of historical season, a transformation in the way people organize their livelihoods and defend themselves that is so far-reaching that it will inevitably transform the whole of society. The change will be so profound, in fact, that to understand it will require taking almost nothing for granted. You will be invited at almost every tum to believe that the coming Information Societies will be very like the industrial society you grew up in. We doubt it. Microprocessing will dissolve the mortar in the bricks. It will so profoundly alter the logic of violence that it will inevitably change the way people organize their livelihoods and defend themselves. Yet the tendency will be to downplay the inevitability of these changes, or to argue about their desirability as if it were within the fiat of industrial institutions to determine how history evolves.

\switchcolumn
但是,认识到或不认识到,我们正在经历历史季节的变化,这种变化将是彻底的,将不可避免地改变整个社会。事实上,这种变化将如此深刻,以至于理解它几乎需要将几乎一切都视为理所当然。几乎每一天都会邀请您相信即将到来的信息社会将非常类似于您成长的工业社会。我们怀疑这一点。微处理将溶解砖中的灰泥。它将如此深刻地改变暴力的逻辑,以至于它必然改变人们组织其生计和自卫的方式。然而,趋势将是淡化这些变化的必然性,或者争论它们的可取性,仿佛工业机构能够决定历史的演变方式。

\end{paracol}

\subsection{The Grand Illusion}
\begin{paracol}{2}  
Authors who are in many ways better informed than we are will nevertheless lead you astray in thinking about the future because they are far too superficial in examining how societies work. For example, David Kline and Daniel Burstein have written a well-researched volume entitled \emph{Road Warriors: Dreams and Nightmares Along the Information Highway}. It is full of admirable detail, but much of this detail is marshaled in arguing an illusion, the idea "that citizens can act together, consciously, to shape the spontaneous economic and natural processes going on around them." Although it may not be obvious, this is equivalent to saying that feudalism might have survived if everyone had rededicated himself to chivalry. No one in a court of the late fifteenth century would have objected to such a sentiment. Indeed, it would have been heresy to do so. But it also would have been entirely misleading, an example of the snake trying to fit the future into its old skin.

\switchcolumn
很多方面比我们更了解的作者仍然会误导您对未来的思考,因为他们在研究社会运作方面过于肤浅。例如,大卫·克莱因和丹尼尔·伯斯汀写了一本名为《公路勇士:信息高速公路上的梦想与噩梦》的良心力作。它充满了令人钦佩的细节,但这些细节中有很多是在争论一种错觉,即“公民可以共同有意识地塑造周围的经济和自然过程”。虽然这可能不明显,但这相当于说如果每个人都重新致力于骑士精神,封建主义可能会存活下来。15世纪晚期的法院中没有任何人会反对这种情感。实际上,这将是异端邪说。但它也是完全具有误导性的,它是蛇试图将未来适应其旧皮肤的一个例子。

\switchcolumn*
The basic causes of change are precisely those that are not subject to conscious control. They are the factors that alter the conditions under which violence pays. Indeed, they are so remote from any obvious means of manipulation that they are not even subjects of political maneuvering in a world saturated with politics. No one ever marched in a demonstration shouting, "Increase scale economies in the production process." No banner has ever demanded, "Invent a weapons system that increases the importance of the infantry." No candidate ever promised to "alter the balance between efficiency and magnitude in protection against violence." Such slogans would be ridiculous, precisely because their goals are beyond the capacity of anyone to consciously affect. Yet as we will explore, these variables determine how the world works to a far greater degree than any political platform.    

\switchcolumn
变革的基本原因确切地说是无法受到意识控制的。它们是改变暴力得分条件的因素。实际上,它们与任何明显的操纵手段相距甚远,甚至在一个充斥政治的世界里也不是政治策略的主题。从未有人在示威中喊着:“增加生产过程中的规模经济”。也从未有过要求:“发明一种武器系统,增加步兵的重要性”的横幅。也从未有任何候选人承诺“在保护暴力方面改变效率和规模的平衡”。这些口号是荒谬的,因为它们的目标超出了任何人有意识地影响的能力范围。然而,正如我们将探讨的那样,这些变量在更大程度上决定了世界的运作方式,而不仅仅是任何政治平台。

\switchcolumn*
If you think about it carefully, it should be obvious that important transitions in history seldom are driven primarily by human wishes. They do not happen because people get fed up with one way of life and suddenly prefer another. A moment's reflection suggests why. If what people think and desire were the only determinants of what happens, then all the abrupt changes in history would have to be explained by wild mood swings unconnected to any change in the actual conditions of life. In fact, this never happens. Only in cases of medical problems affecting a few people do we see arbitrary fluctuations in mood that appear entirely divorced from any objective cause.

\switchcolumn
如果你仔细思考,就应该明显,历史上重要的转变很少主要是由人类意愿驱动的。它们并不是因为人们对一种生活方式感到厌倦,突然就转换了。一个瞬间的反思就能说明这一点的原因。如果人们的想法和愿望是发生的唯一决定因素,那么历史上所有突然的变化都必须用无法连接到任何实际生活条件变化的狂躁情绪来解释。实际上,这从来没有发生过。只有在影响少数人的医疗问题的情况下,我们才会看到情绪的任意波动,这些波动似乎完全脱离了任何客观原因。

\switchcolumn*
As a rule, large numbers of people do not suddenly and all at once decide to abandon their way of life simply because they find it amusing to do so. No forager ever said, "I am tired of living in prehistoric times, I would prefer the life of a peasant in a farming village." Any decisive swing in patterns of behavior and values is invariably a response to an actual change in the conditions of life. In this sense, at least, people are always realistic. If their views do change abruptly, it probably indicates that they have been confronted by some departure from familiar conditions: an invasion, a plague, a sudden climatic shift, or a technological revolution that alters their livelihoods or their ability to defend themselves.

\switchcolumn
通常情况下,大量的人不会突然一齐决定放弃他们的生活方式,只是因为他们发现这样做很有趣。没有一个采集者会说:“我厌倦了生活在史前时代,我宁愿过一个在农村村庄里的农民的生活。”行为和价值观决策的决定性转变往往是对实际生活条件的变化的反应。在这个意义上,至少人们总是现实的。如果他们的观点突然发生了变化,这可能表明他们已经面临了一些熟悉条件的改变:入侵、瘟疫、突然的气候转变或技术革命,这些都影响了他们的生计或他们保卫自己的能力。

\switchcolumn*
Far from being the product of human desire, decisive historic changes more often than not confound the wish of most people for stability. When change occurs, it typically causes widespread disorientation, especially among those who lose income or social status. You will look in vain at public opinion polls or other measures of mood for an understanding of how the coming megapolitical transition is likely to unfold.

\switchcolumn
决定性的历史变革远非人性的产物,往往会让大多数人渴望稳定的愿望不成反而陷入困惑。当变革发生时,通常会引起广泛的失望,尤其是那些失去收入或社会地位的人。你徒劳地在民意调查或其他情绪测量中寻找理解即将到来的大政治转变的方法。

\end{paracol}

\section{没有远见的生活}

\begin{paracol}{2}
If we fail to perceive the great transition going on around us, it is partly because we do not desire to see. Our foraging forebears may have been just as obdurate, but they had a better excuse. No one ten thousand years ago could have foreseen the consequences of the Agricultural Revolution. Indeed, no one could have foreseen much of anything beyond where to find the next meal. When farming began, there was no record of past events from which to draw perspective on the future. There was not even a Western sense oftime divided into orderly units, like seconds, minutes, hours, days, and so on, to measure out the years. Foragers lived in the "eternal present," without calendars, and indeed, without written records at all. They had no science, and no other intellectual apparatus for understanding cause and effect beyond their own intuitions. When it came to looking ahead, our primeval ancestors were blind. To cite the biblical metaphor, they had not yet eaten of the fruit of knowledge.    

\switchcolumn
如果我们未能感知到正在我们身边发生的巨大变革,这部分原因在于我们不愿意看见。我们的采集祖先可能也同样顽固,但他们有更好的借口。一万年前没有人能够预见到农业革命的后果,实际上,没有人能够预见未来的很多事情,即使是下一个饭食从哪里得到。农业开始时,并没有过去事件的记录可以借助,以此来预见未来。甚至没有西方的时间概念,将时间划分为有序的单位,如秒、分钟、小时、日等,来测量年份。采集者生活在“永恒的现在”,没有日历,事实上,没有任何书面记录。他们没有科学,也没有任何其他的知识体系来理解因果关系,除了他们自己的直觉。当我们看向未来时,我们的原始祖先是盲目的,以圣经的隐喻为例,他们还没有吃过知识之果。

\end{paracol}

\subsection{从过去学习}
\begin{paracol}{2}
Luckily, we have a better vantage point. The past five hundred generations have given us analytic capabilities that our forebears lacked. Science and mathematics have helped unlock many of nature's secrets, giving us an understanding of cause and effect that approaches the magical when compared to that of the early foragers. Computational algorithms developed as a result of high-speed computers have shed new insights on the workings of complex, dynamic systems like the human economy. The painstaking development of political economy itself, although it falls well short of perfection, has honed understanding of the factors informing human action. Important among these is the recognition that people at all times and places tend to respond to incentives. Not always as mechanically as economists imagine, but they do respond. Costs and rewards matter. Changes in external conditions that raise the rewards or lower the costs of certain behavior will lead to more of that behavior, other things being equal.

\switchcolumn
幸运的是,我们拥有更好的视角。过去的500代人给予我们比我们的祖先更好的分析能力。科学和数学帮助我们解开了许多自然的秘密,使我们对因果关系的理解接近于魔法,尤其是与早期采集者相比。由于高速计算机的发展,计算机算法对于像人类经济这样的复杂动态系统的运作方面带来了新的见解。政治经济学的艰苦发展,虽然远未达到完美,但已经磨练了对影响人类行为的因素的理解。其中一个重要的内容是认识到,在所有时间和地点,人们都倾向于响应激励。虽然并非总是像经济学家所想象的那样机械化,但他们确实做出了回应,成本和奖励很重要。当外部条件的变化提高某种行为的奖励或降低成本时,其他条件保持不变,将会导致该行为的增加。

\end{paracol}

\subsection{刺激很重要}
\begin{paracol}{2}
The fact that people tend to respond to costs and rewards is an essential element of forecasting. You can say with a high degree of confidence that if you drop a hundred-dollar bill on the street, someone will soon pick it up, whether you are in New York, Mexico City, or Moscow. This is not as trivial as it seems. It shows why the clever people who say that forecasting is impossible are wrong. Any forecast that accurately anticipates the impact of incentives on behavior is likely to be broadly correct. And the greater the anticipated change in costs and rewards, the less trivial the implied forecast is likely to be.    

\switchcolumn
人们往往会对成本和奖励做出反应,这是预测的一个重要因素。例如,你可以非常自信地说,如果你在纽约、墨西哥城或莫斯科的街上掉落了一张一百美元的钞票,很快就会有人来捡起来。这并不是看似微不足道的事情。这表明了那些聪明人说预测是不可能的是错误的。任何能够准确预见激励对行为的影响的预测,可能大体上是正确的。而且,预期成本和奖励的变化越大,预期的预测变化就越不微不足道。

\switchcolumn*
The most far-reaching forecasts of all are likely to arise from recognizing the implications of shifting megapolitical variables. Violence is the ultimate boundary force on behavior; thus, if you can understand how the logic of violence will change, you can usefully predict where people will be dropping or picking up the equivalent of one-hundred-dollar bills in the future.    

\switchcolumn
所有最深远的预测可能都源于认识到政治变量转移的含义。暴力是行为的最终边界力量;因此如果你能理解暴力逻辑将如何变化,你就可以有用地预测未来人们将在哪里弃置或拾起一百美元的等值物品。

\switchcolumn*
We do not mean by this that you can know the unknowable. We cannot tell you how to forecast winning lottery numbers or any truly random event. We have no way of knowing when or whether a terrorist will detonate an atomic blast in Manhattan. Or if an asteroid will strike Saudi Arabia. We cannot predict the coming of a new Ice Age, a sudpen volcanic eruption, or the emergence of a new disease. The number of unknowable events that could alter the course of history is large. But knowing the unknowable is very different from drawing out the implications of what is already known. If you see a flash of lightning far away, you can forecast with a high degree of confidence that a thunderclap is due. Forecasting the consequences of megapolitical transitions involves much longer time frames, and less certain connections, but it is a similar kind of exercise.    

\switchcolumn
这并不意味着你可以知道不可知的事情。我们不能告诉你如何预测中奖彩票号码或任何真正的随机事件。我们无法知道恐怖分子何时或是否会在曼哈顿引爆原子爆炸。或者是否有一颗小行星会撞击沙特阿拉伯。我们无法预测新的冰河时期、大规模火山爆发或出现新疾病。可能改变历史进程的不可知事件数量很大。但是,知道不可知的事情与推断已知事物的影响是截然不同的。如果你看到远处闪电,你可以非常有信心地预测一声雷鸣即将响起。预测大政治转型的后果涉及更长时间范围和不太确定的联系,但它是一种类似的锻炼。

\switchcolumn*
Megapolitical catalysts for change usually appear well before their consequences manifest themselves. It took five thousand years for the full implications of the Agricultural Revolution to come to the surface. The transition from an agricultural society to an industrial society based on manufacturing and chemical power unfolded more quickly. It took centuries. The transition to the Information Society will happen more rapidly still, probably within a lifetime. Yet even allowing for the foreshortening of history, you can expect decades to pass before the full megapolitical impact of existing information technology is realized.

\switchcolumn
导致变化的大政治催化剂通常在其后果显现之前就已经出现了。完全意义上的农业革命需要五千年才能浮出水面。从基于制造业和化学制品动力的农业社会向工业社会的转变进展更快。它需要数个世纪的时间。向信息社会的转变将更快地发生,可能在一生内。然而,即使缩短了历史,你也可以期望几十年过去,才能实现现有信息技术的完整大政治影响。

\end{paracol}

\subsection{重要和次要的大政治转变}

\begin{paracol}{2}
This chapter analyzes some of the common features of megapolitical transitions. In following chapters we look more closely at the Agricultural Revolu- tion, and the transition from farm to factory, the second of the previous great phase changes. Within the agricultural stage of civilization there were many minor megapolitical transitions such as the fall of Rome and the feudal revolution of the year 1000. These marked the waxing and waning of the power equation as governments rose and fell and the spoils of farming passed from one set of hands to another. The owners of sprawling estates under the Roman Empire, yeoman farmers in the European Dark Ages, and the lords and serfs of the feudal period all ate grain from the same fields. They lived under very different governments because of the cumulative impact of different technologies, fluctuations in climate, and the disruptive influences of disease.

\switchcolumn
这一章分析了城市政治转变的一些常见特征。在接下来的章节中,我们会更仔细地看待农业革命和从农场到工厂的转变,即前一次伟大的阶段性变革中的第二个转变。在文明的农业阶段中,有许多次较小的城市政治转变,如罗马帝国的崩溃和公元1000年封建革命。这标志着政府的兴衰和农业战利品从一组人手中流入另一组人手中。罗马帝国广阔庄园的所有者、欧洲黑暗时期的自耕农,以及封建时期的领主和农奴都在同一片田地上种植谷物。他们生活在非常不同的政府统治下,这是由于不同技术、气候波动以及疾病的破坏性影响的累积影响造成的。

\switchcolumn*
Our purpose is not to thoroughly explain all of these changes. We do not pretend to do so, although we have sketched out some illustrations of the way that changing megapolitical variables have altered the way that power was exercised in the past. Governments have grown and shrunk as megapolitical fluctuations have lowered and raised the costs of projecting power.Here are some summary points that you should keep in mind as you seek to understand the Information Revolution:

\switchcolumn
我们的目的并不是彻底解释所有这些变化。虽然我们已经勾勒出了一些变化的方式,但我们并不假装完全做到了这一点。随着城市政治波动降低和提高实行权力的成本,政府的规模不断扩大和缩小。以下是一些摘要观点,您应该将这些观点牢记在心,以便更好地理解信息革命:

\switchcolumn*
\begin{itemize}
  \item A shift in the megapolitical foundations of power normally unfolds far in advance of the actual revolutions in the use of power.
  \item Incomes are usually falling when a major transition begins, often because a society has rendered itself crisis-prone by marginalizing resources due to population pressures.
  \item Seeing "outside" of a system is usually taboo. People are frequently blind to the logic of violence in the existing society; therefore, they are almost always blind to changes in that logic, latent or overt. Megapolitical transitions are seldom recognized before they happen.
  \item Major transitions always involve a cultural revolution, and usually entail clashes between adherents ofthe old and new values.
  \item Megapolitical transitions are never popular, because they antiquate painstakingly acquired intellectual capital and confound established moral imperatives. They are not undertaken by popular demand, but in response to changes in the external conditions that alter the logic of violence in the local setting.
  \item Transitions to new ways of organizing livelihoods or new types of government are initially confined to those areas where the megapolitical catalysts are at work.
  \item With the possible exception ofthe early stages offarming, past transitions have always involved periods of social chaos and heightened violence due to disorientation and breakdown of the old system.
  \item Corruption, moral decline, and inefficiency appear to be signal features of the final stages of a system.
  \item The growing importance of technology in shaping the logic of violence has led to an acceleration of history, leaving each successive transition with less adaptive time than ever before.
\end{itemize}

\switchcolumn
\begin{itemize}
  \item 实际引起权力转变的转型在城市政治基础方面通常会提前很久发生。
  \item 当一个主要的转型开始时,收入通常会下降,往往是因为一个社会由于人口压力而使资源边缘化,从而使自身陷入危机。
  \item 看待系统的“外部”通常是禁忌的。人们经常看不到现有社会暴力逻辑的盲点,因此他们几乎总是盲目地看不到这种逻辑的潜在或明显的变化。大都市政治转变很少在发生之前被认识到。
  \item 主要的转变总是涉及文化革命,并且通常涉及到支持旧价值观和新价值观的人之间的冲突。
  \item 大都市政治转变从来都不受欢迎,因为它们会淘汰艰苦学得的智力资本并迷惑既有的道德准则。它们并非是应市民的要求而进行的,而是对外部条件变化的反应,这些变化改变了本地暴力逻辑。
  \item 转变为新的谋生方式或新的政府类型最初仅限于大都市政治催化剂起作用的地区。
  \item 在农业的早期阶段可能除外,过去的转变总是涉及到社会混乱和暴力加剧的时期,这是由于旧的系统失序和崩溃所导致的。
  \item 腐败、道德沦丧和低效似乎是一个系统的最后阶段的显著特征。
  \item 技术在塑造暴力逻辑方面的日益重要作用加速了历史,使得每个后续转型的适应时间都比以往任何时候都要短。
\end{itemize}
\end{paracol}

\subsection{历史加速了}
\begin{paracol}{2}
With events unfolding many times faster than during previous transformations, early understanding of how the world will change could turn out to be far more useful to you than it would have been to your ancestors at an equivalent juncture in the past. Even if the first farmers had miraculously understood the full megapolitical implications of tilling the earth, this information would have been practically useless because thousands of years were to pass before the transition to the new phase of society was complete.

\switchcolumn
随着事件的发展比之前的转型快很多倍,对于世界将如何改变的早期理解对你来说可能比对你祖先在相同时期的理解要有用得多。即使最初的农民神奇地理解了耕地的全部超级政治影响,这个信息也会实际上是无用的,因为成千上万年要过去,直到社会的新阶段的转型完成。

\switchcolumn*
Not so today. History has sped up. Forecasts that correctly anticipate the megapolitical implications of new technology are likely to be far more useful today. If we can develop the implications of the current transition to the Information Society to the same extent that someone with current knowledge could have grasped the implications of past transi~ions to farm and factory, that information should be many times more valuable now. Put simply, the action horizon for megapolitical forecasts has shrunk to its most useful range, within the span of a single lifetime.

\switchcolumn
而今天则不同了。历史加速了。能正确预测新技术的超级政治影响的预测可能会更有用。如果我们能够像拥有当前知识的人一样深入发展信息社会的当前转型的影响,就像某些过去的转型到农场和工厂的影响一样,那么这个信息现在可能会有多倍的价值。简而言之,超级政治预测的行动地平线已经缩短到最有用的范围,即单个生命的时间跨度。

\switchcolumn*
Our study of megapolitics is an attempt to do just that --to draw out the implications of the changing factors that alter the boundaries where violence is exercised. These megapolitic~l factors largely determine when and where violence pays. They also help inform the market distribution of income. As economic historian Frederic Lane so clearly put it, how violence is organized and controlled plays a large role in determining "what uses are made of scarce resources."

\switchcolumn
我们对城市政治的研究正是试图做到这一点——揭示改变暴力实施边界的因素所带来的影响。这些城市政治因素主要决定了何时、在何地暴力才是有利可图的。它们还有助于说明收入的市场分布情况。正如经济历史学家弗雷德里克·莱恩所明确阐述的那样,暴力如何组织和控制在决定“稀缺资源的使用方式”方面发挥着重要作用。

\end{paracol}

\section{A CRASH COURSE IN MEGAPOLITICS}
\begin{paracol}{2}
The concept of megapolitics is a powerful one. It helps illuminate some of the major mysteries of history: how governments rise and fall and what types of institutions they become; the timing and outcome of wars; patterns of economic prosperity and decline. By raising or lowering the costs and rewards of projecting power, megapolitics governs the ability of people to impose their will on others. This has been true from the earliest human societies onward. It still is. We explored many of the important hidden megapolitical factors that determine the evolution of history in \emph{Blood in the Streets} and \emph{The Great Reckoning}. The key to unlocking the implications of megapolitical change is understanding the factors that precipitate revolutions in the use of violence. These variables can be somewhat arbitrarily grouped into four categories: topography, climate, microbes, and technology.

\switchcolumn
《街头流血》和《大恐慌》中我们探讨了很多重要的隐含的城市政治因素,这些因素决定了历史的演变。揭示城市政治变革的影响的关键在于理解导致暴力使用革命的因素。这些变量可以被任意地分成四类:地形、气候、微生物和技术。

\switchcolumn*
1. \textbf{Topography} is a crucial factor, as evidenced by the fact that control of violence on the open seas has never been monopolized as it has on land. No government's laws have ever exclusively applied there. This is a matter of the utmost importance in understanding how the organization of violence and protection will evolve as the economy migrates into cyberspace.

Topography, in conjunction with climate, had a major role to play in early history. The first states emerged on floodplains, surrounded by desert, such as in Mesopotamia and Egypt, where water for irrigation was plentiful but surrounding regions were too dry to support yeoman farming. Under such conditions, individual farmers faced a very high cost for failing to cooperate in maintaining the political structure. Without irrigation, which could be provided only on a large scale, crops would not grow. No crops meant starvation. The conditions that placed those who controlled the water in a desert in a position of strength made for despotic and rich government.

As we analyzed in \emph{The Great Reckoning}, topographic conditions also played a major role in the prosperity of yeoman farmers in ancient Greece, enabling that region to become the cradle of Western democracy. Given the primitive transportation conditions prevailing in the Mediterranean region three thousand years ago, it was all but impossible for persons living more than a few miles from the sea to compete in the production of high-value crops of the ancient world, olives and grapes. If the oil and the wine had to be transported any distance overland, the portage costs were so great that they could not be sold at a profit. The elaborate shoreline of the Greek littoral meant that most areas of Greece were no more than twenty miles from the sea. This gave a decisive advantage to Greek farmers over their potential competitors in landlocked areas.

Because of this advantage in trading high-value products, Greek farmers earned high incomes from control of only small parcels of land. These high incomes enabled them to purchase costly armor. The famous hoplites of ancient Greece were farmers or landlords who armed themselves at their own expense. Both well armed and well motivated, the Greek hoplites were militarily formidable and could not be ignored. Topographic conditions were the foundation of Greek democracy, just as those of a different kind gave rise to the Oriental despotisms of Eygpt and elsewhere.


\switchcolumn
1.\textbf{地形}是至关重要的因素,如控制海上暴力从未像在陆地上那样垄断。没有政府的法律曾经被独家应用在那里,这对于理解暴力和保护的组织方式随着经济转移到网络空间的演变而演变的方式至关重要。

地形与气候结合在一起,在早期历史中起着重要作用。第一个国家出现在洪水平原上,被沙漠所包围,如在美索不达米亚和埃及,那里有大量用于灌溉的水源,但周围地区太干旱无法支持自给自足的农业。在这种情况下,个体农民面临着维护政治结构失败的极高成本。没有灌溉,只有在大规模下才能提供,庄稼就不会长。没有庄稼意味着饥饿。这些控制沙漠中的水源的人处于优势地位的条件导致了专制和富裕的政府。

正如我们在《大清算》中分析的那样,地形条件也在希腊古代独立农民的繁荣中起着重要作用,使该地区成为西方民主的摇篮。在地中海地区三千年前盛行的原始交通条件下,生活在离海岸线几英里以上的地方几乎不可能竞争古代世界中的高价值作物,如橄榄和葡萄。如果必须将油和酒运输到远距离的内陆地区,搬运成本是如此之高,以至于无法盈利销售。希腊的复杂海岸线意味着希腊的大多数地区距离海岸线不超过二十英里。这使得希腊农民在潜在的内陆竞争对手面前具有决定性的优势。

由于在高价值产品交易方面的优势,希腊农民可以从控制仅有的小块土地中获得高收入。这些高收入使他们能够购买昂贵的盔甲。古希腊著名的重装步兵是农民或地主,他们自费武装自己。希腊的重装步兵既装备精良又具有很强的动力,军事上是令人敬畏的,不能被忽视。地形条件是希腊民主的基础,就像不同种类的地形条件导致了埃及和其他地方的东方专制主义一样。

\switchcolumn*
2. \textbf{Climate} also helps set the boundaries within which brute force can be exercised. A climatic change was the catalyst for the first major transition from foraging to farming. The end of the last Ice Age, about thirteen thousand years ago, led to a radical alteration in vegetation. Beginning in the Near East, where the Ice Age retreated first, a gradual rise in temperature and rainfall spread forests into areas that had previously been grasslands. In particular, the rapid spread of beech forests seriously curtailed the human diet. As Susan Alling Gregg put it in \emph{Foragers and Farmers}:

%\footnotesize{\emph{The establishment of beech forests must have had serious consequences for local human, plant and animal populations. The canopy of an oak forest is relatively open and allows large amounts of sunlight to reach the forest floor. An exuberant undergrowth of mixed shrubs, forbs, and grasses develops, and the diversity of plants supports a variety of wildlife. In contrast, the canopy of a beech forest is closed and the forest floor is heavily shaded. Other than a flush of spring annuals prior to the emergence of the leaves, only shade-tolerant sedges, ferns, and a few grasses are found.}}

Over time, dense forests encroached on the open plains, spreading throughout Europe into the Eastern steppes. The forests reduced the grazing area available to support large animals, making it increasingly difficult for the population of human foragers to support themselves.

The population of hunter-gatherers had swollen too greatly during the Ice Age prosperity to support itself on the dwindling herds of large mammals, many species of which were hunted to extinction. The transition to agriculture was not a choice of preference, but an improvisation adopted under duress to make up for shortfalls in the diet. Foraging continued to predominate in those areas farther north, where the warming trend had not adversely affected the habitats of large mammals, and in tropical rainforests, where the global warming trend did not have the perverse effect of reducing food supplies. Since the advent of farming, it has been far more common for changes to be precipitated by the cooling rather than the warming of the climate.

A modest understanding of the dynamics of climatic change in past societies could well prove useful in the event that climates continue to fluctuate. If you know that a drop of one degree Centigrade on average reduces the growing season by three to four weeks and shaves five hundred feet off the maximum elevation at which crops can be grown, then you know something about the boundary conditions that will confine people's action in the future. You can use this knowledge to forecast changes in everything from grain prices to land values. You may even be able to draw informed conclusions about the likely impact of falling temperatures on real incomes and political stability. In the past, governments have been overthrown when crop failures extending over several years raised food prices and shrank disposable incomes.

For example, it is no coincidence that the seventeenth century, the coldest in the modem period, was also a period of revolution worldwide. A hidden megapolitical cause of this unhappiness was sharply colder weather. It was so cold, in fact, that wine froze on the "Sun King's" table at Versailles. Shortened growing seasons produced crop failures and undermined real income. Because of the colder weather, prosperity began to wind down into a long global depression that began around 1620. It proved drastically destabilizing. The economic crisis of the seventeenth century led to the world being overwhelmed by rebellions, many clustering in 1648, exactly two hundred years before another and more famous cycle of rebellions. Between 1640 and 1650, there were rebellions in Ireland, Scotland, England, Portugal, Catalonia, France, Moscow, Naples, Sicily, Brazil, Bohemia, Ukraine, Austria, Poland, Sweden, the Netherlands, and Turkey. Even China and Japan were swept with unrest.

It may also be no coincidence that mercantilism predominated in the seventeenth century during a period of shrinking trade. Economic closure was perhaps most pronounced at the end of the century, "when a terrible famine occurred." By the eighteenth century, especially after 1750, warmer temperatures and higher crop yields had begun to raise real incomes in Western Europe sufficiently to expand demand for manufactured goods. More free-market policies were adopted. This led to a self-reinforcing burst of economic growth as industry expanded to a larger scale in what is commonly described as the Industrial Revolution. The growing importance of technology and manufactured output reduced the impact of the weather on economic cycles.

Even today, however, you should not underestimate the impact of suddenly colder weather in lowering real incomes --even in wealthy regions such as North America. There is a strong tendency for societies to render themselves crisis-prone when the existing configuration of institutions has exhausted its potential. In the past, this tendency has often been manifested by population increases that stretched the carrying capacity of land to the limit. This happened both before the transition of the year 1000 and again at the end of the fifteenth century. The plunge in real income caused by crop failures and lower yields played a significant role in both instances in destroying the predominant institutions. Today the marginalization is manifested in the consumer credit markets. If sharply colder weather reduced crop yields and lowered disposable incomes, this would lead to debt default as well as tax rebellions. If the past is a guide, both economic closure and political instability could result.

\switchcolumn
2. \textbf{气候}也有助于设置可以行使蛮力的边界。气候变化是从采集到农业的第一个重大转变的催化剂。大约一万三千年前,最后一次冰河时期的结束导致植被发生了根本性的变化。从近东地区开始,随着冰河时期的撤退,温度和降雨量逐渐上升,将森林扩展到以前是草原的地区。特别是,山毛榉森林的快速扩散严重限制了人类的饮食。如苏珊·阿林·格雷格在《采集者和农民》中所说:

%\footnotesize{\emph{植被恢复为山毛榉森林对于当地的人、植物和动物种群可能有重大的后果。橡树林的树冠相对较开阔,能够让大量阳光照射到地面。丰富的灌木、草本和草类植被发展,植物多样性支持了各种野生动物。相比之下,山毛榉森林的树冠封闭且森林地面沉重阴暗。除了在叶片出现前的春季一批草本植物,只能看到能耐阴性的莎草、蕨类及少数草本植物。}}

随着时间的推移,茂密的森林蔓延至原本开阔的草原,并向东方草原延伸至整个欧洲。森林减少了大型动物的牧草生长区域,使得人类采集者的种群越来越难以维持生计。

在冰川时期的繁荣期,狩猎采集者的种群已经过度膨胀,无法支持自身在数量上的增长,许多大型哺乳动物物种被猎杀至灭绝。农业转型不是一种偏好的选择,而是在为弥补饮食的缺陷而迫不得已采取的措施。在更远北的地区,向北的趋势没有严重影响到大型哺乳动物的栖息地,因此继续采集和狩猎成为主要生产方式;在热带雨林中,全球变暖趋势并未产生减少食物供应的反效果,采集和狩猎也成为当地人们的主要生产方式。自从农业出现以来,气候变化更多地是由于气温变冷而不是变热引起的。

对过去社会气候变化动态的了解,对于未来气候继续波动时可能会证明有用。如果您知道平均温度下降一摄氏度会使种植季节缩短三至四周,会使作物的最大种植海拔高度减少五百英尺,那么您就了解到这些限制条件对未来行动的影响。您可以利用这些知识预测从谷物价格到土地价值的变化。您甚至可以基于相应的温度下降对实际收入和政治稳定性的影响得出明智的结论。在过去,由于长期几年的农作物歉收导致食品价格上涨和可支配收入下降,曾导致政府被推翻。


例如,不是巧合的是,现代时期最寒冷的17世纪是世界范围内的革命期。这种不幸的情况的一个隐藏的超级政治原因是异常寒冷的天气。事实上,气温如此之低,以至于凡尔赛宫“太阳国王”桌子上的葡萄酒都结冰了。缩短的生长季节导致了农作物收成的减少,破坏了实际收入。由于天气寒冷,繁荣逐渐转向接近1620年开始的全球长期的萧条。这证明了其极其不稳定的趋势。17世纪的经济危机导致了世界范围内的反叛,其中许多聚集在1648年,恰好是另一个更著名的反叛周期的正好两百年之前。在1640年到1650年期间,爆发了爱尔兰、苏格兰、英格兰、葡萄牙、加泰罗尼亚、法国、莫斯科、那不勒斯、西西里、巴西、波希米亚、乌克兰、奥地利、波兰、瑞典、荷兰和土耳其的反叛。甚至中国和日本也因骚动被席卷。

在十七世纪贸易缩减时期主导了重商主义,也许并非巧合。在世纪末发生可怕的饥荒时,经济封闭可能尤为明显。在18世纪,尤其是1750年之后,更温暖的气候和更高的农作物产量已经足以提高西欧的实际收入,以扩大对制成品的需求。采取了更多的自由市场政策。这导致了自我加强的经济增长爆发,因为工业在常常被描述为工业革命的更大规模扩张。技术和制造业产出的不断增长降低了天气对经济周期的影响。

然而,即使在像北美这样的富裕地区,突然变冷的天气降低实际收入的影响也不容小视。当现有机构配置已经耗尽潜力时,社会往往倾向于使自己陷入危机。过去,这种趋势常常表现为人口增长,将土地的承载能力推到极限。这在1000年前后的过渡期和15世纪末再次发生过。由于作物歉收和产量降低引起的实际收入下降,在两种情况下都对主导性的制度的破坏起了重要作用。今天,这种边缘化表现在消费信贷市场上。如果突然变冷的天气降低了作物产量并降低了可支配收入,这将导致债务违约和税收反叛。如过去所示,经济封闭和政治不稳定可能会导致。
    
\switchcolumn*
3. \textbf{Microbes} convey power to harm or immunity from harm in ways that have often determined how power was exercised. This was certainly the case in the European conquest of the New World, as we explored in \emph{The Great Reckoning}. European settlers, arriving from settled agricultural societies riddled with disease, brought with them relative immunity from childhood infections like measles. The Indians they encountered lived largely in thinly populated foraging bands. They possessed no such immunity and were decimated. Often, the greatest mortality occurred before white people even arrived, as Indians who first encountered Europeans on the coasts traveled inland with infections.

There are also microbiological barriers to the exercise of power. In \emph{Blood in the Streets}, we discussed the role that potent strains of malaria served in making tropical Africa impervious to invasion by white men for many centuries. Before the discovery of quinine in the mid-nineteenth century, white armies could not survive in malarial regions, however superior their weapons might have been.

The interaction between humans and microbes has also produced important demographic effects that altered the costs and rewards of violence. When fluctuations in mortality are high due to epidemic disease, famine, or other causes, the relative risk of mortality in warfare falls. The declining frequency of eruptions in death rates from the sixteenth century onward helps explain smaller family size and, ultimately, the far lower tolerance of sudden death in war today as compared to the past. This has had the effect of lowering the tolerance for imperialism and raising the costs of projecting power in societies with low birthrates.

Contemporary societies, comprising small families, tend to find even small numbers of battle deaths intolerable. By contrast, early modern societies were much more tolerant ofthe mortality costs associated with imperialism. Before this century, most parents gave birth to many children, some of whom were expected to die randomly and suddenly from disease. In an era when early death was commonplace, would-be soldiers and their familiesfaced the dangers of the battlefield with less resistance.

\switchcolumn
3.\textbf{微生物}以各种方式传达着可能导致伤害或免疫免受伤害的力量,这在许多情况下决定了权力的行使方式。在我们探讨的《大清算》中,欧洲殖民者来自遍布疾病的定居农业社会,带来了相对免疫于麻疹等儿童感染病的免疫力。他们遇到的印第安人主要生活在人迹稀少的采集族群中。他们没有这样的免疫力,因此被大量歼灭。通常,最大的死亡率出现在白人甚至还没到达之前,因为第一批遇到欧洲人的印第安人带着感染病到达内陆。

此外,微生物存在也限制了权力的行使。在我们探讨了强大疟疾菌株在多个世纪内使热带非洲对白人的入侵无法成功的角色《血在街头》中。在19世纪中叶奎宁被发现之前,白人军队在疟疾地区无法生存,无论他们的武器多么优越。

人类与微生物之间的相互作用还产生了重要的人口统计学影响,改变了暴力的成本和收益。当由于流行病,饥荒或其他原因导致的死亡率波动较大时,战争中的相对死亡风险则会降低。从16世纪以后死亡率爆发的频率不断下降,有助于解释家庭规模的缩小,最终,与过去相比,今天对于战争中突然死亡的容忍度大大降低,这降低了帝国主义的容忍度,提高了在人口出生率低的社会中推行权力的成本。

当代社会由小家庭组成,即使是小规模的战斗死亡也让人难以容忍。相比之下,早期的现代社会对帝国主义所带来的死亡代价更加容忍。在本世纪之前,大多数父母会生下许多子女,其中一些被视为可能会因疾病而随机突然死亡。在早死很常见的年代里,未来的士兵和他们的家庭会更不抵制战场上的危险。

\switchcolumn*
4. \textbf{Technology} has played by far the largest role in determining the costs and rewards ofprojecting power during the modem centuries. The argument of this book presumes it will continue to do so. Technology has several crucial dimensions:
\begin{itemize}
  \item A. \emph{Balance between offense and defense.} The balance between the offense and the defense implied by prevailing weapons technology helps determine the scale of political organization. When offensive capabilities are rising, the ability to project power at a distance predominates, jurisdictions tend to consolidate, and governments form on a larger scale. At other times, like now, defensive capabilities are rising. This makes it more costly to project power outside of core areas. Jurisdictions tend to devolve, and big governments break down into smaller ones.
  \item B. \emph{Equality and the predominance of the infantry.} A key feature determining the degree of equality among citizens is the nature of weapons technology. Weapons that are relatively cheap, can be employed by nonprofessionals, and enhance the military importance of infantry tend to equalize power. When Thomas Jefferson wrote that "all men are created equal," he was saying something that was much more true than a similar statement would have seemed centuries earlier. A farmer with his hunting rifle was not only as well armed as the typical British soldier with his Brown Bess, he was better armed. The farmer with the rifle could shoot at the soldier from a greater distance, and with greater accuracy than the soldier could return fire. This was a distinctly different circumstance from the Middle Ages, when a farmer with a pitchfork --he could not have afforded more --could scarcely have hoped to stand against a heavily armed knight on horseback. No one was writing in 1276 that "all men are created equal." At that time, in the most manifestly important sense, men were not equal. A single knight exercised far more brute force than dozens of peasants put together.
  \item C. \emph{Advantages and disadvantages of scale in violence.} Another variable that helps determine whether there are a few large governments or many small ones is the scale of organization required to deploy the prevailing weapons. When there are increasing returns to violence, it is more rewarding to operate governments at a large scale; therefore governments tend to get bigger. When a small group can command effective means of resisting an assault by a large group, which was the case during the Middle Ages, sovereignty tends to fragment. Small, independent authorities exercise many of the functions of government. As we explore in a latter chapter, we believe that the Information Age will bring the dawn of cybersoldiers, who will be heralds of devolution. Cybersoldiers could be deployed not merely by nation-states but by very small organizations, and even by individuals. Wars of the next millennium will include some almost bloodless battles fought with computers.
  \item D. \emph{Economies ofscale in production.} Another important factor that weighs in the balance in determining whether ultimate power is exercised locally or from a distance is the scale of the predominant enterprises in which people gain their livelihoods. When crucial enterprises can function optimally only when they are organized on a large scale in an encompassing trading area, governments that expand to provide such a setting for enterprises under their protection may rake off enough additional wealth to pay the costs of maintaining a large political system. Under such conditions, the entire world economy usually functions more effectively where one supreme world power dominates all others, as the British Empire did in the nineteenth century. But sometimes megapolitical variables combine to produce falling economies of scale. If the economic benefits of maintaining a large trading area dwindle, larger governments that previously prospered from exploiting the benefits of encompassing trading areas may begin to break apart --even if the balance of weaponry between offense and defense otherwise remains much as it had been.
  \item E. \emph{Dispersal of technology.} Still another factor that contributes to the power equation is the degree of dispersal of key technologies. When weapons or tools of production can be effectively hoarded or monopolized, they tend to centralize power. Even technologies that are essentially defensive in character, like the machine gun, proved to be potent offensive weapons, that contributed to a rising scale of governance during the period when they were not widely dispersed. When the European powers enjoyed a monopoly on machine guns late in the nineteenth century, they were able to use those weapons against peoples at the periphery to dramatically expand colonial empires. Later, in the twentieth century, when machine guns became widely available, especially in the wake of World War II, they were deployed to help destroy the power of empires. Other things being equal, the more widely dispersed key technologies are, the more widely dispersed power will tend to be, and the smaller the optimum scale of government.
\end{itemize}  

\switchcolumn
4.\textbf{技术}在决定现代世纪投射力量的成本和回报方面发挥了迄今为止最大的作用。本书的论点认为它将继续发挥作用。技术具有几个关键的维度:
\begin{itemize}
  \item A. 进攻与防御的平衡。随着攻势能力的增强,在支配武器技术的进攻和防御之间能够取得平衡,有助于确定政治组织的规模。此时能够以远距离投射力量的能力占主导地位,管辖区 tend 类往往会整合,政府形成较大规模。在其他时候,比如现在,防御能力正在上升。这使得在核心地区之外投射力量的成本更高。管辖区 tend 类往往会分散,大政府分裂为小政府。
  \item B.平等与步兵的优势。决定市民平等程度的一个关键特征是武器技术的性质。那些相对便宜,可由非专业人员使用,并增强步兵的军事重要性的武器,往往会使力量平等化。当托马斯·杰斐逊写下“所有人被创造平等”时,他说的比几个世纪以前类似的说法要真实得多。手持狩猎步枪的农夫不仅装备像典型的布朗贝斯武装英国士兵一样,而且更好。持枪的农民可以在更远的距离上射击士兵,并比士兵更精准。这与中世纪完全不同,当时一个只拿着草叉的农民——他买不起更多——几乎没法指望抵挡重装骑士的攻击。当时没有人写下“所有人被创造平等”。那时,在最明显的意义上,人们是不平等的。单个骑士的力量远远超过数十个农民的力量之和。
  \item C. 暴力规模的优劣势。决定政府数量是少数大政府还是众多小政府的另一个变量是部署主流武器所需的组织规模。当暴力收益不断增加时,大规模运作政府更具有回报性;因此,政府往往变得更大。当少数人能够指挥有效的方式抵抗大团体的攻击时(中世纪就是这种情况),主权往往会分裂。小型独立机构行使了许多政府职能。正如我们在后面的章节中所探讨的那样,我们相信信息时代将带来网络士兵的曙光,他们将是分权的先驱。网络士兵不仅可以由国家部署,还可以由非常小的组织甚至个人部署。下一千年的战争将包括一些几乎是用计算机进行的无血战斗。
  \item D. 生产规模效益。决定最终权力是地方行使还是由远处行使的另一个重要因素是人们谋取生计的主导企业的规模。当至关重要的企业只有在拥有一个包括经营区域的大规模组织时才能发挥最佳效益,扩展以提供这种环境给企业的政府可以获得足够的额外财富来支付维护大型政治体系的成本。在这种情况下,整个世界经济通常在一个最高的世界强权主导下比其他所有国家都更有效,就像19世纪英国帝国一样。但有时跨度变量相结合会导致规模经济递减。如果维护大型经营区域的经济利益减少,之前从利用包括经营区域在内的优势而繁荣的大型政府可能开始分裂 - 即使进攻和防御之间的武器平衡仍然保持差不多。但有时跨度变量相结合会导致规模经济递减。如果维护大型经营区域的经济利益减少,之前从利用包括经营区域在内的优势而繁荣的大型政府可能开始分裂--即使进攻和防御之间的武器平衡仍然保持差不多。
  \item E. 技术分散。还有另一个因素有助于权力平衡,那就是关键技术的分散程度。当武器或生产工具能够有效地垄断或垄断时,它们往往会集中权力。即使是本质上是防御性的技术,例如机枪,也被证明是有效的攻击武器,这加剧了在这些武器不被广泛传播的时期,政府规模的上升。当欧洲大国在19世纪晚期垄断机枪时,他们能够将这些武器用于对边缘民族的攻击,从而大大扩展殖民帝国。后来,在20世纪,特别是在二战后机枪变得广泛可用后,它们被用于帮助摧毁帝国的力量。其他条件相等时,关键技术分散得越广泛,权力就越分散,政府的最优规模就越小。
\end{itemize}
\end{paracol}

\section{跨度政治变化的速度}
\begin{paracol}{2}
While technology is by far the most important factor today, and apparently growing more so, all four major megapolitical factors have played a role in determining the scale at which power could be exercised in the past.  
\switchcolumn
虽然技术是当今最重要的因素,而且显然越来越重要,但过去四个主要的“超级政治”因素都在决定权力可以在何种规模下行使方面发挥了作用。
\switchcolumn*
Together, these factors determine whether the returns to violence continue to rise as violence is employed on a larger scale. This determines the importance of magnitude of firepower versus efficiency in employing resources. It also strongly influences the market distribution of income. The question is, What role will they command in the future? A key to estimating an answer lies in recognizing that these megapolitical variables mutate at dramatically different speeds. 

\switchcolumn
这不这些因素共同决定暴力的回报是否随着暴力规模的扩大而不断上升。这决定了火力强度与资源利用效率之间的重要性。它也强烈影响收入的市场分配。问题是,它们在未来将发挥什么样的作用?估算答案的关键在于认识到这些超级政治变量以戏剧性不同的速度突变。

\switchcolumn*
Topography has been almost fixed through the whole of recorded history. Except for minor local effects involving the silting of harbors, landfills, or erosion, the topography of the earth is almost the same today as it was when Adam and Eve straggled out of Eden. And it is likely to remain so until another lee Age recarves the landscapes of continents or some other drastic event disturbs the surface of the earth. At a more profound scale, geological ages seem to shift, perhaps in response to large meteorite strikes, over a period of 10 to 40 million years. Someday, there may again be geological upheavals that will alter significantly the topography of our planet. If that happens, you can safely assume that both the baseball and cricket seasons will be canceled.

\switchcolumn
地形几乎在整个记载历史时期都是固定的。除了涉及海港淤积,填海造陆或侵蚀等轻微局部影响外,地球的地形今天几乎与阿当和夏娃走出伊甸园时一样。并且在地球表面再次受到冰川时代的侵袭或其他剧烈事件扰乱之前,它很可能保持不变。在更深刻的层面上,地质时代似乎会在1000万到4000万年的时间内发生变化,可能是为了响应大型陨石撞击。有一天,可能会再次发生改变,从而显著地改变我们星球的地形。如果发生这种情况,您可以放心地假定棒球赛和板球赛季都将取消。

\switchcolumn*
Climate fluctuates much more actively than topography. In the last million years, climatic change has been responsible for most of the known variation in the features of the earth's surface. During Ice Ages, glaciers gouged new valleys, altered the course of rivers, severed islands from continents or joined them together by lowering the sea level. Fluctuations in climate have played a significant role in history, first in precipitating the Agricultural Revolution after the close of the last Ice Age, and later in destablizing regimes during periods of colder temperatures and drought. 

\switchcolumn
气候波动比地形波动活跃得多。在过去的一百万年中,气候变化负责地球表面大部分已知变化。在冰河时代期间,冰川挖出新的山谷,改变了河流的走向,通过降低海平面将岛屿从大陆上隔离或将它们连接在一起。气候波动在历史上发挥了重要作用,首先是在最后一次冰河时期结束后促成了农业革命,以及在寒冷和干旱时期破坏政权。

\switchcolumn*
Lately, there have been concerns over the possible impact of "global warming." These concerns cannot be dismissed out of hand. Yet, taking a longer perspective, the more likely risk appears to be a shift toward a colder, not a warmer climate. Study oftemperature fluctuations based upon analysis of oxygen isotopes in core samples taken from the ocean floor show that the current period is the second warmest in more than 2 million years. If temperatures were to turn colder, as they did in the seventeenth century, that might prove megapolitically destabilizing. Current alarms about global warming may in that sense be reassuring. To the extent that they are true, that assures that temperatures will continue to fluctuate within the abnormally warm and relatively benign range experienced for the past three centunes.

\switchcolumn
最近,人们对“全球变暖”的可能影响表示担忧。这些担忧不能轻率地被排除。然而,从更长远的视角来看,更有可能的风险是转向更冷的气候,而不是更暖和的气候。基于分析从海洋底部取出的核心样本中的氧同位素的温度波动研究表明,当前时期是超过200万年中第二个最热的时期。如果温度像17世纪那样变冷,那可能会导致超级政治动荡。有关全球变暖的当前警报在这个意义上可能是令人放心的。在这个意义上,只要它们是真实的,就可以保证温度将继续在过去三个世纪体验的异常温暖和相对温和的范围内波动。

\switchcolumn*
The rate of change in the influence of microbes on the exercise of power is more of a puzzle. Microbes can mutate very rapidly. This is especially true of viruses. The common cold, for example, mutates in an almost kaleidoscopic way. Yet although these mutations proceed apace, their impact in shifting the boundaries where power is exercised have been far less abrupt than technological change. Why? Part of the reason is that the normal balance of nature tends to make it beneficial for microbes to infect but not destroy host populations. Virulent infections that kill their hosts too readily tend to eradicate themselves in the process. The survival of microparasites depends upon their not being too rapidly or uniformly fatal to the hosts they invade.

\switchcolumn
微生物影响权力运作的变化率更像是一个谜题。微生物可以迅速发生变异。这在病毒中尤其明显。例如,普通感冒以一种几乎万花筒般的方式变异。尽管这些变异正在快速进行,但它们对于移动权力边界的影响远不及技术变革的突然。为什么?部分原因是自然环境中的平衡通常使感染微生物对宿主人群有益而不是摧毁。太具有致命性的病毒感染会在这个过程中自我灭绝。微生物寄生体的生存取决于它们对入侵的宿主不会过于迅速或一致地致命。

\switchcolumn*
That is not to say, of course, that there cannot be deadly eruptions of disease that alter the balance of power. Such episodes have figured prominently in history. The Black Death wiped out large fractions of the population of Eurasia and dealt a crushing blow to the fourteenth-century version of the international economy.

\switchcolumn
当然,这并不是说不能有致命的疾病暴发会改变权力的平衡。这种情况已经在历史上占据了重要地位。黑死病摧毁了欧亚大陆人口的大量部分,并给14世纪版本的国际经济以沉重打击。

\end{paracol}

\subsection{可能发生的情况}

\begin{paracol}{2}
History can be understood in terms of what might have been as well as what was. We know of no reason that microparasites equid not have continued to play havoc with human society during the modem-period. For example, it is possible that microbiological barriers to the exercise of power, equivalent to malaria but more virulent, could have halted the Western invasion of the periphery in its tracks. The first intrepid Portuguese adventurers who sailed into African waters could have contracted a deadly retrovirus, a more communicable version of AIDS, that would have stopped the opening of the new trade route to Asia before it even began. Columbus, too, and the first waves of settlers in the New World might have encountered diseases that decimated them in the same way that indigenous local populations were affected by measles and other Western childhood diseases. Yet nothing of the kind happened, a coincidence that underlines the intuition that history has a destiny.

\switchcolumn
历史可以通过可能发生的事情以及发生的事情来理解。我们不知道为什么微生物寄生体不能在现代时期继续对人类社会造成伤害。例如,等同于疟疾但更具致命性的微生物障碍可能会阻止西方侵略边缘。第一个翻越非洲水域的葡萄牙冒险家可能感染了一种致命的反转录病毒,一种更具传染性的艾滋病病毒,这会在开辟前往亚洲的新贸易路线之前就阻止了这一过程。哥伦布和新世界的第一波移民也可能遭遇到像本土的人口一样受麻疹和其他西方儿童疾病影响的疾病。然而,没有发生任何这样的事情,这种巧合强调了历史有一个命运的直觉。

\switchcolumn*
Microbes did far less to impede the consolidation of power in the modem period than to facilitate it. Western troops and colonists at the periphery often found that the technological advantages that allowed them to project power were underscored by microbiological ones. Westerners were armed with unseen biological weapons, their relative immunity to childhood diseases that frequently devastated native peoples. This gave voyagers from the West a distinct advantage that their antagonists from less densely settled regions lacked. As events unfolded, the disease transfer was almost entirely in one direction --from Europe outward. There was no equivalent transfer of disease in the other direction, from the periphery to the core. 

\switchcolumn
微生物并不像现代时期那样阻碍了权力的巩固,而是有利于它。在边缘地区的西方部队和殖民者经常发现,使他们有能力施展影响的技术优势得到了微生物方面的加强。西方人武装了无形的生物武器,在童年时期经常摧毁当地人的疾病上则相对免疫。这给西方的航海者带来了与来自人口稀少地区的对手所缺乏的明显优势。随着事态的发展,疾病传播几乎完全呈单向性——从欧洲向外传播。没有等价的疾病从边缘地区向核心地区传播。

\switchcolumn*
As a possible counterexample, some have claimed that Western explorers imported syphilis from the New World to Europe. This is arguable. If true, however, it did not prove to be a significant barrier to the exercise of power. The major impact of syphilis was to shift sexual mores in the West.

\switchcolumn
作为一个可能的反例,有些人声称西方探险家从新大陆引入了梅毒到欧洲。这是可以争辩的。然而,如果真是这样,它并没有证明它成为行使权力的一个重要障碍。梅毒的主要影响是改变了西方的性道德观念。

\switchcolumn*
From the end of the fifteenth century to the last quarter of the twentieth, the impact of microbes on industrial society was ever more benign. Notwithstanding the personal tragedies and unhappiness caused by outbreaks of tuberculosis, polio, and flu, no new diseases emerged in the modern period that even approached the megapolitical impact of the Antonine plagues or the Black Death. Improving public health, and the advent of vaccinations and antidotes, generally reduced the importance of infectious microbes during the modern period, thereby increasing the relative importance oftechnology in setting the boundaries where power was exercised. 

\switchcolumn
从十五世纪末到二十世纪最后一个季度,微生物对工业社会的影响越来越温和。尽管肺结核、小儿麻痹症和流感暴发所引起的个人悲剧和不幸,但现代时期没有出现新疾病,甚至不会接近安东宁瘟疫或黑死病的宏观政治影响。改善公共卫生和疫苗及解毒剂的出现,通常减少了现代时期传染性微生物的重要性,从而增加了技术在设定权力边界方面的相对重要性。

\switchcolumn*
The recent emergence of AIDS and alarms over the potential spread of exotic viruses are hints that the role of microbes may not be altogether as megapolitically benign in the future as it has been over the past five hundred years. But when or whether a new plague will infect the world is unknowable. An eruption of microparasites, such as a viral pandemic, rather than drastic changes in climate or topography, would more likely disrupt the megapolitical predominance of technology.

\switchcolumn
最近的艾滋病爆发和对外来病毒扩散的警报表明,微生物的作用在未来可能不完全是宏观政治上的同情。但新的瘟疫什么时候或是否会感染世界是不可预测的。微生物寄生物的爆发,例如病毒大流行,而不是气候或地形的剧烈变化,更有可能打破技术的宏观政治优势。

\switchcolumn*
We have no way of monitoring or anticipating drastic departures from the nature of life on earth as we have known it. We cross our fingers and assume that the major megapolitical variables in the next millennium will be technological rather than microbiological. If luck continues to side with humanity, technology will continue to grow in prominence as the leading megapolitical variable. It was not always such, however, as a review ofthe first great megapolitical transformation, the Agricultural Revolution, clearly shows.

\switchcolumn
我们没有办法监测或预测地球上生命的性质如何彻底改变。我们交叉着手指,假定未来一千年的主要宏观政治变量将是技术而不是微生物学。如果运气继续站在人类一边,技术将继续成为主要的宏观政治变量。然而,对第一次巨大的超大城市转型——农业革命的回顾清晰地表明,情况并非总是如此。

\end{paracol}