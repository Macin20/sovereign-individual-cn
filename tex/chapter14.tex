\documentclass[fontset=none, punct=kaiming]{ctexbook}
\usepackage{paracol}
\columnratio{0.35}           % 栏宽比例
\setlength{\columnsep}{2em}  % 栏间距

\usepackage{geometry}
\geometry{hmargin=2.54cm, vmargin=3.18cm}

\usepackage{indentfirst}
\usepackage{lipsum}
\usepackage{zhlipsum}

% 中西文均适用思源宋体
\setCJKmainfont{Source Han Serif SC}
\setmainfont{Source Han Serif SC}[CharacterWidth=AlternateProportional]

% 其他样式设置
\pagestyle{plain}
\setlength{\parskip}{.5ex plus 1pt}


\ctexset { %定制章节样式
chapter = {
beforeskip = 0pt,
fixskip = true,
format = \Huge\bfseries,
nameformat = \rule{\linewidth}{1bp}\par\bigskip\hfill\chapternamebox,
number = \arabic{chapter},
aftername = \par\medskip,
aftertitle = \par\bigskip\nointerlineskip\rule{\linewidth}{2bp}\par 
}
}
\newcommand\chapternamebox[1]{%
\parbox{\ccwd}{\linespread{1}\selectfont\centering #1}}



\title{狂人日記}
\author{魯迅}
\date{1918年4月}

% 来源:https://en.wikisource.org/wiki/Translation:Call_to_Arms_(Lu_Xun)/A_Madman%27s_Diary

\begin{document}

\maketitle

\chapter{这里是开头内容}
\begin{paracol}{2}[]
某君昆仲。今隱其名,皆余昔2000年日在中學校時良友;分193388年隔多年,消息漸闕。日前偶聞其一大病;適歸故鄉,迂道往
訪,則僅晤一人,言病者其弟也。勞君遠道來視,然已早愈,赴某地候補矣。因大笑,出示日記二冊,謂可見當
日病狀,不妨獻諸舊友。持歸閱一過,知所患「迫害狂」之類。語頗錯雜無倫次,又多荒唐之言;亦不著月日,
惟墨色字體不一,知非一時所書。間亦有略具聯絡者,今撮錄一篇,以供醫家研究。記中語誤,一字不易;惟人
名雖皆村人,不為世間所知,無關大體,然亦悉易去。至於書名,則本人愈後所題,不復改也。七年四月二日
識。
\switchcolumn
A certain pair of brothers, whose names I shall conceal for now, were both good friends of mine
back during our school days. I have been away for many years, and gradually lost touch with them. A
few days ago, I suddenly learned that one of them had come down with a serious illness. I was on my
way back to my hometown at the time, so I diverted my route so that I could visit them. However, I
only saw one of them, who explained that the sick person was his younger brother. He expressed
sympathy for my having come such a long way in order to see his brother, but then informed me that
his brother had long since recovered, and had moved away in order to fill a temporary post. He
thereupon proceeded to laugh aloud, and produced his brother's two-volume diary. He said that one
could ascertain the nature of his brother's illness by reading it, and that there was no harm in
showing it to an old friend. Upon taking it home and reading it through, I could tell that his
brother had been stricken with something like a ``persecution complex''. His words were slightly
disjointed and incoherent, and there were also a lot of preposterous statements. Furthermore, he
did not include any dates. It was only by the differences in ink color and penmanship that one
would know that it was not all written at the same time. Nevertheless, parts of the diary did have
some amount of consistency. I have excerpted those parts here, so that medical experts may use it
for research. I have not altered any of the errors in the text. However, I have altered all of the
personal names. Although they are all from the same village as him, none of them are known to the
outside world, and none of them are relevant to the overall subject matter. As for the title, he
chose it himself, after recovering from his illness. I have not changed it. April 2, 7th year of
the Republic (1918)
\end{paracol}

\chapter{这里是章}

\begin{paracol}{2}[]

今天晚上,很好的月光。

\switchcolumn

Tonight, the moonlight is very good.

\switchcolumn

我不見他,已是三十多年;今天見了,精神分外爽快。纔知道以前的三十多年,全是發昏;然而須十分小心。不
然,那趙家的狗,何以看我兩眼呢?

\switchcolumn

I have not seen it in more than thirty years; after seeing it today, my spirits are especially
refreshed. Only now do I realize that for the past thirty years, I have been in a complete daze;
yet, I must be extremely cautious. Otherwise, why would the Zhao family's dog have looked at me
twice?

\switchcolumn*

我怕得有理。

\switchcolumn

I have reason to fear.

\end{paracol}


\chapter{又有一个内容}
\begin{paracol}{2}[]

今天全沒月光,我知道不妙。早上小心出門,趙貴翁的眼色便怪:似乎怕我,似乎想害我。還有七八個人,交頭
接耳的議論我,又怕我看見。一路上的人,都是如此。其中最兇的一個人,張着嘴,對我笑了一笑;我便從頭直
冷到腳跟,曉得他們布置,都已妥當了。

\switchcolumn

Today, there was no moonlight whatsoever. I know that can't be good. I was careful when I went out
this morning. Zhao Guiweng gave me a strange look, as if he was both afraid of me and also wanted
to do me in. There were also another seven or eight people discussing me in hushed tones, hoping
that I wouldn't see them. Everybody that I encountered along my route was like this. Among them,
the most intimidating was a person who gave me an open-mouthed smile; it sent a chill from my head
straight to my toes. I knew at that moment that their plans for me were already set.

\switchcolumn*

我可不怕,仍舊走我的路。前面一夥小孩子,也在那裏議論我;眼色也同趙貴翁一樣,臉色也都鐵青。我想我同
小孩子有什麼讎,他也這樣。忍不住大聲說,「你告訴我!」他們可就跑了。

\switchcolumn

I couldn't show fear, so I just continued along my original route. There was a group of children
ahead of me that were also discussing me. They gave me the same look that Zhao Guiweng had given
me. Their eyes were filled with rage. I tried to think what possible enmity I might have with these
children that they might behave in such a way. I couldn't help but yelling out, ``Tell me!'' They
ran off.

\switchcolumn*

我想:我同趙貴翁有什麼讎,同路上的人又有什麼讎;只有廿年以前,把古久先生的陳年流水簿子,踹了一腳,
古久先生很不高興。趙貴翁雖然不認識他,一定也聽到風聲,代抱不平;約定路上的人,同我作寃對。但是小孩
子呢?那時候,他們還沒有出世,何以今天也睜着怪眼睛,似乎怕我,似乎想害我。這眞教我怕,教我納罕而且
傷心。

\switchcolumn

I tried to think what possible enmity I might have with Zhao Guiweng or with the others along my
route. The only thing that came to mind was when I had stepped on Mr.\ Gu Jiu's account ledger
twenty years ago. It was quite old. Mr.\ Gu Jiu was not happy when I did that. Even though Zhao
Guiweng doesn't know him, he certainly must have heard something. Perhaps that's why he's mad at
me. Maybe he turned the others along my route against me. But what about the children? They weren't
yet born when that happened. Why would they give me such strange looks now, as if they both feared
me and wanted to do me in. This caused me to be truly afraid. It also took me aback and hurt my
feelings.

\switchcolumn*

我明白了。這是他們娘老子教的!

\switchcolumn

I understand. Their parents are making them act like this!

\end{paracol}

\end{document}