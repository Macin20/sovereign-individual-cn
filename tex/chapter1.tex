\chapter[人类社会第四个阶段]{2000年转折: \\ 人类社会的第四阶段}

\section[预言]{PREMONITIONS\\ 预言}

\begin{paracol}{2}[]
The coming of the year 2000 has haunted the Western imagination for the past thousand years. Ever since the world failed to end at the turn of the first millennium after Christ, theologians, evangelists, poets, seers, and now, even computer programmers have looked to the end of this decade with an expectation that it would bring something momentous. No less an authority than Isaac Newton speculated that the world would end with the year 2000. Michel de Nostradamus, whose prophecies have been read by every generation since they were first published in 1568, forecast the coming of the Third Antichrist in July 1999. Swiss psychologist Carl Jung, connoisseur of the “collective unconscious,” envisioned the birth of a New Age in 1997. Such forecasts may easily be ridiculed. And so can the sober forecasts of economists, such as Dr. Edward Yardeni of Deutsche Bank Securities, who expects computer malfunctions on the millennial midnight to “disrupt the entire global economy.” But whether you view the Y2K computer problem as groundless hysteria ginned up by computer programmers and Information Technology consultants to stir up business, or as a mysterious instance of technology unfolding in concert with the prophetic imagination, there is no denying that circumstances at the eve of the millennium excite more than the usual morbid doubt about where the world is tending.

\switchcolumn
在过去的一千年里,2000年的到来一直困扰着西方人的想象力。自从世界在基督之后的第一个千年之交未能结束以来,神学家、传道者、诗人、先知,甚至现在,甚至计算机程序员也期待着这个十年的结束,期望它会带来一些重大的东西。不亚于艾萨克·牛顿(Isaac Newton)推测世界将在2000年结束的权威。米歇尔·德·诺查丹玛斯(Michel de Nostradamus)的预言自1568年首次出版以来,每一代人都在阅读,他预言了1999年7月第三敌基督者的到来。瑞士心理学家卡尔·荣格(Carl Jung)是“集体无意识”的鉴赏家,他在1997年设想了新时代的诞生。这样的预测可能很容易被嘲笑。经济学家的清醒预测也是如此,例如德意志银行证券(Deutsche Bank Securities)的爱德华·亚德尼(Edward Yardeni)博士,他预计千禧年午夜的计算机故障将“扰乱整个全球经济”。但是,无论你是否将Y2K计算机问题视为计算机程序员和信息技术顾问为搅动业务而制造的毫无根据的歇斯底里,或者作为技术与预言想象力一起展开的一个神秘例子,不可否认的是,千禧年前夕的情况比通常对世界走向何方的病态怀疑更令人兴奋。
\end{paracol}
\begin{paracol}{2}[]
A sense of disquiet about the future has begun to color the optimism so characteristic of Western societies for the past 250 years. People everywhere are hesitant and worried. You see it in their faces. Hear it in their conversation. See it reflected in polls and registered in the ballot box. Just as an invisible, physical change of ions in the atmosphere signals that a thunderstorm is imminent even before the clouds darken and lightning strikes, so now, in the twilight of the millennium, premonitions of change are in the air. One person after another, each in his own way, senses that time is running out on a dying way of life. As the decade expires, a murderous century expires with it, and also a glorious millennium of human accomplishment. All draw to a close with the year 2000.
\begin{tcolorbox}
For there is nothing covered that shall not be revealed, neither hid that shall not be known. \\ 
\begin{flushright}
——MATTHEW 10.26
\end{flushright}
\end{tcolorbox}
\switchcolumn
对未来的不安感已经开始为过去250年来西方社会特有的乐观情绪着色。世界各地的人们都在犹豫和担心。你可以在他们的脸上看到它。在他们的谈话中听到它。看到它反映在民意调查中并在投票箱中登记。正如大气中无形的物理变化甚至在云层变暗和闪电袭击之前就预示着雷暴即将来临,所以现在,在千年的黄昏,空气中弥漫着变化的预感。一个又一个人,每个人都以自己的方式,感觉到时间在垂死的生活方式上已经不多了。随着十年的过去,一个杀戮的世纪也随之结束,也是人类成就的光荣千年。所有这些都在2000年结束。
\begin{tcolorbox}
因为没有什么是不能透露的,也没有隐藏的,是不应该知道的。
\begin{flushright}
—— 马太福音10.26
\end{flushright}
\end{tcolorbox}
\end{paracol}


\begin{paracol}{2}[]
We believe that the modern phase of Western civilization will end with it. This book tells why. Like many earlier works, it is an attempt to see into a glass darkly, to sketch out the vague shapes and dimensions of a future that is still to be. In that sense, we mean our work to be apocalyptic—in the original meaning of the word. Apokalypsis means “unveiling” in Greek. We believe that a new stage in history—the Information Age—is about to be “unveiled.”
\switchcolumn
我们相信,西方文明的现代阶段将随之结束。这本书讲述了原因。像许多早期的作品一样,它试图黑暗地看到玻璃,勾勒出未来仍然存在的模糊形状和维度。从这个意义上说,我们的意思是我们的工作是世界末日的——在这个词的原始含义中。Apokalypsis在希腊语中的意思是“揭幕”。我们相信,历史的新阶段——信息时代——即将“揭开面纱”。
\end{paracol}

\begin{paracol}{2}[]
\begin{tcolorbox}
We are watching the beginnings of a new logical space, an instantaneous electronic everywhereness, which we may all access, enter into, and experience. We have, in short, the beginnings of a new kind of community. The virtual community becomes the model for a secular Kingdom of Heaven; as Jesus said there were many mansions in his Father’s Kingdom, so there are many virtual communities, each reflecting their own needs and desires.
\begin{flushright}
——MICHAEL GRASSO
\end{flushright}
\end{tcolorbox}
\switchcolumn
\begin{tcolorbox}
我们正在观察一个新的逻辑空间的开始,一个瞬间的电子无处不在,我们都可以访问,进入和体验。简而言之,我们已经有了一种新的社区的开端。虚拟社区成为世俗
天国的典范;正如耶稣所说,在他父的王国里有许多豪宅,所以有许多虚拟社区,每个社区都反映了他们自己的需要和愿望。
\begin{flushright}
—— 迈克尔·格拉索
\end{flushright}
\end{tcolorbox}
\end{paracol}


\section[人类社会的第四阶段]{THE FOURTH STAGE OF HUMAN SOCIETY \\ 人类社会的第四阶段}
\begin{paracol}{2}[]
The theme of this book is the new revolution of power which is liberating individuals at the expense of the twentieth-century nation-state. Innovations that alter the logic of violence in unprecedented ways are transforming the boundaries within which the future must lie. If our deductions are correct, you stand at the threshold of the most sweeping revolution in history. Faster than all but a few now imagine, microprocessing will subvert and destroy the nation-state, creating new forms of social organization in the process. This will be far from an easy transformation.
\switchcolumn
本书的主题是权力的新革命,它以牺牲二十世纪的民族国家为代价解放个人。以前所未有的方式改变暴力逻辑的创新正在改变未来必须存在的界限。如果我们的推论是正确的,那么你就站在了历史上最彻底的革命的门槛上。微处理的速度比现在除了少数人想象的要快,它将颠覆和摧毁民族国家,在此过程中创造新的社会组织形式。这远非易事的转变。
\end{paracol}

\begin{paracol}{2}[]
The challenge it will pose will be all the greater because it will happen with incredible speed compared with anything seen in the past. Through all of human history from its earliest beginnings until now, there have been only three basic stages of economic life: (1) hunting-and-gathering societies; (2) agricultural societies; and (3) industrial societies. Now, looming over the horizon, is something entirely new, the fourth stage of social organization: information societies.
\switchcolumn
它将带来的挑战将更大,因为与过去看到的任何东西相比,它将以令人难以置信的速度发生。纵观人类历史,从最早的开始到现在,经济生活只有三个基本阶
段:(1)狩猎和采集社会;(2)农业社会;(3)工业社会。现在,迫在眉睫的是某种全新的社会组织阶段:信息社会。
\end{paracol}

\begin{paracol}{2}[]
Each of the previous stages of society has corresponded with distinctly different phases in the evolution and control of violence. As we explain in detail, information societies promise to dramatically reduce the returns to violence, in part because they transcend locality. The virtual reality of cyberspace, what novelist William Gibson characterized as a “consensual hallucination,” will be as far beyond the reach of bullies as imagination can take it. In the new
millennium, the advantage of controlling violence on a large scale will be far lower than it has been at any time since before the French Revolution. This will have profound consequences. One of these will be rising crime. When the payoff for organizing violence at a large scale tumbles, the payoff from violence at a smaller scale is likely to jump. Violence will become more random and localized. Organized crime will grow in scope. We explain why.
\switchcolumn
社会以前的每个阶段都与暴力演变和控制的明显不同阶段相对应。正如我们详细 解释的那样,信息社会有望大幅减少暴力的回归,部分原因是它们超越了地方性。小说家威廉·吉布森(William Gibson)称之为“自愿幻觉”的网络空间的虚拟现实将远远超出欺凌者的想象范围。在新的千年里,大规模控制暴力的优势将远远低于法国大革命前的任何时候。这将产生深远的影响。其中之一将是犯罪率上升。当大规模组织暴力的回报下降时,小规模暴力的回报可能会增加。暴力将变得更加随机和局部化。有组织犯罪的范围将会扩大。我们解释原因。
\end{paracol}

\begin{paracol}{2}[]
Another logical implication of falling returns to violence is the eclipse of politics, which is the stage for crime on the largest scale. There is much evidence that adherence to the civic myths of the twentieth-century nation-state is rapidly eroding. The death of Communism is merely the most striking example. As we explore in detail, the collapse of morality and growing corruption among leaders of Western governments are not random developments. They are evidence that the potential of the nation-state is exhausted. Even many of its leaders no longer believe the platitudes they mouth. Nor are they believed by others.
\switchcolumn
暴力回归下降的另一个逻辑含义是政治的黯然失色,这是最大规模犯罪的舞台。有很多证据表明,对二十世纪民族国家的公民神话的坚持正在迅速侵蚀。共产主义的死亡只是最引人注目的例子。正如我们详细探讨的那样,西方政府领导人道德的崩溃和日益严重的腐败并不是随机的发展。它们证明民族国家的潜力已经耗尽。甚至许多领导人也不再相信他们所说的陈词滥调。他们也不被其他人相信。
\end{paracol}

\subsection[历史重演]{History Repeats Itself \\历史重演}
\begin{paracol}{2}[]
This is a situation with striking parallels in the past. Whenever technological change has divorced the old forms from the new moving forces of the economy, moral standards shift, and people begin to treat those in command of the old institutions with growing disdain. This widespread revulsion often comes into evidence well before people develop a new coherent ideology of change. So it was in the late fifteenth century, when the medieval Church was the predominant institution of feudalism. Notwithstanding popular belief in “the sacredness of the sacerdotal office,” both the higher and lower ranks of clergy were held in the utmost contempt—not unlike the popular attitude toward politicians and bureaucrats today.
\switchcolumn
这种情况在过去有着惊人的相似之处。每当技术变革将旧形式与新的经济动力分离时,道德标准就会发生变化,人们开始越来越蔑视那些控制旧机构的人。这种普遍的厌恶往往在人们形成一种新的连贯的变革意识形态之前就已经显现出来了。因此,在十五世纪后期,中世纪教会是封建主义的主要机构。尽管人们普遍相信“神圣职位的神圣性”,但神职人员的上级和下级都受到了极大的蔑视——这与今天对政治家和官僚的普遍态度没有什么不同。
\end{paracol}

\begin{paracol}{2}[]
We believe that much can be learned by analogy between the situation at the end of the fifteenth century, when life had become thoroughly saturated by organized religion, and the situation today, when the world has become saturated with politics. The costs of supporting institutionalized religion at the end of the fifteenth century had reached a historic extreme, much as the costs of supporting government have reached a senile extreme today.
\switchcolumn
我们认为,通过类比15世纪末的生活被有组织的宗教彻底浸透的情况与今天世界 政治饱和的情况,可以学到很多东西。十五世纪末支持制度化宗教的成本已经达到了历史的极端,就像支持政府的成本今天达到了衰老的极端一样。
\end{paracol}

\begin{paracol}{2}[]
We know what happened to organized religion in the wake of the Gunpowder Revolution. Technological developments created strong incentives to downsize religious institutions and lower their costs. A similar technological revolution is destined to downsize radically the nation-state early in the new millennium.
\switchcolumn
我们知道在火药革命之后,有组织的宗教发生了什么。技术发展为缩小宗教机构规模和降低成本创造了强烈的动力。类似的技术革命注定要在新千年初期从根本上缩小民族国家的规模。
\end{paracol}

\begin{paracol}{2}[]
\begin{tcolorbox}
Today, after more than a century of electric technology, we have extended our central nervous system itself in a global embrace, abolishing both space and time as far as our planet is concerned
\begin{flushright}
——MARSHALL McLUHAN, 1964
\end{flushright}
\end{tcolorbox}
\switchcolumn
\begin{tcolorbox}
今天,经过一个多世纪的电气技术,我们已经在全球范围内扩展了我们的中枢神经系统本身,就我们的星球而言,废除了空间和时间
\begin{flushright}
——马歇尔·麦克卢汉,1964
\end{flushright}
\end{tcolorbox}
\end{paracol}


\subsection[信息革命]{The Information Revolution \\信息革命}

\begin{paracol}{2}[]
As the breakdown of large systems accelerates, systematic compulsion will recede as a factor shaping economic life and the distribution of income. Efficiency will become more important than the dictates of power in the organization of social institutions. This means that provinces and even cities that can effectively uphold property rights and provide for the administration of justice, while consuming few resources, will be viable sovereignties in the Information Age, as they generally have not been during the last five centuries. An entirely new realm of economic activity that is not hostage to physical violence will emerge in cyberspace. The most obvious benefits will flow to the “cognitive elite,” who will increasingly operate outside political boundaries. They are already equally at home in Frankfurt, London, New York, Buenos Aires, Los Angeles, Tokyo, and Hong Kong. Incomes will become more unequal
within jurisdictions and more equal between them.
\switchcolumn
随着大型系统的加速崩溃,系统性强迫作为影响经济生活和收入分配的一个因素将会消退。在社会机构的组织中,效率将变得比权力的要求更重要。这意味着,能够有效维护财产权和提供司法行政的省份甚至城市,同时消耗的资源很少,在信息时代将是可行的主权,就像过去五个世纪以来通常没有的那样。一个不受身体暴力影响的经济活动全新领域将在网络空间出现。最明显的好处将流向“认知精英”,他们将越来越多地在政治边界之外运作。他们已经在法兰克福、伦敦、纽约、布宜诺斯艾利斯、洛杉矶、东京和香港同样自在。辖区内的收入将变得更加不平等,它们之间的收入将变得更加平等。
\end{paracol}

\begin{paracol}{2}[]
The Sovereign Individual explores the social and financial consequences of this revolutionary change. Our desire is to help you to take advantage of the opportunities of the new age and avoid being destroyed by its impact. If only half of what we expect to see happens, you face change of a magnitude with few precedents in history.
\switchcolumn
《主权个人》探讨了这一革命性变化的社会和财务后果。我们的愿望是帮助您利用新时代的机遇,避免被其影响所摧毁。如果我们期望看到的只有一半发生,你将面临历史上很少有先例的巨大变化。
\end{paracol}

\begin{paracol}{2}[]
The transformation of the year 2000 will not only revolutionize the character of the world economy, it will do so more rapidly than any previous phase change. Unlike the Agricultural Revolution, the Information Revolution will not take millennia to do its work. Unlike the Industrial Revolution, its impact will not be spread over centuries. The Information Revolution will happen within a lifetime.
\switchcolumn
2000年的转变不仅将彻底改变世界经济的性质,而且将比以往任何阶段性变化都快。与农业革命不同,信息革命不需要几千年才能完成其工作。与工业革命不同,它的影响不会延续几个世纪。信息革命将在有生之年发生。
\end{paracol}


\begin{paracol}{2}[]
What is more, it will happen almost everywhere at once. Technical and economic innovations will no longer be confined to small portions of the globe. The transformation will be all but universal. And it will involve a break with the past so profound that it will almost bring to life the magical domain of the gods as imagined by the early agricultural peoples like the ancient Greeks. To a greater degree than most would now be willing to concede, it will prove difficult or impossible to preserve many contemporary institutions in the new millennium. When information societies take shape they will be as different from industrial societies as the Greece of Aeschylus was from the world of the cave dwellers.
\switchcolumn
更重要的是,它几乎会同时发生在任何地方。技术和经济创新将不再局限于全球的一小部分。这种转变几乎是普遍的。它将涉及与如此深刻的过去决裂,以至于它几乎将像古希腊人这样的早期农业民族所想象的那样,使神的神奇领域栩栩如生。在比大多数人现在愿意承认的更大程度上,在新千年中保留许多当代机构将证明是困难或不可能的。当信息社会形成时,它们将与工业社会不同,就像埃斯库罗斯的希腊与穴居者的世界一样。
\end{paracol}

\section[主权个体的崛起]{PROMETHEUS UNBOUND: THE RISE OF THE SOVEREIGN INDIVIDUAL\\普罗米修斯不受束缚:主权个体的崛起}


\section[卢德分子回归]{RETURN OF THE LUDDITES\\卢德分子的回归}


\section[带宽胜过边界]{BANDWIDTH TRUMPS BORDERS\\带宽胜过边界}


\section[复兴游行的法律]{REVIVING LAWS OF THE MARCH\\复兴游行的法律}


\section[愿望的虚荣]{THE VANITY OF WISHES\\愿望的虚荣}


\section[大型机和Y2K定时炸弹]{MAINFRAMES AND THE Y2K TIME BOMB\\大型机和Y2K定时炸弹}


\section[Y2K与核武库]{Y2K AND THE NUCLEAR ARSENAL\\Y2K与核武库}



\begin{paracol}{2}[]
\switchcolumn
\end{paracol}

