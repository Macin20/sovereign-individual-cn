\chapter{译者序}

看完上面彼得·泰尔的序言,可能很多中文读者会感觉不舒服,甚至受到了冒犯,但也许,感觉越不舒服的人,越需要读这本书。因为,“几乎所有你自认了解的东西都是错的”。


《主权个人》写作于 1990 年代中期,首次出版于 1996 年,距今整整四分之一个世纪。2020 年该书进行了再版,本中文版本翻译自 2020 年最新版本。这本书之所以在这几年还魂回潮,不只是它预言了加密货币,而加密货币已经成为了今天的显学;而是因为它的论点和逻辑经受住了时间考验,书中的预言得到了普遍的验证,潮水正在往它指示的方向涌动。就像英文读者的评论,“不读这本书,你根本不理解过去 20 年发生了什么,你也不知道接下来 20 年会发生什么。”

这本书在西方被认为是“加密货币界的圣经”,它深刻影响了或影响着科技精英和加密一代。可以看到,很多人在以“主权个人”中的逻辑创造和生活,在以“主权个人”中的观念思考和表达。但它的意义并非仅止于此,事实上,加密货币的内容在全书中只占不到十分之一。它更大的价值和意义,在我最早发布的书评中做了更多的解释,这里不再赘述。

作者在书中很多地方使用了“管辖区”(Jurisdiction)一词,是因为在作者看来,现代形式的民族国家,将会在这场信息革命中解体或消亡,未来会出现大量的新型主权,所以用“管辖区”的概念与国家形成区隔,可以简单理解为国家。

这是我第一次翻译这种篇幅和深度的书,而且时间非常紧迫,翻完甚至没来得及仔细从头再过一遍。译文中肯定有大量纰漏和错误,请大家理解,同时欢迎批评指正,后续有时间再做修订。

\begin{flushright}
\kaishu 陈诗\footnote{译者个人网站:https://macin.org,欢迎来访。} \\
2025 年 9 月,重庆
\end{flushright} 