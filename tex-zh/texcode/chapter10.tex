\chapter{民主制度的黄昏}

\begin{tcolorbox}
民主理论是当今民族国家进行道德说教的世界语,是把所有国家真正联合起来的语言,是现代世界里公开的伪善辞令,但它其实是一种价值可疑的货币——只有十足的蠢货才会完全相信它的表面价值,这么说一点也没有夸张。
\begin{flushright}
—— 约翰·邓恩(John Dunn)
\end{flushright}
\end{tcolorbox}


众所周知,在政府的历史上,民主是一段相对罕见且短暂的时期。古今中外,凡是民主制占上风的时代,它的成功都仰赖于那些加强了军事力量和群众重要性的大政治条件。在《武器系统与政治稳定性》(Weapons Systems and Political Stability)一书中,历史学家卡罗尔·奎格利(Carroll Quigley)探讨了这些条件,它们包括:

\begin{enumerate}
    \item 廉价且广泛分散是武器装备。当购买有用武器的成本较低时,民主制度往往会蓬勃发展。
    \item 武器可以被业余人士有效地使用。当任何人不需要长时间的训练,就可以正确地使用武器时,民主制生存的可能会更大。
    \item 步兵在作战中占据着数量上的军事优势。奎格利指出,“步兵占优势的时期,政治权力在社会中更加分散,民主制度更有机会胜出。” 
\end{enumerate}

这几点,很难说是对民主生存条件的全面指南。如果按照这些条件,民主就不会在 20 世纪末成为一种优胜的制度。在工业时代的末期,武器可以说比以往任何时候都昂贵。而且,很多最高效的武器,绝对需要专业人士才能有效地使用。此外,美国及其盟友与伊拉克之间的海湾战争,已经证明了大型的步兵队伍是多么的脆弱,即使是窝在战壕和挖好的防御工事里。那么,在这些条件下,为什么在20 世纪行将结束的时候,民主制看上去依然很繁荣呢?

\section{民主制,共产主义的孪生兄弟?}
在第五章中,我们提出了一个充满悖论的解释,即民主制度是共产主义的孪生兄弟,它的繁荣昌盛,是因为它有利于国家横行无阻地控制社会资源。从工业时代的“常识”来看,这个结论可能很愚蠢。我们并不否认,在工业社会的条件下,民主制与共产主义是截然对立的。但从大政治的角度,特别是从信息时代的有利位置来看,这两种制度的共同点,比你所能猜测的还要多。

在武器装备机器昂贵的情况下,民主成为了国家最大限度控制资源的决策机制。

与国家社会主义一样,民主制度提供了巨额的资金,可以资助庞大的军事机构。

但与国家社会主义不同的是,民主福利国家掌握了更多的资源。这就可以说明一些问题,因为国家社会主义或共产主义制度,几乎把控了一切有价值的资产。

冷静地看,民主之所以比国家社会主义制度更优越,因为它是国家富裕的秘诀。

我们在前面解释过,民主可以为军队提供更多的资金,因为民主与私有制和资本主义的生产力相适应。

国家社会主义的理论前提是国家拥有一切。相比之下,民主福利国家在一开始不会提太多的要求。它们假装允许私有制——尽管是一种不确定的私有制,从而利用优越的激励机制来提高产出。西方的民主政府不是一开始就横加干涉,而是允许个人拥有财产,积累财富。只有在财富被创造出来之后,民主民族国家才会出面征税,拿走其中的一大部分。

这个“大”字要大写。例如,在 1996 年,美国最高等级的联邦终身税率为 1 美元 73 美分。对于通过分红获得收入的公司股东来说,税率为 1 美元 83 美分。而对于任何想把财产留给孙子女的人来说,联邦税率为 1 美元 93 美分。如果把各州和地方税也算进去,各级民主政府没收了在美国赚到的每一美元中的绝大部分。占到所有收入 3/4 甚至 9/10 的掠夺性税率,使得民主国家和社会主义国家成为了事实上的同伙。当然,这两种制度不是一回事,但关系非常密切。

民主国家之所以存活得更久,因为它更加灵活,而且相比莫斯科或东柏林,它能征收到的资源量更加惊人。

\subsection{“效率低下,反而更成功”}
民主作为强大政府的决策机制,它的大政治优势曾被我们称之为“效率低下的优势”。与共产主义相比,福利国家的系统效率确实要高得多。但与殖民地时期的香港——一块真正自由放任的飞地——相比,福利国家的效率就差多了。香港的增长速度非常惊人,而它的优越性恰恰在于:快速增长带来的好处,其中 85\%进入了香港居民的腰包,而不是香港政府的。

当然,香港从来也不是一个民主国家。其实,它是我们希望看到的那种,能在信息时代蓬勃发展的管辖区模型。在工业时代,香港没必要成为一个民主国家,它不必为了养一支强大的军队而去征收资源,惹得人民怨声载道。香港可以从外部获得保护,所以它能够维持真正的自由经济。

正因为民主具有搜刮资源的能力,它才在工业时代的大政治条件下占据了主导地位。大众民主与工业主义是相伴而生,携手并进的。正如阿尔文·托夫勒所言,大众民主“是大众生产、大众分配、大众消费、大众教育、大众媒体、大众娱乐及其他一切,在政治上的表现。”如今,信息技术正在取代大规模生产;预期大众民主即将进入黄昏,是合乎逻辑的。大众民主在工业时代获胜的关键性大政治需求,已经消失了。因此,大众民主步其孪生兄弟——共产主义的后尘,只是时间问题。

\subsection{大众民主与信息时代不兼容}
稍加思考就会发现,信息时代的技术,在本质上不是一种大众技术。在军事层面,就像我们所说的,它释放了“智能武器”和“信息战争”的潜能,其中的“逻辑炸弹”可以破坏中央指挥控制系统。信息技术明确指出,武器的完善方向是只能由专业人士操作;不仅如此,它还降低了战争的决定作用,相对提高了防守的地位。微技术使个人也可以大幅提高自己的军事力量,大规模步兵编队的重要性则被削弱了。在兰德公司给国防部长的报告中写道:“互联网不仅可能遭到国家的破坏,也可能受到非国家行为者的攻击,甚至包括分散的团体或个人。”更重要的是,它意味着,庞大的集权式系统所固有的规模不经济性,将在网络战中彻底被暴露。

用兰德公司专家的话说,“信息技术使地理距离变得无关紧要。美国大陆上遥远的目标,和战区内的目标一样容易受到攻击。”在过去,居住在美国这样的超级大国的境内,就等于安全得到了保障;而在信息时代,集结力量的逻辑要被翻转了。皮奥利亚(美国伊利诺州第二大城市,译注)虽然远离战争前线,但它也不再是安全的,可能受到所有潜在敌人的网络攻击。居住在超级大国境内,就相当于把自己放在了靶心。所以,各地方与其联合在一起,不如分解开来,反而会更加安全。总之,网络战的出现,会使中央控制指挥系统更加脆弱,但同时加强了分散式系统的竞争活力。

这种变化所激发的反馈机制,将加速权力分解的进程。兰德公司的专家认为,为了降低在民族国家后期发展起来的指挥控制系统,面对网络攻击的脆弱性,政府将不得不增加“对新的软件加密技术的利用”。这会使一些主要的私营部门的系统,更不容易遭到破坏;同时加速硬加密技术向商业组织的传播,有助于它们摆脱国家的控制。这也将促进权力的下放和分解。就像我们前面所说的,它将进一步推动资源向网络空间转移,而那里是政治的鞭长莫及之地。

最终,这意味着大众民主的结束,尤其是它的主要表现形式,即代议制恶政(misgovernment),不管是国会型的还是议会型的。

\section{代议失范的大政治学}
当大政治条件发生重大变化时,比如现在,政府的组织形式也必然随之改变。事实上,代议形式的政府,历来是原始权力分配的产物。这一点从一件事就可以看出来,那就是:代议制的代表是以地域为基础选出来的,而不是通过其他方式。

仔细想一想。原则上,立法机构的成员可以根据任意的人口划分来选举出来,并且具有同等的代表性。议会或国会的选区,可以用生日,或者按字母顺序来划分选区。1 月 1 日出生的人可以根据一份候选人名单投票,1 月 2 日出生的根据另外一份。或者,名字以“Aa”至“Af”开头的人针对一份名单投票,以“Ag”开头的针对另一份。以此类推。

但是现在没有这种制度,有这么几个原因。第一个也是最充分的原因是,在 18世纪,这样的制度安排在技术上是不可行的。但更重要的是,以生日或字母顺序划分选区,无法反映选票在当时应该体现的原始权力的分配,甚至一点关联都没有。有共同的生日或者名字的前几个字母,这样的人群,在过去和现在都很难组织成任何牢固的权力基础。

\subsection{为什么地理区域更重要?}
投票确实是作为军事竞赛的替代物而出现的。它今天也是,只是以一种更隐晦的方式。这种竞赛可以按照地理界线进行组织,更罕见的有按照亲属关系或宗教界线;但它没办法按照生日或名字的首字母来组织。它也无法按职业类别来组织,除非职业被限定在世代相传的行会里,如印度那种种姓制度;或者像爱荷华州的农民那样联系在一起。

当前代议制的全部重点在于,它代表和调动的是地理上的利益,而非其他方面的。

历史上,军事成功关键就是控制领土。而所有的军事威胁也都是在当地形成的。

代议制的目的,就是为这种权力的表达提供一种不同的竞技场。代议制的方式,会不可避免地倾向于促进地方的既得利益。地域选区的安排,会使代表们以国家所有居民的共同利益为代价,去为特殊群体谋福利。

\subsection{未来的新可能}
根据公共选择经济学家的分析,选举结构或机票方式的微小变化,就会给选举结果造成巨大且可预测的改变。正因为如此,今天认真研究政治的学生也必须认真地研究宪法。也是出于这种考虑,我们决定超越宪法的范畴,把目光投向,由具体环境中普遍存在的大政治条件所决定的终极元宪法。

技术的变革,已经打破了把投票局限在地域选区内的基础。在十八和十九世纪,当现代的代议制出现时,几乎所有的通讯手段都是地方性的。大多数人从生到死,都在他们出生地的几英里范围内活动,他们全部的商业和通信也都发生在当地。

而今天,在全世界都可以即时通讯。和五千英里以外的人做生意,和跟邻居做生意差不多。经济正在越来越彻底地超越地理的限制。社会的流动性不断在增强。

信息时代的财富也是如此。计算机程序不是钢铁厂,它不会很容易就被地方的政治势力所挟持。当立法者决定征税或监管业主时,钢铁厂是没办法移动的。而计算机程序通过调制解调器,能以光速传输到世界任何地方;业主只需收拾下他的手提电脑,就可以坐飞机离开。这种变革,也会破坏地域选区的大政治基础。

根据我们的分析,所有的代议制民主,都面临一大共同难题,那就是地域选区的安排,必然会过度代表工业企业的既得利益。“失败者”或“被遗弃者”是完美的选民,他们在地理上很集中,在政治上有需求。工业民主的历史已经证明了这一点。即使在 20 世纪 30 年代工业时期的高潮阶段,来自新兴产业的“赢家”,在立法审议中的代表也长期不足。政客们更愿意代表现有的、成熟的竞争者,而不是未来的新企业或新企业的潜在客户。这可能是代议制政府的一个天然特性。

就像曼库尔·奥尔森(Mancur Olson)在《国家的兴衰》(The Rise and Decline ofNations)一书中的论述,寿命长的产业会发展出有效的“分配联盟”,为政治利益进行游说和争夺。

但是涉及到信息时代的经济时,这个问题就会无限放大。因为在信息经济中,更有创造力的参与者在地理上是很分散的。他们不可能像苏格兰的鲑鱼渔民或萨斯喀彻温省的麦农一样,大规模地集中在某地,为立法者所关注。实际上,信息经济中许多有影响力的人物,甚至不是最大的管辖区的公民。因此,他们在代议制民主国家的立法审议中,几乎没有发言权。有一个很有说服力的例子:美国的数学博士们,为了阻止外国科学家在美国工作,曾做出很多不光彩的干涉活动;他们向国会提出排外主义的提案,阻止雇主择优录取,很容易就得到了听信。工业时代遗留下来的地域代表权,已经过时了;它毫不理会那些为经济做出重大贡献,但不是选民的外国数学家或其他任何人。

\begin{tcolorbox}
为什么人们会相信民主体制的合法性?回到这个问题,和解释人们为什么会相信特定的宗教教条一样难。因为和宗教信仰一样,人们对民主制度的理解程度、怀疑和信仰的程度,在整个社会的不同层面及历史的不同时期,都有很大的不同。
\begin{flushright}
—— 胡安·林茨(Juan J. Linz)
\end{flushright}
\end{tcolorbox}

还没有人开始以系统性的方式去思考,技术变革对破坏工业主义和改变收入分配将产生的后果。很显然,如果信息经济中的收入差距大到惊人,那么民主就只是一种披着合法外衣的寄生主义。也很少有人注意到,工业政府的一些制度,与后工业社会的大政治条件之间,隐含着不兼容性。然而,无论人们是否能够看清这些矛盾,随着世界各地政府失败的案例与日俱增,它的后果会越来越显眼。现代时期出现的政府制度,反应了一个或多个世纪前的大政治条件。它们在农业社会向城市工业社会的转型中,幸存了下来。但信息时代需要新的代表机制,避免工业政府那种长期的功能失调,甚至出现苏联式的崩溃。

可以预见,随着政治承诺不能兑现,政府信用和制度支持被耗尽,很多国家都会出现政府破产的危机。最终,必将会出现新的制度形式,它可以在新的技术条件下,维护个人自由,同时使人们的共同利益得以表达和生存。

所有这一切,都指向我们在 20 世纪所熟知的大众民主的终结。现在的问题是,拿什么来替代它。如果大众民主的唯一替代品就是独裁——在这种制度下,人们对自己的命运毫无发言权;那么,很多人可能会受到诱惑,加入到新卢德派,去“反抗未来”。

\subsection{新制度}
幸而,令人欣慰的是,抛弃大众民主,不等于只能选择独裁。信息技术提供了新的选择。人们无需在“大众生产、大众消费、大众教育、大众媒体、大众娱乐及其他一切”的约束下,进行集体选择;信息技术可以使消费者真正地定制主权服务。这是很有可能的,因为大规模运作不再是必要的。所以,我们相信,信息时代会出现新的治理形式,就像农业革命和工业革命,都催生出了各自独特的社会组织形式一样。

这种新制度可能是什么?要理解这一点,请忘掉你在名不副实的“政治学”教科书中读到的一切。信息时代的治理制度,将突破传统的思维界限。朝着这种新制度发展的历史进程已经开启。它们是对一种尚未被充分利用的资产——主权利益——的重组,现在还只是鲜为人知的临时排演。世界上的民族国家,已经被各地的独立运动和分解权力的强大力量搞得惊恐不安,他们不仅严密监视着这一切,而且已经联合起来,形成了垄断联盟,全力以赴地加强边界固定。虽然在 20 世纪 90 年代,世界上新独立的国家数量有所增加,但那主要出于两个事件,即前苏联和前南斯拉夫的多民族共产主义独裁政权的解体。值得注意的是,包括美国在内的其他先进的民族国家,都在操纵着,尽最大可能去延长苏联的存在;而很少有政府支持南斯拉夫的解体。前南斯拉夫各共和国的独立,直到独立主义者通过军事斗争夺取了实际的控制权以后,才开始得到承认。世界大国们很乐意看到,手无寸铁或武装薄弱的独立主义者,被他们的塞族折磨者所屠杀。即使是在遥远东方的中国,一个强大的民族国家,在维护南斯拉夫的残余方面并没有直接的利益,也极力反对被科索沃压迫的阿尔巴尼亚人实现民族自决。不过,讽刺的是,这种坚守边界的狂热,并不能真正预防,反而更有可能导致主权被分解。全世界脆弱的民族国家,对公开的独立运动和政治瓦解行为强烈抵制,使曾被公认的主权,成为了一种有价值的超验资本,可以被拥有它的国家自愿分割或租赁。

厄加勒斯弯特许自由区就是一个例子,它包括西非海岸的普林西比岛 50 平米公里的土地;该自由区说明了,一个国家可以自愿分割出部分主权,用来打造免税的、私有的管辖区。虽然该部分领土仍然处于圣多美和普林西比民主共和国境内,但它的管理完全是私营的。那里的治理,是根据 WADCO 公司所管理的合同进行的;WADCO 即西非发展有限公司,是一家注册在南非的私营企业。该自由区的书面语言是英语,而不是圣多美的官方语言——葡萄牙语。通行的货币也不是圣多美毫无价值的本地货币——多布拉,而是全球硬通货——美元。它的安全防卫,不是由圣多美和普林西比民主共和国的警察提供的,而是由 WADCO 公司雇佣的私人安保。圣多美的商法不适用于该区的商业交易;所有的商业争端,都必须根据巴黎国际商会的规则,通过国际仲裁来解决,圣多美的法院没有管辖权。除了部分受到严格管制及细微方面的例外,圣多美的税收政策及其他官方垄断,在这里都不适用。例如,电信管制,在自由区内被自动解除。在支付租金并遵守其他特许条款的前提下,WADCO 有权自动地、重复地续租这一私有的分散型主权管辖区。第一次续租时间为 2047 年,每次为期 50 年。

WADCO 在圣多美和普林西比所实现的成功,可以并必然会被其他管辖区的很多人所效仿。杰昆·阿吉雷(Joaquin Aguirre)是 21 世纪发展的真正先驱之一,他在东玻利维亚的阿吉雷中央港湾(Central Aguirre Portuaria),打造了一个类似的私人主权区。阿吉雷是一位千万富翁、小说家和发明家,他是联合国的共同创始人,也是玻利维亚共和国的前参议员,在很多方面都是一位开拓者。在帮助建立联合国的半个世纪之后,阿吉雷现在又成为了 21 世纪主权个人的原型。他创建的“自由区”,免除了玻利维亚大部分的税收、关税和监管措施,为新形式的私有化城邦指明了方向;在信息时代,越来越多的成功者将朝着这一目标迈进。它还确凿地证明了,大政府的辩护者经常讴歌的大众生活,在阿吉雷发起的自由贸易区带来的经济繁荣下,其质量会得到大幅度的提高。随着时间发展,全世界事实上的城邦国家会不断涌现。作为一个个人,如果你在经济上实现了足够的独立,那么你就会像阿吉雷一样,获得终极的独立。如果其他商业化的分散型主权,不能为你提供舒适的栖息地,你可以推出自己的专属小国,像中世界所有的公爵一样享有独立的地位。所以,在未来,你不用再与煽动家和政治掮客进行拉锯战,防止你的资产被大众民主的叫嚣者夺走并瓜分;你可以建立自己的私人领地,进行私人治理。

从大众民主到个人自治的终极形式——主权个人,是一场彻底的阶段性变革。它不需要从根本上改变大众舆论,也不需要幻灭的选民发起奇迹般的投票,废除掉大众民主制度。这场革命,现在就可以开始;而事实上,它也已经开始了,以不为人知的方式,通过把主权出租为免税区、“自由区”和自由港。到时机成熟,主权将被不断分割,直到完全分裂,达到无法再进一步分割的程度,不然所产生的利益将不足以覆盖移交的成本。基于摩尔定律,以及吉尔德的推论,即带宽将每年增加三倍;现在来看,没有任何依据可以预测,权力分解的趋势会提前结束。

相反,我们预计,虽然目前致力于维护大众民主的民族国家,其力量看上去还坚固无比,但它们很快将分裂成数以万计的碎片;未来的治理系统,会更多地让人想起中世纪,而不是工业时代。

到那个时候,即使还残存一些国家,保留着大众民主,它们的政策也将发生重大的改变,以便适应元宪政的现实要求。民主的忠实拥趸威廉·基赫(WilliamKeech),在《政治经济学:民主的代价》(Economic Politics: The Costs of Democracy)一书中写道:“人们学会了要求他们觉得自己能得到的东西,而如果他们觉得不喜欢自己想要的和已经得到的,他们也会改变主意。”换句话说,在 20 世纪的末尾,大众民主连同代议制政府制度,虽然还大行其道,但这可能是一个“卖出的信号”。即使在它们自己设定的条件下,这样的决策机制,也难以确保能经受住时间的考验。如果你把目光投到政治之外,你会发现,几乎没有任何的高管、行政人员、教练或其他专业的领导者,是通过民主选举产生的。恰恰相反,最成功的领导者,通常是业主通过选拔机制雇佣的;在这个过程中,那些拥有最大利害关系的人,在决定结果方面也拥有完全不平等的且不成比例的发言权。如果民主选举真的是一种优越的方法,能够选出有能力的领导,那么你应该会看到,它是一种普遍适用的决策机制。然而,它基本上局限于政治领域。简而言之,依据目前的证据,假设主权服务的不良供应,是被占据主导地位的民主决策所阻碍,比假设相反的情况更加合理,即认为是受到了公司和商业组织的影响,因为它们是由业主认命的行政人员管理的,而不是通过举手表决。

到 21 世纪中期,当主权分散的自营管辖区大量激增,可能会确凿地证明专有管理(proprietary administrative)的优势。选民们会看到,在大众民主的束缚下,他们其实是在受苦。因此,就像基赫教授所指出的,他们将会认识到,由雇员控制的政府所带来的好处远低于其成本。醒悟之后,他们会走上改革之路。即使是欧洲和北美的选民,虽然现在看起来强烈地反对革新,最终他们会投票使自己的选区更适合专有管理。多数人可能会愿意,甚至很高兴抛弃掉政治的闹剧,转而支持政府的专有管理;这将为缔结和执行合同创造最佳的环境。

只要政府以及它惯用的配套设施还存在,它就能以全新的方式被组织起来。在末日到来前的某个时候,在某个管辖区的某个地方,某些人会意识到计算机技术的潜力,利用其实现真正的代议制政府。庞大的竞选开支以及长时间的政治选举,都是饱受怨言的;这些问题其实可以在瞬间内解决。代表没必要经由选举产生,完全可以通过随机的分类挑选出来,这些代表的才能与观点,在概率上更有可能与广大民众的水平相一致。

这其实只是古希腊抽签选拔制度的现代版本。在他的权威历史著作《希腊和罗马的投票与选举》(Greek and Roman Voting and Elections)中,斯塔夫利(E. S. Staveley)详细阐述了,希腊的许多职位,从地方法官到执政官,都是抽签产生的,而不是选举。尽管机会的随机性比较机械,有一定的局限,但通过使用抽奖机(allotment machine),或雅典人所说的“cleroterion”,他们很巧妙地实现了想要的效果。

抽奖机是用“一系列的黑球和白球”作为随机计数器,用它们来决定哪些人可以进补候选的职位,并且“决定议会中各派系轮流执掌市政厅的顺序”。这个想法的经典来源,可能使它更加可信。但其实它的主要吸引力在于,它可以避免政治游戏中自我选择的弊端。从统计学上看,它能够保证在公共事务中,出现更少的律师和自大狂。

通过这种方法,立法机构将由真正的代表所组成。他们不是因为追逐权力而被召集到了一起,毕竟,他们被再次选中的机会微乎其微;所以,他们可以自由地处理政府事务,理性地分析问题,并制定出合理的政策。

\subsection{正直的委员会}
今天的政客们,一心只想优化自己的选票,基本没有动力去连贯一致地分析问题。

因此,与企业家、公司高管和运动教练相比,他们在实际解决问题方面的表现乏善可陈,一点也不足为奇;因为企业家他们是根据业绩表现获得报酬的。按业绩给立法者支付报酬,可能不会使每个当选者都像李光耀那样有效率。但是,根据领导者业绩表现支付报酬,只是李光耀“弹性工资”(flexiwage)计划的一个逻辑延伸;该计划在新加坡非常成功,它规定,根据新加坡经济的实际增长情况向政府雇员支付薪酬。所以,我们相信,如果将立法和行政人员的薪酬,与一些客观的绩效标准(如税后人均的增长)相挂钩,他们的表现肯定会大不一样。根据业绩付酬,可能会使他们的表现提高一千倍。

能够提高人们实际税后收入的政策,对社会整体是大有裨益的。所以,如果总理和总统们的政策能够带来收益,为什么不能奖励他们其中的一小部分呢?这部分奖金可以通过一种微小的税收来获得。通过这样的制度安排,整个社会可能就不会面临,像尼克松和克林顿这样具有特殊政治才能的野心家所制造的威胁了。

\begin{tcolorbox}
他们给他带来了金银和衣物,但这位‘基督’把这些东西都分给了穷人。当有人献上礼物时,他和他的女伴会匍匐在地,祈祷致谢;但随后他又起身,命令众人向他敬拜。后来,他组织了一支武装队伍,带领他们穿越乡村,拦截和抢劫路上遇到的旅行者。但他的野心不是为了发大财,而是要受人崇拜。他把所有的战利品都送给了那些一无所有的人,我们估计,应该也包括他的追随者。
\begin{flushright}
—— 诺曼·科恩(Norman Cohn)
\end{flushright}
\end{tcolorbox}


\subsection{弥赛亚式的人物}
人们很少注意到,选举政治会诱使不正常的、大救星式的人物进入权力的中心。

即使在民主制度出现之前的农业社会,这种人也是存在的,而且通常都会对社会秩序产生严重的威胁。回顾一下布列塔尼的基督尤多·德·斯泰尔、8 世纪的阿德尔伯特、11 世纪的埃昂、安特卫普的坦切姆、梅尔基奥尔·霍夫曼、贝伦特·罗斯曼及其同道的生涯,有几个特点非常突出。他们的施政才能越是立竿见影,造成的危害就越大。由于当时的国家还没有建立大规模的强制体系,这些早期的原始政客(protopolitician),经常自发地进行抢劫和掠夺,以便获得现金分发给他们贫穷的追随者。

\subsection{原始政客的行动}
他们的滑稽故事,给人的感觉就是生错了时代的人才,就像在篮球被发明之前,读到有七英尺高的人在操场上跑来跑去。今天,因为有了 NBA,身材畸高的可以靠运球和扣篮赚到几百万美元。如果篮球运动消失了,他们就只能回缩到社会的夹缝里,大多情况下只能作为马戏团的景点,或者玩玩杂耍来生存。在现代政治被发明之前,有煽动力的人被吸引到了农业世界里最接近政治的地方:巡回布道。他们在人群中大声疾呼,像政治家一样,向任何愿意追随他们的人夸下海口,承诺带给他们更好的生活。过去和现在一样,穷人是最主要的煽动目标。诺曼·科恩在他关于千禧年信徒运动的卓著《千禧年的追求》(The Pursuit of theMillennium)中,记述了在投票政治诞生之前,众多弥赛亚(救世主)式领袖的职业生涯。从他的描述中可以很容易看出,这些人的个性类型,与现代时期的魅力政治家如出一辙:

领袖——像法老和其他诸多“神王”——具有一个理想父亲的所有属性:他睿智无比,他公正无偏,他保护弱小。但另一方面,他也是一个儿子,他的任务是改造世界,他是要建立新天新地的弥赛亚;他可以对自己说,“看,我使万物更新。”既是父亲又是儿子,这种人物都是巨人,是超人,是全能的。在人们的想象中,他们具有无穷的超自然力量,像光一样发射出来……。此外,身披着这种神圣的光辉,宣扬末世论的领袖,更具有创造奇迹的独特能力。他麾下的军队可以攻无不克战无不胜,他的存在可以使土地里五谷丰登,他统治的时代完美又和谐,是腐败的旧世界闻所未闻的。

当然,这种形象是一种纯粹的幻象,可以说它与背后的人的真实本质和能力都没有关系,不管这些人是曾经存在的或是未来可能存在的。但是,这种形象可以投射到一个活生生的人身上,而想要接受这种投射的人也从不缺乏,他们热切渴望成为无懈可击、制造奇迹的救世主。……而他们之所以能够得道飞升,秘诀不在于他们的出身,也不在于他们的教育,而在于个性。对于这些穷人的弥赛亚,当代人通常认为他们具有出众的口才、指挥才能和个人魅力。总之,这些人给人的印象是,他们中间可能有人自知是冒牌货,但大多数都真的把自己看成是神的化身。而这种全心全意的信念,很容易为那些最渴望获得末世救主的大众所接受。 那些未来的千禧年救世主,实则常常使中世纪社会陷入动荡与不安;这段对他们的描述简明扼要,令人赞叹,但它还不能完全体现出科恩权威大作的味道。在阅读整部作品时,从这些预言家的滑稽行为中,我们清楚地辨认出现代煽动者的熟悉特征:雄辩的口才、“个人的磁性”、“弥赛亚式的自命不凡”,以及反复出现的渴望作为护民官被穷人崇拜的欲望。

对于这些冒牌货,中世纪的处理方式与 20 世纪末民主政治的做法大不相同。在中世纪,这些人基本上都会被处死,而在 20 世纪末,现代民主政治为他们提供了一个公开的渠道,使之可以合法地取得民族国家的权力。

把地球上最大、最致命的组织的控制权,通过人气竞赛的方式,经常性交到煽动力十足的赢家手中,这样一种制度,从长远来看,必然会使人们遭受痛苦。

\subsection{付费给领导,以获得更好的表现}

如上所述,一个组织要想获得优秀的领导力,有一个很简单也更优越的方法,就是雇用制。这是在竞争性经济中广泛使用且普遍成功的方式。合理的选拔程序,加上建设性的奖励机制,对业绩良好的领导者进行奖励,就可以使有能力的人成为政府的掌舵者。这种方法还能调动新型人才;在传统模式下,这些人通常不会对政治和治理问题感兴趣。

如果根据实际取得的社会成果来支付报酬,就能吸引到世界上最具才华的行政人员,来管理岌岌可危的政府。在每一个先进的西方国家,一个领导人如果能够明显提高公民的实际收入,按理说,他的薪酬应该远高于迈克尔·艾斯纳\footnote{译注:曾任迪士尼 CEO}。在一个更美好的世界里,每一位成功的政府首脑都会是千万富翁。

\subsection{电子投票}
另外一种可以明显解决代议制弊端的方法是电子投票。全体公民,或通过防篡改程序抽签,选出来有代表性的一部分人,对立法提案直接进行投票。有了计算机技术,电子公投已经不是问题。电子投票可以很容易地与抽签结合在一起,缩小对特定问题的投票人数。总之,从原则上讲,对于未来的选民来说,了解政治问题的难度,要远远小于揣摩政客以及评价他们的施政观点,更不要说预测政客上任后会做什么。这一点尤其困难,因为政客及其幕僚越来越擅于包装和操纵他们在公众面前的形象。

\section{商业化的主权}
我们期望看到有新的东西出现,来取代政治。虽然我们上面所谈到的每一种方法,都可以被尝试,也能取得一些成效;但是,我们想要的不是政治改革或改进,而是政治将沦为一种过时的东西,被普遍地抛弃。这并不是说我们想看到独裁统治,相反,我们期待看到创业型的政府,即主权的商业化。

不同于独裁,甚至也不同于民主,商业化的主权不会阻止人们选择的自由。它将为每个人提供更大的表达空间。而对于有能力利用其优势的人来说,商业化主权提供的决策与自决的空间,比以往任何形式的社会组织都要大。

\subsection{定制政府}
为避免这听起来像是千禧运动,提醒大家不要忘了微技术,它带来的小型化和对权力的分解。微技术有利于定制化而不是大规模的生产。现在你去商店买蓝色牛仔裤,可以根据你的尺寸定制,裁切版型,然后发到半个世界之外的工厂缝制。

等到新的制度发展成熟,能够最终适应信息时代的大政治条件时,你将可以像选购牛仔裤一样,去定制符合你个人需求和品位的管理服务。

阿尔文·托夫勒(Alvin Toffler)曾批评过信息技术可以使公民变成顾客的观点。

托夫勒说,“这种模式太狭隘了。不管我们是否喜欢,外面都有一个宗教和情感的世界,不能简单地人与人之间的关系归结为契约关系。”我们认为这个说法是错误的。出于前文探讨过的原因,我们同意把“民族主义的情感世界”简化为“契约关系”是很难的;但这并不等于不可能,更不等于说契约关系不好。毕竟,民族主义中如果能少一点非理性的狂热,就可以挽救千千万万的生命。

\subsection{“进入、退出”与“发声”}
当然,主权商业化是一个陌生的概念,显然连阿尔文·托夫勒都不熟悉。但是,它的中心思想——经济的表现形式,是生活在 20 世纪末的人们耳熟能详的。在任何稍微有点自由度的经济体重,消费者都可以通过购买服务和产品,来直接表达自己的欲望;或者是通过不购买、不消费。当你对某一产品或某一服务感到不满时,你可以选择“退出”来表达你的意见。换句话说,你可以把你的业务转移到其他地方。

在前面的章节中,我们分析了,由于信息技术的发展,很快你就可以在网络空间创造财富,并且完全不受民族国家的掠夺。这将会形成一种事实上的元宪政要求,也就是说,政府在要你支付账单之前,必须真正地提供令你满意的服务。为什么呢?因为在这里,你缴纳所得税实际上是自愿的,就像这套理论原本所设想的那样。

\subsection{避免“冗长的政治渠道”}
如果信息技术如期所愿地发展下去,它将确保政府真正地被客户所控制。作为客户,你会有丰富的选择,一开始是数百种,然后可能数千种;你可以与民族国家签订私人税收协议,或者完全逃离民族国家,去到新兴的小型主权管辖区,这样可以直接降低你的保护成本。“进入”和逃离或者说“退出”这些合约,就是你作为客户的欲望的经济表达。用你的脚和钱来投票,有一个很大的好处,那就是它能带来你想要的结果。

你作为客户,有选择“进入”和“退出”的自由,相比之下,民主政治的表达方式是怎样的呢?如果有人对某种产品或服务感到不满,特别是对政府提供的或受政府严格监管的,他可以给美国总统写信,或与当地议员及其他相应的民选官员会面,来“发声”表达自己的意见。这种请愿方式有时候会有效果,但不一定,通常都会失败。如果不能成功,想通过自己的“声音”进行变革的人,可以组织示威游行,或在报纸上刊登整版的广告,甚至自己直接参加竞选。

政治的表达模式确实给人提供了一个发声与演讲的渠道。但它的弊端是,你很少能通过自己的行动,获得满意的结果,或改善自己的处境。面对政府提供的劣质产品和服务,你不得不继续买单,直到你能说服整个政治进程同意你的要求,做出改变。

在西方国家,现在几乎是在所有国家,要达到你的目的,意味着必须确保达到民主制度要求的多数支持。这个需要多数人参与的门槛,对你想要实现的相对直接且理性的目标,施加了巨大的交易成本。

米尔顿·弗里德曼(Milton Friedman)在《资本主义与自由》(Capitalism and Freedom)一书中,为了推进他的教育代金券的建议,指出了经济表达相对于政治表达的优点:家长可以把孩子从一所学校退学,然后送到另外一所,以此来直接表达对学校的意见;(有了这种券)他们可选择的程度就比现在大得多。现在他们基本上只能通过改变居住地,来表达自己的不满;其他的手段,就只有繁琐的政治渠道。 阿尔伯特·赫希曼以政治党派人士的身份发言,他对弗里德曼倾向于“以退出‘直接’表达自己对某组织不满的方式”表示有异议。当然,一个在经济学方面训练不足的人,可能会天真地认为,表达自己意见的直接方式,就是把它说出来。

是通过市场机制,如作为客户给予或撤回支持;还是通过“繁琐的政治渠道”来表达意见?哪一种更直接、更有效,这是一个复杂而有争议的问题。不同的人会以不同的方式给出自己的答案。对于那些在政治表达中主要以他人利益为代价为自己谋利的人来说,转用经济表达方式所能得到的利益,可能确实比给政客写信所能实现的要惨淡得多。

\subsection{经济表达与“互惠社会”}
对于那些相与伙伴展开“互惠”,而非“胁迫”或寄生型关系的人来说,利用经济型表达方式,能以更短的时间、更少的麻烦、更低的成本,获得更大的满足。

尽管有赫斯菲尔德教授\footnote{译注:Hirschfield,此名全书只出现一次,有点懵。},这一点也很容易证明。

任何一套经济表达方式,包括进入、合同执行和退出,只要让众多的人参与决策,就会沦为政治型的“声音”表达。可以做一个实验来证明。你所需要的就是找几百个人,这些人觉得自己生活中的政治还不够多。在一年的时间里,不要让他们把自己的可支配收入,分散地用到数千次的购买中,而是把这些大量的经济决定转换为少数的政治决定。

首先,所有人同意把他们的可支配收入集中起来,然后放弃以个人为单位的购买行为。也就是说,以后不再是个人拥有数千美元,进行数千次的个人消费;而是每个人将获得一张或几张选票,这取决于集体的职位数量。在任何时候,你都不能直接花钱去买你想要的东西;而只能在少数的几次选举中,使用你的一票或几票,选出代表,由他们来决定,如何使用那一大笔集体财产。

然后,你和其他人一起,来分享买来的东西,而且只能分享执政委员会以多数人的名义批准购买的东西。

这是不是有点像一个“繁琐的政治渠道”了?这还不算。这种模式确实蕴含了,人们在国家政治层面所看到的,有关演讲和说服方面的所有可能。但还有大部分是挫败的可能。

例如,如果你喜欢吃新鲜的西兰花,而这个群体在食物口味上的分布很平均,那你就麻烦了。你所在的小组中,可能有部分或大部分人,宁愿用食品补贴去买更多的红肉,而不是新鲜的蔬菜。为了阻止食堂委员会把整个年度的蔬菜预算,都挥霍在豌豆罐头和玉米上,你可能要站出来“发声”,表达自己的观点。你也许会提请小组成员注意,与红肉中的饱和脂肪及胆固醇相比,摄入更多的维生素和植物营养素(如西兰花中的磺胺素),是更加有益健康的。

当然,在这种政治模型中,究竟你如何让人理解这一点或你的任何观点,是一个很大的难题,就像倡议任何政治计划或参与竞选一样。你可以发表演讲,当然这需要你把群体中的大部分人聚集在一起,让他们愿意听你讲,然后你还要说服他们。你可以印刷传单,如果你参与的政治游戏允许这种“竞选支出”的话;你还可以写信。但这两种方法都需要其他的参与者有足够的识字能力。

\begin{tcolorbox}
“它描绘出这样一个社会,其中的绝大多数美国人,并不知道他们不具备,在这个日新月异的技术社会和国际市场上谋生所需的技能。
\begin{flushright}
—— 理查德·莱利,美国教育部,《美国成年人的教育程度》
\end{flushright}
\end{tcolorbox}

\subsection{九千万老年痴呆症患者?}
在你的这个政治模拟演习中,如果参与者恰好都是美国人,那你很难得到任何有说服力的信息,特别是这些人跟美国选民的整体素质相似的话。世界上最强大的民族国家的公民,文化程度很低者的比例惊人;在有史以来对美国成年人素质进行的最彻底的调查中,很无奈地证实了这一点。这项名为“美国成人识字率”的研究表明,为任何的政治辩论找到一个识字的听众,都绝非易事。很大一部分美国人,可能占到 15 岁以上的大多数,都缺乏评估观点及作出判断的基本技能。

另外,根据美国教育部的数据,9000 万美国人不会写信,看不懂公共汽车时刻表,甚至不会用计算器做加减法。如果说有 9000 万美国人有程度不同的老年痴呆症,大概也就是这个样子。其中有 3000 万可以被认为非常无能,连回答问题的能力都不具备。

所以,在你的政治实验中,如果你关于健康的论点不能扭转局势,那你可以向动物权利活动价求助。也许你可以请他们到食堂委员会抗议你的对手,或者在有影响力的成员家里,大肆宣扬杀牛的罪恶。

这个例子还可以无限地延伸下去,可能远远超过理性人所能容忍的限度。它清楚地表明了:(1)任何进入或退出的经济型表达,都可以通过集体决定,而转变成一种政治型的声音表达;(2)集体决策,虽然欢迎参与者进行辩论,但它冗长又累赘,而且很难实现合理的结果。

这些都已为历史的经验所证明。要发动必要的人力和物力去改变一个民主国家的政治进程,难如登天。所以,重申一下,这很可能就是几个世纪以来,民主福利国家能够在与其他政府组织形式的竞争中幸存下来,并在工业时代末期占据主导地位的原因。民主作为一种政治制度之所以成功,正式因为它的运作模式,使客户难以控制政府或限制国家对资源的攫取。

然而,在信息时代,国家对你个人事务的无限参与,不再具备军事上的优势,聪明的人会找到更好的方式,来获得政府提供的少数真正有价值的服务。如前所述,现行的权力,将会从无力支撑它们的集体机制中承包出去。我们期待看到效率战胜大规模的权力。尼尔·芒罗曾简明扼要地指出;“美国的经济,将越来越多地由计算机化的信息驱动,而不再是人力或大规模的生产,它将在这个拥有 500台电视频道的世界里赢得战争。计算机化的信息存在于赛博空间——一个由计算机网络、通信卫星、调制解调器、数据库和公共互联网不停复制所创造出来的时空新维度。”在这样的世界里,组织大规模的军队,意义甚微。效率将前所未有地重要。我们在第 6 章及其他章节讨论过,由于微技术开辟了一个全新的保护维度,个人能够在任何单独政府暴力垄断的领土范围之外,创造并保护自己的资产,这是人类文明史中的第一次。这些资产基本上只受个人的支配。因此,对你和其他主权个人来说,“用脚退票”,退出主要的民族国家,与一个边缘民族国家或小的新型主权签订私人保护合约,是完全合理的;这样一来,你只需支付具有商业效益的最低金额,而不是你的大部分资产。简单来说,你可能会因为 5000 万的差额,而搬到百慕大。

\subsection{先退出,后签约}
在早期阶段,能够刺激主权商业化发展的,只能是来自于用退出表达自己经济利益的个人。这种选择在美国是最难的,但也是最有价值的。比尔·克林顿总统与共和党国会,强加给资本家的“柏林墙”,不仅与《独立宣言》相抵触,因为宣言主张个人有权抛弃掠夺性政府的统治;而且它也与美国民族主义者在 1960 年代的口号相矛盾,当时他们自信满满地叫喊着“要么爱她,要么离开她”。离境税是对那些选择离开的人征收惩罚性的税收,以此来强迫忠诚。然而,这种报复性的立法,只会让人联想到罗马帝国落幕时期,对携财产逃离者实施的惩罚;它也可能在无意之中,为其后的信息时代制定更合理的政策,设定了一个参照的框架。

等到某个时间点,当大量有能力的人选择离开,并在海外积累了足够多的财富,美国及其他高税率国家的政府,为了提升自身的吸引力,将允许公民或绿卡持有者通过支付离境税来免除以后所有的税负,但并不要求他们真的离开。毕竟,从经济上讲,把效益高的人留在境内创造价值,比放他们去到其他条件更好的管辖区,更符合主权国家的利益。换句话说,离境税可以成为一次性买断税负的模式。

征收离境税的政府,根据类似于瑞士和其他地方目前提供的私人条约的条件,接受那些已退出的人重新回来居住,可以获得更多的利益。

美国或其他国家的政府,如果采取这种举措,那将是一种优化收入的理性姿态。

终究而言,对保护服务的竞争会迫使税率下降,并将征收条件调整到更文明的标准。未来的主权个人,将不再受制于立法机构颁布的税收制度,而是将通过谈判达成私人条约,实现可接受的、定制化的一揽子服务政策。

\section{真正的信奉者被冒犯}
当然,我们完全不认为,上述的大部分做法会很快受到普遍的欢迎。个人的去国家化,以及它所意味的主权商业化,会触犯到 20 世纪政治陈规的真正信徒。和已故的克里斯托弗·拉什一样,这些人觉得,政治的萎缩会对大多数人的福祉造成威胁。在他们看来,复兴工业时代的政治,加强收入再分配,可以解决信息技术的竞争压力给很多人带来的痛苦。

小迪昂(E. J. Dionne, Jr.)是《华盛顿邮报》的政治记者。和拉什一样,他也在回味、留恋着政治。他为一种呼吁社会民主平权的激情运动在发声;在未来的几十年里,这种运动会制造出更大的声响,因为信息时代的大政治现实,会更强烈地破坏现代世界遗留下来的体制。迪昂认为,20 世纪富裕地区的生活水平和物质条件的普遍改善,主要是由于民主政治,而不是技术或经济的发展。他发出的信号是,未来的希望在于扩大政治对信息技术的主导权:美国及整个民主世界的当务之急,是重新开始民主改革,它是工业时代获得成功的政治引擎。

信息技术本身并不能构建一个成功的社会,就好像只靠工业主义,世界也并不会变得更好。……即使最非凡的技术突破和最精妙的网络应用,也不能使我们免于社会崩溃、犯罪丛生或公义泯灭。只有政治,即人类自我组织的艺术,才可以处理这些问题。 迪昂和其他像他这样的人并不明白,20 世纪的社会之所以特别有利于系统性的强制,这其中的条件,并不是由任何人类机构所选择的。“人类自我组织的艺术”,这种说法在现代之前无人理解。社会过于复杂,没办法完全确认为是有意识自我组织的努力结果。现代时期的民族国家是自发出现的,它是工业技术提高了暴力回报率的偶然的副产品。而今天,信息技术正在减少暴力的回报。这就会使政治变得过时,无论人们多么希望它能在下一个千年延续下去,也挽留不住。

\begin{tcolorbox}
它们\footnote{译注:指上天的律法。}的存在不限于昨日和今日,而是永久的;它们从哪里来的,没有人知道。
\begin{flushright}
—— 索菲克勒斯《安提戈涅》 
\end{flushright}
\end{tcolorbox}


\subsection{“他们不再像以前那样制定法律了”}

对“制定法律”的强烈渴望,好像是 20 世纪政治常识的一部分,但这绝不是人类全部文化中的普遍现象。如果它在未来消失,那也只是一个周期中的一段;这个周期在几个世纪的时间里,不断地盈亏涨落。例如,早期的希腊人及其他地方的人,认为法律是无法人为制定的。用哲学家卡西尔(Ernst Cassirer)的话说,希腊人认为“‘不成文的法律’,即正义的法律,在时间上是没有源头的”。像其他前政治时期的民族一样,他们认为,没有人可以改进自然的、“几何的”正义法律,这些法律不是由任何人类力量所创造。

他们不相信有“立法者”。正如卡西尔所说:“我们要通过理性思考来寻找道德行为的标准;正是理性,也只有理性,才能赋予它们权威。”从这个意义上说,任何试图通过立法行为,把法律强加给社会的做法,就如同企图用立法改写几何。

\subsection{立法是一种亵渎}
在中世纪的大部分时间里,类似的对“立法”的抵触情绪也普遍存在,只是原因非常不同。约翰· 莫拉尔(John B. Morrall)写道:“对德国人来说,法律自古就有。”它是对部落中个体成员的“权利保障”。国王和议会:还没有制定新法律的想法;从中世纪早期的观念来看,这种想法不仅是多月的,而且是半亵渎的。因为法律和王权一样,拥有神圣不可侵犯的地位。相反,国王和议员们认为,自己只是在解释和澄清,已有的完整法律体系的真正含义。

日耳曼人的习惯,给中世纪的人们传递了一种永远无法磨灭的理念,即使在现实中并没有据此而实践。这种理念就是,良好的法律可以被重新发现或重新表述,但永不会被重新制定。 在经历了 20 世纪的过度立法之后,这种古老的态度看起来有些奇怪。现在,希望把国家的强制力用于私人目的,特别是收入再分配,几乎成为了人类的第二天性。

\subsection{哀悼}
因此,在政治最后的日子里,总是会出现一些悲伤的歌曲;这一点也不意外。不仅因为它们反映出,大多数思想家对大政治必然性的盲视;也很少有政治记者,像迪昂,愿意接受政治明显的萎缩和消亡,因为那可能会使他们重回犯罪的行当。

在中世纪末的时候,支持复星骑士精神的呼声就很高。可以看一下《Ii Libro delCortegiano》即《廷臣论》这本书,由巴尔达萨雷·卡斯蒂廖内伯爵(BaldassareCastiglione)写于 1514 年,1528 年在威尼斯由阿尔杜斯出版。

卡斯蒂廖内渴望骑士美德的回归,这一点深入人心。但对于一种已经无效的生活方式,无论人们多么渴望,也无法将其挽回;在 16 世纪不会,在 21 世纪也不会。

我们在解释大政治理论时曾经论述过,当今世界变革的最重要力量,来自于是技术的驱动,而不是大众的意见。如果我们的大政治理论站得住脚,那么,现代社会——以公民身份的概念以及围绕国家而组织的政治制度为特征,之所以能够取代封建制度以及围绕个人誓言与忠诚组织起来的骑士精神,原因不在人们的想法,而是新技术带来了成本和收益的变化。骑士精神的消亡,并不是因为卡斯蒂廖内或者其他人,没能说服对其有控制力的正义的民众,在国家事务中不能没有荣誉和道德。恰恰相反,卡斯蒂廖内的《廷臣论》对主教提出了批评,也对他同时代的尼科洛·马基雅维利在其《君主论》中称赞的行为进行了抨击。但那又怎么样呢?最终,马基雅维利赢得了更多的读者,这不是因为他在《君主论》中的观点更加雄辩,而是因为他的建议更适合现代的大政治条件。

20 世纪杰出的哲学家恩斯特·卡西尔(Ernst Cassirer),在讨论“马基雅维利的道德问题”时说道:该书以完全超然的态度,讲述了获得及维持政治权力的方式和方法。而关于如何善用权力,书中只词未提。从当下来看,没有人会怀疑,政治生活中充满了恶行、背叛和重罪。但在马基雅维利之前,没有一个思想家剖析过这些罪行的艺术。很多人在做,但没有人教过。马基雅维利乐于成为传授手腕、背叛与残忍艺术的老师,是一件闻所未闻的事。 简而言之,《君主论》是一部激进的著作,它贡献了一个近代的政治秘诀,即有野心的统治者,可以不惜以他人为代价来推动其事业的成功。事实证明,马基雅维利所推崇的行为,非常匹配权力时代的政治特性。但是,被现代政治家视为精明的政治手腕,两面三刀的艺术,相比前几个世纪成长起来的骑士文化,是令人愤慨和难以接受的。

如我们此前所述,骑士精神的美德,强调对誓言的极度忠诚。在一个以个人服侍换取保护的社会里,这是一种必需。封建社会赖以生存的交易方式,并不会自发地在人们中间重复出现,使他们可以在被胁迫的条件下,自由地确定自己的最大利益所在。因此,作为骑士精神基础的封建主义承诺,必须充满强烈的荣誉感。

在这种情况下,没有什么比马基雅维利的建议更具颠覆性的了;他认为在任何时候,君主都应该毫不犹豫地撒谎、欺骗与窃取,只要这么做于他有利。

在 20 世纪即将结束的时候,马基雅维利的论点依然在被研究,因为它对于理解现代政治,以及分析 20 世纪各种罪行与暴政,具有重要的意义。相比之下,卡斯蒂廖内的作品早已被遗忘。也许只有少数读文学的研究生,和一些礼仪史的鉴赏家,会花一年的时间,把《廷臣论》从头到尾读一遍。

在未来几十年后的某个时候,信息时代的大政治条件将会使《君主论》过时。主权个人需要新的成功秘诀,这个秘诀会高度强调正直的品性,因为它对于在国家控制之外运行资源至关重要。可以预测的是,小迪昂及其他还活着的社会民主党人,看到这个建议会很不高兴。

\subsection{由客户制定政策}
在转型初期尤其是这样,这时的大部分管辖区仍然需要制定政策,其倡议者还能获得大多数人的普遍同意。其后,随着民主的消退和主权服务的市场深化,制约“政策”的市场条件会为更多人所理解。

我们现在所认知的“政治”领导力,都是在民族国家的概念框架内,未来它会变得越来越类似企业家精神,政治的意涵会越来越少。在这种情况下,为一个管辖区制定政策,在选择它的可实行范围时,就必须适当地缩小;就好像一个企业家,在设计一家顶级的酒店或其他产品及服务时,他必须考虑到客户愿意为什么东西而买单。例如,很少有度假酒店会设置这样的经营条件,即要求客户做苦力来修理甚至出资添置经营设施。即使是一家由雇员拥有或控制的酒店——就像典型的现代民主国家,想强迫顾客接受这样的要求,也是不可能的,特别是有了更好的住宿选择后。如果客户都愿意打高尔夫,而不愿在烈日下从事繁重的体力劳动,那么至少在这个问题上,想在市场上任意强加什么要求,是行不通的。在这样的条件下,目前的“政治”问题会退居为企业性质的判断,因为分散的管辖区都在努力去寻求,什么样的政策组合能够真正吸引到大面积的客户。

\subsection{政治的萎缩}
当人们理解了这一点,他们的态度会发生巨大的变化。那些权力被分解地区的民众,不会再指望 20 世纪的政治政策,从它冗长的政治辩论所开出的愿望清单中进行选择。另外,在信息时代,人们赚取收入的能力比工业时代更加倾斜,各管辖区会更愿意去满足那些最具商业价值的客户;而这些客户对开展业务的地点又具有最大的选择权。

有鉴于此,对于一个管辖区而言,它最理想的商业政策,能否吸引到焦点群体中的“中层选民”,可能没有我们通常所认为的那么重要。

总之,主权的商业化会促进客户对政府的控制,而非客户的意见将变得无关紧要,或不那么重要。就像吃巨无霸的人对鹅肝的批评,根本不会影响到三星级法国餐厅(如巴黎的 L'Arpege)的成功一样。

\section{“对民主的背叛”}
和已故的克里斯托弗·拉什一样,信息技术的反对者不仅会抱怨它破坏了工作,还会攻击它背叛了民主,因为它使个人可以把资产置于政治强制力之外。也正因为此,新千年的反动派会发现,信息技术对金融隐私的保护特别具有威胁性;他们对所得税和资本税的征收前景会感到不安,因为这些税收将真的取决于纳税人的“自愿遵守”。他们会支持使用一些新奇甚至激烈的手段,来榨取任何看上去富有者的资源,比如“推定税”或直接扣押富人的钱。

\subsection{共同财产}
在我们写作的时候,未来的发展的蛛丝马迹已接近浮出水面。已有早期的证据表明,政府对国际市场的控制力在消失;那些认为个人理当成为民族国家资产的人,也走向了末路。他们希望向政府施加压力,要把国家的公民当做资产,而不是客户。这些反动派认为,所有人的收入都应该视为全社会的收入,也就是说,应该由国家来支配。

我们讨论过拉什在《精英的反叛与民主的背叛》中提出的观点;这不是唯一一篇为民族国家鼓吹的檄文。哈佛大学的政治理论家迈克尔·桑德尔(Michael Sandel),在《民主的不满》一书中认为,“如果政治无法制全球的经济力量,那么今天的民主是不可能生存的;因为没有这种控制,企业将控制一切,人们投谁的票根本不重要。”换句话说,国家必须保留它在个人身上的寄生权力,以此确保政治结果与市场结果相背离。否则的话,通过集体决策,去强制推行不经济的结果有什么意义呢?在我们看来,桑德尔的哀叹和拉什的一样,也只是对了一半。我们承认,如果政府缺乏强迫个人按照政客的要求去行使的权力,民主的重要性将大大丧失。这一点显而易见。而事实上,19 世纪和 20 世纪的民主注定要消失。但桑德尔没有看到,市场力战胜强制力的真正的重要性。他认为民族国家崩溃后,“企业的统治”将是一种伴生的危险,这显然是不合时宜的。

企业很难统治未来的全球经济市场。其实,就像我们所说的,企业甚至不会以人们所熟悉的现代形式继续存在;这一点并不显为人知。一点也不。在信息时代引发的大政治革命中,企业也必然会被改变。我们在前面谈论过,微处理技术将改变“信息成本”,而这些成本决定着企业的“合同关系”(nexus of contracts)。

正如经济学家迈克尔·詹森(Michael C. Jensen)和威廉·麦克林(William H. Meckling)所指出的,公司只是一种法律形式,它为“个人之间一系列的契约关系提供了一个纽带。”用经济学家路易斯·普特曼(Louis Putterman)和兰德尔·克罗斯纳(Randall S. Kroszner)的话说,企业能否继续存活,更不用说作为“不受市场力量影响的官僚指导系统”而“进行统治”,这本身就取决于“市场力量的完备性及市场力量穿透到企业内部关系的能力”。

我们在前面论述过,随着市场力量日益渗透到迄今还属于“企业的内部关系”中去,企业能否生存下去是很值得怀疑的。因为,通过信息技术,依靠价格机制和市场竞标来完全相应的工作任务,而不是把这些工作内化到一个正式的组织中,会获得更大的经济回报。所以,企业会趋向于解体。随着信息技术越来越多地使生产自动化,它将消除企业存在的部分理由,如雇佣和激励管理人员来监督工人。

\subsection{“为什么会有公司?”}
请记住,“为什么会有公司?”这个问题,并不像偶尔闪过的那么微不足道。微观经济学家一般认为,价格机制是调配资源到最有价值用途的最有效手段。普特曼和克罗斯纳就注意到,这意味着像公司这样的组织,并不具备内在的“经济存在的理由”(economic raison d’être)。在这个意义上,公司主要是为解决信息和交易成本而存在的人工产物,而信息技术可以大大降低这些成本。

因此,信息时代将会成为独立承保商的时代,没有“工作”,也没有长期存在的“公司”。随着技术降低了交易成本,个人不仅会摆脱政客的统治,也会阻止“公司统治”。现有企业将与全球各地的“虚拟公司”进行竞争;其激烈程度,恐怕只有少数企业能免幸免于是。随着市场愈加完善,对于大多数公司机构来说,经过竞争还能活下来的,就是烧高香了。

可以预期,这一切的后果,不是个人将受公司的摆布;而是恰恰相反。公司并不会比政客更有能力操控市场。个人最终将在一个真正自由的市场中,自由自主地决定自己的命运。既不受大政府的统治,也不受公司等级制度的约束。

交易成本的削弱,也将终结最近流行的“利益相关者资本主义”(stakeholdercapitalism)的概念。英国工党的托尼·布莱尔和比尔·克林顿的一些幕僚,都喜欢这种概念,它的前提是国家有能力来操纵公司。社会主义已经崩溃,干预主义者在梦想对公司进行严格的监管,以更有市场效率的手段来达到共产主义的目的。这种新的再分配理论认为,管理层、股东、员工和“社区”都是公司的“利益相关者”,因为他们都从公司的长效生存中获得利益,甚至有赖于此。所以,监管应该保护管理者、员工和地方税务机关,延续他们在与企业的历史关系中形成的利益。

“利益相关者资本主义”理论,不仅在终极层面预设了,国家操纵公司决策的能力;而且在基本层面上,它也预设了公司能够独立于市场竞标的价格信号,而长期生存下去。

我们相信,市场深化不仅会削弱民族国家的征税能力,也会削弱政客利用管制措施,把自己的意志任意强加给资源所有者的能力。在未来的世界里,管辖权优势会受到市场的考验,许多地方市场会向来自世界各地的竞争开放,很难指望地方“社区”会有什么办法,能把在当地受宠的企业与全球竞争隔离开来。因此,他们无法保证那些高成本的公司(例如,需要保留不必要的员工和管理人员,或者因为地方政治压力而保持无用设施的开发)能够抵消掉这些成本,而继续经营下去。在工业时代,政客可以关闭市场,限制少数广受青睐的公司的进入,来保护地方的就业或达成其他目的。在未来,当人们可以在全球任何地方进行信息交易时,政府想使当地企业免受全球竞争的压力,犹如螳臂当车。

有人呼吁,构建一种以所谓的独立或志愿部门为中心的“新社会契约”,去吸收“社区”中失业的或被边缘化的工人。这也是行不通的。

杰里米·里夫金(Jeremy Rifkin)就想象着,“在政府和第三部门间之间,构建新的伙伴关系,以重建社会经济。……为穷人提供食物,提供基本的卫生保健服务,为年轻人提供教育,建设便宜的住房及保护环境。”

\subsection{公共用品的消亡}
当然,支持强制措施的人会说,国家权力被削弱,会导致人们无法购买或使用公共产品。这根本就是不可能的,不管是由于竞争还是其他原因。首先,当大部分的地域优势被技术所瓦解,不能提供基本公共产品(如法律和秩序)的管辖区,很快就会失去客户。在最极端的失败案例中,如索马里、利比里亚、卢旺达和前南斯拉夫,大批身无分文的难民会越过边界,去寻找能提供更好的法律与秩序的居住地。当然,这些极端的逃亡,或者说用脚投票的例子,是出于紧急所需,与直接选购管辖区是两回事。不管哪种情况,企业都会迫使地方管辖区去满足客户的需求。

\subsection{“竞争性领土俱乐部”}
“用脚投票”由经济学家查尔斯·丁波(Charles Tiebout)于 1956 年首次提出,它已不仅仅是一种理论。正如经济学家弗雷德·弗尔德瓦里(Fred Foldvary)在《公共物品与私人社区》(Goods and Private Communities)一书中的论述:社会服务和许多公共物品必须通过政治手段来提供,这中说法并没有什么本质上的原因。弗尔德瓦里及其他人的研究,也证实了诺贝尔经济学得主罗纳德·科斯的一个定理——该定理饱受争议;科斯认为,“无需通过政府干预来解决外部性问题”,比如污染问题。企业家可以通过市场手段提供公共产品。世界各地已经有很多企业家在这么做了。弗尔德瓦里的案例研究证明,社区的私有化可以形成一种新的机制,来提供和资助公共产品与服务。

\subsection{通往繁荣之路}
微技术可以带来新的融资和管理手段,来提供一直被视为公共产品的物品。回过头来看,有一些公共产品其实是变相的私人用品。高速公路就是一个主要的例子。

只要交通拥堵不严重,道路和公路就可以被视为是公共产品,尽管亚当·斯密曾提出过批评,认为它们使住在附近的人过度收益,而牺牲了偏远地区人们的利益,因为这些人被迫支付了修路费,却很少用得到。

在信息时代,区分高速公路、飞机跑道及其他基础设施,进行精准的使用定价,并在不中断交通的情况下,征收包括拥堵费在内的通行费用,在技术上是完全可行的。因此,可以谨慎地对交通基础设施进行私有化,并由使用服务的人直接出资。经济学家保罗·克鲁格曼估计,如果由市场对美国的交通基础设施进行定价,可以是美国每年增加 600 亿到 1000 亿美元的 GDP,同时可以提高资源的利用效率,并减少污染。

此外,不要忘了,现代民族国家所做的成本最高的一件事——即收入再分配,根本不是提供公共产品,而是以公共开支提供私人用品。这里的“公共开支”是一种委婉的说法,其实是“以纳税人为代价”。

那么,真正的公共利益该如何解决呢,比如一支能对抗强敌入侵的军队?这样的军事力量历来都是代价不菲的。我们已经探讨过,很显然,一个政府如果不能无限制地没收公民的收入和财产,就没办法为另一场像二战那样的大国冲突提供资金。

但是,这种财政上的限制所构成的威胁,并没有反动派所假设的那么大。原因很简单,不会再有第二次世界大战那样大规模的冲突了。正在解放个人的信息技术,可以保证这一点。

\subsection{超越政治}
所以,与其听任政治来摆布公共服务的质量和性质,不如把“政府”转变为弗尔德瓦里所说的“竞争性领土俱乐部”,以企业化的方式来运作。我们觉得,这些“竞争性领土俱乐部”的决策机制,相比它们满足市场需求的成功表现,最终会变得无关紧要。今天的消费者,在购买产品或服务的时候,很少会关心销售它们的公司是独资公司、有限责任公司,还是由退休基金提名的外部董事控制的公司。

同样,我们也不认为,在信息时代,理性的主权服务消费者,会计较新加坡是大众民主国家还是李光耀的独资企业。