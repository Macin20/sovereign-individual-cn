\chapter[民族国家的生与死]{民族国家的生与死:\\ 民主与民族主义作为暴力时代的财政策略}

\begin{tcolorbox}
打赢战争关键的关键,就在于有足够的资金提供军队所需要的一切。
\begin{flushright}
—— 罗伯特·德·巴尔扎克(ROBERT DE BALSAC) 1502 年
\end{flushright}
\end{tcolorbox}


\section{历史的废墟}

1989 年 11 月 9 日和 10 日,欣喜若狂的东柏林人用大铁锤拆除了柏林墙,这一幕通过电视机在全球直播。喧闹之中,一些初具企业家思想的人,收集了该墙的碎片,后来作为镇纸纪念品卖给了世界各地的资本家。此后多年,这种遗物生意都非常火爆。就在我们写这本书的时候,小杂志上还会偶尔登出广告,售卖来自东德的柏林墙碎片,价格已经达到了高品质银矿石的水平。我们认为,那些购买了柏林墙镇纸的人不要急着转手,因为它们的纪念意义要远大于共产主义的崩溃。我们相信,自从 1495 年 2 月圣乔瓦尼的城墙被炸成碎片,柏林墙是近五个世纪以来最重要的一堆历史废墟。

法国国王查理八世将圣乔瓦尼的城墙夷为平地,那是火药革命的第一次爆炸。正如我们前面所述,它标志着封建历史时期的结束和工业主义的开端。柏林墙的倒塌则是另外一个历史分水岭,那就是工业时代与信息时代的分野。在人类历史上,从来没有出现过如此伟大的象征性胜利,效率第一次战胜了权力。圣乔万瓦围墙的倒塌鲜明地揭示出,暴力的经济回报已经急剧上升。柏林墙的倒塌则是另外一番内涵,它意味着当今世界的暴力回报正在下降。目前还很少有人想到这一点,但它带来的后果将是颠覆性的。

在本章中我们将会探讨,柏林墙对整个民族国家和工业时代的象征意义,远超当晚在柏林的群众或世界各地看直播的数百万观众的理解。柏林墙的建造目的和圣乔瓦尼城墙截然不同,它是为了防止里面的 人逃跑,而不是抵御外来的侵略者。

仅此一点就说明了,从 15 世纪到 20 世纪国家权力的增长,而且远不止一个方面。

几个世纪以来,民族国家使所有对外的城墙都变得多余和没有必要。在国家最初扎根的地区,它对强制力的垄断,使它在内部更加和平;在军事上,它也比人类有史以来所有的主权组织都更加强大。国家从基本被全部解除武装的人口中提取资源,用来压制小规模的对手和敌人。民族国家成为历史上最成功的夺取资源的工具;它的成功基于其从公民身上榨取财富的超级能力。


\begin{tcolorbox}
MTV 不仅仅是在播放音乐视频,以及促销唱片。它是第一个真正的全球网络,是第一个在每个国家都提供单一节目流的电视网络。在这个过程中,MTV 为它的观众——儿童和年轻人——创造了一种单一的全球共享的现实感。最近的研究发现,地球上的年轻人,不仅越来越趋向于喜欢共同的偶像、有共同的口味和职业期望,而且对生活的意义和恐惧都有共同的价值观。他们也普遍认为,政治对自己未来的影响没有自己的能力重要。
\begin{flushright}
—— Jim Taylor and Watts Wacker, the 500-Year Delta: \\ What Happens After What Comes Next  
\end{flushright}
\end{tcolorbox}

\subsection{“要么爱它,要么离开它”(除非你很有钱)}
在民族国家完全过渡到信息时代的新式主权国家之前,西方最强大民族国家的许多居民,将会像 1989 年的东柏林人一样,想方设法地逃离出去。对于在二战前或者冷战初期成年的几代人来说,跨越边境是一种创伤性记忆。但更具有全球视野的年轻一代,并不像那些被民族国家的意识形态灌输过的老年人,认为放弃出生时的国籍是一个不可想象的决定。在一项针对五大洲 25000 名中产阶级高中生的大规模调查中,吉姆·泰勒和沃兹·瓦克报告了很有意思的结果。在 1995-96学年,纽约消费者研究公司 BrainWaves Group 进行的抽样调查中,10 个学生中有 9 个同意,“要得到我想要的生活就只能靠自己。”更加引人注目的是,“几乎有一半的青少年都希望,离开他们出生的国家去追求人生的目标。”作为第一个在 MTV 上发布竞选广告的总统候选人,比尔·克林顿对 MTV 一代的态度可能更加敏感,所以他也在想办法使美国人更难“离开出生国去追求人生目标”。1995年,就在美国的高中生准备发布他们的独立宣言时,美国总统提议制定离境税,要求富有的美国人在放弃国籍时缴纳巨额的赎金,哪怕他们只是带着部分资产出走。这就是“资本的柏林墙”。

克林顿的赎金,不仅让人联想到在东德后期,国家把公民视为其财产的政策;也让人想起衰落中的罗马帝国,为了支撑财政而采取的越来越严厉的措施。《剑桥古代史》中的这一段就说明了这一点:


\begin{tcolorbox}
于是,国家开始使用激烈的方式,去榨取人口中的最后一滴价值。由于经济资源不能满足社会所需,强者为夺取主要份额而斗争,他们的暴力和不择手段,很符合当权者的出身以及惯于掠夺的士兵习性。法律的严厉全部释放到了人民身上。士兵假冒执行官,或者充当秘密警察,在土地上四处游荡。受影响最大的当然是有产阶级,对他们的财产下手相对更容易。在时局紧张的时候,他们是最快遭受到频繁勒索的阶层。
\end{tcolorbox}

当失败的制度尚有余力的时候,他们会对那些试图逃离的人施加刑事处罚。我们再看一下《剑桥古代史》,“如果有产阶级把钱埋起来,或者牺牲 2/3 的财产逃离地方政府,甚至不惜放弃全部财产以摆脱交租;而无产者干脆直接逃跑,政府的回应则是增加压力。”当你谋划未来时,有必要记住这一点。从过去的历史看,一种国家制度的暮光时分,很少是礼貌的、有序的。在第二章中,我们提到了罗马税吏的无耻行径。罗马帝国崩溃后,西欧出现了大量被抛荒的农田,这只是更大问题的一小部分。实际上,在高卢以及包括今天的卢森堡和德国在内的边境地区,税收是相对温和的。

而在罗马土地最肥沃的地区——埃及,因为有灌溉设施,那里的农业产量更要。

罗马崩溃以后,埃及的土地遗弃现象更加严重。是不是要逃跑,也就是拉丁语里所说的,去寻找“最后的避难所”,几乎成为了所有有产者的首要难题。历史记录显示,“在埃及,人们经常向神谕提出的问题,有三种标准类型:‘我会成为乞丐么?’‘我应该逃跑么?’‘我逃跑的话会被抓住么?’”克林顿的提案说“会的”。这个提案还只是设置逃离障碍的初期版本,随着民族国家财政资源的流失,障碍会越来越高。当然,比起埃里希·昂纳克(前东德领导人)的混凝土和铁丝网,美国的版本要温和多了。它对价格也很敏感,只涉及应税财产超过 60 万美元的“亿万富翁们”。尽管如此,它被提出的理由,与昂纳克为前德意志民主共和国(即东德)最著名的公共工程(即柏林墙)辩护时,提出的论点非常相似。昂纳克声称,东德在那些想要逃难的人身上进行了大量的投资,让他们自由离开会影响经济,国家需要他们的努力。

如果你接受人民是或者应该是国家的资产,那么昂纳克的墙就是站得住脚的。对共产党来说,没有墙的柏林是一个漏洞;对克林顿的国税局来说,逃离美国的税收管辖也一样。克林顿关于亿万富翁离境的论点,与昂纳克的差不多,除了显示出政客一贯无视数字的准确性以外,在某种程度上也不太符合逻辑。毕竟,美国政府并没有对那些想要离开的富裕公民进行大量的投资。他们并不是接受了国家资助的教育,然后想溜到别的国家去当律师赚大钱。绝大多数可能被征收离境税的人,都是靠自己的努力创造了财富;尽管有美国政府,但不是因为美国政府。

在美国,前 1\%的纳税人缴纳了总所得税的 30.2\%(1995 年),所以,这里的问题,并不是美国富人没有回报政府对其教育或经济繁荣所做的真正投资。恰恰相反,那些支付了大部分账单的人,所付出的费用远超所得利益的全部价值。

美国前 1\%的纳税人,每人年均纳税额超过 12.5 万美元,这个税收成本的影响,他们并没有认识到。如果在 40 年的时间周期内,他们能从支付的超额税款中获得哪怕 10\%的回报,那么每多交 5000 美元的税,就会使他们的净资产减少 220万美元;如果是 20\%的回报,5000 美元的税就减少 4400 万美元的净资产。

随着千禧年的到来,在信息时代的大政治条件下,人们将越来越清楚地看到,从工业时代继承下来的民族国家是一个掠夺性机构。年复一年,它已不再是繁荣的助推器,而更像是一个牢笼,一个个人想要逃离的牢笼。但是,已经陷入绝望的政府决不会允许这样的逃离。西方福利国家的稳定乃至生存,都取决于他们是否能够继续抽取世界总产出的大头,重新分配给经合组织国家(发达国家)的一部分选民。这就要求必须以超级垄断的价格,对当前富裕国家中最具生产力的公民进行征税;相比政府所提供服务的实际成本,该价格要高出数百甚至数千倍。

\section{民族国家的生与死}
柏林墙的倒塌不仅是共产主义灭亡的明显标志,它也标志全世界民族国家体系的一次失败,以及效率和市场的一场胜利。历史背后的权力支点已经发生了转移。

民族国家从法国大革命开始,已经持续了两百年,这是一个奇怪的历史阶段;我们认为,1989 年柏林墙的倒塌就是民族国家时代的顶点。国家存在了六千年之久,但是在 19 世纪之前,它们只是世界主权形式中的一小部分。它们的崛起始于革命,也终于革命。1789 年的大事,使欧洲走上了真正的民族国家之路;1989年的大事,宣告了共产主义的死亡和市场对大众力量的控制。这两场革命正好相隔 200 年,定义了由民族国家主导大国体系的时代;大国反过来又主导了整个世界,它们的政治体系甚至传播和强加到了最偏远的部落飞地。

国家作为组织暴力的主要工具,它的胜利并不是一个意识形态问题,而是冷酷的暴力逻辑。就像我们常说的,这是一个大政治事件,它跟理论家和政治家们的期望关系不大,甚至也不是因为将军们的操纵;它是由隐藏的暴力杠杆所决定的,它以阿基米德梦想的方式推动了历史的发展。

在过去两百年的现代时期,国家是一种常态;但放到更长远的历史看,国家是罕见的主权形式。它们的生存能力一直依赖于非常特殊的大政治条件。在现代时期以前,大多数国家都是“东方专制主义”,沙漠中的农业社会,依赖对灌溉系统的控制。即使是罗马帝国,通过对埃及和北非的控制,也间接成为了一个水利社会。但这不足以使其生存下去。像很多前现代国家一样,罗马最终因缺乏迫使人们臣服于垄断暴力的能力而崩溃,而这种能力是可以饿死人的。在非洲地区以外的罗马政府,没有能力阻止叛逆的群众使用灌溉系统,没办法切断水源供应。在古代经济中,相比其他所有的大政治配套设施,水利系统提供了更多的暴力杠杆。

在当时的社会中,谁如果控制了水利,那它从社会中提取战利品的水平,就可以媲美于现代民族国家抽走社会总产出的比例。

\subsection{规模胜于效率}
火药的出现,使国家更容易在稻田与干旱河谷的范围外扩张。在战争中,火药武器的性质和工业经济的特点,创造了巨大的规模优势。这导致暴力的回报率高企且不断上升。历史学家查尔斯·蒂利(Charles Tilly)说,“拥有最严厉强制手段的国家往往能赢得战争,效率(投入产出比)会败给效益(总产出)。”由于多数政府都是大规模组织起来的,少数幸存的小国,如摩纳哥和安道尔,需要大国的承认才能保持独立。也只有掌握了更多资源的大国政府,才能够在战场上展开较量。

\subsection{未解答的大问题}
这就带给我们一个现代史上重大的未解之谜:为什么在大国体系行将结束时爆发的冷战中,最后的竞争对手是共产主义独裁政权和民主福利国家?这个问题很少被研究,以至于在柏林墙倒塌后,美国国务院的分析员弗朗西斯·福山(FrancisFukuyama)宣称“历史的终结”时,很多人都觉得有道理。然而,他的热心读者太想当然了。很显然,福山和其他人都没有费心去问一个基本问题:国家社会主义和民主福利国家有什么共同点,使它们成为了世界统治的最后竞争者?这个问题至关重要。毕竟,在过去的五个世纪里,有几十种制度在争夺主权,你方唱罢我登场,包括绝对君主制、部落飞地、红衣主教制、教皇直接统治、苏丹王国制、城邦国家和再洗礼派殖民地。今天的人可能会觉得很惊讶,一个拥有自己武装的医院管理公司,可以统治一个国家长达几个世纪;但与此类似的情形就在历史上发生过。在 1228 年后的三百年间,耶路撒冷圣玛丽医院的条顿骑士团,后来与利沃尼亚之剑骑士团联合,统治了东普鲁士及东欧的多国领土,包括立陶宛和波兰的部分地区。然后爆发了火药革命。这导致在短短的几十年间,条顿骑士团就从它所有的领土上被驱逐了出去;他们大团长的军事重要性,也落得跟一个国际象棋冠军差不多。为什么会这样?为什么在工业时代结束时,在争夺世界霸权的伟大斗争中,那么多的主权形式都变得无足轻重,只有大众民主制度与国家社会主义体系并驾齐驱,相与争锋?

\subsection{为所欲为的控制}
我们的大政治理论给出了答案。这就好比问为什么相扑运动员都体型庞大。答案很简单,一个瘦小的相扑手,无论他的力量体重比多么惊人,都不可能打赢一个重量级的对手。正如蒂利所言,重要的是“效用”(总产出),而不是“效率”(投入产出比)。在一个暴力日盛的世界里,经过五个世纪的竞争而占据主导地位的制度,必然是有利于最大限度获取大规模战争所需资源的制度。

这是怎么实现的?就共产主义而言,答案显而易见。在共产主义制度下,控制国家的人基本控制了一切。如果你是冷战时期的苏联公民,当克格勃认为有必要时,他们可以拿走你的牙刷,还可以拔掉你的牙。1992 年前苏联的档案开放以后,根据更可信的记录估计,在 74 年的统治中,苏联的秘密警察及特工夺走了 5000 万人的生命。国家社会主义可以调动国内的一切用于军事目的,该制度下的人民不可能提出异议。

对西方民主国家来说,答案没有那么明显。部分原因在于,西方人习惯认为,民主制度与共产主义是形成鲜明对比的。从工业主义的角度看,这两种制度确实存在很大的对立面。但从信息时代的角度看,二者的共同之处比你想象得要多。它们都有利于政府不受限制地控制资源。不同的是,与国家社会主义制度相比,民主福利国家掌控了更加丰富的资源。

少即是多,是一种比较罕见的现象,这里就是一个明显的例子。国家社会主义的理论前提是国家拥有一切。相比之下,民主福利国家的制度要求则比较温和,并采用了更优越的激烈措施动员人民,从而获得更大的产出。西方国家的政府,不是一开始就拿走一切,而是允许个人先拥有财产、积累财富。当财富积累到一定程度之后,政府就通过税收掠走一大部分。与国家社会主义制度相比,高额的财产税、所得税和遗产税,为民主福利国家提供了大量的资源。

\subsection{效率低下也有优势}
与共产主义相比,福利国家确实效率更高。但与其他积累财富的制度相比,例如殖民地时期的香港——一块真正自由放任的飞地,福利国家的效率是很低的。“少即是多”又一次得到了体现。在工业时代的大政治条件下,恰恰是这种低效率使福利国家成为了制度的优胜者。

当你理解了这其中的原因,你对柏林墙倒塌和共产主义死亡的真正意义,就会有更进一步的认识。它远不是人们所普遍认为的,民主福利国家将确保取得最终的胜利;它更像是看到双胞胎兄弟中的一个率先老去。

杀死共产主义的这场大政治革命,也将破坏和摧毁我们在 20 世纪所熟知的民主福利国家。

\section{政府控制在谁的手里?}
要理解我们这个非正统的结论,关键就在于认识到民主政府控制在谁的手里。这个问题并不像看起来那么简单。在现代社会,谁控制政府总是被当作一个政治问题。这个问题有很多答案,但几乎都一致认为,是在特定时刻控制特定国家的政党、团体或派系。你听说过政府控制在资本家手里,控制在劳工手里,在天主教徒手里,在伊斯兰原教旨主义者手里,在部落和种族团体手里,在胡图人、在白人手里。你也听过说政府控制在职业团体手里,如律师和银行家;或者控制在农村利益团体手里,在大城市机器工厂主手里,在郊区居民的手里。当然你也听说过政府控制在政党手里,如民主党、保守党、基督教民主党、自由党、激进党、共和党和社会主义政党。

但你可能没听说过政府控制在顾客手里。我们前面讨论过,经济史学家弗雷德里克·莱恩撰写了一系列条理清晰的文章,论述了暴力的经济后果,同时也为理解政府控制权归属的新思路奠定了基础。莱恩将政府视为一种出售保护服务的经济单位,从经济而非政治的角度,去分析政府的控制权。依据该观点,对政府的控制有三种基本的选择:业主,雇员和客户;而每一种选择都对应一套完全不同的激励机制。

\subsection{业主}
政府控制在一个业主手里,这种情况比较少见,但今天也有。他通常是一个世袭的领导人,几乎拥有国家的一切。例如,文莱苏丹对待文莱政府的态度就像是一个业主。这种情况在中世纪的领主中很常见,他把自己的领地当作一家独资企业,通过各种方式去优化收入。

对于“生产性企业的所有者”的激励机制,莱恩做了如下描述:为了追求利润最大化,业主会在维持价格不变的同时,尽量降低他的生产成本。就像英国的亨利七世或法国的路易十一,他们会使用尽可能便宜的手段,维护自己的合法地位,维持国内的秩序,并分散邻国王公们的注意力,以便降低军事开支。从降低的成本中,从他牢固的垄断地位增加的税收中,或者从二者的结合中,他积累了财富盈余。 由业主控制的政府,有强烈的动机去降低在某地提供保护或垄断暴力的成本。但是,只要他们的统治是安全的,他们就不想降低向客户收取的价格(即税收),降到低于优化收入的比率。垄断者收取的价格越高,实际成本越低,他的收益就越大。对一个由业主控制的政府来说,理想的财政政策是有巨额盈余。当政府能够保持高收入且削减成本时,对资源的利用有很大的好处。劳动力或其他有价值的资源——原本可能浪费在不必要的昂贵保护上,就可以转而用于投资或其他目的。君主通过降低成本获取的利润越高,就有越多的资源被释放。用这些资源去投资,就会刺激增长;即使是用于炫耀性的消费,也可以创造和哺育出新的市场。

但如果资源被浪费在低效的“保护”上,这些市场就不会存在。

\subsection{雇员}
雇员控制的政府,它的激励机制很容易描述,在其他由雇员控制的组织中也很常见。首要的一点是,雇员控制的机构倾向于赞成任何能增加就业的政策,并反对减少就业的措施。正如莱恩所说,“当雇员作为一个整体进行控制时,他们对最大限度地减少保护费没有兴趣,对最大限度地减少以劳动力成本和自己工资为代表的那部分成本也没有兴趣。规模的最大化更符合他们的口味。”一个由雇员控制的政府,很少有动力去降低政府的成本或向客户收取的价格。而当客观条件造成很强的价格阻力,即反对提高税收时,雇员控制的政府更愿意入不敷出,而不是削减他们的开支。换句话说,他们的激励模式意味着他们更倾向于长期的财政赤字,而由业主控制的政府则不会。

\subsection{客户}
有没有由客户控制的政府?有的。莱恩就是受到了中世纪商人共和国的启发——例如威尼斯,开始从经济角度来分析政府的控制权。在商人共和国,一群需要保护的贸易批发商在几个世纪内有效地控制了政府。他们实际上是政府保护服务的客户,而不是业主。他们为这种服务付费,而不寻求从政府对暴力的垄断中获益。

如果有人想这么做,也会被其他长期客户阻止。其他由客户控制的政府,包括被有限特许的民主国家或共和国;如古代的民主国家,或建国初期的美利坚共和国。

当时,只有那些为政府付费的人才被允许投票,约占美国人口的 10\%。

由客户控制的政府,会像业主的政府一样,有动力尽可能降低运营成本。但与业主和雇员控制的政府不同,他们还有动力去压低政府收取的价格。在客户统治的地方,政府精简干练,一般不引人注目,运营成本极低,有最少的雇员与温和的税收。由客户控制的政府,制定的税率不是为了优化政府能收多少钱,而是为了优化客户能留住多少钱。与竞争性市场中的典型企业一样,就算客户控制的政府是垄断性的,它也不得不向效率的方向发展。它以税收形式收取的服务成本,只能维持微薄的利润,而不可能太高。

\section{民主的角色:作为雇员和客户的选民}
莱恩以传统的方式看待民主,认为它使制造和利用暴力的企业“越来越受客户的控制”。这是一种政治正确的结论。但它是真的吗?我们认为不是。我们来仔细观察一下现代民主国家是怎么运作的。

首要的一点是,它们基本没有竞争性行业的特征;在竞争性行业里,贸易条件显然是由客户支配的。这从一点可以看出,民主政府通常只将总支出的一小部分用于保护服务,而这原本应该是它的核心功能。例如美国,州和地方政府用于警察、法院和监狱的开支只占总支出的 3.5\%;加上军事开支,用于保护的费用占收入的比例也只有 10\%左右。另一个可以揭示大众民主不受客户控制的现象是,从工业时代继承下来的当代政治文化认为,如果关键问题的决策由实际支付账单者的利益来决定,那就太离谱了。想象一下,如果一位美国总统或英国首相提议,支付了大部分税款的公民,可以决定哪些政府项目可以继续,哪些雇员群体可以被重要;那将引起多大的骚动。这会深深地改变人们对政府运作方式的期望,而由政府雇员决定应该提高谁的税收就不会出问题。

然而仔细想想,如果客户真的占据了主导地位,却得不到他们想要的东西,那才是离谱的。就像你去一家商店买家具,销售员收了你的钱,却对你的要求置之不理,而是转身跟别人商量怎么花掉你的钱,你肯定会感到不满。如果商店员工认为你不配拥有这些家具,应该运给他们认为更有资格的人家里,你当然会认为这不正常,也不合理。事实上,在与政府打交道的过程中,类似这样的情况有很多,这只说明政府的“客户”根本没什么控制权。

无论从哪个角度看,民主政府的成本都已经失控了,这明显不符合正常的规律,即客户偏好会迫使卖家提高效率。大多数民主政体都有长期赤字。这是由雇员控制的政府的财政特点,政府对降低运营成本有明显的抵触。人们对当代政府有一个普遍抱怨,就是政治项目一旦确立,要削减它就面临极大的困难。减少政府雇员几乎是不可能的。以前国有部门的私有化带来了一大好处,因为私人控制可以更容易地淘汰掉冗余的就业。从英国到阿根廷,新的私人管理者将以前的国家雇员裁掉 50-95\%的情况都不少见。

再想想政府提供保护服务的财政条款,它的定价基础是什么?在大多数情况下,政府服务定价依据的是税率,但如果你想寻找哪些因素对税率产生了竞争性的影响,那将注定徒劳无功。即使在最近几年,正常的政治辩论偶尔会被降税的话题给打断,但这也只是暴露出民主政府距离被客户控制有多么遥远。有些主张降低税收的人会指出,降税实际上可以使政府的收入增加,因为以前的税率太高,阻碍了经济发展。

通常来说,这些人权衡的不是国家之间的竞争,而是一些更让人惊讶的想法。他们并没有争论说,香港的税率只有 15\%,所以德国或美国的税率一定不能高于15\%。恰恰相反。这些税收辩论通常会假定,纳税人面临的选择不是在这个还是那个司法管辖区做生意,而是以惩罚性税率做生意还是干脆关门放假。你肯定听到有人说过,如果不减轻税负,那些遭受掠夺性税收的有产者就会停掉他们的生意,去打高尔夫球。

连这种论点都可以出现,说明民主福利国家所强加的保护成本,根本没有竞争的基础。自从 20 世纪以来,每个民主福利国家都出现了累进所得税制,与客户偏好的定价方式截然不同。这其间的差别,拿这种为了支持和保护垄断而征收的税率,与电话服务的费率做一下比较,就可以很容易地看出来。直到不久前,电话服务在大部分地方还是垄断的。但是,如果电话公司以所得税的方式收电话费,用户肯定会大喊大叫,不答应。比如你打了一通电话到伦敦,结果电话公司寄来了一张 5 万美元的账单,因为你碰巧在电话里达成了一笔 12.5 万美元的交易。

相信你或者任何脑子正常的人都不会付这笔钱。但这正是每个民主福利国家评估所得税的基础。

当你仔细思考民主工业国家的运作条件时,你会发现,把它们视为由雇员控制的政府更加符合逻辑。把大众民主看作是由雇员控制的政府,可以解释为什么改变政府的政策那么困难。因为在很多方面,政府是服务于雇员的利益。例如,在大多数民主国家,公立学校一直都很糟糕,而且没有补救措施。如果政府真的由客户主导,他们很容易就可以制定出新的政策方向。那些为民主政府付费的人,很少能够设定政府支出的条件。相反,政府作为一个合作社,既不在业主的控制范围内,又以一种天生的垄断形式在运行。它收取的价格与成本完全脱节;相比于私营企业,它的服务质量又普遍很低;顾客有什么不满,也很难得到处理和补救。

简而言之,大众民主导致政府被其“雇员”所控制。

但是,等一等。你可能会说,在大多数国家,选民都比政府工资单上的人要多得多。在这种情况下,雇员怎么可能占据主导地位呢?福利国家的出现,正是为了解决这个难题。因为不可能有足够的政府雇员去形成一个有效的多数,所以,越来越多的选民实际上是被放到了工资单上,可以接受政府的各种转移支付。接受转移支付和补贴的人,就成为了政府的假雇员,只是省去了每天上班通勤的麻烦。

这是由工业时代的大政治逻辑决定的。

在 1989 年之前,强制力的大小比资源配置的效率更加重要,在这种情况,大部分政府都不可能被他们的客户所控制。直到几年之前,即便浪费严重,国家还可以在世界范围内行使巨大的权力,前苏联就是一个很好的例子。当暴力的回报率很高且不断上升时,规模比效率更重要;较大的实体比小的实体更有优势。那些可以更有效地调动军事资源的政府,即使以大量的浪费为代价,也往往会战胜那些更高效利用资源的政府。

这意味着什么?它必然意味着,当规模优于效率时,由客户控制的政府就没有获胜的机会,而且往往无法生存。在这种情况下,能够征用最多战争资源的实体,就是军事效果最好的实体。而真正由支付账单的客户控制的政府,不太可能获得全权委托,无法把手伸到每个人的口袋里提取资源。

客户都希望他们为任何产品或服务(包括保护服务)支付的价格能降低,并且在控制之中。冷战期间,如果西方民主国家是由客户主导的,那仅凭这一点,它们就会成为更弱的军事对手,因为它肯定会限制向政府输送资源的规模。请记住,在客户主导的地方,价格和成本都会受到严格的控制。但历史并非如此;福利国家显然是冷战支出竞赛的赢家。在各种评论家看来,西方让苏联花费到破产的超支能力,正是它们获胜的原因之一。

而恰恰是这一点,凸显了在暴力回报上升的时期,民主的低效率使它在大政治上占据了主导地位。大规模的军事开支,及其所有的浪费,明显是一种为私人利益服务的次优资本配置。我们在前面提到过,与国家社会主义制度相比,福利国家在经济上是高效的;但比起那些自由放任的飞地,它们创造财富的效率要低得多,如香港。具有讽刺意味的是,相比束缚更少的自由市场体系,民主福利国家正是以它的低效率,在工业化的大政治环境中获得了成功。

民主制度所孕育的低效率,为什么会成为它在暴力时代获胜的要素?要解开这个明显的悖论,关键在于认识到以下两点:

1. 在现代时期,一个主权国家的成功,并不在于创造财富,而是能够建立一支军事力量,使它有能力对任何国家使用压倒性的暴力。要做到这一点需要金钱,但金钱本身并不能赢得一场战争。竞争的关键,并不是建立一个经济效率最高或增长速度最快的系统,而是建立一个能够提取更多资源并将其用于军事的系统。就其性质而言,军事开支的财政回报本身很低,甚至不存在。

2. 要想使人们同意把资金投入到很少或者几乎没有直接经济回报的活动中去,比如缴税,最简单的方法是向出钱者以外的人请求许可。荷兰人用价值 23美元的珠子购买了曼哈顿,因为和他们做交易的那些印第安人,并不真正地拥有它。就像营销人员所说的,在这种情况下,“达成协议”要容易得多。

比如说,作为这本书的作者,我们希望你不是按定价购买,而是用你年收入的 40\%买一本。如果我们不问你,而是问其他人,那我们更有可能得到许可,达成这笔交易。而且,如果我们能获得几个你可能都不认识的人的同意,会更有说服力。我们甚至可以举行一次专门的选举,就像门肯(H.L.Mencken)所描述的,也许没他想的那么夸张,就是“一场赃物的高级拍卖”。为了使这个例子显得更真实,我们还同意把从你这里收到的钱,分一部分给匿名的旁观者,以换取他们的支持。

这就是现代民主福利国家演化出来的功用。在工业时代,它是一个无与伦比的系统;因为在关键的地方,它既有效率又无效率。它将私有制的效率和创造财富的激励机制,与一个基本不受限制获取这些财富的体系结合到了一起。

民主使财富生产者的口袋一直敞开着。在全世界暴力的回报率上升到巅峰的时期,它在军事上取得了成功,因为它使客户无法有效地限制政府的税收,以及其他为军事开支汲取资金的方式,如通货膨胀。

\subsection{为什么客户无法成为主宰}
在现代社会,那些为“保护服务”付费的人,即使集体行动,也无法抵抗向主权国家提供资源。因为不提供的话,他们可能会受到敌意更强国家的压迫。

冷战期间,很多人明显考虑到了这一点。在西方先进的工业国家中,他们的客户,或者说纳税人,已经不成比例地承担了政府的成本,但他们没有能力拒绝高额的税收。如若不然的话,他们的财富就可能被苏联,或者另外一个有能力组织暴力侵略的集团,完全没收。

\subsection{工业主义和民主}
长远来看,大众民主是一种不合时宜的东西,当工业时代结束以后,它不会长期存在。大众民主和民族国家,是在 18 世纪末随着法国大革命一起诞生的,是对人们实际收入激增的一种回应。1750 年左右,西欧人的收入开始大幅上升,部分原因是天气变暖了。这时也正值技术革新时期:非熟练工,甚至妇女和儿童都能操作的机器设备,取代了工匠的熟练工作。这些新的工业设备提高了非技术工人的收入,使收入分配更加平等。

革命的关键触发点,并不像很多人通常持有的那种反常想法,认为人们会在条件改善后进行反抗。更大的可能反而是,当收入上升到一定水平,早期的现代国家,终于可以绕过以前必须讨价还价的私人中介和有权有势的大人物,转而采用“直接统治”的制度:由政府直接与公民个体打交道,向他们征收越来越高的税负,并要求他们服兵役以换取各种福利,对兵役的补偿则少得可怜。

新兴的中产阶级很快就有足够的钱可以被征税了,统治者不用再像以前那样,必须与强大的地主或富商进行谈判。正如历史学家查尔斯·蒂利所说,这些人“有能力防止一个强大政权的成立”,因为它会“夺取他们的资源并限制他们的交易”。

不难看出,当政府与数百万公民单独打交道,而不是对付数量相对较少的领主、公爵、伯爵、主教、雇佣兵、自由城邦和其他半主权实体时,它在攫取资源方面更加成功。在 18 世纪中叶之前,欧洲的统治者就不得不与上述实体进行谈判。

实际收入的增加使政府可以采取一种新策略,将更多的资源置于它们的控制之下。从数百万人身上征收小额税款,比少数权贵支付的大额税收,数量更大,政府的收入要更高。更重要的是,很多人比少数人更容易对对;因为这些少数人通常不愿意把钱交出去,而且更有抵抗能力。

毕竟,与国家本身相比,普通的农民、小商人或工人拥有的资源少得可怜。在法国大革命前夕,一个典型的西欧人,根本不可能与国家进行什么讨价还价,不管是降低他的税率,还是反对威胁他利益的政府政策。但这恰恰是过去几个世纪以来,那些有权有势的私人大亨们一直在做的,并且会继续这么做。他们能有效地抵抗统治者,与其讨价还价,限制其征用资源的能力。

\begin{tcolorbox}
打仗加速了从间接统治到直接统治的进程。几乎所有发动战争的国家都会发现,它们积累的储备和当前的收入不足以支付战争的费用,它们需要大量的借贷,增加税收,并从不情愿的公民手中争夺战略资源——包括男人;这些资源本来是公民另作他用的。
\begin{flushright}
—— 查尔斯·蒂利 
\end{flushright}
\end{tcolorbox}

18 世纪中期的波兰是一个完美的例子,很好地说明了这一点。1760 年,波兰的国家军队只有 18000 名士兵。与邻国奥地利、普鲁士和俄罗斯统治者麾下的军队相比,简直微不足道;他们中最小的一支常备军都有 10 万人。实际上,与当时波兰境内的其他武装力量相比,1760 年的波兰国民军也是很小的。波兰贵族的总兵力都有 3 万人。

如果波兰国王能与数百万波兰人直接打交道,直接向他们征税,而不是受制于权贵巨头们的捐款,只能间接地榨取资源;那么毫无疑问,波兰王室就能筹集到更多的钱,就养得起更庞大的军队。

在任何地方,要对付那些虽有数百万之众但无法协同行动的普通人,中央当局的力量都是压倒性的。但是在 1760 年,波兰国王缺乏直接向其公民征税的选项。

他只能与领主、富商及其他贵族进行交易,而这些人是一个小型的、有凝聚力的团体。他们可以而且确实采取了一致行动,防止国王不经同意就征收他们的资源。

鉴于波兰贵族的兵力远胜于己,波兰国王无法坚持。

历史证明,如果不能解决富人和强人对征集资源的阻碍,在暴力时代是一个决定性的军事劣势。短短几年之内,波兰作为一个独立的国家就不复存在了。它被来自奥地利、普鲁士与俄国的入侵所征服,这三个国家的军队都比波兰的大很多倍。

它们的君主都找到规避富商和贵族限制其征用资源的途径。

\subsection{法国大革命之后}
法国大革命导致军队规模进一步激增,这证明了在暴力回报率上升时期民主战略的作用。从法国大革命开始,政府直接与普通人交易,它给予人们前所未有的社会生活参与度;作为交换,人们代替雇佣兵参加战争,并从他们不断增加的收入中支付越来越重的税收。

正如蒂利所言:“国家的范畴远远超出了它的军事核心功能,它的公民开始从保护、裁决、生产和分配等各个层面,向它提出要求。随着国家立法机关将自己的权力扩大到批准税收之外,它们就成为了所有组织严密的团体索求的对象,这些团体的利益受到了或可能受到国家的影响。直接统治与大众政治一起成长,相辅相成,互相强化。”18 世纪被证明为真的逻辑,在 1989 年柏林墙倒塌时依然为真。随着工业时代的发展,非熟练工的收入持续上升,使得大众民主成为优化资源提取的更有效途径。

因此,政府也不断壮大,蒸蒸日上,在 20 世纪,工业国家抽取其年收入的比例平均每年增加了 0.5%。

在 1989 年之前的工业时代,民主是最具军事成效的政府形式;因为有了民主,要想限制国家征用资源,就变得非常困难或者不可能。向所有人提供慷慨的福利,使大多数选民成为事实上的政府雇员;这成为了主要工业国家的主要政治特征。

选民作为保护服务的客户,在对政府的控制方面处于弱势地位。他们不仅面临着咄咄逼人的共产主义——该制度下的国家控制了整个经济,可以生产大量的资源用于军事目的。而还有另外一个原因,使得纳税人想真正控制政府也不现实。

数以百万计的普通公民,无法达成有效的合作去保护他们的利益。因为他们合作的门槛很高,而且对其中的任何个体而言,成功捍卫群体共同利益的回报是微不足道的。所以,在抵抗政府扣缴资产时,他们不会像那些激励机制更有利的小团体一样成功。

因此,在其他条件相同的情况下,你可以预期,大众民主制度下的政府,它征用的资源占社会总资源的比例要高于寡头政治;也高于一个分散的主权体系,在这种体系中很多大人物都握有军事力量,拥有自己的军队,而在 18 世纪之前,现代早期的欧洲各地都是如此。

所以,西方世界的民主制度得以兴盛,一个很关键但又很少被研究的因素就是,在暴力回报上升的时候,谈判成本的相对重要性。从少数人那里获取资源总是比从多数人那里获取的成本要更高。

相比一大群普通公民,一个人数较少的精英富贵群体,往往可以更加团结一致并有效合作。这个小团体有更强的激励机制来共同进退,它必然会比大众群体能更好地保护自身的利益。而且,即使小团体中有多数成员选择不参与共同行动,少数富有的人也可能部署足够的资源,把事情搞定。

相较于对付一小撮很容易形成组织去捍卫共同利益的权贵,有了民主决策,民族国家可以更彻底地对数以百万计的人行使权力,因为这些人很难因自身利益达成合作并集体行动。

民主制还有一个压倒性的优势:它创造了一套合法的决策机制,使国家可以利用富人的资源,而无需为了获得同意直接与他们讨价还价。简而言之,民主作为一种决策机制,很适合工业时代的大政治环境。它与民族国家相辅相成,因为在这个时代,军事的规模比调动军力的效率更加重要;而民主制有助于将军事力量集中到国家统治者的手里。

法国大革命明显证明了这一点,它提高了战场上的军事规模。此后,其他参与竞争的民族国家几乎别无选择,只能向类似的组织形式靠拢,而其合法性最终与民主决策联系到了一起。

总之,在过去的两个世纪里,民主民族国家的成功有以下几个隐蔽的原因:

\begin{itemize}
    \item 暴力的回报不断上升,使军事的规模比效率更重要,这也催生了相应的统治法则。
    \item 人们的收入上升,不再限于维持生计的程度,使国家可以征集到大量的资源,而不用再和有反抗能力的巨头进行谈判。
    \item 事实证明,民主制度与自由市场可以良好兼容,有利于创造越来越多的财富。
    \item 民主制度加强了“雇员”对政府的支配,从而使削减政府开支变得更加困难,包括军事开支。
    \item 作为一种决策机制,民主是一种有效的解毒剂;它可以防止富人协同行动,阻碍民族国家征税或以其他方式保护自身财产不受侵犯。
\end{itemize}

在军事上,民主制也成为了制胜之道,因为它有利于将更多资源聚集到国家手里。

其他形式的主权,它们的合法性依赖于不同的规则,如封建征收、国王的神权、公司的宗教责任或富人的自愿捐赠。相比之下,大众民主在军事上是最强有力的,因为它是工业经济中征收资源最可靠的方式。

\begin{tcolorbox}
民族,作为一种文化共同体,是现代性价值的最高象征。它被赋予了一种准神圣性,只有宗教可以与之媲美。事实上,这种准神圣的特征就来自于宗教。在实践当中,民族不是成为了现代的、世俗的宗教替代品,就是成为了宗教最强大的盟友。在现代时期,民族所激发的共同体情感受到高度重视和追捧,成为了群体忠诚的基础。……这一切的受益者往往是现代国家;鉴于其至高无上的权力,并不令人惊讶。
\begin{flushright}
—— 何塞普·R·洛贝拉(Josep R. Llobera) 
\end{flushright}
\end{tcolorbox}

\subsection{民族主义}
民族主义也是如此,它是大众民主的一个必然结果。能够利用民族主义的国家发现,用较小的代价就可以动员起更庞大的军队。民主主义是一项发明,使一个国家可以有效地扩大它的军事规模。与政治本身一样,民族主义主要是一种现代发明。社会学家何塞普·洛贝拉在他关于民族主义兴起的书中,通过详实的记录指出,民族是一个想象中的共同体,在很大程度上,它是在法国大革命期间作为动员国家力量的一种手段出现的。他说,“在现代意义上,民族意识只存在于法国大革命之后,是从 1789 年制宪会议把法国人民等同于法兰西民族开始的。”民族主义使调动权力和控制大量人口变得更加容易。民族国家的形成,是通过突出和强调人们的共同特征而实现的,特别是所讲的语言。它简化了官僚机构的任务;有利于摆脱中间人的干预进行统治。与那些必须翻译成各种语言的法律相比,只使用一种语言的法律,可以更快地颁发,在执行中也更少出现混乱。所以,民族主义可以降低控制较大面积领土的成本。在民族主义出现之前,早期的现代国家需要通过领主、伯爵、公爵、主教、自由城邦以及其他企业和种族的中介,帮助解决农民税收、军火商和雇佣兵等问题,从而获得财政收入,以建设军队并履行其他政府职能。

通过鼓励群体对国家利益的认同,民族主义还显著降低了军事动员的成本。在利用群体情感为国家利益服务方面,民族主义的优势无与伦比,以致于大多数国家,甚至是所谓国际共产主义的苏联,都把民族主义作为一种辅助性的意识形态。

从更长远的角度看,民族主义和国家一样,本身是一种反常现象。正如历史学家威廉·麦克尼尔(William McNeill)所记录的那样,多民族主权在过去是一种常态。麦克尼尔说,“一个政府理应只统治一个民族的公民,这种想法到中世纪末才在西欧发展起来。”民族主义的一个早期实体是普鲁士联盟(Prussian League),成立于 1440 年,是为了反对条顿骑士团的统治。前面我们强调过骑士团的一些特点,作为一种主权形式,它与民族国家几乎是在两个极端。条顿骑士团是一种特许社团,它的成员几乎没有一个是普鲁士本地人。在不同时期,它的总部从不莱梅和吕贝克转移到耶路撒冷、阿克里和威尼斯,再转移到维斯瓦河畔的马里恩贝格。它曾一度统治着特兰西瓦尼亚(今罗马尼亚中部)的布尔岑兰地区。一种与国家如此不同的主权形式,成为早期尝试动员民族情感以构建权力者的攻击对象,这并不奇怪。但是,早期的民族主义与它后来的变体大不相同,一个标志性的事件就是,普鲁士联盟的德语贵族向波兰国王请愿,请求将普鲁士置于波兰的统治之下。这主要是因为,那个时候的波兰国王是一位相对软弱的君主,不太可能像条顿骑士团那样严酷地统治。

民族主义的早期化身,在火药革命之前就已经出现了。它随着早期现代国家的发展而发展,到法国大革命时期,其重要性有了质的飞跃。我们认为,民族主义作为一种理念,它的力量已经开始消退了。它的全盛阶段可能是在第一次世界大战结束后,当时美国总统伍德罗·威尔逊想让欧洲每个民族都拥有自己的国家。

我们预计,当西方民主福利国家崩溃时,民族主义将成为低技能人士举行集会、怀念强权的一大主题;稍后我们将对此进行探讨。现在你们什么都还没有看到。

对于大部分西方人来说,共产主义的死亡带来的影响是相对温和的。你们看到了军费开支的下降,铝价的暴跌,以及 NFL 冰球运动员来自新的国家\footnote{译注:指前苏联国家。}。这是好消息;这是令大多数在 20 世纪长大的人都欢欣鼓舞的好消息,特别是冰球爱好者。而注定不受欢迎的消息还没有到来。

随着工业时代的过去,满足民主所需的大政治条件很快将不复存在。在信息时代的新型大政治条件下,大众民主和和福利国家能否长期生存,值得怀疑。

\begin{tcolorbox}
国会不是民主的殿堂,它是一个交易法律的市场。
\begin{flushright}
—— 秘鲁总统阿尔贝托·滕森(ALBERTO FUJIMORI)
\end{flushright}
\end{tcolorbox}

其实,未来的历史学家可能会记述到,我们这一代人已经看到了第一场后现代的政变——1993 年在秘鲁发生的引人注目的国会封锁事件。这件事在主要的工业民主国家几乎没引起什么关注,也没得到积极的评价。但放到时间的长河中,其意义可能远超传统分析家的认知。少数对此有所思考的人,认为它不过是那种人们已经熟悉得不能再熟悉的拉美式夺权。但在我们看来,这或许是使一种统治方式失去合法性的第一步,因为随着向信息时代的过渡,这种统治方式直接赖以生存的大政治条件正在消失。滕森关闭国会是政治承诺贬值的终极象征。当其他立法机构的信用被耗尽,等待它们的是同样的命运。

技术的变革,正在侵蚀工业主义,并且使很多国家的政府失灵,或者运转不力。

尤其是立法机构,似乎越来越不能发挥作用。它们制定的法律,在 50 年前可能还只是愚蠢,在今天却很危险。这一点在秘鲁十分明显,1993 年,秘鲁的国家内部主权已濒临崩溃。

\begin{tcolorbox}
袭击、绑架、强奸和谋杀,与攻击性的驾驶习惯和街头暴力相同步,与日俱增。警察逐渐失去了对局势的控制,一些警员还卷入了丑闻,沦为老练的罪犯……人们慢慢习惯了没有法律的生活。盗窃、非法扣押和工厂接管成为了日常现象……
\end{tcolorbox}
\begin{flushright}
—— 赫尔南多·德索托(HERNANDO DE SOTO)
\end{flushright}

\subsection{废墟中的秘鲁}
从某种意义上说,1993 年的秘鲁已不再是一个现代民族国家。虽然它还有一面国旗和一支军队,但大部分机构都成为了摆设,甚至连监狱都被囚犯接管了。秘鲁的崩盘有多方面的原因,不过大多数试图解释的专家都没有抓住真正的要点。

秘鲁只是技术变革的一个早期牺牲品,该变革正在使世界各地的封闭经济体失灵,使中央权威受到破坏。而像秘鲁国会这样的决策机制,陷于不正当的激励机制,把他们最需要解决的问题变得更加严重,从而使整个大政治压力雪上加霜。

秘鲁的代议制民主就像一副灌了铅的骰子\footnote{译注:即欺诈手段。}。作为扩大政府规模的机制,它是无与伦比的。但是,当新的形势要求下放权力时,在旧日大政治条件下使民主如此有效的固有偏见,使它越来越不符合发展的需要。国会通过的法律,在迅速摧毁所有的价值基础,以及人们对法律本身的尊重。德索托在《另一条道路》(the Other Path)中写道:“小型利益集团相互争斗,导致政治破产,公职人员牵连其中。政府分发特权,法律予取予求,远超道德允许的范畴。”像秘鲁那样的国会,完全受制于特殊利益集团,其道德地位类似在围栏内拍卖赃物。它使自由市场沦为非法,使法律成为笑谈。正如德索托对前滕森时代的描述:


\begin{tcolorbox}
对目的和手段的彻底颠覆,使秘鲁人的生活发生了翻天覆地的变化,以致于那些法定的犯罪行为,都不再受到集体意识的谴责。走私就是一个典型的例子。从贵族小姐到底层草根,都使用走私货物。没有人对此有所顾忌,相反,人们认为这是对个人智慧的一种挑战,或者是对国家的一种报复。\\
随着暴力和犯罪在日常生活中蔓延,贫穷与匮乏不断加剧。总的来说,秘鲁人的实际收入在过去 10 年内不断下降,现在已经回到了 20 年前的水平。全国各地垃圾堆积成山。乞丐、洗车工和拾荒者成群结队,不分白天黑夜,到处围攻路人,向他们索要钱财。精神病患者赤身裸体涌上街头,浑身散发着尿骚味。儿童、单身母亲和瘸子在每个角落乞求施舍。\\
……事实证明,我们这个社会传统的中央集权主义,显然无法满足一个转型期国家的多重需求。
\end{tcolorbox}

在滕森封锁国会之前,秘鲁人已经放弃了畸形的合法经济,转向了黑市;德索托称之为正在进行的“一场无形的革命”。我们对自由市场的好处持肯定态度,但是在一个法律和金钱同样堕落的社会,人们的承诺是否可靠,我们就不那么肯定了。

德索托把 1993 年之前的秘鲁描述为一个“发条橙”\footnote{译注:库布里克的著名电影。}的世界,过度集权又功能失灵的政府机构正在摧毁公民社会。

这正是滕森要着手改变的。他关停印钞机,大大降低了通货膨胀。他还想方设法裁掉了五万名政府雇员,并削减掉一些补贴。他开始平衡政府的财政预算。滕森要实施全面的改革计划,包括创建自由市场和工业私有化。但是,和前苏联一样,大部分重要的改革措施在 1993 年都没有被采纳,包括对国有银行、矿业公司和公共事业的第一轮大规模私有化。秘鲁的国会,就像在莫斯科挑战叶利钦改革的俄罗斯国会一样,并没有通过这些必要的提案,而是试图走回头路。他们的计划是:从空荡荡的国库中恢复补贴,提高工资,保护所有一切的既得利益,尤其是官僚机构的。这正是你对一个由雇员控制的政府所能期望的一切。

滕森指称秘鲁国会犹豫不决,腐败不堪;这一点大家都同意。他进一步控诉,国会的犹豫和腐败使秘鲁不可能改革崩溃的经济,也无法打击来自毒品恐怖分子以及虚无主义的“光辉道路”(Sendero Luminoso)游击队的暴力袭击。

\subsection{70\%支持率的解决方案}
于是,滕森关闭了国会。在很多人看来,这一行为可能意味着他和许多之前的拉美领导人一样,是个独裁者。但我们认为,而且我们当时也这么评价,滕森找对了路,他发现了改革的一个根本障碍。美国的社论记者和国务院官员对秘鲁国会进行了奢侈的官方赞颂,但并没有得到秘鲁人民的认同。当北美人把秘鲁国会当成自由和文明的化身时,秘鲁人民却在欢呼。当滕森总统把国会送回老家时,他的支持率飙升到了 70\%以上;后来他又以压倒性优势连任。到 1994 年,秘鲁的实际经济增长率达到 12.9\%,全球最高。

\subsection{政治承诺的通货紧缩}
在我们看来,秘鲁的动荡与其说是对昔日独裁统治的回溯,不如说是更广泛转型危机的早期阶段。可以预期的是,随着政治承诺的落空与政府信用的衰竭,许多国家都会出现类似的治理危机。最终必将出现新的制度形式,能够在新的技术条件下维护自由,同时让所有公民共享的利益得到表达与成长的空间。

工业时代的政府机构与后工业社会的大政治条件是不兼容的,这一点还很少有人意识到。然而,无论这其中的矛盾是否得到了明确的认知,随着世界各地政治失败的例子越来越多,它的后果会越来越明显。现代时期出现的统治机构,反应的是一个或几个世纪前的大政治状况。信息时代需要新的代表机制,去避免长期的功能失调甚至社会崩溃。

1989 年柏林墙的倒塌,不仅标志着冷战的结束;也是世界权力基础遭到无声地震的外在表现。它是漫长的暴力回报上升期的结束。在 1987 年的《血流成河》以及更早的《战略投资》(Strategic Investment)月刊中,我们就曾经预测,共产主义的垮台不仅仅是对一种意识形态的摒弃。它是过去五个世纪以来暴力发展史上最重要的外在标志。如果我们的分析是正确的,随着大规模使用暴力的经济性越来越差,社会的组织形式必将发生改变。限定未来的边界已经被重新划分。