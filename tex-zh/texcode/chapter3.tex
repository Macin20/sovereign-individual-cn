\chapter[伊甸园之东]{伊甸园之东}

\begin{tcolorbox}
耶和华对该隐说:“你的兄弟亚伯在哪里?”他说:“我不知道!我岂是看守我兄弟的么?”耶和华说,“你作了什么事呢?你兄弟的血有声音从地里向我哀号。
\begin{flushright}
—— 创世记 4:9-10 
\end{flushright}
\end{tcolorbox}

五百代人之前,人类社会组织的第一场阶段性变革开始了。在几个不同的地区,我们的祖先极不情愿地拿起粗糙的工具,削尖的木桩和拼装的锄头,开始下地干活。他们种下了第一批庄稼,同时为权力的世界奠定了新的地基。农业革命是第一场伟大的经济和社会革命。它从亚当和夏娃被逐出伊甸园开始,发展非常缓慢,以至于到 20 世纪开幕,在全球适合农耕的所有地区,农业还没有完全地取代狩猎和采集。专家们认为,即使在最早出现农业的近东地区,它也是在“一个漫长的过程中逐渐引进的”,可能用了五千年甚至更长的时间。

将一个蔓延数千年的进程描述为一场“革命”,似乎有点虚夸。然而,正是农耕的出现,开启了一场慢动作的革命,使人类的生活因暴力逻辑的变化而彻底改变。

凡是农业扎根的地方,暴力都会成为社会生活中的主要特征,善于操纵或控制暴力的阶层就开始主宰社会。

理解农业革命是认识信息革命的第一步。耕作与收割技术的导入,为我们提供了一个范例,它说明了:一个明显很简单的工作性质的转变,就可以从根本上革新社会的组织形式。看透这场过去的革命,你就可以更准确地预测,当微处理器的导入再次革新暴力的逻辑,历史会对此做出怎样的反应。

要理解农业革命的特征,你首先需要了解农耕之前的原始社会是如何运作的。我们在《大清算》中对此做了研究,下面会做进一步的概述。在漫长的史前沉睡期,人类的生活经过一代又一代人,都变化不大,甚至可以说毫无改变。狩猎和采集是那时唯一的社会组织形式。人类学家认为,自从人类出现在地球,在 99\%的时间内都是狩猎采集者。狩猎采集的部落能够长久生存但最终又彻底消亡,关键在于,他们必须在一个非常广阔的区域内以一个很小的规模去活动。

觅食者(即狩猎采集者)只能在人口密度低的地方生存。你要问为什么,请想一下大群体的问题。首先,1000 名猎人在同一个地方一起狩猎,会引起很大的骚动,吓跑他们的猎物。更糟糕的是,如果一群猎人偶然捕获了大批的猎物,他们得到的食物,包括野外发现的水果和能吃的植物,也没办法长期保持充足和新鲜。

大规模的觅食队伍,就会像三十年战争中饥饿难耐的军队一样,因为过度地采摘而毁坏了农田。因此,为了减少过度捕杀,狩猎的组织必须是小团伙。正如斯蒂芬·博伊登(Stephen Boyden)在《生物学视角下的西方文明》(Western Civilizationin Biological Perspective)中所写的:“狩猎采集的团体最常见的规模是 25 到 50 人之间。”今天,在温和的气候下生活在上万英亩的土地上是一种奢侈,只有超级富豪才能享受到。但是,一个狩猎采集者的家庭,如果没有这么多的地就很难活下去;即使在最肥沃的觅食区,他们通常也需要人均数千英亩的土地。这就可以说明,为什么在特别适合觅食的历史时期,人口的增长会导致后来的人口危机。由于养活一个人需要那么大面积的土地,所以,狩猎采集社会的人口密度非常稀疏。在农耕出现以前,人类的居住密度跟熊差不多。

那时候的人在饮食上跟熊也大同小异。觅食社会赖以生存的食物,主要来自于开放的田野或附近的水体中。虽然有些采集者是渔民,但大部分是猎人,他们 1/3到 1/5 的饮食依靠大型哺乳动物的蛋白质。除了一点简单的工具和随身携带的物品外,狩猎采集者几乎不掌握任何技术。他们没办法有效地储存大量的肉类或其他事物,以供日后食用。大部分食物在得到以后就要尽快吃完,不然就只能任其变质。当然,这并不意味着他们不吃变质的食物。博伊登在书中说,“据说爱斯基摩人就非常喜欢吃腐烂的东西。”他复述了专家们的观察,爱斯基摩人会“把鱼头埋起来,让它们腐烂,等到骨头变得和鱼肉一样黏稠,他们就会把这些散发着恶臭的东西揉成面团状然后吃下去。”他们还喜欢“生吃驯鹿尸体上肥大的蛆状幼虫……鹿的粪便,像浆果一样嚼着吃……还有腐烂一年多的骨髓,上面爬满了驱虫。”除了这些美味佳肴之外,觅食者基本没开发出其他的食物。人类学家格雷格指出,“流动人口一般不储存食品,以应对资源供应的季节性或意外性的短缺。”因此,觅食者也没什么可偷的。在无法储存多余食物的情况下,劳动分工包括暴力的专业化是行不通的。狩猎的逻辑也决定了,狩猎觅食部落之间的暴力永远也不可能上升到很大的规模,因为这些群体自身就必须保持小规模。

在另一个方面,小规模对觅食组织也大有裨益。小群体的成员彼此都很熟悉,合作起来更加高效。因为随着成员人数的增加,激烈的难度会激增,做决策就越来越困难。你可以想一下,组织成百上千的人,在一个流动的宴席上走来走去是多么的让人绝望。狩猎采集的部落没有稳定独立的政治组织或官僚机构,专门去处理战争事宜,所以他们只能依赖于说服和共识,这在关系温和的小团体中最好用。

至于狩猎采集者是否随和,还有待商榷。亨利·梅因爵士(Henry Maine)提到过“原始人普遍好战”。用他的话说,“在自然界和原始社会中,普遍状态不是和平,而是战争。”他的观点得到了进化生物学家研究的支持。保罗·肖和王玉华(Paul Shaw and Yuwa Wong,后者为音译)评论说,“有明显的迹象显示,在第四冰川期及之前阶段,在欧洲的南方古猿人、直立人和智人的遗骸中,有许多伤痕明显是由打斗造成的。”但也有人对此表示怀疑。像斯蒂芬·博伊登这样的专家认为,原始群体通常不喜欢战争,不容易发生暴力。原始社会的习俗主要是为了减少内部的紧张关系,促进猎物的分享。特别是在捕猎大型动物的地区,这些猎物对猎人个体来说是一种崇拜,因此出现了宗教和社会信条,用来支持对集体捕获的任何猎物的重新分配。与其他猎人分享热量来源是摆在首位的,所以,必要性而不是情感,才是激励的动因。首先分得资源的是最有经济能力及最具战斗力的人,而不是老弱病残。这种优先权的形成,毫无疑问,是因为正值壮年的猎人是整个团队种最具战斗力的。保证他们的优先权,整个团队就可以最大限度地减少潜在的致命内斗。

只要人口密度保持比较低的水平,觅食者就不会崇拜好战的神,而是崇拜自然力量或者他们狩猎的动物的化身。因为资本匮乏,边境开放,大部分情况下根本没必要打仗。除了自己的小家庭或者部落,也很少有什么邻居能构成威胁。觅食者往往都是四处漫游去寻找食物,超出最低生存限度的个人财产其实是一种负担。

财产不多,就不容易遭到抢劫。如果发生冲突,争执的双方会倾向于走开,反正他们在任何地方都没有什么特定的投资。对于个人之见的恩怨或者其他过分的要求,逃离是个很简单的解决办法。但这并不意味着早期的人类都爱好和平,他们的暴力和凶残可能超出我们的想象。但如果他们使用暴力,那主要是出于个人原因,或者更糟糕的,只是为了做运动。

狩猎采集者的生存取决于他们在小范围内的运作能力,除了性别的分工,几乎没有其他分工的空间。他们没有政府组织,通常也没有永久的居住点,也不可能积累财富。甚至像书面语言这种基本的文明构成在原始社会也不存在,没有书面语言,就不可能有正式的记录和历史。

\subsection{过度猎杀}
觅食动力学发展出来的工作动机,与我们在农业出现后所习惯的工作动机大不相同。觅食者所需要的生活资本是最小的,只需一点原始的工具和武器就够了。他们没有投资渠道,也没有土地等私有财产,除了偶尔在采石场凿几块打火石或肥皂石。人类学家苏珊·艾琳·格雷格在《觅食者与农民》中写道,“(觅食者的)资源的所有权和使用权”是由“群体共同拥有的。”因为没有固定的住所,他们也不需要努力干活获得财产或维护财产;他们没有抵押贷款,不交税,也不需要购置家具。他们为数不多的消费品是动物皮毛,以及由团体成员自制的个人装饰品。没有什么可买的,也就没有什么动力去获取或积累可以被当作金钱的东西。

在这种情况下,对于觅食者来说,储蓄只是一个还未发展起来的概念。

没有赚钱的理由,就基本没有劳动分工,勤奋工作作为一种美德概念,对狩猎采集者来说肯定是陌生的。除了在非常的困难时期,需要长时间出去觅食,在大部分时间内,他们很少干活,因为没什么需求。超出最低限度生存所需的工作,实际上也不会得到更多东西。所以,狩猎采集部落里的成员,每周只需工作 8-15个小时。猎人的劳动并不能增加食物的供应,反而会消耗它。如果一个人殷勤地加班,猎杀了更多的动物或者采集了更多的水果,在腐烂变质之前吃不掉,那他对集体的繁荣其实没有任何贡献。相反,过度地猎杀会影响未来觅食的前景,对集体的福祉产生不利的后果。这就是为什么一些觅食者,如爱斯基摩人,会惩罚或排斥那些过度猎杀的成员。

爱斯基摩人惩罚过度猎杀的例子特别有说服力,因为相比其他人类,他们更有条件和能力通过冷冻去储存更多的肉类。或者,至少可以储存一些从大型海洋动物中提炼的油。但觅食者很少这么做,这反映出他们在与自然的互动中非常被动。

这可能也说明了,人类的认知和智力发展受文化影响的程度。复杂环境对学习和行为的限制,使得要采用某些策略比看起来要困难得多。就像保罗·肖和王玉华所写的那样,“生态位千差万别,对学习的偏见也是如此。”从这个角度看,农业的出现不仅仅改变了人类的饮食结构,还改变了暴力的逻辑,启动了经济生活和文化组织的重大变革。农业创造了大规模的土地资本,有些地方还发明了灌溉系统。农民培养的农作物和饲养的动物都是有价值的资产,它们可以被储存和囤积,也会被盗窃。由于农作物从种植到收获的整个生长季节都需要照顾,特别是在干旱地区,农业只局限在有可靠水源的小块土地上,这就使得人们即使面临威胁,也不太愿意逃离到别处了。逃离越来越难以选择,有组织的抢劫和掠夺就增加了。到了收割季节,农民特别容易遭到袭击,这逐渐提高了战斗的规模。

社会的规模因此而扩大,因为在暴力的竞争中,获胜的往往是人数更多的群体。

对土地的争夺以及对土地产出的控制越来越激烈,社会变得更加固定。劳动的分工越来越明显,第一次出现了职业和奴隶制。农民和牧民负责生产食物,陶工负责制造储存粮食的容器,祭司们祈求风调雨顺和粮食丰收。专门从事暴力的人——也就是政府的前身,越来越多地致力于掠夺以及保护自己的社团免受掠夺;这些人和祭祀们一起,成为了人类历史上的第一批富人。

在农业社会早期,这些武士开始控制每年收成的一部分,作为提供保护的成本。

在威胁最小的地方,自耕农有时候还能保留较大的自主权。但是,随着人口密度的增加,对食物的竞争加剧,特别是在沙漠周围,那里能产粮食的土地非常宝贵,保护集团就可以拿走总产量的很大一部分。这些武士抽取的利益高达粮食产量的25\%,以及驯养的动物增长量的一半,他们用这些抽成建立了第一批国家。因此,农业极大地提高了压制的重要性。可掠夺资源的激增,带来了掠夺的爆发。

农业革命的全部逻辑用了几千年的时间才完全实现。很长一段时间内,在人口稀疏的温带地区,农民还像他们的觅食者祖先一样生活。在土地和雨水充足的地方,没有太多暴力的干扰,农民们可以小规模地耕作。但随着几千年来人口不断增加,即使在人口稀少的地区,农民也会不定期地遭到掠夺,有时候甚至没有足够的种子去播种下一年的作物。在那些没有保护的社区,由于没有暴力垄断的专门组织,可能就会出现一种极端的情况,就是竞争性的掠夺,或者说是无政府状态。

随着时间推移,农业固有的暴力逻辑,深入到了越来越广阔的地域。不受政府掠夺还可以从事农牧的范围,退缩到了少数真正偏远的地区。举个极端的例子,阿富汗的卡菲尔地区,到 19 世纪的最后十年之前,都一直抵制强加的政府组织。

但为了抵制,在几个世纪之前,他们就演进成了一个好战的民族,按照亲属关系组织起来;即使这样的组织没有能力集结大规模的暴力。在英国人把现代武器带到该地区之前,卡菲尔人一直在偏远的巴什加尔和瓦加尔山谷中保持着独立,高山和沙漠等地形的保护,在他们的堡垒与外部的征服者之间形成了天然的屏障。

慢慢地,在所有从事农业的社会,农业革命的基本逻辑开始强大起来。农耕大大提高了人类社区的规模。大约一万年以前,城市开始出现了。虽然以今天的标准看,这些城市很小,但它们是第一批“文明”中心,这个词来自于 civitas,在拉丁语中是“公民”或“城市居民”的意思。农业创造了可掠夺和需保护的财产,所以也催生了对库存会计的要求。因为如果不能汇编记录和开具收据,就没办法征税。会计师账簿中使用的符号就是书面语言的雏形,这种创新在狩猎采集者中间从未出现过。

农业还拓宽了人类解决问题的范围。狩猎者生活在一种即时的时间范畴,他们很少开展超过几天的活动。但种植和收割庄稼则需要几个月的时间。从事长时间框架的活动,让农民把注意力投入到了星空之上。详细的天文观察,是制定历法和日历的前提条件;而历法则是作物种植及收割的最佳指南。所以,随着农业的出现,人类的视野扩大了。

\section{财产}
向定居型农业社会的转变,诞生了私有财产。很显然,没有人愿意眼睁睁看着,他忙碌了一整季种植的作物被别人跑过来抢走。财产的概念是农业发生的必然结果。但是,私有财产作为一个概念,它的明确性也被随农业而来的暴力逻辑给削弱了。因为在觅食社会,每个健康的成年男人都是猎人,且同等武装;而在农耕时代,个人的大政治力量不再像觅食社会那么平等,这使得财产的情况变得很复杂。农耕使得暴力开始专业化。因为它生产了可以偷和抢的东西,投资于更好的武器装备就变得有利可图。结果就是大部分的盗抢都是高度组织化的。

强力之人开始构建一种新的掠夺形式:地方暴力的垄断组织,或者叫政府。社会因此而急剧分化,通过掠夺获益的人和耕种农田的广大穷人,分别落入了不同的境地。少数控制武装力量的人,以及受到他们宠幸的,开始富有起来。神王和他们的盟友,即那些统治近东最早一批国家的小型地方势力,比起在下面劳作的广大民众,享有更多更接近现代形式的财产。

当然,在农业社会的早期,要想区分私人和公共财产是不合时宜的。执政的神王拥有国家的全部资源供其支配,跟拥有一个大庄园差不多。和欧洲的封建时期很像,所有的财产都受制于更高的权贵;处于等级制度下层的人,会发现他们的财产会因统治者的兴致所至而减少。

不过,专制君主虽然不受法律约束,但并不意味他可以随心所欲地夺取任何东西。

成本与回报限制着法老的自由,就像今天对加拿大总理的影响一样。而且,比起当代的领导人,法老更加受制于交通和通讯的困难。仅仅是转移战利品,特别是当战利品是农产品时,就会因为变质和盗窃而造成大量损失。官员们要互相监督,以减少盗窃,这会增加法老不得不承担的管理成本。权力的分散,在某些情况下会优化产出,但也催生了强大的地方势力,有时候会发展为对王朝统治的全面挑战。所以,即使东方的专制者也不可能肆意妄为。他们别无选择,在他们发现原始权力的时候,也只能接受其间的平衡。

虽然每个人的财产都可能遭到任意的征用,包括富人,但还是有些人可能积累起自己的财产。当时和今天一样,国家把大部分收入用于公共工程。灌溉系统、宗教纪念堂和国王的陵墓等大型项目,都为建筑师和工匠们提供了赚钱的机会。一些有条件的人因而积累了大量的私人财产。事实上,在现存的美索不达米亚早期文明——苏美尔人的楔形文字中,就有很大一部分记录了各种贸易行为,其中广泛涉及财产所有权的转让。

农业社会的早期确实出现了私有财产,但它不属于社会金字塔的底层。占人口绝大多数的农民,往往都一贫如洗,不可能积累起财富。实际上,除了少数历史阶段,直到现代时期,大部分农民都处于勉强维生的状态,一旦有干旱、洪水或病虫害造成作物减产,他们就面临被饿死的危险。因此,农民不得不以某种形式去组织他们的生计,以尽量减少坏年景下的风险。于是,在整个社会广大的贫困阶层中,一种更原始的财产组织方式出现了。它增加了农民的生存概率,但却剥夺了他们积累资本及在经济系统中上升的大部分机会。

\subsection{农民们的保险}
这种亏本生意的形式,就是人类学家和社会历史学家所描述的“封闭的村庄”。

在前现代,作为主要的经济组织形式,“封闭的村庄”几乎存在于每一个农业社会。在更现代的经济生态中,个人倾向于到一个开放的市场上,与众多的买家和卖家做生意。而封闭的村庄与此不同,它是村里的家庭联合起来,像一个非正式的公司或者大家庭一样运作;它不是在一个开放的市场上,而是在一个封闭的系统内,因为它所有的交易往往都交给一个垄断者完成,那就是当地的地主,或者他在村里的代理人。整个村庄与地主签订合同,通常是以实物的形式,交付很高比例的粮食收成,而不是固定的租金。按比例交租意味着地主吸收了庄家歉收的部分风险,当然,地主也拿走了大部分的潜在收益。地主一般还提供种子。

这样的安排将饥荒的危险降到了最低。它要求地主,而不是农民,从他的部分收成中节省出不成比例的份额。因为在过去,很多地区的农业产量都低得吓人,每收成三粒粮食都要种下两粒。在这种情况下,收成不好就意味着大规模的饥荒。

农民理性地选择了一种制度安排,由地主对他们的生存进行投资。以垄断价向地主购买种子、然后廉价地把粮食卖给地主,并且向地主提供实物劳动,以此为代价,农民增加了他们的生存机会。基于类似的想法,封闭乡村经济中的典型农民,往往都放弃了自由保有财产所有权。把自己置于村长的摆布之下,农民家庭增加了从定期的土地重新分配中获益的机会,因为村长总是会把最好的地分给自己和他喜欢的人。这是农民必须承受的风险,为了享受村里混乱的土地所有权带来的生存保险。因为相隔一百码远的两块土地,生产条件可能就很不一样;在作物产量低得可怜的时候,往往就是生存与饥饿之间的差距。农民通常倾向于低风险的选择,甚至不惜放弃任何繁荣昌盛的机会。

一般来说,所有活在生存边缘的群体,都会倾向于规避风险。在前现代社会,纯粹的生存挑战一直制约着穷人的行为模式。在《大清算》中我们探讨过,风险规避很有意思的一点是,它缩小了社会允许个人从事经济活动的范围,即使是和平的经济行为。禁忌和社会束缚限制了实验和创新,甚至会放弃对既定做事方式明显有利的潜在改进。这是对现实的一种合理反应,因为实验会增加结果的可变性。

可变性增加虽然意味着更大的收益,但对处于生存边缘的人来说,则预示着可能遭遇毁灭性的损失。在贫穷的农业社会,一直有很大一部分文化能量致力于压制实验和创新。这种压制,实际上是他们为保险政策寻找的替代品。如果他们有保险,或者有足够的储蓄,能够为他们的实验行为提供自保,就不需要如此强烈的社会禁忌来保证生存了。

文化不是品味的问题,而是适应特定环境的系统。某种环境中的文化,放到其他环境中可能格格不入,甚至适得其反。人类生活的环境千差万别。丰富多样的环境生态,要求我们做出行为的改变。而这些变化太过复杂,是本能无法应对的。

因而,行为是由文化编程的。对于农业社会的大多数人来说,文化给他们编写的程序就是生存,而超越生存、可以参与公开市场的奢侈则被保留给了其他人。

个人能力及自主选择——现代意义上个人“追求的幸福”——往往被社会禁忌和限制所压制,这在穷人身上表现得最明显。在生产力有限的社会中,这些限制极难被取代。在农业生产力较高的地方,如古希腊,就发生了小规模的大政治革命。

人们采用了更现代的财产形式,出现了“Allod”,即自主地(保留绝对所有权的土地)。在这种情况下,土地往往以固定的费用出租,承租人承担经济风险,而当收成良好时,也可以获得更高的收益。更高的储蓄可以承保风险更大的经济活动。这使得自耕农可以超越农民阶级,有时甚至可以积累独立的财富。随着社会逐渐摆脱贫困,在经济等级制度的顶端,开始出现类似市场性质的产权和关系;在比较罕见的情况下,可能会发展到整个经济体中。这种趋势是社会组织的一个重要特征。同样重要的是,应该注意到,历史上最常见的农业社会基本都是封建主义的,市场关系在顶层,封闭的村庄系统在底层。在几乎所有的前现代农业社会中,广大的农民都被束缚在土地上。只要农业的生产力依然很低,或者较高的生产力取决于中央的水利系统,底层农民的个人自由及财产权利就是最低的。封建主义的财产形式也因此普遍存在。农民往往只拥有土地的使用权,而不是自由产权;出售、赠送和继承的权力都受到限制。

形形色色的封建主义,不仅仅是对无处不在的暴力掠夺的回应;也是对低得惊人的生产率的一种反应。在农业社会中,这二者往往是相辅相成,互相促进的。当公共权力崩溃时,财产权和经济繁荣也会相应消退;当生产力崩溃时,公共权威也会遭到破坏。虽然不是每次干旱或不利的气候变化都会瓦解公共权力,但很多时候都会如此。

\section{1000 年的封建革命}
1000 年时的变革就是如此,它开启了封建革命。当时的经济和大政治状况与我们所认为的中世纪大不相同。在罗马灭亡后的最初几个世纪里,西欧的经济萎靡不振。在前罗马帝国领土上扎根的日耳曼王国,承担了罗马政权的诸多职能,但水平难望罗马人之项背。基础设施差不多都荒废了。几个世纪过去,桥梁和水渠年久失修,无法再使用。罗马的硬币依然有效,但实际上它已经从流通中消失了。

罗马时代繁荣的土地市场也基本都退出了历史舞台。城镇曾经是罗马行政管理的中心,它伴随着国家征税权一起衰竭了。其他几乎所有的文明附属品,也都不复存在。

这一时期之所以被称为“黑暗时代”,是有原因的。识字率严重下降,以致于任何有读写能力的人,不管犯下什么罪行,都可能被免于起诉,包括谋杀。罗马时代高度发展的艺术、科学和工程技术都消失了。许多曾广为人知且达到很高标准的技术,从道路建设到葡萄等果树的嫁接,在西欧都停止使用了。甚至像陶轮这样古老的工具在很多地方都找不到了。采矿业、冶金业全都萎缩了。地中海地区的灌溉系统,也因疏于管理而破败不堪。就像历史学家乔治·杜比(Georges Duby)所观察到的,“六世纪末的欧洲是个极不文明的地方。”尽管到了 800 年左右,在查理曼大帝的统治下,中央权力曾有过短暂的复兴,但他去世之后,一切又很快陷入了困顿。

在这种枯寂的景象之下,有一个推论也许会让人感到惊讶,那就是:罗马帝国的崩溃,可能在随后的几个世纪里,反而提高了小农们的生活水平。在黑暗时代统治西欧的日耳曼王国,从他们祖先的部落中继承了一些相对随和的习俗,如自由民在法律面前一律平等。因此,黑暗时代的农民远比他们在封建时期要自由得多。

通过这一点,我们也可以推断出,他们要更加富裕。我们在上文探讨过,在不同的生产力条件下,会出现不同的财产形式。自由财产权与小农的相对繁荣,在历史上是携手并进的。在农民谋生都值得怀疑的地方,往往就会出现封闭的村庄和封建的财产形式。

可以肯定的是,黑暗时代的商业事实上已经崩溃,这使小农失去了贸易的好处及广阔市场的优势。城镇的消亡破坏了现金经济,但这也意味着农村人口可以摆脱官僚机构的沉重负担。正如居伊·布瓦所写的,罗马的城镇是寄生的社区,而不是生产的中心。“在罗马时期,城市的主要功能是维护政治秩序,它靠从周边地区抽取的土地税过活,……实际上,城镇几乎没有为周边的农村生产任何东西。”罗马当局的崩溃,在很大程度上将农民从税负中解放了出来;因为税收吸走了“土地总产值的四分之一到三分之一,这还没算中小土地所有者遭受的其他征收。”真可谓税负猛于虎,有时候甚至要通过处决来强制收税,这导致土地所有者遗弃财产的现象非常普遍。而日耳曼的蛮族统治者,则仁慈地放弃了这些税收。

\subsection{被搁荒的农田}
因为蛮族的政府,来自政府的负担被大大减轻,这就为穷人自由获得和持有财产创造了机会。在罗马帝国最后的一些年头里,一些因为业主逃避税收掠夺而被搁荒的土地,又重新投入了生产。尽管当时环境很恶劣,而且按照现代的标准,作物的产量可谓低得离谱,但对欧洲的小农户来说,黑暗时代是一个相对繁荣的时期。事实上,他们在那时候的地位,是在现代时期之前最高的。首先,他们有大片肥沃的土地,而能够耕地的人口却越来越少,大片土地被搁荒。因为瘟疫、战争以及罗马帝国的崩溃,土地所有者纷纷四散逃离,土地被遗弃,而人口也大幅减少。其次,六世纪时发明了新的农业技术,黑暗时代的小农户带来了发展的优势,那就是安装在轮子上的重型犁。这种工具与改进的马具搭配在一起,使农民可以驱使多头牛,整理北欧的林地就变得很容易。

在这种环境下,土地市场几乎完全萎缩。任何人,只需要整理土地并与有关分享一部分产出,就可以获得新的耕地。这个过程被称为“垦伐”,在罗马崩溃后的几个世纪里,它很好地支持了人口的增长。到了 8 世纪,气温变高,农业效率也提升了,在人口稀少的北欧地区,垦伐变得特别抢手。

日耳曼部落的首领们,征服了前罗马帝国的领土,现在他们是最大的土地所有者。

剩下的大部分人口,耕种小块的田地,但是条件与其后的封建时期非常不同。比较富裕的土地所有者或者主人,约占人口的 7-10\%。在 1000 年之前,法国一个典型的村庄里,似乎有三分之二的村民都是土地所有者;他们拥有大约一半的土地。农奴很少,佃农的数量不超过人口的 5\%。奴隶制依然存在,但规模比罗马时期小很多。

为日耳曼王国提供军事保护的都是自由人,这些人由国王在地方上的代表——伯爵——召集并武装。即使是“小业主和中等业主”也要加入,要派一个人加入步兵一起作战。在《皮特雷斯法令》中,秃头查理要求所有能力适宜的人,都要被集结起来,上马作战。而在一个世纪之前,公元 732 年,教皇格雷戈里三世曾下令禁止人们食用马肉,意图推动这种军事需求。来自自由民的步兵在地位和权利上,与骑兵并没什么区别。所有的自由人都可以参加本地的司法会议,并向伯爵提出解决争端的请求。伯爵这个职位自从罗马后期就开始存在,而在此前,并没有类似的贵族身份。


\begin{tcolorbox}
在十世纪 80 年代,一种社会现象,大规模发生的新现象,突然出现在人们的视野中,那就是社会的向下流动。而它的第一批受害者就是小型的土地所有者。
\begin{flushright}
—— 居伊·布瓦(GUY BOIS)
\end{flushright}
\end{tcolorbox}

然而,伴随着黑暗时代的到来,发生了几件事,破坏了日耳曼王国维护的、自耕农与自由民(永久产权持有者)之间相互独立的关系。

\begin{enumerate}
    \item 人口逐渐恢复,对土地的利用产生了更大的压力。几个世纪以来,大面积最肥沃的无主之地都被投入了生产,特别是在北欧。相对土地供应,农民人口的增长使单个农民的劳动价值降低。在黑暗时代,孩子往往可以平等地分享父母的财产;经过继承,大多数自由民的土地被分割成了越来越小的地块。在人口不断增加、地块越来越小的情况下,土地再次产生溢价;曾经消失的土地市场,在 10世纪中期也再次活跃起来。
    \item 在 10 世纪的最后几十年,气候突然变冷,给农业造成了毁灭性的打击。从982 年到 984 年,连续三年作物歉收,导致了严重的饥荒。994 年,歉收和饥荒再次袭来。随后,在 997 年,爆发了瘟疫,使粮食问题变得更加复杂。这场瘟疫对小型家庭结构的打击特别大,因为小农户缺乏资源去替代失去的家庭劳动力。连年的作物歉收和灾难使自耕农陷入了债务危机。当粮食产量无法恢复,他们就无力支付抵押贷款。
    \item 重装骑兵日益重要导致权力关系日趋不稳。中世纪历史学家弗朗西斯·吉斯(Frances Gies)描述了装甲骑兵向中世纪骑士的转变:骑士原本地位平庸,但因其昂贵的马马和盔甲而高于农民。慢慢地,他们提升了自己在社会中的地位,最终成为贵族阶级的一部分。尽管他们仍然处于上层社会的最低等级,但骑士身份被赋予了一种荣誉;这种荣誉为大贵族乃至皇室所珍视。它的魅力主要是教会将骑士身份进行了基督教化,它通过各种政策,将骑士的仪式神圣化,并且宣扬一种被称为骑士精神的行为准则。这些准则被违反的次数可能远超过被遵守的次数,但是它们对后人的思想和行为产生了无可争议的影响。 正如我们在《大清算》中所指出的,马镫的发明,使全副武装的马背骑士拥有了强大的攻击能力。他现在可以全速地进攻,而不会被长矛刺中敌人产生的冲击力给甩出马鞍。重型骑兵的军事价值,因为一项亚洲人的发明而得到了进一步的提升。这项发明在 10 世纪传到了欧洲,那就是马蹄铁,它延长了马匹在道路上奔跑的时间和寿命。此外,提升骑士作战效率的发明还有,方便骑士挥舞重武器的马鞍、马刺,以及用一直手就能控制马匹的马鞍。显然,这都是一些很小的技术革新,但它们组合在一起,就大大降低了小农在军事上的重要性,因为他们无力饲养战马及武装自己。在专门为打仗而饲养的马匹中,比较便宜的是一种叫做destriers 的大体型冲锋战马,价值 4 头牛或 40 只羊,更好的战马则需要 10 头牛或 100 只羊。盔甲也是小户人家负担不起的,一套相当于 60 只羊的价格。
    \item 寒冷的天气、农作物的歉收、饥荒和瘟疫,在 1000 年降临之前,纷至沓来,这在一定程度上影响了人们的行为。许多人相信,世界末日或者基督再临就在眼前。大大小小的土地所有者,虔诚的信徒也好,被吓坏了也好,纷纷将他们的土地交给教会,准备迎接世界末日。
\end{enumerate}

\subsection{“只有穷人才卖地”}
十世纪末不稳定的社会状况为封建革命铺平了道路。一连串的农作物歉收和天灾,使自耕农陷入了债务困境。粮食产量迟迟不能回复,自由民也面临着绝望的境地。市场总是会把最大的压力放到最弱的财产持有者身上,而这其实是它的一个优点;借此可以把资产从弱者手中转移出去,提高利用效率。

但是,在十世纪末的欧洲,自给自足的农业几乎是唯一的职业。一个家庭失去土地,就失去了他们唯一的生存手段。面对这种悲惨的前景,在封建革命期间,许多或大多数自由民决定放弃他们的土地。用居伊·布瓦的话说,“农民想要守住自己的耕地,唯一可靠的方法就是把土地的所有权让给教会,自己保留用益权。”其他人则把全部或部分土地让给了他们信任的富农,可能是关系比较好的邻居,或者是亲戚。


这些财产转让的条件是农民以及他的家人和后代要继续在田里工作。贫穷的农民可以享受到一些互惠的支持,来自于持有大量土地的人,现在被称为“贵族”,他们有能力购买马匹和盔甲,从而为不断扩大的庄园提供保护。这样的交易可以被视为是一种新农奴主义,它是延续经济所有权和丧失赎回权的中途站。但更多时候,它是农民无法拒绝的贱卖。

生产力的下降不仅使贫农陷入了生存的困境,还激发了暴力掠夺的增长,从而破坏了社会的财产安全。因为可获得的马匹和饲料是有限的,那些没有能力从中抢到一部分资源的人,突然发现他们和他们的财产都不再安全了。他们的境遇放到今天,就好像你必须用一种新型武器保护自己,而需要支付的价格是 10 万美元;如果你付不起,你就只能受制于那些买得起的人。

短短几年时间,国王和法院维护秩序的能力就崩溃了。任何拥有盔甲和战马的人都可以制定自己的法律,结果就像是 10 世纪末的《银翼杀手》,一场抢夺和斗争的大混战,当局根本无力阻止。武装骑士的抢劫和袭击扰乱了乡村的秩序,但这绝不是说受害者都是穷人;相反,大土地所有者中的老弱病残及准备不足的人,才是更有吸引力的目标,他们有更多的财产。

这种情况发生在降温、饥荒和瘟疫对资源造成压力的时刻,绝不是一种巧合。可能引发权力变革的大政治条件在之前已经存在了,但它的潜能要等到危机被触发才会释放出来。歉收和饥荒就是这样的催化剂。虽然很难准确重构事件的发展顺序,但抢掠是出于绝望的经济环境,至少是部分正确的。而暴力一旦被释放出来,显然已经没有人有能力再动员力量去阻止它了。绝大多数农民都武装单薄,当然无能为力;即使几十个农民也打不过一个重武装的骑士。自由民,就像他们的国王和伯爵,也同样挡不住土地被掠夺。

\subsection{“来自上帝的和平”}
穷途末路之下,在暴力肆虐的农村,教会出面进行休战谈判,这在一定程度上帮助启动了封建主义革命。历史学家居伊·布瓦这样说,“政治当局失能,教会发动了一场名为‘上帝的和平’的运动,代替政权进行秩序恢复。他们以‘和平议会’的名义,发布了一系列的禁令,违反者将受到革出教门的惩罚。他们还举行了大规模的‘和平集会’,得到了骑士们的宣誓效忠。这项运动起源于法国的中部地区(989 年的夏鲁瓦议会,990 年的那波内议会),然后逐渐蔓延开来……”教会达成的交易是,承认武装骑士在当地社区的统治地位,以换取暴力掠夺的终止或缓和。在十世纪末期,由于暴力事件激增,在土地的地契上突然开始出现带有“nobilis”或“miles”的头衔,以表示领主的身份。仅仅在几年前,同一个人的财产交易记录,都没有显示出这种区别。贵族作为一种独立的身份,是封建革命创造出来的。

由于生产力下降,小农丧失经济保障,武装其实作为大政治的主导力量,不可避免地导致了领主占有的财产所有制形式(即封建主义)。到 11 世纪的前四分之一世纪末,自耕农基本上消失了;他们现在只是兼职务农,所持有的土地已经缩减到从前的一小部分。小农户及其后代沦为了农奴,大部分时间都在封建领主的庄园里劳动,包括教会的及非教会的。

伴随着封建革命,原有的社会秩序崩溃了,人们的行为方式也发生了变化,封建主义越来越被强化。其中一个很典型的现象就是,人们开始大兴土木,建设城堡。

最早的城堡出现在西北欧,是 9 世纪为应对维京人的突袭而打造的,都是原始的木制结构。城堡最初是卡洛琳王朝官员的指挥中心,封建革命之后,成为了世袭财产。这些早期的堡垒比后来的要原始得多,但还是很难被攻破。城堡一旦建立100起来,想把它夷为平地,就要费尽千难万险。随着城堡在乡野之间遍布林立,国王或他的伯爵们越来越不可能挑战地方领主的霸权地位。

\subsection{教会对生产力的贡献}
封建主义是农业社会在生产力低下时对秩序崩溃的一种反应。在封建社会早期,教会扮演了至关重要的角色,并促进了经济的发展。教会的贡献如下:1、在军事力量分散的环境中,教会具有独特的优势,它可以超越割据的地方势力,制定为各方所遵守的秩序规则,维护和平。这是任何世俗权力都无法胜任的。

伟大的宗教权威 A.R.拉德克利夫-布朗的观察,很好地解释了这一点。他指出,“宗教的社会功能与它的真假无关”,即使是那些“荒唐的、令人厌恶的宗教,如野蛮部落的,也是社会机制中重要的、有效的组成部分。”封建社会早期的教会就发挥了这样的功能。它做了只有宗教组织才能做到的事,它帮助制定了规则,并帮助人们摆脱激励的陷阱和行为的困境,其中一些是人类生活中普遍发生的道德困境,有些则是当时大政治背景下特有的地方性困境。在 10 世纪的最后几年,中世纪的教会,在恢复农村秩序方面发挥了重要的作用。教会向地方政府提供宗教和仪式上的支持,降低了在地方形成暴力垄断的成本,或者至少是一种弱垄断。

通过这种方式,教会帮助建立了社会秩序,并为最终达成更稳定的权力结构创造了条件。

在其后的很长时间内,在世俗权力无力遏制的私人战争和过度的暴力冲突中,教会都发挥着重要的调节作用。相对于世俗权威,教会越来越重要,从一点即可以看出。到 11 世纪,西欧大部分地区的行政区划是基于教区,而不再是从罗马帝国到黑暗时代一直存在的民事权力划分,即基于 ager(地块)或 pagus(城镇)。

1012、教会是保存和传播技术知识及信息的主要机构。在中世纪,教会资助大学,并为社会提供最低限度的教育。教会还建立了一种机制,可供人们复制书籍和手稿,包含当时几乎所有的农牧信息。那时候还没有印刷术,但本笃会修道院的缮写室可以被视为是一种替代的印刷形式。尽管缮写的成本很高、效率很低,但它是封建时期复制和保存书面知识的唯一手段。

3、教会极大地提高了欧洲农业的生产力,特别是在封建社会早期,这部分归功于农地的管理者往往是识字的。在 13 世纪以前,那些帮助世俗领主管理土地的人,几乎都是文盲,他们靠一套精心设计的符号进行记录。尽管他们都是很聪明的农民,但是,他们没办法学到那些非自己发明或非亲眼所见的、改良的生产方法。因此,教会对提高谷物、水果和种畜的质量至关重要。因为教会的机构广泛分布在整个欧洲大陆,它可以将产量最高的种子和种畜送到低产的地区。北欧对圣酒的需求,促使修士们培植了更耐寒的葡萄品种,可以在寒冷的气候中生存。

教会还以其他方式提高了中世纪的农业生产力。在封建革命期间,很多捐给教会的土地,由于地块太小,经济效应不高,被教会重新分配,以提高耕作效率。教会还提供小型农业社区需要的辅助服务。例如,在许多地区,教会拥有的磨坊,可将谷物磨成面粉。

4、教会承担了今天被政府吸收的诸多职能,包括提供公共基础设施。在权力分散的时代,这是克服经济学家所谓的“公共物品困境”的一种方式。在中世纪早期,有特定的教团致力于工程建设,如修路、补桥、重修罗马时代的水渠。他们还开垦荒地,建造水坝,排泄沼泽。一个新的修道会,卡尔图斯会,在法国的阿图瓦打出了第一口“自流井”。他们用冲击钻,挖了一个足够深的小洞,创造了一个不需要泵水的水井。在欧洲的低地国家,西多会负责维护海堤,并修建堤坝。

102农民将土地租给西多会的修道院,再租回来,而僧侣们则承担维护和管理的全部职责。西多会还率先开发了水动力的机器,这些机器被广泛应用于“敲打、举重、研磨和压榨”。明谷修道院则从奥布河挖了一条两公里长的运河。在那些已经迁移到古罗马驻军道路之外的人口中心,教会也参与进去修建了新的道路和桥梁。

对那些修建或维修过河道以及为路人提供收容的地方领主,主教还为他们颁发了赎罪券。由圣贝内泽成立的修士会,即“桥之兄弟”,建造了当时最长的几座桥,包括阿维尼翁桥,一座横跨罗纳河的大桥,有 20 个拱形结构,在一边有一个小教堂和收费亭。就连一直到 19 世纪还屹立不倒的伦敦桥,也是由一位教士修建的,资金中还有 1000 马克的捐款是来自教皇的使节。

5、教会帮助孵化了更复杂的市场。比如说,建造大教堂与桥梁或水渠等公共设施的意义是不同的。至少在原则上,教会建筑只用于宗教仪式,而不是商业场所。

但是,不要忘记,兴建教堂有助于创造一个可以深化手工业和工程技能的市场。

就像在冷战期间,民族国家的军备竞赛无意中孵化了互联网;中世纪大教堂的兴建也催化了其他种类的衍生品,这就是商业的孵化。教会是建筑商和工匠的主要客户。教会采购用于圣餐仪式的银器、烛台和装饰教堂的艺术品,创造了一个本不存在的奢侈品市场。

在诸多方面,教会缓和了武装骑士在“封建革命”期间和之后释放出的凶残暴力。

教会还对提升农业经济的生产力做出了巨大贡献,特别是在封建主义早期的几个世纪。总之,教会是一个至关重要的机构,它很好地满足了黑暗时代末期农业社会的需要。

\subsection{暴力面前的脆弱}
103就像五个世纪之前罗马的陷落一样,“1000 年左右的封建革命,经历了三四十年的暴力骚乱”,它看似是一个很不寻常的事件,但也是由很多相互影响的复杂因素造成的。但是,这场恶人的胜利以及他们创造出的压迫机制,完美地反映了农业社会在暴力面前的脆弱性。与人类的觅食阶段相比,在暴力的组织和压迫的强度上,农业社会都取得了质的飞跃。

这从一开始就反映在农耕民族更加好战的文化中。在农业社会早期,人们崇拜的神是雨神和水神,因为农民认为这些神决定着农作物的产量。然而,降雨或发水的神也往往是战争之神,最早期的国王都会召唤他们;而国王则又是战争之王。

人们的生活因农业革命的变革而改变,农业与战争之间的紧密联系,反映在他们的宗教想象之中。被逐出伊甸园可以看作是人类社会从觅食到耕作的转变,它非常形象地描述了,人类从毫不费力就能在大自然的恩赐中获得食物的自由生活,沦落到了艰苦劳作方能糊口的境地。

\section{失乐园}
农业使人类走上了一条全新的道路。第一批农民确实播下了文明的种子。从他们的辛勤劳作中,诞生了城市、军队、算术、天文学、地牢、葡萄酒和威士忌、文字、国王、奴隶制和战争。尽管农耕为生活增添了很多乐趣,但从原始经济转变到农业,从一开始就不那么受人欢迎。《创世纪》的记录就可以证明这一点,它讲述的就是一个人类被逐出乐园的故事。圣经中伊甸园的寓言,就是人们对觅食者在荒野中享受轻松时光的深情回忆。有学者指出,“伊甸园”这个词似乎就来自苏美尔语的“荒野”一词。

从自由自在地生活在人口稀少的野外,到定居在从事农耕的村落,是一件令人深104感遗憾的事,这不仅表现在《圣经》中,也表现在人类对早起和出门工作的永恒怨恨之中。在《生物学视角下的西方文明》一书中,斯蒂芬·博伊登写到,伴随农耕而来的新生活方式是“逃避”(evodeviant)。在农业出现之前,成千上万代的人们,都像亚当一样生活在伊甸园中,而且他的造物主邀请说:“园中各样的果树,你都可以随便吃。”狩猎采集者没有庄家需要照料,没有牛群需要看管,没有税收需要支付。他们像无业游民一样,随心所欲地漂流,很少干活,不用对任何人负责。

自从有了农业,一种新的生活方式开始了,而且总体上更加压迫人。“地必给你长出荆棘和蒺藜来,你也要吃田间的菜蔬。你必汗流满面才得糊口。”农耕是艰苦的劳作,在人类的记忆中,农耕之前的生活就是失乐园。

农民凭双手创造的一切极大地改变了暴力的逻辑,这远远超出他们的想象。《创世纪》把第一个杀人犯该隐确立为“耕地的人”,这不是巧合,而是圣经不可思议的预言能力的一部分。而圣经的故事之所以由牧羊人讲述,是因为他们更容易理解,农业是怎样给暴力提供了杠杆。在短短的几节经文中,圣经的记载概括了一种逻辑,在人类社会中演绎了几千年。农业是纷争的孵化器,农业创造了大规模的固定资产,提高了暴力的回报,也极大地加剧了财产保护的挑战。有史以来第一次,农业使犯罪和政府都成为了赚钱的买卖。