\chapter[超越地域局限]{超越地域局限:网络经济的兴起} 

\begin{tcolorbox}
\kaishu 真正的问题在于控制。互联网无远弗届,任何一个政府都无法轻易控制。它将创造一个无缝链接的、不可监控的、反主权的全球经济区,并对民族国家的概念发出质疑。\begin{flushright}
—— 约翰·佩里·巴洛(JOHN BARLOW )
\end{flushright}
\end{tcolorbox}

信息高速公路,已经成为数字时代早期人们比较熟悉的比喻之一。它之所以引人注目,不仅因为它无处不在,更因为它其实是对网络经济的普遍误解。高速公路只是人行道的工业化版本,是人员和货物的物理运输网络。信息经济并不是高速公路、铁路或运输管道。它将信息从一点传输到另一点的方式,并不像重型卡车沿着横贯加拿大的高速公路,把货物从艾伯塔省运到新不伦瑞克省。世人眼里的“信息高速公路”,并不仅仅是运输路线;它其实就是目的地。

网络空间超越了地域局限。它可以让人们彻彻底底即时共享数据,无时无地又无影无踪。新兴的信息经济,建立在数百万计算机用户的连接与再连接所构成的互连之上。它的本质是这些连接将孕育出来的新可能。正如约翰·巴洛所说:“网络有望带来一个全新的社会空间,它将是全球性的,是反主权的。在这个空间,任何地方的任何人,都可以畅所欲言,向其他同类表达他的真实想法。这种新的媒介预示着智识与经济的自由,它可能会推翻地球上所有的专制权力。”网络空间,就像人们想象中荷马众神的领域,与我们熟悉的农场和工厂的地面世界全然不同。但是,网络空间的到来,并不是凭空想象,而是真实的。而且,远超现在人们的认知,信息的即时共享会像溶剂一样,瓦解掉大型机构。我们已经讨论过,它会改变暴力的逻辑;不仅如此,它还将从根本上改变信息与交易的成本,而这决定着商业的开展与经济的运行。我们预计微处理技术将改变世界上的经济组织。

\begin{tcolorbox}
\kaishu 相比历史上的任何时期,今天的企业,都更有可能在任何地方落户,使用来自任何地方的资源,生产能够销往任何地方的产品。
\begin{flushright}
—— 米尔顿·弗里德曼(MILTON FRIEDMAN)
\end{flushright}
\end{tcolorbox}

\section{地域限制}
正在衰亡的工业时代在想象信息经济时,第一反应就是把它看成是一项巨大的公共工程项目。由此我们可以看出,过去的范式在人们思维中的地位是多么地根深蒂固。这就好比听到 18 世纪末的农民说工厂是“一个有屋顶的农场”。然而,“高速公路”的比喻更成问题,它暴露出我们被地方限制绑架的严重程度。即使我们在技术上已经可以超越地域的局限性,但解放我们的工具被赋予的绰号,依然象征着一条从一个地方到另一个地方的路线。就像鲑鱼总是游回出生地的天生本能一样,地方性的理念还深深地蚀刻在我们的意识之中。

从古至今,人类的经济活动都被拴在地方性的区域内。20 世纪以前的人,大部分人终其一生,都像被软禁的囚犯一样,很少离开他们出生地超过几天的时间。一趟去往远方的旅程,往往需要几代人的努力。只有偶然发生的危机才会导致广泛的迁移,如战争、瘟疫或气候的不利变化。要把人类从一个悲惨的村庄迁出去,需要泰山压顶般的理由。没有什么能刺激他们背井离乡,去外面寻找更好的生活。

一直到最近,那些离家外出发展的少数人,还常常变得很出名。想想马可·波罗,直到今天还因为穿越欧亚大陆访问忽必烈的宫廷而闻名。他是那个时代的异类。前现代时期的游记留存下来的不多。在被广泛阅读过的作品中,《曼德维尔游记》(Mandeville’s Travels)是 1357 年用法语写成的。不过值得注意的是,它可能是由一个从未离开过欧洲的人创作的。曼德维尔描述了世界各地令人愉快、引人遐想的生活细节,包括很多埃塞尔比亚人只有一只脚的说法,“这只脚硕大无比,当他们躺下来休息的时候,可以用它为全身遮阳。”很显然,与曼德维尔同时代的人,很少有人知道他笔下的“埃塞尔比亚大脚”并不存在。

直到 15 世纪末,现代的探索之旅开启,各大洲之间才有了持续的接触。像哥伦布和达伽马这样无畏的船长,远航去开辟香料贸易,他们的非凡成就足以让每个有文化的家庭,在过去五个世纪的时间里铭记。

从农耕出现直到最近几代人,人类生活的特点就是死水一滩,波澜不惊。今天的人已经忘了这一点,特别是在“新世界”殖民地定居的欧洲人,那里的人员流动更加自由,每个人都倾向于从移民的角度看待问题。北美小学教育的一个主题就是,殖民者从欧洲过来是为了寻找自由和机会。这诚然是事实,但很少有人告诉我们,为什么大部分人不愿意出远门,即使家里贫困潦倒也不愿意。用今天的话说,为数不多的移民为了自立遭受了难以想象的磨难,只有最具进取心或最绝望的穷人才会移民。17 世纪中叶,被关在伦敦臭名昭著的布赖德维尔感化院的囚犯通过反抗表示“他们不愿意去弗吉尼亚”。1720 年,巴黎的街头发生暴乱,要求释放即将被驱逐到路易斯安纳州的流浪汉、小偷和杀人犯。

\subsection{狭隘的视野}
通信和运输的实际困难,加上在大部分时间和地点内,人们的语言技能都很有限,这使得人类的活动一直处于狭隘和地方性的状态。就在 20 世纪初,在中国的沿海地区还很常见,相隔几英里的村庄说着互不相通的方言。几乎所有地方性的经济组织都会受到市场狭窄和机会丧失的惩罚。由于竞争有限,要素成本就一直很高,人们获得专业技能的机会也很少。全世界大部分的小农都生活在贫困之中,他们收入极低只够勉强维持生计,而且无法受益于外部的资本和高效的保险市场。我们前面探讨过,封闭的乡村生活施加给农民的困难与挑战。即使现在,在我们写作的时候,也至少有 10 亿人——大部分在亚洲和非洲,还在为每天不足1 美元的生活成本而挣扎。

\subsection{所有的政治都是地方性的}
人员和资产的不可移动性塑造了我们看待世界的方式,并超出人们普遍认知的程度。在 20 世纪末地球只是一个小地方,而即使是那些最愿意接受这种观点的人,也还在使用过时的、工业政治的概念进行思考。在 20 世纪 80 年代,环保主义者中间流行一句口号,就突出了这一点:“全球思考,本地行动。”这个口号映射出一种政治逻辑,该逻辑一直建立在地方性权力优势的基础之上。

在过去所有的社会形态中,地方性的思维习惯都是由大政治因素所决定的。阻碍或促进权力行使的所有地形特征都是地方性的。河流、山川、岛屿都是地方的。气候也是地方性的,温度、降雨量和生产条件,会在你爬上或翻过一座山之后发生变化。每一种微生物都在某个地方循环,而不在别处。

所以,地方专制渗透到我们关于社会组织和运作的概念之中,一点也不奇怪。迄今为止,一个群体或另一个群体能够在地方形成暴力垄断,他们的权力优势总是发起于某个地方,然后沿着划定边界的大政治边缘而消逝。这也是为什么从来没有出现过一个世界政府。

地方对权力行使的重要性,很少有人明确指出,但早在 1930 年代,一些想强制重新分配人类活动回报的人,已经开始感觉到地方的影响力在下降。借助现代交通体系的例子,他们看到了社会等级在高收入者和贫困者之间的区分。约翰·多斯·帕索斯(JOHN DOS PASSOS)在他的《大钱》(the Big Money)一书中,捕捉到了这种恐惧:“破产的‘流浪汉’坐在高速路边,饥饿不堪。在他们头顶上,掠过一架可以横跨大陆的飞机,上面坐满了高薪的管理人员。上层阶级坐着飞机,下层阶级坐在路边。他们之间不再有任何联系,他们是两个国家。”这其实从另外一个方面说明了,交通条件的改善,增加了成功人士自由选择地点的可能性,从而削弱了敲诈勒索的杠杆。在地面道路上的流浪者,当然不可能向头顶上的飞行者索求施舍。多斯·帕索斯在 60 年前观察到的这种趋势,现在变得更加明显了。

\subsection{大众运输}
1995 年,每天在世界各地有 100 万人在跨越边界。与过去相比,这是一个惊人的变化。在 20 世纪以前,旅行极为少见,以致于大部分的边界只是名义上的,基本不构成过境的障碍。护照也还是未知事物。远洋轮船、火车及其他交通方式的改进,极大地促进了人员流动。交通与通讯方式的发展,使平民旅行更便宜也更容易;但它也同时加强了国家权力,使之可以对人员流动进行更严格的监管。

电影,尤其是电视的出现,极大地开阔了人们的视野,刺激了旅行和移民。然而,直到今天,社会和政治组织的根基依然固定在地方之上。


\begin{tcolorbox}
\kaishu ……为了避免勇气的失败——历史对此给予过严厉的惩罚,我们必须敢于遵循技术推断得出的所有逻辑结果。
\begin{flushright}
—— 阿瑟·克拉克 ARTHER C. CLARK    
\end{flushright}
\end{tcolorbox}

\section{预期不足的错误}
直到现在,地理还严重约束着人们的想象力。就在 1995 年,一些研究互联网的专家认为,除了作为聊天的电子媒介和色情产品的销售渠道之外,它毫无商业潜力和意义。很多人怀疑网络空间的经济重要性,都是信息时代的“飞艇上校\footnote{指自得自满的军官。}”。他们的自满情绪,可以媲美上世纪 30 年代面临帝国衰落的英国当局。每当发现自己受到威胁时,精英们的第一反应就是否认。这一点,从他们热切期望互联网无所作为就可以看出来,这种态度有时还会得到更懂行的权威的支持。我们前面提到过大卫·克莱恩和丹尼尔·波斯坦的作品《公路勇士》(Road Warriors),他们对网络经济的否定就再一次证明了,技术上博学广识并不等同于深刻理解技术的后果。

即使是过去最专业的技术观察家,也往往看不透新技术带来的影响。1878 年,英国议会委员会审议了托马斯·爱迪生的白炽灯的前景,报告说,爱迪生的发明“对我们跨大西洋的朋友来说足够好了,……但没有实用价值,不值得科学人士关注。”爱迪生本人虽然富有远见,但他也认为自己发明的留声机将主要被商人用于听写。就在莱特兄弟证明飞机会飞的前不久,美国杰出的天文学家西蒙·纽科姆(Simon Newcomb),还权威地证明了比空气重的物质不可能飞行。他总结道:“已知物质、已知机械形式和已知力的形式,不可能组合出一种实用的机器,使人们可以在空中进行长距离的飞行。在作者看来,该证明与对其他所有物理事实的证明一样完整。”而在飞机开始飞行后,另一位著名的天文学家威廉·皮克林(William H. Pickering),又向公众解释为什么商业飞行永远不可能实现:“在公众的脑海中,经常会出现一种图景,类似现代蒸汽船的巨大飞行器,载着无数乘客飞跃大西洋……(然而)很明显,以目前的设备,我们不可能超越汽车或火车的速度。”在前文中我们曾回顾过,另外一则对新技术潜力的错误判断——奔驰公司在 20 世纪初预测,全球的汽车数量将不会超过 100 万辆。他们对汽车的了解可能比任何人都多,但对汽车带来的社会影响,他们的估计同样大错特错。

鉴于这种愚蠢的误解传统,观察家们迟迟不能把握信息技术的意义也就不足为奇了,而它最重要的影响就是对地方专制的超越。第一次,新的技术为经济活动创造了一个无限的、不止于地面的领域。它为探索网络经济的新空间、为“全球思考、全球行动”提供了选择。本章将对此做出阐释。

\section{超越地域局限}
对信息的处理和使用正在快速取代和改变实物产品,成为最重要的利润来源,这意义深远。信息技术将获得收入的可能性与特定地理位置的约束相脱离。产品和服务的价值,越来越多地是由注入其中的思想和知识创造出来的,因此,能够被地方管辖攫取的增值部分越来越小。思想可以在任何地方形成,并以光速在全球传播。这意味着信息经济与工厂经济势必存在巨大的差异。

我们承认某些批评家的观点,在 1998 年你能用互联网做的事确实平淡无奇。毕竟,在网上读一篇关于园艺的文章,或者从遥远的地方购买一箱葡萄酒,并没有什么了不起的革命性。但是,你不能仅仅根据它的早期表现,来判断网络经济的潜力;就像你不可能通过在 1900 年看到的东西,去断言汽车对社会的改变一样。

我们预计,网络经济的发展会经过以下几个阶段:

1. 在信息时代最原始的阶段,网络只是作为一种信息媒介来促进工业时代的日常交易。在这一时期,网络不过是一个外来的传递销售目录的系统。例如,虚拟葡萄园(Virtual Vineyards)是最早的网络商户之一,而它只是在万维网的一个网页上销售葡萄酒。这种交易还没有直接颠覆旧的机构;它使用的还是工业货币,并在可识别的管辖范围内开展。互联网的这些用途还不具备大政治意义上的影响力。

2. 在互联网商业的中间阶段,会以工业时代闻所未闻的方式使用信息技术,如远程会计或医疗诊断。在下文中,我们会举出更多的例子,来说明计算能力的发展带来的全新应用。网络商业的第二阶段,仍将在旧的制度框架内运作,使用国家货币并服从民族国家的管辖。利用网络进行销售的商人只是为了赚取收入,还不会通过网络把钱存入银行。这时期网络交易的收入还会被征税。

3. 这是一个更高级的阶段,标志着向真正的网络商务的转变。交易不仅在网络上进行,还将迁移出民族国家的管辖范围。付款将以网络货币的形式,收入将在网络银行入账,投资将通过网络经济公司。很多交易将免受征税的侵扰。在这个阶段,网络经济开始释放出巨大的冲击力,产生我们之前阐述的那种大政治后果。政府在传统经济领域内保有的权力,将会遭到网络新逻辑的革命。地面以上的监督权力将崩溃,管辖权会发生转移。公司的结构以及工作和就业的性质,都将发生改变。

信息革命可能是有史以来影响最深远的一次经济大转型,以上的阶段概述,只是对这场变革的最简单勾勒。

\section{商业的全球化}
在信息时代,目前大部分的管辖权优势将很快被技术所侵蚀。新型的优势将会出现。通信成本的下降,已经削弱了必须在近距离内做生意的限制。在 1946 年,伦敦的投资者向纽约的经纪人下单,技术上已经是可行的;但只有最大笔和最有赢利可能的交易才会这么做,因为在当时的伦敦和纽约之间,打 3 分钟的电话,费用为 650 美金。而今天它的价格是 0.91 美元。半个世纪以来,一通洲际电话的费用已经下降了 99\%以上。

\subsection{通信融合}
在不久的将来,洲际聊天和本地通话之间的差别将微乎其微。你的电话、电脑和电视之间的区别也可能如此。所有这些都将是互动式的通信设备,相对于它们的功能,从人体工程学的角度更容易做出区分。你可以通过个人电脑上的麦克风和扬声器在网上进行语音对话,或者看电影。你还可以与你的电视对话,并通过电视娱乐媒体提供的网络进行数据交换。总之,工业时代对通信方式的区分将会被打破,成本降会大幅降低,越来越多的服务会按照使用的时间进行收费,而不再区分通信的目的地。世界上所有地方的通信和数据传输费用,将比 1985 年大多数地区的国内通话成本要低得多。

\subsection{无线网络}
未来的低轨道卫星和其他的无线技术,可以直接与你口袋里的电话、便携式电脑或工作站之间来回传输信息,根本不需要与当地的电话运营或电视电缆系统打交道。简单来说,未来的互联网都是无线的。当然,通往这个方向的第一步注定会受到一些阻碍。因为早期无线媒体的数据传输速度相对较慢,可能很难“听到”来自用户设备的微弱信号;其中一些设备会是移动的并由电池供电。不过,随着带宽的增加,这些技术问题都会得到处理和解决。

\subsection{商业无国界}
计算能力的持续发展会带来更好的压缩技术,加速数据的传输。现有的公钥/私钥加密算法,会得到广泛使用。它可以使供应商,如卫星系统,将收费功能内置到服务当中,进一步降低成本。即在提供服务的时候,供应商可以从个人计算机上加载的账户中直接扣费,类似法国电信在巴黎电话亭中使用“智能卡”的形式。

\subsection{手机银行}
不同的是,在不久的将来,你可以随身携带你的公共电话亭,还可以通过各种交易为你的账户赚取积分。你的个人电脑将成为你的银行以及全球货币经纪公司的分支机构,它还相当于你在巴黎购买匿名电话卡的小卖部。小偷撬开智能卡付费电话亭什么也得不到,同样,只会使用撬棍的无赖,得到你的电脑也只能望之兴叹,因为只有那些有能力破解复杂电脑密码的人才能进入它。有了合适的加密,你电脑中的所有东西都不可能被破译或盗用。

到了千年之交,在南极洲以北的任何地方,在可以使用有线或数字蜂窝电话的所有地方,在开通了交互式有线电视系统的地方,在有卫星或其他无线传输方式覆盖的地方,你都可以进行商业交易。通过虚拟现实,你可以随心所欲额地跨越边界,进行通话、传输数据和旅行。通过区号识别通话人所在地的电话号码,可能会被全球性的接入号码所代替;通过这种号码,你可以在地球上的任何地方与任何人联系。请见证铱星公司的实现。

\subsection{听懂中文}
未来,你所能做的不仅仅是交谈和发送传真。有一天,你无需再经过多年的学习,就可以用中文与上海工厂的工头交流。你会不会讲他们的语言或方言,已经不再重要。他讲的是中文,而你听到的则是经过粗略翻译后的英文;同样,他听到的也是被翻译后的中文。随着时间发展,在目前还存在着严重语言和习语障碍的地区,即时翻译的采用,会极大地加剧其间的竞争。当那一天到来时,即使中国政府不希望电话被打出去,也无关紧要了。

\subsection{定制媒体}
随着世界的联系越来越紧密,你将拥有前所未有的机会,去定制你在其中的特殊位置。你从媒体上定期收到的消息,可以是你自己筛选的。现在的大众媒体将变成个性化的媒体。如果你对国际象棋感兴趣,或者喜欢养猫,就可以对晚间新闻进行编程,让它介绍你更感兴趣的猫或者国际象棋的信息。你不用再受丹·拉瑟或 BBC 的摆布,而是可以选择根据你的指令汇总和编辑的新闻。

\subsection{从大规模生产到定制化生产}
如果新闻定制进展缓慢,你可以在万维网上访问一个虚拟目录。当你看到一条很喜欢的长裤,你可以在下单时调整裤腿的宽度。在马来西亚的机器人会进行定制裁剪,并根据从你电脑里扫描到并经网络传输的照片,针对你的提醒进行量身定做。

\subsection{网络经纪}
在未来,你能够使用网络货币进行投资与支付,购买服务和商品。如果你生活在像美国这样的司法管辖区,你的投资可能会受到严格的监管;在这种情况下,你可以选择注册到一个允许公民全方位自由投资的管辖区。无论你住在克利夫兰还是贝洛奥里藏特(巴西东南部城市),你都可以在百慕大、开曼群岛、里约热内卢或布宜诺斯艾利斯开展投资业务。随着网络经济的发展,数字资源的使用范围将不断扩张;无论你在哪里,你都能利用专家系统帮助你管理投资,并通过网络会计和档案管理实时监控你持有资产的走向。

\subsection{虚拟文化}
在你投资之余的闲暇时光,你可以到卢浮宫进行一次虚拟参观。但是,该访问可能需要你支付相当于三分之一版税的费用,给比尔·盖茨或者其他同样高瞻远瞩的人,他们已经购买了参观卢浮宫的虚拟现实权利。当你在怀疑蒙娜丽莎的牙齿是否有问题时,你的电脑可能正在帮你下载熊式一翻译的《西厢记》。在你选定的时间,你的个人通信系统会大声地朗诵它,就像古代的游吟诗人。通过多任务程序,你可以同时进行多项活动。

\subsection{在网上选购管辖区}
如果你从经典作品中受到启发,你可以创办一个虚拟公司,推销著名文学作品衍生出来的戏剧性产品,通过三维视网膜显示器来观看。它的图像不是被投射到空中,而是通过每秒波动 5 万次的低能量激光直接投射到观看者的视网膜上。这项技术已经由华盛顿州西雅图的微视(Micro Vision)公司开发成功,就连天生的盲人都能够看到。在开展这个项目之前,你可以指示你的数字助理,对马来西亚、中国、秘鲁、巴西和捷克等国现行的关于生产设施保护的法律进行研究。

当你选定好一个地点后,只需要一个小时,你就可以在巴哈马注册成立公司,圣乔治信托公司会很荣幸为你提供代办服务。你可以同时在纽芬兰、开曼群岛、乌拉圭、阿根廷和列支敦士登的网络银行开设网络账户,并把公司所有的流动资产存入网络账户中。当任何一个司法管辖区试图撤销你的经营权或者扣押存款时,这些资产将以光速自动转移到另外一个管辖区。

\section{网络体验的提升}
要不了多久,你在网上可以体验的很多活动,在工业时代都是绝不可能的,这不仅仅是因为打破了语言障碍。你的数字助理可以帮助查找发表在匈牙利科学期刊上尚未翻译的论文,但相比你与图书馆员直接交流还是有本质的不同。远在五千英里之外坐着听牛津大学的课程,和睡在距离卡法克斯(牛津市中心)六英里以内的地方听,感觉也是不一样的。在乌拉圭的埃斯特角城(Punte del Este)的聚会上,通过虚拟现实,到蒙特卡洛的巴黎酒店玩轮盘赌,更是一种新奇的体验。

\subsection{在网上看医生}
现实生活中的很多活动,很快将会迁移到网络经济当中,可能比很多专家预期的还要快。新技术会以前所未见的方式组合起来,超越地域的束缚和工业经济的陈旧体制。在未来的某一天,如果你肚子疼,你可以在网上咨询数字医生,这是一个对病症、治疗及开药具有百科全书式知识的专家系统。它通过加密方式访问你的病历,询问你的疼痛发生在饭前还是饭后,是锐感的还是钝感的,是持续性的还是急性的。反正医院的医生问什么问题,它就问你什么问题。然后它会诊断出你是喝了太多的酒,还是喝得不够多,或者把你介绍给网络医疗专家。如果需要动手术的话,在百慕大有网络外科医生,借助专门的设备,可以进行远程的微创手术。

\subsection{生命信息处理}
你可能觉得这听起来像科幻销售。但网络外科手术需要的很多设备组件已经到位了。当你读到这本书的时候,会有更多的部件投入使用。通用电气已经将一种新型核磁共振治疗仪(MRT)引入到了全球 15 家医院。这台机器预计会经过三年的研发阶段,然后就可能迅速普及开来,并成为进行多种手术的标准配置。这是科技改变生活的一个很好的例子。

我们大部分人都熟悉核磁共振成像仪(MRI),它利用核磁共振技术为医生提供软组织图像,帮助进行诊断。它所能提供的图像比 X 射线和超声波更好,已经成为现代诊断技术的重要组成部分,特别是在癌症方面。但到目前,这种仪器存在两个明显的局限,一是管子不能自由地进入病人体内,二是机器的功率有限。

\subsection{网络外科手术}
通用电气重新设计了核磁共振机器,使其既可以用于诊断,也可以用于治疗。机器的功率提高了五倍;检查管被切成了两半,不再是完全封闭的,病人将躺在两个甜甜圈装的装置中间。医生可以在做手术的同时,从仪器中看到他在做什么,而不再是先拍摄图像,然后根据图像进行手术。MRT 将与无创或微创手术相结合。外科医生不用再做很大的手术切口,而是可以使用探针从微小的切口进行操作,并且能在手术时看到探针的成像。医生可以根据图像进行手术,而无需直接观察身体。所以,原则上探针手术能够远程操作。它可以利用激光、低温冷冻或加热的设备非常精准地摧毁肿瘤。

很多现在做不了的手术,利用这种技术都可以进行,特别是神经外科方面的。因为在神经外科手术中,肿瘤往往距离大脑的重要部位非常近。这种技术还允许重复操作,而传统手术留下的创伤是无法重复的,否则会导致难以接受的损伤。

一些研究人员认为,到 2010 年,用于软组织手术的手术刀会成为过时的产物。如果这是真的,手术带来的恐惧和其他后遗症都将会消失。对病患来说,这无疑是大好的消息。现在需要几个小时才能完成、且必须在医院住几天甚至几周时间的手术,未来只需要半个小时就可以了,甚至根本不需要住院。而且,实际上外科医生和病人可能根本不会出现在同一个房间里。这将对医院和外科手术产生什么样的影响呢?

\subsection{更少的显微外科医生,做更多的手术}
这将导致外科手术的一场革命。在培训中,有 1/3 的年轻外科医生掌握不了显微手术的技能,有 1/3 只是勉强能做,剩下的 1/3 则会非常优秀。在向高级外科医生进阶的课程中,也存在类似的比例。所以更少的外科医生,将在更短的时间内,操作更多的手术。医疗保险商和需要做手术的病人,可能都会要求查看每个外科医生的手术结果统计,而这里面的差距会相当大。病人当然希望找手术效果最佳的医生,特别是在病情危急生命时。在某些情况下,最好的外科医生会进行远程手术。整个手术可能会放在另外一个司法管辖区,那里税率比较低,法院也不太支持对医疗事故的过高索赔。

\subsection{数字律师}
在同意实施手术之前,有经验的外科医生会请出数字律师,根据核磁共振图像现实的肿瘤大小和特征,即时起草一份合同,以明确并限制自己的责任。数字律师是一套自动选择合同条款的信息检索系统,它采用神经网络等人工智能程序,定制符合跨国法律要求的私人合同。大多数高额或重要交易的参与者,不仅会付费去寻找合适的商业伙伴,他们也会为交易寻找合适的住所地。

\subsection{紧急咨询}
继续以网络外科为例。信息时代的技术,会为医术最高超的外科医生带来服务溢价。其实,几乎所有的工作都会如此。只要能排上队,患者就愿意支付这样的溢价。但在过去,由于信息的限制,以及在紧急状态下,要在某特定地区找到外科医生所面临的困难,使得外科手术市场非常不完善。信息技术将会弥补和改写这一切。一个病人,可能在 24 小时甚至 45 分钟内就需要做手术,他可以委托数字助理,找到全球前十名可以远程执行这种手术的外科医生,回顾他们在类似病例中的成功率,并从相应的数字助理那里获得他们对特殊病例的报价。而这一切可以在几分钟内完成。因此,最受青睐的前 10\%的外科医生,在全球手术市场中的份额会大大增加。MRT 仪器,加上显微外科技术,将提高他们的工作溢价。而技术能力较低的外科医生,将专注于剩余的本地市场。

这个关乎生死的例子有助于人们理解,当经济从当地的限制中解脱出来以后,会产生怎样的革命性后果。有人可能会反对说,通用的 MRT 仪器不是用于远程操作的。也许吧,但他们没有抓住重点。它或者其他类似的仪器,很快就会用于远程。当外科医生看着屏幕比看着病人能更好地完成手术时,这个医生和他的屏幕在什么地方,就没有我们现在想象的那么重要了。信息技术会使人们在地球上的任何地方进行任何互动,即使是动手术这样精微的活动。越来越多的服务必将被重新配置,以反映这一现实。而在其他对设备精确度要求不高、失败风险较低的领域,网络经济会更加蓬勃迅猛地发展。

\begin{tcolorbox}
\kaishu 福利国家的财政政策使得财富所有者无法保护自己。
\begin{flushright}
—— 艾伦·格林斯潘\footnote{艾伦·格林斯潘(Alan Greenspan),1926年3月6日出生,美国犹太人,美国第十三任联邦储备委员会主席(1987年-2006年),任期跨越6届美国总统。许多人认为他是美国国家经济政策的权威和决定性人物。在他人生的巅峰时刻,他被称为全球的“经济沙皇”、“美元总统”,无论走到哪里,都会在红地毯上受到国家元首一般的接待。}    
\end{flushright}
\end{tcolorbox}


\section{强制力的贬值}
世界上的每一个竞争性领域,包括价值数万亿美元的投资交易活动,为了逃避掠夺性税收,以及通货膨胀对以法币形式持有财富者抽走的税收,会以高压水枪般的动力和速度转移到网络空间。

\subsection{逃离保护性勒索}
只要对信息时代的大政治稍加思考,你就会认识到,最富有的工业国家对其公民施加的掠夺性税收和通货膨胀,在今天看来是理所当然的,但在网络空间的新疆土内,它们将彻底失去竞争力。在千年之交后不久,凡是还按当前的税率缴纳所得税的人,都是出于自由自愿的选择。就像弗雷德里克·莱恩所指出的,历史表明在“边境和公海上,没有人能持久地垄断暴力,当商人可以通过其他方式更廉价地获得保护,他们就会抗拒缴纳高额的苛捐杂税。”网络经济正好提供了这样的选择。在这里,任何政府都无法形成垄断。而且,它所包含的信息技术为金融资产提供的保护,将比大多数政府所能提供的更加便宜、更加有效。

\subsection{复利的黑魔法}
请记住,在 40 年里每年多缴纳 5000 美元的税,按 10\%的资本回报,你的净资产少增加 220 万美元;在 20\%的回报率下,复利损失就高达约 4400 万美元。对高税率国家的高收入者来说,在一生中因掠夺性税收导致的损失,累计起来是很惊人的。大多数人失去的比得到的还要多。

这听起来好像不太可能,但数学不会骗人,你可以用袖珍计算机亲自确认。美国前 1\%的纳税人,平均每年要缴纳超过 12.5 万美元的联邦所得税,而只需要其中的一小部分,即每年 4.5 万美元,就可以在瑞士的私人税务条约下生活,享受可能是世界上最诚实的警察和司法系统提供的法律和秩序。从这个角度看,每年 8万美元的所得税超出这个优越的水平线,完全应该归类为贡品或战利品。考虑到警察的保护本来就是一种公共产品,4.5 万美元用于维护法律和秩序,无疑是一笔不小的费用。理论上,公共产品扩大到更多的使用者,其边际成本为零。瑞士人会很乐意与你协商,然后接受你每年缴纳 4.5 万美元(5 万瑞士法郎)的税款,因为他们会从每个报名的百万富翁身上每年赚取 4.5 万美元的利润。

与瑞士的方案相比,对于一个年平均回报率能达到 20\%的投资者来说,按美国税率缴纳联邦所得税,终生的损失将达到 7.05 亿美元。而且要记住,这假设的是每年缴纳 4.5 万美元。与百慕大这样的避税天堂相比,它的所得税为零,按美国税率交税,终生的损失为 11 亿美元。

你可能会反对说,每年 20\%的回报太高了。你说的也没错。但要看到,近几十年来亚洲的经济增长是很惊人的,世界上的很多投资者都能达到这个回报率,甚至更高。自 1950 年以来,香港房地产的复合回报率每年都在 20\%以上。即使在一些增长速度并不显山露水的经济体,也有机会轻松获得很高的投资回报。过去30 年,如果你在乌拉圭银行有美元存款,就可以把年化 30\%的实际回报率收入囊中。高投资回报率在某地地方更容易实现,但投资老手在好的年头肯定能实现20\%以上的收益,即使他们的表现无法一直与乔治·索罗斯或沃伦·巴菲特相提并论。

显然,你能达到的资本回报率越高,掠夺性税收和资本利得税所产生的机会成本就越大。但是,我们说你的损失会超过你的所得,和你是否能获得超高的回报率没有关系。在美国运作的一些共同基金,半个多世纪以来,平均年收益率只有10\%。如果你是美国收入最高的 1\%,即使你连 10\%的年化收益也做不到,那么你每年支付的超过 4.5 万美元以上部分的所得税,会使你的净资产少增加 3300多万美元。而与一个不征收所得税的司法管辖区相比,你的损失是 5500 万美元。

\subsection{55美元而不是5500万美元}
如果经济学家关于利润最大化的假设是正确的——我们相信一般情况还是对的,那么你可以很有把握地预测,只要有机会大多数人都会采取行动去挽救 5500万美元的损失。我们也是这么想的。当复利的黑魔法在高税率国家的成功人士心中变得愈加清晰,他们就会在不同的司法管辖区之间进行对比选择,就像现在买汽车或者上保险一样。如果你对此表示怀疑,你可以到纽约或多伦多的大街上,随机拦住一些人,问他们会不会为了 5500 万美元搬到百慕大。如果经济学家的利润最大化假设是正确的,我们相信它们一般都是正确的,那么你可以做出的一个比较确定的预测是,如果可以的话大多数人都会采取行动来挽救 5500 万美元。这就是我们的预测。当复利的黑魔法在高税率国家的成功人士心中变得更加清晰时,他们就会开始认真地在不同的司法管辖区之间进行挑选,就像他们现在购买汽车或比较保险单的利率一样。如果你怀疑这一点,只需在纽约或多伦多的大街上随机拦住人们,问他们愿不愿意为了 5500 万美元搬到百慕大。答案不言自明。这就好像问马克吐温,更愿意与赤身裸体的莉莉安·罗素共度良宵,还是与全身戎装的格兰特将军,他可没斟酌太久。

成熟福利国家的居民,尤其是美国的,接受起来可能会比较慢,这只是因为他们还不知道自己所面临的选择。有朝一日,他们会发现的。任何想要追求美好生活的人,包括你,都会看到减少掠夺性税收所导致的损失是无法拒绝的诱惑。而你所要做的,只是把交易转移到网络空间。当然,这在很多司法管辖区是非法的。但旧法律挡不住新技术。20世纪 80 年代,在美国发传真也是违法的。美国邮局认为传真属于一级邮件,它对这种业务拥有古老的垄断权。美国还就此发布了一项法令,重申所有的传真都要送到最近的邮局,与普通邮件一起投递。后来人们发了数十亿条的传真信息,也不知道有没有人遵守这项法令。即使有,恐怕也是稍纵即逝。新兴网络经济带来的运营优势,比绕过邮局发送传真更加强大。

用不了多久,当公钥/私钥加密技术被广泛采用,你就可以在任何地方从事各种经济活动。正如《战略投资》的科技编辑詹姆斯·贝内特(James Bennet)所写的那样:法律的执行——尤其是税法,已经严重依赖对通信和交易的监控。一旦下一个合乎逻辑的发展步骤被采用,即离岸银行网点提供经过 RSA 强加密的电邮沟通服务,这些电邮的账号又是从公钥系统中提取出来的,那么,银行或通信中的金融交易就几乎不可能被监控到。即使税务机关在离岸银行安插内鬼,或者盗取银行记录,也无法识别存款人的身份。

所以,在前所未见的程度上,个人可以自主决定他经济活动的住所地,以及希望支付所得税的额度。在信息时代,许多交易将完全不需要在任何主权的领土范围内进行。它们会越来越多地发生在百慕大、开曼群岛、乌拉圭或类似的司法管辖区内,这些地方不会对商业活动征收所得税或其他高昂的交易费用。

\subsection{从垄断到竞争}
各国政府已经习惯于强行提供“保护服务”。用弗雷德里克·莱恩的话说,这些服务“质量差劲,价格高得离谱”。收费远高于政府服务的实际价值,这种惯例是在几个世纪的垄断之中形成的。就是因为政府垄断或近乎垄断了强制手段,所以任何看上去有支付能力的人都被施加以无情的税收。这种垄断的传统,将与网络商务带来的新的大政治可能,发生深刻的冲突。

加密技术将使网络交易很容易得到保护。一个有效的加密程序,如 PGP,它的成本比一个全面服务的经纪人交易 1 手(100 股)股票的佣金还要低。但是,它却可以使你所有的交易在未来的许多年内,不被政府或盗贼所发现。只需要 55 美元,而不是 5500 万美元,网络经济的参与者就可以获得比工业时期或历史任何时期更好的资产保护。易于使用的加密算法,加上可以自由选择的地面住所,为他们提供了切实有效的保护,使其免受最大的掠夺侵害,即来自于民族国家本身。

这并不是说领土政府将完全被击溃。它们仍然可以利用人身易受伤害的弱点,去收取人头税,甚至直接要挟富人,让他们支付赎金;它们还可能强制征收消费税。但不管如何,政府所能提供的最重要的保护服务,将被放到一个更接近竞争的位置之上。生产者为保护所支付的成本,能够被当局没收和重新分配的部分会减少。

技术创新将使全球的大部分财富,越来越多超越于政府的掌控之外。贸易的风险将因此而下降,就像历史学家珍妮特·阿布·卢格(Janet Abu-Lughod)所说,这将大幅降低“所有成本的占比”;不然的话,这些成本费用将不得不“分配给过境税、权力贡品或敲诈勒索。”在历史上,政府很少真正地受到竞争的制约。在为数不多的几次比较接近的情形中,政府都很弱小,各辖区之间的技术也大同小异。正如莱恩所言,在这种条件下,决定企业家盈利水平的关键因素,在于他们所支付的保护成本的差异。要支付 20 份通行费才能将货物运到市场的中世纪商人,当然无法与那些只需要支付4 份通行费就能把商品送到客户手中的同行相竞争。类似的竞争条件将随着信息时代的到来而回归。企业的盈利能力将再次取决于,与其说是技术优势,不如说是你能否把被迫支付的保护成本降到最低的能力。

这种新的经济动能,将与工业时代遗留下来的、政府对保护服务实行垄断定价的欲望,发生直接的冲突。但喜欢也好,不喜欢也罢,在信息时代的竞争环境中,旧制度将没有生存的余地。任何政府如果还坚持向公民施加其他政府并不要求的重税,那肯定会使利润和财富流失到别的地方。因此,成熟福利国家长期以来的减税失败,将会得到自我修正。课税过重的政府,只会使其治下的所有住所都成为破产的负债。

\begin{tcolorbox}
\kaishu 通过手中的特权,国王能以他喜欢的任何物质和形式制造金钱,并设定它的标准。同样,他也可以改变他金钱的实质与表象,提高或降低它的价值,甚至完全取消和废除它。
\begin{flushright}
—— 摘自 1604 年英国法院判决书  
\end{flushright}
\end{tcolorbox}

\section{铸币税的死亡}
政府不仅会失去对各种形式的收入和资本征税的权力,也必将失去对货币的垄断。从历史来看,大政治转型总是伴随着货币性质的改变。
\begin{itemize}
    \item 硬币的使用,帮助启动了古代经济五百年的扩张周期,在耶稣诞生时达到顶峰,当时的利率是现代时期以前最低的。
    \item 伴随黑暗时代的来临,铸币厂实际上也在不断关闭。罗马硬币虽然还在继续流通,但数量随着贸易量的减少而减少,形成了一种自我加强的螺旋式下降。
    \item 封建革命时期,货币、硬币、汇票和其他商业结算工具被重新引入。特别是来自德国拉梅尔斯堡新矿的欧洲白银产量激增,提高了流通中硬币的数量,润滑了商业发展。
    \item 在信息时代之前,最大的货币革命发生在工业注意来临之际。早期现代国家在火药革命中巩固了权力;随着其控制力的增强,国家权力也牢牢地掌控了货币,并极大地依赖于工业主义的标志性技术——印刷术。印刷机是第一种大规模生产的工具,在现代时期被政府广泛应用于生产纸币。
\end{itemize}

纸币是一种特殊的工业产品。在印刷机被发明之前,复制收据或证书使之成为纸币,显然是不切实际的。在缮写室里的修士,当然不会安心地坐在那里画面值五十镑的纸币。纸币极大地促进了国家的权力,不仅因为货币贬值产生的利润,它还将一种杠杆交到了国家手里,由它来决定什么人可以积累财富。就像阿布·卢格所说:“当国家支持的纸币成为被认可的货币时,与国家机器相对立,或者独立于国家机器去积累资本,就变得非常困难了。”

\section{网络货币}
今天,信息时代的来临,意味着货币性质的又一场革命。网络商务经济的开启,必将带来网络货币。这种新形式的货币,将重新设定游戏的胜率,削弱民族国家对谁可以成为主权个人的决定能力。这一变革的关键来自于信息技术的影响,它\uline{将使财富拥有者从通货膨胀的剥削中解脱出来}。在不远的将来,你在网上进行的任何交易,在下单的同时就可以完成支付,而且是以网络货币的形式。

这种电子形式的货币,将在网络商务中发挥核心作用。它由数百位质数的加密序列组成,它是独一无二的、匿名的、可验证的,它可以容纳最大的交易量,也可以被分割成最小的价值单位。它将可以在一个无国界的、数万亿美元规模的市场上一键交易。

\subsection{不用美元做交易}
无可避免的是,\uline{这种新的网络货币将是非国家化的}。当主权个人可以在一个没有物理实体的领域内跨越国界进行交易时,他们无需再忍受长期以来政府利用通货膨胀贬低其货币价值的行径。何必再忍受呢?对货币的控制,将从权力的殿堂转移到全球市场。任何可以上网的个人或公司,都能轻而易举地从处于贬值风险的货币系统中转移出去。未来不同于今日,使用法币进行交易将完全没必要。实际上,当交易跨越全球,那么在每笔交易中,至少有一方会发现,交易中的货币对他来说并不是法定货币。

\subsection{以物易物将更有市场}
在网络经济中,你可以使用任何你想要的媒介进行交易。已故的诺贝尔经济学奖得主哈耶克认为“货币与非货币之间并没有明确的区别。”他写到:“什么是货币,什么不是货币?尽管我们通常假设二者之间存在一条明显的界限,法律也一般会努力做出这种区分,但就货币活动的因果效应而言,并不存在这样明确的分别。货币系统是一种连续统一体,具有不同程度流动性的物质,或价值不受其他波动影响的物质,在其中发挥货币作用的程度,其实是渐次变化的。”全球计算机网络上的数字货币,将使哈耶克的流动性连续统一体上的每一样物体,都变得更有流动性,除了政府的纸币。这会产生的一个结果就是,以物易物将变得更加适用。越来越多的物品和服务,将通过特定的竞价方式,去交换其他物品和服务。

通过网络,这些潜在的交易会在世界范围内广为宣传,其流动性将成倍增加。以物易物的主要缺点之一,就是需求匹配的问题,一个有特定需求的人很难找到另一个人,刚好可以提供他想要的东西;而后者想要的,前者又刚好可以满足。

原始的易货贸易之所以行不通,是因为在地方市场上,要使交换双方的需求完全匹配,其难度令人望而生畏。现金超越了以物易物的局限性,它的优势会在大部分交易中继续保持。但是,计算能力的大幅提高与网络商业的全球化,弥补了易货贸易的缺陷。当你可以在全球范围内去寻找,而不再局限于你在当地市场时,找到与你需求完全对等的交易伙伴的概率就大大增加了。

\subsection{不受伪造的影响}
纸币无疑还将继续流通,作为一种残留的交换媒介,供穷人和电脑盲使用;但用于高价值交换的货币将被私有化。网络货币不像工业时代的纸币,只能以国家单位来命名。它可能会以黄金的克数或盎司来定价,并且像黄金本身一样容易分割;或者被定义为某种真正的价值存储。即使使用不同的定价方式,或者法币仍然主导着某些交易,网络货币也能为消费者提供更好的服务。日新月异的计算能力,可以将不同交换媒介间价格调整的困难化为无形。在网络货币的每笔交易中,转移的是数百位质数的加密序列。在金本位时期,政府发行的纸币收据可以被随意地复制;而与此不同的是,新的数字金本位或其易货等价物,几乎不可能被伪造,原因就在于一条基本的数学原理:要解开数百位质素的乘积是不可能的。所有的数字收据都将是唯一的,并且是可验证的。

像“英镑”和“比索”这样的传统货币名称,反映了一个事实,它们源自于对特定数量贵金属的称重。曾几何时,英镑的价值就是一磅的纯银。在开始的时候,西方的纸币是贵金属存储的仓库或保险箱收据。发行这些收据的政府很快就发现,它们可以大量地印刷,远远超出它们的金银供应所能实际兑换的数量。这根本毫不费力。任何一个持有金银证书的人,都无法从他的收据中分辨出,有关贵金属实际供应的任何信息。除了序列号之外,所有的收据都是一样的;这种情况不仅吸引了造假钞的,也吸引了政客和银行家,他们从货币供应量的膨胀之中谋取暴利。

网络货币不可能以这种方式造假,无论是官方还是非官方的手段。数字收据的可验证性,杜绝了利用通货膨胀抽取财富的经典权宜之计。信息时代的新型数字货币,把对交易媒介的控制权交还到希望保留财富的所有者手中,而不是交给只想抢夺财富的民族国家。

\subsection{自由货币的交易成本}
使用这种新型的网络货币,会使你从国家权力中得到极大的解脱。在过去的半个世纪,全世界的民族国家,在维持其货币价值方面都乏善可陈;在此前的篇章,我们引用过它们的失败记录。自二战以来,德国马克因通货膨胀而贬值的比例最小。但即便如此,从 1949 年 1 月 1 日至 1995 年 6 月底,马克的价值消失了 71\%。

在此期间,作为世界的储备货币,美元贬值了 84\%。这很好地衡量了,利用对领土内法币的垄断地位,政府所攫取的财富比例。需要注意的是,货币并不是天生就应该贬值,名义生活费用也不是必然要每年上升。恰恰相反。保持储蓄的购买力在技术上根本没有难度。看一下黄金的长期购买力,你就能发现这一点。从 1949 年 1 月 1 日到 1995 年 6 月底,最好的国家货币损失了将近 3/4 的价值,而黄金的购买力实际上是上升的。罗伊·雅斯特罗姆(Roy W. Jastrom)教授在他的《黄金常数》(The Golden Constant)一书中记载,根据可靠的价格记录——在英国可以追溯到 1560 年,黄金一直保持着它的购买力,波动幅度非常小。

在没有紧急军事需求的情况下,与黄金挂钩的国家货币也一直保持着购买力。在相对和平的 19 世纪,英镑的价值不降反升,尽管它与黄金的联系很弱。在信息时代新的大政治条件下,切实可行货币制度的不是金本位这样的弱联系,而是一种强联系;消费者手中更高质量的信息和计算资源,将第一次强化这种联系。


\begin{tcolorbox}
\kaishu 如果他们达不到人们的预期,就会面临迅速失去全部业务的威胁(而任何政府组织都一定会滥用这个机会操纵原材料价格!),相比任何设计出来的制度,这种威胁提供了防范政府垄断的更强有力的保障。
\begin{flushright}
—— 弗里德里希·冯·哈耶克  
\end{flushright}
\end{tcolorbox}

\subsection{货币私有化}
弗里德里希·冯·哈耶克在 1976 年指出,使用相互竞争的私人货币将消除通货膨胀。哈耶克说,法币迫使一个管辖区内的人们接受通货膨胀的货币;没有了法币,通过市场竞争,私人货币的发行者将不得不维护其交换媒介的价值。任何不能保持购买力的私人货币,很快就会失去客户。加密网络现金的发展,将以生动的形式使哈耶克的逻辑成为现实。

被人们称为“自由银行”的理论,并不是假设性的学术猜测。从 18 世纪初到 1844年间,私人发行的竞争性货币一直在苏格兰流通。在此期间,苏格兰并没有中央银行;银行业务的准入规定和限制很少。私人银行接受存款,并发行以自家金条为支撑的私人货币。根据劳伦斯·怀特(Lawrence White)教授的记录,这种制度运行良好。相比同期英国的银行和货币系统,它更加稳定,通货膨胀率更低。

金融时报的迈克尔·普劳斯(Michael Prowse)总结了苏格兰自由银行的经验:“几乎没有欺诈,也没有证据表明纸币发行量过大。银行通常不会持有过多或不足的准备金。银行挤兑非常罕见,也没有传染性。自由银行得到了公民的尊重,为经济增长打下了坚实的基础,在这一时期的大部分时间里,苏格兰的经济增长速度都超过英格兰。”在 18 和 19 世纪的技术条件下行之有效的东西,透过 21 世纪的技术,会发挥出更好的效果。很快你就可以使用由私人公司发行的数字货币进行交易\footnote{‌USDC的发行公司是Circle‌,该公司成立于2013年,总部位于美国,是全球第二大稳定币USDC的主要发行方和运营管理者。‌‌}。它的发行方式,就像美国运通发行旅行支票作为现金收据一样。一家比政府更有声望的机构,比如领先的矿业集团或瑞士银行公司,可以用大量的黄金或独一无二的金条做储备,来制作加密收据,通过分子签名做识别,甚至可以刻上全息图像。用这些收据作为货币进行交易,几乎不可能被伪造或发生通货膨胀。

在过去,把黄金直接用作货币,受到很多实际问题的困扰;新的数字黄金将会克服这些问题。处理大量的黄金,将不再是一件麻烦、繁琐或危险的事情;数字收据也不会太重而难以携带。实际上,它们唯一的物理存在就是一种复杂的计算机代码。要把这种数字收据分割成很小的单位,去支付极低数额的交易,也根本不是难事。一块实物黄金,如果被分割到小得可以购买一块口香糖,那么它就很容易丢失,或者与能购买两块口香糖的黄金相混淆。但计算机就很容易区分这些数字黄金的面额,在计算机看来,它们的大小差别,就像一只花栗鼠和一头犀牛。

数字货币的小额支付能力,将催生出以往不可能存在的商业模式,那就是对低价值信息的组织和分发。这些信息的供应商,现在是通过直接借记版税获得补偿,克服了以前令人生畏的交易成本。要知道,当计费成本超过交易价值时,可能就不会发生交易了。网络货币为费用很低的同步计费提供了便利,在这种计费方式中,账户可以在使用中被扣费。在前文中我们举了一个例子,想象你通过虚拟现实参观卢浮宫,你可能会向比尔·盖茨或者其他拥有相关权利的人,支付相当于1/3 便士的版税。你可以把这个数字乘以一千倍。虚拟现实将创造出近乎无限的许可机会,而这些机会只要求微额的版税费用。有一天,你可能会通过虚拟现实,重温 1969 年世界杯系列赛的第三场比赛;而为了使虚拟现实更加逼真,其中使用了某些球员的照片,你就可以用网络货币,向这些球员支付微额的版税。

\section{彻底消灭通货膨胀}
微支付固然是很可能的,但新型数字货币带来的最重要的结果,将是通货膨胀的结束和金融系统的去杠杆化。它的经济影响将是深远的。在《血流成河》与《大清算》中,我们论述过,20 世纪通货膨胀的兴趣与世界权力的平衡密切相关。

暴力的回报越来越高,刺激了军事开支的急剧增加,反过来又促使权力更加积极地征用财富。各国政府发现,它们可以有效地持有本国货币的人征收年度财富税。这种税负也可以看作是一种交易费用,因为货币的使用者通过发行者提供的便利形式来保存他们的财富。

把通货膨胀理解成因为持有货币的便利性而收取的交易费用,可能有点不太寻常,但仔细考虑一下,就说得通了。在工业时代,我们习惯认为货币的供应是一种无须直接付费的服务,以致于很容易忘记美元、比索、英镑、法郎的发行者,也就是政府,不仅要我们付费,而且价码还不低,而它们的手段就是通货膨胀。

这种通货膨胀式的货币交易费,它的费率在过去的半个世纪里,从德国马克每年2.7\%的低点,到接近 100\%的高危表现,不一而足。例如,从 1960 年到 1991 年,在梅内姆总统启动阿根廷货币委员会的改革后,通货膨胀使阿根廷的货币上出现了连续 17 个 0。世界上所有的财富,如果在 1960 年都兑换成阿根廷比索并且埋起来的话,那么到 1991 年,就根本不值得再费气力把它们挖出来了。

阿根廷的例子就是下一个千年的先行指标。未来的货币不会再陷入通货膨胀,因为其他的民族国家无法再逃脱通胀的惩罚,阿根廷也一样。不同的是,相比阿根廷货币委员会的自动发币系统,分散在网络上的私人货币更不容易受到政策逆转的影响;前者还可能受到其他国家输入的信贷收缩的破坏。而由于市场竞争的压力,私人发行的货币不会出现通货膨胀。

通货膨胀一旦死亡,货币发行垄断者以前从中收获的隐形利益,将统统被带走。如果货币发行的隐形利益都被消灭了,那么就需要一种新的支付方式,来直接补偿货币发行者。因此,使用新的货币体系,可能要付出更明确的交易成本,可能是每年 1\%左右的费用。与民族国家每年 2.7\%到 99\%的通胀惩罚相比,这是一个很小的代价。更重要的是,随着垄断的削弱以及世界范围内竞争的加剧,未来的总体价格还可能进一步下降。

\subsection{收缩杠杆}
数字货币的出现,不仅将一劳永逸地战胜通货膨胀,还将收缩世界银行系统的杠杆。在市场全球化之下,世界各地的人民,都可以绕过监管当局,直接通过互联网转移资金,这是划时代的历史进步。任何政府都将无力监管。当政府不能通过印钞使货币贬值,也不能通过圈养银行系统随意扩大信贷来欺骗储户时,它们就会失去间接支配资源的大部分能力。

\subsection{更高的利率}
大部分西方政府将因此而陷入严重的困境。它们的税收将锐减,货币体系的杠杆作用接近清零。而与此同时,它们还保有着无资金准备的负债和对社会支出的膨胀预期,这些都是从工业时代继承下来的。可以预见,一场剧烈的财政危机必将发生,并伴随着惨烈的社会副作用;在后面的章节中,我们会对此进行讨论。这场转型危机的经济后果之一,可能是实际利率的一次性飙升。随着在旧制度下签订的长期负债被清算,优惠信贷枯竭,债务人将受到挤压。

\section{因竞争而改变}
面对货币垄断的激烈竞争,各国政府可能收紧信贷,并向本国货币储蓄者提供更高的实际收益率,与收费网络货币打价格战。为了应对私人货币,一些政府甚至会诉诸另外一个权宜之计,即尝试将黄金再货币化。它们的如意算盘是,与其让本国货币完全被网络货币所取代,不如恢复 19 世纪的金本位制度,虽然在控制上会松散一些,但还是能获得更高的铸币税利润。

然而,不是所有的政府都会做出同样的反应。在那些计算机使用率及网络参与度较低的国家,在网络经济的早期阶段,当地政府可能会延续老式的高通胀表现。但这并不会使政府从富人那里抽走财富,只能让它从没多少钱也上不了网的人那里榨取资源。不过,使用这种策略的政府,也可能会在国际上借入网络货币。

还有一些政府会抓住信息经济创造的机会,为当地的网络货币交易提供便利。这些管辖区会率先承认数字签名的有效性,并支持地方法院强制执行拒不履行的网络债务;长期借贷资本会因此而大量地涌入,促进其经济发展。显然,如果地方法院会惩罚网络货币,或允许债务人违约且无法追偿,在这样的地方就不会出现网络货币的长期信贷。

\subsection{收益率差距}
信贷危机、各国货币当局的竞争性调整、早期网络货币贷款的过渡性障碍等等,这些因素综合在一起,会在信息经济初期形成收益率差距。网络货币的利率可能会低于国家货币,并且要求明确的交易成本。这显然是持有数字货币的劣势,但考虑到它可以加强保护资产免受掠夺性税收和通货膨胀的破坏,这些劣势也足以被抵消了。而且网络货币很可能会与黄金挂钩,它也将从黄金的升值中受益。

不管政府的哪一种替代性政策占据主导地位,黄金的价格相对其他商品都可以大幅上涨。为什么这么说?因为在通货紧缩时期,黄金的实际价格基本都会上涨。毕竟,通货紧缩反映了流动性的短缺;而黄金是流动性的终极形式。

\subsection{工业时代的通货紧缩}
较高的实际利率将刺激高成本、非生产性的活动进入清算,并暂时减少消费。在《血流成河》与《大清算》中,我们探讨了信贷周期的逻辑以及应对的方法,这里不再赘述。可以说,通货紧缩的状况会持续一段时间,它所造成的不利影响,对于北美和西欧等高成本工业经济体更加严重,相比亚洲和拉美等低成本国家。

\subsection{长远来看,利率会更低}
网络经济的出现,在早期可能会促使提高利率,但长期来看,结果将正好相反。

随着资源摆脱政府的控制,储户的税后收益会大幅增加。资源的利用效率将急剧攀升,资本得到解放,可以在全球范围内寻求最高的回报,这将迅速弥补转型危机初期的产出损失。

\subsection{投资者对资本的控制}
传统的思想家在审查我们的论点后,可能会得出这样的结论:如果主要民族国家的收入再分配系统崩溃,将会导致世界经济大崩盘。不要相信他们。我们并不否认转型危机很可能会发生。但是,认为国家可以通过大规模的资源再分配来改善经济,这样的观点已经不合时宜了。它是一种信仰,大致相当于中世纪末期普遍存在的迷信,认为禁食和鞭笞是利国利民的行为。请不要忘记,政府总是大规模地浪费资源,而浪费资源就会使你陷入贫穷。在历史上,每当政府占用的收入由真正的人才掌控时,资源的利用效率就会显著提高。

数百亿、数千亿美元的资产,将会控制在数十万、数百万主权个人的手里。实践将会证明,这些世界财富的新管家,在资源利用和投资配置方面,比传统的政客更有能力。在信息时代的大政治条件下,资本最终将由最具才干的投资者和企业家掌控,而不再是暴力专家,这是史无前例的。我们可以合理预期,这种分散的、由市场驱动的投资,相对民族国家时代由政治驱动的预算分配产生的微薄回报,前者的收益可能是后者的两倍或三倍。在 20 世纪的最后几十年里,政府投资收益为负数的例子非常普遍。在 1992 年 11 月出版的《大清算》修订版中,我们引用了俄罗斯的官方统计数字,表明俄罗斯的整个经济“只值 300 亿美元,还不到其原材料投入价值的 1/3。言下之意,如果彻底关闭国内的制造业和服务业,俄罗斯经济产出的价值将增加三倍以上。它们不但没有增加价值,反而还减损了。”诚然,共产主义崩溃后的俄罗斯是一个极端的例子,但有充分的证据表明,削弱国家对资源的控制,往往会提高经济的效率。《经济学人》杂志列举的增长率显示,经济的自由度与增长率密切相关,最自由的国家经济增长速度最快。而信息时代的网络经济,将比历史上的任何商业领域都更加自由。因此,我们有理由预期,网络经济很快将成为新千年里最重要的新型经济。它的成功将会吸引世界各地的新用户不断加入,就像传真机的广泛普及,使传真对那些没有用过的人越来越有吸引力。更加重要的是,摆脱了掠夺性暴力的侵扰,网络经济的复合增长率将远高于民族国家主导的传统经济。

如果要预测政府的垄断税收与通货膨胀崩溃之后,会对经济产生什么样的冲击,那么上述可能就是最重要的一点。抛开可能持续几十年的困难的过渡时期,全球经济的长期前景值得高度看好。每当条件允许,人们可以降低保护成本,并尽量减少给垄断暴力者的朝贡时,经济就会大幅增长。就像莱恩所说;“我认为,在大多数经济增长时期,如果说有一个因素起到了最重要的作用,那就是降低了用于战争和警察的资源比例。”减少被掠夺的资源,以及对这种掠夺的生存依赖,会释放出巨大的经济效益。如果在竞争的基础上对保护服务进行定价,地方垄断势力依靠价格和质量去争夺客户,也同样可能获得巨大的经济效益。我们可以预期,未来的税率会大大降低,政治活动损耗的资源和能量将减少,政客们也不会再享受以前的巨大红利。

选民们是否愿意放弃他们已经习惯的政治福利?这个问题我们会在其他地方详细讨论。一个简单的答案是,我们可能别无选择。没有人会抗议下雨,也没有人会抗议干旱,不管它们会对经济造成怎样的伤害。再丧心病狂的罪犯,也不会绑架一个乞丐,以死亡为要挟去索要巨额的赎金。当政客们不可能再攫取资源进行重新分配,也许公众会以理性的方式作出回应。他们会忘掉政治,就像中世纪结束之后,那些好心的人们也不再举行忏悔游行一样。