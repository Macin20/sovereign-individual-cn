\chapter[大政治变革]{历史视野下的大政治变革}

\begin{tcolorbox}
\kaishu 在历史上,如同在自然界中,生与死是同样平衡的。
\begin{flushright}
    —— 约翰·惠泽加(JOHAN HUIZINGA)
\end{flushright}
\end{tcolorbox}

\section{现代世界的没落}
在我们看来,你们正在亲眼目睹现代时期的衰亡;驱动它的是一种无情且隐蔽的逻辑。人们普遍还没有认识到,CNN 电视和报纸更没有告诉你们,下一个千年将不再是“现代”。我们这么说,并不是想暗示未来是野蛮或落后的,虽然这种可能性很大;我们是想强调,现在开启的历史阶段与你出生时的阶段,有本质的不同。

变革正在到来。正如农业社会和狩猎采集的部落大相径庭,工业社会与封建社会或自耕农体系有天壤之别,即将到来的新世界,与旧世界的一切也天差地别。

在新的千年里,政治和经济不会再像近代的几个世纪那样,组织规模庞大,笼罩在民族国家的统治之下。给你带来了世界大战、流水线、社保、所得税、除臭剂和烤面包机的文明,正在走向死亡。除臭剂和烤面包机可能还有机会幸存,其他的就没戏了。就像一个古代的强人,民族国家的倒计时以年和日为单位,而不是以世纪或年代为单位。

政府管理和强制的权力,已经大大丧失。共产主义的崩溃,标志着一个长达五世纪的周期的结束;在该周期中,政府权力的规模压倒了组织的效率。那是个暴力回报率很高且不断上升的时期,现在不是了。在世界史的维度,一个阶段性的转变正在开始。实际上,在下一个千年,当未来的吉本(历史学家)编著“曾经的现代时期”的衰亡史时,他可能会宣布,在你读到我们这本书的时候,这个时期就已经结束了。他可能会像我们一样认为,回首往昔,现代时期终结于 1989 年柏林墙的倒塌,或者说是 1991 年苏联的灭亡。这两者中的任何一件,都可以作为文明演进中的代表性事件,定义我们所知的“现代时期”的结束。

人类历史的第四阶段正呼之欲出,而它最难预测的部分,可能是它的称呼。你可以叫它“后现代”、“网络社会”或“信息时代”,或者你自己起一个名字。没人会知道,这个新的历史阶段,会被贴上一个什么样的概念或绰号。

我们甚至也不知道,这刚刚过去的五百年,在以后是不是还会被继续称作“现代”。

如果未来的历史学家理解词语的引申含义的话,可能就不会这么叫了。它更贴切的名称,应该是“国家时代”或者“暴力时代”。但是,这样的称谓,不符合目前定义历史时代的时间谱系。根据《牛津英语词典》的解释,“现代”是关于现在或近代,区别于遥远的过去;在历史应用中,通常是指中世纪以后的时期。

西方人只有发现中世纪已经结束了,他们才有可能意识到自己是“现代人”。在公元 1500 年之前,没有人会认为,封建时代是西方文明的“中间”阶段,是中世纪。道理很明显:一个时代被认为是夹在两个时代中间,如果这么说没错的话,那么在此之前,这个时代肯定已经结束了。那些生活在封建时代的人,不可能想象到,自己是活在古代文明和现代文明的中间阶段;直到他们意识到,不仅中世纪已经结束,而且中世纪文明与黑暗时代或古代文明,有很大的不同。

人类的文化中存在着盲点。关于生活的最大边界的变化,尤其是发生在我们身上的,我们没有相应的词汇去描述。自从摩西时代以来,已经发生了很多剧变,但只有少数异端人士殚精竭虑,思考了人类文明从一个阶段到另一个阶段的转型是怎么发生的。它们是怎么被触发的?它们有什么共同点?有哪些模式可以帮助你判断,它们何时发生、何时结束?英国或美国什么时候会走到尽头?这些问题,你很难通过常规的思维找到答案。

\subsection{预言未来是人类文化的禁忌}
从既有系统的外面看问题,就像一个舞台工作人员试图和剧中人强行对话一样,它违反了那些保持系统运作的惯例。每一种社会秩序,都含有一个重要的禁忌,那就是,生活在其中的人不能思考该秩序将如何结束,以及取而代之的新秩序应该使用什么规则。隐藏的含义就是,不管既存的是什么系统,它都是最后的或唯一的系统,将永世不衰,万寿无疆。这话说得有点直白。读过历史的人,如果在书中看到这样的观点,恐怕很少有谁会认为这种假设是现实的。然而,这就是统治世界的传统。每一种社会体系,不管它对权力的掌控有多强,或者多弱,都会装出一副千秋万代、永不可能被取代的样子。它们一言九鼎,或者定于一尊。原始人认为,他们的系统,就是组织人类生活的唯一可能。经济上更加复杂的系统,也具有了一定的历史感,往往会把自己置于人类发展的顶点。无论是紫禁城里的中国官吏,还是克里姆林宫里的马克思主义干部,或者是华盛顿的众议院议员,当权者要么根本无视历史,要么认为自己登上了历史的巅峰,相比既往的众生以及未来的先锋,他都要更加优越。

出于实际的考虑,这么做也没错。一个体系越是明显地接近尾声,人们就越不愿意遵守它的法律。所以,任何社会组织,都会倾向去阻止或淡化那些预测它灭亡的分析。仅仅这一点就确保了,当重大的历史变革正在发生时,很少有人能够注意到。如果你对未来一无所知,那你可以放心,即使剧变当前,正统的思想家们也会视而不见,更不会通知你。

你不可能依靠传统的信息源,等待它们向你发出客观和及时的警告,告诉你世界将如何变化,以及为什么会变。如果你想知道当下正在发生什么,除了自己去弄清楚,基本没有别的选择。

\subsection{超越显而易见}
这意味着你要有能力穿越显见的事实。历史记录表明,即使是那些事后看来真实确凿、无可否认的变革,在它发生后的几十年甚至几个世纪内,都不被人们接受和承认。想一下罗马的衰亡,这可能是基督教时代第一个千年里最重要的历史进程。但是,在罗马灭亡后很长一段时间内,关于它仍然幸存的虚构故事,还在不断地向公众散布,就像经过防腐处理的列宁的尸体。那些对“新闻”的理解建立在官员的糊弄之上的人,要等到很久以后,当那些信息与现实已毫不相干,他们才会了解罗马已不复存在。

之所以会这样,不仅仅是因为古代的通讯不够发达。即使 CNN 能够奇迹般地在古罗马开展业务,在公元 6 年的 9 月,用录像带记录下一切,情况也好不到哪里去。当时,西罗马的末代皇帝,罗慕路斯·奥古斯都在拉文纳被俘,被迫退居到坎帕尼亚的一栋别墅里,在那里领着奉养金。即使沃尔夫·布里泽\footnote{Wolf Blitzer,他是CNN《战情室》(The Situation Room)节目的联合主持人,与帕梅拉·布朗共同主持该栏目。该节目为观众深入解析当日政治、国际事务及突发新闻事件。布里泽已为CNN服务长达35年。}在 476 年用微型摄像机记录了这一历史事件,但要把它定性为罗马帝国结束的标志,不管是布里泽还是其他人,恐怕都不敢这么说。然则,后世的历史学家正是这么定性的。

CNN 的编辑应该也不会批准一个头条新闻说,“罗马亡于今夜”。当时掌权的人,不会承认罗马已经灭亡。兜售“新闻”的人,基本不会陷入到可能损害自身利益的争论中。他们也许是偏颇的,甚至极度偏颇,但一般不会发出可能导致读者取消订阅的报道。这就是为什么,即使在技术上有可能,也不会有什么人报道罗马的灭亡。专家们会站出来说,讨论罗马的衰落,简直大逆不道。如果不附和这样的说法,就会对生意不利,或者对报道者的健康不利。毕竟,在五世纪的罗马,掌权者是野蛮人,他们不认为罗马已死。

这不仅因为当局会威胁,“不要报道,否则我们就杀了你。”还有部分原因是,在五世纪的后期,罗马已经堕落不堪、腐烂透顶,以致于当时活着的人对它的“衰落”根本没有感觉。事实上,过了一代人之后,马尔切利努斯伯爵(CountMarcellinus)才首次提出,“西罗马帝国和这位奥古斯都一起灭亡了。”又过了几十年,也许几个世纪,人们才普遍承认,西罗马帝国已经彻底作古了。当然了,查理曼大帝在公元 800 年的时候,还认为他是合法的罗马皇帝。

重点并不在于说,查理曼大帝和所有在 476 年之后还以传统思维理解罗马帝国的人都是蠢货。恰恰相反。对历史发展的定性经常是模糊的。当占据权力主导地位的机构被迫讨价还价,去强化一个对它们有利的结论时,即使这个结论是基于伪装和糊弄,也只有少数性格刚硬、观念强烈的人,才敢站出来反驳。把你自己放到五世纪末罗马人的位置上,就很容易想象到,一个“啥都没变”的结论更加诱人;这种态度还很乐观。如果不这么想,那感觉就很可怕了。既然一个让人岁月静好的结论就在眼前,为什么还要自讨苦吃,做出一个令人恐惧的推论呢?毕竟,有一点会被人们拿出来证明,一切事务还在照常进行。过去的人也这么做过。那就是罗马的军队,特别是边境驻军,几个世纪以来就处于野蛮化的状态。

到公元 3 世纪,军队宣告新皇登基已经是很常规的操作。到 4 世纪,甚至军官都日耳曼化了,而且大部分是文盲。在罗慕路斯·奥古斯都被赶下皇位之前,暴力推翻皇权的事,已发生过很多次。那是个混乱的年代,在罗慕路斯的同代人看来,他的下台和其他的动乱并没什么不同;而且他是带着奉养金离开的。就是他领奉养金这件事,让人们觉得整个体系还依然健在,虽然他领了没多久就被杀死了。

在一个乐观主义者看来,废黜罗慕路斯·奥古斯都的奥多亚塞,并没有摧毁罗马帝国,而是重新统一了它。身为阿提拉的随从艾德康的儿子,奥多亚塞是一个聪明人。他没有自宣为王,而是召集了元老院,说服了言听计从的元老院成员,将皇帝职位及整个帝国的主权,交给远在拜占庭的东罗马皇帝芝诺,他只作为一个西罗马的贵族,帮助芝诺管理意大利。

正如威尔·杜兰特\footnote{威尔·杜兰特,1885——1981,男,美国著名学者,普利策奖(1968)和自由勋章(1977)获得者,主要著作有《世界文明史》等。}(Will Durant)在《文明的故事》\footnote{图书链接:https://book.douban.com/subject/36974911/ 。}(The Story of Civilization)中写道,这些变化似乎算不上是“罗马的衰亡”,而只是“国家舞台上微不足道的表面变化”。当罗马衰落时,奥多亚瑟说,罗马经久不衰。和几乎所有人一样,他热衷于假装什么都没变,一如既往。他们知道,“罗马的荣耀”远比取而代之的野蛮人的好,野蛮人自己也这么认为。所以,就像查尔斯·普雷威特·奥顿\footnote{C. W. Previte-Orton,英国的中世纪历史学家。}在他的《剑桥中世纪简史》中所说,五世纪末,当“罗马皇帝被野蛮的日耳曼国王取代”后,是一个“不断虚构(make-believe)”的时代。

\subsection{“不断虚构,制造相信”}
这种“虚构”是为了维护旧制度的门面,虽然它的本质已经“被野蛮人败坏了”。

当最后一个罗马皇帝被野蛮人的“中尉”所取代,政府的规制并未改变。元老院还在开会。“地方行政官员及其他高级职位继续存在,并由有声望的罗马人担任。”执政官依然每年提名一次。“罗马的民政管理部门完好无损地留存”;确实,在某些方面,它一直完整存续到十世纪末封建主义的诞生。在公共场合,使用的依然是旧帝国的徽章。基督教依然是国教。野蛮人也一直假装效忠于君士坦丁堡的东方皇帝,以及罗马法的传统。但事实上,用杜兰特的话说,“伟大帝国在西方已经不复存在了。”

\subsection{那又如何呢?}
在你思考当下的世界时,参考罗马衰亡的远鉴,是很有必要的,其中有诸多关联之处。大都数探讨未来的书,其实都在谈论现在。我们试图弥补这一缺陷,使这本未来之书首先成为过去之书。我们相信,如果能用过去的实例,阐明大政治理论的重要观点,也就是关于暴力的逻辑;那么,就能帮助你更准确地预测未来。

历史是一位神奇的老师,它讲的比我们编的都要有意思;其中有关罗马衰亡的故事,更加值得品味和琢磨。人们记录了许多重要的教训,对你认识信息时代的变迁,可能很有启迪。

首先,在整个人类历史上,罗马的消亡,更加生动鲜明地说明了,在重大转型时期,当政府的规模崩溃时会导致什么后果。公元 1000 年的转折也涉及中央权威的崩盘,不过这次崩盘,也提高了经济活动的复杂性和范畴。15 世纪末的火药革命,同样给权力机构带来了巨大的变化,但这些变化增加而非减小了统治的规模。而今天,一千年来第一次出现,西方的大政治环境,正在破坏政府、企业集团、工会,以及其他大规模运行的组织。

当然了,对比罗马帝国的末期与信息时代的前夜,引发政府规模崩溃的原因,是非常不同的。罗马之所以灭亡,部分原因很简单,就是它的扩张超出了暴力经济能维持的规模。帝国戍边的成本,使古代农业所能形成的经济优势荡然无存。资助军事行动所需的税收和监管负担不断攀升,经济被掏空。腐败遍地。根据历史学家拉姆齐·迈克穆伦\footnote{耶鲁大学历史学家,他在 1988 年出版的经典著作《腐败与罗马的衰亡》(Corruption and the Decline of Rome)至今仍具有深刻启示。}的描述,当时军事指挥官的大部分精力,都花在了“以职权谋取非法利益”上。他们的手段就是敲诈平民,这正是四世纪的观察家辛奈西斯\footnote{古代利比亚昔兰尼加托勒密希腊人主教、哲学家、作家和诗人,生活于公元4世纪至5世纪,希腊化古埃及新柏拉图主义学者希帕提娅的学生。他致力于调和柏拉图主义和基督教思想,曾在君士坦丁堡生活了3年并参与了拜占庭帝国的权力斗争。}(Synesius)所描述的“和平时期的战争,主要由士兵的不守军纪和军官的贪婪腐败所致,它比野蛮人的战争还要糟糕。”罗马崩溃的另一个重要因素,是安东尼时期的瘟疫导致了人口赤字。在当时的很多地区,人口都大量减少,这显然造成了罗马在经济和军事上的衰落。这样的情况今天也没有发生,至少目前还没有。从更长远的角度看,在接下来的千年里,新的“瘟疫”之灾将加剧技术转型带来的挑战。20 世纪人口的空前膨胀,为快速变异的微型寄生虫创造了理想的目标。担心埃博拉或者其他类似的病毒攻击大城市的人口,并非杞人忧天。但是,我们这本书并不是要思考人类和疾病的共同演化,尽管这是个有趣的话题。在这个历史的关口,我们要探讨的不是罗马为什么消亡;甚至也不是要分析,导致罗马灭亡的因素有哪些同样威胁着今天的世界。

我们要谈的与此不同,我们是想搞清楚,每当重大的历史变革发生时,人们是如何认知的,或者更准确地说,是如何错误认知的。

在某种程度上,古今中外的人都有点保守;人们不愿意从某些角度去思考问题,如解散悠久的社会习俗、推翻公认的体制、违抗他们赖以生存的法律和价值观。也很少人会想到,在气候、技术或其他变量上,一些看上去非常细微的变化,就能够切断他们与父辈世界的联系。罗马人不愿意承认,周遭的世界已经改变了。

我们也是如此。

然而,不管你是否承认,我们都正在经历一场历史性的变革,它将革新人们组织生计和保护自我的方式,重塑整个社会。其意义之深远,变化之深刻,你要想理解它,就不能把任何事情视为是理所当然的,天经地义的。无论何时何地,都会有人想让你相信,即将到来的信息时代,和你成长的工业时代没什么两样。我们对此深表怀疑。\uline{在我们看来,微处理技术,将溶解砖头中的砂浆,颠覆暴力的逻辑,必然改变人们谋生与自卫的方式。}然而,对这些变化的必然性和进步性,人们倾向于淡化,或者闭口不谈;好像历史如何演进这件事,只能由工业机构主导和说了算。

\subsection{巨大的幻觉}
有些作者,在很多方面比我们学识渊博,但在思考未来的问题上,却会将你引入歧途,因为他们对社会运转的认识过于肤浅。例如,大卫·克莱恩和丹尼尔·伯斯汀(David Kline and Daniel Burstein)经过精心研究,写了一本专著,名为《公路勇士:信息高速公上的梦想与噩梦》(Road Warrior)。书中充满了令人赞叹的细节,但是,这些细节却是用来论证一种幻觉,那就是,“公民可以自觉地、齐心协力地行动,去塑造他们周围自发的经济和自然进程”。举一个不是很明显的例子,这其实相当于在说,如果当时每个人都重拾骑士精神,封建主义是可以存活下去的。在 15 世纪末的法庭上,没有人会反对这样的观点,反对它甚至被视为是异端。但是,这依然是一种彻头彻尾的误导,就像蛇想把明天的身体塞进昨天的皮囊。

变化发生的深层原因,正是那些不受意识控制的因素,这些因素改变了暴力获利的条件。事实上,它们与任何人为的操纵都相距甚远,在一个充满政治的世界里,它们甚至不是政治操纵的对象。从来没有人在游行中高喊:“提高生产,扩大经济规模”;也没有人打着横幅要求:“发明武器系统,提升步兵地位”;更没有候选人承诺:“改变应对暴力的效率和规模之间的平衡”。这些口号很荒谬,因为它们的目标超出任何人可以主观影响的范畴。然而,正如我们即将探讨的,恰恰是这其中的变量决定了世界的运转,它们的影响超过任何政治宣言。

如果你仔细想一想,显而易见,很少有什么重要的历史转折是由人们的愿望驱动的。它们之所以发生,不是因为人们突然厌倦了某种生活方式,而喜欢上了另外一种。而稍加思考就会明白:如果人们的想法和愿望是事情发生和改变的唯一决定因素,那么历史上的所有突变,都不得不被解释为,与实际生活条件的改变毫不相干的情绪躁动。这实际上从来没有出现过。除了在影响部分人的医疗问题上,我们很少看到完全脱离客观原因的情绪暴走。

一般来说,芸芸众生不会因为觉得很好玩,就突如其来地摒弃他们的生活方式。

一个觅食者不会说:“我厌倦了生活在史前时代,我想住在农村,当个农民。”人类在行为模式和价值观念上任何决定性的改变,毫无例外,都是对生活条件切实变化的因应。至少在这个意义上,人类始终是现实的。如果他们的观点确实发生了突变,那很可能说明,他们熟悉的生活条件被某些东西打断了,一次侵略,一场瘟疫,一次突然的气候变化,或者一场改变他们谋生和自卫能力的技术革命。

大多数人都渴望稳定,而不是翻天覆地的历史变动;当变革来临,他们会感到困惑,并普遍陷入迷失,特别是那些失去收入或社会地位的人。你要是盯着民意调查或者其他测量情绪的指数,想从中了解大政治变革的发生与发展,必将徒劳无功。

\section{缺乏预见性的人生}
我们之所以察觉不到周围正在发生的重大变化,部分原因是变化并非人类之所欲,我们不想看到它。觅食者——人类的祖先,同样顽固不化,但他们有很好的借口。一万年前,没人能预料到农业革命的后果。那时候的人除了在哪里找下一顿饭之外,几乎什么也预见不了。农耕开始的时候,没有任何关于过去的记录,无从预测未来。甚至没有后来西方的那种时间观念,时间还未被有序被分成秒、分、小时、日等单位,以衡量年份。觅食者处于“永恒的现在”,没有日历,实际上也根本没有任何书面的记载。他们也不懂科学,除了直觉以外,也没有足够的智力去理解因果关系。在展望未来时,我们的原始祖先是盲目的。引用《圣经》中的比喻,他们还没有“尝过知识的果实”。

\subsection{以史为鉴}
幸运的是,我们现在的条件要优越得多。经过 500 代人的进化,我们具备了祖先们缺乏的分析能力。借助科学和数学,我们解开了自然界的诸多奥妙。我们对因果关系的理解,相比于早期的觅食者,简直堪称神奇。高速计算机的发展,使我们通过算法对复杂动态系统的运转有了新的认识,例如人类的经济。艰难发展起来的政治经济学,虽然远非完美,但也训练了人们对影响自身行为的因素的理解。

其中很重要的一点就是,无论何时何地,人们都会对激励做出反应,虽然不像经济学家所想的那么机械。成本与回报是其中的关键。在其他条件不变的情况下,通过提高奖励或者降低成本,改变某行为的外部条件,将激励人们更多地从事它。

\subsection{激励措施不可小觑}
人们会对成本与回报做出反应,这是行为预测的一个基本要素。你可以很有把握地说,往大街上扔 100 块钱,不管是在纽约、墨西哥城还是莫斯科,很快就会有人把它捡走。这种情形并不像看起来那么微不足道,它足以证明,那些认为预测是不可能的聪明人是错误的。任何正确衡量了激励措施影响的预测,大体上都可能是对的。而且,成本与回报的预期变化越大,它预示的发展就越不简单。

最具深远意义的预测,可能就来自于,认知和解读大政治变量的细微变化背后的含义。暴力是决定人类行为边界的终极力量。如果你能理解暴力的逻辑,以及它会如何运转,你就可以有效地预测,接下来人们会在哪里丢弃或捡起 100 块的大钞。

这并不是说你能知道不可知的东西。我们无法告诉你怎么预测彩票的中奖号码,或者其他任何完全随机的事件。我们没办法知道,恐怖分子会不会以及什么时候在曼哈顿引爆原子弹;或者一颗小行星会不会掉到沙特阿拉伯。我们也没法预测,会不会有的冰河世纪,会不会有火山突然爆发,或者新的疾病出现。有大量不可知的事情,都可能改变历史的进程。但是,猜测不可知与推导出已知事情的涵义,是完全不同的。如果你看到远处有闪电,就可以有把握地说,很快就会打雷;预测大政治的转型以及它的后果,与此类似,只是它涉及更长的时间框架,以及不那么确定的联系。

在后果展露之前,大政治变革的催化剂就已经出现了。农业革命的影响,用了五千年的世界才全部呈现。工业社会基于制造业和化学力量,从农业社会向它的过渡,就非常快,只用了几个世纪。信息社会的转型会更快,可能只需一生的时间。

不过,即使考虑到历史在缩短,现有信息技术对大政治的影响要完全实现,预期也得几十年的时间。

\subsection{主要及次要的大政治变革}
在本章中,我们会分析大政治变革的一些共同特征。在接下来的章节中,我们将更仔细地研究农业革命,以及第二大的阶段变化,就是从农田到工厂。在农业文明时期,还有一些次级的大政治变革,如罗马的衰落与 1000 年时的封建革命;这些标志着权力等式的消长,政府的兴衰,农业战利品从一个人手上转移到另一个人。罗马时期的大庄园主,欧洲黑暗时期的自耕农,以及封建时代的领主和农奴,他们吃着同一块地里长出来的粮食;但由于不同的技术、气候的波动、疾病的破坏累积起来的影响,他们经验的制度,判然有别。

我们并不打算彻底解释所有这些变化,这不是我们的目的;尽管我们已经勾勒出一些解释,说明了在过往的历史中,大政治变量的不断变化,是如何颠覆权力的行使方式的。简言之,由于大政治变量的波动,降低或提高了运用权力的成本,政府也随之盛衰盈缺。

如果你想要理解信息革命,应该牢记以下要点:

\begin{enumerate}
    \item 远在权力革命爆发之前,权力的大政治基础就已经开始改变了。
    \item 每当重大的变革开始时,人们的收入通常会下降。这是因为,由于人口的压力,社会资源趋向边缘化,从而导致自身危机四伏。
    \item 跳出系统,在“外面”观察,是一种文化禁忌。人们往往对现有社会的暴力逻辑视而不见,所以,他们也看不到这种逻辑的变化,无论是潜在的,还是公开的。大规模的政治变革在发生之前,很少有人能看得出来。
    \item 重大的历史变革总是涉及文化革命,新旧观念的支持者之间会发生冲突。
    \item 大政治的变革从来都不受欢迎,因为它会使苦心积累的知识资本变得过时,使既定的道德标准发生混乱。它不是来自于民众的普遍要求,而是在一定的历史背景下,由于暴力逻辑的外部条件发生了改变,社会必然做出的反应。
    \item 向新的谋生方式或新型政府的转变,在开始的时候,只局限在那些大政治的催化剂起作用的地区。
    \item 旧系统的失序和崩溃,往往导致暴力的加剧,所以,历史上的变革都有一个社会混乱的阶段,可能农业社会的早期是个例外。
    \item 腐败、道德滑坡和效率低下,是一个系统到了最后阶段的信号。
    \item 技术对暴力逻辑的影响日益加强,这导致了历史的加速,并大大缩短了两个阶段转折时的适应期。
\end{enumerate}


\subsection{历史在加速}
今天的事态发展,要比以往的历史演进快上无数倍;所以,在同样的转折点,如果能提前认识到世界将发生什么样的变化,相比你的祖先,这对你要有价值得多。

历史上的第一批农民,即使他们奇迹般地洞悉了农耕在大政治上的所有影响,这些认知其实也没有用;因为还需要几千年的时间,才能完成向新社会阶段的转折。

今非昔比。历史已经加速,准确预测新技术在大政治层面的影响,在当今社会大有裨益。对于这次信息时代变革的潜在影响,如果我们建立起来的认知,能够达到现在普通人对农业革命和工业革命的了解程度,那这些认知将无比宝贵。简单来说,大政治预测起作用的范围,已经缩短到了最有用的时间段,那就是一个人的一生。


\begin{tcolorbox}
\kaishu 回首过往的几个世纪,或者只是看看眼前,我们就可以发现,许多人是靠着使用暴力武器的特殊本领过活的,往往活得还很好;不仅如此,他们的活动还极大地影响着社会稀缺资源的利用。
\begin{flushright}
—— 弗雷德里克·C·莱恩(FREDERIC C. LANE) 
\end{flushright}
\end{tcolorbox}

我们对大政治的研究,就是想做到这一点,即挖掘出那些改变暴力行使边界的因素,以及它们可能产生的影响。\textbf{这些大政治因素在很大程度上决定了:什么时候,在哪里,行使暴力能够得到回报。}它们也有助于了解收入是如何在市场中分配的。

就像经济史学家弗雷德里克·莱恩明确指出的,如何组织和控制暴力,在决定“稀缺资源的利用”上,起着巨大的作用。

\section{大政治学的速成课程}
大政治是一个强有力的概念,它有助于解释历史上的一些大问题:政府的兴亡、机构的演化、战争的时机和后果、经济繁荣和衰退的模式。通过提高或降低行使权力的成本与回报,大政治控制着人们将自己的意志强加于人的能力。从人类社会的最初就是如此,今天依然如此。在《血流成河》和《大清算》中,我们探讨了很多重要的大政治因素,正是它们,在看不见的地方决定着历史的演进。解开大政治变革及其影响的关键,在于了解那些催生暴力革命的变量。某种程度上,这些变量可以笼统地归为四类:\textbf{地形、气候、微生物}和\textbf{技术}。

1、\textbf{地形}是个关键因素,从一点可以明显看出:在公海上,从来没有像在陆地上那样,形成暴力垄断的局面;没有哪个政府的法律在那里是排他的。这一点至关重要,借此人们理解,随着经济迁移到网络空间,暴力及保护的组织将如何演变。

地形结合气候,在早期的历史中发挥了重要作用。第一批国家出现在被沙漠包围的洪泛区,如美索不达米亚和埃及,那里有丰富的灌溉用水,但周边地区又过于干燥,无法支持自耕农经济。在这种情况下,单个农民如果不与他人合作,去维护当时的政治结构,就面临很高的成本而活不下去。那时候的灌溉,只能是大规模地进行;没有灌溉,作物就不会生长。没有收成,就意味着挨饿。在沙漠中,控制水的人处于强势地位,这使得当时的政府既暴虐又富有。

我们在《大清算》中也分析过,地理条件对古希腊自耕农的繁荣居功至伟,并使该地区成为了西方民主的摇篮。3000 年前,地中海地区的高价值农作物是橄榄和葡萄,那时候交通条件普遍还很原始,生活在距海岸几英里之外的居民,就没办法在这些作物经济上竞争。因为如果通过陆路运输橄榄油和葡萄酒,运费成本太高,很难在市场上盈利。希腊在沿海地区精心设计了海岸线,使他们国内的大部分地区离海都不超过 20 英里。这给希腊农民带来了决定性的优势,把他们在内陆的潜在竞争者甩在了身后。

由于在高价值产品的贸易上占据优势,希腊农民只需要控制小块的土地,就能获得丰厚的收入。高收入使他们有能力购买昂贵的盔甲。古希腊著名的重装备步兵,就是自费武装起来的农民或地主。这些步兵装备精良,斗志昂扬,是令人生畏、不容轻视的军事对手。地理条件是希腊民主的基础,而不同的地形催生了埃及和其他东方的专制国家,道理是一样的。

2、\textbf{气候}也影响着暴力的作用边界。从觅食到农耕的第一次大转型,气候变化就是它的催化剂。大约一万三千年前,最后一次冰河时代结束,地表的植被发生了翻天覆地的变化。从近东开始,冰河气候首先在那里退去,温度和降雨量逐渐攀升,森林蔓延到从前的草原地区。特别是山毛榉的迅速繁衍,严重影响了人类的饮食选择。就像苏珊·艾琳·格雷格(Susan Ailing Gregg)在《觅食者和农民》(Foragers and Farmers)所写的:


\begin{tcolorbox}
\kaishu 榉树林大行其道,可能对当地的人、植物和动物都产生了严重的影响。橡树的树冠是相对开放的,阳光可以穿透,照射到林地;由灌木、草本植物和杂草混合的树下灌木丛就能蓬勃地生长,植物的多样性又支撑了各种野生动物的生存和繁衍。与之相反,榉树的树冠是封闭的,森林底部被遮得密不透光。树下的植被,就只有在春季榉树叶长出来之前,有一些一年生的植物;其余时节就剩一点耐阴的苔草、蕨类和杂草。 
\end{tcolorbox}

随着时间的推移,茂密的森林从整个欧洲蔓延到东部大草原,蚕食了广阔的草地。森林削减了可以养活大型动物的牧区,使觅食者人口的生存处境日益艰难。

在冰河时代的繁荣期,狩猎采集的人口膨胀得太厉害,无法再靠日益减少的大型哺乳动物养活自己,其中很多物种都被猎杀到灭绝。向农业的转变并不是人们想要的选择,而是为了弥补饮食不足而被迫采取的一种权宜之计。在遥远的北部地区,觅食者依然是多数;因为气候变暖的趋势,并没有对那里的大型哺乳动物的栖息地产生不利的影响。在热带雨林地区,全球气候变暖也没有减少那里的食物供应。自农业出现以来,更多的变化是由气候变冷而不是变暖促成的。

适当了解一点历史上的气候变化动力学,在以后气候波动的时候会很有用。如果你知道,平均气温每下降一摄氏度,植物的生长期就会减少三到四个星期,可种植作物的最高海拔就下降 500 英尺\footnote{换算后约 152.4 米。},那么,你就会对未来人们行为的边界条件有一个概念。你可以用这些知识去预测各种变化,从粮食的价格到土地的价值。你甚至可以就气温下降对实际收入和政局稳定的影响,得出非常有见地的结论。在过去,如果粮食连年歉收,导致食物价格上涨和可支配收入减少,人们就会揭竿而起,推翻政府。

举例来说,近代以来最冷的十七世纪,也是全世界普遍爆发革命的时期,这可不是巧合。这种混乱背后的大政治因素,就是气候骤降。那时天气严寒,连凡尔赛宫中“太阳王”桌上的酒都冻成了冰块。天气变冷,作物生长期缩短,粮食歉收,收入减少。所以,大约自 1620 年开始,在全球范围内,繁荣景气转入大萧条。

后来的事实证明,社会因此而陷入大乱。十七世纪的经济危机导致世界被叛乱淹没,其中很多起集中爆发在 1648 年,刚好是另一次更著名的革命周期的 200 年前。从 1640 年至 1650 年间,爱尔兰、苏格兰、英格兰、葡萄牙、加泰罗尼亚、法国、莫斯科、那不勒斯、西西里、巴西、波西米亚、乌克兰、奥地利、波兰、瑞典、荷兰和土耳其都发生了叛乱;甚至中国和日本也被动乱所席卷。

所以,在贸易萎缩的 17 世纪,重商主义占据主导地位,应该也不是巧合。也许是在该世纪末,“当时发生了可怕的饥荒,”经济封闭达到了顶峰。到了 18 世纪,特别是 1750 年以后,气温回升,作物产量也上去了,西欧人的实际收入开始增加,对工业制成品的需求也随之攀升。更多构建自由市场的政策被采纳,工业规模也不断扩大,形成了经济增长的自我强化,这就是人们常说的工业革命。

技术和工业产出的重要性日益增加,减轻了天气对经济周期的影响。不过,即使在今天,你也不能低估气温骤降对实际收入的影响,哪怕是在北美这样的富裕地区。当既有机构的配置已经潜力耗尽时,社会就会形成一种强烈的崩坏趋势。在过去,这种趋势主要出现在人口暴涨把土地的承载力推到极限时。这种情况在 1000 年封建革命前发生过,在 15 世纪末又出现了一次。在这两次变革中,作物减产歉收、实际收入大幅下降,是统治集团被颠覆的重要诱因。今天,这种临界性主要表现在消费信贷市场。如果气温下降,作物减产,可支配收入减少,就会引发债务违约,甚至抗拒交税。以史为鉴的话,经济封闭和政治混乱,都会上演。

3、\textbf{微生物}传递了伤害或免于伤害的力量,所以它也常常决定着权力的运行。欧洲人征服新世界就是典型的例子,对此我们在《大清算》中讨论过。欧洲定居者,从满是病菌的农业社会来到新大陆,他们对一些儿童传染病具备了相对的免疫力,如麻疹等。但他们遇到的印第安人,还以觅食为主,以部落的形式散居在人口稀少的地区,没有相应的病毒免疫力,所以就被灭绝了。而且,死亡率最高的时候,发生在白人真正抵达之前。因为有印第安人在海边遇到了欧洲人,被感染之后带着病毒回到了内陆。

有时候,微生物也是权力机器前行的障碍。在《血流成河》中,我们谈到,热带非洲有强力的疟疾病菌,在很多个世纪内,阻止着白人的入侵。在 19 世纪中期发明奎宁以前,白人军队无法在疟疾地区生存,无论他们的武器多么先进。

人类与微生物之间的互动还产生了重要的人口效应,改变了暴力的成本和回报。

如果死亡率因为瘟疫、饥荒或其他原因而波动较大时,死于战争的相对风险就会降低。从 16 世纪开始,死亡率暴涨的频率不断下降,这就很好地解释了,为什么家庭的规模越来越小,而且今天的人们,对于突然死于战争的容忍度比过去要低得多。这也降低了人们对帝国主义的忍耐度,并且提高了在低出生率社会中行使权力的成本。

当代社会由小家庭组成,如果发生战争,即使很小的死亡人数,也难以被接受。相比之下,在现代社会早期,人们对帝国主义产生的死亡成本接受度要高得多。在本世纪之前,大多数父母都会生很多孩子,预计其中有个别孩子会随机地突然病死。在一个早夭司空见惯的时代,未来的士兵和他们的家人,对于上战场,并不是很抗拒。

\begin{tcolorbox}
\kaishu 机器具有侵略性。编织者成为了网,机械师沦为了机器。如果你不使用工具,你就被工具使用。
\begin{flushright}
—— 爱默生(EMERSON)
\end{flushright}
\end{tcolorbox}

4、到目前为止,在现代时期的几个世纪里,在决定权力运用的成本和回报上,\textbf{技术}的影响力最大。本书论证的基础,就是假设这种影响会一直延续。技术有几 个关键的维度:

\begin{itemize}
    \item 进攻与防守之间的平衡。先进的武器技术左右着攻防之间的平衡,进而决定着政治组织的规模。当攻击能力增强时,远距离施加权力的力量就会占主导地位,管辖区域内趋于稳固,会形成更大规模的政府。在其他时期,比如现在,防守能力正在上升,这就提高了在核心区域外运用权力的成本。在这种趋势下,管辖权倾向于下放到地方,大政府会分解成小政府。
    \item 公民平等及步兵的优势。决定公民之间平等程度的一个关键因素是武器技术的性质。相对便宜、非专业人士能够操作,以及能提高步兵军事地位的武器,往往会促进权力的平等。当托马斯·杰斐逊写“人人生而平等”的时候,比几个世纪前发表类似的口号要现实得多。和拿着青铜燧发枪的典型英国士兵相比,一个拿着猎枪的农民不仅同等武装,甚至装备得更好。农民可以从更远的地方向士兵射击,而且精准度更高。中世纪与此截然不同。手上只有一把草叉的农民——他甚至买不起更多的草叉,根本不可能对抗骑着高头大马、全副武装的骑士。在1276 年,没人提出过“人人生而平的”。因为事实很明显,人就是不平等的;一个骑士的杀伤力超过几十个农民。
    \item 大规模暴力的优势和劣势。社会可能是几个大政府还是许多小政府,另外一个决定性的变量是使用先进武器所需的组织规模。当暴力的回报越来越高时,规模越大,回报越高,政府当然就会变得更大。当一个小团体可以组织有效的手段抵抗大团体的攻击时(中世纪就是这样),国家主权就会趋于分裂;小型的、独立的组织就能行使政府的许多职能。就像我们在后面的一章中将探讨的,随着信息时代的到来,我们将看到赛博战士(cybersoldier)到来的曙光,他们是分权革命的信使。赛博战士不再专属于民族国家,很小的组织甚至个人,都可以部署和使用。下一个千年的战争,会有很多由计算机主导的不流血的战争。
    \item 大规模生产的经济性。权力应该分散在当地,还是集中在遥远的中央,另外一个重要的决定因素是,那些提供民生所需的重要企业的规模。如果重点企业只有在广阔覆盖的贸易区内大规模地组织起来,才能够获得最佳效益时,政府就可以通过为企业提供这种条件和保护,从中抽取大量的财富,用以支撑大规模政治体系所需的成本和费用。在这样的情况下,如果有一个世界霸主主宰其他所有国家,整个世界的经济反而会更有效地运行,就像 19 世纪大英帝国那样。但在有些时候,大政治变量的结合会降低经济规模。如果维持大贸易区的经济利益降低,以前利用这种模式繁荣起来的大国政府就会开始分裂,即使攻防之间的军事平衡还一如既往。
    \item 技术的分散。关键技术的分散程度,是决定权力等式的另外一个重要变量。当武器或生产工具被有效地囤积或垄断时,它们就会使权力集中化。事实证明,即使在本质上属于防御性的技术,如机枪,如果分散程度不够高,也往往沦为强有力的攻击性武器,从而促进统治规模的扩大。19 世纪末,欧洲列强享有对机枪的垄断,他们可以轻松地征服边远的国家和地区,大幅扩展殖民帝国。后来在20 世纪,特别是二战之后,机枪被广泛采用,比较容易获得,又成为了破坏帝国主义的力量。在同等条件下,关键技术越分散,权力就越分散,政府的最佳规模就越小。
\end{itemize}

\section{大政治变量的演进速度}
虽然技术是决定大政治变革的最重要因素,而且明显越来越重要,但在历史上,这四种因素都对权力的规模发挥了关键的影响。它们交织在一起共同决定了,当暴力扩张到更大的范围,它的回报是否随之上升。这进而决定了,在权力规模和资源运用效率之间进行取舍的重要性,并强烈影响着收入在市场上的分配。现在的问题是,它们在未来会扮演什么样的角色?要回答这个问题,就必须认识到,这些变量的演进速度是完全不同的。

在整个有记载的人类历史中,地形几乎是一成不变的。除了港口淤塞、垃圾填埋或地质侵蚀等局部的小变化之外,今天的地形状况与亚当和夏娃走出伊甸园时基本没差;而且,它还会继续保持现在,除非新的冰川期来临,或者发生剧烈的事件。在更深远的时间范畴,地质年代可能改变,也许是对大型陨石撞击的反应,发生时间为 1000 万至 4000 万年之间。有一天,可能会再次发生地质运动,颠覆我们星球的地形地貌。如果这样的话,你可以准确地预测到,棒球和板球的比赛都将被取消。

气候波动比地质运动要活跃得多。在过去的 100 万年里,气候是造成已知大部分地表变化的推手。在冰河时代,冰川开凿山谷,改变河流走向,将岛屿从大陆割裂,或者降低海平面使它们相连。气候波动对历史影响甚巨。它先是在上一次冰川期结束后催生了农业革命,后来又在低温和干旱期破坏了制度的稳定。

最近,人们又对“全球变暖”表示担忧,也不无道理。但是,长远来看,更可能发生的气候变化似乎是变冷,而不是变暖。根据从海底采集的样本中的氧同位素分析,对温度波动的研究表明,目前是 200 多万年以来第二温暖的时期。从 17世纪的历史我们知道,如果天气变冷,就会影响社会稳定。从这个意义上说,全球变暖的警报反而令人比较安心。如果真的变暖,那就保证了气温将在异常温暖和相对良好之间继续波动,就像过去三个世纪所经历的。

微生物的演变速度对权力的影响,令人比较迷惑。微生物变异非常快,特别是病毒。例如,普通感冒就以几何量级的速度变异。尽管微生物变异速度很快,但在改变权力边界的影响上,它们却没有技术的变革那么突然。部分原因可能是,自然界为了保持平衡,让微生物只是感染而不杀死宿主,对它们更为有利;毒性太强、太容易杀死宿主的病毒也会使自己难以存活。寄生微生物能够生存,关键就在于它们对宿主不能过于致命,或者全部是致命型的。

当然,打破权力平衡的致命疾病,也不是不可能爆发。这类事件在历史上占据重要地位。黑死病消灭了欧亚大陆的大部分人口,给 14 世纪的国际经济造成了毁灭性的打击。

\subsection{历史的如果与可能}
我们可以从两个角度去理解历史,一是思考它本可能发生什么,二是看它事实上发生了什么。我们知道,微寄生虫不可能不继续对现代人类社会造成破坏。例如,有些微生物类似疟疾,但毒性更强,它本可能对权力的扩张构成障碍,阻止西方人对周边地区的侵略。第一批葡萄牙冒险家驶入非洲水域时,他们本可能感染一种逆转录病毒,就是传染性更强的艾滋病,从而使新的亚洲贸易路线在开辟之前流产。哥伦布和第一波新大陆的定居者,原本也可能遇到各种疾病,遭到灭顶之灾;就像那些被西方人带来的麻疹和其他儿童传染病灭绝的原住民一样。但是,这些本可能发生的事情,都没有发生。这种巧合强化了人们的直觉,那就是历史自有定数。

在现代时期,微生物对政权巩固造成的阻碍,远远小于对其的促进作用。在外围殖民地扩张的西方军队和殖民者往往会发现,他们在技术上的优势会被微生物因素所放大。西方人被看不见的生物武器所武装,对某些儿童流行病具有相对的免疫力,而这些流行病则足以灭绝当地的原住民。所以,来自欧洲的航海者具有明显的优势,而他们的对手生活在人口密度较低的地区,完全处于下风。随着事态的发展,疾病的传播几乎是单向的,那就是从欧洲向外扩散;而在相反的方向,从外围向欧洲核心,并没有对应的转移。

可能有一个反例,有人说西方探险家将梅毒从新世界传入了欧洲。这一点存在争议。但即使是真的,梅毒也没有影响权力的扩张,它主要影响了西方的性道德。

从 15 世纪末截至 20 世纪的最后 25 年之前,微生物对工业社会的影响是日趋良性的。尽管肺结核、小儿麻痹症和尿毒症导致了很多个体的悲剧与不幸,但在整个现代时期,类似安东尼瘟疫或黑死病那样大政治影响力级别的疾病,并没有爆发。公共卫生的改善,疫苗和解毒剂的发明,普遍降低了传染性微生物对现代社会的威胁,这也相对提高了技术对设定权力边界的重要性。

但是,现代近期出现的艾滋病,以及对可能大面积传播的外来病毒的警告,也许暗示着,未来不会再像过去 500 年那样,微生物对大政治的影响完全是良性的。是否以及何时会出现新的瘟疫,造成全世界大面积感染,是不可知的。相比气候或地形的剧烈变动,寄生微生物的爆发,如病毒大流行,将更有可能挑战技术在大政治变量中的主导地位。

我们没办法检测或预测,地球生命的性质与我们已知的会发生多大的背离。我们只能祈祷,在下一个千年,最重要的大政治变量是技术而不是微生物。如果运气继续站在人类一边的话,技术作为大政治变量中主导因素,将越来越重要。然而,通过对第一场大政治变革,也就是农业革命的回顾,我们可以清楚地看到,历史并非总如人们所愿。