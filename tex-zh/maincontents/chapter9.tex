\chapter{民族主义、反动与新卢德分子}

\begin{tcolorbox}
当然,民族主义在本质上是荒谬的。为什么生而为美国人、阿尔巴尼亚人、苏格兰人或斐济人——无论幸与不幸,都要被强加一种忠诚;这种忠诚统治了个人的生活,构建了一种社会,并使之在形式上与其他社会相冲突。在过去,存在地方对家族、氏族和部落的忠诚,存在地方对领主或地主、对王朝或领土战争的义务,但主要的忠诚是对宗教、对上帝或神王,也可能是对皇帝,或对类似这样的文明。当时没有国家,有的是对乡土——祖宗之地——的依恋,或爱国情怀;但在现代之前,谈论民族主义是不合时宜的。
\begin{flushright}
—— 威廉·普法夫(WILLIAM PFAFF)
\end{flushright}
\end{tcolorbox}

IBM 的官方网站\footnote{http://www.ibm.com}说“世界正在变小”,是一个信息量很大的说法,像 IBM 的广告公司这样有声望的机构,又进一步加强了它。他们为 IBM网络创作的多元文化广告,“小小星球的解决之道”,提醒那些尚未觉察的体育迷们,分散在世界各国的人们,他们之间的关系已经被技术所改变。就这一变化的影响,我们参考了杰出的历史学家威廉·麦克尼尔(William McNeill)的一个脚注,很有价值。他说:“通信和运输的不断加强,不但不利于国家的巩固,反而开始起反作用;因为它们的运作范围超越了现有政治和民族的界限。”随着世界的“变小”和通信的变强,“偶然的”及“本质上荒谬的”国家和民族主义,都必将被削弱。

\section{大变革}
这种预期是合理,但问题在于,过去所有的历史告诉我们,它不可能以合理的方式得以实现。它所预示的转型涉及一场危机。应对这场危机,需要一种全新的思维方式,一种超越了民族主义和民族国家的、对社会的全新想象。正如迈克尔·比利格(Michael Billig)所强调的,“我们对民族性(nationhood)的信念,以及对属于一个国家的自然性(naturalness)的信念,”是“一个特定历史时期的产物”。

这个时期,即现代时期,可能已经失灵了。它的主要机构,即民族国家,则依然存在;但它们生存的地基已被严重侵蚀,摇摇欲坠。当另外一只靴子落地,当民族国家崩溃时,我们估计会产生可怕的后果,特别是富裕国家;在这些国家,20世纪的“国民经济”为非熟练工人提供了高收入。

我们相信,当一切进入轨道时,信息技术所引起的大政治条件的变化,会导致翻天覆地的制度变革。本书的论点是,民族国家所享有的大规模权力,注定将被私有化和商业化。和所有激进的制度变革一样,主权的私有化和商业化,关乎一场理解世界运作方式的“常识”革命。这样的革命,很少以渐进的、线性的方式发生。事实恰恰相反,几乎不存在这种可能,原因我们在《大清算》中讨论过。我们预计信息时代会导致突变——与过去的体制和意识的急剧决裂。以下是在此过程中需要关注的地方:

\begin{enumerate}
    \item 由于微处理技术的影响,经济组织发生了前面几章所描述的相关变化。
    \item 所有局限于地理边界以内,不能超越边界运作的组织,其重要性或多或少都将被迅速削弱。在信息时代,政府、工会、有执照的职业和游说者的重要性,都将低于工业时代。因为从政府手里争取的商贸利益,其价值将大打折扣;浪费在游说上的资源也将大幅减少。
    \item 更多人意识到民族国家已经过时,全球各地普遍上演要求独立的分离运动。
    \item 传统精英的地位和权力将下降;把民族国家合理化的符号与信仰,其受尊重的程度也将下降。
    \item 那些丧失地位、收入和权力的人,当他们开始认为,他们的“日常生活”,被政治权力的下放和新型市场配置方式而破坏时,一场激烈甚至暴力的民族主义反应,就会以这些人为中心而爆发。这种反应包括:
    \begin{itemize}
        \item 怀疑和反对全球化、自由贸易、“外国”所有权及其对当地经济的渗透。
        \item 敌视移民,尤其是与原民族群体明显不同的异族移民。
        \item 普通民众仇视信息精英、富人及受过良好教育的人,抱怨资本外逃与工作岗位的消失。
        \item 民族主义者采取极端措施,意图组织个人或地区从摇摇欲坠的民族国家中分离出去,包括诉诸战争和种族清洗,以加强对国家的民族主义认同,并把国家对人民和资源的占有合理化。
    \end{itemize}
    \item 显然,信息技术有助于主权个人从国家权力下逃离,所以,对强制力崩溃的反应还包括,这些新技术以及使用新技术的人,将遭到新卢德主义的攻击。
    \item 在不同的地区和人口群体中,民族主义卢德分子的反应是不一样的:
    \begin{itemize}
        \item 在快速增长的经济体中,反应将不那么强烈,因为在工业时代,这些经济体的人均收入较低,而市场的深化提高了所有技能群体的收入。
        \item 在目前的富裕国家,反动情绪会最为强烈,特别是在那些过去享受着高收入、但既不掌握什么技能又不贡献什么价值的群体占比较高的社会。
        \item 尽管会有“智能炸弹客”,但新卢德派的大部分追随者,将来自于先进民族国家中收入能力最低的三分之二人口中。
        \item 民族主义和卢德派中反应最为激烈的,不是那些穷困潦倒的人,而是掌握了中等技能的阶层,他们是在工业时代成长起来的,面临着向下流动的威胁。
    \end{itemize}
    \item 随着新的大政治条件孕育出新的身份意识,以及与之配套的新的意识形态和道德观念,民族主义过时的要求就会失去吸引力。
    \item 民族主义的反动情绪会在新千年的早期几十年达到顶峰,然后随着事实证明,分散的主权在效率上远远优于民族国家的大规模权力运作,这种反动就会消退。我们相信,民族国家生性而来的对其他国家的霸凌行径,如俄罗斯入侵车臣,会使民族和民族主义的狂热者,失去在信息时代大政治条件下成长起来的新生代的同情。
    \item 民族国家最终会在财务危机中崩溃。当失败的机构不断面临支出上升和收入 下降时,通常都会爆发系统性危机。随着 21 世纪初退休福利和医疗支出的膨胀, 这种情况势必给领先的民族国家造成威胁。就在我们写作的时候,英国和美国都 背负着几十亿美元无准备金的养老金负债(在人均可比的基础上),且都无法收 拾。其他主要国家也面临着同样的破产负担。
\end{enumerate}


\section{与文艺复兴的相似之处}
保姆式国家崩溃的后果,与 500 年前圣母教会垄断体制的崩溃所产生的结果,将 如出一辙;在前面我们简要阐述过其中的原因。而与今天的民族国家不同,当时 的教会,在数百年的时间里,都处于不受挑战的优势地位。在某些方面,教会的 地位比 500 年后的国家更加稳固。长久以来,教会一直宣称自己是“基督社会之 首的普遍权威”;这是中世纪思想史学家约翰·莫拉尔(John B. Morrall)的描述。

然而,尽管在 1490 年的技术革命之前,很少有欧洲人质疑教会在基督教世界中 至高无上的地位;但在革命之后,教会的传统角色勉强生存了一代人的时间。

\subsection{良知的私有化}

到了 1520 年代初期,数百万善良的欧洲人已经拒绝了天主教会的普遍权威,而 在仅仅几十年前,这种异端行为是要受到酷刑和死刑惩罚的。事实上,欧洲许多 中世纪的大教堂和普通教堂,都装饰这异教徒被恶魔拔掉舌头的雕像。这些酷刑 所传达的教训,肯定给那些不识字的教徒留下了深刻的印象;仅仅通过惩罚他们 就能认识到,那些受害者都是异教徒。这种惩罚尽管很严酷,但相比对异教徒的 终极惩罚——烤死在火刑柱上,它只能算是一种热身。

然而,令教会失望的是,这些教训并没有形成足够的威慑力。印刷机的出现,使 异端观点层出不穷,可怕的惩罚并不能阻止异端分子的涌现。实际上,在近代早 期的欧洲,确实有不少追求宗教自由的先驱,非常不幸地为他们精神独立的主张 付出了代价,被割掉了舌头,或被烧死在火刑柱上。宗教裁判所中的反动分子, 仅仅因为有人说了一些我们认为是普通的良心话,就把他们烧死了。

总而言之,宗教改革和它所激起的反动行为,使数百万人付出了生命的代价。仅 在三十年战争的最后半年,死在战场上的人数就达到 151,000 人。还有更多的人, 死于饥荒、疾病、宗教裁判所和其他权力当局手中,绝不是所有的暴力事件都是 天主教会所为。后来人们在伦敦塔中发现了一千多具尸骨,普遍认为是英王亨利 八世残忍杀害的英国重要的天主教徒。其中一些人,包括托马斯·莫尔爵士和圣 约翰·费舍尔主教,都是因为拒绝放弃旧信仰而被公开处决。亨利八世的天主教 女儿玛丽王后,因为从父亲那里遗传了梅毒而精神失常。在她统治的最后两年, 将 300 名新教异教徒烧死在了火刑柱上。当不同教派的信仰者坚持自己的信仰, 选择自己支持的教会时——该权利被长期剥夺,这就是他们要付出的代价。

从我们 20 世纪末的角度来看,这些人对个人信仰的表达,完全在宗教自由和言 论自由的范围内。但在十六世纪初,人们既没有宗教自由,也没有言论自由。当  时的权威,还在从日渐式微的中世纪世界观中汲取养分。在他们眼里,个人自治 与权威相对抗的姿态,尤其是相对于教皇的“至上全权”(plentitude potestatis), 是明显的颠覆,是无法容忍的。正如神学历史学家尤安·卡梅伦(Euan Cameron) 所说,像马丁·路德这样的宗教改革家的观点,“意味着与旧教会的制度和精神 连续性,进行了深思熟虑且坚决果断的分裂。” 

\subsection{异端与叛逆}
正是基于这种精神,我们预计,民族国家的体制和意识形态的连续性,将被“深 思熟虑且坚决果断地打破”。到下个世纪的前四分之一世纪末,数以百万计正直 的个人,将会被判以一种低级的叛国罪,相当于 16 世纪宗教异端的世俗版本。

他们要主张自己的主权,而不再效忠于摇摇欲坠的民族国家;而他们的权力,不 是选择自己的主教或礼拜堂,而是作为客户选择合适的治理方式。主权的私有化, 就是五个世纪之前的良知私有化。二者都是统治机构以前的支持者进行的大规模 反叛。阿尔伯特·赫希曼(Albert O. Hirschman),研究“对公司、组织和国家 衰退的反应”的专家,他写到,这种类型的退出困难重重,因为“退出常常被打 上犯罪的烙印,它一直被贴有逃兵、变节、叛国的标签。” 主权个人不再接受作为国家的人力资源,以及被强加的一切。数百万人将摆脱公 民义务,成为政府提供的有价值服务的客户。他们甚至会创建和赞助相应的机构, 把与公民身份相关的大部分服务,完全置于商业化的基础上。在 20 世纪的大部 分时间里,生产者一直被国家当作资产,就像奶农对待奶牛一样,他们被压榨得 越来越厉害。现在,奶牛要长出翅膀了。

\subsection{脱离公民身份}
就像在 16 世纪,新的大政治条件打破了教会的垄断;我们认为,信息时代的大 政治条件,最终将决定 21 世纪的治理方式,不管在那些被现代政治价值观洗脑 的人看来,它的新规矩是多么的离谱。从“公民”身份到“客户”身份的演变, 以为着对过去的反叛;就像在现代早期,从骑士身份到公民身份的转变一样激烈。

信息精英对公民身份的抛弃,会产生一种刺激;而类似的刺激,在五百年前,导 致数百万欧洲人抛弃了教皇的绝对正确。

如果说与宗教改革的类比没有说服力,部分原因可能是,在今天很难直接看出, 放弃对宗教机构的忠诚曾经是非常严重的问题,就像 20 世纪的叛国罪一样。在 20 世纪末的今天,除了少数伊斯兰国家外,异端邪说只是一种精神不正常,对 个人名誉的打击,跟一张在限速 30 英里的道路上开到 45 英里而吃到的超速罚单 差不多。现在的异端,需要达到公然的魔鬼崇拜的程度,才会引起人们的注意。

在大多数西方国家,宗教教义的整理过于涣散和草率,以致于很少有人了解,过 去成为异端争论焦点的神学观念。这反映出人们的注意力已经普遍从宗教上转移 开了。

20 世纪末的宗教领袖,他们已不再把精力用于关注人们的精神问题,而是成为 了游说者和社会煽动者,这在某种程度上,确实帮助导演了人们对精神世界严肃 性的反叛。宗教领袖像散落的铁屑一样,被权力的磁铁所吸引,把大部分的活动 用于向政治领导人施压,促使其采取对民族主义生存至关重要的收入再分配政 策。在阿根廷,天主教会大声疾呼,向卡洛斯·梅内姆总统的政府施压,要求其 放弃经济改革,转而支持严重通货膨胀的传统货币,并采用凯恩斯主义的财政政 策。在新西兰和其他国家,宗教领袖也对削减臃肿预算的计划表达了类似的抱怨。

天主教的主教还不遗余力地游说,反对美国的福利改革。

\subsection{财政裁判所?}
简单来说,当代的宗教领袖,正在把他们日益衰落的道德权威,主要用在世俗的 救赎,以及对国家政治的影响和鼓动上,而不是精神救赎。鉴于他们的这种表现 记录,可以预期,他们会作为帮凶,参与到即将到来的对世俗改革的反动中去。

当民族国家受到挑战并风雨飘摇,它将不再能够履行对物质利益的承诺,而这正 是民众支持它的核心基础。在法国大革命时达成的事实交易,即将失效。国家无 力再保证公民接受低成本或免费的教育,更不要医疗保健、失业保险和养老金, 而人们以此所交换的是质量极差的军事服务。虽然现在战争的条件已经发生了变 化,政府不用再使用大规模的军队,也可以保卫自己及其统治下的领土,但这并 不能免除对政府打破交易承诺的批评,况且这种交易业已过时。

实际上,当新的大政治逻辑全面展开,它所产生的后果,在新型信息经济的失败 者中间,将极不受欢迎。因此,完全可以肯定,许多宗教领袖以及政府支出的既 得利益者,将会站在怀旧与反动的最前线,试图重振民族主义的主张。他们会宣 称,不能让任何美国人、法国人、加拿大人或其他任何民族的人,饥肠辘辘地上 床睡觉。即使是那些一直处于改革前沿,并将从“市场友好型全球主义”中获得 巨大利益的国家,如新西兰,也会被反动的失败者所折磨。他们会想方设法,阻 挠资本和人员的跨界流动。而且,他们不会就此善罢甘休。政治煽动家们,像新 西兰第一党的领导人温斯顿·彼得斯(Winston Peters),懒得从原本去思考世界 将如何运作。但是,等到适当的时候,温斯顿和他的手下会得知信息经济的逻辑。

他们就会设法阻止计算机、机器人、电子通讯、加密及其他信息技术的扩散;而  这些技术正在全球经济的各个领域内代替人工。无论你在哪里,永远会有一群政 客,为了阻挠个人从政治中宣布独立,不惜破坏长期繁荣的经济前景。

\subsection{20/20 愿景}
到 2020 年,或者说在马丁·路德把他颠覆性的 95 条论钢,钉在威腾堡教堂大门 上,大约 5 个世纪之后,人们对公民身份的成本收益比的认识,会经历类似的颠 覆性觉醒。能人和富人,即未来的主权个人,对民族国家看法的改变,相当于经 历一场政治上的激光近视手术。他们将会看到 20/20。在 20 世纪,和整个现代时 期一样,暴力回报的持续高涨,使大政府成为了一种赚钱的买卖。在经合组织国 家,尽管存在对收入和资本的掠夺性税收,但大规模权力的决定性地位,依然左 右了富有的、雄心勃勃的公民对国家的忠诚。在第二次世界大战后的十年里,每 个经合组织国家的政客,都能够以接近甚至超过 90\%的边际税率进行征税。

对此我们讨论过,富人别无选择,只能屈从。现实形势迫使他们不得不依靠政府, 来保护自己,因为后者掌握着大规模的暴力。除了有机会到香港任职的英国警察 (香港比英国税率低,译注),经合组织国家征收垄断性的税收,很少被当成一 回事。在工业时代,任何有高收入能力的人,如果想享受领先的经济机会,除了 居住在高税率的经济体中,几乎没有别的选择。这意味着,他们要承受昂贵的税 收负担,与政府服务的价值不成比例。

\subsection{政治的算术}
19 世纪的美国副总统约翰·卡尔霍恩(John C. Calhoun),曾经很精明地勾勒出 现代政治的算术。卡恩霍恩的数学公式,将民族国家的全部人口分为两类:一类  是纳税人,他们对政府服务成本的贡献大于消费;另一类是税收消费者,他们从 政府获得的利益超出过成本的贡献。除了几个明显的例外,大多数经合组织国家 的企业家,在 20 世纪结束时都是净纳税人,而且程度很夸张。例如,在 1996 年,前 1\%的英国纳税人承担了所得税总额的 17\%;他们比底层 50\%的收入者支 付的税款还多 30\%,后者只贡献了 13\%的所得税总额。在美国,富人的税负更 夸张;1995 年,前 1\%的人支付了总所得税的 30.2\%。正如弗雷德里克·莱恩所 提示的,富人不仅必须为那些“质量差劲但价格高得离谱”的服务付费,而且他 们的付费价格与服务质量不成比例。上层纳税人付出的利益,基本完全流到了他 人的口袋。在大多数情况下,富人根本不稀罕消费政府的服务,这些服务的质量 一般都很差。几乎所有国家的政府,效率之地下都是出了名的,这主要因为它们 控制在雇员手里,这些人欠缺提高生产率的动力。其实无论从哪个角度看,在工 业时代,缴税最多的纳税人,他们为政府服务所支付的费用,是这些服务在竞争 性市场上真实价值的许多倍。

这一点很少有人注意到。而不幸的是,用莱恩的话说,认识到为政府服务支付的 费用,“按理想标准而言是一种浪费”;这样的洞察在 20 世纪中叶几乎没什么 用处。相反,它只是一个可被接受的缺陷,是“社会组织必然的各种浪费之一。” 对此感到不满的人,他们的选择不是从英国搬到法国,或者从美国搬到加拿大。

这基本没什么区别;主要的民族国家都有同样的缺陷。或多或少,它们都采取了 没收性的税收制度。你如果想大幅度地提高自己的自主权,就必须脱离欧洲和北 美的核心国家,迁移到世界上的边缘国家。在亚洲部分地区、南美洲和一些偏远 的岛国,税收负担明显低很多。但是,逃避掠夺性税收通常也要付出一些代价, 如经济机会的丧失,生活水平一般也会下降。对此我们在前面分析过,在工业时  代的条件下,除了那些肆意施加没收性税负的核心工业民族国家,其他大多数国 家的经济机会都非常有限,生活质量也往往低于平均水平。

以共产主义制度为例。与许多第三世界的政权一样,共产主义国家通常不征收高 额的所得税,甚至根本不征收任何所得税。但是,在苏联存在的四分之三个世纪 里,几乎没有,或很少有企业家去苏联寻求税收庇护。虽然苏联的所得税率不高, 但它没有任何优势,因为苏联不承认财产权。这比税收负担更加糟糕。在共产主 义制度下,人们基本不可能经营企业,并真正赚到什么大钱。就效果而言,共产 主义没收了人们的税前收入。

此外,如果有人通过奇怪的方式获得了稳定的收入,然后选择住在莫斯科或哈瓦 那时,他很难用钱买到体面的生活质量。除了上好的雪茄和鱼子酱、出色的交响 乐团和芭蕾舞之外,前共产主义制度下的生活几乎没什么可消费的乐趣。生活中 大部分稀缺的好东西,用钱都无法买到,或者是基于政治的影响力而严格配给的。

后现代生活的批评者,往往会突出“消费在后现代经验中的重要性”,冒着证实 这种刻板印象的风险,我们要指出,自共产主义垮台以来,全球范围内可获得的 商品数量和服务质量不断提高,必将导致各国之间的竞争更加激烈,从而有利于 削弱国家和地方之间的联系。

在旧的政权之下,消费者的选择如此有限,就连卡斯特罗本人,如果想清理牙齿 上的高斯巴雪茄碎片,也很难买到一包像样的牙线。一直到最近,甚至全世界许 多地方的富人,都享受不到西欧或北美中产阶级日常的生活质量。看到这种悲惨 的境况,大部分的杰出人士都感动地接受了工业时代的民族主义交易。他们待在 原地,缴纳高到离谱的税款,与在他们出生的领土上垄断了暴力的那个民族国家, 交换质量低劣的保护服务。

\begin{tcolorbox}
现在,天堂已经被天使关上了,锁住了,所以我们必须向前走,踏遍世界,看看是否在某 个地方,有一条后路可以进去。
\begin{flushright}
—— 海因里希·冯·克莱斯特(HEINRICH VON KLEIST) 
\end{flushright}
\end{tcolorbox}


共产主义的垮台,拆除了“铁幕”,它曾经阻碍了旅行,阻断了商业的全球化,人为地维持着世界的“大”(相对于现在的小星球,译注)。喷气式飞机,与破 坏共产主义的信息技术相结合,加剧了对高端旅行消费的竞争。即使在最偏远的地区,也有银行家在进进出出,这对全世界的住宿和餐饮标准,都是一种极大的 刺激。而我们所说的,并不是指麦当劳和肯德基,虽然它们的特许经营,在莫斯科和布加勒斯特等以前禁止的地方快速蔓延。更重要的,是领先的连锁酒店与高 质量的餐厅的扩张,这一点很少人注意到。这些地方提供的是特急葡萄酒,而不是伏特加和可乐。这种转变意味着,任何有能力的人,在地球上的任何地方,都 可以享受到高标准的物质生活。事实上,现在很少有哪个国家,没有一家一流的 酒店,或一家能引起米其林调查员兴趣的餐厅。

正如赫希曼在四分之一个世纪前的预测,作为对政府服务的质量与价格感到不满的解决之道,退出离开——得益于技术的进步——其吸引力大大增加了。他写道: “另一方面,我们不需要对国家的忠诚,也可以生活……只有当国家因为通讯进 步与全面现代化而变得大同小异时,才会出现过早和过度退出的威胁,目前的‘人 才外流’就是一个例子。”请注意,我们在第 8 章指出过,赫希曼的“过早和过 度退出”的标准,是出于将被抛弃的民族国家的角度,而不是个人追求美好生活 的角度。

尽管如此,他的结论是无可非议的,即国家之间的相似性会增加反叛与退出的吸引力。现在你去到任何地方,都比过去更容易生活得很好;在这种情况下,你会更愿意去生活成本最低的地方。更重要的是,现在你不仅可以在任何地方都过上 好日子,你还可以在任何地方都获得更高的收入。不是非要住在高成本的国家, 才能积累足够的财富,然后像凯恩斯勋爵所说的那样,过上“明智、愉快且美好的生活”。基于我们已经探讨过的原因,微技术正在改变民族国家赖以生存的大政治基础。在信息时代,全新的网络经济将会出现,它超越任何政府的垄断范围。

有史以来第一次,技术将使个人能够,在一个难以被系统性强制所征服的领域中, 去积累自己的财富。

新的社会,也是新的文化,它的一端将由机器和自动化的作用来定义,机器会在 某些事情上做得比人类更好,自动化将使越来越多的低技能工作消失;它的另一端,则由信息技术赋予真正能够利用它的人的力量来定义。这样的社会将产生更大的张力,一个小阶级(可以称之为信息贵族),和一个不断扩大的下层阶级(可称为信息贫民)。他们之间的区别之一是,信息贫民要么被地理环境所束缚,要么无法从移动中获益。而信息贵族,就像我们在前面说过的,将会到处流动,因为他们有能力在任何自己想去的地方赚到钱;就像受欢迎的小说作家,一直以来都是这样。一百年前,罗伯特·路易斯·史蒂文斯\footnote{《金银岛》、《化身博士》的作者。}(Robert Louis Stevenson)可以在太平洋的一个小岛上谋生;信息贵族现在一样可以。

\subsection{管辖区之间的市场竞争}
由于信息技术超越了地方专制,自然而然地,它会使世界各地的管辖区(即国家,译注),面临事实上的全球竞争,竞争的基础就是政府服务的质量与价格。换句话说,行使着地方领土垄断权的政府,就像大多数其他实体一样,最终会遭遇真正的市场竞争,就看它们怎么服务自己的客户。不可避免的结果是,很快人们就会发现,以前有利于高成本制度的逻辑已经被逆转。领先的民族国家,因为它掠夺性的、再分配的税收制度及严厉的监管,将不再是居住地的首选。理性观察就会发现,它们以垄断性的价格,提供着劣质的保护,并减少了经济发展的机会。

未来的发展可能会证明,它们将比传统上收入更不平等的亚洲和拉美国家更不受欢迎,更加充满暴力。目前领先的福利国家,会因为遗弃逃离而失去最优质的公民。

\subsection{未来的“超国家”(Extranational)时代}

随着“主权个人”时代的成型,许多最杰出的人士,不会再认为自己是任何一个 国家的人,不再是“英国人”、“美国人”或“加拿大人”。一种全新的“跨国 的”或“超国家的”世界观,以及确立自己在世界中的位置的新模式,在新的千 年里等待着人们去发现。如前所述,这种心智模式的早期证据,就是在 MTV 的 青少年观众中,几乎一半人都想离开他们的国家,去追求自己想要的生活。未来 的身份形式,与国籍不一样,它不再是系统性强制的产物;这种强制在 20 世纪 里导致民族国家和国家制度无所不在。

涵盖全球的事态发展通常被人们描述为“国际事务”,这一点本身就已经表明, 民族主义的定式,已经深深地渗透到了构想世界的思维方式之中。经过两个世纪 对“国际关系”和“国际法”奥秘的灌输,人们很容易忽视,“国际”并不是一 个由来已久的西方概念。事实上,这个词是杰里米·本瑟姆(Jeremy Bentham) 在 1789 年发明的;在他的《道德与立法原则导论》(An Introduction to the Principles of  Morals and Legislation)一书中首次被使用。本瑟姆写道:“必须承认,国际这个词 是一个新词,希望它能有足够的可类比性,并且易于理解。”这个词现在非常流 行,而且不仅在本瑟姆原本的狭义层面上。“国际”成了一个万用的近义词,可 以指代全球范围内发生的任何事情。

国际时代开始于 1789 年,与法国大革命同年。它持续了两个世纪,直到 1989 年,欧洲开始反抗共产主义。我们认为,那第二次大革命(指推倒柏林墙,译注) 标志着国际时代的结束,这不仅仅是因为,已声名狼藉的共产主义的主题歌就叫 “国际歌”(The International)。国家所有制的计划经济,是民族国家最野心勃 勃的表现。国家权力与民族主义的密切关系,也体现在语言上。现代时期最具侵 略性的动词就是“国有化”,意思是将某样东西纳入到国家所有和控制之下。在 国际时代,这个动词经常从世界各地的政治煽动家嘴里蹦出来。现在,它已经是 过去的词汇了。国有化已经不合时宜,因为国家权力已经不合时宜。

在现代时期的黄昏,国家的集权,正在被技术创新和市场力量间的互动所削弱。

现在,市场的下一个胜利阶段即将展开。在这个阶段,不仅单个的民族国家开始 解体,在我们看来,就连民族国家的俱乐部——联合国,也注定要破产。如果联 合国在千年之交后的某个时候被清算,我们一点也不会感到惊讶。如果“国际” 概念是一种股票,现在就是清仓的时候。在新的千年里,这个概念可能会被取代, 或至少缩小回它原本的含义;因为未来的世界,不再由连在一起的主权国家的体 系所主导。新的关系会呈现出一种“超国家”的形式,它将由日益重要的微型管 辖区和主权个人说了算。拉布拉多海岸的一块飞地,与一个主权个人之间的争端, 不能再被称之为“国际”争端。它应该是超国家的。

在即将到来的新时代,社区和效忠不再与领土绑在一起。个人身份会更加确切地  建立在真正的亲和关系、共同的信仰、共同的利益和共同的基因之上,而不是民 族主义者特别强调的虚假亲缘关系。保护工作会通过新的方式组织起来,这些新 方式,用六分仪、铅垂线或土地测量员工具箱中其他划定边界的早期现代工具,是解析不了的。

\section{被发明的共同体和传统}

人类一生下来就被置于一个“发明”出来的共同体当中,这个共同体就叫做国家; 这种观念,在下个世纪里,会被世界主义的精英们认为很古怪,又不合理,和在 人类生存的大部分时间里一样。社会学家安东尼·吉登斯(Anthony Giddens)写 到,民族国家“在历史上没有先例”。研究民族主义的权威迈克尔·比利格(Michael Billig),进一步充实了这一说法: 在其他时代,人们并没有语言和方言的概念,更不用说领土和主权的概念了;而在今天它们 是如此普遍,对“我们”来说,又是如此地真实。这些概念已经深刻地嵌入了当代常识之中, 以致于人们很容易就忘记了,它们的永久性是被发明出来的。中世纪在蒙泰罗或圣马特奥的 作坊里干货的鞋匠,在 700 年后的我们看来,也许是狭隘的、迷信的。但他们也会觉得,我 们关于语言和民族的想法是奇怪的神秘主义,他们会搞不懂,为什么这种神秘主义会成为一 个生死攸关的问题。 我们猜测,在超国家的未来,有思考能力的人同样会感到困惑。苯尼迪克特·安 德森(Benedict Anderson)说过,国家是“想象中的共同体”。我们不是说想象 中的东西就一定微不足道。正如约翰逊博士所言,要不是想象力作怪,一个男人 就会很乐意“与侍女睡觉,就像和公爵夫人一样。”不过,对于那些从 20 世纪 走过来的人而言,“国家”好像是一个天经地义的组织单位;要让他们理解国家 是“想象的”而不是自然的,简直就不可能。为了更好理解,未来与我们熟悉的  现在会有多大的不同,有必要看一看,民族主义是如何被强加到工业时代的“常 识”之上的。

人们很容易忽视,“民族共同体”在多大程度上是由持续的想象力加持形成的。

要知道,并不存在什么客观的标准,可以准确界定哪些群体应该是一个“民族”, 哪些不应该是。严格来说,也不存在什么“天然的边界”;就像著名历史学家欧 文·拉蒂莫(Owen Lattimore)和查理斯·惠特克(C. R. Whittaker)所阐明的。

“帝国的主要边界”,拉蒂莫在写到中华帝国时说,“不仅是一条划分地理区域 和人类社会的分界线,它还代表了一个特定社会所能增长的最大极限。”或者像 哥伦比亚大学的经济学家罗纳德·芬德利(Ronald Findlay)所说,“只要在经济 学中思考这些问题,一个特定的经济体或‘国家’的边界,通常被认为是既定的, 生活在边界内的人口也是确定的。但是,很明显,不管这些边界在国际法中变得 多么神圣不可侵犯,它们都曾经在敌对的主张者之间发生过争夺,最终是由各方 经济和军事力量的平衡所确定的。” 一个人即使拥有,世界上一半民族国家的所有数据和高清的卫星地图,也猜测不 到其他民族国家的边界在哪里。也没有任何科学的方法,能够从生物学或语言学 上区分开,一个民族和另一个民族的成员。无论多么先进的解剖技术,都无法从 基因上,把飞机失事的美国人、加拿大人和苏丹人的遗体区分开。国家和民族之 间的界限不是自然的,不像物种之间的界限,或动物品种之间的自然区别。相反, 它们是过去和现在,为了实施权力而锻造的人工制品。

\begin{tcolorbox}
语言就是由陆军和海军护航的方言。  
\begin{flushright}
—— 马里奥·佩 MARIO PEI 
\end{flushright}
\end{tcolorbox}


\section{语言是权力的人工产物}

令人惊讶的是,这种说法也适用于语言。

在民族国家统治世界几个世纪之后,如果说“语言”并不构成区分民族的客观基 础,似乎有点不恰当,甚至是荒谬的。但请再仔细看看。现代语言的历史,清楚 揭示了它们被塑造为加强民族主义认同的工具的程度。我们现在所理解和使用的 西方“语言”,不是自然而然地演变成了今天的形式。在客观上,它们也无法与 “方言”区分开来。在现代社会,没有人愿意说“方言”,每个人都希望自己的 母语被认为是真正的正品,是一种“语言”。


\begin{tcolorbox}
现在,谁也不要再说,言语毫无用途。言语和行动是一体的。抚慰人心的积极话语,具有 强大的能量,它创造了行动——也就是生育了。在此,行动是言语的仆人,顺从跟随其后。
 就像创世的第一天:神说了话,就有了世界。
\begin{flushright}
—— 米什莱(MICHELET) 1792 年 8 月
\end{flushright}
\end{tcolorbox}

\subsection{“语言和行为是一体的”}

例如,在法国大革命之前,法国南部使用的混合版本的拉丁语,即 la langue d’ oc,也就是奥克西坦语;与后来成为法语基础的巴黎方言 la langue d’oil 相比, 奥克语与西班牙北部加泰罗尼亚地区的方言有更多共同之处。事实上,当“人权 宣言”以巴黎风格的语言发表时,生活在目前法国边界内的大多数人都是无法理 解的。法国革命者面临的挑战之一,就是要绞尽脑汁把他们的大字报和法令翻译 成无数村庄的土语;这些村庄之间,也很难理解对方的语言。

生活在后来的“法国”境内的人们,原本讲着各种差异很大的语言,作为一种政 治策略,这些语言被有意识地异文合并,形成了一种官方语言。自 1539 年弗兰  西斯一世颁布《维勒-科特雷茨敕令》以来,书面法语一直是法院的官方语言, 但这并不意味它能被普遍地理解,就像在 1200 年后,“法律法语”在英格兰成 为了法院的官方语言,但能理解它的人不多。它们都属于一种“行政方言”,不 是在全国范围内为人们所使用和理解的标准化语言。

法国革命者希望创造更全面的东西,一种国家语言。历史学家杰尼斯·兰金斯 (Janis Langins)在《语言的社会历史》一书中评论到,“革命者中一个有影响 力的意见团体认为,如果有意识地在共和国的土地上,强行推广一种标准的法语, 就会进一步推动大革命的胜利和启蒙运动的传播。”这种“有意识的努力”包括 在个别词语的使用上费尽心机。例如,米拉波在 1789 年首次使用“革命的” (revolutionary)这个词,就是一个很有说服力的例子。兰金斯指出,(“革命 的”一词)在经历了一段“广泛且不加区分的使用”之后,“在恐怖时期,被压 制和遗忘了一段时间,长达几十年……1795 年 6 月 12 日,国民公会决定改革语 言以及我们的前暴君(即被砍头的罗伯斯庇尔)建立的机构,以在官方称谓中替 换‘革命的’一词。”这种语言工程的传统,在后来针对从英语转入法语的词汇 ——如“周末”等,法国当局的微妙反应中一直延续了下去。

然而,在两个世纪前,法国的国家语言工程师们,不只是歧视来自英吉利海峡对 岸的词汇;他们面临着一项更大的工程,那就是消灭共和国领土内的地方语言变 体。这项工作不仅限于压制奥克西坦语。在当时的里维埃拉地区,人们说的“法 语”比起巴黎的法语,更接近于东边的“意大利语”。同样,阿尔萨斯的语言可 以被归为德语的一种形式,而德语本身也有很多地方性变体。在比利牛斯山一带, 人们说的是巴斯克语。和在法国西北海岸使用的布列塔尼语一样,巴斯克语与作 为法语基础的拉丁语的任何地方土语都毫无共同之处。在法国东北地区,还有相  当数量的人,讲的是佛兰芒语。总之,就像迈克尔·比利格所提示的,“巴黎风 格的法语”,不是通过自发的市场行为传播的,而是“作为‘法语国语’,在法 律和文化上被强加的。” 法国如此,其他民族国家的构建也是如此。语言往往由军队携带,由殖民国家强 加。比如,非洲独立之后的地图,就是根据欧洲列强行政语言占主导地位的地区 来编制的。学校很少教授当地方言。被承认的“语言”和“方言”之间的区别, 在很大程度上是政治性的;前者主要用以界定“国家”,即使是任意划定边界的 殖民国家,后者则不是这样。

简而言之,在世界范围内,强加“民族语言”是巩固国家权力的一部分。在国家垄断暴力的领土上,鼓励或迫使每个人讲“母语”,对促进权力的行使大有裨益。

\subsection{语言统一的军事好处}

在一个暴力回报率不断上升的世界里,采用一种民族语言可以带来军事上的优 势。民族语言,可以说是民族国家巩固中央权力的一个先决条件。中央当局鼓励 公民讲同一种语言,能够削弱地方豪强的军事力量。法国大革命之后,语言的标准化,使得最廉价、最有效的招募现代军力的形式——国家兵役——成为一种可 行的制度。共同的语言,使来自“全国”各地的士兵可以顺畅地交流。这是大规 模兵役取代独立兵营的先决条件,后者通常是由地方豪强征集和控制的,而不是中央当局。

我们在第五章中讨论过,在法国大革命之前,军队是由地方豪强组建和指挥的。

他们可能响应,也可能不响应,来自巴黎或其他首都的战争号召。而无论响不响应,他们的立场都是经过慎重谈判后确定的。正如查尔斯·蒂利(Charles Tilly)所指出的,“给予或撤回支持的能力是……强大的讨价还价的能力。”此外,对 中央当局而言,独立的军事力量还有一个缺点,就是能够抵抗政府对国内资源的 征用。显然,所有的中央当局,无论是国王还是革命大会,都面临一个困难的挑 战,就是从地方势力手中征税或以其他方式剥削资源,而地方势力控制着私人军队,有能力保护这些资源免受剥夺。

“国家军队”极大地增强了,国家政府在整个领土范围内实施其意志的权力。而强制推行一种国家语言,对国家军队的构建起着决定性的促进作用。在国家军队 组建并运转之前,使各个成员能够流利地互相沟通,显然是很有用的。

因此,如果一个辖区内的每个人都能理解命令和指示,并能沿着官僚智慧系统把 情报传递回去,是一个有利的军事条件。这一点的价值,得到了法国革命者立竿 见影的证明。除了开办相当于语言学校的课程班以外,他们还专门开设了为期一 个月的“速成班”;在这些班里,就像兰金斯所写的,“来自法国各地的数百名 学生,将接受火药和大炮制造技术的培训。” 法国人的做法所带来的军事优势,体现在拿破仑时期的成功,以及其他国家的反 例之上;这些例子让人们看到,在战争期间,没有利用共同语言动员优势的政权, 遭到了什么样的下场。在第一次世界大战初期,造成俄军惨败和士气低落的诸多 因素之一,就是沙皇的贵族军官喜欢用德语交流(罗曼诺夫家族的另一种宫廷语 言是法语),而普通士兵根本不懂德语,更不用说俄国公民了。

这点出了共同语言的另一个重要的军事优势;它减少了缺乏打仗动力的障碍。官方进行的宣传,如果人们无法理解,是没有用的。在这一点上,法国革命者也很 好地适应了新的可能性。按照兰金斯的说法,他们的“主导思想”是“人民的意志,因此,他们必须用人民特殊的语言,去表达人民的意志,从而使自己认同人民的意志。”在 1789 年以前,“公民”之间无法理解彼此的语言,是表达“人 民意志”的一大障碍,也是在国家层面行使权力的一种制约。在工业化时期,多语言国家或帝国在战争动员方面,面临着更多挑战。

因此,在边缘地区,它们往往会被更善于激励公民作战和调动战争资源的民族国家所淘汰。民族主义的发展和巩固就是很好的例证,如 18 世纪末法国和法国人 等概念的发明。而民族主义的分权也证明了这一点,如一战后奥匈帝国的崩溃。

哈布斯堡帝国分崩离析之后,出现的新民族国家,如奥地利、匈牙利、捷克斯洛 伐克和南斯拉夫,用凯恩斯的话说,是“不完整与不成熟的”。但是,他们围绕 着民族身份——至少部分是由语言定义的——建立独立的民族国家的主张,说服 了伍德罗·威尔逊(当时的美国总统,译注)和其他的盟国领导人,从而起草了《凡尔赛条约》。

第一次世界大战后中欧的分裂,说明语言在国家构建中是一把双刃剑。当暴力的 回报不断上升时,共同的语言促进了权力的行使和巩固。而当巩固权力的动力减 弱时,少数族裔围绕着语言争端会分帮立派,这往往会导致多语言国家的分裂。

19 世纪中叶的奥匈帝国,在一场流行病夺走了大量的德语人口之后,多个城市都掀起了分离主义的高潮。在 19 世纪的开端,布拉格还是一个德语城市。和其他欧洲城市一样,在进入 19 世纪后,它也迅速发展起来;主要是通过农村来的 移民,大量无地的讲捷克语的农民被同化。一开始,新来的移民觉得有必要学好 德语,以便更好地与人相处,于是他们就学了。但到 19 世纪中期,饥荒和瘟疫 带走了大量将德语的城市居民,取而代之的是讲捷克语的农民。突然之间,讲捷克语的人多了起来,新移民不用再学德语了。布拉格成了一个捷克语城市,也成 了捷克民族主义的温床。

当代的分离主义运动,主要围绕着语言的争执,在多语言国家展开。比利时和加 拿大显然就是这样,我们在前面提到过;在新的千年里,这两个国家可能是经合 组织中最早解体的。加拿大政府很难越过魁北克省的魁北克党,强行实施语言统 一的高压行动。更出人意料的是,在意大利北部的分离主义的早期活动中,语言问题也是一大诱因;意大利同样面临着解体。1980 年代初,当时的伦巴第联盟, “宣布伦巴第语是一种独立于意大利语的语言。”比利格评论说:“如果联盟的 方案在 80 年代初获得成功,伦巴第从意大利分离出去,建立起自己的国家,那 我们可以预测:越来越多的伦巴第人会认为他们不是意大利人。”这个说法并非 毫无根据,历史上发生过类似的情况。例如,在 1905 年挪威独立后,挪威民族 主义者齐心协力,去划定并强调“挪威语”与丹麦语和瑞典语的不同之处。同样, 白俄罗斯的独立主义活动家,把路标都换成了“白俄罗斯语”;但很显然,他们没能证明白俄罗斯语是一种独立的语言,而不是俄罗斯的方言。

现在,有利于语言统一的军事需求基本不存在了,我们预计国家语言将逐渐消失, 但一场恶战是免不了的。可以想见,“战争使国家强健”这句被反复打磨过的俗 语,会重新复辟,考验人们的选择。随着民族国家变得无足轻重,煽动者和反动 分子会鼓动战争和冲突,掀起种族和部落战争,就像在前南斯拉夫,以及从布隆迪到索马里等很多非洲国家一样。冲突将是很容易挑起的,因为它可以为那些想 要阻止主权商业化趋势的人提供借口。战争有利于维持更严苛的税收,并对那些逃避公民义务的人施加严厉的惩罚。战争还有助于加强民族主义“他们,和我们” 的维度区分。对于系统性强权的支持者来说,主权商业化,个人可以根据质量和 价格选择主权服务,简直罪恶深重;其罪不亚于在宗教改革期间,个人主张否决教皇的判决,并要自己选择救赎之道。

这两种罪之间的相似性,也基于这一事实:15 世纪末的印刷术也 20 世纪末的信 息技术,都解放性地将以前神秘的知识,交到了个人手里。印刷术将圣经和其他 圣典直接带到了个人身边;而在那之前,人们必须依靠牧师及教会上层,由他们来解释上帝的话语。同样,新型信息技术是任何有电脑联网的人,都可以获得关 于商业、投资和时事的信息,而在过去,这些信息只有政府和企业高层才能获得。


\begin{tcolorbox}
印刷术和出版业的发展,使新的民族意识成为可能,促进了现代民族国家的兴起。
\begin{flushright}
—— 杰克·韦瑟福德(JACK WEATHERFORD)
\end{flushright}
\end{tcolorbox}

\subsection{赛博空间的摇滚}

毫无疑问,互联网对民族主义的破坏力,与火药和印刷术对民族主义的促进作用 一样强大。全球计算机的互连,不会使拉丁语重新成为一种通用语言,但它有助 于把商业从地方方言的范畴,转移到新的、互联网时代的全球语言中去,那是奥 迪斯·雷丁(Otis Redding,美国灵魂乐歌手)和蒂娜·特纳(Tina Turner,摇滚 女王)教给世界的语言,那是摇滚乐的语言,那就是英语。

新的网络媒体,可以创造超越地理界限的新型亲和关系,从而削弱民族主义。它 们将吸引分散在世界各地的受众,无论这些人在哪里,只要他们受过教育,在网 上碰巧发现了彼此,就会形成这种关系。这些新的、非地域性的亲和关系将会蓬 勃发展;在发展过程中,它们可能会创造一个关于“爱国主义”的新视点。或者 说,它们会形成一种新的“内群体”(ingroup),个人可以在不牺牲经济理性的 情况下,认同这些团体。犹太人在过去两千年的历史表明,在长期面对敌对的地 方条件下,这种团体是有可能达成的。正如本章开头引用的威廉·普法夫的评论 所说,认为忠于自己祖先的土地,必然意味着要忠于类似民族国家这样的结构,  这种认识既不符合历史,也是错误的。杰弗里·帕克(Geoffrey Parker)和莱斯 利·史密斯(Lesley M. Smith),在《十七世纪的普遍危机》(The General Crisis of the 17teenth Century)一书中,更清楚地阐明了这一点。他们指出,那些看似 是早期民族主义的例子,其实更多的是爱国者(patriot,这里翻译成爱国者其实不 太妥,译注)在捍卫一块有限的家园(patria),而且往往是为了抵抗国家的吞 并。他们写道:“很多时候,所谓的对一个民族共同体的忠诚,仔细考察后就会 发现并非如此。家园本身和整个国家一样,完全有可能是故乡的一个城镇 (hometown)或省份。” 在《野蛮人与文明》(Savages and Civilization)一书中,杰克·韦瑟富德(Jack Weatherford)清晰地解释了,印刷术作为第一种大规模生产的技术,它的出现, 对政治的诞生发挥了巨大的作用;而政治要求人们效忠于一个更宽泛的民族国 家。到 1500 年,欧洲 236 个地方都有印刷厂,“它们总共印刷了 2000 万本书。” 古腾堡(印刷术发明者)印刷的第一本书,是一本拉丁文的《圣经》。随后,他 又用拉丁文出版其他流行的中世纪书籍。韦瑟富德解释说,印刷术的发展方向打 破了一些人早期的期望,他们本以为,随手可得的书籍会使拉丁文甚至希腊文得 到普及。事实却恰恰相反。印刷术没有促进拉丁文的使用,有两个重要的原因。

首先,印刷术是一种大规模生产的技术。正如本尼迪克特·安德森所指出的,“如 果说手稿知识追求的是稀缺的、高深的学问,那么印刷知识的生存之道,则是可复制性和传播性。”在 1500 年的欧洲,没有多少人能使用多种语言;这意味着 拉丁文作品的受众不是大众读者。当时绝大多数人都只懂一门语言,他们构成了一个更大的潜在阅读市场。此外,读者如此,作者更是如此。而出版商需要能卖 得出去的作品。在十五或十六世纪,当时的作家很少能用拉丁文创作出令人满意的新作品,出版商在市场需求的趋势下,开始出版地方语言的作品。因此,印刷术帮助把欧洲区分成了不同的语言区。它的促进作用,不仅是通过出版新的作品, 确立新语言的独特性,如西班牙语和意大利语;而且通过采用特色的字体,如罗马体、斜体,以及直到 20 世纪初德国出版业普遍采用的厚重的哥特体。新的方 言出版业,即安德森所说的“印刷资本主义”,大获成功。最值得注意的是,印刷术对异端思想的传播,起了决定性的推进作用;我们期待互联网对个人的非国 有化,也会产生同样的推进力。尤其是,“路德成为了第一个为大众所知的畅销 书作家,或者说,他是第一个能凭自己的名字‘卖’出新书的作家。”令人震惊 的是,路德的作品占到了“1518 年到 1525 年间所有德语书籍销量的 1/3 还多。” 在很多方面,信息时代的新技术将会对抗 15 世纪的新技术——印刷术,后者对 刺激和支持民族国家崛起发挥了部分大政治影响。万维网创造了一个商业场所, 它使用一种全球语言,即英语。它最终会被同声传译软件所加强,几乎人人都将可以顺利地使用多种语言;这也有助于语言和想象力的去国家化。印刷术破坏了对中世纪的统治机构——圣母教堂的忠诚,同样,我们期待信息时代的新技术摧毁保姆国家的权威。当时机成熟,基本每个地区都会成为多语言的地区;方言的重要性将上升。来自中央的宣传将失去大部分的凝聚力,因为移民和讲少数民族 语言的人敢于抵抗国家的同化。

\section{军事神秘主义}
从同样的角度看,“狩猎采集部落”是客观的共同体,而国家则远远不是;国家是在一种神秘主义的启发下被想象出来的,而这种神秘主义又是由一种已经不复 存在的军事需求激发出来的。那是一种被认为比生命本身还重要的身份需求,它将生活在同一领土内的每个人联系起来。康托洛维茨(Kantorowicz)说,“在历 史的某一时刻,抽象的国家或作为公司运营的国家,成为了一种神秘主义的团体; 而为该团体牺牲,其价值简直相当于为上帝而死的十字军战士!”从这个意义上, 可以把民族国家理解为一种神秘的构建。不过,正如比利格所指出的,民族主义 是“一种平庸的神秘主义,而且它是如此平庸,似乎所有的神秘色彩早就蒸发了”。

它“把‘我们’与祖国绑在一起,祖国是如此的特殊,它不仅仅是一个地方,地球上的一块物理区域。通过这一切,祖国被赋予一种家国情感,它不容质疑,当有需要的时候,完全值得为之而牺牲。而男人,尤其是男人,收到了特别的、充 满快感的提示,提示他们可能的牺牲。” 民族主义者只要一有机会,就会强调国与家之间的想象性联系。就像比利格所说 的,国家被“想象成温馨的空间,人们在边界内舒适、安全地生活,并把危险的 世界挡在外面。而‘我们’这个在祖国内的民族,可以很容易地把‘我们自己’ 想象成某种家庭。”民族主义者不知疲倦地重复着这些陈词滥调,还有许多关于 亲情和身份的庸俗隐喻。他们把国家与个人的一种“整体适应性”(inclusive fitness)意识联系在一起,这种意识会激发强大的利他主义和牺牲动机。


\begin{tcolorbox}
牺牲性的利他主义,确实存在于社会性昆虫、其他非人类动物和人类当中,这意味着自我 利益的最大化,不能仅由单个有机体的欲望和需求来定义。事实上,利他主义的存在,尤其 是对粉丝的利他主义,要求人们对生物科学中传统的适者生存观念,进行全面的重新思考。这也使得越来越多的人相信,自然选择的法则最终不会适用于个人。
\begin{flushright}
—— 保罗·肖 王玉华(音译)R. PAUL SHAW and YUWA WONG  
\end{flushright}
\end{tcolorbox}

\section{民族主义与整体适应性}

在本书中,我们主要关注的是,改变人类选择的成本与回报的客观“大政治”因素。分析的预测能力,依据的基本前提是,个人会追求回报而规避成本。这是查 尔斯·达尔文所说的“自然经济”的一个基本真理;但这并不是全部的真理。简单的回报优化并不能解释生活中的一切。不过,它确实阐明了人类社会性三种互 动方式中的两种,皮埃尔·范登伯格(Pierre Van Den Berghe)称之为“互惠和胁 迫”。范登伯格所说的“互惠”是指“互利合作”,最复杂、影响最深远的例子, 就是市场互动:交易、买卖、生产和其他经济活动。“胁迫”则是为了单方面的利益,即为了同物种间的寄生或掠夺而使用暴力。正如在本书及前两本书中所论述的,我们认为,胁迫是人类社会活动中的一个关键因素,它比人们通常所认为的要更加重要。胁迫可以决定财产的安危,可以限制个人进行互利合作的能力。

胁迫是一些政治的基础。范登伯格的人类社会类型学中的第三种互动方式,是“亲 缘选择”(kin selection),即动物与它们的亲属之间展开的合作行为。下文会更 详细地阐述亲缘选择,它也是“自然经济”的一个重要特征。

杰克·赫舒拉发(Jack Hirshleifer)写到,“达尔文的进化选择理论被复兴,被应用于社会行为问题的研究,已经形成了一门社会生物学”,它具有“明显的经 济学属性”;而且: 纵观所有的生命领域,社会生物学试图找到,决定生物体之间种种联系的一般规律。例如: 为什么某些生物有性别有家庭,有的只有性别没有家庭,有的既没有性别也没有家庭?为什 么有的动物成群结队,有的则单独活动?为什么在有的群体中,我们可以观察到等级支配模 式,有的却没有?为什么有些生物会划分地盘,而另一些则不会?是什么决定了社会性昆虫 的无私奉献,为什么这种模式在自然界非常罕见?在哪些情况下,资源的分配是和平的,什 么时候是通过暴力的手段?这些问题都是用公认的经济学术语提出和解答的。

社会生物学家问的是,观察到的这些关联模式,对呈现相应模式的生物体的净优势是什么, 这些模式在社会平衡状态下持续存在的机制又是什么?社会生物学认为人类与其他生物之  间,存在经济行为的一致性。可能正是因为这种主张(被一位反对者称为“遗传资本主义”), 让一些意识形态者对社会生物学充满敌意。 我们将把社会生物学引入到对民族主义的分析中去,因为它提供了一种视角,可 以帮助我们理解,为什么人性会促进系统性的胁迫。我们同意自然科学家科林·塔 吉(Colin Tudge)的观点,他是《历史以前的时间》(The Time Before History) 一书的作者;他认为在我们理解当前的世界之前——先不要说对未来有什么看 法,我们需要了解历史的前言部分。

这意味着我们必须“在宏大的时间尺度上审视自己”。塔吉提醒我们,“在我们 动荡的生活表层之下,有更深层、更强大的力量在发挥作用,最终影响到我们所 有人和所有的同类……”。我们怀疑,在这些“更深层、更强大的力量”中,可能有一部分是受基因影响的动机力量,为民族主义提供了支撑。正如赫舒拉发所 指出的,套用亚当·斯密与罗纳德·科斯(R. H. Coase)的话说,“人类的欲望, 最终是由人类的生物本性和在地球上的处境所形成的适应性反应。”在大多数关 于民族主义的讨论中,这一观点因为明显的生物学意味而引人注目。毕竟,即使 在美国,这个显而易见的多民族国家,政府也被人格化为家族式的“山姆大叔”。

\subsection{生物学遗传}
简单来说,在理解人类社会的持续演化时,要考虑到人性、物种起源以及它们在 自然选择下的发展等因素。在当前的情况下,我们思考的是,对信息技术带来的 新环境,人类可能做出的反应。我们尤其关注人们对网络经济及其后果的反应, 包括前所未见的经济不平等。在这些预期可能发生的反应中,至少有一部分,其 关键在于我们遗传的基因。

当一个新的物种形成时,它并不会抛弃在以前结构中携带的所有基因,而是会增 加它。人类和黑猩猩之间的全部差别,只包含在各自不到 2\%的 DNA 里;超过 98\%的 DNA 是两者共同的,其中一些 DNA 可以追溯到非常原始的早期生物,比 历史发展的链条要远得多。

\section{遗传惰性}

人类文化同样包含着普遍的因素,其中一些确实可以是遗传自前人类的祖先。我们如何寻找事物,如何交配,如何组成家庭,如何与陌生群体打交道,如何保护 自己,这些都是本能与文化的复杂混合物,有着非常原始的根源。它们也能够进 行现代化的调整,比如现在时期民族国家的特点。如果从这个角度思考文化,我们可以把它看作是与遗传平行发展的。但它们之间有三个重大的区别:文化是通过人与人之间的信息链传播的,而不是通过代际之间的基因链;文化在一定程度 上——可能比我们想象的要容易——可以通过有意识的智能活动来改变;文化会随着决定成本与回报的环境条件发生改变,这比基因变化要快得多。从生理上看, 我们与三万年的祖先还非常相似;但从文化上,我们离他们已经很远了。

\subsection{进化模式}

关于物种的进化,有两种生物学模式。科学的正统观念是新达尔文主义。按照这 种理论,随机的基因变化产生出不同的自然种类。大多数种类不具备生存优势, 如白化黑鸟,它们往往会消亡。少数对生存有帮助的,会在物种间传播。这个理 论还面临着很多难题,也许会在下个世纪被科学家们解决,但随机性和有利于生 存的适应性,是目前科学界的正统观念,具有一定的解释力。另一种模式是 20  世纪初法国哲学家亨利·伯格森(Henri Bergson)理论的一些变体,他认为自然 界的创造具有一些非随机性的目的,是一种寻求解决方案的智能力量。这一理念, 在当代权威人士如大卫·雷泽(David Layzer)和斯蒂芬·古尔德(Stephen Jay Gould) 的工作中,得到了回应;他们强调遗传变异不是简单随机性的,而是呈现出明确 的倾向性。这不是严格意义上的圣经创造论,但它避免了正统达尔文主义的许多问题。

\begin{tcolorbox}
社会生物学理论的重大贡献,是将适应性(fitness)的概念扩展为了“整体适应性”。动物的确可以通过自身的繁殖,直接复制其基因;也可以通过与它共有特定比例基因的亲属进行繁殖,来间接复制其基因。因此,我们可以预料,动物之间会有合作行为,从而在其基因相关的范围内增强彼此的适应性。这就是亲属选择的涵义。简而言之,动物具有裙带关系,它们喜欢亲属而不是非亲属,喜欢近亲而不是远亲。这可能是有意识的,像人类就是;但更多的是无意识的。
\begin{flushright}
—— 皮埃尔·范登伯格(PIERRE VAN DEN BERGHE) 
\end{flushright}
\end{tcolorbox}


\section{受基因影响的动机因素}

1963 年,威廉·汉密尔顿(W.D.Hamilton)在他的论文“利他行为的演化研究” 中,提出了“整体适应性”的概念,为人类行为的研究增加了生物学视角。哈密 尔顿认识到,虽然人类被赋予的行为模式,从根本上是自我导向的,但他们偶尔也会从事一些利他主义或自我牺牲的行为,这些行为并不能给个人生活带来什么 直接的好处。汉密尔顿试图调和这些明显的矛盾,他提出来,需求利益最大化的 基本单位不是生物个体,而是基因。任何物种中的个体都会寻求利益最大化,但 不仅仅是为了它们自身的福祉,而是汉密尔顿所说的“整体适应性”。他认为, “整体适应性”不仅包括达尔文意义上的个体生存,还包括可以增强具有相同基因的近亲属的繁衍和生存。汉密尔顿的“整体适应性”理论,有助于解释人类社 会中许多奇怪的特征,包括民族国家的政治表现。

\subsection{利他主义:是误会还是顽固不化的亲属选择?}

范登伯格认为:“利他主义主要适用于亲属,尤其是近亲属,所以,事实上,它是一个错误的说法,用词不当。它其实代表了终极的基因自私性;只是人们没有认识到最大化的整体适应性,所以使用了一种盲目的叫法。”不过,这也不等于 说,如果不存在汉密尔顿和范登伯格所说的近亲遗传关系,就不存在利他主义。

人类是有性繁殖的,而不是无性克隆,这本身所带来的不确定性,在“整体适应 性最大化”的遗传倾向下,几乎肯定会激发大量的利他主义行为,使“自私的基 因”以外的等位基因获益。首先来说,有一种情况,总是有可能发生的,就是有 些人在帮助他人时,他可能是误以为是在帮助自己的近亲属。为后代而牺牲的父亲,可能并不是他们的生父,只是自己不知道。这不仅是一个肥皂剧题材,它还 解释了一个原始的谜题:如果每一个表面上的父亲,表现得“好像”是真正的父 亲——即使它可能不是,这也很可能促进“自私的基因”的幸存。

不过,正如赫舒拉发所指出的,从正确的角度看,“利他主义”的诸多悖论,都是由于语义上的模糊,它们经常迷惑或误导人们忽视竞争的背景,在这种背景下, 实施“帮助”更有助于生存:“在生存的竞争当中,如果利他主义的策略选择要 胜过非利他主义,那么利他主义对自我生存的贡献就要超过非利他主义,这样的话,它就不可能真的是利他主义。这真的是一团浆糊。但是,如果我们抛掉‘利他主义’这个词,转而问道;可以被完全客观地称之为‘帮助’的现象,它的决 定因素是什么?这一切就清晰多了。”  可能在“帮助亲属”时,问这个问题最有意思。在汉密尔顿关于整体适应度的基 本表述中,涉及到一种生物学上的成本效益分析。根据这种分析,一个个体,或 者说“控制帮助行为的基因”,会把一个与自己相同的副本的生存,放在与自己 的生存同等重要的位置。因此,一个个体是否愿意提供帮助,甚至为他人牺牲, 它的意愿程度,会随着另一个个体拥有相同基因的比例而变化。“具体来说,负责亲属帮助的基因,会指示一个人(在其他条件都相同的情况下),如果他能拯 救两个兄弟姐妹、四个同父异母的兄弟姐妹、八个堂兄弟姐妹,他就会献出自己的生命。” 

\section{整体适应性的概率问题}

这种生物逻辑在原则上好像很清楚,但仔细研究后会发现,它掩盖了一些问题。

例如,一个人的兄弟姐妹和他的子女,可能有 50\%的概率具有相同的基因,但按 照严格的逻辑,这并不意味着这种情况会真的在他们身上发生。每个人身上都携 带有两套基因,一套来自父亲,一套来自母亲。当然这意味着,但也只意味着, 父亲或母亲个体所携带的基因,只有一半会必然遗传给后代。另外,生殖过程中 也一直存在着基因突变,虽然概率不但,但也降低了基因成本效益分析的确定性。

所以,如果认真思考“基因是优化器”这个隐喻,就会意识到它存在很多方面的 问题,那种不是后代父辈的繁衍,就是一个再清楚不过的例子。如果“自私的基 因”确实可以通过为近亲牺牲而优化生存,那么,任何导致另一个等位基因替代 “自私基因”的相同拷贝的可能性,都可以看做是大自然母亲对自己玩的复杂戏 法之一。

\section{不确定的后果}
因此,针对亲属的利他主义是有问题的。对于“自私的基因”来说,不仅存在着 概率的问题,也就是说,它的宿主表面上的亲属,事实上可能并不具有相同的基 因副本。还有一个问题是,在不确定的条件下,如果做出特定的牺牲,怎么确定 最终的实际受益人主要是自己的亲属,而不是其他人。(如果主要受益人是其他 人,那这样的牺牲实际上可能会损害自私基因的整体适应性,因为它降低了自己 在后代族群中出现的可能性。)举一个可怕的例子,这是我们在写作的时候,受 到新闻的启发而想到的。假设苏格兰邓布兰地区的一位家长,接到了紧急通知, 有一个武装分子冲进了当地一所学校,显然是要搞伤害活动。如果这个家长立刻 行动,他或她可以采取英勇但也许注定会失败的行动,与疯子对抗,从而有可能 救出他们在学校的孩子。

也可能救不出。

即使是一个残忍的变态,想要杀死地球上所有的孩子,但他也可能打光弹药,或 被他人制服,在此之前,他所能造成的杀伤也是有限的。不管怎样,如果做出牺 牲的父母决定不去救,他的孩子也很有可能活下来,就像学校里大多数孩子一样。

在这里,勇敢的牺牲行为所避免的伤害,可能主要使其他孩子受益。所以,当事 的父亲或母亲冒着生命危险,主要是为了他人的孩子,这实际上降低了他们自己 的“整体适应性”。由于葬送了他们所有孩子的父母之一,可能会让这些孩子在 达尔文式的竞争中处于更不利的位置。

显然这是一个很压抑的例子,但它也是现实的。它反映了一个事实:在生活中无 数的情况下,大大小小的帮助行为都会产生有益的影响。在很多时候,这些行为 的直接受益者,都无法被局限在关系密切的亲属身上。而讽刺的是,这可能会使  那些具有较少歧视性帮助基因的人,获得部分生存利益,使他们能够经受得住迄 今为止数千年来的所有磨难;我们会在下文对此进行阐述。

\subsection{利他主义与遗传惯性}

我们认为,如果“自私的基因”理论,是对人类行为动机的准确模拟,那么,认为由它所引起的帮助或牺牲行为,可以狭隘地、仅仅为了实际亲属的利益而实施, 就未免太简单了。在某些情况下,由于不完全知情,区分亲属是一门不确定的艺 术。而且,即使亲属是已知的,对于特定的“自私的基因”,它在亲属人群中的 实际分布也不能确定,这不仅仅是个概率问题。直到最近,还无法区分出个体之 间真实的遗传标记。而要在实践中区分出,哪些近亲属代表了哪种正在优化生存 的“自私的基因”,我们距此还有一段距离。除此之外,要将利益限制在亲属而非他人身上,是一个更大的难题。

还有,从经验中也可以看出,如果没有合适的亲属,人类有时候会把自己的“养 育本能”转移到非亲属的利益上。最明显的例子就是父母领养孩子的行为,或者 某些人(通常无子女)对家中宠物的行为。这些人为了营救困在树上的猫咪,不 惜受到重伤甚至死亡;这样的事时有发生。在每一年中,也有不少人因为家庭事 故而丧失,这些事故都是由掉入险境中的宠物,以某种方式造成的。对宠物如此, 对养子女更是如此。可以肯定地说,领养孩子的父母对待他们的方式,就“好像” 对待亲属一样;这就赋予了“亲属选择”的概念另一种含义。

这样的例子,并不像一些批评者所希望的那样,会否定“自私的基因”理论。恰 恰相反。我们看到,人们通过“好像”是为近亲属做牺牲的行为,增加了自己的 整体适应性;这些例子说明了“基因惯性”的现象。换句话说,它们反映了,霍  华德·马格里斯(Howard Margolis)在《自私、利他主义和理性》(Selfishness, Altruism,and Rationality)一书中所指出的事实,即“人类社会的变化”比人类基 因的进补(makeup)要快。因此,现在人们的行为方式,仍然“像是生活在一个 小型的狩猎采集群体当中。”就像范登伯格所说,这类群体的一个主要特征就是: “他们是由几百人组成的、近亲繁殖的小规模团体。部落的成员,虽然被细分为较小的亲属 群体,但他们把自己看成一个单一的种族,团结在一起,抵御外面的世界;他们通过亲属关 系和婚姻网络互相联系,使部落成为了一个事实上的超级家庭。高度的近亲繁殖率,保证了大多数配偶也是亲属。” 简而言之,在农业出现之前的所有人类历史中,族群(ethic groups)都是“近亲 繁殖的超级家庭”。由于过去的家庭与族内人(in-group,或叫内群体,译注)之间的认同感,很可能存在一种受基因影响的行为倾向,就是把族内人当成亲人。

不难想象,在古时候,当“近亲繁殖的超级家庭”中的每个成员都是亲属时,这 种行为是具有生存价值的。正如马格里斯提到的,很容易想象,对于“这样小型 的狩猎采集群体,彼此之间关系密切,只靠这种整体性的自私(不存在互惠或复 仇的问题),就可以独立支撑起对群体利益的保障。那么,我们可以认为,某些 幸存至今的倾向于群体利益的动机,其实是一种顽固不化的亲属利他主义(fossil kin-altruism)。”换句话说,由于我们保留了狩猎采集者的基因构成,我们对内 群体的行为反映了一种“利他主义”,这种利他主义有望优化由“近亲繁殖的超 级家庭”组成的内群体的生存与成功。

马格里斯推测,这种为群体利益付出的行为倾向——起因于“顽固不化的亲属利 他主义(fossil kin-altruism)”或基因惯性,很可能帮助了智人的生存,“而其 他的类人物种则灭绝了。” 

\subsection{表观遗传学(Epigenesis)}

我们认为这种“好像”的行为,是“表观遗传”的一个典型例子;或者说,受到遗传影响的动机因素,会使人先天性地倾向于某些选择而非其他选择。换言之, 人类的心智并不是一块白板,或一张白纸,而是一个带有预设电路的硬盘,它使 某些反应比其他反应更容易被学习和受欢迎。因此,有一种说法认为,人类的心 智倾向于从两个角度思考问题:一个是能激发敌意或敌视的外群体;一个是让人 感到友好或忠诚的内群体,而这种忠诚通常是留给亲属的。

这种对待内群体就像对待近亲属一样的表观遗传倾向,使人类非常容易受到操纵 的影响;而这一点又往往会被民族主义者所利用,去鼓动人们支持国家,甚至不 惜付出生命。由此来看,民族主义的宣传,到处都在用充满亲情的词汇来装扮, 绝不是一个巧合。

\begin{tcolorbox}
炮声惊天动地,美丽的法兰西呼唤她的孩子们奋起。周围的士兵正武装起来,呐喊吧,呐 喊吧,我们的母亲在呼唤。
\begin{flushright}
—— 法国士兵的军歌
\end{flushright}
\end{tcolorbox}

\subsection{虚假的亲属关系}

世界各地的政客,都有一种强烈的偏好,就是喜欢用亲属关系的术语来描述国家。

国家是“我们的父土(fatherland)”或“我们的母国(motherland)”。它的公民是“我们”,是“家庭成员”,我们的“兄弟姐妹”。像法国、中国和埃及这 样的国家,文化千差万别,但都使用这样的比喻;在我们看来,这可不是修辞上 的巧合,而是“表观遗传”的一个典型例子,或者说,受遗传影响的动机因素会 使人类先天地倾向于某些选择。

表观遗传是如何操作的?民族国家为了驾驭人民的情感忠诚,通过各种方式建立 起了身份识别机制;这些方式在原始社会原本属于亲属关系的象征,却被用来将 “个人对整体适应性的关切”与国家的利益联系在一起。例如,肖和王通过深入 研究,总结了现代民族国家用来动员民众对抗外族的五种识别方式,它们是:
\begin{enumerate}
    \item 共同的语言
    \item 共同的祖国
    \item 共同的外形特征
    \item 共同的宗教遗产
    \item 共同血统的信念
\end{enumerate}

当然,通过这些特征,在原始时代就可以区分出核心的民族群体。而民族主义的吸引力,在很大程度上可以归因于,这些识别手段被运用的方式,以及亲属语言的乔装改扮,像上文引用的法国军歌。把国家称为“父国”或“母国”的动员手段,在世界范围内都很普遍,因为它效果卓著。

\subsection{基因计算}
这种涉及国家的亲属关系,是凭空虚构出来的,因为它根本不具备实际亲属关系所具有的那种亲疏远近。即使是一个很大的家庭里,每个人都是亲戚,但他们之间的亲近程度是不一样的。父母和兄弟姐妹是最亲的,祖父母和表兄弟姐妹就没那么亲了,远亲的、联系稀疏的表兄弟姐妹关系已经非常遥远,他们之间拥有共同基因的可能性跟陌生人差不多。丈夫和妻子之间的基因联系,也不再像石器时代那样密切。不管怎么样,所有真实的亲属关系,都可以用一个“亲缘系数”(coefficient of relatedness)的数学术语来定义,汉密尔顿把它作为计算和衡量基因重叠程度的标准。

相比之下,想象中的民族“大家庭”,它与国家的领土范围完全重合,并且是弹性的。国籍就像液体一样,均匀地渗透到严格限定的边界内的每一条缝隙。本尼迪克特·安德森写道:“在现代的概念中,国家主权在法定领土的每一平方厘米上的运行,都是完全的、绝对的、均匀的。”当然,在需要为国家做出牺牲时,想象出来的亲缘系数永远是 1。

整体适应性与民族国家的密切关联,这种现象很有意思,因为它可以帮助我们了解,面对即将到来的新千年的变革,人们的意愿取向是怎么样的,是欢迎还是抵制?我们在前面说过,在信息时代之前,所有类型的社会都是以领土为基础的。

它们要么是围绕着核心族群的家乡领土而形成;要么是像民族国家一样,利用群体团结的共同动机,动员力量保护地方领土,抵抗外来者。不管是哪种情况,领土一步之遥外的陌生人,都被视为可怕的敌人。考虑到原始时代亲属选择的设想,这也不无道理。毕竟,当人类以目前的基因类型出现时,部落里的成员都是近亲,他们都归属于一个核心的种族群体,一个“近亲繁殖的超级家庭”。

此外,关于亲属选择的必要性,即个人将直系亲属的生存与繁荣,与部落或超级家庭的繁荣昌盛捆绑在一起,确实有实际的经济意义。狩猎采集部落的成员,确实要依靠这个部落的成功来实现自己的繁荣。他们没有独立的财产,个人或家庭如果脱离部落,根本没有办法生存与繁衍。在这种情况下,个体利益与群体利益就紧密地捆绑在了一起。用赫舒拉发的话说,“只要一个群体内的成员承担着共同的命运或结果,互助就成了自助。”

\begin{tcolorbox}
显然,原始人——洛夫图人可被视为数百个类似民族的代表——认为,一个社会的标准应该是,任何人在任何时候的处境都是完全平等的。  
\begin{flushright}
—— 赫尔穆特·舍克(Helmut Schoeck) 
\end{flushright}
\end{tcolorbox}


\subsection{新环境,老基因}
在当今时代,微处理技术正在加速创造新的环境条件,与人类在石器时代的基因所处理的条件大不相同。信息技术导致的收入不平等,远非我们在平等的石器时代的原始祖先所能想象。信息技术也正在创造超地域的资产,这有助于颠覆内群体的现代化身——民族国家。具有讽刺意味的是,新的网络资产可能具有更高的价值,而这正是因为它们建立在离家很远的地方。如果在富裕的工业国家,出现我们所预期的那种对信息技术日益渗透导致的经济不平等的反击,上述发展只会更加强烈。这个事实本身,就会使远程持有资产变得更有价值。这种做法不仅不会遭人嫉妒,还可以将资产置于民族国家控制的范围之外;而民族国家是个人必须应对的最具掠夺性的群体。

\subsection{大自然与民族主义的不经济性}
很少有人注意到,把对内群体的认同与民族国家联系起来,是多么的讽刺;而这种忽视,可能也是表观遗传在影响态度方面的重要性的一个标志。现代时期的暴力逻辑,往往从一开始就混淆,那种激发了认同内群体适应性的倾向的冲动。此话怎讲?因为在一个敌对的世界里,将个人的“整体适应性”与一个国家级内群体认同在一起,不仅无法促进近亲属的生存与繁荣,反而会把个人牺牲的价值,稀释到对亲属无足轻重的程度。一个典型的现代民族国家,实在过于庞大;这种国家的公民,与任何向他提出要求的其他公民之间,都不太可能具有统计学意义上的“亲缘系数”。国家内群体中的近亲比例,从石器时代的几乎全部都是,到20 世纪已经锐减至用化学痕迹法基本检测不到了;不仅如此,个人与国家其他公民之间的“亲缘系数”,在大多数情况下,都不会比他与整个人类的系数更高。

一个拥有数千万甚至数亿(中国甚至有十几亿)成员的内群体,使任何人做出的牺牲或利益付出,相比它的超级体量,不过是沧海一粟。因此,从严格的逻辑上讲,现代的民族主义者,与石器时代的狩猎采集者不同,他不可能合理地期望,通过牺牲或帮助内群体的行为,得以切实提升其家庭的生存前景。

在现实时期,尽管国家经济已经成为衡量社会福祉的基本单位,但是对有才华的个体的成功——也是其亲属的成功,最大的障碍却是以国家的名义,即内群体的名义强加给他的负担。至少对那些主要从事互惠性,而非胁迫性社会活动的人来说是这样——重温一下范登伯格的人类行为类别。

民族国家的逻辑表明,公民身份的终极代价是牺牲和死亡。吉恩·艾尔斯坦(JeanBethke Elshtain) 观察到,民族国家对公民的教化,更多是为了牺牲而不是侵略。

“年轻人上战场与其说是为了杀戮,不如说是为了死亡,为了大的政治实体而放弃自己的身体。”在纳税方面,做出牺牲的冲动同样强烈。纳税就像携带武器一样,是一种义务,而不是一种交换;因为在交换当中,一个人付出金钱可以得到某种价值相等或者更大的产品或服务。这一点从日常对话中就可见一斑。谈到纳税时,人们总是说“税收负担”,但是他们不会说“食品负担”,买汽车不会说“汽车负担”,出去旅游也不会说“度假负担”,就是因为商业买卖一般都是公平交易,不公平的话人们就不会进行交易。

在这方面,民族主义展示了,表观遗传能够以及如何扭转达尔文的“自然经济”理论。民族国家加强了系统性的、基于地狱的掠食。与石器时代狩猎采集者的遭遇不同,20 世纪末的个人,遭受的主要寄生虫和掠食者,不可能是“外人”,及外敌;而是想象的内群体的化身,既在地的民族国家本身。因此,信息时代出现的超越地域性的资产,它的主要优势就是,这种资产可以被安置在民族国家所动员的系统性强迫的范围之外,而那就是未来的主权个人居住的领域。

如果我们的思路是对的,借助微处理技术,个人很可能在技术上摆脱公民身份的负担。在未来的“虚拟城市”中,他们将成为自己的超国家主权者,而不是臣民;他们通过合同或私人条约来定义效忠,这种方式让人更多地联想到前现代的欧洲。在那里,商人通过商业条约和宪章保护自己的“财产不被任意扣押”,并获得对“领主法的豁免”。在网络文化中,成功人士将会被免除因出生的意外而产生的公民义务。他们不会再认为自己主要是英国人或美国人;而是将成为整个世界的超国家公民,只是碰巧住在其中一个或多个地方而已。

\section{网络经济与遗传基因}
然而,这个技术奇迹与它孕育的经济奇迹——摆脱地方专制,它所面临的主要障碍,在于个人是否愿意把自己的大部分财富和未来委托给陌生人。当然,如果按照严格的基因计算,这些陌生人,并不一定比我们大都数的“同胞”在基因上更陌生;而近几个世纪以来,我们都一直在依赖着所谓的同胞。

问题在于,当民族国家的内群体友好行为,收获到违反常理的结果,这对网络经济来说是积极的还是消极的指标。当那些“被遗弃者”(left-hebinds),面临失去从强制再分配中所获得的利益,他们会不会把民族国家的死亡看作是对亲属关系的攻击?新千年的第一个四分之一世纪将证明这一点。人类的情绪反应是很复杂的。要知道,在 20 世纪,有 1.15 亿人为民族国家献出了生命。这个数字残酷地证明了表观遗传的力量。它说明了,确实有很多人认为,国家的存在是生死攸关的事情。现在的问题是,这种态度会不会延续到一个具有不同大政治规则的新时代。

其实,受基因影响的为民族国家而牺牲的行为,与亲属选择的进化目的是相悖的;这一点也说明了,人类有足够的适应能力,可以适应很多在石器时代条件下没有被基因编程过的环境。塔吉(Tudge)在描述人类的“极端普遍性”(extremegeneralness)时说,“我们是相当于图灵机的动物,是可以转用于任何任务的通用设备。”在即将到来的转型危机中,哪种倾向会付出水面呢?可能二者皆有。

主权的商业化,本身取决于数十万主权个人和数百万的其他人,是否愿意把他们的资产存入“第一无地银行”(first bank of Nowhere),从而确保不受直接的强制征收。这种类型的信托,在原始的过去找不到明显类似的存在。石器时代的人们资产很少,确实有一些的话,也是囤积在部落的控制之下;而部落是一个“近亲繁殖的超级家庭”,对外人充满着疑虑和不信任。不过,尽管网络经济在进化上看起来很新奇,但它也给人类带来一个机会,去表现我们最新奇的基因遗传,那就是伴随我们超级容量的大脑而来的智慧。信息时代的精英肯定具有足够的智慧,当他们看到一个好东西时,就能够识别出来。

此外,创造出不受掠夺的资产,反过来也可以切实地增加主权个人的“整体适应性”。虽然网络经济的逻辑颠覆了民族国家存在的理由,但这种逻辑是令人信服的,特别是对高技能人士而言。

为了优化在不同管辖区之间选购服务的优势,个人必须愿意退出民族国家,把对自己的人身保护委托给其他管辖区的警察,或者委托给基于市场激励的安保人员,而这些人所在的地方,可能距离自己出生和长大的地方非常遥远。这意味着,会多门语言和热爱全球文化会有更大的优势,而不是极端的爱国主义。它还意味着,任何真正想要实现网络经济的潜力,为自己和家庭带来解放的人,都应该开始在他商业生涯中主要居住的地区以外,寻找几个管辖区,为自己打下坚实的基础。要了解这方面的更多细节,可以参考我们在附录中对实现财务独立的战略讨论。

\subsection{真正的亲和关系}
一种新的对于世界的超国家理解,以及一种确认自己在世界中位置的新方式,如果说这不能改变我们天生的行为倾向,至少可以改变人类文化所塑造的习惯。我们期望看到,新的身份方程式将在新的千年里生根发芽,并能比所预期的更好地适应新的世界。新的身份认同与国籍不同,它不是系统性强制的产物,这种强制使得民族国家及其体系在 20 世纪无所不在。在未来的新时代,共同体和效忠不再依附于领土。身份认同会更加精准地建立在真正的亲和关系、共同的利益或真实的亲属关系之上,而不是传统政治不厌其烦地宣传的、公民身份中虚假的亲缘关系。未来的保护服务,会以新的方式组织起来,而这些方式,都是过去划定边界的勘测工具所无法比拟的。越来越多的资产将存放在网络空间,而不是某个特定的地点;这种安排将促进新的竞争,可以减少“保护成本”或大多数领土管辖区内的税收。

\begin{tcolorbox}
所以,有雄心壮志的都明白,迁徙的生活方式是获得成功的代价。
\begin{flushright}
—— 克里斯托弗·拉什(Christopher Lasch)
\end{flushright}
\end{tcolorbox}


\section{逃离民族国家}
作为“内群体”的民族国家,尽管还牢牢地控制着现代人的想象力,但是,加入这样一个代价高昂的“想象的共同体”,到底效用几何?那些从未对此产生过怀疑的能人,很快就会发出疑问了。事实上,民族国家的拥护者已经开始抱怨,认知精英们正在与国家日益疏离。已故的克里斯托弗·拉什,在他的檄文《精英的反叛与民主的背叛》(The Revolt of the Elites and the Betrayal of Democracy)中,抨击了那些(精英),“与其说他们是靠财产所有权发财,不如说是靠对信息的操纵。”拉什对新兴信息经济的超国家性表示遗憾。他写道:“新式精英们所经营的市场,现在是全球性的了。他们的财富与跨国企业联系在一起。他们更关心的是整个系统的平稳运行,而不是其中的某一部分。他们的忠诚——如果这个词用在这里还没过时的话——是国际性的,而不是区域性的,不是全国性或地方性的。他们与布鲁塞尔或香港的同行之间的共同之处,相比于尚未接入互联网的广大美国同胞,要多得多。” 尽管拉什远非一个冷静的观察家,而且他的描绘显然是想对信息精英进行抹黑;但是,他对那些从地方专制中解放出来的人的蔑视,是建立在本书重点关注的相同的发展基础之上的。当我们在阅读拉什的批评,或米奇·考斯(Mickey Kaus)的《平等的终结》(The End of Equality),或迈克尔·瓦尔泽(Michael Walzer)的《正义的领域》(Spheres of Justice),或罗伯特·莱克(Robert Reich)的《国家的工作》(the Work of Nations)等批评时;我们看到我们的分析被部分证实了,这些作者都是极不高兴的,他们对市场深化的各种结果非常反对,更不要说主权个人的去国家化了。拉什痛斥那些怀“觊觎新的头脑贵族的成员资格”的人,怀着超国家的野心,“在快速流动的金钱、魅力、时尚和流行文化中,培养与国家市场的联系。”他继续写道:“他们是否还把自己当成美国人,这真的是一个问题。爱国主义,当然在他们的美德等级中排名不高。另一方面,‘多元文化’则非常适合他们。这个词让人联想到一种全球集市的惬意场景,在那里,异国美食,异国服饰,异国音乐,异国的部落习俗,都可以肆意品尝,没有任何问题,也不需要什么承诺。新精英们的家只存在于旅途中,在去参加高级别的会议、出席新的特许经营权开幕典礼、参加国际电影节或去往未被发现的独家胜地的途中。本质上,他们的世界观是游客式的;持有这种观念的人,你不太可能激发他们对民主进行激情的投入。” 

\subsection{经济民族主义}
这是这些“过客”构成了信息时代的虚拟社区,对他们的批评背后,其实隐藏着一个认知,那就是对许多精英阶层来说,成为过客的收益已经超过了他们付出的成本。像拉什和瓦尔泽这样的批评者并没有争论,在清醒的成本收益分析下,公民身份对高技能人士来说已经过时了。他们也没有提出,被他们鄙视的信息精英,算错了自己的最大利益之所在。他们也没有假装复利表真的表明,继续把钱注入到国家社保计划——更不用说所得税中,产生的回报比私人投资更高。恰恰相反,他们很懂算术。他们已经看到了,他们这些明显的结论加在一起意味着什么。但是,他们不承认经济理性的颠覆性逻辑,而是退避三舍,认为信息精英们超越地方专政、抛弃“未开窍者”是“背叛”。

与帕特·布坎南(Pat Buchanan)一样,社会民主党人也是经济民族主义者,他们不满于市场对政治的胜利。他们谴责“新的头脑贵族”脱离地方,不热衷他们对群众最大利益点的看法。虽然他们还没有明确认识到个人的去国家化,但他们抨击其早期的苗头和表现,即瓦尔泽所说的“市场帝国主义”,或金钱“跨界渗透”的去世,以及购买那些如拉什所谓的“不应该被出售的东西”,如兵役豁免权。需要注意的是,反动派把民族国家的军事要求,看作是货币和市场不可侵犯的神圣领地。

这些对信息精英的批评,预示了在下一个千年中,人们对主权个人崛起的反应及说辞。随着新型的、更加市场化的保护方式的出现,大量有能力的人,会越来越清楚地看到,大部分所谓的国籍利益都是虚构的。这不仅使人们有机会,更好地去核算公民身份的机会成本,还将创造出新的方式,去构建所谓的“政治”甚至“经济”问题。正如弗雷德里克·莱恩对一个古老的两难问题的表述,有史以来第一次,“一个为自己、并自主行事的企业家个人”,可以通过在不同管辖区之间的移动,来调整自己的保护成本,而无需再等待“群体决策和群体行动”去达成它们。

当为保护付出的代价,开始受制于“替代原则”(即面临竞争,译注),强制行为背后的算术就会赤裸裸地暴露出来,而这将激化信息时代新的全球主义精英,与“信息贫民”之间的冲突,后者基本上只会一种语言,不擅于解决问题,也不具备什么可在全球市场销售的技能。这些“失败者”或“被遗弃者”,正如托马斯·弗里德曼所说,无疑将继续把他们的福祉,与现有民族国家的政治生活联系在一起。

\section{大部分政治议程都将是反动的}
随着 21 世纪的到来,大多数热衷于搞政治的人,无论是民族主义者、环保主义者、还是社会主义者,都会团结起来,维护摇摇欲坠的民族国家。渐渐地,一个事实将变得越来越明显,即民族国家和民族主义情感的生存,是维护政治强制力的前提条件。正如比利格所指出的,“民族主义是常见(政治)战略的运作条件,不管在什么政治环境中。”因此,可以想见,在未来的岁月里,所有政治议题中的民族主义含量,都会像贪吃蛇的肚子一样迅猛膨胀。例如,环保主义将不太注重保护“地球母亲”,而是更关心“祖国”。比起他们现在的理解,他们可能会更加同意克里斯托弗·拉什的观点,他跟随汉娜·阿伦特宣传:“是公民身份创造了平等,而不是平等创造了公民权利。”主权私有化将切断财富创造者与国家和地方之间的联系,从而降低工业时代那种对平等的重视程度。公民身份不再是这样一种机制,即在限定的领土内,根据选票平等的原则强制进行收入再分配。这种变革会对进步史观造成又一次的挫伤。

因为,与 20 世纪开始时,那些被认为富有远见的人士所预期的相反,经过几十年的时间考验,自由市场并没有被摧毁,而是胜利地离开了。马克思主义者预测,资本主义将会消亡,并导致民族国家的超越,工人之间会出现普遍的阶级意识,而这从未发生。事实上,国家将会消亡,但方式截然不同。即将发生的事情,与马克思主义者的期望几乎正好相反。资本主义的胜利,会会资本家中间出现一种全新的、全球性的,或者说是超国家的意识,很多人将成为主权个人。最有能力、最富有的人,根本不像马克思主义者所想象的那样,依靠国家来约束工人,他们其实是民族国家操作下的净输家。显然,随着市场战胜强制,超越民族主义,对他们来说才是最有利可图的。

很快,也许不是马上,但肯定在一代人的时间内,几乎每一位信息精英,都会选择把他的营收活动安排在低税或无税的管辖区内。随着信息时代改变全世界,它会给人们上一堂明白无误的复利实战课程。在几年内,更不用说几十年内,人们将普遍认识到,任何有才能的人,都可以通过放弃高税率的民族国家,积累更高的净资产,享受更好的生活。我们在前面已经提示过,先进的民族国家所施加的惊人成本,但由于这是一个鲜为人知的问题的核心,值得再次强调一下国籍的机会成本。

\subsection{机会成本}
信息精英们退出民族国家,非但不会因为,目前由高税收资助的政府服务的失去或削减而受到影响;反而,他们会以前所未有的方式蓬勃发展。只是摆脱现有的超额税负这一项,就会让他们获得巨大的利润空间,可以改善家庭的物质福利。

如前所述,如果你每年能从投资中赚取 10\%的收益,那么缴纳 5000 美元的税款,就会让你一生的净资产减少 240 万美元。而如果你能赚到 20\%的收益,每年多缴纳 5000 美元的税,会让你在四十年里少赚 4400 万美元。累计而言,每年缴纳5000 美元会使你每年损失超过 100 万美元。按照这个比例计算,每年 25 万美元的税,很快就相当于每年 5000 多万的损失,算下来一生中损失 22 亿美元。当然,中间可能有零星的更高的收益,哪怕只有几年,特别是在人生的早期阶段,就以为着掠夺性税收导致的财富损失更加惊人。

高于 20\%的收益是可能的,你们的作者自己已经实现了。在我们撰写本书的几年里,我们在百慕大的莱恩斯(Lines)海外资产管理公司的同事,实现了三位数的回报,平均每年为 226\%。他们的经验突出了复利表所说明的,即对于高收入者和资本所有者来说,掠夺性税收所导致的终身成本,相当于一笔巨额财富。

一个高收入者按照香港税率纳税,另一个具有同等收入的人按照北美或欧洲的税率纳税,前者最终拥有的财富,可能是后者的一千多倍。如果你的资本不断受到高税收管辖区的侵犯,那就像在赛跑比赛中,你每跑一步都有人朝你开枪。如果能在有适当保护的情况下参加同样的比赛,并且不受伤害地奔跑,显然你会跑得更远、更快。

未来的主权个人,会利用克里斯托弗·拉什及其他信息精英批评者所不满的“过客”偏好,他们会选择最有利的管辖区居住。虽然这与民族主义的逻辑相悖,但却符合令人信服的经济逻辑。10\%,更不要说 10 倍的底线收益差距,就会促使追求利润最大化的人们,改变他们的生活方式和生产技术,以及他们的居住地。

西方文明的历史,原本就是一系列不安分的变动记录,在曲折迂回的大政治条件刺激下,人口与繁荣总是一再迁徙到新的机会之地。底线收益的千倍之差,匹配最有力的刺激措施,将让理性的人们行动起来。或者换句话说,大多数人,即使是托马斯·弗里德曼所说的“失败者和被遗弃者”,如果有机会,也会很高兴为了 5000 万美元离开任何一个民族国家;更不用说那前 1\%的纳税人,民族国家仍然在从他们身上榨取更大的成本。因此,选购管辖区的主权个人之崛起,是关于信息时代的最可靠的预测之一。

\section{主权的商业化}
从成本收益的角度看,在 20 是行将结束时,公民身份已经是一个赔钱的买卖。

澳大利亚议会研究处的伊恩·艾尔兰,在 1995 年 8 月编写的,题为“女王是澳大利亚公民吗?”的研究报告,就强调了这一点。艾尔兰对 1948 年的《澳大利亚公民法》进行了梳理,并回顾了人们获得澳大利亚公民身份的四种途径,与其他主要民族国家的获得方式差不多,即:出生,收养,血统,授予。

这些方式原本都没什么,但它使人们注意到,主权与公民权之间的区别。正如艾尔兰所说,“根据传统的法律和政治概念,君主是主权者,人民是他或她的臣民。

臣民通过效忠和服从而受制于君主。”显而易见,女王伊丽莎白二世是主权者;看到这一点,艾尔兰得出结论:“有一种说法认为,女王不是澳大利亚公民。”的确,她不是。女王,愿她长命百岁;幸运的是,她已经不在乎自己是否是个公民了。她是主权者,是她臣民们的君主。和世界上其他少数几个君主一样,女王是天生的主权者,她的地位是通过现代以前的习俗继承下来的。君主制的观念很古老,可以追溯到人类生活的最早历史记录。那些保留了君主制的国家,将其形成归因于古老的历史,但它仍然有助于决定其社会的形态,即使不是政治权力,也是阶级威望。后现代的个人,没有女王的先发优势,将不得不发明新的法律依据,作为信息技术赋予他们事实上的主权地位的基础。

主权个人还将不得不应对嫉妒的腐蚀性后果,这个难题有时候让君主们都感到束手无策;但对于那些非传统的尊贵人士,而是自我创造主权的人,它的难度无疑更大。正如赫尔穆特·舍克在他的综合调查报告《嫉妒》中所写的:“在只有一个国王、一个美国总统的情况下,或者说,在特定地位上只有一个成员,那么他可以过着相对免责的生活;而他的那种生活方式,如果被该社会中更大的职业或社会团体的成功人士所采用,即使是在一个很小的圈子里,也会引起极大的愤慨。”所以,君主作为国家的化身,享有一定的嫉妒豁免权,而这种豁免不会延续到主权个人身上。

信息社会中的“失败者和被遗弃者”,肯定会羡慕和怨恨成功者;特别是随着市场的深化,未来将越来越是一个“赢家通吃”的世界;现在的奖励已经更多地基于相对表现,而不是工业生产中的绝对表现。打工者的工资要么是根据考勤表记录的时间,要么是根据某些产出标准,如制造的件数、组装的单位数或其他类似的衡量标准。因为使用同样的工具,每个人的产出都差不多,所以才可能实行标准化的薪酬。但是,创意财富的创造,就像艺术表现一样,使用同样工具的人,表现有云泥之别。从这个角度来看,未来的整体经济会越来越像歌剧行业,最高的报酬给那些嗓音最好的人,而那些唱歌走调的人,无论多么认真,也不可能获得很高的收入。随着各种领域的开放,迎接真正的全球化竞争,普通表演者的回报率必然会下降。中等人才将大量涌现,其中一些人还能以先进工业国家主流价格的一部分,出租自己的时间。失败者将是那些拿着“滑球速度棒”(slider speedbats)的小联盟外野手,他们的反应速度距离打出大联盟的快球差上半秒钟;没有全垒打,没有年薪百万,也没有明星代言的补充收入,他们一年只挣 25000美元。剩余其他人将完全被三振出局。

\begin{tcolorbox}
一旦一个国家向全球市场开放,那些有能力利用它的公民就会成为赢家,而没有能力的则成为失败者或被遗弃者。通常会有一方……宣称自己能够对抗或减轻全球化的痛苦。那就是美国的帕特·布坎南,俄罗斯的共产党,以及现在土耳其的伊斯兰福利党。……因此,土耳其发生的事情比原教旨主义的接管要复杂得多。当全球化的扩大使越来越多的失败者被淘汰,当民主化的扩大使他们都有了投票权,而宗教党派却有效地利用这种巧合来夺取政权时,就会发生这样的现象。
\begin{flushright}
—— 托马斯·弗里德曼(Thomas Friedman) 
\end{flushright}
\end{tcolorbox}

谁将是信息时代的输家?一般而言,税收消费者会是输家。他们通常没能力转移到另一个管辖区来增加自己的财富。他们的大部分收入,是存放在一个国家的政治规则之中,而不是通过市场估值来传递的。因此,取消或大幅减少税收,会对他们的净资产产生负的复利影响,他们的生活不会变得更好,因为降低收税的代价是转移支付的削弱。他们可能会失去收入,因为他们无法再依靠政治强制力,从比自己更具生产力者的口袋里掏钱。那些没有储蓄而依靠政府支付退休福利和医疗费用的人,生活水平有很大可能会下降。这种收入损失会转化为金融作家斯科特·伯恩斯(Scott Burns)所说的“超验的”或政治的资本大贬值。这种“超验的”或想象中的资本,并不是基于经济上的财产所有权,而是对一种收入流的事实性(de facto)拥有,这种收入流是由政治规则和条例所建立的。例如,政府转移计划的预期收入,可以转化为按现行利率计算的债券资本。这种由想象中的共同体资助的想象中的债券,就是“超验”资本。它将因“大转型”而突然贬值,这种转型必将削弱政府对现金流的控制,使之无法兑现承诺。

\begin{tcolorbox}
在边境和公海上,没有人能够长久地垄断暴力,商人们不会支付高额的苛捐杂税,因为通过其他方式可以更便宜地获得保护。
\begin{flushright}
—— 弗里德里克·莱恩
\end{flushright}
\end{tcolorbox}

显而易见,未来的信息精英会利用网络新经济的机会,追求自我解放与个人主权。同样可以预料,随着信息技术在新千年的影响越来越大,那些“被遗弃者”会越来越愤怒和具有攻击性。场面会在什么时候变得丑陋,很难给出精确的时间点。

但我们估计,当西方国家像前苏联那样,解体的前景一目了然时,攻击将会加剧。

但同时,每当一个民族国家崩溃,都会促进权力进一步下方,激励主权个人的自治。我们预计,主权实体的数量将大幅增加,因为从国家的废墟中,会出现许多类似于城邦的飞地和管辖区。不乏有新的实体,将提供高度竞争性的保护服务定价,对收入和资本征收极低的税收,甚至根本不征税。在保护服务的定价上,新世纪必然会比经合组织的主要民族国家更具吸引力。可以把这看成一个简单的市场细分问题;市场上服务最差的领域,一般都是在高效率、低成本的那一端。谁要是想缴纳高额的税收,去交换一系列复杂的政府安排,那他有充分的机会可以做到。因此,对于一个新型的迷你主权来说,最有竞争优势也最有利可图的战略,应该是提供高效率、低成本的替代方案。毕竟,微型主权要想提供比现有民族国家更全面的服务,是不太现实的。考虑到所有的民族国家肯定不会马上就崩溃,所以,替代国家主义的方案会有充足的供应,这是很可能的,特别是在早期过渡阶段。另外来说,一个拥有过得去的法律和秩序、能满足基本需求的政权,也可以相对便宜地先推出来。如果社会动荡和犯罪活动,在旧的核心工业国加剧到我们预期的程度,那么,一个有基本法律和秩序的管辖区,比国家太空计划、国家赞助的妇女博物馆,或为失业管理人员提供补贴的再培训计划,都要有吸引力得多。

\section{个人的去国家化}
随着新实体的出现,公民身份的魅力将日益消退,其逻辑也不再令人信服;因为新实体允许公民对政府现在提供的服务进行选择,首先就是保护服务。这样一来,个人不再以国家名义来确定自己的身份,将成为现实。不过,公民身份的祛魅(去神秘化)是一个漫长的过程。在日常生活中,你会不断接触到各种无聊的信息,这些信息都旨在加强你对民族国家的认同。在这些信息的熏染下,你不可能忘记“你的国籍”。对很多人而言,国籍是重要的身份徽章。“我们”被教导以国籍来看待世界。这是我们的国家,我们的运动员在奥运会上竞争;当他们获胜时,颁奖仪式上飘扬的是“我们”的国旗,我们的国歌让裁判和其他国家的选手肃然起敬。“我们”被引导,相信这是“我们”的胜利;尽管我们从来也没搞清楚,“我们”是如何参与其中的,除了作为同一领土内的一个公民。

\subsection{从第一人称复数到单数}

随着信息技术走上前台,它将促进全球视野的形成,这种趋势在 MTV 观众的态度中已经明显流露出来;它还将创造各种条件和潜在的可能,使主权个人可以借之摆脱民族主义的沉重负担。举例来说,在今后的几十年内,窄播将取代广播,成为个人获取新闻信息的方式。这一点意义重大。它相当于把千百万人的想象力,从第一人称的复数调频为单数。当个人开始担当自己的新闻编辑,选择自己感兴趣的话题和新闻报道时,他就不太可能再去选择,那些灌输为民族国家牺牲的必要性的内容。事实上,相比他们订阅的高度个人化的“窄播”新闻,他们的态度更可能受到自己消费的全球文化的影响。同样,在技术的推动下,教育的私有化也会产生类似的效果。在中世纪时期,教育牢牢地控制在教会的手里。在现代时期,教育一直在国家的控制之下。用埃里克·霍布斯鲍姆(Eric Hobsbawm)的话说,“国家教育把人变成了特定国家的公民,‘把农民变成了法国人’。”在信息时代,教育将被私有化与个性化。它不会在像工业时代那样,背负着沉重的政治抱负。民族主义被灌输到灵魂生活每个角落的日子,将成为过去。

向互联网的转移,还将降低地点在商业种的重要性。不受地域限制的个人地址将会出现。基于卫星的数字电话服务,将超越基于地理位置的固定线路系统,后者只能共享一个国际拨号代码。在未来,个人将拥有自己独一无二的全球电话地址,就像互联网网址一样;无论他身处何地,你都联系得到。到了一定的时候,国家邮政系统的垄断地位将崩溃;与现有民族国家毫无联系的私人邮递系统,将在全球内提供服务。

这里一步,那里一步,虽然是一些明显的小步骤,但有助于使普通消费者以及认知精英,摆脱对民族国家的刻板认同。当国家垄断的领土上,出现切实可行的主权替代方案,公民身份的祛魅化将大大加速。网络经济的基石——网络货币、网络银行及不受监管的全球证券网络市场,必将大规模涌现。它们的出现,将使贪婪的政府没收“公民”财富的能力,日趋萎缩。

当然,先进国家肯定会尝试合作,建立垄断联盟,限制加密技术,并防止公民逃离自己的领地,以维护它们的高税收和通胀的法币,但它们最终会一败涂地。地球上最有生产力的人,会找到通往经济自由的道路。国家甚至没办法有效地在物理上把人关起来,用以敲诈和收税。禁止非法移民的措施往往都会失效,这已经令人信服地说明了,民族国家想封锁边界以防成功人士逃跑,是不可能的。富人逃跑时,就像未来的出租车司机和服务员想逃进来一样,充满魄力和开拓精神。

自从主权分散的中世纪以来,又一次,边界将无法被明确地划分。正如我们前面所探讨的,未来大量的金融交易,都不会发生的确定的领土内。越来越多的主权个人,不会因为出生的巧合而继承国家的债务,他们会利用这种模糊性,摒弃其纳税义务,超越公民身份,成为政府的客户。就像我们在第 8 章所分析的,他们将作为客户与瑞士谈判私人税收条款。一份典型的与瑞士法语区各州达成的私人税收协定,允许个人或家庭每年固定缴纳 5 万瑞士法郎(目前约合 4.5 万美元)的税款为交换,在瑞士境内居住。请注意,这不是统一的税率,而是不考虑收入的固定税额。如果你的年收入是 5 万瑞士法郎,你不就应该签这样的私人税收条约,因为你的税率是 100\%。但如果你的收入是 50 万瑞士法郎,你的税率就是10\%。如果是 500 万,税率就只有 1\%。如果是 5000 万,那就是 0.1\%。与纽约市 58\%的边际税率相比,如果你觉得这个条件好得难以置信,那只说明了,工业时期的政府服务定价,其掠夺性和垄断性是多么地凶残。

事实上,对于政府提供的必要且有用的服务来说,5 万瑞士法郎已经是一笔充裕的年费。每一位移居瑞士的百万富翁,每年都为他们获得的特权缴纳 5 万瑞士法郎,瑞士人从中肯定没少赚。而且在很多时候,多接收一个百万富翁,其边际成本基本为零;因此,相当于瑞士政府在一笔交易中的年净利润接近 5 万瑞士法郎。

任何服务,如果可以被压低价格,还能让低成本的供应商获得约 100\%的利润,那就是垄断性的,其定价过高超越了极点。让人吃惊的,并不是像瑞士这种情况,征收的税率会随着收入的增长而降低;而是在 20 世纪,不同的人为政府支付天差地别的税额,看上去居然是“公平”的。这真的是奇怪得很,因为使用政府服务最多的人缴的费用最少,而使用最少的人却缴得最多。所有这些都证明了,对于一个高收入的美国人来说,在瑞士居住相比美国国籍的优势,相当于一生数千万美元的收入差别。所以,除非美国进行税收改革,相比其他国家更具竞争力,并且不再以国籍为基础进行征收,否则,即使克林顿设置了离境税的障碍,有头脑的人还是会放弃美国身份,去获得税负较轻管辖区的护照。

工业时代的政府是根据纳税人的成功来为其服务定价,而不是根据其服务的成本或价值。当政府的服务被转向商业定价,人们就能以远低于传统民族国家的价格,获得更加满意的保护服务。

\subsection{公民身份走上骑士身份的老路}
简而言之,公民身份注定要走上骑士身份的老路。随着提供保护的制度基础再次被重组,与该制度相配套的合理化叙事以及激励的意识形态,也将不可避免地发生改变。500 年前,在中世纪结束时,当以服务去换取保护不再是一个划算的买卖时,人们的反应是可以预见的。他们摒弃了骑士精神,宣誓和忠诚不再像过去五个世纪那么重要。今天,信息技术有望同样颠覆公民身份。民族国家和民族主义的理念将被祛魅,就像 500 年前处于垄断地位的教会一样。

虽然反对派会通过诋毁创新者,以及重振民族主义情绪来负隅顽抗,但我们怀疑,在大政治层面已经死亡的民族国家,还能够召唤起足够强大的忠诚度,去抵御信息技术释放出来的竞争压力。在一个政府纷纷破产的世界里,大多数有头脑的个人,都宁愿作为客户享受良好的保护服务,而不是作为民族国家的公民被掠夺。

富裕的经合组织国家,对其境内的商人施加了沉重的税收和监管负担。在以前,当经合组织国家是仅有的、可以相对舒适地经商和居住的管辖区时,这些费用还可以忍受。但那一天已经过去了。最富有国家的居民为税收和监管支付的溢价,永远也不可能得偿其成本。随着各管辖区之间的竞争加剧,这种情况越来越无可忍受。有资本、有能力应对信息时代挑战的人,会选择到任何地方居住和经营。

在有选择的情况下,只有最爱国或最愚蠢的人,才会继续住在高税率的国家。

出于以上原因,可以预见,一个或多个民族国家会秘密行动,破坏过客生活方式的吸引力。生物战可以有效地阻断旅行,例如引爆致命的流行病。这不仅会打击人们对旅行欲望,还会给世界各国提供封锁边界、限制移民的借口。

\subsection{国籍税的弊端}
除非政策发生惊人的、近乎奇迹般的变化,否则,信息时代成功的投资者或企业家,将因为居住在某些国家,而在一生中支付数千万、数亿、甚至数十亿美元的罚款;因为这些国家的财政政策,类似于 20 世纪生活水平最高的那些国家。

如果不发生彻底的改变,美国人受到的惩罚将是最严重的。美国是世界上仅有的三个根据国际而非居住地征税的国家之一。另外两个,一是菲律宾,前美国殖民地;二是厄立特里亚,它的一位流亡的领导人,在长期反抗埃塞俄比亚统治的过程中,受到了美国国税局的蛊惑。厄立特里亚现在征收 3\%的国籍税。虽然这只是对美国税率的苍白模仿,但这样的税负,也使厄立特里亚的公民身份在信息时代成为累赘。而美国的现行法律,使其公民成为更大的麻烦。国税局已经成为美国主要的出口产品之一。因为相比其他国家,美国更有能力把手伸到地球上的任何角落,从它的国民身上榨取财富。

如果一家波音 747 喷气式客机,满载着来自世界每个国家的一名投资者,在一个刚刚独立的国家降落,每个投资者在新的创业冒险中投入 1000 美元,那么,美国人将面临比其他任何人都高得多的收益税。包括针对在外国投资的特殊惩罚性税收,如所谓的 FPIC 税,再加上美国国籍税,可能导致在美国境外持有长期资产,将遭到 200\%甚至更高的税负。一个成功的美国人,如果成为全球其余 280多个国家任何一国的公民,都可以大大减少他一生中的总税收负担。

美国拥有全球最具掠夺性的、吃大户(soak-the-rich)的税收制度。与其他所有国家的公民相比,生活在国内或国外的美国人,更多地被视为国家的财产,而不是客户。因此,美国的税制,甚至比臭名昭著的斯堪的纳维亚福利国家的税制,更加不合时宜,更不符合信息时代的成功要求。丹麦或瑞典公民,如果要实现他们日益提升的技术自治权,几乎没有什么法律障碍。如果他们想要谈判个人税率,他们可以自由地通过私人条约在瑞士交税;或者搬到百慕大,完全不用缴纳所得税。一个瑞典人或者丹麦人,如果相信斯堪的纳维亚国家的福利制度,值得并且愿意支付高额税款,那他其实是做出了选择。他有能力选择文明或不文明世界里任何国家的任何税率。要改变税率,他只需要搬家即可。科技让这种选择变得越来越容易。然而,美国人却被剥夺了这个选项。

要实现信息革命带来的个人自治,持有美国护照注定将成为一大绊脚石。在工业时代生而为美国人,是一个幸运的意外;而在信息时代,即使是在早期阶段,它也已经成为了一个数百万美元的包袱。

要想搞清楚这个包袱有多大,可以做一个比较。在合理的假设下,一个新西兰人,如果他的税前收入与美国前 1\%纳税人的平均收入相同,那么他缴的税要少得多,但是节税产生的复利,就会使他比美国人更富有。在一生结束时,这位新西兰人会比美国人,多留给子孙 7300 万美元。然而,新西兰还不是一个公认的避税天堂。还有其他 40 多个管辖区,其收入税和资本税比新西兰还要低。在我们看来,低税率管辖区的数量应该会增加,而不是减少。所有这些地区都将提供比美国更多的居住地优势,其价值即使达不到数亿美元,终生算下来也有几千万。除非美国进行税制改革,相比其他管辖区更具优势,并且不再根据国籍征税,否则,尽管克林顿设置了离境税障碍,有头脑的人还是会放弃美国身份。

信息时代的竞争条件,几乎可以使人们在任何地方获得高收入。民族国家赖以征收超额税负的地域垄断将被打破,实际上,它们正在被打破。随着垄断被进一步削弱,竞争的压力必将驱使最具进取心、最有能力的人,逃离高税率的国家。而这些国家,就像《经济学人》的前编辑诺曼·麦克雷(Norman Macrae)所说,“将主要由傻瓜居住”。

\begin{tcolorbox}
到 2012 年,预计用于福利和国债利息的支出,将耗尽联邦政府的所有税收收入。……将不会有一分钱留给教育、儿童项目、高速公路、国防或其他任何可自由支配的项目。
\begin{flushright}
—— 美国两党权利与税收改革委员会
\end{flushright}
\end{tcolorbox}


富人从先进福利国家逃离的时间,恰好是人口统计学上的不利阶段。到 21 世纪初,欧洲和北美大量的老龄人口将会发现,自己没有足够的储蓄来支付医疗费用及退休后的生活方式。例如,65\%的美国人完全没有退休储蓄金,一点也没有。

而那些有储蓄的,则存得太少。普通的美国人,到了 65 岁,在去世之前预计还将面临超过 20 万美元的医疗费用;而他们的净资产不足 7.5 万美元。即使拥有私人养老金的少数人,也不可能过上舒适的生活;他们的养老金,平均只能替代退休前收入的 20\%。典型的退休人员,他们的大部分资产并不是真正的财富,而是“超验资本”,即转移支付的预期价值。大多数人已经习惯于,依赖这些转移支付来弥补个人资源的缺口。问题是,这些钱很可能会收不到。现收现付制度将缺乏现金流或资源去兑现它们。尼尔·豪(Neil Howe)进行的一项研究表明,即使美国的税前收入比过去 20 年增长得更快,但到 2040 年,美国的平均税后收入将不得不下降 59\%,以资助当前水平的社会保障和政府医疗项目。

这个问题没有回旋的余地。福利国家正面临破产。欧洲的财政困境比北美更严重。

意大利可能是最糟糕的,紧随其后的是瑞典和其他北欧福利国家,它们为收入支持计划设定了慷慨的标准。《金融时报》估计,如果“将意大利国家养老金的现值计算在内,该国公共部门的债务将超过 GDP 的 200\%。”这种程度的负债,从数学上讲是没有希望了。几年前,针对在多伦多政权交易所上市的公司,就其商业负债状况,进行过一次全面的调查,研究表明,面临当今主要福利国家的那种极端负债率,只有四分之一的企业活了下来。简单地说,这些国家已经破产了,数万亿没有资金支持的福利承诺将被勾销。面对这个现实吧,即使不情愿但也无可避免。

这就是网络经济的逻辑。它面临的一个障碍,可能只是简单的惯性,即人类不愿意承担风险去移动的筑巢本能。如果说还有其他的障碍,那可能是人类本性中的硬伤。在网络空间部署资产的经济逻辑,与根深蒂固的怀疑外人的生物逻辑相违背。在每一种文化中,儿童都会表现出对陌生人的厌恶。主权商业化的反对者,会竭尽全力地煽动人们的怀疑情绪,反对信息时代带来的新的全球文化,并拒不接受它将导致的民族国家的消亡。另外一个障碍可能来自表观遗传,或受基因影响的动机因素;受此影响,“失败者或被遗弃者”对于破坏民族国家的发展,可能会做出像狩猎采集者保护家庭式的愤怒反应。当迷失方向和被疏离的个体,拥有越来越强大的破坏力,在这样的环境中,对信息技术的反击,可能将充满暴力与痛苦。

\begin{tcolorbox}
从历史上看,集体暴力经常从西方国家的中央政治进程中流泻出来。想要夺取、掌控或重整权力杠杆的人们,在其斗争过程中总是会诉诸集体暴力。被压迫者以正义的名义发起攻击,特权阶级则以秩序的名义进行镇压,而中间阶层则因为恐惧而陷入骚乱。权力局势的巨大转变通常就发生在——而且往往依赖于——集体暴力的特殊时刻。
\begin{flushright}
查尔斯·蒂利(Charles Tilly)
\end{flushright}
\end{tcolorbox}


\section{暴力行为透视}

在变革的环境下,是什么引发了暴力?关于这个问题,至少有两种不同的理论。

历史学家查尔斯·蒂利总结了其中一种。他认为,“激发集体暴力的源头主要在于人们对既定体制崩溃的焦虑感。如果又叠加了痛苦与危险,那么理论认为,人们的反应会更加残暴。”不过,在蒂利看来,与其说暴力是“焦虑”的产物,不如说它是一种更为理性的企图,即出于“正义感被剥夺”的动机,压迫当局“履行其责任”。根据蒂利的解释,“重大的结构性变化”往往会刺激“政治”性质的集体暴力。“此外,暴力斗争非但不会形成与‘正常’政治生活的急剧决裂;它反而是伴随、补充和扩展了同一批人,为实现目标而进行的有组织的、和平的抗争。暴力与非暴力的斗争属于同一个世界。”无论哪种暴力理论更正确,大变革期间的社会,其和平前景都是很有限的。显然,民族国家的崩溃,肯定算得上是“既定体制崩溃”的典型例子。因此,人们的焦虑情绪很可能全面爆发,政治上的暴力刺激也会随之而来。先进的福利国家可能会尤其严重,因为那里的人已经习惯了相对的收入平等。生活在信息经济早期的民众,都是在工业时代长大的,而当时的政府确实有能力,用物质利益堵住人们的不满;考虑到这一点,可以预期“被遗弃者”会继续要求物质利益。经合组织国家的人民,要在网络经济的现实中,经过漫长而痛苦的教育,才能摆脱对大规模强制性的收入再分配的期待。总之,无论由于什么原因,是出于“焦虑”也好;还是为了利用系统性强制的好处,而进行的更有心机的争抢也好,在大变革的环境条件下,暴力基本是无可避免的。

\subsection{失败者的选区}

强制性收入再分配系统的崩溃,必然使那些寄希望于数万亿转移支付的陷入不安。这些人基本上都是“失败者或被遗弃者”,他们缺乏在全球市场上竞争的技能。前苏联的养老金领取者,构成了祖加诺夫(Zuganov)的共产主义核心支持群体;同样,垂死福利国家失望的养老金领取者,将形成一个反动的选区,他们将致力于组织民族国家的主权被私有化;而正是这种私有化,剥夺了国家盗窃的资格。当他们意识到,他们以前控制的政府,正在失去对资源享有的主权,以及强制进行大规模收入转移的能力时,他们会变得像法国公务员一样,坚定地与算法作战。

你可能还记得,法国总理阿兰·朱佩曾经提出,要削减国家工作人员“在人口统计学意义上不可持续的”退休福利,以及节约国有化铁路系统的运营成本;尽管他的提议相当温和,但还是激起了强烈的反对。法国人称他们的制度为“福利国家”(État-Providence),其荒谬程度的一个象征是,法国规定,允许“在计算机化的高速 TGV 列车上工作的工程师,可以在 50 岁时退休,就像在燃煤机车上辛苦工作的前任一样。”显而易见的是,在任何一个经合组织国家,要减少不可持续的福利,都很可能遭到暴力的回应。即使在有些国家,民众的反应没那么强烈,你也可以预料,未来的失败者会竭尽所能地阻止,对国家强制力的瓦解。

整个过程可能会出人意料地曲折。例如,在美国历史上,本土主义情绪就一直带着浓厚的种族主义色彩。这个传统始于 19 世纪的“白帽运动”(White Caps)和三 K 党。然而,作为一个群体而言,黑人其实是收入转移、平权运动和其他政治强制成果的主要受益者。他们在美国军队中的比例也非常高。因此,黑人很可能与蓝领白人一道,成为美国民族主义最狂热的支持者。

那些处于阿蒙的萝卜模型底端的人,将充满不安全感;而刻意迎合这些人的政客,不管在哪个国家,都会在一篇聒噪中走上政治的前台。送塞尔维亚的斯洛博丹·米洛舍维奇,到美国的帕特·布坎南,到新西兰的温斯顿·彼得斯,再到土耳其原教旨主义的伊斯兰福利党的领袖内吉梅丁·埃尔巴坎,这些煽动家都在大力抨击市场全球化、移民和投资自由。

那些认为自己是“经济全球化牺牲品”的人,将对富人和移民产生特别的敌意。

用安德鲁·希尔的话说,他们会“仇视移民的加入,因为移民的准入标准好像就是财富,或者说是缺乏财富(即难民,译注),而按照他们似是而非的逻辑,后者会成为他们的福利负担。”

\subsection{对自由的恐惧}
民族国家可能会在新千年的初期退出历史舞台,那些容易受此影响的人,他们的生活将天翻地覆。社会将因此而充满戾气。已经有不少的观察家注意到了一种反应模式,这种模式普遍存在于,那些感觉被无边界的未来所抛弃的人中间。当更广大、更具包容性的国家群体开始瓦解,同时伴随着更具流动性的“信息精英”进行业务全球化,“失败者和被遗弃者”,将回归到某个种族亚群、部落、帮派、宗教或语言上的少数群体中。在某种程度上,这也是在过去由国家提供的服务(包括法律与秩序)崩溃之后,而产生的一种实际又实用的反应。对于那些没有多少市场资源的人来说,往往很难买到什么机会和渠道,去替代已经崩溃的公共服务。

而对于那些有足够资源购买高质量私人替代品的人而言,把以前被视为公共产品的东西,如教育、清洁的水源或社区治安,转换成私人产品显然更容易管理。不过,需要现金的话,最实际的替代方案还是依靠亲属,或者加入一个按种族身份组织起来的互助团体,比如东南亚古老的华裔“福建人”,或者是通过宗教组织。

世界上那些宗教有力、传教活跃的地区,宗教活动之所以受欢迎,部分原因在于,它们往往有能力回归到提供社会福利和公共服务的前现代机制。例如,在南非开普敦,由穆斯林领导的民团,就在打击暴力帮派方面发挥了主导作用。然而,尽管这些种族和宗教团体的互助切实有用,但未来的反动行为更多是由于国家的萎缩。另外,在对全球化的反对中,好像也存在某种强烈的心理因素。

这种论点,与埃里希·弗洛姆(Erich Fromm)在 1942 年首次出版的著名作品《对自由的恐惧》(Escape from freedom)中,对法西斯主义的吸引力所做的心理学解释不谋而合。弗洛姆认为,资本主义带来的社会流动性,破坏了传统乡村生活的固定身份。农民的儿子不确定他必将成为一个农民,他也不必在父亲耕种的贫瘠土地上面朝黄土背朝天。他现在有了广泛的职业选择。他可以成为教师、商人、军人;也可以做医生,当海员。即使身为农民,他也能移民到美国、加拿大或阿根廷,在远离故乡的地方生活。然而事实证明,资本主义带给人们的这种“创造自我身份”的自由,对于那些不准备创造性利用它的人来说,是相当可怕的。正如比利格所言,很多人渴望“固定身份的安全感”,并且“被民族主义和法西斯主义宣传的简单性所吸引。”同样,比利格在谈到工业时代的黄昏时写道:“有一种全球性的心理因素,它在从上面打击国家,用身份的自由消灭传统的忠诚。

另外,还有一种源于种姓的或部落的炙热心理,它以一种强大到偏执的承诺和激烈的情感,从下面攻击国家的软肋。”安德鲁·海尔(Andrew Heal)从另一个角度看待这个现象。他认为有两个重大的“全球性政治和经济趋势……趋势一是全球经济的增长。……趋势二是民族主义、种族主义和地区主义情绪的上升。无论是毛利人、苏格兰人、威尔士人还是反移民的团体,即使政府在把他们推向无边界的新未来,他们自己却一直在奋力地往相反的方向回撤。”总之,无论你怎么看待这种现象,是一种主要的“趋势”,还是一个“心理课题”;显而易见的是,一股强烈的支持民族主义、反对边界消失和市场深化的反动情绪,正在世界范围内集结声势。

\section{文化多元主义与被害者}
在落幕时分,福利国家要从空空如也的口袋中,兑现不劳而获的承诺,已经力有不逮;这时它会发现,继续编造、发动其新的歧视策略,只能是一个权宜之计。

官方会指定出各种类型的“被压迫者”,特别是在北美地区。还有被指定为“受害者”的群体,官方会告诉他们,生活中的不如意不是他们自己的错。相反,这一切都是因为欧洲后裔的“死去的白人男性”,以及操纵剥削被排斥群体的压迫性权力结构。黑人、女性、同性恋者、拉丁裔、法语裔、残疾人等等,都有权为过去受到的压迫和歧视而获得补偿。

按照拉什的观点,强化受害感的目的是要破坏国家,这会让新兴的、可自由流动的信息精英,更容易逃避公民的责任与义务。我们并不确信,新精英们,尤其是来自大众媒体的大多数人,有足够的精明,可以理智地采取这种行动。要是他们有这样的本事,那就令人放心了。在我们看来,受害学的兴盛,主要是为了买一个社会和平,手段不仅有拉什所说的,扩大英才制度的成员,还包括重新构建收入再分配的合理性。新一轮的受害学运动,会在北美表现得最为夸张,因为信息技术在那里渗透得最深。但是,我们相信,虚构新的歧视叙事,在所有衰落的工业社会中,或多或少都会发生。北美的多民族福利国家,只是更容易受到诱惑,把收入再分配的成本强加给私营部门。它们通过把社会中各种亚文化的经济缺陷,归咎于整个社会结构,以及普通的白人男性,激起人们的不满与被亏欠的感觉,就有能力做到这一点。

\subsection{创新的大政治学}
其实,在信息技术对工业经济的“创造性破坏”形成威胁之前,它就已经使马克思主义者和社会主义者所膜拜的许多神话明显过时了。我们在前一章中研究了创新的大政治学,其中我们强调了一点,对正确认识信息革命的社会影响具有重要意义。从近几个世纪的发展来看,技能进步能够扩大社会就业,好像已成为经济生活中一条可靠的法则;但它未必就靠得住。社会收入完全有可能集中在少数富人手里。

\section{实际工资下降 50\%}
这种情形,确实发生在近代时期的头两个世纪甚至更长的时间里。从 1500 年左右的火药革命到 1700 年,在西欧的大部分地区,60-80\%底层人口的实际收入下降了 50\%或更多。在很多地方,实际收入的下降持续到 1750 年,而且直到 1850年才恢复到 1500 年的水平。

与最近 250 年的经验不同,在近代的前半个阶段,即西欧经济急剧扩张的时期,收入的增加是集中在少数人身上的。当前的信息技术创新,与近几个世纪以来,世界所经历的工业技术创新大相径庭。其不同之处在于,当前的大部分技术创新,具有节省劳动力的特点,它创造的是需要技能的工作任务,它会削弱规模经济的效应。而这与 1750 年左右迄今的发展经验正好相反。

工业创新为无技能的人提供了工作机会,并提高了企业的经济规模。这不仅使穷人不费气力地就提高了收入,而且还增强了政治体制的力量,使其更有能力控制社会动荡。在工业革命的早期,被机械化和自动化所取代的人,往往是熟练的工匠、手艺人和熟练工,而不是无技能的劳动力。纺织业尤其如此,它是最早大规模采用机械和动力设备的行业;这导致了卢德分子的暴力反动,在 19 世纪初的一次暴动中,他们砸毁了纺织机器,并杀害了工厂主。在另一个战场上,1830年在英国东南部发生了一场骚乱,其领袖是神秘的斯温上尉,他的追随者则主要是干农活的日工(day laborer)。他们的诉求包括:向当地富人征税,然后给他们发钱或提供啤酒;要求雇主提高日工的工资;以及“摧毁或要求摧毁新的农业机械,特别是脱粒机”,以减少农民对农村日工的需求。

马克思主义者与其他一些人,编造出充满浪漫主义的呓语,把节省劳力技术的暴力反对者,给美化成了英雄;而事实上,他们只是一群令人不齿的暴徒,他们反对提高了全世界生活水平的新技术,纯粹是出于自私。

内德·卢德和斯温上尉的暴力追随者,虽然在英国危害了好几个月的公共秩序,然而,一旦中央当局开始镇压,其运动的流产是必然的。大部分穷苦的、没有技能的人,不可能长期投入到一项事业中去,因为这项事业追求的是,摧毁供他们养家糊口的机器;这些机器不仅给他们提供了工作,还降低了他们生活必需品的成本,如保暖衣物和面包,从而提高了他们的生活质量。

\subsection{无技能者获得了高收入}
很快,工业和农业的自动化,就对一无所有的人产生了吸引力,因为这给他们创造了赚钱的机会,并降低了他们的生活成本。新工具使那些身无长技的人,也能够生产出与高技能者同等质量的产品。在流水线上,一个天才和一个白痴,生产同样的产品,赚取同样的工资。

在过去的两个世纪里,工业自动化极大地提高了无技能者的工资,特别是在资本主义蓬勃兴起的一小部分地区。而先进的工业企业的大规模发展,不仅使无技能劳动力获得了前所未有的工资回报,还促进了收入的再分配。

福利国家的产生,是符合工业主义技术发展逻辑的必然结果。领先的工业雇主,因为规模大、资本成本高,成为了最容易被征税的目标。而且,政府可以要求他们保存会计记录及扣押工资,这就使个人所得税成为了实际可行的税收技术;在之前几个世纪,经济比较分散的时候,这是不可能的。这一切的最终效果是,工业创新促进了规模经济的发展;政府变得更加富有,当然也更有能力维持秩序了。

\subsection{历史进程被逆转}
根据我们的判断,今天的形势则正好相反。信息技术正在提高有技能者的收入,并破坏大规模运作的组织,包括民族国家。

这也指出了信息时代另一件讽刺的事情,那就是自由市场的批判者对于工业主义工作的兴衰,持有的精神分裂般及根本性的阻挠态度。在工业主义的早期,他们对工业工业的所谓罪恶感到窒息,因为那些工作把无地农民从“我们已经失落的世界”中引诱了出来。根据批判者的说法,工厂的工作岗位,是前所未有的罪恶,是对工人阶级的“剥削”。但如今看来,唯一比工厂工作的出现更糟糕的是它的消失。那些曾经为工厂工作的出现而哀嚎的人,今天是他们的曾孙辈,又要为低技能技能、高收入回报的工厂工作的短缺而哀嚎。

贯穿这些抱怨的一条主线,是对技术创新与市场变革的坚决抵制。在工厂制度的早期阶段,抵制导致了暴力事件。在信息时代,这可能会重演。

而这次不是因为资本家在“剥削工人”。计算机作为一种技术典范,揭示了这种说法的荒谬性。对于某些思维涣散的人来说,假设一个几乎不识字的汽车工人,在生产过程中,被那些构建工厂并出资雇人的企业主所“剥削”,可能还有一些可信度。在有形产品的生产和销售中,概念资本的关键性,不如在信息时代的产出中那么明显,因为后者显然是需要大量脑力的。所以,认为企业家以某种方式攫取了,员工创造的信息产品的价值,这种说法就不太合理了。如果产品的价值明显是通过脑力劳动创造的,如消费类的软件,但却被认为是实际创作它的技术人员以外的他人的产品,那就很荒谬了。马克思主义者和社会主义者,在 19 世纪和 20 世纪的大部分时间里,都认为是工人创造了所有的价值,事实却远非如此;现在的问题恰好相反,一个日益明显的就业趋势就是摆脱无技能的劳动者。

这也引发了人们普遍的担忧,那就是无技能劳动者是否还能为经济发展做出任何贡献。

因此,要求收入再分配的理由,也从“剥削论”转为了“歧视论”。因为“剥削”假设的是低收入者具有生产能力,而“歧视”则不然。低技能者之所以未能发展出更高价值的技能,是因为他们受到了“歧视”。

这种歧视也成为了一种合理合法的依据,用来强加非最佳雇用标准及其他标准,更准确地说,就是向落后群体重新分配收入。例如,在美国,以种族为基础的成绩与能力的规范化测试,使黑人在客观分数较低的情况下,可以获得比白人和亚裔更多的机会。通过这种及其他手段,政府迫使雇主以较高的工资,雇用更多黑人和其他官方指定的“受害者”群体。任何不遵守规则的雇主,都面临着代价高昂的法律诉讼,其中涉及巨额的惩罚性赔偿。

指定受害者的目的,不是为了在工业社会的重要亚群体中培养偏执的迫害妄想症,也不是为了削弱反作用价值观的传播,而是为了减轻破产国家对收入再分配的财政压力。灌输迫害的妄想,只是一个不幸的副作用。讽刺的是,对“歧视”现象关注的激增,恰好发生在技术革命的初期;而技术革命却必然会使真正的、任意性歧视,比以往大幅减少。因为在互联网上,没有人知道或在乎,一款新软件的开发者是黑人还是白人,男的还是女的,同性恋还是素食主义的侏儒。

虽然现实中的歧视压迫感在未来一定会减轻,但这不一定能减弱要求“补偿金”的压力,要求弥补各种有的没的不公平待遇。每个社会,不管它的客观情况怎么样,都会制造出各种合理化收入再分配的理由。这些理由,有的微妙,有的荒诞;从圣经中“爱邻如己”的劝诫,到黑魔法的召唤。巫术、女巫和邪恶之眼,是宗教情感的反面,是税务局或国税局的精神写照。当人们不能被爱感动去资助穷人时,穷人自己也想看到他们被恐惧所驱动。制造恐惧有各种形式。有时候是赤裸裸的敲诈,刀子架在喉咙上,枪口对着脑门。其他时候,是通过伪装的或虚幻的威胁。近代早期的“女巫”,大多数都是寡妇或者贫困的未婚女性,这并不是巧合。她们用诅咒恐吓邻居,借此索求财务,往往都可以得逞。很显然,付钱的人绝不只是因为迷信;邪恶之眼的恶意可不是迷信,而是事实。即使是一个可怜的女人,她也可以放走别人家的牛,或者放火烧掉别人的房子。从这个意义上说,近代早期的巫术审判,并不完全像看起来的那么荒唐。虽然惩罚很残酷,而且毫无疑问,有很多无辜的人,因为邻居在麦角碱中毒的幻觉下,受到了牵连;但是,对女巫的起诉,可以理解为对敲诈勒索的一种间接控诉。

我们预计,随着信息时代的展开,以分享(敲诈)成就奖励为动机的敲诈勒索会重新出现。那些对以往的歧视感到委屈的群体,不会仅仅因为他们的要求变得不合理或更难实现,就很快放弃受害者的身份,这个身份对他们具有明显的价值。

他们会继续坚持自己的诉求,直到当地社会环境的发展,使他们确信再也得不到回报了。

非裔美国人和非裔加拿大人中间反社会行为的增长就说明了这一点。它说明了,在黑人的愤怒与对这一问题的根源进行现实评估之间,很难达到平衡。难以评估黑人的问题在多大程度上,是由自身反社会行为造成的。黑人的愤怒还在不断上升,即使他们的生活方式已经因此而日益紊乱。非婚生育飙升,教育程度下降,越来越多的黑人青年卷入犯罪活动,以致于在监狱里的黑人男性比在大学里的还要多。

这些违背情理的结果可能产生了暂时的效果,即在工业主义的黄昏时期,通过提高对整个社会的敲诈威胁,增加了资源向下层社会的流动。但这种效果转瞬即逝。

为了符合生产规范,福利国家降低了对表现不佳者的挑战,消除了竞争的积极意义,从而制造出大批技能地下、头脑偏执、适应能力差的群体,整个社会就是一个火药桶。民族国家的死亡和大规模收入再分配的消失,无疑会导致这些悲惨灵魂中的一些变态,对任何比他们生活更好的人发起攻击。因此,我们有理由相信,随着信息时代的展开,社会和平将受到威胁,尤其是在北美与西欧的多民族飞地。

\begin{tcolorbox}
我们永远不会放下武器,直到下议院通过一项方案,取缔所有伤害共同利益的机械,并废除绞死破框者的方案。但是我们……我们不会再请愿了,请愿没有用,我们必须战斗。  
\begin{center}
救世军将军内德·卢德 签名 \\
书记员 救世军万岁 阿门
\end{center}
\end{tcolorbox}


\subsection{新卢德分子}
鉴于 19 世纪初反技术暴乱的历史经验,以及欧洲和北美长期以来的集体暴力传统;接下来,如果看到新卢德派对信息技术和使用该技术的人发起攻击,应该没有人会感到惊讶。前面提到的卢德派,是集中活动在英国西约克郡的纺织工人,在 1811-12 年,他们发起了一场恐怖运动,针对自动裁切机及使用该机器的工厂主。卢德分子把脸抹成黑色,在西约克郡横行肆虐,焚烧工厂,杀害勇于利用新技术的工厂主。大部分的暴力事件都出自“裁缝”之手;在此之前,他们的工作是挥舞重大 50 磅的巨大剪刀,是毛布生产中的关键一环。但是裁缝的所有工作,就是用“起绒草使棉絮起绒,用剪子裁断布匹”;正如罗伯特·里德(Robert Reid)在《失落之地》(Land of Lost Content)——关于卢德运动始末最好、最全面的著作——一书中所言,“(这些工作)太简单了,不得不进行机械化。”这种机械化的裁切机,达芬奇就曾经勾画出了设计图;但是,他的构思却被搁置了几个世纪。最后直到 1787 年,有人重新发明了一种类似达芬奇的装置,在英国投入生产。里德指出,“这项技术的所有组成部分都早已广为人知,但令人惊讶的是,它竟然没有在更早的时候被引入。……工业革命带来的新设备,使用起来不需要什么力气和技术,于是很多岗位都被妇女和小孩抢走了,工资一开始都很低。其中一台新机器,即使由相对不熟练的工人操作,他在 18 个小时内完成的工作量,一个熟练的裁缝用手剪要干上 88 个小时。”需要注意到是,那些抨击机械化的工人,在反对新技术方面是很有讲究的。他们只攻击那些取代自己工作岗位和降低对熟练劳动力需求的技术。有一位名叫威廉·库克的企业家,在西约克郡引进了地毯编织机,就没有引发任何暴力事件。

没有人企图烧毁库克的工厂,或砸烂他的机器,更没有人想要谋杀他。罗伯特·里德在他的“卢德分子骚乱史”(即上述那本书,译注)中解释到,库克的新技术之所以没有激起任何反对情绪,因为地毯是一种“在那之前山谷中没有人专门生产的产品。”里德继续说道:“由于库克引入了一种新产品,它不基于任何传统的做饭,并且创造了就业机会,所以他的工厂得以蓬勃发展。”这个例子对未来的发展极具启发意义。它指出来,在下一个千年,有头脑的企业家,应该首先在欠缺生产任何产品或服务传统的地区,开展削减劳动力的自动化经营。

以史为鉴,可想而知,在新千年最初的几十年里,最暴力的恐怖分子不是无家可归的穷人,而是流离失所的工人,他们以前享受着中产阶级的收入和地位。1812年的卢德运动就是如此,其中大部分的卢德分子不是贫困的无产阶级,而是熟练的工匠,他们习惯了赚取比普通人高五倍甚至更多的收入。放到今天,他们的同类可能是被淘汰的工厂工人。然而不幸的是,在仔细查阅了大多数经合组织国家的人口统计数据后,我们发现,有更多的地区可以被列为潜在的暴力发生地。

全世界的民族国家,将想方设法对抗网络经济和利用它积累财富的主权个人。愤怒的民族主义情绪将席卷全球;其中一部分是对新科技的反对,类似工业革命时期英国的卢德运动和其他反技术的动乱。对这个问题,我们应该慎之又慎,因为它可能是新千年治理方式演变的关键。在未来的剧变中,核心的一大挑战是,任何在暴力升级的情况下维持秩序;或者说怎么逃避暴力的冲击。与信息时代息息相关的公司和个人,包括硅谷,甚至包括为新技术提供动力的电力供应商,都必须对自由职业者、新卢德派的恐怖主义保持特别的警惕。

不幸的是,随着人们对收入下降的失望和对成就者的不满与日俱增,像“智能炸弹客”这样的疯子,可能会刺激很多人进行模仿。我们估计,未来的大部分暴力事件都会涉及到爆炸。正如《纽约时报》所报道的,美国国内的恐怖主义活动在20 世纪 90 年代急剧上升。“在过去的五年里,它们增加了 50\%以上;在过去的十年里,增加了几乎三倍。爆炸性犯罪活动与未遂事件,从 1985 年的 1,103 起,上升到 1994 年的 3,163 起。在小城镇和郊区,以及在市中心的街头帮派中,普通人变身炸弹客的现象,在不断蔓延。”

\subsection{防卫将私有化}
尽管民族国家征收着惩罚性的税收,作为保护服务的代价,但在今后的日子里,它们不可能提供有效的保护。信息时代的新技术,意味着暴力规模的下降,这也使得大规模军事组织的作用大大降低。这不仅意味着它们在战争中的决定性会下降,还意味着国家切实保护公民的能力也会减弱;而且,美国作为世界超级大国的超国家霸权,在下个世纪将不如英国在 19 世纪的霸权那么好使。因为在第一次世界大战之前,以相对较低的成本,就可以强有力地把权力从核心区域向周边推进。在 21 世纪,大国对生命和财产安全构成的威胁,必然会随着暴力回报的下降而减小。暴力回报率的下降预示着,有能力大规模使用军事力量的民族国家或帝国,不可能在信息时代生存或形成。

在未来,要获得充分的防御,对财务上的要求会下降,在这种情况下,把保护服务看作是一种私人用品,是很有道理的。毕竟,安全威胁的规模越来越小,也可以更多地由商业化的保安力量去解决,如通过围墙、栅栏和安全边界等方式,把麻烦制造者隔绝在外。此外,富有的个人或公司也可以聘请安保人员,来应对信息时代可能出现的大部分威胁。在边境地区,军事威胁规模的缩小,可能增加无政府状态的危险,或者说在单一领土内出现暴力竞争的场面。但它也会促使各管辖区在竞争的基础上提供保护服务。这也意味着,各管辖区在保护、护照、领事及司法服务方面,竞争会更加激烈。

当然,从长远来看,主权个人可能将会凭借非政府文件旅行,例如由私人机构或亲和团体签发的、类似信用证一样的证照。假设在网络空间出现一个团体,一个商业共和国,像中世纪的汉萨同盟那样组织起来,促进各管辖区之间私人条约与合同的谈判,并为其成员提供保护,这绝不是痴人说梦。想象一下,未来会出现一种特殊的护照,由主权个人的联盟颁发,持有人受该联盟的保护。

不过,即使出现了这样的证件,它也只是摆脱民族国家及其豢养的官僚时代的临时产物。在近代以前,边境的界定都是很松散的,通过边境一般都不需要护照。

虽然中世纪的边境有时候需要安全通行证,但它们通常是由被访问国的当局签发的,而不是旅行者的国籍国。那时候比护照更重要的是介绍信和信用证,它们有助于旅行者找到住所并进行商业谈判。这一天会再次到来。最终,有身份的人出去旅行将无需携带任何文件;他们可以通过声纹系统或视网膜扫描识别自己的身份,这是一种万无一失的生物识别。

简而言之,我们预计,在下个世纪的上半叶,人类世界将迎来真正的主权私有化。

伴随着一些可以预见的条件变化,强制性领域将被压缩到逻辑上的最低限度。不过,对于下一个千年的世俗裁判者和反动派来说,把曾经“神圣”的国籍身份放到市场之上,作为一种通过成本效益核算的交易进行买卖,既令人愤怒又充满威胁。

在本书中,我们认为,要打赢一场信息战争,不再需要一个民族国家,由计算机程序员部署大量的“机器人”或数字仆人就可以。比尔·盖茨已经具备比大多数民族国家更强的能力,可以引爆全球脆弱系统中的逻辑炸弹。在信息战争的时代,任何一家软件公司,甚至山达基教会,它们所能造成的威胁,比在联合国拥有席位的大多数国家累积起来的还要大。

从逻辑上看,民族国家权力的丧失,是低成本、先进计算能力出现的必然结果。

微处理技术既减少暴力的回报,并为政府在工业时期垄断的保护服务,首次创造了一个竞争性的市场。

在主权商业化的新世界里,人们将可以选择他们的管辖区,就像现在选择保险公司或宗教一样。任何的管辖区,如果不能提供适当的服务组合——不管服务的内容是什么,都将面临破产和清算,就像无能的企业或失败的宗教团体一样。因此,竞争会调动各地管辖区的积极性,使其更加经济高效地提供优质的服务。各管辖区在提供公共产品方面进行竞争,它所产生的影响,和我们在生活中其他方面观察到的很类似。通常而言,竞争都会提高顾客的满意度。

\section{竞争与无政府状态}
需要牢记的一点是,我们所预期的管辖区之间的竞争,不是发生在同一地区内暴力组织之间的竞争。如前所述,暴力组织之间的竞争,往往会使暴力在生活中无处不在,从而破坏了经济发展的机会。正如莱恩所指出的:当与敌对的暴力集团相竞争或在领土内建立垄断时,暴力的运用显然具有很强的规模优势。

一个从经济层面对政府进行分析的基本事实是:使用暴力、控制暴力的行业是一种天然性的垄断,至少在土地方面是这样。在一个管辖区的领土范围内,里面所提供的服务,垄断者都可以用更廉价的方式生产出来。可以肯定的是,在某些历史时期,在同一块领土上,会有不同的暴力组织竞相要求人们支付保护费,例如三十年战争时期的德国。但这一状况,比同一地区内电话系统间的竞争更加不经济。 莱恩的评论在两个方面具有参考价值。首先,我们同意他的一般性结论,即主权国家倾向于在领土内实行垄断,因为这可以使它们提供更便宜、更有效的保护服务。第二个有意思的方面是,莱恩用过时的电话服务的垄断进行了比较。当然,现在我们都知道,电话系统不一定是垄断的。这就给我们的分析提供了一个警示。

人们普遍认为,在具体的领土范围之内,无政府状态是行不通的;但技术条件的变化,可能会在某种程度上,使这个一般性的结论变得过时。比如说,在一个不受强制的领域内,当网络资产发展到很大的规模,那么,保护服务的定价就更少是一个“需求”问题,而更多是一个市场谈判的问题。

不过,我们在此所指的与一般的无政府状态不同;我们指的是各管辖区之间互相竞争,而每个管辖区在自己的领土内则垄断着暴力。我们认为,这些管辖区会以符合成本效益分析的方式,尽可能提供最大价值的保护服务,以吸引其“客户”。

另外,毫无疑问的是,在信息时代,保护服务的供应情况会更加模糊,私营的治安和防卫服务,会比我们在过去所看到的要完备得多。而且,我们所设想的竞争,与多家保护机构在同一地域内争夺不同客户所导致的冲突不同,后者是真正的无政府状态。

尽管如此,随着主权数量的增加,以及个人在积累了足够多的资源之后,承担更多主权者的职能,这必然意味着,世界上无政府状态的范围会扩大。主权国家之间的状态一直是无政府主义的;也从来没有一个世界性的政府,来规范各个主权国家的行为,无论是部落国家、民族国家还是帝国。就像杰克·赫舒拉发所写的那样,“从原始部落到现代民族国家,所有的关系在内部都有某种形式的法律来管理;而它们之间的外部关系,则基本上还是无政府状态。”当世界上的主权实体越来越多,不可避免会有更多的关系涉及到不止一个管辖区,无政府状态必然会增强。

需要注意的是,无政府状态或缺乏压倒性的权力进行争端仲裁,并不等同于完全混乱,或无规则、无组织。赫舒拉发指出,无政府状态是可以分解的,“部落间或国家间的系统,也有其可分析的规律性和系统性模式。”换句话说,就像数学中的“混沌”,其实包含着复杂而高度有序的组织形式;同样,无政府状态也不是完全无序或结构不清。

赫舒拉发分析了一系列无政府状态的现象。除了主权国家之间的关系外,还包括禁酒时代的芝加哥帮派斗争,以及“加州淘金热中矿工与强占者(Claim Jumper强占他人土地的人)之间的对抗。”请注意,在 1849 年淘金热爆发时,加州已经是美国的一部分,但金矿区的状态还是被恰当地描述为了无政府状态。赫舒拉发指出,(因为)“官方的法律机构无能为力。”他认为,尽管缺乏有效的执法保护,但山区营地的地形条件,加上矿工为打击强占者而进行的紧密组织,使得外来者很难夺取金矿。换句话说,在一定的条件下,即使处于无政府状态,有价值的财产也能得到有效保护。

问题是,赫舒拉发基于达尔文“自然经济”理论对自发秩序的动态分析,对信息时代的经济发展是否有指导意义。我们觉得是有的。虽然我们并不认为会出现普遍的无政府状态,也不会到处都是金矿区那样的环境条件,但我们预计在整个世界体系中,无政府关系的数量会增加。基于这种预期,赫舒拉发的论点是有启发意义的;他认为,在一定条件下,“两个或多个无政府主义竞争者”,可以“在平衡的状态下,从社会可用资源中分获供各自生存的份额。”他还特别探讨了,无政府状态在什么时候容易“崩溃”,沦落为暴政或等级统治;当无政府主义的各方被一个压倒性的权威所征服时,就会发生这种情况。

在信息时代弄清楚这些问题,比在工业时代更加重要。对无政府状态进行更细微的区分,在近几个世纪也不如在新千年中那么重要。部分原因在于,现代时期的暴力回报率是不断上升的。这意味着,集结越来越多的军事力量——像民族国家在过去几个世纪所做的那样,往往会赢得决定性的战争。根据定义,决定性战争就是,将资源控制权的多方竞争者,归置到一个更强大权力机构的统治之下,从而制伏了无政府状态。另一方面,与军事技术中防御优势的下降相对应,战斗的决定性作用也在下降,这有助于无政府状态保持动态的平衡。因此,信息技术的强大冲击,会使军事行动的决定性作用降低,从而使小国之间的无政府状态更加稳定,不容易被大国所征服所取代。战争的决定性的降低,也意味着战争将会减少;对信息时代的世界来说,这是一个令人鼓舞的推论。

\subsection{可行性}
无政府状态得以维持的另外一个重要条件,是生存能力,或者说有充足的收入。

缺乏足够收入以维持生活的个人,有可能会(1)为了夺取生活资源而投入主要的精力去斗争,或(2)向另外一个竞争者屈服,以换取食物和生存所需。在 1000年时的革命过程中,封建主义的兴起就引发了类似的情况。我们估计,在西方国家中,以前依赖于国家转移支付的低收入者,未来会越来越多地以侍从的身份附属于有钱人。尽管如此,在霍布斯式的混战(或所有人对所有的战争)中,部分竞争者无法生存的单一事实,并不能说明什么。就像赫舒拉发所言:“在无政府的状态下,仅仅是低收入的事实,……本身并不能表明接下来会发生什么。”
\subsection{资产的特性}
无政府状态可持续的另外一个有趣的条件是,资源的“可预测性和可防御性”。

在赫舒拉发的分析中,“无政府状态是一种社会制度安排,竞争者在其中为了控制和捍卫耐用资源(durable resource)而斗争。”他定义的耐用资源包括“土地和可移动的资本财物”。在信息时代,数字资源可能被认为是可预测的,但它们不属于赫舒拉发所认定的,“领土性”或无政府状态下的“耐用资源”。事实上,如果数字货币能以光速在地球上的任何地方转移,那么征服网络银行所在的领土就是浪费时间。想要压制主权个人的民族国家,必须同时夺取世界上所有的银行避风港和数据避风港。即便如此,只要加密货币设计得当,民族国家也只能破坏或摧毁某些数字货币,而无法夺取它们。

所以,可以得出一个结论,在即将到来的信息时代,对于富人而言,最可预测的也最脆弱的资产,可能是他们的人身,也就是他们的生命。这也是为什么我们会担心,在未来几十年会发生卢德式的恐怖主义活动;其中一些活动,可能还会受到民族国家雇佣的煽动者的暗中怂恿。

不过,从长远来看,我们不认为主要的民族国家能够成功压制主权个人。首先,现有的国家,特别是资本匮乏的归家,它们会发现,相比保持与北大西洋民族国家的团结,以及维护“国际”体系的神圣性,为主权个人提供庇护会得到更多的好处。破产的、高税负的福利国家,想把“它们的公民”和“它们的财产”留在“它们的国家”内,这个动机和理由,对其他数百个分散的主权来说,并不足以令人信服和遵从。

我们要说,尽管世界上有数以千计的多国组织,旨在制约各个主权国家的行为;而且毫无疑问,其中一些组织是很有影响力的,比如欧盟和世界银行。但请记住,那些欢迎主权个人的国家,会从他们的加入中获得巨大的利益。即使像美国这样的猪头(pigheaded,顽固刚愎的)强国,在当前的趋势下,也必须大力防止出现不受政府控制的网络经济;而且说到底,它也不希望把全世界那些在银行有存款、但不想成为美国人的居民排斥在国门外。这一点尤其关键,因为购物是现在旅行者的主要兴致所在。所以,最终,虽然会晚于其他国家,但迫于竞争压力,美国或其部分地区,将加入到主权商业化的行列。

\subsection{有需求就有供应}
资产负债表最薄弱的民族国家,会更早更强烈地感受到这些压力。未来的“离岸”中心,将是当前民族国家中分裂的地区和飞地;如加拿大和意大利,几乎肯定会在 21 世纪的 1/4 世纪结束前解体。一个全球性市场的诞生,有助于高质量、高成本效益的管辖区的形成。和普通商业中一样,小规模的参与者更加灵活,更有能力进行竞争。所以,人口稀少的管辖区,更容易打造自己的制度结构,实现高效运作。

信息精英们将通过签订合同的方式,以合理的价格寻求高质量的保护。虽然这笔费用远不足使民族国家的全体人口重新分配到明显的利益,因为它们往往有几千万到几亿人之多;但在一个人口只有几万或几十万的管辖区,这笔费用绝非微不足道。少数巨富者所带来的税收和其他经济优势,对一个人口更少而不是更多的管辖区而言,人均的获益要高得多。

在信息时代,一个人的企业在哪里注册已经不重要了,除了一些纯粹的消极意义,如某些地址意味着更多的义务;小的管辖区会发现,它们更容易打造出能获得商业成功的保护服务。所以,人口较少的管辖区,在制定对主权个人具有吸引力的财政政策方面,享有决定性的优势。

我们相信,民族国家的时代已经结束了;但这并不等于说,民族主义作为一种人类的情感牵系,它的魔力会立即平息。作为一种意识形态,民族主义深深地植入了人类普遍的情感需求之中。我们都有过敬畏的经历,如第一次看到巨大的瀑布,或第一次站在大教堂的入口。我们也都有过归属感的体验,比如在家庭圣诞聚会上,或者作为某项运动中胜利团队的医院。对这两种强大的情感,人类势必要在文化上做出回应。我们被自己国家的历史文化所照亮,而它又是更大的人类文化中的一部分。因为知道自己归属于某个文化群体,我们会倍感欣慰,它使我们获得了参与感和认同感。

这些文化符号的影响,可以激发最强烈的情感效应。美国人对国旗、国歌或感恩节的联想,英国人对君主制或板球的联想,都分别会对美国人和英国人的想象力产生真实的影响,并通过重复而不断加强,然后深入到潜意识之中。这些符号会告诉我们,我们是什么样的人,并提醒我们有着什么样的民族文化。当反越战的示威者想要撼动其他美国人时,他们点燃了美国的国旗。被社会疏离的英国人为了攻击君主制,甚至会在板球场上挖洞。

这些情绪的触发点是肤浅的,但并非无足轻重。它们的背后,是我们被教导要为之流血牺牲的东西。无论大政治条件如何变化,或由此导致的体制如何变革,在那些和我们一样在 20 世纪长大的人的想象中,它们依然占据着重要的一席之地。