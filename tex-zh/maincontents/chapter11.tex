\chapter[信息时代“自然经济”中的道德与犯罪]{信息时代“自然经济”中的\\道德与犯罪}

\begin{tcolorbox}
腐败现象,远比人们以前想象的要普遍和广泛。腐败的证据随处可见,主要在发展中国家,在工业国家中也日益高发。……包括国家总统和部长在内,著名的政治人物都被指控腐败。……在某种程度上,这代表了国家的私有化,但它并不像通常所说的私有化,即把国家权力转移到市场上,而是转移给了政府官员和官僚系统。
\begin{flushright}
—— 维托·坦吉(Vito Tanzi)
\end{flushright}
\end{tcolorbox}


我们认为,随着现代民族国家的解体,当代的野蛮人会越来越多地在幕后行使实权。像继承了前苏联衣钵的俄罗斯黑手党、其他种族犯罪团伙、前苏联党政组织中的头面人物、毒枭以及叛逆的秘密机构,都会逐渐地制定自己的法律。他们其实已经在这么做了,而且远远超出常人理解的程度;当代野蛮人已经渗透到了民族国家的形体之中,只是没有明显地改变它的表象。他们是一群微寄生虫,以这个垂死的系统为食。他们在更小的范围内使用国家级的技术手段,并且像战争中的国家一样残暴、不择手段;其日益增长的影响力和权力,是政治萎缩后果的一部分。微处理技术,降低了有效使用和控制暴力的必要的团队规模。随着这场技术革命的深入,掠夺性暴力会越来越发生在中央控制之外的边缘地区。遏制暴力的方式,也将从依赖于权力的大小,转移到更加依靠效率的手段。

在这场世界剧变的大戏中,民族国家内部隐蔽的犯罪活动与激增的腐败现象,将会成为重要的辅助剧情。你可能会看到一部隐秘又险恶版的恐怖片《天外魔花》\footnote{Invasion of the Body Snatchers,又译人体入侵者。}。在大多数民族国家陷入明显的崩溃之前,它们将被当代的野蛮人所控制。而就像上世纪 50 年代这部著名的B 级片里所演的,野蛮人往往会精心地伪装自己。所以,未来的豆荚人不是来自太空的外星人,而是来自各种背景的罪犯,他们会担任官方职位的同时,还至少会部分地效忠于宪法以外的某种秩序。

一个时代将尽时,通常都是腐败严重的时期。旧制度的约束被瓦解,社会风气也随之消散,一种新环境应运而生。而在这种环境里,身居高位的人会把公共事务与私人犯罪结合起来。

遗憾的是,你无法通过常规的信息渠道,来准确、及时地了解民族国家的衰败程度。曾经掩盖了罗马帝国消亡的那种“持续的虚构”,也许是大型政治实体衰败时的典型特征。现在它正在掩饰和掩盖民族国家的崩溃。出于种种原因,你不能总是依靠新闻媒体告诉你真相。很多人在某种意义上都是保守的,因为他们代表着过去的政党。有些人则被不合适应的社会主义和民族国家的意识形态所蒙蔽。

有些人因为具体的原因不敢揭露腐败,眼看着它在一个腐朽的系统里越滚越大。

其中一些人是单纯地缺乏做这种事的勇气,另外一些是担心自己的工作,或者害怕因发声而遭到报复。当然,也没有理由认为,相比意大利修路承包商,记者和编辑更不容易搞腐败。那些热衷于报道各种新闻的重要信息机构,其实远不像人们想象的那样,具有可靠的信息来源。其中很多人都有着其他的动机,包括为一个摇摇欲坠的系统进行鼓吹;他们会把这些动机放在为你提供真实的信息之前。

他们看到的很少,说出来的更少。

\section{超越现实}
随着虚拟现实和电脑游戏技术的不断进步,你甚至可以在夜间新闻报道中,订购一份你想听的虚拟新闻。想在新闻中看到自己获得了奥运会十项全能冠军吗?没问题。它将是明天的新闻头条。你可以在你的电视或电脑上,看到任何自己想看的新闻故事,无论真的假的,比 NBC 或 BBC 现在所做的一切都要逼真。

我们正在快速进入一个由人类智慧创造的新世界,在这里,信息将完全摆脱现实的束缚。当然,这也会极大地影响你接受到的信息的质量和性质。在一个人造现实的世界里,所有信息都是即时传送的,准确的判断和区分真假的能力,将更加重要。

由于科技放大的缘故,真假之间的区别将普遍变得模糊。但与我们目前的环境相比,变化程度会比很多人想象的要小。之所以这么说,是我们看到了,信息革命的结果,大都是解放性的。

技术已经在超越地理距离和政治统治。政府可以设置壁垒来阻碍贸易,但它们很难阻止信息的传输。在香港的任何一家餐厅,几乎每一位顾客都可以通过手机与全世界联系在一起。1991 年 8 月,强硬派在莫斯科发动了政变,但他们无法关停叶利钦的通讯,因为他有手机。

\subsection{更多信息,更少理解}
当信息传递的壁垒被打破,信息将越来越多,这当然是好事。但这有什么意义,很多人会为此感到困惑。现代技术,不仅有助于把信息从政治控制和时空障碍中解放出来,它也会提高老式判断力的价值。那种洞察力——能够从堆积如山的事实和幻象中,分辨出什么是重要的和真实的,其价值与日俱增。主要基于三个原因:

1. 现在可获得的信息是海量的,简明扼要就很重要。而简明会导致缩略。缩略会漏掉很多不熟悉的内容。当你有大量的事实要消化,有很多的电话要回复的时候,你自然希望每个相应的信息都尽量简洁。但不幸的是,简略的信息会导致糟糕的理解。更深层、更丰富的历史纹理,往往会被 25 秒的新闻片段给编辑掉,或者在 CNN 上被误传。传播一个已被广泛理解的主题的衍生信息,比探索一种新的理解范式要容易得多。你可以很轻松地报道一场棒球或板球比赛的结果,但要解释棒球或板球的玩法、规则及意义,就没那么简单了。

2. 日新月异的技术发展,正在破坏社会和经济组织的大政治基础。如此一来,关于世界运行的普遍的理解范式,或隐藏的理论,比过去更容易过时。这增加了全面概览的重要性,而降低了个别“事实”的价值;因为这些事实,任何会使用信息搜索系统的人都能轻易获得。

3. 生活中日益严重的部落化和边缘化,对对话乃至思维都产生了抑制。很多人因此养成了一种习惯,对他们所掌握的事实中明显包含的结论避而不谈。最近一项伪装成民意调查的心理学研究表明,个别职业群体的成员,几乎是全体一直地拒绝接受,任何表明他们的收入将会减少的结论,无论该结论背后的逻辑多么严密。鉴于专业化程度的增加,为大多数专业化职业群体所提供的大部分解释信息,都是为了迎合群体本身的利益。而对于那些可能不文明、无利可图或政治不正确的观点,他们往往不感兴趣。这是一种普遍的趋势,其中最好的例子,莫过于蛊惑投资股市前景一片光明的信息。这些信息基本都是由经纪公司炮制出来的,而很少有经纪公司会告诉你,股票或股市被高估了。他们的收入就来自于交易佣金,而这取决于大多数的客户是否准备买卖。总之,你很少能听到独立的、反面的声音。

由于种种原因,信息时代尚未成为理解时代。相反,公共讨论的严禁性会急剧下降。现代人所能知道的东西,是过去任何时候的祖先都望尘莫及的。但是,几乎没有什么公共的声音来评估事件的意义,指出什么是真的,什么是假的。这就是为什么我们一直饶有兴致地观察着,媒体对报道美国政府高层耸人听闻的腐败事迹兴趣淡薄,特别是美国媒体。

在本书当中,我们要解决的一个核心主题是,不断变化的技术和其他“大政治”因素会如何改写“自然经济”理论。“自然经济”基于达尔文式的“自然状态”;在这种状态下,结果由自然力量决定,所以有时候是不公平的。在“自然经济”中,有一个重要的行为链,就是生物学家所说的“干扰竞争”(interferencecompetition,或译为互涉竞争)。

\subsection{干扰竞争}
“干扰竞争的参与者”,就像杰克·赫舒拉发所说的,“会直接攻击或阻碍对手,来获得并保持对资源的控制。”无论我们是多么地希望,人类的行为会遵循法律规则和“其他被强制执行的社会游戏规则”(“政治经济学”的说法);但有充分的证据表明,许多人只在有利于自己的时候才会“按照规则行事”。冲突问题的权威赫舒拉发是这么说的,“犯罪、战争和政治的持续存在告诉我们,现实中的人类事务,在很大程度上仍然受制于自然经济的潜在压力。”换句话说,经济结果只是部分地像教科书中所描述的,由经济理性人和平守法的行为所决定;这些人尊重产权,“不会直接拿走不属于他们的东西。”而实际上,经济成果也是由冲突造就的,包括公开的暴力。经济学家赫舒拉发指出,“即使在法律和政府的约束之下,理性的、自利的个人获取资源的手段,也会在合法与非法之间取得一个平衡——一方面是生产和交换,另一方面是盗窃、欺诈和勒索。”

\section{信息时代的抢劫}

关于暴力、犯罪和政治,米歇尔·加芬克尔和斯特吉奥斯·斯卡帕达斯(MichelleR. Garfinkel and Stergios Skaperdas),出版了一本非常有价值的书籍《冲突与侵占的政治经济学》(The Political Economy of Conflict and Appropriation),书中就谈到了这一点:“个人和团体可以从事生产并由此而创造财富,也可以夺取他人创造的财富。”他们援引了一个现代版干涉竞争的故事,最早由《经济学人》杂志报道出来:“说有一个美国商人,最近来到莫斯科开设办事处,在他住的旅馆里遇到五个人。这些人戴着金表,拿着手枪和他公司净资产的打印材料;他们所要他公司未来收益的 7\%。美国商人坐第一班飞机回了纽约,那里的歹徒还没这么老练。”这个信息时代的抢劫故事,要更多地归功于新技术的进步,而不单是俄罗斯黑帮通过互联网,就可以获得受害者的财务资料和信用报告这么简单。

\subsection{军事力量的决定性下降}

无论是福是祸,大规模的军事力量不再具有决定性的作用。信息技术从根本上削弱了民族国家,在这个不守规矩的世界里,强行施加权威的能力。如果像伏尔泰说的那样,上帝曾经“站在阵营更强大的一边”;那么,随着暴力能够获得巨大回报的日子一天天消失,上帝的支持也会越来越小。所以,我们将会看到相反的情况。有更多的证据表明,暴力的回报率正越来越低,这强烈地预示着,像民族国家这样的大型集团,无法再说服人们接受,其巨额的管理成本是合理的。

中央权力定于一尊的能力的下降,最明显的证据就是恐怖主义的兴起。90 年代中期在美国发生的引人注目的爆炸事件表明,就是是世界上的超级军事强国,也不能免于被攻击。

暴力回报率下降的另一个重要表现,是黑帮主义和有组织犯罪及其必然后果——裙带关系和政治腐败——在世界范围内的增长。它们反映出一种普遍的非道德化的氛围,在这种氛围中,国家可以进行强制,但不提供保护。那么,随着国家对暴力垄断的逐渐耗尽,新的竞争者会进入到这个领域,例如在莫斯科向美国商人征收私税的黑帮分子。

小型团体、部落、黑社会、帮派、流氓、民兵,甚至是单独的个人,都获得了越来越多的军事效能。在下一个千年的“自然经济”环境中,他们能行使的实权将远远超过 20 世纪。采用微芯片的武器,已经使暴力的平衡向防守的方向偏移,决定性的侵略不再有利可图,因此也就不太可能再发生。例如,像“毒刺”导弹这样的智能武器,就把富裕的大国以前部署的昂贵的空中力量,相对于贫穷小国的优势给有效地抵消了。

\subsection{未来的信息战争}
“信息战争”的可能性,已经映入眼帘;虽然它被广泛讨论,但却少有人理解。

信息战也表明暴力的回报率越来越低。“逻辑炸弹”可以瘫痪掉空中交通管制系统、铁路调配机制、发电机和配电网络、供水和污水处理系统和电话继电器,甚至军方自己的通信系统都会失效或被破坏。随着人类社会越来越依赖于计算机的控制,“逻辑炸弹”的破坏力,一点不亚于物理爆炸。

而与传统炸弹不同,“逻辑炸弹”可以被远程引爆;而且不仅是敌方政府,自由职业的程序员团体,甚至是有天赋的个人黑客,都可以引爆它。请注意,一名阿根廷少年就因多次入侵五角大楼的计算机,而于 1996 年被捕。虽然到目前为止,黑客们还很少以破坏性的方式篡改计算机控制系统,但这并不是因为有真正有效的方式,可以阻止他们。

当信息战真正打响的时候,参战的对手可能不只是政府。像微软这样的公司,进行信息战的能力,肯定比世界上 90\%的民族国家都要强。

\subsection{主权个人的时代}
这也是我们将本书命名为《主权个人》的部分原因。随着战争规模的缩小,防御和保护也可以小规模地进行。因此,它们会逐渐成为私人物品而不是公共物品,将由私人承包商基于营利的目的来提供。这一点在北美的警务私有化中已经很明显。在美国,“安保”(security guard)是增长较快的职业之一。根据预测,到2005 年,私人安保的数量,将比 1990 年增加 24\%到 40\%。

警务私有化已经是一个明显的趋势。不过,就像盎格鲁-爱尔兰人、哈米什·麦克雷(Hamish McRae)大师所指出的,这不可能是政府深思熟虑后决定的结果。

他在《2020 年的世界》(The World in 2020)中写道:没有一个政府做出过具体的决定,要退出部分警务工作;事实上,也没有一个政府退出,是私人部门在进入。部分原因是人们认为警察服务的失败,部分是由于其他的社会变化;总之,私人安保公司逐渐接手了很多保护工作,如针对办公大楼或购物中心内的普通平民。洛杉矶的门禁社区显示,在某种程度上,人们甚至在向中世纪的城市概念迈进;在那里,市民生活在由警卫巡逻的城墙后面,只有通过被把守的大门才能够进入。 我们认为,20 世纪政府的几乎所有职能,将迎来全面的私有化,警务工作只是一个前奏。因为信息技术削弱了,政府实施权力以及为大规模组织提供物理安全的能力;而且,“自然经济”中每家企业的最优规模都在下降。

为了应对这一技术变革,需要大量的投资需求(或机会),把中心化的脆弱系统,重新设计为分布式系统。如果不解决大规模的脆弱性,相应的系统将面临灾难性的后果。

即使不进行设计,迟早有一天,大型官僚机构和公司所提供的产品于服务,也会自动成为高度竞争的市场,通过一种分布式的、分散的网络来管理,而不是由一个“总部”来控制。

拥有一个总部的公司是脆弱的,因为它可能被官方纠察队包围,或者被恐怖分子所破坏。就像《连线》杂志主编凯文·凯利在《失控》中所写的,公司最终将成为没有实体地点的“虚拟公司”,“同时在许多地方运营”。凯利看到,技术已经改变了生产过程,再把它置于集中控制之下已经没有必要;“在工业革命的大部分时间里,真正的财富都是通过把所有工序集中在一个厂房下创造出来的。规模越大,效率越高。”现在不是了。

凯利预见到,未来的汽车,即“新贵”汽车,可能只需要十几个人,通过一家虚拟公司合作设计出来,然后就可以投入生产。

在未来,规模过大不仅会适得其反,而且还很危险。大型企业总是会成为更诱人的猎取目标。地下经济从业者都知道,避税的秘诀之一就是避免被发现;相比那些总部位于摩天大楼、招牌闪亮的老牌公司,这对小型的“虚拟公司”要容易得多;前者肯定更容易受到“带着金标、手枪和公司资产打印材料的人”的注意。

这些黑帮分子会像在俄罗斯一样,在全球各地征收打着自己品牌的私人税收。各种规模的企业都可能遭到有组织犯罪团伙的敲诈和勒索。

\begin{tcolorbox}
敲诈勒索者的定义,就是制造威胁,然后收钱减少威胁的人。按照这个标准,政府所提供的保护,基本上都可以算作敲诈勒索。
\begin{flushright}
—— 查尔斯·蒂利
\end{flushright}
\end{tcolorbox}

\subsection{自然界讨厌垄断}
随着“大部队”(bigger battalions)对暴力的垄断被打破,可以预见,首先会出现的后果就是有组织犯罪的日益猖獗。毕竟,有组织犯罪团体,是各民族国家进行暴力掠夺的主要竞争者。这么说可能有点不礼貌,但政治学家查尔斯·蒂利提醒我们,不要忘记,政府本身——“具有合法性优势的典型保护网”——就堪称是最大的有组织犯罪集团。

如果你对这个世界一无所知,只知道有一个重要的垄断组织正在瓦解,那么你可以做出的一个最简单、最确定的预测是,最接近它的竞争对手将获得最大的利益。

因此,贩毒集团、黑帮、黑手党和各种类型的黑社会,在世界各地泛滥,并非巧合。

\subsection{权力系统}
从俄罗斯到日本再到美国,有组织犯罪在经济运行中的重要性,远远超过经济教科书上的描述。西西里人称之为“sistema del potere”,即“权力系统”的有组织犯罪,将对经济的运行发挥越来越重要的决定作用。

欧洲警察的官方报告说,包括俄罗斯和意大利黑手党在内的国际犯罪集团,在资助近年来蹂躏了巴尔干地区的种族灭绝战争中,发挥了“主导作用”。

在全球其他地区最近的内战和叛乱中,毒贩的资助也起了关键的作用。西班牙加他罗尼亚的国家警察缉毒队对长胡里奥·费尔南德斯说:“从 1986 年到 1988年,西班牙 80\%的海洛因是由泰米尔猛虎组织的游击队,与住在巴塞罗那或马德里的巴基斯坦人合作贩运过来的。当我们逮捕这些人并摧毁该网络之后,取而代之的是来自土耳其的库尔德人,他们在接下来的两年里完全主导了这个网络。”问题是,每当新的内战或叛乱发生时,那些穷困潦倒的战斗人员,都可能会通过贩毒和清洗毒资来资助他们的军事活动。

\subsection{由毒品资助的折扣}
有组织犯罪集团的活动,对毒品以外的其他商品价格造成了下行压力。在微观层面上,犯罪集团是利用犯罪企业搞到的赃物,来补贴表明上合法的企业。他们可以用低于成本的价格销售普通商品,以此来清洗毒品利润或其他非法资金,从而压低了合法竞争者的价格,导致很多破产。

\subsection{山口组与通货紧缩}
势力强大的日本黑帮山口组,在 1980 年代末的日本房地产泡沫中,扮演了很关键的角色。尽管九万名山口组成员每年的总收入在 101.9 亿美元(官方估计)到713.5 亿美元(夏藤高津教授的估计)之间;但导致日本银行破产的大量无法收回的贷款中,很大一部分都贷给了由山口组支持的交易。这导致的一个后果就是日本经济的通货紧缩——日本人习惯称之为“价格破坏”。

\subsection{熟视无睹}
叶利钦自己承认,俄罗斯的黑帮已经与“商业组织、行政机构、内务部门、城市当局等”融合到了一起。黑帮与警察融合后就获得了犯罪豁免权,所以,他们可以公然地暴力强征私人税收。权威人士指出,现在俄罗斯的企业,每 5 家有 4家都要交纳保护费。“根据一些报道,俄罗斯当地的小企业要向勒索者交纳 30\%到 50\%的利润,而不是向美国商人索要的 7\%那么微薄。”在 1993 年,俄罗斯发生了 355,500 起被官方认定为“敲诈勒索”的罪行,其中包括“30,000 起有预谋的杀人案”,大部分都是黑帮团队对商人的暗杀。据前内政部长维克多·叶林将军说,“主要都是因为商业和金融领域的冲突,买凶杀人。”而政府当局往往都“熟视无睹”。

经济学家詹卢卡·菲奥伦蒂尼(Gianluca Fiorentini)和萨姆·佩尔茨曼(SamPeltzman),在《有组织犯罪的经济学》(Economics of Organized Crime)中写道,犯罪组织“通过对胁迫和腐败的控制”,在经济生活中扮演着关键的角色。从理论上讲,这种影响有时候也是正面,因为它制约了监管,并可能激励政府改善其公共服务。一个强大的黑手党,可以“制约政府当局的垄断程度”。当境内存在实力强大的有组织犯罪集团时,政府只能勉为其难地接纳黑帮反对的政策。

\subsection{官匪勾结}
事实上,需要看到的是,大多数政府都很少直接对抗黑帮,虽然黑帮是它们在有组织强制方面的主要竞争对手。从严格的经济学角度看,这一点也不奇怪。公共行政部门的当选者所能成就的最有利可图的措施,就是与有组织犯罪集团达成“合谋协议”。菲奥伦蒂尼和佩尔茨曼指出,“有证据表明,在大量的协议中,犯罪组织确保对候选人团伙的政治支持,而后者则通过在公共采购和公共服务方面的便利或进行补贴来回报。”与好莱坞电影给人的印象相反,渗透和欺诈政府,现在是西西里黑手党及其他犯罪组织的重点工作之一。“大多数学者认为,到目前为止,西西里黑手党最大的业务,恰恰是侵占不同来源的公共开支,以及针对地方、国家和欧共体的补贴项目的欺诈。”

\subsection{毒枭共和国}
在《大清算》一书中,我们提出过警示,世界上很多国家的政府都被毒枭彻底腐蚀了。墨西哥就是一个无可争议的例子。前墨西哥联邦副总检察长爱德华多·瓦莱·埃斯皮诺萨在辞职声明中,把墨西哥的制度说得一清二楚。“任何人设计的政治项目,如果不让毒贩头目或者他们的金主分一杯羹的话,都是不可能的。如果你坚持要做,你就会死。”瓦莱表示,收受贿赂使墨西哥警察局长的收入无比丰厚,以致于有人为了得到录用,最高支付 200 万美元去贿选。即使经过严格的盈亏核算,买通一个地方的警察办公室也是一笔超值的投资。贩毒集团甚至会像级别很低的墨西哥官员行贿,因为这可以使他们的犯罪行为被免于起诉。

哥伦比亚是另外一个政府高层被毒枭控制的国家。美国政府最近撤销了哥伦比亚总统埃内斯托·桑佩尔的签证,因为他在明知的情况下接受毒贩的政治现金,为其提供好处。

\subsection{五十步笑百步}
任何人如果在 1990 年代关注过,我们《战略投资》的通讯报道的话,都会立刻意识到,克林顿政府对桑佩尔的姿态颇具讽刺意味。因为有可靠的证据表明,桑佩尔被指控的行为,美国总统比尔·克林顿都做过,而且更严重。如果你不相信我们的话,那么有两本做过深入研究的书籍,双方的作者持相对的政治立场,书中都用丰富的细节强调了克林顿不堪的背景。

罗杰·莫里斯(Roger Morris),总体上持左翼观点,他曾是尼克松政府的国家安全官员,也是迪安·艾奇逊(Dean Acheson)、林登·约翰逊(Lyndon Johnson)总统和沃尔特·蒙代尔(Walter Mondale)的高级助手。莫里斯拥有哈佛大学博士学位。他的《权力的伙伴》一书,详细描述了克林顿肮脏的过去,相比之下,桑佩尔简直就是一个童子军。

莫里斯讲述了克林顿在阿肯色州温泉市的童年,他自幼失父,当地是赌博、卖淫和有组织犯罪的中心,他的大部分家人都与这些活动有一定的联系。他的继叔雷蒙德·克林顿,一直被比尔·克林顿视为是“父亲的形象”,据说他是迪克西黑手党的“教父”级人物。

莫里斯指称,比尔·克林顿入职了中情局,在牛津大学的学生时代,他一直在监视反越战的活动家。在莫里斯看来,克林顿在担任州长期间,一直是中情局的资产,为中情局在阿肯色州以梅纳市为中心,进行的毒品和枪支走私活动提供便利。

莫里斯似乎是把中情局作为一个整体,来控诉其贩毒行为,而没有考虑克林顿也是该腐败机构的一份子;而在我们看来,后者的可能性更大。但不管是哪种情况,都表明了美国政府重要的秘密情报机构,直接或间接地参与了大规模的有组织贩毒活动。如果中情局还不是有组织犯罪的附属机构,那么它已经很危险地接近于这个角色了。

\subsection{250,000,000 分之一的概率}
尽管如此,《权力的伙伴》中包含的细节,会让每个研究现代美国政治腐败的学生感兴趣。不过,莫里斯的矛头也不是全部指向比尔·克林顿,还有他的妻子。

例如,关于希拉里进行商品交易的这段描述,希拉里的操作堪称神奇。“在 1995年,奥本大学和北佛罗里达大学的经济学家们,利用所有的交易记录以及华尔街日报的市场数据,对第一夫人的交易进行了复杂的计算机模型统计,结果发表在《经济与统计杂志》上。根据他们的计算,希拉里·罗德姆进行合法交易的概率,不到 2.5 亿分之一。”在克林顿任州长期间,阿肯色州的贩毒和洗钱活动蓬勃发展;莫里斯收集了大量详细的罪证。“经由航班贩运带来的毒品和资金规模巨大,使得阿肯色州小小的梅纳市,在 20 世纪 80 年代,成为了世界麻醉品交易中心之一……”莫里斯引用一位亲信的证词说,克林顿“他知道”。

克林顿不仅知道可卡因走私,而且告诉州警察 L.D.布朗——克林顿的前保镖,克林顿帮他在中情局获得一个职位,毒品走私不是中情局的行动。“哦,不,”克林顿说,“那是拉萨特的生意。”丹·拉萨特,已被判刑的可卡因经销商,是克林顿的主要政治金主之一,此人从阿肯色州的生意中赚取了数百万美元;据说他曾把 30 万美元现金装在牛皮纸袋里,送给了当时的肯塔基州州长约翰·布朗。据莫里斯说,拉萨特“并不只是另外一个需要克林顿恭敬相待的大捐赠者,他们的关系非常亲近,克林顿定期到他的经纪公司拜访,并随时会到拉萨特的豪宅里里私会。”莫里斯记述说,经常带克林顿去豪宅的拉萨特的司机,“是一个判过刑的杀人犯,身上带着枪,而且人人都知道他兼职贩毒。”总之,根据莫里斯的描述,美国总统与毒贩之间的关系,似乎比他所指称的,哥伦比亚总统埃内斯托·桑佩尔和卡利集团(贩毒集团,译注)之间的关系还要好。

\begin{tcolorbox}
哇!鲍勃讲了一些连希拉里都不知道的比尔·克林顿的事。
\begin{flushright}
—— P. J. O’Rourke
\end{flushright}
\end{tcolorbox}

埃米特·小泰勒尔(R. Emmett Tyrell, Jr.),是《美国旁观者》杂志的主编,并非莫里斯那样的左派自由主义者。但在他的《少年克林顿》一书中,也包含了莫里斯引用到的诸多细节,把克林顿描述为一个腐败的政客,与毒品交易和其他犯罪活动密切相关。事实上,《少年克林顿》的序言中,就引述了克林顿前保镖 L. D.布朗的话,布朗提出一个耸人听闻的指控,说克林顿是暗杀小组的同谋,这个组织是为了解决掉那些梅纳毒品交易的证人。

布朗更具体地指证说,1986 年 6 月 18 日,他本人被派往墨西哥的巴亚尔塔港,随身携带着一支比利时制造的 F.A.L 轻型自动步枪。布朗在旅行中化名为迈克尔·约翰逊,他的任务是杀掉特里·里德(Terry Reed)。

里德,你们可能还记得,《妥协:克林顿、布什和中情局》(Compromised: Clinton, Bush and the CIA)一书的作者之一,在 1994 年引起了公众的注意。《妥协》一书的论点是,中情局“已经拉拢了总统”,它的“黑暗的活动,就像癌症一样,已经转移到了政府机关”。更具体地说,里德与他的合著者指称,克林顿和布什都被拉下了水,参与了阿肯色州的非法活动,包括贩毒。

布朗没有按照指示杀死里德。他们设法活了下来,并至少讲出了他们的部分故事;相比当时及后来与克林顿有关的人,他们要幸运得多。想想已经去世的杰里·帕克斯,他在 1992 年为克林顿-戈尔的总部提供安保,在 1993 年 9 月的一次黑帮式暗杀中被枪杀。据伦敦的《星期日电讯报》报道,帕克斯的遗孀独家披露,是已故的文森特·福斯特(Vincent Foster)雇用了帕克斯,让他监视比尔·克林顿。

这个故事真的是一波三折,扭曲离奇。

福斯特为什么要编制一份对克林顿不利的信息档案,谁也不知道。(他自己说是为了希拉里。)但不管怎么样,这都证明了官方对福斯特的描述是虚伪的,说什么他是一个天真的乡下孩子,被华盛顿无情的政治所震惊,最后在绝望中自杀了。

随着每一点新的发现,这个荒诞的故事都变得更加荒诞。

\subsection{黑帮的总统}
虽然证据明确证实,美国总统与有组织犯罪和犯罪分子联系紧密,污点重重,但全世界都回避这个棘手的结论。莫里斯引用了一位前美国检察官的话,他曾负责追踪有组织犯罪团伙及其利益链条。他声称,克林顿在 1984 年当选州长的选举,“是黑帮分子真正开始介入阿肯色州政治的一场选举。他们是些斗狗、赛马的年轻人,看到好东西就会下手……他们超越了这里的老迪克西黑手党,相比之下,后者就是小玩闹。东岸和西岸的犯罪资金,跟合法的公司一样,都注意到了在这里发展的可能性。”显然,其他有类似想法的人,也一直在关注克林顿的可能性(指犯罪方面,译注)。

在《读者文摘》早前发表的一章文章之后,《纽约》杂志继续报道说,“总统在工会中的主要盟友,都与美国最肮脏、最黑道的一些工会关系紧密。”其中特别令人感兴趣的,是克林顿与亚瑟·科亚的关系。科亚是克林顿的“首要筹款人之一”,他是国际劳动联盟北美地区的主席,该组织是“劳工史上最张扬的腐败工会之一”。

很明显,克林顿领导下的司法部与科亚达成了某种合作,《纽约》杂志称之为“奇怪的慷慨交易”,这使得科亚“虽然面临司法部提出的证据确凿的指控,即他是有组织犯罪分子的长期同谋,但仍然能够保住工作”。

无论特里·里德关于“中情局与总统合署办公”的说法是否正确,但对于一个有权进行“黑色行动”的秘密组织内的个人来说,显然会受到极大的诱惑,放纵于赫舒拉发教授所说的理性选择,即采用“非法手段去获取资源”。

鉴于技术变革正在削弱大规模军事力量的决定性作用,接下来,人们应该会看到腐败现象的激增,如果有组织犯罪集团没有直接接管政府的话。

我们同意赫舒拉发的一个观点,即“政治经济学的制度永远不可能完美到完全取代……自然经济的基本现实”。

维托·坦吉睿智地指出,政治腐败“代表了国家的私有化,而这种私有化,并不像通常的私有化那样,把权力转移到市场上,而是转移到了政府官员和官僚系统中”。事实上,克林顿领导下的联邦调查局和其他警务部门,就是这样。“法治”正在变成克林顿和他的亲信门想要的样子。

目前,似乎还没什么证据表明,这些腐败关系的细节会对选民产生任何影响,即使它们在被大众媒体报道和讨论。相反的是,对于美国总统涉嫌参与贩毒、洗钱及其他更恶劣的罪行,好像没什么人关注。

这让人想起已故的沃尔特·李普曼(Walter Lippmann)的忧虑,即选民缺乏看透他所说的“虚构人格”(fictitious personalities)的洞察力。他认为,选民被“奉承和谄媚所误导,然后被卑鄙的虚伪所背叛。这种虚伪告诉他们,他们可以用手中的选票来决定,什么是真,什么是假;什么是对,什么是错。”李普曼察觉到,“宪政秩序的崩溃”,可能导致“西方走向急剧的灾难性衰落……在短短的时间内,我们已经严重堕落。……我们所看到的已不能称之为衰败——尽管旧的架构在大面积瓦解——而是历史性的灾难”。

问题在于,政治判断预期说是对现实世界的反应,不如说是普通公众对超出他们直接认知之外的现象,所构建的一种伪现实的反应。但如果你被别人的认知限制所支配,那就是你的错。即使你不在乎文森特·福斯特是否被谋杀,以及他的谋杀是否被美国政府所掩盖,包括警察部门的高层和相关的官员,甚至包括现任的特别检察官肯尼斯·斯塔尔,即使你对这一切都无所谓,你也可能要考虑了解一下,有组织犯罪集团和白宫之间更广泛的关系模式。

从长远来看,最高层的政治腐败,会使传统上对民主之可能性的赞美成为无稽之谈;传统认为,民主是处理公共问题的审慎之道。在信息时代,政府的规模和权势都不重要,重要的是它的诚实。政府在以往所提供的大部分服务,在下一个千年肯定会下放到私人市场。从世界各地的证据来看,长期依靠一个腐败的系统和腐败的领导人,来保障你的家庭和投资安全,其可行性是非常值得怀疑的。

正如莫里斯所言:“克林顿夫妇不仅仅是个表象,而是在更大层面上,象征着两党制度在世纪末走进了死胡同。”维托·坦吉在他关于腐败的文章中指出,“遏制腐败的唯一方式,就是大幅缩减公共干预的规模。”而信息革命正是如此,它将极大地“缩减公共干预的规模”,并在此基础之上,为道德和诚实的重生带来希望。信息革命对于道德还将产生另外一个重要的影响:当网络商务和虚拟公司都用牢不可破的加密技术进行交流,它们会变得更加脆弱。因为这使得组织内部的窃贼,或者虚拟组织,变得更加难以被发现;被偷走的钱,或者被出卖的商业机密、专利或其他有价值的经济资产,几乎不可能被追回。

犯罪是可以收到回报的,很多人觉得,通过非法的掠夺来补充合法的生产活动,是值得尝试的。与过去两个世纪的大部分时间里,西方社会普遍存在的情况不同,罪犯并不是在社会上毫无地位的边缘人。当犯罪有所回报的时候,你会看到一个更好的罪犯阶层,因为犯罪很少会遭到什么社会公愤。例如,西西里的黑手党,以及许多以高价雇用当地劳动力的毒贩,在他们的地盘上都颇受群众的支持和尊重。

\section{道德秩序及其敌人}
所有强大的社会都具有坚实的道德基础。关于经济发展史的所有研究都表明,道德和经济之间有着紧密的联系。成功的国家和群体之所以成功,部分原因在于它们的道德观,普遍鼓励自力更生、勤奋工作、注重家庭和社会责任、高储蓄、为人诚实等经济美德。社会中的亚群体尤其是这样。犹太人,特别是信教的犹太人,新英格兰的清教徒,十八、十九世纪英国经商的贵格会教徒,或者现代美国的摩门教教徒,他们的商业成功都显示出,具有强大道德支撑的文化,所能带来的经济效益。

以贵格会为例。贵格会成员在商业上非常成功,特别是作为银行家,原因有很多。

他们给自己设定了尽可能高的信任标准。他们虽然不会发誓,但认为每一项商业承诺都像誓言一样具有约束力。对他们来说,“言出必行”(my word is my bond)是一个绝对的原则。他们信奉平静、体面又节俭的生活方式。而出于宗教责任,他们不会把钱用于虚荣的世俗消费上。他们避免参与争吵,认为战争永远是罪恶的。作为商人,他们觉得负有提供公平价值的道德义务,通过以适当的价格保持高质量的服务,他们建立了良好的声誉。“Caveat emptor”即“买家自己小心”,对他们来说是不够好的。在大多数商人遵循高价格、高利润的贸易理论的时代,贵格会成员的道德观,使他们自然而然地采取了低利润、高周转率的策略。就像亨利·福特后来所展现出来,这种策略可能更加赚钱。他们之所以采用这样的经商原则,因为他们觉得不能欺骗客户,是事实证明这也是扩大业务的最佳途径。

经实践检验,贵格会的人是值得信赖的生意伙伴,所以他们的回头客越来越多,双方实现了共赢。贵格会以履行义务为荣,他们是一个储蓄率很高的社群,所以,经营银行业有天然的优势;而贵格会的成员资格本身就是一种商业资产,更激发了贵格教徒的信心。

不幸的是,这种商业优势可能会被其成功所反噬。国家有一个发展的周期,这构成了 18 世纪亚当·弗格森(Ferguson)社会学的理论基础,即从贫穷开始勤奋工作,然后到富裕,到奢侈,再到颓废,最终衰落。古罗马人自己也回顾过共和国时期的美德,当时帝国正在建立,并为后来人们的奢侈和懒惰痛恨不已,认为这是导致帝国衰落的主要原因。繁荣富裕对勤劳节俭的美德的侵蚀,速度可能是惊人的。现在的德国人,仍然是一个能力和效率都很出众的民族,但他们已经不像1945 年在战败的废墟上重建国家时那么努力了。在两代人的时间里,德国人从一穷二白中长时间地工作并白手起家,变成了只需工作很短的时间,就可以获得地球上最高的工资和最好的福利。

1995 年 10 月,16 个德国雇主协会签署了《彼得堡宣言》。宣言中充满了抱怨,但是理由充分;它反映了德国工业士气的下降:德国的税收负担在 1995 年达到了历史最高水平,特别是由于团结附加费和护理报销费。德国的公司税总额超过了 60\%以上,远远高于 35\%至 40\%的国际同类水平。公共部门的惯例,如晋升的监管、终身制的工作和高额的养老金,必须被自由市场的择优晋升和补偿规则所取代。德国是世界上劳动力成本最高的国家,工资政策必须考虑减轻企业的成本,以此来降低失业率……工资增长应该根据竞争力和生产力来衡量……。工会的行为必须改变。每年一次的运动、呼吁、工人动员、威胁和警告式罢工,是很有害的。 德国人,特别是年轻人和富裕的继承者,已经丧失了勤劳工作的优良传统,这引起了德国社会包括科尔总理的普遍焦虑。

大众汽车公司现在的劳动合同,给每个汽车工人的工资,都是世界上最高的,还要加上福利税,换来的是每周 28 个小时的工作,即四天,每天 7 个小时。战后的德国现在已经是一个大规模的就业输出国。在 19 世纪中叶,英国被认为是世界上效率最高的工业国家;100 年之后,它们的这个声誉肯定已经易主了。而繁荣的循环总是会破坏辛勤工作和适度期待的美德,这些美德往往存在于工业成功发展的早期阶段。国家无法延续其早期的社会美德,就像个人很容易因为成功而变得懒惰又贪婪。

毋庸置疑,全球性投资将会奖励勤劳的美德,并惩罚那些贪婪和懒惰的人。这是理所应当的。事实上,你可以说,稳健的投资必须建立在道德以及纯粹的财务评估之上。18 世纪的英国人认购贵格会银行的资本,可能获得了很好的收益。在19 世纪,贵格会教徒投资于巧克力企业,因为他们觉得可可比酒精更健康;这大概也是对的。但对弗莱公司(Fry’s)或吉百利公司(Cadbury’s )的投资,肯定是很好的投资。投资者应该避开颓废期。即使德国仍在欧洲市场上保持着强势地位,而且工业技能很高,但高昂的劳动力成本和极短的工作时间,已经降低了德国未来的潜力。

社会道德与经济成功密不可分。但是,什么因素有助于维持或倾向于破坏社会道德呢?20 世纪上半叶伟大的历史哲学家阿诺德·汤因比,提出了挑战与回应的理论。社会因挑战而焕发出活力,并能发展出人们甚至不知道自己拥有的美德。

一直以来,人类都知道,困难时期会比繁荣时期激发出更健康的反应,确实如此。

在我们个人的生活中,每个人都努力想让自己生活得舒服一些,希望住在漂亮的房子里,有一份自己喜欢的工作,银行里有足够的钱,等等。为实现这些目标而进行奋斗是有意义的。我们上学,参加培训,在生意场和职场上打拼,都是为了梦想成真。

而有很多人,在功成名就之后,却陷入到了困境之中。也许是,苦斗好过成就。

本世纪初,伟大的瑞士心理学家卡尔·荣格收治了一位美国商人。这个商人在年轻的时候就充满雄心壮志。他努力工作建立了自己的企业,并在 40 岁之前赚到了足够的钱,可以退休。他娶了一个年轻美丽的女人,买了一所漂亮的房子,组建了新的家庭。他的生意非常成功,在 40 岁,他确实卖掉企业退休了,成为了一个富有而独立的人,无忧无虑。一开始,他确实享受自由的生活,终于能够做他想做的事情。他带着家人去欧洲旅行,参观美术馆等等。但渐渐地,这些兴趣和自由感,都开始变得苍白。他开始回想那段不自由的日子,那时候他全身心地扑在工作上,有很多生意上的烦恼,但那时候他很快乐。这一切让他陷入了抑郁症,最后由他的妻子带着他来找荣格看病。荣格做了诊断,认为其实是他的创造力没有了出口,那些能量转向了他自身,正在摧毁他。这个诊断可能是对的,但没能治愈他。这位商人始终未能从他的精神崩溃中恢复。

对我们人类来说,重要的是奋斗而不是成就,我们是为行动而生的,而成就可能会让人大失所望。雄心壮志总是会催人奋进,不管是为了什么目的,而奋斗的过程也往往比结果更令人享受,即使最终得偿所愿。当然,对大部分人来说,目标都无法完全实现。我们没有挣到那么多钱,没能住上梦想中的房子,我们不得不满足有缺憾的生活。

有一种社会意识,在 19 世纪通过不同的方式得到了强化,它认为美德是动态的,美德在于努力而不在于结果。阿瑟·休·克拉夫(Arthur Hugh Clough)有一首著名的诗歌,在第二次世界大战的生死斗争中,给许多人带来了安慰。值得注意的是,交战国的自杀率在二战中均有所下降;也许,战争的挣扎也比无所事事的压抑要好。

\begin{tcolorbox}
\begin{center}
不要说奋斗终是徒劳,辛劳和创伤白费无功,\\
敌人尚未力竭,更未被打倒,事物一切照旧,依然毫无变动。\\

倘若希望是虚妄,恐慌则是骗徒,你可看见远处的硝烟弥漫?\\
即便在此刻,战友们也在将逃敌驱逐,\\
若非你拖后腿,早已将敌军阵地攻占。\\

疲惫的波浪徒然拍打海滩,\\
仿佛再努力也无法向前一寸,\\
可是就在附近的河口与港湾,\\
大海的满潮已经无声涌进。\\

每当白昼来到世间,光芒不仅仅射进东窗,\\
在正面,太阳攀援多么迟缓,\\
但向西看吧,遍地洒满霞光!
\end{center}
\end{tcolorbox}


这种积极的竞争,依然触动着现代人的感官。而很多人确实就过着这样的生活方式,在不断的奋斗中,抓住恶劣环境中的潜在机会。我们生活在一个充满竞争的世界里,而且大多数并不想从中退缩。当然,也有人属于沉思性的精神气质,那是非常少见的。

美国最伟大的哲学家威廉·詹姆士(William James),1891 年在耶鲁大学的哲学俱乐部的讲话中,对上述的动态道德观,提出了一种类似的 19 世纪的观点:在人的道德生活中,最深层的差别其实是轻松和紧张的状态差别。当处于轻松的状态时,对眼前病痛的畏惧就会统治着我们的思虑。相反,紧张的状态会使我们对当前的折磨无动于衷,一心想着实现更大的理想。紧张情绪的潜能在每个人身上沉睡,但有些人更加难以唤醒。它需要更加狂野的激情予以刺激,需要巨大的恐惧、爱和愤慨;或者一些更高程度的忠诚,如正义、真理和自由的深刻呐喊。

强烈的解脱是打开视野的必需,如果所有的山峰都被推倒,所有的山谷都被填平,这样的世界并非理想的居住之地。这也是为什么那种激烈的情绪,在一个孤独的哲学家身上可能会永远沉睡不醒。因为他的各种理想,仅为自己所知,那只能算是他的偏好,而且其价值限于同一个维度;他可以随意地在头脑中摆弄,或快或慢。

同样,这也是为什么,在一个没有上帝只剩下人类的世界里,我们对道德的呼唤,远没有激发出它最大的能量。当然,即使在这样一个世界里,生活也是一首真正的道德交响曲;但它只在几个可怜的八度空间内演奏着,无限的价值尺度未能被打开。 威廉·詹姆士认为,动态的道德可以延伸到宗教领域,它的重点在于做而不是存在,在于行动而非畏缩。关于竞争与生存的道德,在亚当·斯密(1776 年)、托马斯·马尔萨斯(1798 年)和查尔斯·达尔文(1859 年)的作品中,都做了强有力的拓展。这是主导着当今世界经济秩序的道德律,对它的中心思想需要认真地审视。

达尔文主义的主导思想是:物竞天择、适者生存,自然选择的过程塑造了物种的特性。在动物中,这个过程是随机突变的结果;现在的人知道这属于基因遗传,而达尔文本人则只能猜测。但是,人类社会的生存取决于以人类智慧为基础的文化选择。文化改变人类社会,就像基因改变其他物种一样。所以,我们的社会可能会变化得很快;它不用像随机的基因突变那样,要经过很多代的演化。人类发展出了文化选择,取代了动物界的自然选择;在人类历史中的某些阶段,有些文化发展出了新的技术,使它们在创造财富和集结力量方面享有了决定性的优势。

新技术的文化优势,往往是决定性的,如铁器时代的人胜过青铜时代的人,电子时代的人胜过机械时代的人。亚当·斯密也许不是第一个把国家福利归结于个人行动的经济学家,但他说得最简洁、最令人信服:每个人都在不断地努力,为他所掌握的各种资本找到最有利的用途。的确,他考虑只是自己的利益,而不是社会的利益;但他对自身利益的计算,自然而然地,或者说必然地,会促使他选择对社会整体最有利的工作。 托马斯·马尔萨斯,是人口学研究的奠基人,他看到亚当·斯密的观点不仅适用于国家经济的发展,而且可以应用到人类人口的生存问题。马尔萨斯以提出以下主张而闻名,他认为“人口在不受控制的情况下,会以几何比率增长,而生存的资源则按算术比率增长。稍微对数字有所了解的人,都会发现,第一种增长的力量相比第二种是多么的惊人。”马尔萨斯甚至早于达尔文看到,同样的原理也适用于整个自然界:在动物和植物王国中,大自然极其慷慨大方地到处撒播生命的种子;但在给予培育种子所必需的营养和空间方面,它却一直比较吝啬。我们这个地球上的生命种子,如果得到充足的营养和空间,经过几千年的繁殖,就会挤满几百个地球。然而,必然性,这一强硬又无处不在的自然法则,把它们限制在了规定的范围内。 到 18 世纪末,即使是在亚当·斯密和马尔萨斯的时代,人们已经理解到,世界是以动态的方式在发展和演变——实际上它一直都是如此。人类作为众多物种中的一类,由于无限的生殖能力与有限的食物种植能力不相匹配,而不得不进行竞争。人类社会的生存,和动物的生存一样,取决于对环境的适应。因此,动态的道德观,重点关注的是如何克服适应的问题。这种道德的最好体现,是那些能够调整自身行为去抓住机会,并利用社会中的现有资源获得最大竞争优势的个体。

马尔萨斯还意识到,亚当·斯密的思想已经改变了世界。他写道,他关于人口的论点并不是全新的:“它所基于的原则,部分由大卫·休谟解释过,部分由亚当·斯密博士解释过。”他还看到,持续不断的生存竞争是一个道德问题,而不仅仅是实践问题。在他 1798 年的“论文”中,最后一段是这样写的:世界上存在着邪恶,不是为了使人悲观绝望,而是为了刺激活动。我们不应当忍耐和屈服于邪恶,而应尽力避免作恶。竭尽全力消除自己身上以及能力所及范围内的邪恶,不仅是每一个人的利益所在,也是每个人的义务。每个人愈是尽力地履行这个义务,他努力的方向就越正确,收获的成果也越丰厚;他就越有可能改善和提高自己的精神,从而更全面地实现造物主的意志。 在达尔文划时代的著作《物种起源》(1859 年首次出版)中,从第三章的内容概要可以看出,他对上述观念之重要性的认识。他把这重要的一章命名为“生存斗争”,副标题为:“自然选择下的熊——生存斗争的广义用法——按几何比率的增长——归化的动物和植物的迅速增长——抑制增长的性质——斗争的普遍性——气候的影响——个体数量的保护——一切动植物在自然界中的复杂关系——同物种的个体间和变体间的斗争最为剧烈,同属的物种间的斗争也很剧烈——有机体与有机体间的关系是一切关系中最重要的”自 1776 年(亚当·斯密)以来,显然人们可以认识到,优化国家财富的最佳途径,是允许个人在自由竞争条件下最大化自己的资本回报。自 1798 年(马尔萨斯)以来,显然人们可以认识到,人口的相对生存,取决于整个社会是否在政治和经济上足够成功,能够养活人口,并保护其不会因传染病或战争而大量减少。

自 1859 年(达尔文)以来,显然人们可以认识到,在人类、动物和植物的世界里,整个生命的戏剧就是一场持续不断的生存斗争,其中最大的对手,是彼此最接近的物种或文化。这种斗争需要一种动态的道德观,它不仅是在邪恶来临被动地做出反应,而是要积极主动地防御邪恶。

这些思想的力量是如此强大,自从它们被提出之后,就没有人不会为其所震动,进而去思考人类的本质或道德的问题。卡尔·马克思和查尔斯·达尔文一样相信生存斗争,但他认为这是一场不同社会阶级之间的斗争,而阶级本身是由经济力量所形成的。阿道夫·希特勒也相信为生存而奋斗,这个理念可以说贯穿他政治生涯的始终;但他认为,这场斗争是不同种族之间的斗争。马克思、列宁、斯大林、毛泽东和希特勒,都可以称为社会达尔文主义者,因为他们把生存斗争,即希特勒所说的“我的奋斗”,看作是政治的核心问题。马克思主义者把社会阶级当成独立的物种,纳粹党人则以同样的眼光看待种族。

然而,这并不是马尔萨斯所设想的动态道德,而是动态不道德。马克思主义和纳粹主义希望解决同一个问题,即生存斗争的问题,但他们的方式是通过破坏竞争。

他们侵略外国;他们不断制造冲突,在争夺社会权力的不同阶级之间,或在互相视对方为剥削者的不同种族之间(常见的就是反犹者对犹太人的指控),或针对被认为很危险的下层阶级(白人对黑人的恐惧)。第二次世界大战,是阿道夫·希特勒试图摧毁潜在的竞争者,特别是斯拉夫人和犹太人,来确保德国人的生存优势,但以失败而告终。一个有意思的悖论是,历史证明,战争的失败比纳粹的胜利,对德国人更加有利。

破坏性的“干涉”竞争,可以被合作竞争所替代。合作竞争是亚当·斯密的核心思想,也是马尔萨斯和威廉·詹姆斯的核心思想。破坏性竞争的典型是征服者。

他摧毁竞争对手,夺取他们的资产,可能还会接管他们的国家,甚至奴役他们的人民。而合作竞争的典型是商人。让顾客对交易感到满意,是商人的利益所在;因为只有这样,顾客才会回来做更多的交易。顾客的繁荣富裕,也符合商人的利益;因为只有这样的顾客,才有钱继续进行交易。征服意味着摧毁对方,商业意味着满足对方。在现代的技术环境中,征服已成为异常危险的政治策略,所以,商业是解决生存问题的唯一合理的途径。

亚当·斯密的另外一个核心观点——对他来说并不新鲜,即职能的专业化,进一步强化了商业中相互依存的关系。在《国富论》的开篇,就有一段著名的论述,斯密指出,“劳动生产力的最大提高,以及在运用劳动时所体现出来的更高的熟练程度、技巧和判断力,似乎都是分工的结果。”他举例说,“以这种方式,制造大头针的重要业务,可以被分为 18 道不同的工序。在有些工厂,这 18 种不同的操作由 18 个不同的工人来负责。”职能的专业化程度越高,制造业的效率就越高;但很显然,这种经济是高度相互依存的。它要获得成功,就必须是合作性的。

因此,一种成功的社会道德应该具备某些特性。它必须是强大的——软弱的道德无力发挥作用。它必须有助于生存斗争,但应该是合作性而非破坏性的;希特勒具有强烈的生存道德,但它的破坏性摧毁了整个社会。它必须是动态的,能够匹配现代技术乃至整个社会制度的动态变化。它必须有利于经济发展;列宁主义体系中的平均主义与专制主义的大杂烩,根本就行不通。当然,这些并不是一种社会道德应有的全部特征。它还应有一个更宏大的目标,就是使整个社会成为一个良好的生活环境,并把人们联系在一起。此外,道德本身应该能够适应变化并生存下去,一种脆弱的道德可能在一代人中间被接受,到下一代就被抛弃掉。一种传统的道德又可能会过于僵化,无法适应社会结构的不断变动。而另一方面,纯粹相对主义的道德体系则根本算不上道德,因为它无法明确指导人们该如何行事。

我们可以先把所有的社会道德放到一个背景中去考察。一个强大的共同体,即使是虚拟的共同体,都有赖于其道德被广泛接受。人类社会中最成功的历史时期,往往是集体道德被充分认同的时期。这样的道德不仅能发挥具体的功能,如减少犯罪、稳定家庭与社会;它还能赋予公民以目标感和方向感。从历史来看,这样的道德共识,似乎取决于有一种占主导地位的宗教。无论是罗马帝国早期的国教;还是犹太教,它一直是四散分离的犹太人的生存线索;或者是包含社会规则的伊斯兰教;以及中世纪的天主教,新英格兰地区早期的新教等等。一个民族,一种道德,一种宗教,这三种观念相互依存,并相互加强。

在这样的道德社会中,公民个人可以在社会支持的架构内,努力去实现人生目标。

诚然,道德率可能有些武断,或至少在外人看来是武断的。正统的犹太人失去了吃猪肉或贝壳的自由,并且不能在安息日工作。虔诚的天主教徒不能使用人工避孕药,更不要说堕胎了。穆斯林不能饮酒。儒家的信徒要为父亲服丧很长一段时间,这很不方便——甚至孔子也觉得,服丧仪式被夸大了。然而,这些信仰体系中的每个人都认为,遵循这些守则,只是为一个共享的、一致的社会秩序所付出的小小代价;有了这种秩序,每个人都有一个安定的位置。一个犹太人完全可以说,遵守安息日只是有一点点不便,但对律法和家庭都有好处。一个宽容的社会,拥有共同的道德,是约翰·洛克和早期自由主义哲学家所追求的理想。他们根本不相信,任何一个社会,可以在没有规则的情况下正常运转;但他们认为,规则应该服从于最好的理性,而且人们只能被强迫接受基本的规则。他们的确承认,强制执行社会道德是不可避免的,特别是在保护生命或财产安全方面;因为在他们看来,没有不能保证安全,任何社会都无法生存。而对于不影响他人福祉的个人选择,他们持几乎绝对宽容的态度。儒家人要为父亲服丧 40 天,住在隔壁的可能是犹太人,他要遵守安息日;但是他们互不打扰,也不会强迫对方遵守自己的宗教习俗。

在基本的事务上遵循社会道德,而在个人决定上则给予宽容,通过这种把二者结合在一起的理论学说,人们实际上得到了,一套必须强加给所有公民的核心道德标准,以及公民作为个人或社会亚群体成员所资源接受的道德标准。一个本笃会的修士宣誓要坚守贫穷、贞洁与服从时,他是作为一个亚群体的成员,对自我的要求。他不是要求所有的天主教徒,更不是所有的同胞,都要发同样的誓言,或遵守同样的规则。他会服从修道院院长的命令,但根本不期望外人有任何的关注。

所以,对于社会道德中可以自行选择的部分,不需要做普遍的要求;而核心的道德的确应该是共享的。不接受核心道德的人会损害社会,也损害到自己。在极端的情况下,一个充斥着激情杀人的强盗社会,就像罗马帝国灭亡后的欧洲大部分地区一样,谁也无法过上安稳的生活,强盗自己也不行,他们更容易受到其他杀人犯的威胁。在今天的美国,一些城市的中心就是这样。无政府状态绝不是理想的社会;没有法律维护秩序,就没有安全。

当我们审视那些敌视社会道德的力量时,需要意识到,人类社会的核心道德,在大多数现代宗教体系中是大概一致的。在基督教的《旧约》或犹太教的《摩西五经》中,十诫中的两条,可以说在任何宗教中都是被普遍认同的。即“不可杀人”和“不可偷窃”。甚至不止于此。几乎所有严肃的不可知论者,也都会把谋杀和偷窃视为禁忌,并认为社会有权惩罚做出这两种行为的人,因为这是对生命和财产的终极威胁。他们可能会对特定犯罪行为的惩罚尺度是否适当产生分歧,但不会争论社会是否有权进行惩罚。

约翰·洛克的原话就是这个意思。每个人都享有“生命、自由和财产”的权利。

1776 年,托马斯·杰斐逊又增加了一句,即“追求幸福的权利”。这一句非常精彩,也是非常良好的愿望。但相比之下,“生命、自由和财产”更接地气。社会存在绝对依赖于生命权和财产权。而历史表明,只有在拥有自由的情况下,这些权利才能得到保护。如果国家无所不能,那么它就会像在侵略战争中一样,成为生命强大的敌人;它也会对个人财产造成巨大的威胁,因为它会从社会财富中攫取过分的比例,用于自己无谓的挥霍浪费。

然而,在那些最先进的国家,攻击核心道德的部分力量,恰恰来自于赋予了这些国家技术优势的现代化力量。美国是世界领先的技术强国。在 20 世纪 60 年代初以前,许多人,包括大部分美国人,都把美国视为其他国家的道德榜样。现在,很少有人再表达这种观点了,即使是那些为祖国感到骄傲的美国人。像全世界人一样,一个人不可能在听了 O.J.辛普森的审判之后,还把美国看成是最初那个单纯的美德共和国。

如果回顾一下旧日美国身上的标签,它们反映的其实是一个边境社会的发展需求;即使是在大城市,它也受公民态度的熏染。边疆是民主发生的地方;在那里,人们觉得自己是平等。早期的美国人还抛弃了欧洲的等级制度。即使是作为囚犯从英国运来的契约劳工,等到契约到期,他们就成为了独立的商人、农民或自由劳工,在社会上确立了自己的地位。工资比欧洲高,生活必需品的成本很低,尽管进口的制成品比较贵。在边疆,人们彼此之间互相依赖,虽然生活艰苦,但以欧洲的标准看,还是很不错的。新移民可能从波士顿和纽约的贫民窟起步,从低工资的工作干起,一般很快就走出去了,一代又一代的人,获得了繁荣和富裕的生活。南北战争后,黑人开始把自己看做是另一个移民群体,他们中很多人都认同美国的价值观和生活目标。由此,黑人中间阶级也发展起来了。

这种志向抱负,得益于边疆生活的实际经验,以及新教和天主教教会的影响,而不断加强,形成了美国人的爱国主义。他们相信自己生活在上帝的国家,这是一种在民主理想和基督教信仰导引下,所形成的独一无二的观念,美国也成为了世界上第一个也是最成功的一个民主国家。这幅熟悉的图景体现在我们所有人,或几乎所有人对亚伯拉罕·林肯的印象之中;尽管在南部还有一些美国人,把林肯看成是为了阻止自由州脱离他们不再信任的联邦,而发动了第一场现代战争的恐怖人物。

即便如此,林肯留给世人的形象,嶙峋、朴实、诚恳、雄辩,仍然是最伟岸的美国人的形象,而且基本上是一个道德的标杆。现在还有很多美国人能够感受到,新国度的民主活力与欧洲疲惫的等级制度间强烈的对比。不过,在今天的洛杉矶、纽约、伦敦或华盛顿,外国人很难认识到这种充满能量的贤能社会的理想;而在广大的郊区或农村地区,还能找到它的痕迹,以及更多的存在。美国清教徒的道德伦理,及其所有重要的历史,在雪线以北地区保存得最好,而创业精神的推动力则传播到了更广泛的地方。

美国人看到,大城市的衰败,已经成为犯罪,特别是毒品交易的温床,是公共道德下降最严重的表现。大多数美国人也认识到,在几种不同的道德文化之间存在冲突,它们都在争着扩大自己的主张和权威。政治正确的文化,排挤掉许多支撑旧文化的道德原则,虽然不是全部。政治正确咄咄逼人,强调在历史上占据主导地位的白人男性文化,认为其剥削了其他群体的价值和权利;同时,它也拒绝接受这种文化,虽然这是美国的创始文化。

20 世纪上半叶由男性所主导的文化,是以核心家庭的生存为中心。这从历史发展上,给予了丈夫-父亲在家庭中至少是名义上的支配地位,尽管实际上家庭往往是由妻子-母亲在管理,名义上的主人通常是温顺地接受。这也使男老板在职场上获得了主导地位,这种地位如今受到女权运动的挑战,但还没有被逆转。家庭的利益,以及基督教的历史教诲,使堕胎被非法化。过去的道德认为,堕胎是非法杀人,是绝对禁止的。传统道德的坚持者仍然持这样的观念;而新道德的支持者则持相反的观点。以前,堕胎问题通常由各州自行处理,而在“罗伊诉韦德”(Roe v. Wade)一案中,最高法院将堕胎的宪法性权利,置于在隐私权的基础之上,而隐私权本身与宪法或修正案中的任何条款都相去甚远。

法律上认为,妇女的隐私权包括要孩子或不要孩子的权利,不管对胎儿可能造成什么后果。最高法院不认为胎儿享有任何宪法权力,20 世纪后期的胎儿与 19 世纪上半叶的奴隶一样,是宪法外的实体。“生命、自由及追求幸福的权利”并不适用于奴隶,罗伊诉韦德案的大法官,也没有把《独立宣言》的规则适用于胎儿。

关于堕胎的争论,是新旧道德冲突的极端例子;不过在其他领域,旧的社会组织及道德也受到新道德的挑战,同样存在显著的冲突。在新教和天主教中,传统的基督教道德都非常强调性的角色。不允许进行婚前或婚外的性行为,男性之间不能发生生殖器关系。女同性恋没有被特别强调,因为社会基本不承认它的存在。

当维多利亚女王第一次被告知此事时,她坚决拒绝相信女人之间会发生这样的事。政治正确是所谓的受压迫群体的道德。同性恋者声称他们的生活方式与其他人是平等的,并挑战传统反对他们发生性行为。“恐同症”被认为是一种令人发指的歧视,和种族歧视一样。新的道德观认为,批评同性恋与批评黑人、犹太人或妇女一样,是不可接受的。

与此同时,其他诸多的性禁忌都遭到了侵蚀,或干脆被废除了。在 1960 年代,出现了自由恋爱的浪潮,部分原因是女性避孕药明显安全了很多,但也受到了情绪药物和流行音乐的推动。这导致了越来越多的非婚同居。到 20 世纪 90 年代,在比美国普遍保守的英国,人们认为爱德华王子和他女朋友在白金汉宫睡觉是很正常的,就像 20 世纪 60 年代的学生在宿舍一起睡觉一样,是一种虽未婚但稳定的亲密关系。女王伊丽莎白二世作为英国教会的领袖,纵容自己最小的儿子胡作非为,但很少人觉得这有什么奇怪的,而她三个更大的孩子的婚姻都已经破裂。

少数抱怨的人,还被认为是无可救药的落伍者和假正经。不过,仍有许多人认为旧的道德观念是可取的,尽管他们自己不遵循,也不真切期望他们的孩子从很小的年龄就遵循。

政治正确运动也有它清教徒的一面。因为它是从妇女的利益出发,而妇女被认为是最大的受压迫群体,所以它对男性的性行为有一定的敌意,包括攻击性的和过去被认为是无害的形式。一些妇女认为,所有的男人天生就是强奸犯;对强奸的自然恐惧就被夸大到对男性的普遍谴责。另一些人则集中在性骚扰问题上,这种抱怨是真实的——很多男人的性举止特别粗俗;但对一些很微不足道的行为的抱怨,就显得有点可笑了。仅仅是眼神,没有说任何话,更不要说身体接触,都可能被指控为性骚扰。所以,新的道德观可能审查性非常强。白人可能被指控有种族歧视,不是因为他们真的有歧视,而是因为他们是白人。男人可能被指控为性骚扰,因为他们的表情显示,他们觉得一个女人很有吸引力;而在上一代人看来,这是一种恭维而非侮辱。

政治正确派和基督教原教旨主义者,互相猛烈地攻击,然而在现代世界里,他们其实颇为相似。尽管他们的道德教义不同,但他们都认为自己的教义具有权威性,是普适的。事实上,他们存在同样的缺陷,都是一种夸大的、自大的道德主义,缺乏深度,缺乏历史感,也缺乏宽容。两者都被职责为类似 17 世纪的请教主义;或类似于英格兰的奥利弗·克伦威尔(Oliver Cromwell)——他差点移民到新英格兰,一个自大的到的家;或类似于塞勒姆(Salem)的猎巫者。无论是教条化的妇女运动,还是圣经地带(美国中西部及南部有强烈基督教信仰的地区,译注)保守的传教士,都不能说他们缺乏道德,只能说他们做得有些过头和僵化。这些道德的核心好像已经成为了一块石头。这种道德的动脉硬化对社会道德共识的破坏,并不亚于它们所反对的“怎样都行”的无政府主义。

这是对道德力量的扭曲,是粗暴的自以为是。法利赛主义,和人类的历史一样古老,认为自己的美德独一无二,对耶稣基督尤其反感。而一种新兴的对道德的腐蚀,则认为道德选择纯粹是个人喜好,和选衣服一样,是个人的私事。这种信念认为根本不存在任何共同的道德。它把古典的自由学说带到了一个新的阶段,把“追求幸福”从约翰·洛克的本意以及 1776 年杰斐逊对它的诠释,变成了一种不计后果的享乐主义。

“追求幸福”一语出自约翰·洛克的《人类理解论》(1691 年出版)。其中写道:“对智性的最高完善,就在于认真地追求真正的、踏实的幸福;所以,关照我们自身,不要把想象的幸福误认为真正的幸福,是自由的必要基础。”他接着说:“每个人所认为的幸福并不相同……心灵和味觉一样,有不同的喜好……。

人们的选择可能会各有不同,但都是对的;假设他们像一群可怜的昆虫,其中有些可能是蜜蜂,喜欢鲜花与它的香甜,有些可能是甲虫,则喜欢其他的食物。”此外,他还论述了,宁要恶习不要美德,“显然是错误的判断”。他特别看重宗教观点,但也认为“这里的恶人要更坏”。他认为,“道德如果是建立在真正的基础之上,就必然会决定所有愿意遵循它的人们的选择。”相比那些强求一视同仁、行为统一的专制道德体系,毫无疑问,洛克的自由学说给人们的选择偏好提供了更大的空间。不过,很快,经典自由学说就认识到了集体道德要求的必要性,包括尊重他人,特别是他们的生命,及依法和平地享有的财产所有权。集体道德的普遍腐蚀威胁着自由,这既有直接的一面,如它导入了无政府主义的因素;也有间接的一面,如它鼓动了社会中最独裁的力量。我们可以把公共道德的历史,看成是无序与独裁之间的循环;现代的专制性道德,包括女权主义和原教旨主义,都是对 60 年代享乐主义的周期性反应。

对于下个世纪的新世界,我们已经描述了它的一些属性。它将主要由两种力量所塑造,一是技术的变革,它正在开放亚洲的经济;二是新的全球电子通讯系统,它会使人们对当地政府的依赖逐渐降低。新技术将取代或已经取代了众多中等技能的劳动者——如流水线工人、办公室文员,以及越来越多的中层管理者。但是,稀缺的技能将得到特别的回报,一个由高技能人才组成的国际认知精英群体将会出现,新的通信技术会为他们的技能打开最广阔的市场。和大部分精英一样,认知精英也会有点高高在上,相当傲慢,认为可以制定自己的标准。所以,他们会与社会比较疏远。

在下个世纪的上半叶,大量财富会从旧的西方转移到新的东方。政治上的失败——中国依然是一个政治落后的国家——可能会推迟这种财富和战略力量的转移,但决不可能阻止。这一趋势无可逆转。

这一财富大转移,必然会给北半球白人主导的国家带来最大的压力,即欧洲和北美地区的先进国家,目前大约有 7.5 亿人口。直到最近,日本才成为唯一一个达到欧美生活水平的亚洲非白人国家,尽管在新西兰、澳大利亚和非洲南部的白人中也有欧洲人种。即使在 1990 年,先进工业国家的总人口也只占世界 50 亿人口的 15\%左右。世界财富的分配状态是 15\%的富人,85\%的穷人,和 100 年前先进工业社会的收入分布非常像。经过一段加速的进程,到 2050 年,世界的总人口可能达到 70 亿,先进经济体的人口预计会有 30 亿;或财富的分配比例是 40\%的富人,60\%的穷人。到下个世纪末,这个数字可能会翻转,即达到 60\%的富人和 40\%的穷人,而贫穷人口主要集中在非洲地区。国与国之间会朝着财富更平等的方向发展,但在国家内部,很可能会更加地不平等。能够高效利用人才和资本的人,相比那些只具有中等技能或较少资本者,将享有决定性的优势。未来的财富高度流动。先进国家的穷人,不可能再像 20 世纪那样吃大户,依靠政府大规模地向富人征税;这么做的国家会在激烈的竞争中倒下。

当然,如果不发生世界大战,世界经济的总生产力会继续提高,也许每年平均提高 3\%。如果是这样的话,那么,全球经济总产值将每 25 年翻一番;到 2050 年,总产值将是现在的 4 倍以上,到 2100 年,将达到现在的 16-20 倍。即使 2100年的世界人口增加到 80 亿,也会使那时的人均 GDP 达到现在的 10 倍。这样的财富增长,除了照顾新兴工业社会的崛起,以及认知精英每年的数百万美元收入,仍然可以为其他先进的劳动力,提供一个体面的并不断提高的生活标准。但未来的情况与 20 世纪会有很大的不同。从全球范围来看,穷国的收入会比富国的增长更快;从国家范围来看,富人的收入增长,将比中低收入者要快得多;就像20 世纪 90 年代的美国。在下个世纪,我们将见证一个世界超级阶级的诞生,也许由 5 亿超级富豪组成,其中 1 亿人的富有程度足以成为主权个人。

这个过程会导致一个不可避免的结果:社会的同质性将大大降低;民族国家将被削弱,甚至完全崩溃;认知精英将更多地认为自己是世界公民。在全球范围内从事同样职能工作的人,已经在发展一种文化;在这种文化中,相比在旧式民族国家中的同胞,他们与世界各地的同行之间更加亲近。一个伦敦的投资银行家去到首尔,可能比在格拉斯哥更有家的感觉;一个华盛顿的公务员在波恩,可能比在华盛顿当地的黑人区更自在。我们应该可以看到,这个过程对道德价值观产生的分裂效应。个人的道德部分是由教育所决定的,主要来自于他的童年;部分是由生活经验所塑造的。认知精英的教育和经验都是世界性的,这往往会使他们与当地的社区脱离开来。

当我们迈向下一个世纪,在不断壮大的认知精英群体中,会有很大一部分人几乎没从家庭中接受过宗教或道德教育。精英中间最常见的宗教是一种不可知论的人文主义。很多这样的家庭会因为离婚、再婚及第三次婚姻而严重分裂。好莱坞的婚姻模式虽然在美国不具有普遍性,但欧美认知精英的离婚率都很高,平均可能达到三分之一或更高。父母离异的孩子很少接受基本的宗教教育,他们也会意识到父母、继父母、继兄弟姐妹见的道德观念各有不同。如果把这个群体的基础道德教育和爱尔兰或波兰村庄里的相比较,显然农民的教育提供了更强的宗教训练。一个无神、无根、有钱的精英是不可能幸福的,也不会得到爱。

下个世纪在经济上占据主导地位的这群人,他们基础道德教育的不足,很可能将被他们的生活经验所强化。这些人会接受某种高等技术教育的训练,以适应他们在未来电子世界中作为领导者的新角色。但是,对于历史上作为人类社会行为框架的道德课程,他们从中学习的不过。按照孔子、佛陀或柏拉图(公元前 500年)、圣保罗(公元 50 年)或穆罕默德(公元 600 年)的标准,他们属于道德文盲。他们被灌输的是经济效率,资源利用、金钱追求等理念,没有谦逊或自我牺牲的美德,贞洁就更不要提了。基本上,他们中的大部分是作为异教徒长大的,其价值观更接近于罗马帝国后期的价值观,而不是基督教的。而且,这些价值观还是高度个人化的,不是共通的。我们前面论述过,只有当真正的道德价值被广泛认同时,一个社会才能强大起来。所以,先进国家已经步入了这样的境地:很多人持有着微弱或有限的道德价值观,而其他人则以强烈的非理性价值观作为补充,整个社会共同持有的价值观念非常稀薄。不过,毫无疑问的是,我们前面所说的“竞争性领土俱乐部”,会对其居民实行严格的道德标准。

贫富之间的财富差距,本身在历史上并没有导致宗教价值观的根本差异。在一个传统强大、稳定且坚实的社会中,陡峭的等级结构——“富人在自己的城堡里,穷人在他的门口”,可能会掩盖贯穿其等级制度的价值观,但这取决于富人和穷人之间集体感情的强度,以及社会传统的强度。而这两者现在都不复存在了,而且社区的感觉和传统都被正在发生的经济和技术革命给削弱了。大众和少数人的生活正变得原来越疏远。技术革命的实现,往往意味着打破旧的方式方法。在所有的领域,都是激进的人获得胜利;而以传统方式思考的人则拉在了后面,实际上他们已经退出了比赛。我们的政治可能是由传统的思想者所领导的,如比尔·克林顿、赫尔穆特·科尔、约翰·梅杰;而我们最成功的企业则是由激进者领导的,他们对新的技术世界有着敏锐的理解,其中的典型就是比尔·盖茨。传统思维会因为无法应对变化的速度和力量,而一败涂地。

然而,道德并非如此。以形成于约公元前 1000 年的摩西经典为例,如果我们要从中寻找科学的话,那收获不多。《创世纪》中关于创世的描述,可能包含了一个神学真理——上帝创造了宇宙和人类,但它并没有对物理结构的实际发展做出科学的解释。但是,如果我们学习摩西的道德观,即十诫,那就有很多值得借鉴的地方。

尊重父母,忠于婚姻,是维护家庭生活的最好方式;家庭生活是培养道德健康的孩子的最好方式。偷盗对窃贼和失主都有损害,也不利于工作和储蓄。社会秩序有赖于证人不做假见证。杀人是错误的,等等。

在科学方面,三千年的发展彻底改变了人类的知识内容;而在道德上,我们其实可能倒退了。一般的心理医生给病人提供的,关于如何生活的道德建议,可能还不如普通的犹太人,在摩西时期从老师那里得到的好。当然,基督教依然存在,但对世界上的大多数人来说,它只是以往自身的一个苍白幽灵。没什么人还怀有早期时代的信仰,甚至社会还不那么复杂时的信仰;没有人会在公园大道上寻找圣徒。

对传统的破坏是科学进步的必要条件。如果我们至今都还相信,是太阳绕着地球转,那我们就不可能发明出卫星通信。事实上,我们所认为的科学,本身只是一系列的假设,是不完美的解释,注定要被其他的解释所取代,更强大的解释,但依然是不完美的。然而,对传统的破坏给人类的道德秩序带来了灾难。

孔子教导说,我们应该行为适度(他把黄金分割 Golden mean 称为中庸之道 chumyum,至少 17 世纪的学者是这么翻译的)。他还教导我们要尊重师长,待人如己。这些教诲已经有 2500 年的历史了。作为一种传统,它影响了中国有文字记载的所有历史;但对于许多现代中国人来说,儒家思想是一种过时的传统,他们不注重节制,尊重权力而不是师长,当然也不会以自己希望被对待的方式去对待别人。随着传统的沦丧,社会将失去其道德共识中的全部表述。中国的实力虽然不断增强,但与西藏相比,却是一个道德落后的国家,虽然西藏人贫穷且受到压迫。(严厉谴责作者分裂中国,译者坚决支持一个中国的原则。)在我们看来,良好的社会道德具有某些特征。它应该以动态而非静态的方式,促进社会与个人的生存。它应该鼓励宽容,避免自以为是。它应该是宗教性的,而不仅是不可知论的。它不应该假装可以解决科学事实的问题。它既不应该是无政府主义的,也不应该是独裁主义的。它应该得到广泛的认同和深深的用户。这样的社会道德,对于家庭,对于把孩子培养成独立、负责任的成年人,都至关重要。

它提供另一个良好社会的着力点。

我们发现,所有这样的道德,都会为从商业和同情共感中,产生的相互依存的逻辑所支持,但却会因为肤浅的科学主义的攻击、超阶级和亚阶级间的疏离、已丧失根基的旧式地域经济,而受到威胁。对于这些趋势,人们也许会做出反应。必须认识到,在下个世纪,它们对社会的危害会非常大。

随着以赛亚·柏林所说的“西方历史上最恐怖的世纪”随风而逝,社会结构的巨人主义时代也进入了尾声。20 世纪最后的日子,注定是一个规模缩减、权力下放、机构重组的时期。这将是社会恐龙(指大机构,译注)困在沥青坑里的时代;也是一个清道夫的时代。鸟儿将衔走恐龙的骨头。政府、企业和工会将不得不违背自己的意志进行调整,去适应微技术渗透下所确立的元宪政的新环境。微技术已经深刻地改变了行使暴力的边界。当今世界的变化,也已远远超出了我们通常的理解,超出 CNN 和报纸所告诉我们的。而它的变化方向,恰恰是对大政治条件的研究所指明的。先是在《血流成河》中,然后在《大清算》中,我们都论述过,当技术或其他决定暴力行使边界的因素发生改变时,社会的特征必然会随之改变。一切附着于人类互动方式的东西,包括我们的道德和看待世界的常识,都将随之改变。一段道德的松懈期过去了,也表明一个时代结束了;在此之后,我们会看到更严厉的道德的觉醒,它会提出更严格的标准,以满足一个竞争性主权世界的更高要求。

可以预见新道德会有几个特点。首先,它会强调生产力的重要性,强调创造收入者保留收入的正确性。由此可以推出另外一点,它将强调投资效率的重要性。信息时代的道德观会为效率而喝彩,会承认把资源用于最高价值的用途是一种优点。换句话说,信息时代的道德将是市场的道德。而正如詹姆斯·贝内特(JamesBennett)所言,信息时代的道德也是一种信任的道德。网络经济是一个高信任度的社区。在这样的环境中,因为有牢不可破的加密技术,贪污犯或小偷可以把犯罪所得安全地转移和存储,而且无法被追回;所以,为了避免损失,人们从一开始就会有强烈的动机,避免与不诚实的人做生意。就像前面所举的贵格会的例子,诚信的声誉是在网络经济中的重要资产。由于网络空间的匿名性,这种声誉不一定直接指向某个已知的个人,但它能够通过识别加密密钥而得到可靠的验证。如果加密方式或加密身份的验证被不法之徒或其他人所破坏,它可能导致的辐射性灾难令人望而生畏;这一点就足以使人们强烈排斥雇用任何缺乏可信度的人。贝内特设想了“赛博绅士俱乐部”(A Gentleman’s Club of Cyberspace),这是一块被保护的领域,需要有强安全的措施才能参与,“可能会使用声纹等生物特征进行识别和验证。业主要担保参与者的身份,在一定程度上保证他们的可信度,并承担相应的责任,以此来实现一个‘赛博空间的绅士俱乐部’(当然现在也欢迎女士)。相比一般的网络空间,在这样的地方,人们可以更加安全更有信心地进行交易。因此,在 21 世纪,人们可能会看到,在任何一个维多利亚人都想象不到的环境中,社会对信用和品格的重视将回归到维多利亚时代的程度。”网络空间保护区也可以提供降低风险的担保,类似于香槟区伯爵为往返香槟集市的商人,而提供的超地域保护担保。即其他的管辖区会“赔偿行旅商人在通过特定贵族所辖领土时可能遭受的任何损失”。

当时的“集市守卫”,原本是由伯爵认命的官员,为集市上的商人提供安全保护和“正义法庭”。最终他们演变成了更独立的实体,有专门的印章,对合同进行公证并强制执行,并有权“禁止任何被发现不支付债务或不履行合同的商人参加未来的交易会。这显然是一个很严厉的惩罚,所以很少有人会冒险犯禁,被剥夺未来赚钱的机会。不过,除此之外,守卫们还可以没收违约债务人的货物,并为债权人的利益而将其出售。”当替代性市场的数量增加时,作为执行合同手段的排斥主义就没那么重要了。但随着现在信息技术的出现,在社会进入下一个主权分散的阶段时,对欺骗着和违约者的排斥,可能再次成为一种强有力的执行机制。计算机的联动验证,可以使信用和欺诈信息不可伪造,从而对网络空间提供警戒。在互联网的意义上,整个世界是一个非常小的社区,骗子和小偷将面临更大的压力。

未来的道德观除了重视收入和效率,并重新强调品质和信用的重要性,它可能还会着力突出暴力的邪恶,特别是绑架和勒索;因为这两种“要挟”个人的手段,会变得越来越炽盛,不然那很难从个人手里夺走资源。

还有一个因素,也会提高道德的标准,那就是特权和收入再分配的结束。对于那些被社会拉下的人,当他们的获救希望主要来自于私人和慈善机构时,在自愿给予慈善救助的人看来,接受者在道德上应该是值得帮助的,这一点相比 20 世纪时更加重要。

\begin{tcolorbox}
补贴、意外之财和火热的经济前景,削弱了节制与留存的紧迫性。民主、再分配和经济发展的咒语,提高了人们的期望值和生育率,促进了人口增长,进而导致经济和环境的螺旋式下降更加迅猛。
\begin{flushright}
—— 弗吉尼亚·阿伯内西(VIRGINIA ABERNETHY)
\end{flushright}
\end{tcolorbox}


在某些方面,新的信息世界能够更好地鼓励人们以严肃的态度处理道德问题。收入再分配的承诺,在美国、加拿大和西欧地区,满足了不幸者和失败者的期望,在国际社会上也产生了不利的影响。有充分的证据表明,外国的援助和干预,原本旨在解决落后经济体的饥荒,并提高当地的生活水平,但往往是刺激人口增长的主要因素,使之超出了受援助地区的承受能力。第二次世界大战以来,全球人口的惊人增长,以及对森林、土壤和水资源的破坏,都可以归因到全球范围内的干预。原本,在当地人口与其供养所需的资源之间,通过负结果反馈机制保持着平衡,而这种干预使得该机制失灵了。

当然,那些生活在资源匮乏、增长乏力或毫无希望的环境下的人们,会非常高兴,因为他们得到了保证,贫乏的乡村生活就要结束了。他们热切地接受着国际援助工作和、和平队志愿者、当地革命者,以及冷战期间不同意识形态的竞争者所传递的乐观信息,告诉他们,明天的日子会更美好。而这恰恰是一个错误的信息。

不同文化间的收入再分配,会产生一个重要的后果,它使那些生活在非工业文明中、持有非工业价值观的人,获得了虚假的竞争力。国际援助、抗击饥荒和疾病的救援工作以及技术干预,愚弄了很多人,使他们相信自己的生活前景已极大地改善,所以没必要更新自己的价值观,或费力改善自己的行为。

国家间的收入再分配,不仅促使世界人口不可持续地激增,并以重要的方式助长了文化相对主义,而且,对于文化在影响人们创造当地经济繁荣中所扮演的关键角色,造成了普遍的困惑。今天,大多数人都认为,文化更多是一种品味问题,而不是行为指南的来源;这些指南既可能误导人,也会启发人。我们太热衷于相信所有的文化都是平等的,而对于认识反作用力文化的弊端,则非常迟钝。本世纪在世界的很多地方,在补贴和干预的温室中培育出来的混合文化,更是如此。

就像美国城市中心区的犯罪亚文化一样,它们保留了一些来自于早期经济发展阶段的零碎文化片段,而这些片段将与信息时代的行为价值观结合到一起。

因此,信息革命不仅会释放出天才精神,也会释放出天惩精神(nemesis 报应)。

在未来的一个千年里,二者之间将进行前所未有的较量。

工业社会向信息社会的转变,必将令人心潮澎湃,叹为观止。从经济生活的一个阶段过渡到另一个阶段,总是要经历一场革命。我们认为,相比农业革命和工业革命,信息革命的影响可能是最深远的,它将更彻底地颠覆和重组人们的生活。而它的影响力,你很快就将感受到。请系好安全带。