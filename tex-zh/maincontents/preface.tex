\thispagestyle{empty}
\addcontentsline{toc}{chapter}{前言}
\chapter*{前言}

中世纪的人对意志没有信心,认为人类是容易受伤的、软弱的,但他们尊重智力。

他们认为只要认真思考,即使是人,也有能力回答关于上帝和宇宙的最深奥的问题。


现代人崇拜意志,但他们对智力感到绝望。乌合之众,随机粒子的偏转,无意识偏见的影响:所有这些当代的陈词滥调,都在谈论智力的弱点,或者说也在谈论我们自己。


威廉·里斯·莫格勋爵和詹姆斯·戴尔·戴维森,并没有承诺也没有给出任何关于上帝和宇宙的答案。但是,他们对“大政治”的研究,对历史上各种力量的剖析,以及对不久将来的一系列预测,是非比寻常的,甚至是反文化的,因为他们运用人类的理性,去思索那些我们被教导为“机遇”或“命运”的事情。


在《主权个人》首次出版近四分之一个世纪之后,回顾过去,最容易做、也是我们周围的文化最鼓励做的事,就是检测他们的错误;这也算是一种自我安慰:那么费心去思考未来有什么意义呢。


当然,有一些事情他们没有料到:首先就是中国的崛起。在共产党的领导下,21世纪的中国创造了自己的信息时代,具有明显的民族主义、种族同化和深刻的国家主义特征。这可能是该书出版以来最大的“大政治”现象。仅举一个关键的例子,共产主义中国已经粉碎了香港这个城邦(城市国家),而里斯·莫格和戴维森曾将香港描述为“一种心智模式,一种会在信息时代繁荣昌盛的管辖区模式”。

从某个角度看,这是作者的盲点之一。从另一个角度看,中国的政治局委员一定是《主权个人》的热心读者。在不断重温列宁斯大林主义的同时,他们也在积极地展望信息时代,只有这种特有的、长期的警惕意识,才使得党的领导人能在本书分析的趋势中获得胜利。


这些趋势在今天依然适用:赢家通吃的经济、管辖权的竞争、大规模生产的转移,以及国家间的战争可能会过时。中国的崛起,与其说是对里斯·莫格和戴维森的反驳,倒更像是对他们所描述的利害关系的剧烈提升。


事实上,未来大政治的重大冲突才刚刚开始。在技术层面上,这场冲突的两极是:人工智能和加密技术。人工智能展现出一种前景,它能够最终解决经济学家所说“计算问题”(计划经济的关键,译注)。理论上,它使集中控制整个经济成为可能。中国共产党最喜欢的技术,就是人工智能,这绝不是巧合。强加密技术在另外一极,它带来的远景是一个去中心化和个性化的世界。如果说人工智能是共产主义的,那么加密技术就是自由主义的。


未来可能就落在这两极之间。而要知道,我们今天采取的行动,会决定日后全局性的结果。在 2020 年,阅读《主权个人》,是你认真思考自己的行动将塑造何种未来的一种方式,是一次不容浪费的学习机会。

\begin{flushright}
\kaishu 彼得·泰尔(Peter Thiel)\\
2020 年 1 月 6 日,洛杉矶
\end{flushright} 

\let\cleardoublepage\clearpage