\chapter[公元2000年的转折点]{公元2000年的转折点:\\ 人类社会的第四个阶段}

\begin{tcolorbox}
\kaishu 感觉有大事即将发生,各种图表向我们展示了世界人口每年的增长,大气中二氧化碳的浓度、网站地址、一美元能买到的存储容量,它们都在世纪之交后飙升到一个渐进点:奇点。在那里,一切我们已知的都将结束,某些我们可能永远无法理解的将会诞生。

\begin{flushright}
——《千年时钟》丹尼·希利斯(Danny Hillis)
\end{flushright}

\end{tcolorbox}


\section{预言}
在耶稣降临后的第一个千年之交,世界并未像传说的那样毁灭。其后,在过去的一千年里,公元 2000 年的到来一直困扰着西方人的想象。神学家、传教士、诗人和预言家都在张望着本世纪最后十年的结束,期待着历史性事件的发生。权威如伊萨克·牛顿,也曾经预测,整个世界将随着公元 2000 年的到来而结束\footnote{另外一种说法,牛顿预言世界将会在 2060 年重置。}。米歇尔·德·诺斯特拉达穆斯(Michel de Nostradamus),他的预言首次出版于 1568 年,每一代人都读过。他预告,基督的敌人将会在 1999 年 7 月到来。瑞士心理学家卡尔·荣格, “集体无意识”的鉴别大师,预言一个新时代会在 1997 年来临。这些预言也许很可笑,但不可否认的是,当大众不确定该相信什么的时候,预言会焕发出一些病态的吸引力。


在过去的 250 年里,一种对未来的不安,给西方社会特有的乐观主义染上了阴影。各地的人们都犹豫不决,忧心忡忡。你可以从他们的脸上看到,从他们的谈话中听到;它反映在民意调查中,登记在选票箱中。就像在乌云密布、闪电到来之前,大气中看不见的离子的物理变化,已经预示了雷雨即将降临。

如今,在千禧年的黄昏,空气中弥漫着变革的预感。一个又一个人,在一种行将结束的生活方式下,感受到时间就要燃尽。随着最后十年的过去,一个肃杀的世纪,同时也是人类成就辉煌的一千年,就此告以终章。所有的一切,都将因 2000年的到来而画上句号。



\begin{tcolorbox}
\kaishu 所以不要怕他们,因为被遮盖的事没有将不被显露出来的,隐秘的事也没有将不被知道的。
\begin{flushright}
——《马太福音》10:26
\end{flushright}
\end{tcolorbox}

我们相信并将论证,西方文明的现代阶段也将随之结束。像许多早期探索一样,本书尝试在黑暗之中,凝视一个模糊的玻璃杯,勾勒出它尚未成型的、未来的模样和尺寸。从这个意义上说,我们的作品,就是 Apokalypsis 这个词的希腊文原意,即末日启示录。Apokalypsis 在希腊语中的意思是“揭开面纱”。我们相信,一个新的历史阶段,即\textbf{主权个人}的时代,即将被揭幕。


\begin{tcolorbox}
\kaishu 我们正在目睹一个崭新逻辑空间的开始,一个即时的电子万物场,我们可以访问、走进和体验。
简而言之,一种新型社区在向我们走来。虚拟社区将成为世俗天国的模范;正如耶稣所说,在他父的天国里有众多华厦,因而也会有众多的虚拟社区,每一个都反映他们自己的需求和愿望。
\begin{flushright}
—— 迈克尔·格鲁索(MICHAEL GRASSO)
\end{flushright}
\end{tcolorbox}

\section{人类社会的第四阶段}
本书的主旨是探讨一场新的权力革命,它将以 20 世纪民族国家的毁灭为代价,解放出个体。创新,以前所未有的方式改变了暴力的逻辑,并且正在革新未来的边界。如果我们的推论是正确的,你们正站在一场有史以来最宏大的革命的门槛上。微处理的速度之快,超出所有人的想象,它将颠覆和摧毁民族国家,并在此过程当中,创造出新的社会组织形式。这场变革绝非儿戏。


它所带来的挑战是前所未有的,与既往的任何事物相比,它发生的速度都难以置信。纵观历史,从最初时期到现在,人类的经济生活只经历过三个基本阶段:(1)狩猎和采集社会;(2)农业社会;(3)工业社会。\uline{现在,在地平线上若隐若现的,是一个崭新的阶段,也就是人类社会组织的第四阶段:信息社会}。


前面的几个社会阶段,对应着暴力演进以及人类控制暴力的不同时期。我们将详细阐释,信息社会有望极大地减少暴力的回报,部分原因是它超越了地域性。新的千年,掌控大规模暴力的优势,可能会远远低于法国大革命之前的任何时候,这将产生深远的影响。

其中之一是犯罪率的上升。当有组织、大规模的暴力的回报率下降时,较小规模暴力的回报率则很可能会上升。暴力将变得更加随机和局部,有组织犯罪的范围会扩大。我们将就此作出解释。

从逻辑因果来看,暴力回报率的下降,将导致政治的黯淡。很多证据表明,20世纪支撑起来的民族国家的公民神话,正在被快速地抛弃。共产主义的死亡只是一个最显著的例子。我们将详细探讨,西方政治领导人的道德败坏,以及日益严重的政治腐败,绝非偶然。它证明了民族国家的潜力已经耗尽。人们不再相信政客口中的陈词滥调,甚至连他们自己都不相信。

\subsection{历史将会重演}
这种情形与过去惊人地相似。每当发生技术变革,旧的规制就会被新的经济驱动力甩脱。社会的道德标准随之改变,人们开始越来越不屑地对待旧体制的掌权者。


在新的革命意识形态达成一致之前,这种普遍的反感往往已经显露出来了。十五世纪末就是如此,当时的教会是中世纪封建主义的主导机构。尽管人们还普遍相信“教职的神圣性”,但是无论高级还是低级的神职人员,都遭到了极大的蔑视。

这与今天的人们对待政治官僚的态度并无二致。当今的世界被政治所充斥,与 15 世纪末宗教充斥一切的生活相类比,我们可以学到很多东西。15 世纪末,支撑宗教制度的成本已经到了历史的最顶点,就像今天政府的成本已经到了极限,抵达衰败的边缘。


我们都知道,在火药革命之后,有组织的宗教发生了哪些变化。技术的发展创造了强大的动力,促使宗教机构缩减规模,降低成本。在新千年的初期,一场类似的技术革命,注定将彻底缩小民族国家的规模。



\begin{tcolorbox}
\kaishu 今天,经过一个多世纪电力技术的发展,我们已经将自身的中枢神经系统扩展到了全球,无远弗届,就我们这个星球的范畴而言,空间和时间已经被废止。
\begin{flushright}
—— 马歇尔·麦克卢汉(MARSHALL McLUHAN),1964 年
\end{flushright}
\end{tcolorbox}

\subsection{信息革命}
随着大系统的加速崩溃,作为塑造经济生活和收入分配的一个因素,系统性的强制将会式微。很快,在社会机构的组织中,效率将会比权力的分配更加重要。在网络空间,一个全新的经济领域将会出现,它不受人身暴力的制约。最显著的利益将会流向“认知精英”,他们将越来越多地在政治的边界之外进行运作,他们在法兰克福、伦敦、纽约、布宜诺斯艾利斯、洛杉矶、东京和香港都有住所。国家内部的收入会更加不平等,但是在这些管辖区,收入将更加平等。

《主权个人》这本书,会探讨这场信息革命将带来的社会和金融后果。我们的愿望是,帮助你抓住新时代的机遇,并避免被它的冲击波给摧毁。如果我们预期的事情有一半会发生,你就将面临史所罕见的沧海剧变。

2000 年的变革将改变世界经济的特征,不仅翻天覆地,而且比以往任何阶段都迅猛无比。它不会像农业革命,需要几千年的时间;也不会像工业革命,蔓延几个世纪。信息革命将在一代人的时间内发生。

更重要的是,在世界各地,它几乎会同时上演。技术和经济的创新,将不再局限于一小部分地区。它对旧世界的突破,如此高远,乃至古希腊等早期农业民族所幻想的众神的仙境,都将成为现实。多数人不愿意承认,但事实将会证明,大多数的当代组织,在新的千年里,在很大程度上都很难或者不可能存在。信息社会一旦形成,它与工业社会的区别,将像埃斯库罗斯\footnote{Aeschylus(公元前525~公元前456),埃斯库罗斯出生于希腊埃莱乌西斯,在雅典的黄金时代长大,甚至在公元前490年的马拉松战役中与入侵的波斯军队作战。在此之前,大约在公元前500年,他开始写戏剧,到484年,他在酒神节上获得了一等奖,这是希腊最重要的悲剧戏剧节,也是希腊剧作家的巨大荣誉。有“悲剧之父”的美誉。 代表作有《被缚的普罗米修斯》、《阿伽门农》、《善好者》等。}的希腊与穴居人的世界的差别一样大。

\section{解缚的普罗米修斯:主权个人的崛起}

\begin{tcolorbox}
\kaishu 最令人鼓舞的事情莫过于,人类通过有意识的努力来提升自己的生活能力;而这种能力毋庸置疑。
\begin{flushright}
——亨利·戴维·梭罗(HENRY DAVID THORAU)
\end{flushright}
\end{tcolorbox}

这场转型,带来的既有好消息,也有坏消息。好的一面,信息革命将使个人得到前所未有的解放。有能力自我教育的人,将第一次完全自由地创造自己的工作,实现自身生产力的全部利益。天才将得到释放,既不受政府的压迫,也不受种族和民族偏见的拖累。