\chapter[公元2000年转折点]{公元2000年转折点:\\ 人类社会的第四个阶段}

\begin{tcolorbox}
\kaishu 感觉有大事即将发生,各种图表向我们展示了世界人口每年的增长,大气中二氧化碳的浓度、网站地址、一美元能买到的存储容量,它们都在世纪之交后飙升到一个渐进点:奇点。在那里,一切我们已知的都将结束,某些我们可能永远无法理解的将会诞生。

\begin{flushright}
——《千年时钟》丹尼·希利斯(Danny Hillis)
\end{flushright}

\end{tcolorbox}


\section{预言}
在耶稣降临后的第一个千年之交,世界并未像传说的那样毁灭。其后,在过去的一千年里,公元 2000 年的到来一直困扰着西方人的想象。神学家、传教士、诗人和预言家都在张望着本世纪最后十年的结束,期待着历史性事件的发生。权威如伊萨克·牛顿,也曾经预测,整个世界将随着公元 2000 年的到来而结束\footnote{另外一种说法,牛顿预言世界将会在 2060 年重置。}。米歇尔·德·诺斯特拉达穆斯(Michel de Nostradamus),他的预言首次出版于 1568 年,每一代人都读过。他预告,基督的敌人将会在 1999 年 7 月到来。瑞士心理学家卡尔·荣格, “集体无意识”的鉴别大师,预言一个新时代会在 1997 年来临。这些预言也许很可笑,但不可否认的是,当大众不确定该相信什么的时候,预言会焕发出一些病态的吸引力。


在过去的 250 年里,一种对未来的不安,给西方社会特有的乐观主义染上了阴影。

各地的人们都犹豫不决,忧心忡忡。你可以从他们的脸上看到,从他们的谈话中听到;它反映在民意调查中,登记在选票箱中。就像在乌云密布、闪电到来之前,大气中看不见的离子的物理变化,已经预示了雷雨即将降临。

如今,在千禧年的黄昏,空气中弥漫着变革的预感。一个又一个人,在一种行将结束的生活方式下,感受到时间就要燃尽。随着最后十年的过去,一个肃杀的世纪,同时也是人类成就辉煌的一千年,就此告以终章。所有的一切,都将因 2000年的到来而画上句号。



\begin{tcolorbox}
\kaishu 所以不要怕他们,因为被遮盖的事没有将不被显露出来的,隐秘的事也没有将不被知道的
\begin{flushright}
——《马太福音》10:26
\end{flushright}
\end{tcolorbox}

我们相信并将论证,西方文明的现代阶段也将随之结束。像许多早期探索一样,本书尝试在黑暗之中,凝视一个模糊的玻璃杯,勾勒出它尚未成型的、未来的模样和尺寸。从这个意义上说,我们的作品,就是 Apokalypsis 这个词的希腊文原意,即末日启示录。Apokalypsis 在希腊语中的意思是“揭开面纱”。我们相信,一个新的历史阶段,即\textbf{主权个人}的时代,即将被揭幕。


\begin{tcolorbox}
\kaishu 我们正在目睹一个崭新逻辑空间的开始,一个即时的电子万物场,我们可以访问、走进和体验。
简而言之,一种新型社区在向我们走来。虚拟社区将成为世俗天国的模范;正如耶稣所说,在他父的天国里有众多华厦,因而也会有众多的虚拟社区,每一个都反映他们自己的需求和愿望。
\begin{flushright}
—— 迈克尔·格鲁索(MICHAEL GRASSO)
\end{flushright}
\end{tcolorbox}

\section{人类社会的第四阶段}
本书的主旨是探讨一场新的权力革命,它将以 20 世纪民族国家的毁灭为代价,解放出个体。创新,以前所未有的方式改变了暴力的逻辑,并且正在革新未来的边界。如果我们的推论是正确的,你们正站在一场有史以来最宏大的革命的门槛上。微处理的速度之快,超出所有人的想象,它将颠覆和摧毁民族国家,并在此过程当中,创造出新的社会组织形式。这场变革绝非儿戏。


它所带来的挑战是前所未有的,与既往的任何事物相比,它发生的速度都难以置信。纵观历史,从最初时期到现在,人类的经济生活只经历过三个基本阶段:(1)狩猎和采集社会;(2)农业社会;(3)工业社会。\uline{现在,在地平线上若隐若现的,是一个崭新的阶段,也就是人类社会组织的第四阶段:信息社会}。


前面的几个社会阶段,对应着暴力演进以及人类控制暴力的不同时期。我们将详细阐释,信息社会有望极大地减少暴力的回报,部分原因是它超越了地域性。新的千年,掌控大规模暴力的优势,可能会远远低于法国大革命之前的任何时候,这将产生深远的影响。

其中之一是犯罪率的上升。当有组织、大规模的暴力的回报率下降时,较小规模暴力的回报率则很可能会上升。暴力将变得更加随机和局部,有组织犯罪的范围会扩大。我们将就此作出解释。

从逻辑因果来看,暴力回报率的下降,将导致政治的黯淡。很多证据表明,20世纪支撑起来的民族国家的公民神话,正在被快速地抛弃。共产主义的死亡只是一个最显著的例子。我们将详细探讨,西方政治领导人的道德败坏,以及日益严重的政治腐败,绝非偶然。它证明了民族国家的潜力已经耗尽。人们不再相信政客口中的陈词滥调,甚至连他们自己都不相信。

\subsection{历史将会重演}
这种情形与过去惊人地相似。每当发生技术变革,旧的规制就会被新的经济驱动力甩脱。社会的道德标准随之改变,人们开始越来越不屑地对待旧体制的掌权者。


在新的革命意识形态达成一致之前,这种普遍的反感往往已经显露出来了。十五世纪末就是如此,当时的教会是中世纪封建主义的主导机构。尽管人们还普遍相信“教职的神圣性”,但是无论高级还是低级的神职人员,都遭到了极大的蔑视。


这与今天的人们对待政治官僚的态度并无二致。


当今的世界被政治所充斥,与 15 世纪末宗教充斥一切的生活相类比,我们可以学到很多东西。15 世纪末,支撑宗教制度的成本已经到了历史的最顶点,就像今天政府的成本已经到了极限,抵达衰败的边缘。


我们都知道,在火药革命之后,有组织的宗教发生了哪些变化。技术的发展创造了强大的动力,促使宗教机构缩减规模,降低成本。在新千年的初期,一场类似的技术革命,注定将彻底缩小民族国家的规模。



\begin{tcolorbox}
\kaishu 今天,经过一个多世纪电力技术的发展,我们已经将自身的中枢神经系统扩展到了全球,无远弗届,就我们这个星球的范畴而言,空间和时间已经被废止。
\begin{flushright}
—— 马歇尔·麦克卢汉(MARSHALL McLUHAN),1964 年
\end{flushright}
\end{tcolorbox}

\subsection{信息革命}
随着大系统的加速崩溃,作为塑造经济生活和收入分配的一个因素,系统性的强制将会式微。很快,在社会机构的组织中,效率将会比权力的分配更加重要。在网络空间,一个全新的经济领域将会出现,它不受人身暴力的制约。最显著的利益将会流向“认知精英”,他们将越来越多地在政治的边界之外进行运作,他们在法兰克福、伦敦、纽约、布宜诺斯艾利斯、洛杉矶、东京和香港都有住所。国家内部的收入会更加不平等,但是在这些管辖区,收入将更加平等。

《主权个人》这本书,会探讨这场信息革命将带来的社会和金融后果。我们的愿望是,帮助你抓住新时代的机遇,并避免被它的冲击波给摧毁。如果我们预期的事情有一半会发生,你就将面临史所罕见的沧海剧变。

2000 年的变革将改变世界经济的特征,不仅翻天覆地,而且比以往任何阶段都迅猛无比。它不会像农业革命,需要几千年的时间;也不会像工业革命,蔓延几个世纪。信息革命将在一代人的时间内发生。

更重要的是,在世界各地,它几乎会同时上演。技术和经济的创新,将不再局限于一小部分地区。它对旧世界的突破,如此高远,乃至古希腊等早期农业民族所幻想的众神的仙境,都将成为现实。多数人不愿意承认,但事实将会证明,大多数的当代组织,在新的千年里,在很大程度上都很难或者不可能存在。信息社会一旦形成,它与工业社会的区别,将像埃斯库罗斯\footnote{Aeschylus(-525~-456),埃斯库罗斯出生于希腊埃莱乌西斯,在雅典的黄金时代长大,甚至在公元前490年的马拉松战役中与入侵的波斯军队作战。在此之前,大约在公元前500年,他开始写戏剧,到484年,他在酒神节上获得了一等奖,这是希腊最重要的悲剧戏剧节,也是希腊剧作家的巨大荣誉。有“悲剧之父”的美誉。 代表作有《被缚的普罗米修斯》、《阿伽门农》、《善好者》等。}的希腊与穴居人的世界的差别一样大。

\section{解缚的普罗米修斯:主权个人的崛起}

\begin{tcolorbox}
\kaishu 最令人鼓舞的事情莫过于,人类通过有意识的努力来提升自己的生活能力;而这种能力毋庸置疑。
\begin{flushright}
——亨利·戴维·梭罗(HENRY DAVID THORAU)
\end{flushright}
\end{tcolorbox}

这场转型,带来的既有好消息,也有坏消息。好的一面,信息革命将使个人得到前所未有的解放。有能力自我教育的人,将第一次完全自由地创造自己的工作,实现自身生产力的全部利益。天才将得到释放,既不受政府的压迫,也不受种族和民族偏见的拖累。


在信息社会里,任何一个真正有能力的人,都不会被别人的错误意见所羁绊。地球上的大多数人,不管他们对你的种族、长相、年龄、性取向、发型的看法如何,都无关紧要。在网络经济中,他们永远不会看到你。丑陋的、肥胖的、年老的、残疾的,将完全匿名地,在网络空间的新疆域,与年轻漂亮的人平等竞争。

\subsection{思想即是财富}

在信息时代,英雄不问出处。才能,无论它在哪里出现,都将得到前所未有的回报。在这样一个环境中,财富的最大来源将是你脑子里的想法,而不仅仅是物质资本;任何一个头脑清晰的人都有可能成为富翁。\uline{信息时代将是一个向上流动的时代。}世界上的很大一部分地区,有几十亿人,从来没有充分分享过工业社会的繁荣,信息时代将为他们带来更加平等的机会。那些智慧过人、能力出众、充满抱负的个体,将会作为真正的主权个人涌现出来。

主权个人处在生产力的最高峰,他们之间的竞争和互动,简直与希腊众神之间的关系遥相呼应。下一个千年,网络空间将成为神秘莫测的奥林匹亚山,它没有实体,看不见摸不着,但有望在新千年的第二个十年成为全球最大的经济体。到2025 年,网络经济的参与者将数以百万计。他们中间将会出现很多个比尔·盖茨,每个人都身家百亿美元。那些年收入不到 20 万美元的人,将成为网络穷人。


没有网络战,没有网络税,没有网络政府。未来 30 年最伟大的经济现象,很可能是网络经济,而不是中国。


好消息是,在这个新的领域,政客们要支配、压制、管理大部分商业,其难度不亚于古希腊城邦的立法者想修剪宙斯的胡须。全球经济的很大一部分,会从政治的控制中解放出来,这将迫使残余的政府组织,必须按照更贴近市场规则的方式运行。最终他们将别无选择,只能把国土内的人民当作客户去服务,而不能再像有组织的犯罪分子那样绑架勒索。



\subsection{超越政治}
神话中所描述的众神的世界,将成为个人可行的选择,那就是一种国王和议会无法触及的生活。开始是几十人,然后是几百人,最终是几百万人,个体将纷纷摆脱政治的束缚。他们的成功,将改变政府的性质,缩小强权的领地,扩大私人控制资源的范围。


主权个人的出现,将再次证明神话的奇特预言力。早期的农牧民族,对自然法则几乎一无所知,在他们的想象中,“我们应该称之为超自然的力量”无处不在。


这些力量有时由人运用,有时被“神的化身”运用。在詹姆斯·乔治·弗雷泽爵士的《金枝》中,描述了一种“伟大的民主”,在那里,神的化身看上去和人一样,并且与人们互动。


当古人想象到宙斯的孩子们就生活在他们中间,这激发了他们对魔法的深刻信仰。与其他原始的农耕民族一样,他们敬畏大自然,并且迷信地认为,自然界的造化是由个人的意志和魔力所促成的。在这个意义上,他们对自然和神灵的观念,并非一种自觉式的预言。他们远没有预见到微处理技术的出现;他们无法想象到,在数千年之后,这项技术对提高个人边际生产力所产生的影响;他们当然也不可能预见,权力和效率之间的平衡将因此而被改变,创造和保护财产的方式也因此而被改写。然而,古人们在编织神话故事时所想象的,与你可能会看到的世界,将产生奇特的共鸣。



\subsection{新的 Abracadabra 咒语}
例如,召唤魔法时念的咒语“abracadabra”,就与我们登陆计算机的密码有很奇怪的相似之处。在某些方面,高速计算已经使我们可以施展类似精灵的魔法。初代的“数字仆人”就像被封印在魔灯中的精灵一样,被封在计算机之中,听从机主的召唤,执行他的命令。信息技术的虚拟现实将拓宽人的愿景,使人类能够想象到的一切,都看上去和真的一样。远程呈现赋予人类以超自然的速度跨越距离,从万里之外监控事件的能力,就像希腊人认为的赫尔墨斯和阿波罗那样。信息时代的主权个人,犹如古代和原始神话中的神一样,在适当的时候,会享有一种“外交豁免权”,摆脱古今中外的凡人都备受困扰的政治问题。


新的主权个人,和那些受支配的普通人生活在同样的物理环境中,但在政治上,他们却处于一个独立的空间,就像神。主权个人将掌握更多的资源,并且超越于强权之外。在新的千年里,他们将重新设计政府,重新配置经济。这种变革的深远影响,难以想象。



\subsection{天才与天惩}
对于任何一个追逐理想和成功的人来说,信息时代的回报将无与伦比。这无疑是几代人以来最好的消息,但也是一个坏消息。基于个人自治的新型社会组织,以及建立在能力之上的、真正的机会均等,会使才能出众者,得到超级的回报和个人自主性。但是,个人要对自己担负的责任,也会远远超过他们在工业时期所习惯的。此外,在整个 20 世纪,先进工业社会的居民,享受了不劳而获的优越生活,这种优势也将被削弱。在我们写这本书的时候\footnote{本书初版写于 1997 年之前。},世界上前15\%的人口,人均年收入为 21000 美元;其余 85\%的人,平均年收入只有 1000美元。在信息时代的新环境下,过去囤积起来的巨大优势,必将烟消云散。


随着它的消散,民族国家大规模重新分配收入的能力将崩溃。信息技术加剧了各管辖区之间的竞争。技术是流动的,交易是在网络空间进行的。任何人只要有一台便携式电脑,和一条卫星网络,就可以在任何地方,从事几乎任何信息业务,包括世界上数以万亿美元的金融交易。当这种发展愈演愈烈,政府将无法再为它们的服务收取超出其价值的费用。


这意味着,你不再需要为了高收入,而不得不生活在高税率的国家和地区。在未来,大多数财富可以在任何地方赚取,甚至可以在任何地方消费。到那时,政府试图对它的永久居民收取高额的服务费,只会丢掉它们最好的客户。如果我们的推理是正确的,我们相信它是正确的,那么,大家所知道的民族国家,将不会再以任何类似现在的形式而存在。

\section{国家的终结}
技术的变革,将削弱占支配地位的机构权力,这势必会导致紊乱和危险。就像在现代社会的早期,君主、领主、教皇和权贵们,为了维护他们习惯的特权而发动残酷的战争一样,今天的政府也同样会使用暴力,以隐蔽的、武断的手段,试图推迟或阻止末日的到来。面对技术对权力的挑战,国家会像采用一系列无情的外交手段,就像以往与其他政府打交道一样,去对付主权个人,也就是它的前公民。


1998 年 8 月 20 日,美国发射了一颗价值约 2 亿美元的战斧式 BGM-109 海射巡航导弹,轰炸一个据称与流亡的沙特亿万富翁乌萨马·本·拉登\footnote{1957年3月10日—2011年5月2日,出生于沙特阿拉伯吉达,“基地”组织创始人,被美国认定为2001年“9·11事件”主谋。他当过工程师,从石油和建筑业赚取了巨资,在西方拥有数家公司,涉及建筑、石油、制造和宝石等诸多行业,个人财产估计达数十亿美元。拉登家族是与沙特王室有亲密联系的富庶家族,本·拉登在52个兄弟姐妹中排行第17,有23个同父异母兄弟,父亲是沙特建筑业大亨。}有关的目标。这一事件,为新的历史阶段的揭幕,增添了轰动效应。美国还同时摧毁了苏丹喀土穆的一家制药厂,更为本·拉登增添了荣耀。拉登成为历史上第一人,因为他的卫星电话而成为轰炸目标。


本·拉登作为美国头号敌人的出现,使战争的性质发生了重大的改变。尽管他拥有数亿美元,但是一个个人,现在居然可以被令人采信描述为对美国——这个工业时代最强大的军事力量——的巨大威胁。美国总统和他的国家安全助理,在声明中把本·拉登描绘成一个跨国恐怖分子,是美国的一大劲敌,让人联想到冷战时期对苏联的宣传。


将本·拉登提升为美国主要敌人的军事逻辑,也将体现在政府与其国民的内部关系中。越来越严厉的税收技术,必然会导致政府和个人之间出现一种新型的谈判。


技术将使个人比以往任何时候都更接近主权,而他们也会像主权组织一样被对待。有时候作为敌人被暴击,有时候是平等的谈判对手,有时候则成为了盟友。


但是,无论政府多么冷酷,对它们来说,即使与中情局和国税局的联姻,用处也都不大,特别是在过度时期。要求与主权个人谈判的呼声与日俱增,而这些人的资源可不再那么容易被控制。


信息革命不仅给政府带来巨大的危机,而且会使所有的大型机构解体。在 20 世纪,十四个帝国消失了。帝国的瓦解,是民族国家本身消亡过程的一部分。政府将不得不适应日益增长的个人自主权。税收能力将骤降 50-70\%,较小的管辖区可能会更加成功。制定有竞争力的条件,以吸引有能力的个人及资本,相比在各大洲,飞地更有机会赢得这个挑战。


我们相信,随着现代民族国家的解体,现代的野蛮人将会更多地在幕后行使权力。


就像俄罗斯黑手党这样的团体,包括前苏联的骨干成员,其他种族的犯罪团伙、名流、毒枭和变节的秘密机构,他们将制定自己的法律。他们现在已经这么做了。现代的野蛮人,已经侵入了民族国家的体内,只是没怎么改变它的外表;这一点,远远超出人们普遍的认知。他们是微型的寄生虫,以垂死的系统为食。就像战争中的国家一样,这些团体在更小的范围内使用国家级的技术,残暴且不择手段。微处理技术,缩小了团体必要的规模,便于更有效地使用和控制暴力。随着这一技术革命的展开,暴力掠夺会越来越多地在中央的控制之外。而遏制暴力的努力,也将更多地转移到高效率的手段上,不再取决于权力的大小。



\subsection{历史的逆转}

民族国家在过去五个世纪的演进,在信息时代将被新的发展逻辑所颠覆。地方性的权力中心将重新确立,因为国家主权将变得支离破碎、互相重叠。有组织犯罪的与日俱增,只是这种趋势的一个反映。跨国公司已经不得不把大部分工作分包出去,除了必要的部分。一切大型的企业集团,如 AT\&T、Unisys 和 ITT,为了更好地运作,已经把自己分拆成了几个公司。民族国家就像一个笨重的集团公司,将不得不下放自己的权力;但是,在发生金融危机并迫使它这么做之前,可能性不大。


不仅世界上的权力在发生变化,工作也在变。这意味着商业的运作模式也将不可避免地改写。虚拟公司证明了,随着信息和交易成本的不断下降,企业的性质正在发生全面的转变。信息革命带来的影响,不仅将使公司解体,也会使所谓的“好工作”消失。\uline{在信息时代,“工作”是指一个项目或任务,而不再是你拥有的一个“职位”}。微处理技术创造了全新的经济活动领域,超越领土与边界。这种超越,也许是自亚当和夏娃被造物主从伊甸园赶出去之后,最具革命性的发展。


“你必汗流满面才得糊口”。随着技术革新人们使用的工具,法律将会过时,道德将被重塑,观念将被改写。本书会解释这种转变的发生。


快速发展的计算机和通信技术,使个人能够自由地选择工作地点。互联网上的交易可以被加密,税收部门想在很短的时间内捕捉到,几乎不可能。离岸的免税资金的复利增长,已经远远超过了在岸资产,而这些在岸的,还要遭受 20 世纪民族国家征收的高额税负。\uline{千禧年之后,世界上的大部分商业,都将迁移到网络空间}。在这里,政府的统治力,并不比它们对海底和外星的统治强多少。自古以来,身体暴力一直是政治最初及最后的威胁;在网络空间,这种威胁就失灵了。在网上,温和的人和强壮的人,在同等的条件下相遇。网络空间是终极的离岸管辖区,是免税的经济体,是镶满钻石的空中百慕大。


当这个最大的免税天堂,完全对商业开发时,基本上所有的资金都将成为离岸资金,由其所有者掌控。它带来的后果,将如瀑布般倾泻而下。长期以来,国家已经养成了习惯,对待它的纳税人就像农民对待奶牛,把它们圈在农场里挤奶。但是不久之后,奶牛将长出翅膀。



\subsection{国家的复仇}
就像一个绝望的农夫,国家一开始肯定会采取绝望的措施,去围堵和阻碍逃跑的牛群。它会使用隐蔽的、甚至是暴力的手段,限制人们获得和使用解放性的技术。


但是这种权宜之计根本于事无补,只能暂时奏效。20 世纪的民族国家,带着它所有的虚伪造作,将随着税收的减少而饿死。


当国家发现无法再通过增加税收来满足自己运转的支出时,它会采取更加绝望的措施;其中之一就是印钞票。\uline{所有的政府都已经习惯了对货币发行的垄断,并且可以随心所欲地贬值}。这种任意而为的通货膨胀,是 20 世纪所有国家货币政策的突出特点。即使是战后最好的国家货币,德国马克,也从 1949 年 1 月 1 日到1995 年的 6 月底,贬值了 71\%。美元同期则贬值了 84\%。这种通胀的效果,和对所有持有货币的人征税是一样的。后文我们将会探讨,随着\textbf{加密货币}的出现,通货膨胀将在很大程度上别摒除。得益于新的技术,财富的持有者可以绕过发行和监管货币的现代国家的垄断机构。国家继续控制着工业时代的印钞机,但是,它们控制世界财富的重要性,将被没有实体存在的数字算法所超越。\uline{在新的千年里,私有市场控制的网络货币,将取代政府发行的法币}。只有穷人会成为通胀以及随之而来的通缩的受害者,这是不断地向经济注入法币、制造人工杠杆的必然后果。

即使传统上很文明的国家,当税收和通胀的恶化超出政府习惯的范围,也会变得恶劣不堪。随着征税越来越难,一些古老的、武断的税收方式将重现江湖。终极形式就是预扣税,它实际上是一种公开的劫持人质。当财富外流加剧,一些火烧眉毛的政府将会拿起这一武器。不幸的人会发现自己被选中,以一种近乎中世纪的手段进行敲诈勒索。向个人提供有利于实现自治能力的服务的企业,将受到渗透、破坏和干扰。任意没收财产在美国已经司空见惯,现在每周发生 5000 起,未来会更加普遍。政府将侵犯人权,审查信息的自由流动,破坏有用的技术,甚至更糟。西方政府将使用利用极权手段压制网络经济,就像已经解体的苏联,曾经出于同样的原因,试图阻挠个人使用电脑和施乐印刷机,但最终徒劳无功。



\section{卢德分子的归来}

政府的强制手段,可能会得到某些人的欢迎和支持。个体解放和个人自治,虽然是好事,但是对许多被转型危机吓坏了的人来说,则是一个坏消息,他们不期望自己会成为新的社会结构中的赢家。


1998 年亚洲经济崩溃之后,马来西亚总理马哈蒂尔·穆罕默德实施了严厉的资本管制,明显很受欢迎。这说明了,对民族国家主导的、老式的、封闭型经济仍然抱有热情的,大有人在。无可逃避的转型危机将导致怨恨,而怨恨助长人们对过去的怀旧。最大的怨恨,可能主要来自于目前富裕国家的中等人才;他们会尤其觉得,信息技术对自己的生活方式构成了威胁。强权组织的受益者,包括数百万接受政府再分配的收入的人,也会对主权个人的新自由非常不满。这会很好地说明一个道理:“屁股决定立场”。


\begin{tcolorbox}
\kaishu 有时候我想知道,为什么自己会为一小撮陌生人的命运,感到如此深沉的痛苦?他们在数百英里外的球场上,与另外一群陌生人比赛。答案很简单,因为我热爱我的球队。虽然有风险,但投入关心是值得的。竞技让我热血沸腾,心潮澎湃,充满自豪。我喜欢冒险,在赛场上,生命更加鲜活。
\begin{flushright}
—— 克雷格·兰伯特
\end{flushright}
\end{tcolorbox}

但是,如果把未来转型危机中产生的所有敌对情绪,都归结为以他人为代价来满足自己欲望的思想,也是一种误导。它其实涉及更多层面。人类社会的特点表明,对于即将到来的卢德式反应,必然会发生道德层面的误导。你可以把这种反应看成是,赤裸裸的欲望戴着一顶道德的假发。在激发行动方面,自以为是的愤怒,要比自觉的自私自利有力得多。20 世纪公民神话的魅力虽然正在迅速消退,但它并不缺乏真正的信徒。正如克雷格·兰伯特引用的这段话,很多人需要归属感,他们渴望成为群体中的一员。同样的认同需求,可以激励体育爱好者成立组织,也可以激励很多人成为党员。20 世纪的每个成年人,都被灌输了 20 世纪公民的责任和义务。工业社会的道德残余,至少会刺激一些新卢德分子\footnote{“卢德分子”最初指的是19世纪初期英国的一群工人,他们反对工业化和机械化,认为这些新技术威胁到他们的生计。在现代,卢德分子的概念被重新定义,例如美国心理学家格兰蒂宁在1990年发表的《新卢德宣言》主张为新技术革命中的受害者发声,这标志着“新卢德运动”的兴起。此外,英国人查尔斯·斯诺曾提到,知识分子,尤其是人文知识分子,天生就是卢德分子,这一观点引发了广泛讨论。}对信息技术发起攻击。


这些可能发生的暴力行经,在某种意义上,我们认为它属于一种“道德上的不合时宜”,也就是把一个经济阶段的道德标准,应用到另外一个阶段。每个经济阶段,都有它自己的道德准则,以便在该特定生活方式下的人们,在面临选择时,能够克服它特有的动机陷阱。就像一个农业社会的人,不会按照爱斯基摩人的道德准则去生活;信息社会的道德标准,也不会满足 20 世纪国家的需求,因为后者的道德是为了促进一种好战的工业国家的成功。我们将解释其中的原因。


在未来的若干年内,道德上的不合时宜,将在西方的核心国家出现;就像在过去的五个世纪,人们在边缘国家所看到的那样。西方的殖民者和远征军,在遇到狩猎和采集的原住民,以及那些依然为农耕社会的民族时,就激发了上述的危机。


当新技术被引入到不合时宜的环境中,势必造成社会混乱和道德危机。基督教的传教士,能够成功转化数以百万计的原住民,在很大程度上,可以归功于,外部突然强加的权力变革导致的地方危机。


从 16 世纪,到 20 世纪的上半叶,这样的遭遇一次次地发生。我们预计,在新千年的初期,随着信息社会取代那些按照工业路线组织的社会,类似的冲突也将上演。


\subsection{对强权的怀旧}
信息社会的兴起,不会受到普遍的欢迎,不会被所有人视为是一个充满希望的历史新阶段。每个人都会感到一些疑虑,即使是从中获益最多的人,也不例外。很多人会敌视破坏领土民族国家的创新。


任何形式的激进变革,都会被视为是江河日下,越来越糟,这是人类的一个本性。


五百年前,勃艮第公爵\footnote{勃艮第公爵(英文:Dukes of Burgundy)是勃艮第公国统治者的头衔,843年由西法兰克王国的查理二世创立。}身边的臣仆们会说,破坏封建主义的创新是邪恶的,他们认为世界正在迅速地下滑;而后来的历史学家看到的,则是人类潜力大爆发的文艺复兴。同样,有一天,从下一个千年的角度看,信息革命可能被认为是新文艺复兴式的发展;但在二十世纪疲惫者的眼睛看来,却是可怕的。在这种新的发展方式下,感到受冒犯的人,或者丧失竞争优势的人,他们的反应可能会极不友好。在这场向新型社会组织的激烈过渡中,遭遇这些新卢德分子,在某种程度上,对每个人来说,都是一个坏消息。


准备好躲避吧。因为变化的速度太快,远远超过活着的人在经济和道德上的适应能力。可以预计,尽管信息革命有解放未来的伟大使命,但它将遭到愤怒的抵制。


一系列的转型危机摆在面前,你必须理解并为敌对情绪做好准备。随着时间发展,工业时代遗留下来的过时的国家和国际组织,将被证明不足以再应对新的、分散的、跨国经济的挑战。通货紧缩的悲剧,类似 1997 年和 1998 年,席卷远东、俄罗斯,并传染了亚洲其他新兴经济体的金融危机,将会零零星星地不断爆发。


新的信息和通讯技术对现代国家的颠覆,超过自哥伦布航海时代以来,其他任何对统治地位的政治威胁。认识到这一点很重要。因为当权者对破坏其权威的发展,很少和平以待。这一次也不例外。


新旧之间的冲突,将会影响新千年的最初几年。我们预计,这将是一个充满巨大危险和回报的年代。在某些领域,文明的程度可能大大降低,而在其他方面,发展则是空前的。越来越有自治能力的个人,和破产的绝望的政府,将在一个新的文明分野中互相对峙。我们期待在这次转型结束之前,看到主权性质的彻底改组,和政治的实质性死亡。国家对资源的支配和控制将被取代,政府现在提供的几乎所有服务将注定被私有化。我们将会在书中探讨,由于无可避免的原因,信息技术将摧毁政府就其服务进行收费的能力;而这些服务,对你以及其他付费的人来说,原本就是物非所值的。


\subsection{通过市场实现主权}
仅仅在十年前,很少有人能够想象,面对领土上的民族国家,通过市场机制,个人能够实现越来越大的自主权。现在,所有的民族国家,都面临着破产和权威的迅速消解;尽管依然强大,但它们保有的是抹杀的权力,而不是令人言听计从的权威。它们的洲际导弹和航空母舰,已经变成了文物,就像封建时代的最后一匹战马,威风凛凛,但毫无用处。


通过改变资产创造和保护的方式,信息技术可以使市场急剧地扩张。这完全是革命性的。事实上,它对工业社会的革命力度,将超过火药对封建农业社会的。2000年的变革,意味着主权的商业化和政治的死亡;就像枪支意味着基于宣誓的封建主义的消亡。公民意识将走向骑士精神的道路。


我们相信,个人经济主权的时代即将到来。正如曾经被“国有化”的钢铁厂、电话公司、矿山和铁路,在世界各地被迅速私有化一样,很快你将看到私有化的终极形式——个人的全面非国有化。在新的千年,主权个人不再是国家的资产,不再是国库资产负债表上一个事实的项目。在 2000 年的转折之后,非国有化的公民,不再是我们所熟知的公民,而是政府的客户。


\section{带宽超越边界}
就像封建主义崩溃之后的骑士誓言一样,民族国家针对公民身份的各种条款和条件,在主权商业化之后将会过时。在与国家强权的关系中,21 世纪的主权个人不再是一个被征税的公民,而是在一种“新型逻辑空间”运作下的政府的客户。他们将为自己所需要的、最低限度的政府讨价还价,并按照合同付款。与几个世纪以来人们所习惯的不同,信息时代的政府,将通过新的原则组织起来。一些司法管辖区和主权服务,将建立在一种“同类匹配”的系统上;在这个系统中,亲密关系,包括商业好感,是虚拟管辖区获得忠诚的基础。新的主权国家,还有可能是中世纪组织的延续,虽然这种情况很罕见。例如有 900 年历史的耶路撒冷、罗德岛及马耳他圣约翰主权军事医院骑士团\footnote{耶路撒冷、罗得岛和马耳他圣约翰主权军事医院骑士团,简称马耳他骑士团,是联合国观察员实体,具有“准国家”性质,没有领土,是最为古老的天主教修道骑士会之一。 前身是成立于第一次十字军东征之后的天主教军事组织医院骑士团。著名的三大骑士团之一。医院骑士团的口号:“守卫信仰,援助苦难!”(Defence of the faith, assistance to the suffering!)},更广为人知的名字是马耳他骑士团。该组织是一个富裕的天主教团体,目前有 1 万名成员,年收入达几十亿。


马耳他骑士团发行自己的护照、邮票和货币,并且与 70 个国家\footnote{现在有 100 多个。}建立了全面的外交关系。在我们写这篇文章的时候,它正在与马耳他共和国谈判,以重新获得圣安杰洛堡\footnote{圣安杰洛城堡,又称为天使城堡,位于圣彼得大教堂东侧的台伯河岸,建筑历史超过130年。这座坚固的堡垒始建于1530年,当时医院骑士团刚刚入驻马耳他。在1565年的马耳他大围攻中,这里不仅是医院骑士团的总指挥部,也是大团长瓦莱塔的驻地。}的所有权。占有这个城堡,会使骑士团弥补他们缺少的领土要素,然后可以被承认为一个主权国家。马耳他骑士团有望再次成为一个微型的主权国家,它悠久的历史,使其能够很快被合法认可。在 1565 年的马耳他之围中,正是骑士团从安杰洛堡击退了土耳其人。事实上,在那之后,他们统治马耳他很多年,直到 1798 年被拿破仑驱逐出境。如果马耳他骑士团在未来几年内回归,那将是再清楚不过的一个证明:法国大革命之后迎来的现代民族国家体系,只是历史长河中的一段插曲。在人类历史上,多种主权形式同时共存才是一种常态。


铱星卫星电话网络\footnote{译注:今天的话就是马斯克的星链计划。},是另外一种特殊的后现代主权模式,它也是建立在“同类匹配”的基础之上的。乍一看,你可能会觉得,把蜂窝电话服务当作一种主权很奇怪;然而,铱星公司已经被国际普遍承认为一个虚拟的主权国家。铱星是一种全球移动电话服务,通过它,在地球上的任何地29方,从新西兰的费瑟斯顿到玻利维亚的查科,你都可以用一个独立的号码接听电话。考虑到全球电信的结构,为了使电话能够转接到世界各地的任何一个铱星用户,国际电信当局必须接受铱星是一个虚拟国家,有自己的国家通信代码,即8816。从一个由卫星电话用户组成的虚拟国家,到网络上跨国界的、连接更紧密的虚拟主权社区,从逻辑上说,是很短的一小步。自从晶体管被发明以来,带宽,或者说通信媒介的承载能力,相对于计算能力,一直在成倍地增长。按照设想,如果这个趋势延续下去,那么在千年之交后不久,带宽将足够强大,使“元宇宙”(mataverse)能在技术上得以实现。元宇宙是科幻小说家尼尔·斯蒂芬森\footnote{尼尔·斯蒂芬森(Neal Stephenson),1959年10月31日出生于美国马里兰州,美国赛博朋克科幻作家。其父为电子工程学教授,祖父为物理学教授。1981年获波士顿大学物理学与地理学学位,现居西雅图并兼任技术顾问。1984年发表首部小说《大学》,1992年创作《雪崩》提出"元宇宙"(Metaverse)概念。1996年以《钻石时代》获雨果奖,1999年出版密码学题材小说《编码宝典》。2003至2004年间完成展现科技史的《巴洛克记》三部曲,后续推出《失落的星阵》等硬核科幻作品。其作品融合历史学、语言学等多学科知识,1999年被《时代》周刊评为“50位数字英雄”之一。}在几年前构想出来的一个网络世界,它是一个密集的虚拟社区,有着自己的法律。


我们相信,随着网络经济不断繁荣,它的参与者必将会想方设法,去规避民族国家不合时宜的法律。新的网络社区,至少会像耶路撒冷、罗德岛及马耳他圣约翰主权军事医院骑士团一样富有,并有能力争取自身利益。事实上,借助影响深远的通信和信息战技术,他们将比骑士团站得更稳。我们还将探讨其他一些分散型的主权模式。在这些模式中,小型团体可以租赁弱小民族国家的主权,经营他们自己的经济庇护所,就像今天被广泛许可的自由港和自由贸易区一样。


在未来,怎样描述主权个人之间,以及他们和残留的政府之间的关系,可能需要一套新的道德词汇。我们猜测,当这些描述新型关系的术语成为社会的焦点时,很多来自 20 世纪民族国家“公民”时代的人,会非常生气。国家的终结和“个人的非国有化”,将使一些曾经被热烈拥抱的理念,变得没有市场;如“法律面前人人平等”,因为这些理念的前提是即将被淘汰的权力关系。随着虚拟社区的凝集力越来越强,他们将坚持依照自己社区的法律去承担责任,而不是某个前民族国家的法律,仅仅因为他们碰巧生活在那里。就像在古代和中世纪,同一地理区域内,多种法律体系将再次共存。


在火枪大炮面前,身穿盔甲的骑士妄图维护自己的权力,注定一败涂地;同样,现代的民族主义和公民概念,也必将被微处理技术所终结。它们会落得一个滑稽的下场,就像 15 世纪神圣的封建主义原则,在 16 世纪沦为笑柄一样。在转入2000 年之后,20 世纪所珍视的公民观念,对新的几代人来说,搞笑且不合时宜。21 世纪的唐吉坷德,不是为复兴封建主义而战斗的骑士,而是一个穿着棕色西装的官僚,一个满脑子想着审计公民的税吏。


\section{边区法律的复兴}
除了在最广泛的意义上,我们很少把政府看作是一种竞争性的实体;因而,现代人对主权的范围及其可能性的直觉已经萎缩了。在过去,权力往往是分散的,管辖是重叠的,不同类型的实体行使着主权的一种或多种属性;在这种权力等式中,很难有某个集团能稳定地保持垄断地位。名誉上的最高统治者,在下面并没有多少权力,这种情况在历史上并不少见。现在,比民族国家弱小的政府,它们在地方施加权力的垄断地位,就面临着持续的竞争。这些竞争,曾经改变了控制暴力和吸引效忠的形式,而新的改变很快就会出现。


当领主和国王们的势力单薄,往往就会出现一种现象:对同一块边境地区,有一个或多个团体主张权力,而任何一方都无法占据决定性的支配地位。在中世纪,有很多的边疆或“边区”(March)。在这些地方,主权重叠,暴力丛生。边区在欧洲存在了几十年甚至几个世纪,广泛存在于凯尔特人和英特兰人控制的爱尔兰地区之间,在威尔士和英格兰、苏格兰和英格兰、意大利和法国、法国和西班牙、德国和中欧的斯拉夫人边境之间,以及在西班牙的基督教王国和格拉纳达的伊斯兰王国之间。边区形成了独特的制度和法律,在下一个千年,我们很可能会重温它们。在边区,由于存在两个相互竞争的当局,住在这里的人很少交税。更重要的是,他们往往可以选择遵循谁的法律,通过“宣誓”或“封租”等法律方式。这些法律概念和方式现在都不复存在了;我们认为,它们将会成为信息社会法律的明显特征。


\subsection{超越国籍}
在民族国家之前,要历数存在于世的主权数量,是很困难的,因为行使权力的组织多种多样,错综复杂。这种情形将会再次发生。在民族国家的体系中,领土之间的分界线往往被划定得非常清晰,并固定为边界。到了信息时代,边界将会再次模糊不清;在新的千年里,主权将再次被分割;新的实体将会出现,它们将会履行部分被我们认为专属于政府的职能。


其中一些新的组织,如圣殿骑士团和中世纪的其他宗教军团,可能会掌握庞大的财富和军事力量,但并不控制任何固定的领土。它们的组织原则,将与国籍完全无关。中世纪宗教团体的成员和领袖,他们在欧洲各地的统治权威,无论在何种意义上,都绝非来源于国籍身份。他们具有不同的种族背景,宣誓忠实于上帝,而不是同一种族成员间的亲缘关系。


\subsection{赛博空间的商业共和国}
你还会看到,具有半主权地位的商人和富人的协会将再次出现,例如中世纪的汉萨同盟\footnote{汉萨同盟是德意志北部城市之间形成的商业、政治联盟。汉萨(Hanse)一词,德文意为“公所”或者“会馆”。13世纪逐渐形成,14世纪达到兴盛,加盟城市最多达到160个。1367年成立以吕贝克城为首的领导机构,有汉堡、科隆、不来梅等大城市的富商、贵族参加。拥有武装和金库。1370年战胜丹麦,订立《斯特拉尔松德条约》。同盟垄断波罗的海地区贸易,并在西起伦敦,东至诺夫哥罗德的沿海地区建立商站,实力雄厚。15世纪转衰,1669年解体。}。在法国和佛兰芒集市上经营的汉斯同盟,后来吸纳了六十多座城市的商人。汉萨同盟,在英语中是一个重复的命名,直译为“联盟式同盟”,由日耳曼商人行会组成,为其成员提供保护以及谈判贸易条约。在新的千年里,这种实体将重新崛起,代替苟延残喘的民族国家,在不安的世界提供保护,促进契约的履行。

简而言之,那些满脑子 20 世纪工业社会公民神话的人,对未来的期望会遭到挫败。其中包括对民主社会的幻想,这些幻想,曾经让天才和智者们都心驰神往。

它们假设的前提是,社会应该以政府希望的任何方式演进,当然,最好能根据民意调查和严格统计的选票。过去的 50 年,完美正确地体现了这一点。但现在,它已经不合时宜了,就像生锈的烟囱,是工业时代的产物。公民神话不仅反映出一种思维定式,认为社会问题可以通过工程方法解决;也表现出一种错误的信心,以为在未来仍然像在 20 世纪,资源和个体在政治强权面前脆弱不堪。我们对此表示怀疑。我们相信,市场的力量,而不是政治上的多数,将推动社会进行重新的自我配置,以某些公众舆论既不理解也不欢迎的方式。到那个时候,认为历史将向人们期望的方向发展的观点,将被证明是多么地天真和充满误导性。

因此,你必须重新看待这个世界,这至关重要。由外向内看进去,重新分析你认为是理所当然的东西,从而获得新的认识。当传统思维和现实脱节的时候,如果你不能超越传统思维,那将沦为迷失方向的猎物。迷失方向是未来的流行性疾病,它导致错误的决定,进而威胁你的事业、投资和生活方式。

\begin{tcolorbox}
\kaishu 宇宙,当我们了解它时,就会获得奖励;当我们不了解时,就会遭到惩罚。当我们理解宇宙的规律,就能心想事成,幸福快乐;反之,如果我们跳下悬崖,试图振臂飞翔,宇宙会令我们粉身碎骨。
\begin{flushright}
—— 杰克·科恩和伊恩·斯特沃特(JACK COHEN AND IAN STEWART)
\end{flushright}
\end{tcolorbox}

\subsection{打开新视野}

要对即将到来的世界做好准备,你必须理解,为什么它将与大多数专家告诉你的不一样。这需要仔细研究社会变革的隐形原因。为了实现这一目标,我们尝试了一种非正统的分析方法,称之为“大政治研究”(megapolitics)。在之前出版的两本书《血流成河》(Blood in the Street)以及《大清算》(the Great Reckoning)中,我们提出来,社会变革的最重要原因,不存在于政治宣言或者已故经济学家的声明里,而是隐藏在改变权力的运行边界的要素之中。通常来说,气候、地形、微生物和技术的微妙变化,会改变暴力的逻辑;它们变革了人们组织生计及自我保护的方式。


请注意,我们理解世界变化的途径,与大多数预测型专家的都大相径庭。我们不会假装,在某个特定的“主题”上,比那些在上面投入了整个职业生涯、积累了高度专业知识的人,懂得还要多。从这个意义上说,我们不是任何方面的专家;与此相反,我们是从外面看进去,堪称门外汉。我们是在做出预测的主题的“周边”进行研究,最重要的一点是,看到必然的边界划在哪里。当边界发生变化,社会必然随之改变,无论人们是否心甘情愿。


在我们看来,\uline{理解社会演进的关键,是抓住决定是否使用暴力的成本要素和回报要素}。人类社会的每一个阶段,从狩猎部落到庞大帝国,都是由大政治的各个因素相互作用所决定的,流行的说法就是“自然法则”。无论何时何地,生命都是复杂的。羔羊和狮子之间保持着微妙的平衡,在边缘展开互动。如果狮子突然更加迅猛,就能捕获现在抓不住的羊羔;如果羊羔突然长出翅膀,狮子就会饿死。

利用和抵御暴力的能力,是左右边缘地带生命体的关键变量。把暴力放到大政治理论的核心,是完全有必要的。对暴力的控制,是人类社会的每个阶段都面临的最大困境。正如我们在《大清算》中所写到的:

\begin{tcolorbox}
\kaishu 人们之所以诉诸暴力,是因为能够得到回报。可以说,一个人如果想要钱,最简单的办法就是拿别人的。一支军队夺取油田和一个暴徒抢夺钱包,完全是一回事。就像威廉·普莱费尔所说,权力总是寻求通往财富的最现成的道路,那就是攻击已经拥有财富的人。\\

繁荣富裕所面临的挑战,正是因为在某些情况下,掠夺性暴力能收获丰厚的回报。战争可以改变一切,它改变规则,改变资产和收入的分配,甚至生杀予夺。暴力确实好使,就是这一点,让它难以被遏制。

\end{tcolorbox}

从这些角度思考,帮助我们成功预测了一系列的事态发展;而这些发展,是消息灵通的专家们认为绝不可能发生的。例如,1987 年出版的《血流成河》,是我们研究目前正在进行的大政治革命的最初尝试;当时我们认为,技术的变革正在破坏世界权力的平衡和稳定。我们的主要观点如下:

\begin{itemize}
    \item 美国的优势正在下降,这将导致经济失衡和经济危机,可能再次发生 1929年式的股灾。这个观点遭到了专家们的一致否定。然后短短 6 个月内,1987年 10 月,全球股市迎来了本世纪内最猛烈的抛售和震荡。
    \item 我们告诉读者,等着看共产主义的崩溃吧。这一次,专家们又笑了。然而,1989 年,出人意料的事情发生了,柏林墙倒塌,从波罗的海到布加勒斯特\footnote{布加勒斯特(Bucharest),是罗马尼亚首都,有“小巴黎”之称。此处意指罗马尼亚1989年的七日革命,以齐奥塞斯库政权垮台及其夫妇被处决告终,为东欧剧变的重要组成部分。},共产主义政权被民主革命扫荡一空。
    \item 我们解释道,布尔什维克的领袖们从沙皇手里继承的多民族帝国,将会不可避免地分崩离析。1991 年 12 月底,锤子镰刀旗最后一次从克里姆林宫降下,苏联不复存在。
    \item 在里根政府进行军备竞赛的高峰期,我们指出,世界即将迎来全面的裁军。人们要么觉得荒谬,要么觉得不可能。然而,在随后的七年里,出现了一战结束后最大规模的裁军。
    \item 当北美和欧洲的专家都指着日本,相信政府可以成功地控制市场时,我们不以为然。我们预测,日本金融资产的繁荣将以破产而告终。柏林墙倒塌后不久,日本的股市崩溃,价值被腰斩。我们还认为,它最终的低点,可能相当于甚至超过,1929 年股灾华尔街遭受的 89\%的损失。
    \item 从中产家庭到全球最大的房地产投资者,当每个人都相信房地产只涨不跌的时候,我们警告说,房地产的崩溃即将到来。四年之内,由于房地产衰退,全世界的房产投资者损失超过 1 万亿美元。
    \item 早在专家们还没意识到的时候,我们就在《血流成河》中指出,蓝领工人的收入在下降,而且肯定将长期持续下降。在我们今天写这本书的时候,差不多十年后,沉睡的世界才终于开始意识到这是真的。美国的平均时薪已经低于艾森豪威尔第二任期时的水平。1993 年,按定值美元计算,美国年平均时薪为 18808 美元;1957 年,当艾森豪威尔第二次宣誓就职时,美国的年化平均时薪为 18903 美元。
\end{itemize}


虽然事后证明,《血流成河》里的预测惊人地准确,但是仅仅在几年前,它还被那些固守传统思维的人认为是无稽之谈。1987 年,《新闻周刊》的一位评论家,把我们的分析斥为是“对理性的不假思索的攻击”,反映出工业社会后期封闭的精神氛围。

你可能会觉得,随着时间的推移,像《新闻周刊》和类似的出版物,会认识到我们的分析路线的价值,看到它揭示出一些关于世界变化的真知灼见。根本没有。

和《血流成河》一样,《大清算》的第一版也遭到敌视,被嗤之以鼻。多家类似《华尔街日报》这样的权威媒体,断然否定我们的分析,认为它们就是“你的笨阿姨”在唠叨。随他们笑吧。事实证明,《大清算》中预测的,并不像传统的卫道士所假称的那么荒谬。

\begin{itemize}
    \item 我们延伸了对苏联灭亡的推论,探讨了俄罗斯及其他前苏国家面临的问题,如日益严重的内乱,恶性通货膨胀,以及生活水平的下降。
    \item 我们阐释了 1990 年代将是缩减规模的十年,其中就包括,政府及商业实体首次在全球范围缩减规模。
    \item 我们还预测,对收入再分配的条件要进行重大的调整,重新定义,福利水平将大幅降低。从加拿大到瑞典,都出现了财政危机的苗头,美国的政客们也开始谈论“结束我们所知的福利”。
    \item 我们预测并分析,“世界新秩序”将被证明是“世界新乱序”。早在波斯尼亚的暴行占据新闻头条之前,我们就警告,南斯拉夫将陷入内战。
    \item 在索马里陷入无政府状态之前,我们指出,非洲即将崩溃的政府,会导致那里的一些国家事实上被接管。
    \item 我们还预测并论述了,激进的伊斯兰教会取代马克思主义,成为与西方对抗的主要意识形态。在俄克拉荷马爆炸案\footnote{1995年4月19日上午9时4分,从美国俄克拉何马城的中心传来“轰”的一声巨响。被炸毁的是一座美国联邦办公大楼,通常情况下,大约有500名政府官员和职员在这里上班。这起爆炸案是2001年的911袭击事件发生前,美国本土所遭受最为严重的恐怖主义袭击事件,共计导致168人死亡,另有超过680人受伤。}和世贸中心被炸毁的几年前,我们就点明了,为什么美国面临的恐怖袭击将会激增。
    \item 在洛杉矶、多伦多和其他城市的骚乱登上新闻头条之前,我们就预见了,城市中少数族裔的犯罪亚文化,将导致暴力犯罪四处弥漫。
\end{itemize}

《大清算》还提出了一些有争议的论点,没有得到完全证实,或者说没有达到我们预计的发展水平。例如:- 我们认为,日本的股市会沿着 1929 年后华尔街的道路走下去,导致信贷崩溃和经济大萧条。不过,虽然西班牙、芬兰和其他一些国家的失业率超过了 30年代的水平,包括日本在内的一些国家也确实经历了局部的萧条,但是没有出现 30 年代全世界经济内爆的系统性信贷崩溃。

\begin{itemize}
    \item 我们曾预测,前苏联指挥控制系统的崩溃,会导致核武器扩散到一些小国家、恐怖组织和犯罪团伙手中。令人庆幸的是,这种情况并未发生,至少没有达到我们担心的程度。不过据新闻报道,伊朗在黑市上购买了几件战术核武器。
    \item 更令人担忧的是,《伦敦时报》于 1998 年 10 月 7 日报道:“根据一家主要的阿拉伯报纸的消息,流亡的沙特亿万富翁、恐怖主义领袖乌萨马·本·拉登,已经从前苏阵营的某中亚国家获得了战术核武器”。也就是说,这些从前苏流出的核武器,目前还没有正式确认部署或使用。
    \item 我们还论述了,“禁毒战争”会反过来颠覆警察和司法系统,特别是在毒品泛滥的国家,如美国。每年收获数百亿隐形的垄断利润,使得毒贩有能力也有动力,去腐蚀表明上很稳定廉洁的政府。虽然世界上的媒体偶尔会刊登一些报道,暗示毒资对美国政治体系高层的渗透,但这远不是故事的全部。
\end{itemize}

\subsection{见他人所不见}
尽管有一些错误的地方,或者根据现在已知的情况看是错误的,但我们的预测成绩甚是可观,经得起检验。1990 年代所发生的、未来的经济史要处理的很多课题,在《大清算》中都有预测或预计和解释。我们的预测,并不是对趋势的简单外推或延伸,而是认识到二战后被视为正常的东西已经发生了重大的背离。我们曾经警告过,1990 年代将与之间的 50 年迥然不同。翻阅 1991 年到 1998 年的新闻头条,不难发现,《大清算》的预测几乎每天都在得到证实。

我们所看到的,不是孤立的现象,或者一时一地的麻烦,而是来自同一条断层线的冲击和震荡。旧秩序正在一场大政治的地震上颠簸,而这场地震将颠覆原有的体制,改变有头脑的人看待世界的方式。

毫无疑问,暴力决定着世界的运行和变化,并在其中发挥着核心作用;但奇怪的是,很少有人认真思考这一点。在大多数政治分析家和经济学家看来,暴力好像只是一种轻微的刺激,就像在蛋糕周围嗡嗡作响的一只苍蝇,而不是烘焙它的厨师。


\subsection{另外一位研究大政治的先驱}
实际上,关于暴力在历史中的作用,有清晰思考的人少之又少,以致于关于大政治分析的著作,只需一张纸就能列完。在《大清算》中,我们借鉴了一本大政治分析的经典,并详细阐述了其中的论点。

这本书是威廉·普莱费尔\footnote{威廉·普莱费尔,1759年-1823年2月23日,英国苏格兰工程师、经济学家,数据可视化先驱,提出三种标准化数据可视化方法。代表作品包括《商业与政治图解集》《统计学摘要》。}(William Playfair)的《强盛国家衰落的永恒原因》\footnote{这是一部创作于19世纪初的历史研究著作。该书通过历史案例剖析国家由盛转衰的根本动因,着重探讨了从富强鼎盛走向荒芜衰败的深层机制。本书电子版:https://www.gutenberg.org/ebooks/16575}(An Enquiry into the Permanent Causes of the Decline and Fall of Powerful and Wealthy Nations),出版于1805年,早已被世人所遗忘。另外,弗里德里克·莱恩(Frederic C. Lane)的作品,是我们的出发点之一。莱恩是一位研究中世纪的历史学家,关于暴力在历史中的作用,他在20世纪四五十年代,写了几篇相关文章,洞若观火,清晰透彻。其中最全面的应该是《有组织暴力的经济后果》,发表在1958年的《经济史杂志》上\footnote{论文链接:https://www.jstor.org/stable/2114533}。除了少数专业的经济学家和历史学家之外,鲜有人阅读,大部分看过的人好像也没有意识到它的重要性。和普莱费尔一样,莱恩的著作,也是写给当时还不存在的读者的。


\subsection{对信息时代的洞察}
早在信息时代之前,莱恩就发表了他的作品,探讨暴力和战争的经济意义。他写这些东西,并不是提前预见到了微处理技术及其他技术革新;但是,他对暴力的洞察,为我们理解信息革命下的社会重构,建立了一个框架。


莱恩为未来打开的窗口,来自于他窥视过去的窗口。他主要研究中世纪的历史,特别是威尼斯,一个在暴力的世界中崛起又沉沦的贸易城邦。在思考威尼斯兴衰成败的过程中,莱恩注意到一些东西,可以帮助你认识未来。他看到了这样的事实:在决定“如何利用稀缺资源”的问题上,怎么组织和控制暴力,发挥着至关重要的作用。


我们相信,莱恩关于对暴力的竞争的分析,会对我们理解信息时代的生活变迁有诸多启示。但是,这么抽象的不时髦的观点,不要指望大多数人能注意到,能跟得上的人就更少了。世人的注意力,都盯在充满谎言的政治辩论和刚愎浮夸的名人身上,大政治的蜿蜒迂回,依然无人理会。


普通的北美人,可能会浪费百倍的时间在辛普森和莱文斯基\footnote{莫尼卡·莱文斯基(Monica Samille Lewinsky),女,1973年7月23日出生在美国加利福尼亚州一个富裕犹太家庭,前美国白宫实习生,美国前总统比尔·克林顿绯闻女友。}身上,相比于他们对微处理技术的关注。而正是这些技术,将淘汰他们的工作,颠覆他们赖以获得失业补助的政治制度。


\section{期望的幻觉}

并不是只有窝在沙发上看电视的人,才有这种忽视根本重点的习性。各式各样的专家并没有好到哪里去,他们看到的也只是民族国家的一个假象,认为是人们持有的观念决定了世界变化的方式。一些看上去老谋深算的分析家们,沉溺于各种解释和预测,把历史的重大变迁理解为是人们的期望推动的。就在我们写《再见,民族国家;你好……未知?》的时候,在《纽约时报》的社论版上,出现了一篇文章,很显著地反映了上述推理,作者是尼古拉斯·科尔切斯特。不只是这篇文章的主题——民族国家之死,也就是我们讨论的话题;更重要的是,科尔切斯特把他自己作为一个标杆,来反射我们的思维方式与常规之间有多么大的差距。科尔切斯特不是普通人,他也是《经济学人》情报部的编辑部主任。

如果有谁可以对这个世界形成一种现实的观点,那应该是他。而他在文章中多处明确指出,从逻辑上说,“世界政府的到来”势不可挡。

他为什么这么说呢?因为民族国家已经摇摇欲坠,没办法再控制经济的力量。在我们看来,这种假设近乎荒谬。仅仅因为一种治理方式失败了,就认为某种特定的新型治理方式必然会出现,这简直就是谬论。按照这种推理,海地和刚果早该有更好的政府了,因为他们之前的政府明显不行。


在北美和欧洲少数思考上述议题的人中间,科尔切斯特的观点得到广泛的认同。

但他完全没有考虑到,是更宽宏的大政治力量,决定了某种政治制度的实际可行性。本书的重点就在于此。把正在塑造新千年的技术纳入视野,我们看到的不是一个世界政府,而是微型政府,甚至更接近无政府的状态。

人人按规则行事,而暴力在决定规则上扮演着什么样的角色,就这一主题的严肃分析,我们看到了几十本云遮雾罩的书,里面写的都是小麦补贴;还有几百本神神道道的书,都在讲货币政策。对于真正决定历史进程的关键因素,这种思考上的不足,可能在很大程度上反映了,为什么过去几个世纪以来,权力的配置能保持相对的稳定。睡在河马背上的小鸟,不会想到失去栖息地,直到河马活动起来。


梦境、神话和幻想,在为所谓的社会科学提供信息方面发挥的作用,比我们通常认为的要大得多。

暴力的角色,在关于经济公正的大量文献中,表现得再明显不过了。关于经济的公正与不公正,已经有数以百万级的文字在述说、在描述,每一页都可以用来仔细分析,暴力如何塑造了社会,并设定了经济运行的边界。然而,现代语境下关于经济公正的表述,却预设了这样一个前提:社会是由一种强制性工具支配的,这种工具如此强大,它可以夺走生命中一切的美好,然后重新分配。而事实上,这种权力只存在于现代社会的几代人身上;现在,它正在衰退。


\subsection{社会保障的老大哥}
在 20 世纪,工业技术的发展,使政府掌握了前所未有的控制手段。有一段时间,政府对暴力的垄断越来越高效,留给个人自主的空间越来越小,看上去已经成为历史的必然趋势。在本世界的中叶,没有人会期望主权个人的胜出。

根据当时看到的证据,20 世纪中期一些最睿智的观察家确信,民族国家权力的集中,将会导致对人们生活方方面面的极权统治。在乔治·奥威尔的《1984》(1949年出版)中,老大哥看着每个个体徒劳地挣扎,想维持他们自主和自尊的余地。而这似乎是一场必败的抗争。弗里德里希·哈耶克的《通往奴役之路》(1944年出版),以更学术性的视角论述了,自由在被一种新的控制经济的方式所剥夺,国家正在成为一切的主宰。


这些作品,都是在微处理技术出现之前写的。而微处理,以及它孵化的一系列其他技术,提高了小型团队乃至个人独立于中央权威运作的能力。尽管哈耶克和奥威尔这样的观察家都目光如炬,但他们过于悲观。历史已经展开了它的惊喜。共产主义极权勉强撑过了 1984 年。如果政府能够成功压制微处理技术的解放力,下一个千年,还将出现一种新型的农奴制。但更有可能的是,我们将看到前所未有的机遇,以及个人的自主和自治。我们的父辈所担心的事情,也许将被证明根本不是问题;他们认为理所当然的、永恒不变的社会特征,现在看起来注定会消失。无论必然性给人类的选择设定的边界在哪里,我们都会做出相应的调整,然后重新组织生活。


\subsection{做预测的风险}
毫无疑问,就社会组织以及将组织紧密连接起来的文化,试图对它们深层次的变化做出预测和解释,会使我们小小的尊严受到威胁。人们做出的大多数预测,如果放到很长的时间范畴内,读起来都很蠢。而且他们设想的变化越剧烈,下场越难堪。世界没有终结。臭氧层没有消失。即将到来的冰河世纪,融化在全球变暖之中。与所有的警报相反,石油也没有枯竭。安特罗伯斯先生,《九死一生》\footnote{the Skin of our Teeth,一部关于人类生活三部曲的戏剧。}中的普通人,避开了冰冻,躲过了战争和经济崩溃的威胁,并且无视专家报告的警示,自然地老去。


大多数“揭示”未来的预言,很快都沦为了笑料。即使为了自己的面子,人们有很强的动机做出尽可能清晰的思考,但是,前瞻性的眼光也往往被证明是短视的。


1903 年,梅赛德斯公司说,“全球的汽车数量永远不会达到 100 万辆,原因很简单,全世界不可能有 100 万名技工,能够被训练成司机。”认识到这一点,我们本应该闭嘴的,但是没有。我们不害怕站在队伍中,迎接应得的嘲笑。如果我们大错特错,后人可以尽情嘲笑,假如有人记得的话。敢于表达,就甘冒犯错的风险。我们还没有僵硬无用到害怕犯错的地步,远远没有。我们宁愿冒险提出可能对你有用的想法,也不想因为事后回顾起来可能有些夸大或尴尬,而憋在心里不说出来。


正如阿瑟·克拉克(Arthur C. Clarke)睿智地指出,预测未来经常会失败,两个首要的原因是“缺乏勇气和缺乏想象力”。在这二者当中,他写到:“缺乏勇气似乎更加常见。有时候,即使给定了所有相关的事实,那些冒牌的预言家也看不出来,它们指向一个必然的结论。有些失败是如此滑稽,简直令人难以置信。”如果我们对信息革命的探索,与未来的现实在哪些方面不符,这当然不可避免,但原因更多是因为我们缺乏想象力,而不是缺乏勇气。预测未来始终是一个大胆的举动,因为它激发人们的怀疑。也许时间会证明,我们的推断谬之千里。但是,与诺斯特拉达穆斯不同,我们不会假装自己是预言家。我们不会在一碗水里搅动魔棒,或者使用占星术。我们也不会写神秘的诗句。我们的目的,是就某些对你至关重要的问题,提供一份清醒、独立的分析。


我们的观点可能被视为异端,正因为如此,我们感到有义务讲出来,不然的话,它们就没机会被听到。在工业社会晚期封闭的精神氛围中,思想已经不像它应有的那样,可以通过当前的媒体自由地传播。


本书是我们一起写的第三本,它出于建设性的精神,延续了《血流成河》与《大清算》,分析了正在发生的历史大变革的各个阶段。这是一场思维的练习。本书重点探讨工业社会必然死亡,社会将以新的形式重构。我们预期在未来的几年,会看到惊人的发展悖论。一方面,随着主权个人的崛起,我们将看到一种新式自由的实现,\uline{并有望见证生产力的彻底解放}。与此同时,我们期望看到现代民族国家的死亡。20 世纪成长起来的、被西方人视为理所当然的、诸多关于平等的保障,也势必随之烟消云散。我们预计,现在人们所熟知的代议制民主,将逐渐消失;取而代之的是来自网络市场上新的民主选择。如果我们推论正确的话,下个世纪的政治,将比我们已经习惯的政治更加多样化,但重要性会大大降低。


我们相信本书的论点并不难理解,尽管它穿越的领域类似知识上的荒郊野岭和穷街陋巷。如果书中有什么地方晦涩难懂的地方,那不是因为我们在耍机灵,或者在含糊其辞,像某些所谓的预言家发表的神秘声明。我们绝不会模棱两可。如果我们的论点不够清晰明确,那是由于我们写作能力的不足,没能把引人注目的观点表达得浅显易懂。和大多数预言家不同,我们希望读者能够理解甚至复制我们的思路。它不是出于灵异的遐想或者行星的回旋,而是基于老派的、不讨人喜欢的逻辑。正是基于逻辑,我们认为微处理技术必将颠覆和摧毁民族国家,同时创造出新型的社会组织形式。一种全新的生活方式正在到来,比你想象的更早更快;但你至少可以预见到其中的一些细节,这既有必要,也是完全可能的。


\subsection{预测未来的讽刺性}
几个世纪以来,第二个千年的结束,一直被视为一个重要的历史时刻。805 年前,圣·马拉奇就把 2000 年定为最后审判的日期。美国通灵师埃德加·卡伊斯在 1934年说,在 2000 年,地球的轴心将发生移动,加利福尼亚将裂为两半,纽约市和日本将没入大海。日本的火箭科学家板川秀夫在 1980 年宣布,1999 年 8 月 18日,众行星将排列成“大十字”,地球将遭受大范围的环境灾害,人类生命因此终结。


这些天启式的幻觉,后来遭到无尽的羞辱和嘲笑。公元 2000 年,虽然是一个气势磅礴的整数,但也只是西方采用的基督教历法的一个随意产物。其他文化中的历法和纪年系统,计算每个世纪和千年的起点与此不同。例如,按照伊斯兰历法,公元 2000 年就是 1378 年,不过是很普通的一年。按照中国历法,每 60 年一个轮回,公元 2000 年只是另外一个龙年;它是一个连续周期的一部分,向上延伸到过去的几千年。不过,人们对 2000 年的特别关注,不仅仅因为神学。这一年之所以被看重,除了基督教传统的加持,也受到本世纪中期信息技术局限性的影响。所谓的 Y2K 问题,也就是计算机的千年虫危机,它是潜藏在数十亿行代码中的逻辑缺陷,可以在千禧年的午夜关闭工业社会的基本元素,给人类世界带来毁灭性的打击,接近于末日状态。因为大量的计算机和微处理器使用的,是早期计算机保存和回收的软件。当时每兆字节的内存空间价格达到 60 万美元,比黄金还要昂贵。为了节省宝贵的空间,早期的程序员只用一年的最后两个数字来记录日期。这种两位数日期的惯例,被广泛应用到了大型计算机的软件当中,乃至大部分的个人计算机和嵌入式芯片。微处理器几乎可以控制一切,从录像机到汽车点火系统、安全系统、电话、控制电话网络的交换系统、工厂、发电厂、石油公司、化工厂、管道等的过程和控制系统等等。因此,1999 年缩写为两位数就是 "99"。问题是,当“00”出现在 2000 年的时候,很多计算机会把它读成 1900年。这可能导致大量没有调整过的计算机和其他数字设备,无法识别日期栏中的2000 年。


这种情况会引发大规模的数据破坏,它也意外启发了未来信息战的新潜力。在信息时代,潜在的对手可以通过引爆“逻辑炸弹”,破坏底层数据,进而瘫痪建立在该数据之上的关键系统,从而克敌制胜。比如,在一项军事演习中,如果你能攻击对飞机安全至关重要的数据,就不需要再击落它了。破坏数据,和使用物理武器一样,都能瘫痪现代社会的正常运行。认真思考一下,就很容易发觉,它潜在的影响极其深远。例如,1997 年 12 月 14 日《伦敦邮报》报道,由于担心航空运输控制系统会出事,全球的航空公司计划在 2000 年 1 月 1 日,取消数百次航班。据波音公司称,有大量飞机需要进行千年虫修复。很多设备如果在一个无效的日期记录一个事件,就会导致混乱。操作飞机的线性计算机控制系统,如果在编程运算中得出结论,飞机的最后一次关键维护是在 1900 年,就很可能会出现故障,甚至进入一个错误的死循环而崩溃。


逻辑炸弹会导致有缺陷的控制系统失灵,由此带来的致命反馈效应,可能会使很遗憾地使千年之交成为一个令人难忘的日子。要知道,即使你很幸运,没有在新千年开始的时候坐在半空中的航班里,你也会被很多进入错误循环而关闭的设备所影响。


我们建议你避免使用不符合 2000 年标准的心脏起搏器,或避开醉酒的千禧狂欢者可能引起的事故。如果心脏起搏器关停,电话系统可能也一样,所以救护车永远也不会来。如果你不是住在巴西或乌克兰,你也许习惯了拿起电话或打开车载电话,就能自动获得拨号音;所以,令人高兴的是,你不用关心电话系统运作的技术细节。但事实上,电话网络交换机和路由器是高度依赖日期的。所有通话的日期和时间,都会被记录在一个日志上,这对计费至关重要。如果在 1999 年 12月 31 日的 11:59:30 时,你打了一个一分钟的电话,在而 12 点整的时候,系统把你的通话时长计成了超过 99 年,从而发生错误和崩溃都是可能的。虽然长途电话公司正在花费巨资升级交换机,以使其符合 2000 年的要求,本地服务供应商大概也在做同样的事,但如果有哪怕一小部分公司未达到标准而停机,也会使整个网络受到影响。在 2000 年 1 月 1 日,如果你能获得一个拨号音,你就已经很幸运了。


用千年虫专家彼得·代·雅格(Peter de Jager)的话说,“如果我们不能打电话,那我们就什么都干不了。我们的电子专注,我们的交易,还有银行业务。”而千年虫故障的后果可能远不止这些。


今天还没有人知道,关键系统因千年虫问题而崩溃的话,波及面会有多大。1976年以后制造的汽车、卡车和公共汽车,其中的嵌入式系统都不能重新编程,如果它们对日期敏感不能运行,就必须更换。(也许你不会与戴着不符合要求的心脏起搏器的人,发生交通事故,因为他们的汽车可能也无法启动。)嵌入式系统还广泛存在于所有类型的发电厂、水厂和污水处理系统、医疗设备、军事设备、飞机、离岸石油平台、游轮、警报系统和电梯。虽然很多微处理器组件的功能对日期不敏感,但它们内部的操作可能依赖于一个时钟,而这个时钟受千年虫的影响。


\section{大型主机和 2000 年的定时炸弹}
Y2K 问题最初成为关注的焦点,主要是政府和大公司的大规模指挥控制系统,涉及到部署在大型计算机上的超高交易量。因为它们在大型机上运行,而大多数软件都是几十年前开发的,基本都不符合要求。所以,彼得·雅格在 20 世纪 90年代首次提出关于 Y2K 的警报时,主要就是针对需要升级的大型多处理器主机的操作系统。雅格先生担心,所有使用着脆弱系统的公司和政府机构,即使在几年前开始了崩溃修补计划,可能也找不到足够多精通 COBOL\footnote{COBOL(Common Business-Oriented Language)是一种面向过程的高级程序设计语言,主要用于数据处理,COBOL的起源可以追溯到1950年代中期,当时电子计算机开始用于商业和企业的事务处理。1959年,美国国防部召开会议,讨论建立通用商业语言的要求和可能性,最终确定了COBOL的基本设计思想。}的程序员,来完成对日期敏感代码的修补和替换。因为这些问题还没有爆发,而且很多系统的操作者才刚刚开始评估它们的脆弱性,所以,你可以很有把握地预测,大量的大型机系统无法顺利运行到 2000 年。


这无疑是个重大的问题,因为按照现在的经济机构,除了使用计算机处理,没有其他的选择。大多数企业,如果它们的规模大到需要主机来处理业务,那么它们的生存都依赖于交易量,这么大的交易量是 19 世纪的老式文书系统无法管理的。


如果这些企业不得不恢复到使用纸张,那它们只能完成正常交易量的一小部分。


这种程度的业务下跌对收入的冲击,将危及所有公司的生存,除非是资本化程度最高的公司。


几乎所有与钱相关的东西——发票、采购和工资系统,加上库存控制和执行标准——都会被破坏。大量的数据将会丢失,因为计算机遇到 Y2K 问题时,可能会崩溃或喷涌出虚假数据。在某些情况下,如果系统能够立即崩溃,而不是以复合的方式破坏数据,直到发生大规模的故障引起人们的注意,这反而是一件好事。


当一个备份工具把源自 1999 年 4 月 7 日的文件,复制到 2000 年 4 月 1 日的更新系统中时,会发生什么情况?谁也说不准。计算机会不会把一份“1900 年”\footnote{译注:其实为 2000 年。}1 月 4 日的保险单支付收据,理解为该保单已经违约一个世纪,从而使该保单被取消并从文档中被删除?对于那些跨越新千年的贷款,银行和财务公司的计算机是否会要求支付百年的利息?你的银行和经纪公司是否会保留你账户余额的准确记录,并及时为你提供资金证明?这些只能算是千年虫问题给你带来的一些有趣的窘境。


\begin{tcolorbox}
\kaishu 这可能是千年虫问题中最具破坏力的部分。这不是晚发几天工资造成的不方便,而是将会血流成河。
\begin{flushright}
—— 里昂·克柏曼博士(DR.Leon Kappelman)
\end{flushright}
\end{tcolorbox}

信息管理协会 2000 年工作小组联合主席在你的关注清单上,最重要的是如果千年虫故障导致电力中断会发生什么。没有电,即使大多数没有受到千年虫影响的系统也无法运行,如你的冰箱、冰柜,甚至是你的热源。千年虫的兼容问题,可能会影响到核电站安全的访问和控制功能。


例如,核设施的工作人员需要佩戴计量检测工具,测量他们在核电站中受到的辐射量。这些工具被定期分析,辐射量的数据会保存在计算机系统中,然后计算机控制着工作人员能够进入核设施。显然,如果计算机发生故障,那么这些精心设计的控制措施,原本旨在确保运行安全和正常维护的,将全部化为泡影。而更严重的是,核管理委员会的一份备忘录指出,许多“与安全无关的,但极为重要的计算机系统,主要是关于核电站的数据库和数据收集”,对日期很敏感。


传统型发电站在千年虫问题面前也非常脆弱。首先,以煤为动力的工厂很容易受到影响,因为把煤运送到锅炉的地面运输系统会出问题。在 1997-1998 年的冬季供暖季节,煤电运营商发现,在某些情况下会被迫减少产量,因为南太平洋和联合太平洋铁路系统合并后,西部煤炭的铁路运输变慢了。问题就在于两家铁路公司使用的计算机控制和调度系统之间不兼容。根据联合太平洋公司发言人的说法,尽管联合太平洋技术公司一直被认为是,开发计算机化运输控制系统的行业领导者,但整合这两个系统却成为了一场“噩梦”。由于编程上的困难,该铁路公司无法准确跟踪货运车辆的移动。联合太平洋公司不能顺利完成与南太平洋公司的合并,这是一个很坏的兆头,说明千年虫逻辑炸弹,对交通、电力和其他经济方面,可能造成什么样的危害。


然而,对电网最大的担忧在于,整个系统都受到敏感的监测和计算机控制,以便将电力从过剩的地区转移到不足的地区。这个过程必须由计算机仔细监控,防止电力浪涌和系统故障。所有电力传输都要记录时间和日期,用来确定供电时长,这点与电话很像。虽然用来连接的是重型的机械继电器,但它们是由计算机控制着。这些计算机对负载平衡至关重要,但它们可能会因为与电话网络相同的原因而失灵。实际上,北美的电力负荷分配控制系统,是通过 T-1 线路和电话微波链接联网的。因此,如果电话网络出现故障,你就知道,电力也会瘫痪。要记住,1998 年 1 月加拿大的经验表明,一旦大面积停电,要让系统重新运行是一个很大的挑战。所以,停电可能会持续很长时间,会很麻烦。


\section{千年虫与核武库}
对于现代经济体来说,寒冬时节停电会造成很大的破坏,并有可能威胁到健康,特别是对那些依赖电热源和医疗设备的人。但是,最坏的情况可能比这还要糟。


据克林顿总统的 Y2K 转换委员会的负责人约翰·科斯基宁说,美国的军事武器库可能会在 1999 年 12 月 31 日午夜时分停止运作。虽然科斯基宁表示他不希望引起不必要的恐慌,但他补充说,“有些事需要担心。”关于核导弹,需要担心的是,“如果数据出问题,它们真的会爆炸。”当然,这种担忧同样或更加适用于俄罗斯的核导弹。破产的俄罗斯,要把系统升级到符合 2000 的要求,比美国更难。而且有证据表明,俄罗斯还没有认真对待Y2K 的转换问题。虽然人们会祈祷不要发生意外发射,但毫无疑问,2000 年之交很可能会加剧全球的不稳定风险,原因很简单,就是很多国家的军事通信系统可能无法正常运作。就像科斯基宁所说:“如果你正在某个国家里坐着,突然间你搞不清到底发生了什么,而且你的通讯方式也都失灵了,你会更加紧张。”因此,这一点也应该列入你的千年虫顾虑清单。逻辑定时炸弹可能导致真正的炸弹被发射,这一事实,突出了信息战对中央指挥控制系统的威胁。


如果恐怖分子想攻击任何中央系统,他们很可能会选择在 1999 年 12 月 31 日行动,因为这是各种系统最脆弱的时候。不仅通信系统最紧张,还有可能出现电力故障,车辆无法启动,警察、救援和救护车、911 服务无法工作,等等。而且,很多你现在觉得是理所当然的功能,如空中交通管制,可能都会停止运转。没电意味着没水;污水处理系统会停工;交通灯会熄灭。在交通系统崩溃后的几个小时内,超市的食物会被买光(或抢光)。根据美国城市的最近经验,你可以认为,没有点,没有水,很多人没有暖气,没有灯光,以及与紧急服务相关的部分通讯,包括警察和救援,所有这些加起来就等于没有了人类文明。虽然没有人能确定Y2K 问题的影响究竟有多大,但它可能导致街头抢劫和暴乱,特别是如果人们得知,大范围的工资、福利和养老金都将无法发放。


\begin{tcolorbox}
\kaishu 我们不再是旧日的自己,而是开始成为新的样子。
\begin{flushright}
—— 菲奥雷的约阿希姆(JOACHIM DE FIORE)
\end{flushright}
\end{tcolorbox}


关于新千年的不详预感,并不一定完全来自于和基督教信仰相关的神学,但它确实落在菲奥雷的约阿希姆\footnote{生于意大利南部卡拉布里亚省,卒于皮耶特拉菲塔。曾参与十字军并到耶路撒冷朝圣,受到触动,成为一名西多会隐修士。1177年,被选举为西西里岛上凯拉佐修院的院长。1184年,受到路西乌斯三世教宗的召见,敦促其抓紧经注疏工作。1191年,放弃院长职务和责任,在深山中度隐修生活。}的千年传说之中。约阿希姆的沉思使他相信,基督只是“历史的第二中枢”,而另外一个中枢也注定要打开。哲学家迈克尔·格罗索(Michael Grosso)也持同样的看法,他认为,信息革命正在引导人类的历史,朝着实现西方世界预言的愿景前进;他称之为“科技启示”。不管技术的发展有没有受到千年愿景的影响,千年虫问题,都是西方主流关于时间想象的一个产物。它以一种怪异的方式,补充了人类关于末日的梦境、遐想和幻觉,或者是对幻觉的数字解释,就像牛顿对但以理预言的润色。


这些直觉上的跳跃,始于一种特定的视角,就是把基督的诞生看作是历史的中心。


还有超级整数\footnote{指公元 2000 年。}的心理暗示作用;所有的股票交易员都知道,超级整数有别具一格的影响。这些因素叠加在一起,使得第二个千年,必然成为拥有直觉的人们的想象焦点。


一个批评者很容易指出上述预言的愚蠢之处,他甚至都不用攻击天启和最后审判之类的神学概念,这些概念赋予了预言强大的力量,但是它们含糊不清、充满争议。有意思的是,计算机超越了千年虫问题,没有发生大规模的故障\footnote{译注:此处意指 2000年并没有那么可怕。}。不然的话,就是在基督教自身的框架内,也有证据可以降低 2000 年的重要性。2000 年被认为可能成为下一历史阶段的拐点,只是因为它把新千年的到来错误提前了。严格来说,下一个千年要到 2001 年才开始。


公元 2000 年只是 20 世纪的最后一年,是耶稣诞生后的第两千年。如果耶稣是诞生在基督教时代的第一年,那么也没错;但是他没有。公元 533 年,耶稣的诞辰取代了罗马建国的日子,成为西历纪年的基础。当时引入这套新惯例的僧侣,把耶稣的生日算错了。现在人们认为,耶稣实际上诞生于公元前 4 年;照这么算的话,他出生后的整 2000 年,应该是在公元 1997 年的某个时候。因此,卡尔·荣格为新时代的开始,设置了一个与众不同的日期。\footnote{在前文中,荣格预言新时代会在 1997 年到来。}


如果你觉得好笑的话,不妨一笑;不过,我们并不鄙视或者否定对历史的直觉。


虽然我们的论点建立在逻辑上,而不是遐想中;但是我们对人类意识的预言能力感到震惊。一次又一次,它救赎了疯子、灵媒或圣人的幻觉。2000 年的这次转变可能也会如此。这个长期固定在西方人想象中的日期,可能就是那个宿命的转折点。至少有一半的人确信,历史自有定数。我们无法解释为什么会这样,但我们相信它确实如此。


我们的直觉是,历史自有定数;而自由意志和决定论只是一体两面,同一现象的两个版本。人类的互动塑造了历史,但是他们的表现又好像宿命使然。就像电子等离子体,是一种电子浓度很高的气体,它表现出来的就是一个复杂的系统,人类也是这样。电子个体的自由运动,和高度组织化的集体行为是互相兼容的。就像大卫·博姆(David Bohm)对电子等离子体的描述,人类历史是“作为一个整体在行动的、高度组织化的系统”。


要理解世界的运作方式,就要对人类社会如何服从自然演进中的数学运算,形成一种现实的直觉。和大部分人所期望的不同,现实是非线性的。要理解变化的动力,你必须认识到,人类社会和自然界的其他复杂系统一样,具有周期性和不连续性。历史的某些特征有重复的趋势,但是那些最重要的变化,当它们发生的时候,往往是突然的而不是渐进的。


在人类生活的各种周期中,有一个神秘的五百年周期,好像标志着西方文明史上的重大转折。随着 2000 年的到来,我们被一件怪事所困扰,那就是每一个能被5 整除的世纪,它的最后十年都意味着西方文明的深刻转型。就像生和死划定了人类的世代,这种衰亡和重生的模式;代表着社会组织的阶段更新。至少从公元前 500 年开始,就是如此。公元前 508 年,随着克里斯提尼的宪法改革,希腊的民主制度登上历史舞台。随后的 5 个世纪,是古代经济增长和强化的时期,到公元前 4 年,耶稣诞生的时候,达到了顶点。


在接下来的 500 年里,繁荣衰退,导致罗马帝国在公元 5 世纪末崩溃。威廉·普莱费尔对此的总结值得重温,“罗马最伟大的时刻……当属基督降生时;而同时它也开始逐渐衰落,直到 490 年。”就在那一年,最后一个罗马军团被解散,西方世界陷入了黑暗时代。


随后的五个世纪里,经济萎缩,长途贸易停滞不前,城市人口下降,流通中的货币不断减少,艺术和文化几乎完全消失。西罗马帝国的崩溃,使曾经可以有效解决争端的法律,逐渐被更原始的方式所取代。报血仇的现象,从五世纪末开始明显增多。而第一次有记录的神明审判,就刚好发生在公元 500 年。


历史来到一千年前,第十世纪的最后十年,又一次见证了“社会和经济体系的巨大动荡”。在这些年发生的变革中,最不为当时人们所知的,当属封建革命,它萌芽于一个经济和政治完全混乱的时期。巴黎大学的中世纪史教授居伊·布瓦(Guy Bois),在他的《第一个千年的变革》(The Transformation of the Year One Thousand)中写到,十世纪末的这次大断裂,使古代制度的残余彻底崩盘,然后在无政府的状态中,催生出了一种新事物,那就是封建主义。用劳尔·格拉伯(Raoul Glaber)的话说,“有人说是整个世界全心一致地抖落掉了古代的破烂。”突然出现的新制度,适应了经济的缓慢复苏。在被我们称为中世纪的500年里,货币和国际贸易得到了重生,伴随着算术、识字和时间意识的回归。


然后,在15世纪的最后十年,又一个转折点出现了。当时的欧洲,刚刚摆脱黑死病造成的人口赤字,就立即开启了对全球其他地区的征服。这次的转型,将人类社会带入了现代时期;“火药革命”、“文艺复兴”和“宗教改革”等名称,代表了它的不同侧面。查理八世带着全新的青铜大炮,入侵意大利,以一声巨响,宣布了新时代的开幕。这个时代的变革,涉及到一个开放的世界,1492年哥伦布发现美洲就是它的缩影。对新世界的探索,推动了人类历史上最猛烈的经济增长。


它还涉及到物理学和天文学的革命,催生了现代科学。还有印刷术等新技术的涌现,促进了时代思想的广泛传播。


如今,我们正坐在另外一次千年变革的门槛之上。从工业时代继承下来的大型指挥和控制系统,在千禧年的午夜时分,可能会像古代的马车一样分崩离析。工业社会已经时日无多,不管它是否会在千年虫的逻辑炸弹下立刻崩溃。信息社会的到来将彻底改变世界,这就是本书的主旨。你完全有权利怀疑这一点;毕竟,在一个千年里只重复两次的周期,还没有足够多的迭代数据,使它在统计学上显得令人信服。实际上,即使更短的周期,也会被经济学家们怀疑,要求提供更多能够满足统计学标准的数据。丹尼斯·罗伯逊(Dennis Robertson)教授曾经写道,要确定四年或八到十年贸易周期的存在,“我们最好等上几个世纪。”按照这个标准,罗伯逊教授需要大约三万年,才能确定五百年的周期,是不是统计学上的一种偶然。我们没那么教条,或者说我们更乐意捕捉历史的暗示。我们认识到,现实的发展模式,比大多数经济学家的静态和线性平衡模型,要复杂得多。


我们相信,公元2000年,不只是连绵不绝的时间进程中一个明显的分割点,而是旧世界和即将到来的新世界之间的一个拐点。工业社会正快速成为过去,具有讽刺意味的是,它的消亡可能会因为早期计算机昂贵的内存而加速,它迫使人们广泛采用了两位数的日期设置。当Hallerith牌打卡机只能容纳80个字符时,缩写日期是一个合理的选择。但是,让早期程序员没有想到的是,缩写日期成为了意外的逻辑炸弹,埋伏了40年,它对工业社会的威胁持续到千禧年的结束。美国政府预算管理委员会,专门针对这个问题,在1997年2月7日发布了《让联邦计算机为2000年做好准备》的报告。委员会的结论是,这些计算机“除非被修复或替换,否则在世纪之交,它们可能会因三种情况而报废:拒绝运行正常条目;计算错误;或者根本不工作。” 这三种结果的结合,会使工业社会无法正常运转。


不管怎样,大规模生产的技术,将注定被一种小型化的新技术所取代,短期内的危机只会加快这一进程。信息技术日新月异,催生出一门新的科学,非线性动力学;它已经发展出来的惊人结论,还只是一些线索,尚未编织成一套系统的世界观。虽然生活在计算机时代,但我们的梦想还在织布机上旋转。我们还依照着工业社会的隐喻和观念在生活,还没有从奇怪吸引子\footnote{strange attractor,计算机和物理学名词。}的角度去想象世界。我们的政治还横跨在左翼和右翼的工业鸿沟之上,而绘制这些意识形态的思想家,如亚当·斯密和马克思,在今天活着的人出生之前就已经去世了。在我们看来,随着世界的转变,工业时代的“常识”,在很多领域都没有用了。


奥斯瓦尔德·斯宾格勒(Oswald Spengler),在 1911 年,对即将到来的世界大战和“西方的没落”产生了一种强烈的直觉。85 后的今天,我们也预感到“一个阶段性的历史变革正在发生……而且在几百年前就已经注定了\footnote{斯宾格勒语。}。”和斯宾格勒一样,我们看到西方文明正在衰亡;与之相伴的是,自哥伦布大航海与新世界建立联系以来,主导了过去五个世纪的世界秩序也将崩溃。但与斯宾格勒不同的是,我们相信,在未来的千年里,西方文明将迎来一个崭新的阶段。