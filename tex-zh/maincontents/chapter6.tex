\chapter[信息时代的大政治分析]{信息时代的大政治分析:\\ 效率胜过权力}

\begin{tcolorbox}
美国的经济,将越来越多地由计算机化的信息驱动,而不再是人力或大规模的生产,它将在这个拥有 500 台电视频道的世界里赢得战争。计算机化的信息存在于赛博空间——一个由计算机网络、通信卫星、调制解调器、数据库和公共互联网不停复制所创造出来的时空新维度。
\begin{flushright}
—— 尼尔·芒罗(NEIL MUNRO)
\end{flushright}
\end{tcolorbox}

1936 年 12 月 30 日,希望得到更高薪水的汽车工人,强占了通用汽车在密歇根州弗林特的两家主要的工厂。他们让机器空转,关停了装配线,把工厂当成了家,安营扎寨,进行了持续数周的工业对抗。这场戏剧被暴力骚乱以及警察、密歇根民兵和各级政府官员摇摆的忠诚度不断地打断。后来,工会看到他们的要求推进甚微,于 1937 年 2 月 1 日再次罢工;工会的活跃分子强行接管了通用汽车在弗林特的雪佛兰工厂。通过占领和关闭主要工厂,工人们有效瘫痪了通用汽车的生产能力。在第三家工厂被占领后的 10 天里,通用在全美只生产了 153 辆汽车。

60 年后,我们重温这则新闻短讯,是为了更清晰地认识当前正在发生的大政治革命。通用汽车的占领罢工,就发生在本书一些读者的有生之年。然而我们相信,在信息时代,这种罢工将是不合时宜的,它无疑于奴隶在沙漠中拖着巨石为法老建设墓葬金字塔。工会和他们的威胁策略是工业时期的常态,甚至成为了社会景观中无可非议的一部分,但它们依赖的特殊的大政治条件正在飞快地逝去。在信息时代的高速公路上,不会有雪佛兰,也不会有美国工会的罢工。

政府的命运将随着它们对手的命运一起衰落,也就是工会。在 20 世纪发挥关键作用的那种制度化胁迫,将不再可能发生。技术正在加速改变敲诈勒索与保护的根本逻辑。

\begin{tcolorbox}
不存在什么财产,不存在什么支配,也没有我的和你的之分;每个人能得到的就是他的,只要他能保得住。
\begin{flushright}
—— 托马斯·霍布斯(THOMAS HOBBES)    
\end{flushright} 
\end{tcolorbox}


\subsection{勒索与保护}
纵观人类历史,暴力一直是插向经济心脏的一把匕首。托马斯·谢林(ThomasSchelling)曾睿智地指出:“伤害的力量——摧毁他人珍爱之物,制造悲伤与痛苦——是一种讨价还价的力量,它不容易被使用,但又经常被使用。在黑社会,它促使人们敲诈、勒索和绑架;在商界,它促使人们抵制、罢工和停工。它是纪律的基础,无论在民事上还是在军事上,连神明也用它来执行纪律。”一个政府的征税能力,最终取决于跟私人敲诈勒索相类似的能力强弱。尽管我们很少从这个角度去思考,但是通过犯罪或政府来控制和使用资产的比例,为我们认识勒索与保护之间的大政治平衡,提供了一个粗略的衡量标准。如果技术发展使保护资产变得更困难,犯罪就会更普遍,工会的活动也一样;那么政府的保护就变得更重要,税收将很高。而那些税率较低、工资由市场的力量决定而不受政治干涉或胁迫的地方,就是技术已经使天平向保护的方向倾斜了。

在二十世纪最后 25 年开始的时候,勒索与保护在技术上的不平衡达到了一个极端。在一些先进的西方国家,政府霸占了多数的资源。很大一部分人的收入,要么是由政府规定的,要么是在胁迫的力量下达成的,例如通过罢工或其他形式的暴力威胁。福利国家和工会都是技术的产物,它们分享了 20 世纪“权力胜于效率”的战利品。如果没有军用和民用的技术发展,它们就不可能存在;正是这些技术提高了工业时代的暴力回报率。

在敲诈勒索面前,创造财富的能力总是显得很脆弱。你创造或拥有的资产越多,以这种或那种方式支付的保护费就越高。要么你买通所有获得暴力敲诈杠杆的人,要么你付费购买能运用暴力击退所有敲诈勒索者的军事力量。


\begin{tcolorbox}
你地上不再听见强暴的事,境内不再听见荒凉毁灭的事……
\begin{flushright}
—— 以赛亚书 60:18
\end{flushright}
\end{tcolorbox}

\subsection{财产保护的数学运算}
现在,暴力的匕首很快将不再锋利。信息技术有望极大改变保护与勒索之间的平衡,在很多情况下,财产保护将更加容易,而敲诈勒索则更加困难。信息时代的技术,有可能创造出强制手段触及不到的资产。这将在保护与勒索之间形成一种新的不对称性,而它就建立在一个基本的数学原理之上:做乘法比作除法更容易。

不过,在微处理器出现之前,这个基本原理的深远影响并未被揭示出来。在过去十年中,高速计算机所完成的计算量,是人类此前历史时期所有计算量的几十亿倍。计算能力的这一飞跃,使我们第一次能够探索复杂性的某些普遍特征。计算机揭示出来,复杂的系统只能自下而上被建立和理解。质数乘法很简单,但试图通过分解大质数的积去解析复杂性,则几乎是不可能的。“连线”杂志的编辑凯文·凯利这样说,“把几个质数相乘得出它的积很容易,任何小学生都可以做出来;但要把一个乘积分解成简单的质数,却会难倒世界上的超级计算机。”

\subsection{复杂系统的逻辑}
这一深刻的数学真理,必将主导网络经济的塑造。它已经在强大的加密算法中得到了明显的体现。在本章稍后的内容中,我们将会探讨,通过这种算法,可以创建一个全新的、受到保护的网络商业领域,暴力的杠杆作用在这里将被大大降低。

勒索与保护之间的天平,将会迅速地朝着保护的方向倾斜。新的经济模式将会出现,它将更多地依赖自发的适应机制,而更少依赖有意识的决策以及官僚机构对资源的配置。以保护为首的新体系,将与工业时期以强制为主导的旧体系截然不同。

\subsection{指挥控制系统将变得原始}
我们在《大清算》中指出,计算机正在使我们“看到”以前看不到的、各种系统内在的复杂性。先进的计算能力,不仅使我们可以更好地理解复杂系统的动力学,也使我们可以掌控复杂性,用于生产性的活动。从某种意义上说,人类经济的发展超越由中央控制的僵化阶段,不是一个选择,而是一种必然。依赖于线性关系的系统,从根本上说是原始的。政府拨款不可避免地将资源从高价值的复杂应用,拖入到低价值的原始用途。这是一个受到数学不对称性限制的过程,同样的不对称性阻碍了大质数乘积的分解。分赃永远只能是原始的。

\subsection{一切将更加复杂}
当你眺望浩瀚的宇宙,目光所及之处,你都能看到系统随着演进往往变得更加复杂。在天体物理学中如此,在水坑中也是如此。让雨水留在低洼处,它就会变得更复杂。每一种先进的系统都是复杂的适应性系统,没有中央权威在控制。自然界中所有的复杂系统,都依赖其分散的能力;人类社会中最明显的表现就是市场经济。能够在最宽泛的条件下最有效地运作的系统,靠的是它自发构建秩序去适应新的可能性的能力。生命本身就是这样复杂的系统,数以亿计潜在的基因组合形成了一个人类个体。要对它们进行分类,所有的官僚结构都会茫然无措。

二十五年之前,这只能是一种直觉。今天,它是可以被证明的。借助于计算机,我们对人工生命的数学了解越来越深,从而也可以更好地理解真实生命的数学,也就是生物的复杂性。当信息技术驾驭了复杂性的奥秘,我们的经济就可以被配置为更复杂的新形式。正如凯文·凯利(Kevin Kelly)在《失控》(Out of Control)一书中所提出的,互联网和万维网已经具有了有机系统的特征。用他的话说,大自然是“一个思想的工厂。壮丽的后工业化的蓝图就藏在丛林中的蚁群里……将生物逻辑全面注入到机器之中,会带来让我们充满敬畏的结果。当人造的与天生的完全融合,我们的造物将拥有学习、适应、自愈,甚至是进化的能力。这是我们难以想象的力量。”“将生物逻辑全面注入到机器之中”,它的影响必将是深远的。因为人类的社会系统总是会模仿主流技术的特征。在这一点上,马克思说得很对。超级工厂与大政府出现在同一个时代;而微处理技术正在使机构小型化。按照我们的分析,信息时代的技术发展,最终将创造出更擅于利用复杂性优势的经济模式。

然而,对于这种改变在大政治层面的影响,人们还知之甚少;大多数认识到其数学重要性的人,对它的理解也是不合时宜的。在未来的数年内,技术变革将使现代世界的大部分政治形式和概念变得过时;而要完全理解和内化这种可能性,绝非易事。已故物理学家海因茨·帕格尔斯(Heinz Pagels)在他极富远见的著作《理性之梦》(the Dreams of Reason)中写道:“我确信,能够掌握新的复杂性科学的国家和人民,将成为下个世纪经济、文化和政治的超级大国。”这一预测,令人印象深刻;但我们认为未来肯定不是这样。不是帕格尔斯博士看错了,而是未来比他所说的还要走得更远。那些能够自我重构以更好地适应复杂性的社会系统,确实会繁荣昌盛。但这样的系统不太可能是国家,更不可能是“超级大国”。

在新的千年里,社会系统复杂性增加的直接受益者,更有可能是主权个人。

帕格尔斯的预测,相当于在 500 代人以前,一个狩猎部落的萨满在他们围在一起烤火的时候,对他部落的人说:“我确信,第一个掌握新的灌溉种植技术的部落,比起那些在湖面抓鱼的家伙,将会有更多的空闲时间讲故事。”关于复杂的重要性,帕格尔斯是对的,但他忽略了一个最基本的事实:当暴力的逻辑发生变化,社会必将随之改变。

\section{暴力的逻辑}
要知其然并知其所以然,就必须关注几个少有人知的大政治问题。历史学家弗雷德里克·莱恩(Frederic Lane)对这些问题进行过专门的研究,他的著作揭示了暴力与战争的经济意义,在本书中多次我们曾多次论及。莱恩在本世纪中期写下这些作品时,信息时代还不见影踪。在这种情况下,他认为,随着民族国家的出现,人类对暴力的竞争已经达到了最后的阶段。在他的作品中没有任何迹象表明,他预见到了微处理技术的出现,或者在一个没有物理存在的网络空间创造财富是可能的。对于大规模的商业可能完全不再受制于暴力的深远影响,莱恩完全没有提到过。

虽然莱恩没有预见到今天的技术变革,但是他关于各个历史阶段的暴力垄断的洞察,却是无比清晰的,完全可以适用于当前的信息革命。在对充满暴力的中世纪进行研究的过程中,莱恩被一些传统历史学家和经济学家忽略的问题所吸引。他看到,对暴力的组织和控制在决定“如何利用稀缺资源”的问题上,起着关键的作用。莱恩还认识到,虽然暴力所得通常不被认为是经济产出的一部分,但对暴力的控制,对于经济发展确实至关重要的。政府的主要作用就是提供免受暴力侵扰的保护。他说:

\begin{tcolorbox}
每个经济组织都需要保护,并要为此而支付费用,以免其资本被破坏或遭到武装抢夺,或者其生产劳动被强行中断。在高度组织化的社会中,提供这种保护效用,是一个被称为政府的特殊协会或组织的职能之一。事实上,政府最特殊的一点就是,通过自己使用暴力或其他手段来控制他人对暴力的运用,从而创造出法律与秩序。
\end{tcolorbox}

也许是过于显而易见,莱恩的观点并没有出现在教科书中,也没有成为公民讨论的一部分,这些讨论有望影响到政治的走向。但是,虽然这一点很基本,在你了解正在发生的信息革命时,却不能忽视它。保护生命与财产是人类的根本需要,这个问题困扰着每一个社会阶段。如何抵御暴力的侵略,是历史上的一个核心难题;尽管可以通过多种方式进行保护,但要解决这个问题并不容易。

\subsection{一个时代的结束}
在我们写这本书的时候,信息时代的大政治后果才刚刚显露出来。最近几十年主要的经济变化,就是从以往的制造业占首位,到今天以信息和计算机为主导;从机器动力到微处理器;从工厂到工作站;从大规模生产到小型团队,甚至个人独立工作。企业的规模越来越小,遭到破坏和勒索的可能性也随之降低;工会要把小规模的企业组织起来会更加困难。

微技术使企业变得小巧、自由、不受束缚。很多企业经营的服务或产品,其自然资源的含量低到可以忽略不计。所以,原则上,这些企业可以在地球上的任何地方运营,不必再困于某个特定的场所,如矿场或港口。而随着时间的推移,要对它们进行征税将越来越难,不管是工会还是政府。正如中国古老的民间智慧所言,“三十六计,走为上计。”在信息时代,这种东方智慧将很容易得到实践。一家企业在任何一个地方经营,如果被索求过多而感觉不舒服,搬走是很容易的。甚至,就像我们在下文中预测的,信息时代将会出现虚拟公司,它在任何国家的住所将完全依现货市场而定。

如果来自于政府或其他方面的敲诈急剧增加,虚拟公司的业务和资产就会以光速撤出该管辖区。

微处理技术与工业操作的日益融合,使那些仍然在以规模经济经营制成品的企业,也不像以前那样容易受到暴力的影响了。美国汽车工人联合会对卡特彼勒罢工的失败,就很好地说明了这一点。这场罢工持续了将近两年,在 1995 年的最后几天放弃了。与 1930 年代的装配线不同,今天的卡特彼勒雇佣了更多的技术工人。在海外竞争的压力下,卡特彼勒把很多低技能的工作外包了出去,关闭了效率低下的工厂,还花费 20 亿美元将机床电脑化,并安装了组装机器人。罢工本身甚至促进了精简劳动力的效率。该公司宣称,现在需要的员工比罢工之初还少了两千人。

生产过程中已经发生的大政治变化,其剧烈程度远超多数人的认知。这种变化之所以还没有清楚地显示出来,部分原因在于,在大政治环境的革命与它必将导致的制度变革之间,总是存在一个滞后期。更进一步而言,微处理技术的快速发展,意味着那些即将诞生的产品可能导致的大政治影响,在它们出现之前就可以被预见。它们将带来一个完全不同的世界。

\section{工人对资本家的剥削}
在 20 世纪的大部分时间里,由于技术的特点,强占工厂或者静坐罢工的策略,是业主或管理者们很难应对的。历史学家罗伯特·麦克埃文(Robert S.McElvaine)认为,“雇主很难在不破坏自己工厂设备的情况下”解决静坐罢工。这其实相当于,工人用他们的身体挟持了业主的资本,以此索要赎金。相比小公司,大型工业集团更容易成为工会盘剥的目标,其中原因我们会在后文进行讨论。在 1937年,通用汽车可能是世界上最先进的工业公司。它的工厂是有史以来最庞大、最昂贵的机械集合体,雇佣了数以万计的员工。通用工厂被迫关停的每一小时、每一天,都会让公司损失掉一笔财富。像 1936 年和 1937 年之交的那个冬季,持续数周的罢工,意味着损失的大幅增加。

\subsection{违背供求关系}
在第三家工厂被强占以后,通用公司无法再生产汽车,于是很快就向工会屈服了。

这不是一个基于劳动力供求关系的经济规律做出的决定。在通用同意工会的条件时,美国有 900 万人失业,占劳动力人口的 14\%。大多数失业的人如果能进通用上班,都会很开心。他们当然具备填充装配线岗位的技能;不过,你很难在诸多的当代描述中看到这一点。因为好像有一种微妙的礼节,笼罩着对工业时期劳资关系的直接分析。其中一套托词是:工厂的工作都是需要技能的熟练工,特别是在 20 世纪中期。这是不对的。当时大部分工厂的工作,只要能准时上岗的工人基本都可以完成;几乎不需要培训,甚至也不需要读写能力。即使到了 20 世纪80 年代,通用汽车的大部分员工,还要么不识字,要么不识数,要么两种都不识。到 20 世纪 90 年代,通用装配线上典型的工人,在上岗之前只需接受一天的训练。一天就能上手的工作,可不能算是技术工。

然而,1937 年,在非技术工和技术工都在排队乞讨的情况下,通用的员工却可以迫使雇主加薪。他们的成功,更多是基于暴力动力学,而与劳动力的供求关系无关。1937 年 3 月,在通用罢工事件解决后的一个月,美国又发生了 170 起占厂罢工;大部分都获得了成功。类似的时间蔓延到了每个工业化国家。工人只需要占领工厂,然后让业主赎回去。这种策略很简单,而且对参与的人来说,往往既有利可图,又充满乐趣。一位罢工者写到:“我玩得很开心,看到很多新奇的、没见过的东西,这里还有丰富的食物和音乐。”1936-37 年的通用汽车留厂罢工和当时其他强占工厂的行为,都是我们在《血流成河》中描述的“工人剥削资本家”的例子。这当然不是皮特·西格(Pete Seeger民谣歌手)在他的伤感民谣中植入的观点。但是,除非你打算在蓝领社区从事民谣歌手的职业,否则你要关注的重点,就不应该是流行的解释,而应该是潜在的现实。无论你看哪里的历史,通常都有一层合理化和虚构的东西,掩盖了系统性勒索的真正的大政治基础。如果相信这些经过合理化的表面说辞,你就不可能掌握真相和真理。

\section{破译勒索的逻辑}
要认识到这场迈向信息时代的转型的大政治影响,你必须剥去那些伪善的说教,关注社会暴力的真正逻辑。这就像剥开一个熟透的洋葱,可能会让你泪流不止,但你不能转过头去。接下来,我们会首先研究工作场所中的敲诈逻辑,然后将其扩展到更宏观的层面,分析财富的创造与保护、现代政府的性质等问题。超出大部分人的想象,政府的繁荣,在很大程度上与工会一样,直接取决于它能用来进行敲诈的杠杆的强弱。这些杠杆的作用,在 19 世纪比 20 世纪要小得多,而在下一个千年,它们基本会彻底消失。

统治的全部逻辑和权力的特征已经被微处理技术所改变。当你首次想到这一点时,可能会觉得很夸张。但请仔细想一想。在 20 世纪,政府与工会的繁荣是同步进行的,可谓一荣俱荣。在 20 世纪之前,大多数政府能够征用的资源,都远远小于我们现在所习惯的激进福利国家。同样,在此之前,工会是经济生活中很小、很微不足道的因素。工人迫使雇主支付高于市场平均工资的能力,与政府抽走经济产出的 40\%甚至更多的能力,是基于同样的大政治条件。

\subsection{20 世纪之前工作场所中的勒索行径}
工会勒索资本家的兴衰成败,从生产发展中不断变化的大政治条件,可以得到清晰的解释。1776 年,当亚当·斯密出版《国富论》时,工作场所中的条件是很不利于敲诈勒索的,因为工人想“提高劳动价格”的“联盟”(union 即工会)很难成立。当时大部分的制造业,规模都很小,或者是家庭经营的。规模较大的工业活动才刚刚出现。虽然难免有暴力掠夺的行为,但是它们的杠杆作用很小。

实际上,在斯密的时代直到 19 世纪,在英国、美国和其他普通法系国家,工会都被视为非法组织。亚当·斯密对罢工是这样描述的:“他们常见的借口是生活成本增加了,或者是老板通过他们的劳动赚了大钱。他们总是会极尽吵闹,有时候还诉诸于令人震惊的暴力和暴行。”即便如此,工人们“很少从这些骚乱中得到任何好处”,除了“领头者被惩罚或被处理掉。”19 世纪,工业经济和工会的规模都在扩大。不过,大部分人还是农民或是小业主,为自己工作。工会组织的活动,就像亚当·斯密所说的,仍然是“通常以失败而告终”。第一批成功组织起来的工会,来自于有很高技能的手工业者,他们的联合往往不是经由暴力。他们可以要求涨工资,把工资设定到与替换他们的边际成本相匹配的位置。而非技术工人的工会则是另外一番景象。他们会利用工业组织向大型企业的转变,筛选出比较容易受到胁迫影响的行业;这些行业,要么是经营规模庞大,要么是很容易遭受物理上的蓄意破坏。工会的这种操作模式,从纽卡斯尔到阿根廷都得到了印证。

在美国,劳工暴力运动的一个早期例子,是 1834 年对切萨皮克和俄亥俄运河的攻击。与 19 世纪早期的大部分企业不同,切萨皮克和俄亥俄运河并不是一个封闭的、容易保护的产业。按照最初的规划,它将绵延 342 英里,从波托马克河下游到俄亥俄河上游,有 3000 英尺的落差。挖这样一条沟渠,是一项从未完成过的大工程。

当时有大量的工人在这里工作,其中一些人很快就意识到,他们可以很容易把运河工程瘫痪掉。实际上,如果没有定期的维护,运河很容易被挖地道的麝鼠破坏。

在运营中,运河的水闸与河道也可能因使用不慎、暴雨导致的洪水或船只的撞击而毁坏。对罢工者来说,用沉船或者其他杂物封锁航道是一件很容易的事。

4 年,在运河上工作的爱尔兰人的敌对帮派之间发生了暴动,导致有人想利用这一点占领运河。不过这一企图最终失败了。安德鲁·杰克逊总统派出了麦克亨利堡的联邦军队,驱散了工人,造成了 5 人死亡。

矿山和铁路也是美国工会早期的攻击目标。与运河一样,它们非常容易遭到破坏。

例如,矿场可以被淹没,或在入口处被封锁。只要把从地下矿井中拖矿车的骡子杀掉,就能给矿主制造很大的困难与烦恼。铁路也是一样,它绵延伸展,是很难防护的。相对而言,工会暴徒很容易袭击矿场和铁路,并造成巨大的损失。在工会努力成为更有效组织的过程中,这样的袭击经常发生。在实际工资因为通货紧缩而上升的时期,当业主试图调整名义工资,工会活动最为激烈,往往会导致暴力抗议。在 1873 年恐慌后的大萧条中,这类事件非常普遍。

1874 年 12 月,在宾夕法尼亚州东部的无烟煤田,爆发了公开的战争。工会组织了一支暴力罢工队伍,伪装成一个秘密社团,化名为“希伯尼安古会”;也被称为“莫利·马奎尔”,一个爱尔兰革命者的名字。这个团伙臭名昭著的行为有:“恐吓煤田老板,阻止想工作的旷工上班,破坏财产,毁坏财物,公然谋杀和暗杀。”铁路员工中也经常发生暴力事件。例如,1877 年 7 月,宾夕法尼亚铁路与巴尔的摩和俄亥俄铁路都旨在破坏财产的恶性事件。工人们夺取了控制室,拆毁了铁路,封锁了货仓,捣坏了机车,将火车抢劫一空等等。在匹兹堡,宾夕法尼亚铁路的圆顶机车库被防火烧毁,有数百人在里面。几十人因此而丧生,两千节火车车厢被烧毁和抢劫,机械车间被破坏,连同一台谷物升降机和 125 台机车。最后联邦军队介入干预,恢复了秩序。

这些早期的罢工虽然得到了社会主义者及工会活动家们的同情和理解,但它们并没有得到公众的支持。铁路和矿山等行业固然有其脆弱性,但当时总体的大政治条件还不利于工人剥削资本家。企业规模太小,很难发展起来系统性的敲诈。有些行业虽然很容易遭到攻击,但它们雇佣的工人占比太小,胁迫雇主得到的好处无法被广泛分享。缺乏普遍的支持,敲诈勒索的行当是不可持续的,因为业主可以求助于政府的保护。工会有时候也会恐吓地方官员,组织他们干涉罢工,但很少获得成功。即使是最暴力的罢工,在几天或几周内就会被军事手段镇压。

\subsection{讹诈变容易了}
信息时代可以从工业时代借鉴的一个教训是:如果公司的规模比较小,工会想把工资抬高到超出市场结算的水平,是很难获得成功的。即使是那些明显很容易遭到破坏的行业,如运河、铁路、有轨电车和矿山,工会想要控制住,也不是那么简单。这并不是因为工会顾及使用暴力恰恰相反,他们在暴力手段上毫不节制,有时候还会攻击知名度很高的个人。例如,在美国劳工运动史上被誉为“旷工复仇记”的案例中,爱达荷州的州长弗兰克·斯泰纳伯格,因为反对旷工封锁科达伦矿区,被工会雇佣的杀手用炸弹给炸死了。不过,在 20 世纪的超级工厂与规模化生产兴起之前,即使是通过谋杀或死亡威胁,工会的要求也很难达成。

要想理解为什么工会的地位在 20 世纪发生了翻天覆地的变化,你必须关注生产技术的特点。20 世纪初的几十年里,蓝领工人的数量猛增,这肯定导致了某些变化;这些变化使处于经济前沿的企业特别容易受到勒索。事实上,工业技术的物理特性,几乎就是在邀请工人利用胁迫手段去撼动资本家的地位。请考虑以下几个方面:

1. 大多数工业产品中的自然资源含量都很高。在这种条件下,生产就需要固定在有限的地点,就像矿场必须位于矿体所在地一样。另外,靠近交通枢纽的工厂往往具有显著的经营优势,因为更方便获得零件供应和原材料。这就使得像政府或工会等强制机构,更容易榨取这些优势为己所用。

2. 规模经济催生了巨无霸企业。19 世纪早期的工厂相对还很小;但到了 20 世纪,随着流水线的增加,规模经济不断扩张,处于生产重心的设备的规模和成本也迅速飙升。这样的企业在多个方面都容易受到攻击。例如,有效的规模经济往往伴随着很长的产品周期;产品周期长又意味着更稳定的市场。而这反过来也招致了对企业的掠夺和攻击,因为可以抽取更长期的利润。

3. 主导产业的竞争者数量急剧下降。在工业化时代,只有少数几家公司在竞争数十亿美元的市场,是很常见的现象。这样的公司更容易成为工会敲诈的目标,因为攻击五家公司比攻击五千家要轻松得多。工业竞争的集中,本身就是一个有利敲诈的因素。而这种优势也是自我强化的,这些企业虽然被迫支付垄断性的工资,但它们不太可能受到同行的挑战,即使那些公司不用承担超出市场水平的劳动力成本。因此,工会可以抽干这些企业相当大一部分利润,也不会使它们很快破产。显然,如果工人迫使雇主提高工资,雇主很容易就破产,那这对工人也没什么好处。

4. 为契合公司的发展规模,对固定投资的资本要求越来越高。这不仅增加了资本的脆弱性,放大了工厂关闭的成本损失;也使得现代工厂越来越不可能被个人或家庭所拥有,除非是继承某人以较小规模创办的企业。要解决大型工厂所需的器械和高额成本,需要在资本市场上汇聚成千上万人的财富。

这往往会使分散的、几乎是匿名的业主更难保护他们的财产。业主别无选择,只能依靠职业经理人,而这些人持有的公司股份非常少。对职业经理人的依赖削弱了公司对抗敲诈的能力。当公司财产面临威胁时,经理人不可能冒着生命危险去捍卫;他们的抗争精神,跟那些烟酒商店的老板或小企业的业主无法相提并论。

5. 公司规模的扩大也意味着,与以往所有时期相比,更大部分的总劳动力受雇于更少的公司。有时候,一家公司拥有数以万计的员工。从军事的角度看,业主和管理者的人数,相对于下面工人的数量是不成比例的。三十比一或更低的比例都很常见。这种劣势会随着公司规模的扩大而增加,因为大量的工人聚集在一起,很容易通过匿名的方式使用暴力。在这么多人的情况下,工人不可能与业主发生任何有意义的接触和关系。毫无疑问,这种关系的匿名性,会导致员工轻视业主产权的重要性。

6. 少数公司占有大规模的就业是一种广泛的社会现象。与 19 世纪的美国相比,这进一步加强了工会享有的大政治优势,当时大多数人都是自营职业者,或者是在小公司工作。到 1940 年,60\%的美国劳动力是蓝领工作者。因此,利用敲诈手段挺高工资得到了越来越多人的支持,他们认为自己能从中获益。一项研究很好地说明了这一点。1938 年到 1939 年,在俄亥俄州的阿克伦市,对 1700 人进行了调查,询问他们对公司财产的看法。调查发现,CIO 橡胶厂 68\%的工人对公司的财产没有概念或缺乏认同感,“只有 1\%的人强烈支持公司的财产权。”而另一方面,没有一个商人,或者小业主,是“强烈反对公司财产权的,94\%的人都是极度支持。”

7. 流水线的技术本质是讲究顺序的。在整个生产过程中,零部件要按固定的顺序去移动和组装,这就留下了很多可能遭到破坏的漏洞。其实,流水线就像在工厂围墙内的一条铁路。如果轨道被阻断,或者一个零件的供应被切断,整个生产过程就会停滞。

8. 流水线技术使工作标准化。这减少了不同技工使用相同工具在产出上的差异。

事实上,工厂设计的一个重要目标就是打造一个系统,在这个系统中,天才和白痴轮班工作,可以生产出相同的产品。那些堪称“愚蠢”的机器被设计出来,只为生产某一种产品。这使得凯迪拉克的买家根本没必要去询问生产他那台车的工人的身份。因为所有的产品都是一样的,不管生产它们的工人在技能和智力上有什么不同。

装配线上的非技术工人,与那些能力比他们强的人相比,可以生产出同样的产品。


这让人们开始觉得,每个人的经济贡献是同等的。这种想法推动了平等主义的发展。创业家的技能和精神似乎不那么重要了。现代工业生产的魔力好像就在于机器本身。它们不可能是由每个人设计出来的,但却可以被每个具备一定智力的人所使用。这使得非熟练工被工厂主“剥削”的说法更加可信,因为将工厂主从整个生产公式中剔除,谁也不会受到损失,除了他自己。“我们知道我们可以拿下工厂,”一位通用汽车的罢工者说,“我们已经搞懂了怎么运作它们。如果通用公司不小心的话,我们就可以用所见所学来接手。”工业技术的这些特点,糅合在一起导致了工会的诞生,工会可以利用企业的弱点进行敲诈;它还导致了更大的政府,它们可以对大型工业企业征收高额税负,大发其财。这种情况不是一时一地,而是发生在所有大规模工业扎根的地方。一次又一次,工会冒出来,通过暴力手段要求获得远高于市场水平的工资。它们之所以能够得逞,是因为工业时代的工厂造价非凡、地位显眼、无法移动且成本高昂,不可能被隐藏,也很难搬到别的地方。这些工厂被停工的每一刻,都意味着它们在白白消耗惊人的成本,而没有产生任何的收益。

所有这一切都使它们成为了被威胁敲诈的活靶子。存在于工会历史中的事实真相,远比 20 世纪主流意识形态让你相信的更加明显。著名经济学家亨利·西蒙斯(Henry Simons)在 1944 提出了这个问题:工会如果没有威胁恐吓的强大力量,那它只是一个不真实抽象概念。现在工会拥有了这样的力量,而且只要以目前的形式存在,它们现在拥有,未来也将一直拥有。当权力尚且弱小或不够稳固时,就应该公开地、广泛地使用;当权力巨大且不受质疑时,就会变得像政府的权力一样,自信地持有,被尊敬地对待,而很少惹人注目地运用。 西蒙斯的分析很精辟,但在关键一点上他是错误的。他认为工会“将永远持有”他所说的“强大的威胁恐吓的力量”。事实上,工会正在逐渐消失,不仅仅是在美国和英国,在其他成熟的工业社会都是如此。它们衰退的原因,西蒙斯没有看到,甚至许多工会组织者也无从理解,就是向信息时代的转变已经决定性地改变了财产保护的大政治条件,大大提高了财产的安全性。事实证明,微处理技术已经开始颠覆福利国家赖以生存的敲诈逻辑,因为即使是在商业领域,它也创造出了与工业时期非常不同的激励机制。

1. 信息技术中的自然资源含量可以忽略不计。固定的位置在信息时代没有任何优势。大多数信息技术是高度流动的,它不受地点的限制,它增强了思想、人、和资本的流动性。通用汽车不可能把它在密歇根州弗林特的三个工厂打包,然后坐飞机搬到别的地方。一家软件公司就可以,公司老板可以把他们的算法代码下载到笔记本电脑中,乘下一班飞机离开。对这些公司来说,还有一个额外的诱因,就是可以逃避高额的税收或工会对垄断性工资的要求。小公司往往面临更强烈的竞争。如果有几十个甚至几百个竞争对手在抢你的客户,你不可能向政客或员工支付比其实际价值高太多的报酬。这么做意味着你的成本远高于竞争对手,你就会破产。当固定位置带来的竞争优势不复存在,像政府或工会这样的胁迫性组织,想再利用这些优势为自己分一杯羹,它们手中的杠杆无疑就少了很多。

2. 信息技术降低了企业规模。企业更小,意味着有更多的竞争对手。竞争的加剧降低了敲诈勒索的可能性,因为要将工资或税率提高到竞争水平之上,工会或政府需要实际控制更多数量的企业。在信息技术的推动下,公司的平均规模急剧下降,雇佣员工的人数明显减少。例如在美国,根据大面积的报道估计,在 1996年,有多达 3000 万人在自己的企业里独立工作。这 3000 万人显然不会罢自己的工。另外还有几百万人在雇员很少的公司上班,他们要强迫雇主支付高于市场水平的工资,恐怕也不太现实。

在信息时代,想要通过敲诈勒索提高工资的人,缺乏那种军事上压倒性的人数优势;正是这种优势使工厂工人难以对付。公司雇佣的员工越少,匿名使用暴力的机会就越少。就因为这个原因,把 1 万名工人分散到 500 家公司,他们给业主财产构成的威胁,远远小于 1 万人集中在一家公司,即使工人与业主/管理者之间的人数比例完全相同。

3. 企业规模的缩小也意味着想获得高于市场水平工资的要求,不太可能像在工业时期那样,得到广泛的社会支持。在信息时代敲诈雇主的工会,会发现自己的处境更类似 19 世纪的运河工人、铁路员工和旷工(即很难获得社会支持,译注)。

即使少数拥有规模经济效应的大企业会保留下来,作为工业时代的遗产,但在就业广泛分散于小公司的背景下,它们也很难再向敲诈勒索屈服。小企业和小业主的优势表明,即使收入再分配的愿望没有改变,人们也会更加支持产权的概念和保护。

4. 信息技术降低了资本成本,这会激发创业精神,使更多人可以独立工作,进而促进了竞争。较低的资本要求不仅降低了进入的门槛,也减轻了“退出的障碍”。

换句话说,相对于收入而言,公司可能拥有较少的资产,承受损失的能力也较弱。

信息时代的企业,可能不仅没有多少用于向银行借贷的资源,也可能根本没有什么能被抢夺的实物资产。

5. 信息技术缩短了产品周期。在这种情况下,产品的淘汰更加迅速,通过敲诈享受高工资收益的时间,将变得非常短暂。在高度竞争的市场中,过高的工资可能导致工作岗位很快消失,甚至导致企业破产。以牺牲自己的工作为代价来争取暂时的高工资,就像烧掉家具让屋子里暖和上几度一样。

6. “信息技术”不是连续性的,而是同步性和分散性的。与流水线不同,信息技术可以同时容纳多个进程。它的活动分散在网络上,允许在不同的工作站之间进行冗余替换;而工作站的数量可能高达几千甚至几百万个,并且可能位于地球上的任何地方。人们会在越来越多的活动中,在完全没有物理接触的情况下进行合作。随着虚拟现实和视频会议的技术更加先进,职能分散与远程办公的去世将加速。这相当于是信息时代的“散工制”(putting-out),它打破了中世纪行会的垄断。

在烟雾缭绕的工厂里一起工作的人越来越少,这不仅剥夺了以往工人敲诈资本家时所拥有的人数优势,它甚至使人们很难将工作场所中允许的敲诈与真正的敲诈罪区分开来。迄今为止,只有被同一家公司雇佣、在共同的环境中一起工作的人,使用暴力去要求更高的工资是可以被接受的。但是,如果“工作场所”不再处于一个中心地点,而且公司的大部分只能都分散给了分包商和远程工作人员,那么,这些人从他们的客户或“雇主”那里要求更高收入的行为,就很难与非法的敲诈勒索罪区分开。

例如,一个威胁用病毒感染公司电脑然后索要额外现金的远程工作人员,他是属于罢工工人,还是互联网敲诈犯?不管他是什么,区别的意义都不大。在两种情况下,被攻击公司的反应可能都是一样的。那就是采取防止信息破坏的技术解决方案,如改进加密措施和网络安全,这不仅可以解决外来的黑客攻击,还可以预防心怀不满的员工或分包商对公司各方造成损害,虽然他们之间可能只是偶尔打大交道。

当然,有人可能会建议,工人或远程办公人员可以随时去公司报到,然后在那里举行传统的罢工。但在信息时代,这种做法并不像看起来那么简单。信息技术超越了地理位置,并分散了经济职能,这意味着有史以来第一次,公司雇员和雇主不需要居住在同一个管辖区内。请注意,我们说的不是梅菲尔区和佩克汉姆区的分别,而是雇主在百慕大,远程办公人员在新德里。

此外,如果印度人被 1936-37 年通用汽车大罢工的报道所蛊惑,决定前往百慕大进行抗议,等他们到达那里,却发现根本没有实体办公室。Chiat/Day(李岱艾),一家大型广告公司,已经开始拆除其总部,它的员工和分包商将通过电话和互联网保持联系。如果需要召集团队协商客户项目的工作时,他们会租用酒店的会议室;项目一结束,就把房间退掉。

微处理技术将把生产流程从流水线的固定顺序解放并分散开来,这会大大降低工会和政府等强制性机构过去享有的杠杆作用。如果说流水线是工厂围墙内的铁路,很容易被静坐罢工所占领,那么网络空间则是一个没有物理存在的无界疆域,它无法被武力占领或勒索。在信息时代,想使用暴力作为杠杆谋求更高收入的员工,他们的处境将比 1936-37 年通用汽车的罢工者要差得多。

7. 微处理技术使工作更加个性化。工业技术使工作标准化,任何人使用同样的工具都会得到同样的产出。微技术开始使用更智能的科技取代“愚蠢”的机器,它实现的产出将是高度可变的。使用同样技术的人,产出的差别增加了,这具有深远的影响。在接下来的章节中,我们会探讨更多这方面的内容。其中尤为重要的是,当产出不同,收入也会不同。


在人们的技能有差距的领域,大部分的价值往往是由少数人创造的。这是竞争最激烈的市场的一个普遍特征。比如,体育界就非常明显。全世界有数以百万计的年轻人踢各式各样的足球,但 99\%的人花钱买票,只是为了看极少数球员的表演。

同样,世界上有无数充满抱负的男女演员,只有个别的才能成为明星。此外,每年还有数以万计的书籍出版,而大部分版税都被那些能够娱乐读者的畅销书作家。


赚走了。不幸的是,我们并不在其中。

使用同样的设备,产出却可能天差地别,这给敲诈勒索者制造了另一个障碍,那就是关于如何分配回报,他们之间要反复地讨价还价。如果某项工作中相对较少的人创造了大部分价值,从数学上说,他们不会因为通过罢工得到平均工资而增加收入。一个软件工程师设计出某种控制机器人的算法,可能价值数百万美元;而另一个人使用同样的设备,写出来的程序却毫无价值。生产力更强的程序员,肯定不愿意把他的收入与同事的收入绑在一起,就像汤姆·克兰西不可能同意把他的版税与我们的版税平均。

虽然现在还是信息革命的早期,但相比 1975 年,非常明显的一点是,技术能力和心智能力已经成为了经济产出的关键变量。曾经在工业时代盛行的、把“工人敲诈资本家”给合理化的骄傲说辞,已经被蒸发了。认为非技术工人实际创造的价值被资本家和企业家不成比例地抽走了,也成为了一种不合时宜的幻觉。在信息时代,这种虚构根本就站不住脚。当程序员坐下来写代码时,他的技能与产品之间的归属关系太直接了,谁负责什么、谁做了什么,搞错的可能性不大。一个文盲或半文盲不可能为计算机编程,这显然是毫无争议的;同样明显的是,别人编程产生的所有价值不可能是从他这里偷走的。这就是为什么现在还控诉“剥削”的工人主要是清洁工。

信息技术是人们清楚地看到,低技能劳动者所面临的问题,不是他们的生产力被不公平地剥削了,而恐怕是他们缺乏真正做出经济贡献的能力。正如凯文·凯利在《失控》一书中所说,信息时代的“新贵”汽车公司,可能只是“十几个人”的智慧结晶。相比底特律和东京的汽车公司,他们把大部分零部件外包,但依然能够精心定制出更加符合客户心意的汽车:“这些汽车,每一辆都是定制的,由。


客户在网上下单,生产完成后立即发货。车身模具由计算机控制的激光工具快速成型,并根据客户和目标消费群的反馈生成设计。一条灵活的机器人生产线负责组装,机器人的维修和升级则外包给机器人公司。”

\subsection{“会说话的工具”}
未来越来越普遍的现象是,非技术性工作将由自动化机器、机器人和计算系统来完成,如数字助理。亚里士多德曾把奴隶描述为“会说话的工具”,他说的工具是人。在不远的将来,真正“会说话的工具”将会出现,就像寓言故事中的精灵一样,它们会说话,能够听懂并遵循指令,甚至可以处理复杂的任务。高速发展的计算能力已经催生了一些语音识别的原始应用,如免提电话和根据口头指令进行数学运算的计算机。在我们写这本书的时候,能够将语音转换为文本的计算机已于 1996 年底上市。随着模式识别能力的提升,连接着语音合成器的计算机通过网络运行,将会执行许多以前由人工承担的职能,如电话接线员、秘书、旅行代办、行政助理、国际象棋冠军、索赔处理员、作曲家、债券交易员、网络战争专家、武器分析员,甚至是具有街头智慧可以接听 900 电话的聊骚高手。

卡内基·梅隆大学的迈克尔·莫尔丁(Michael Mauldin)已经变成出了一个“机器人”,一个名为茱莉亚的虚拟人物,她几乎能够骗过在网上跟她聊天的每个人。

据媒体报道,茱莉亚是个“满嘴俏皮话的贵妇,在网上玩角色扮演的游戏打发时光。她聪明,风趣,喜欢调情。她还有一种机智之才,对每一条信息,都可以瞬间蹦出完美的讽刺性回复。然而,茱莉亚并不是一位女士,而是一个机器人,一种只存在于互联网上的人工智能。”对人工智能和数字仆人的开发已经取得了惊人的进展,毫无疑问,更多实际的应用还在后面。它们的大政治影响无可限量。

\subsection{一个人的军团}
为各种用途开发出“会说话的工具”,这创造了一种可能性,它使个体可以同时分散进行多种活动。不久之后,个人将不再是单一的存在,而是可能成为通过智能媒介进行的几十种甚至几千种活动的集合体。这不仅会极大地提高最具天赋个体的生产力;在军事层面,它还将使主权个人比以往所有的个体都更加强大。

通过利用基本上是无限量的智能媒介,一个人可以强力放大他的活动。甚至在死了之后,他还可以发出行动。有史以来第一次,一个人即使在生物意义上已经死亡,还有能力执行他精心设计的计划。无论是战争中的敌人还是日常生活中的罪犯,杀掉他们也不可能完全消灭其实施报复的能力。这是整个人类历史上,对暴力的逻辑进行的最具革命性的创新之一。

\subsection{对信息时代的洞察}
人类生活中最重大的变化,往往来自那些无人关注的变量。或者说,我们会理所当然地认为,在几百年甚至几百代人的时间内都很少波动的变量,会一直如此。

在人类生存的大部分时间内,保护与勒索之间的平衡一直在很小的幅度内波动,而且勒索始终占据着上风。如今这一切即将改变。那些决定着暴力的成本与回报的要素,将发生根本性的转变,信息技术正在为此奠定基础。在未来的保护远景中,对那些使用暴力的人,将会有智能代理人对其进行调查,并以某种方式实施报复。25 年前,如果一个人说,“你要是杀了我,我就把你银行账户里所有的钱都提出来,捐给尼泊尔的慈善机构。”这可能被认为是疯子的胡言乱语。但千禧年之后,恐怕就不是了。它能否构成一种实际的威胁,将由时间和地点等因素来决定。不过,即使未来不法之徒的账户密不透风,智能特工军团也肯定可以通过其他方式对犯罪实施报复,让他们付出高昂的代价。想一想吧。

\subsection{保护方式的新选择}
过去两个世纪以来,政府一直享受着对保护与勒索的垄断。信息时代的技术将开辟出大量加强保护的方法,打破这种垄断;上述只是其中之一。即使不用那些令人眼花缭乱的新技术,也有其他的保护方式;政府不会垄断住一切。

一个人如果感受到威胁,他跑掉就可以了。在世界还年轻、视野还开阔的时候,逃跑是人们常用的保护方式。如果担心因盗窃或破坏而造成财产损失,也可以购买保险覆盖这些风险。诅咒和咒语虽然是很虚弱的保护,但也吓退过盗贼,拯救过生命;在掠夺者很迷信的社会中,有时候它们也会起作用。贵重物品可以藏起来,藏得好的话也是一种有效的保护。把财产埋到地下,上锁,藏在高墙后面,墙上安装警报和电子监控设备。不过,把人或财产藏起来并不是很实用。

在历史上出现过的各种保护手段中,有一种独占鳌头,那就是以暴制暴,以更大的力量压倒所有想攻击你或偷你抢你的人。问题是你去哪里找到这样的服务,你怎么激励别人冒着生命危险帮助你,与那些对你发起攻击的暴徒作战。

这种保护,有时候来自近亲属。有时候,是以部落和氏族为为基础的团体,充当非官方的警察角色,以血仇的方式,报复任何成员遭到的暴力攻击。有时候是雇佣兵或私人卫队,你付钱请他们来提供保护,但这些人不像你想象的那么有用。

信息时代的新型智能特工,虽然它们的活动范围局限于网络空间,但却提供了一种新的保护选择。不同于那些雇佣兵、私人卫队,甚至是你的远方表兄弟,它们的忠诚度无与伦比。

\subsection{权力的悖论}
以暴制暴其实充满了悖论。在既往的历史条件下,你雇佣来保护生命和财富不受攻击的团体,如果它们可以成功地保护你,也必然可以掠夺你。这是一个严重的问题,它没有简单的答案。通常情况下,在经济领域,你可以会指望通过竞争防止卖方无视客户的意愿。但就暴力而言,直接竞争往往会产生反常的结果。当两家潜在的保护机构派人去逮捕对方时,这更像是一场内战而不是保护。在你寻求保护,免受暴力侵害时,你希望能够压制暴力,而不是增加暴力的输出。而压制暴力首先就要保证,你付费购买的保护服务不会反过头来攻击你。

\begin{tcolorbox}
在人类的生活中,当所有人敬畏的共同权力还未出现时,人们处于被称为战争的状态,而这种战争是所有人对所有的战争。在这种环境种,每个人除了靠自己的力量和创造来保护自身安全之外,没有其他任何的保障。 
\begin{flushright}
—— 托马斯·霍布斯    
\end{flushright}
\end{tcolorbox}

\subsection{垄断和无政府状态}

这就是为什么无政府状态,也就是霍布斯所说的“所有人对所有人的战争”,不是人们想要的生活方式。地方上暴力的竞争意味着为保护付出的代价更高,而得到的保护却更少。有些热衷自由市场的人提出,仅靠市场机制就足以提供对产权和生命的保护,并不需要任何的主权。有些说法很有道理,但事实是,在工业化的大政治条件下,由自由市场提供经常和司法服务并不可行。只有在那些行为高度模式化、人口极少且人种单一的原始社会,才能在没有政府暴力提供垄断性保护的情况下生存。

组织水平能够超过狩猎采集部落的无政府社会,这样的例子很少而且很古老;主要存在于一些最简单的农业经济体中,它们往往与世隔绝,依靠雨水生存。例如,皈依穆斯林之前的阿富汗的卡菲尔人;黑暗时代的一些爱尔兰部落;巴西、委内瑞拉和巴拉圭的一些印第安人部落;还有其他散落世界各地的原住民。他们在没有政府的情况下组织保护的方式,只有那些了解人类极端状态的行家才懂得。如果你感兴趣的话,我们在注释里引用了几本书,并包含了更多的细节。原始群体中没有专门从事暴力的组织,他们之所以能在这种情况下运转,只因为他们是小型的、封闭的社会,与世隔绝。他们利用紧密的亲属关系,在有限的范围内抵御大部分的暴力攻击,也是他们唯一可能遇到的威胁。当遇到更大规模的暴力,例如由国家组织的,他们就会被制服,并承受外来组织的垄断性统治。这种情况屡屡发生。无论哪里,尤其是在贸易线上,不同民族接壤的地方,只要社会的规模超过了部落的大小,专业的暴力组织就会出现,去掠夺更和平的人们生产的一切。

当技术发展提高了暴力的回报,那些不能组织起来把更多资源投入战争的社会,就注定要遭殃。

\begin{tcolorbox}
哪些王子在提供警察的服务?哪些是勒索者甚至是掠夺者?一个掠夺者其实可以成为警察局长,只要他能将自己的“所得”规范化,使之与支付能力相适应,保护自身利益不受其他掠夺者的侵犯,然后垄断地盘足够长的时间,直到社会承认其合法性。
\begin{flushright}
—— 弗雷德里克·莱恩
\end{flushright}
\end{tcolorbox}

\subsection{政府作为保护服务的卖方}
我们多次提到过,从纳税人的角度看,政府的主要经济职能就是为生命和财产提供保护。然而,政府的运作就像犯罪组织,它从统治下的人们那里榨取资源作为贡品或战利品。政府的统治不仅是一种保护性服务,也是一种保护性勒索。政府提供保护,防止源自别处的暴力,但它也向客户收取费用,和黑社会收保护费一样。提供保护是一种经济服务,收保护费则是一种敲诈勒索。在实际生活中,这两者的区别是很难分清的。正如查尔斯·蒂利(Charles Tilly)所说,最好把政府理解为“最大的有组织犯罪的例子。”即使是最好的政府,它的治理也是保护性服务与保护性敲诈的混合。从历史上看,如果政府能够在它的领土内实现暴力垄断,那么它在这两方面的表现都能得到优化。在同一块领土上,如果有某一个单独的武装团体,能够占据使用暴力的主导地位,那么相对于存在几个互相争斗的竞争性保护组织,它所提供的服务质量,将会远远优于几个中的任何一个。

\subsection{对领地的天然垄断}

实现对强制力的地方性垄断,不仅能使政府更有效地保护其潜在客户免受外来的暴力伤害,还可以大大降低政府的运营成本。正如莱恩所说:“控制和使用暴力的行业是一种自然垄断,至少在土地方面是这样。在领地范围内,通过垄断,政府提供的服务可以更便宜。”因此,“在一块领土之内垄断对暴力的使用,可以使提供保护的集团改进其服务并降低其成本。”这样一个统治集团,如果不需要经常发动军事行动,去抵御那些前来抢夺其客户的竞争组织,那么它就能以更低的成本提供更多的保护。

主权必须建立在领土垄断之上,这个假设的前提将在信息技术的作用下变得“松弛”;政治理论家已经注意到了这一点。大卫·埃尔金斯的《超越主权》(DavidJ. Elkins:《Beyond Sovereignty》)一书就是关于这个主题。埃尔金斯呼应了我们的论点,即政府的垄断注定将遭到去中介化(脱媒化 Disintermediated),就像1500 年之后的教会一样。他写道:“我们曾经假定,宗教必须有它的领土或‘地盘’。随着国家取代无处不在的宗教,成为生与死的最高裁判者,宗教的‘紧密’与‘边界’让位于我们现在熟悉的同一地区的信徒交融。然而,我们并不支持国家或省份之间的交融,尽管我相信这一假定(主权必须建立在领土垄断之上,译注)正在被打破。”他后面的论证也与我们的观点一致。他说,主权对领土的垄断,可以在不陷入无政府状态下被打破,像加拿大联邦中国家与省级政府之间的主权分裂;以及一些太平洋中的岛屿,在本世纪的大部分时间内,都是由英法联合主权组成的共管政府。因此,主权对领土的垄断虽然很少通过武力解除,但可以通过协议解除。埃尔金斯认为:“领土国家是一个包袱或篮子,我们生活的方方面面都被融入其中。它很像经济学概念中的‘一篮子商品’,你没办法地获取单个商品,而必须接受集体套装。在餐馆里,人们可以逐个‘点菜’;但涉及到我们的身份时,我们必须接受国家捆绑在一起的东西,这相当于‘套餐’。对21 世纪的公民来说,点餐式政府会更自然一些。”埃尔金斯的这个说法我们深表赞同。而对于主权的分解和点餐式政府的崛起,没有什么比完全超越物理边界的网络经济,更能促进这一点。


\begin{tcolorbox}
随着频率的加强和波长的缩短,数字传输的性能将成倍提高。带宽扩大,所需的功率下降,天线的尺寸变小,干扰将无计可施,错误率也会大幅减少。
\begin{flushright}
—— 乔治·吉尔德(George Gilder)    
\end{flushright}
\end{tcolorbox}


\section{电信世界的法律将废除国家法律}

带宽(或通信媒体的承载能力)必将打败领土国家,不只是我们看到了这一点。

《500 年的三角洲》(The 500-year Delta)的作者,吉姆·泰勒和沃茨·瓦克(JimTaylor, Watts Wacker),虽然没有像我们这样定义他们的观点,但他们已经清楚地看到,“互联网的接入创造了全球主义,全球主义使边界的概念过时,从而破坏了政治体系。随着边界的消失,支持政府的税收变得越来越脆弱……随着边界的消失,权利的概念——因为你出生在某个特定的地方,所以享有与之相关的经济优势——也将崩溃,国家的福利制度也会随之崩溃。与这一切相伴随的是,构筑国家地位的所有理念——爱国主义、民主、国家、大熔炉、大一统、公民责任,不管是在哪个国家,以什么形式存在——都将被扫进历史的垃圾桶。”虽然他们没有明确阐述,但显然也感受到历史在朝着解放主权个人的方向发展。正如他们所说,“一种更加纯粹的个人主义正在冉冉升起,它超出我们所了解的民主制度。”这将如何发生呢?泰勒和瓦克看到一股强大的动能,正在促使改变:

\begin{tcolorbox}
一个简单的事实是,宽泛的爱国主义意识——爱一个国家,对它负有忠诚的责任感——将不再是一种特别有用的态度……在全球性社会中发展起来的公民,会认同自己是全球公民。他们所做出的政治、生活、经济上的选择,不是基于国家身份,而是基于这些选择与他们自己以及世界各地像他们一样的人之间的直接关系……在全球社会中蓬勃发展起来的国家和公司,其组织形式会与此相适应。他们会最大化认知、出行、活动及存在的自由。反之,那些抱着怀旧情绪负隅顽抗的国家和公司,将逐渐萎缩。
\end{tcolorbox}

现在,网络带宽在以每年三倍的速度增加,互联网和万维网以几何级数扩展,这意味着物理边界的日益式微,并将加速对政府的去中介化进程。如果带宽继续以每年三倍的速度增长,那么到 2012 年,网络带宽将是 1993 年的 10 亿倍;正是在那个时候,乔治·吉尔德首次提出,带宽的复合增长速度甚至会超过微处理器的发展。如果这一点得以实现,强大的通信能力将促进网络贸易的爆发式增长。

而从最近集成光学领域的突破来看,我们相信这肯定会实现。通过波分复用,一根细如发丝的光线每秒可以传输一万亿字节。换句话说,一根光线所能容纳的数据量,是世界上全部通信网络总负荷的 25 倍。这样的扩张能力令人瞠目结舌。

当这么强大的通信潜力被释放,将有大量的钱花在网络通信上,因为它会很便宜;而像专线电话和电视等既有媒体将被淘汰。万维网向每台计算机传输的信号组合,比今天的消费者在电视上看到的要丰富万倍。随着带宽革命的发展,越来越多的人会进入到网络社区、电子商务等无边界的虚拟世界,这个世界的图景密度将足以形成“元宇宙”,也就是科幻小说家尼尔·斯蒂芬森所想象的赛博空间的另类现实。斯蒂芬森的“元宇宙”是一个虚拟社区,拥有自己的法律、君王,还有各种反派角色。

当越来越多的经济活动进入到网络空间,国家待在边界之内垄断权力的意义将减弱,这会刺激它们分割主权,进行特许专营。今天的民族国家,往往有动力去开设自由港、自由贸易区和法国区;未来它们也会有动力租借其主权。在前文中我们讨论过,具有 900 年历史的耶路撒冷、罗德岛及马耳他圣约翰主权军事医院骑士团(简称为马耳他骑士团),在与马耳他共和国谈判,关于将圣安杰洛堡的主权归还与骑士团。我们期待这次谈判成功结束。其他的将随之而来。一些民族国家会把小块飞地或偏远地区的主权,转让给全新的亲和团体和虚拟社区。而像一些商业实体,如保安公司和连锁酒店,去竞标小块土地的主权,是完全有可能的。

在未来,像 Wackenhut、Pinkerton 和 Argenbright(均为安保及私人侦探公司),就有可能在世界各地气候宜人的地方,提供带有安保服务的混合型退休社及免税区。宗教团体,以马耳他骑士团为代表,也包括其他你能想象到的教派,会在地球上某个偏僻的角落,以他们自己的方式实现人间的天堂。甚至一些富有的个人和家庭,也会拥有自己的地盘,在里面行使有限的主权,发行自己的邮票和护照,并维护一个官方网站。

\section{垄断与掠夺}
需要注意的是,有偿分享或租赁主权的动机,与历史上那些统治者面临地方地方垄断势力的竞争,而不得不分权的动机完全不同。租借主权跟设立自由贸易区差不多,并不会破坏社会稳定。相比之下,军阀和游击队通过军事手段竞争权力,直接影响到未来的政府会更多地保护它统治下的人民,还是掠夺他们。当竞争的团体在不稳定的平衡中缠斗时,使用暴力进行掠夺的动机就会增强,横征暴敛将更有吸引力。由于权力不够稳固,还面临对方势力的挑战,掌权者能够运用暴力的时间就很短。“山林之王”其实是站在一个斜坡上,他可能屹立不到最后,去从平定天下中获得丰厚的回报。在这种情况下,就没有什么能够阻止这些人利用手中的权力去恐吓与掠夺。

因此,根据暴力的逻辑,在一块领土上相互竞争的武装力量越多,他们诉诸暴力去掠夺的可能性就越大。如果没有一个压倒性的力量去镇压这些野生的暴力,它们就会四处滋生蔓延,经发发展及社会合作的成果都将化为乌有。

20 世纪 20 年代,在军阀的统治下,处于无政府状态的中国,就证明了暴力彻底释放后可能导致的伤害。在《大清算》一书中我们讲述了这段故事。相互竞争的军阀,缺乏一家独大的力量控制他们,在地方上造成了巨大的破坏。在索马里摩加迪沙的大街上,CNN 的记者则绘声绘色地向全世界传播了类似的故事,证明了类似的观点。索马里的武装军阀,绰号“技术派”,在受到美国领导的大规模军事干预之前,导致这个悲惨的国家处于无政府状态。当美国军队的指挥力量撤走以后,“技术派”再次拿起了武器,无政府状态又一次复辟。《华盛顿邮报》的一篇新闻记述道:

\begin{tcolorbox}
装有高射炮的卡车再次在尘土飞扬、瓦砾遍地的街道上奔驰。穿着 T 恤、扛着 AK47 的年轻人,也大摇大摆地回来了,他们在临时设置的路障上向过往车辆和巴士勒索钱财。有一个民兵控制的街区,武装力量非常强大,被当地人称为“波黑地区”。
今天,在这个城市狭窄的街道上穿行,会让人不禁想起 1992 年的日子,当时敌对民兵之间的混战,使索马里陷入了无政府状态和严重的饥荒,促使美国领导进行了军事干预。如今昔日重来,为了穿越摩加迪沙,旅行者必须雇佣一车的武装人员,以每天 100 美元的价格外加午餐休息时间,希望能够得到他们的保护。
\end{tcolorbox}

索马里、卢旺达,以及其他你很快会在电视上看到的国家,这些例子提供了一个彩色电影般的证明,为控制领土而进行的暴力竞争,与其他形式的竞争不同,并不产生直接的经济收益。恰恰相反。在无政府状态下互相竞争的土匪强盗,完全缺乏保护生产活动的动机;而当独裁者的统治稳固之后,反而有可能减轻他们的重手。

\begin{tcolorbox}
我们所说的现代社会,首先在西方,它的特点就是某种程度的垄断。个人被禁止自由使用军事武器,这种权力属于各种中央权威。同样,对个人财产或收入的征税也集中在一个中央当局手里。流入到中央的财政维持了它对军事力量的垄断,而这反过来又保证了它对税收的专营。这两者不分先后,是同一垄断的两面;如果一个消失了,另一个也会随之自动消失。有时候,这种垄断统治的一面比另一面会遭到更强烈地动摇。
\begin{flushright}
—— 诺伯特·埃利亚斯(Norbert Elias)
\end{flushright}
\end{tcolorbox}

\section{保护方式的演变}
在想象信息时代将如何展开时,我们挪用了莱恩提出的一个观点。他认为,黑暗时代以来的西方经济史,可以用暴力组织中的竞争与垄断的四个阶段来解释。我们所确定的影响政府规模的大政治因素,莱恩基本没有论及,但他对暴力经济学的研究,与我们在《血流成河》及《大清算》中的论点,以及本书中相应的观点都密切吻合。

我们在前面分析过,在罗马灭亡后西方社会的演进中,某些大政治因素发挥了重要作用。莱恩也研究了这一时期,他重点关注对暴力垄断的竞争所导致的经济后果。他发现,在过去的一千年里,西方的经济发展有四个重要时期,分别对应着暴力组织的不同阶段。

\subsection{走出黑暗时代}
第一个时期是“无政府状态与掠夺”,它标志着一千年前的封建革命。虽然莱恩没有说明四个时期的具体起止时间,但从数学上推理,他的第一个时期的边界是很清楚的。而且,他对“无政府和掠夺”阶段的描述,也很符合从黑暗时代过渡时的情况,当时暴力的运用“竞争极为激烈,甚至是在陆地上。”莱恩没有说明原因,但当暴力“竞争激烈”的时候,一般意味着实施权力面临着巨大的障碍,不管距离的远近。从军事上讲,防守比进攻更占优势。

我们在第三章中解释过,在这个“无政府状态与掠夺”的事情,正好发生了不利的气候变化,导致农业生产力下降。由于当时的技术无法成就有效的规模经济,也就无法保证谁能够形成对暴力的垄断,所以,潜在统治者之间的竞争非常普遍。

结果就是经济活动严重萎缩。

经济的疲软,使建立稳定的秩序更加不可能。因为要达成地方性的暴力垄断,付出的军事成本太高,与微薄的经济回报不成比例。所以,那些武装骑士,守着一块经济上可以养活他们的领土,由于缺乏有效垄断暴力的实力,往往会四处恐吓掠夺,而没有为他们的客户提供任何“保护”。

\subsection{封建主义}

“第二个时期开始于小地方或省级垄断组织的成立。农业产出开始上升,大部分经济盈余被新成立的暴力垄断者抽走。” 这个阶段的盈余还是很微薄的,我们认为它属于中世纪早期。这时候,暴力组织还没有形成规模优势,经济增长被抑制,要实施地方垄断的军事成本依然很高。不过,由于无政府状态得到了控制,经济活动开始复苏,虽然暴力成本依然很高,但小的主权单位能收到的保护费也在增加。

“在第二阶段后期,那些收取贡品者为了吸引客户,会对农业和商业团体提供提别的优惠。对那些开荒耕种新田的人,他们以很低的费用提供保护;为了鼓励贸易,他们还提供特殊的治安服务,像香槟区的伯爵们,对参加他们集会的商人就组织类似的服务。” 换句话说,当地方军阀对领土形成足够的控制,能够进行可信赖的谈判时,他们就会像那些需要扩张市场份额的商人一样,对自己的服务进行打折以吸引顾客。然后,军阀们会利用新增经济活动带来的额外资源,巩固自己对更大领土的控制。一旦这种控制牢牢确立,他们就可以享受更多的垄断优势。

虽然维护治安的成本趋于下降,但他们却可以提高收费价格,也不用担心这会降低对客户的吸引力。

在西方历史上这段复杂的历史时期,掌握暴力的人,主要是中世纪的领主和君主,拿走了人们糊口之余的大部分盈余。那时候商人还很少。最成功的人,是那些能够逃避或能最大程度地少缴,收保护费者所强加的税款、费用及其他成本的人。

\subsection{现代早期}
当非暴力专家的商人和地主“比封地拥有者和君主获得更多经济盈余时,”莱恩所说的第三阶段就达到了。“在这个阶段,专门从事暴力的团体,其收益少于向政府购买保护服务的团体。”成功的商人更有可能把利润再投资,而不是消费挥霍掉;所以,在这个历史阶段,商人实现的高额利润,带来了自我加强的发展成果。

\subsection{工厂时代}
莱恩认为,从第三阶段到第四阶段,出现了技术和工业创新,它们的主要作用是提高了盈利能力,而不是降低保护成本。莱恩所指的应该是 1750 年以来的时期。

从那时起,技术发展开始在地区繁荣中发挥明显的主导作用。举一个极端的例子,像 1840 年之前的新西兰,在某些地区根本没有政府存在,但也不可能因为不用交税而变得高度繁荣。在这一历史阶段,相比降低保护成本,工业技术的创新对于盈利更加重要,因为无论你怎么节约,即使把成本降到零,没有开源,也很难产生更多利润。这时随着政府规模的扩大,最初用于军事筹款的官办信贷和融资机制,开始走向民间,为更大规模的商业实体所用。

虽然莱恩没有这么说,但当技术优势集中在某特定地区,就减轻了不同管辖区之间的竞争,使该地区“专营暴力的机构”——即政府,能够收取更高的价格。如果不同辖区间存在巨大的技术落差,就像工业时代那样,那么,拥有最佳技术的辖区内的企业家,往往能赚到更多的钱,即使他们可能要向政府交更多的税和其他费用。

\subsection{面带微笑的掠夺}
工业时代的政府享受着对剥削的垄断地位,这种垄断却奇怪地令人感到愉悦。相对于征收的税价,政府保护人身安全的成本可谓微乎其微。但是,他们处于一个完全异常的竞争领域;事实上,政府从事的工作更多是掠夺,而不是保护,这一点几乎没有人注意到。这是历史上很罕见的阶段。

在工业化的大政治条件下,由于无政府主义的弊端,在同一块土地上进行保护性服务的竞争,技术上是不可行的。在这里,能够实现有效保护的唯一途径,是有能力运用更强大的暴力。因此,用莱恩的话说,税收中哪部分属于“为保护服务支付的费用”,哪部分属于人们更想称之为的“掠夺”,是很难分清的。反正无论如何,人们都要交税,除了满足病态的好奇心,搞那么清楚也没什么好处。就像莱恩说的,不管税收中有多大比例属于“掠夺”,它都是“人们为避免更严重的损失”而必须付出的代价。

\subsection{工业主义下收入的增加}
在过去两个世纪民族国家的统治之下,这种两难选择之所以被容忍,部分原因是人们的收入在急剧上升,特别是在大部分工业发展受到限制的国家或地区。经合组织国家的管理者,每年从社会收入中抽走的比例都在提高。但伴随着更大的掠夺,也出现了更大的繁荣,以及与世界其他地区间的财富不平等。如此一来,反对税收激增的意见必然被边缘化,不足以改变整个趋势的逻辑发展。在前面几章我们分析过,一个工业化民族国家的军事生存,很大程度上就在于,它征收公民资源的能力不会被压制。

在所有的工业国家,它们的政策都或多或少地朝同一个方向发展。在二战后工业化的高潮阶段,很多国家的边际所得税率达到 90\%甚至更高。这是国家对攫取资源的权力的大肆宣扬,比早期水利文明时期东方专制者的做法还要激进。不过,工业版本的掠夺有它自身的逻辑,它主要由 20 世纪上半叶工业技术的特点所决定,对此我们在前面分析过。

工业技术的特点,不可避免地使国家要拿走国民收入的一大部分,并进行重新分配,而被掠夺的大头就落在了少数资本家身上。因为工业发展严重依赖自然资源,与地理位置紧密相连。一个钢铁厂、一个矿场或一个港口要想移动,必须耗费巨大的成本,或者根本就不可能挪动。所以,这些设施就是固定的靶子,很容易被课税。本世纪以来,财产税、公司税和开采税都猛烈增长;所得税也是一样。首先针对资本家,最终还是落到工人身上。大规模的工厂就业使普遍的所得税得以实现。工资可以从源头被扣税,由税务机关和工业公司的会计部门协调征收。今天,我们对这一切已习以为常,相比在乡下从数以百万级的个体工匠或农民身上榨取部分剩余价值,在工厂门口征收所得税确实简单得多。

简而言之,相比历史上许多早期的税收制度,工业技术使征税更加常规化、更可预测,对税吏个人的危险性也更小。但它抽取社会资源的比例,也比任何形式的主权都要高。

\subsection{到底在保护什么?}
从现实来看,工业社会确实变得更加富裕,而税收抽走的收入比例也大幅上升,这就不仅让人问一个问题,关于政府为工业经济所提供的保护,他们到底在保护什么呢?我们的答案是:主要是资本成本高昂又极易受到攻击的工业设施。在一个充满暴力竞争的无序环境种,大型工业企业是不可能存在的,即使竞争可以削减政府征收占社会总产出的比例。

这也是为什么在美国的贫民窟,以及暴力普遍高发的第三世界,不可能出现资本密集型的经济。工业社会作为一个整体,能够得到持续发展,就在于某种秩序已经确立并得以维护。在这种秩序下,企业受到的是定期的、可预测的敲诈勒索,而不是反复无常的暴力洗劫。

另外,即使在工业化的高峰时期,要说政府“垄断了暴力”也有点夸大其词。尽管所有政府都试图进行垄断,但正如我们所见,工业企业的雇员通常也能对其雇主进行暴力威胁。只要普通民众还能够获得武器,暴动人群还有体力掀翻公交车或向警察投掷石块,那些控制政府的人就没有完全垄断暴力。他们只是控制了占支配地位的武力,而这种支配所达到的程度,使大多数人要在现有条件与他们进行竞争,经济上完全不划算。

\begin{tcolorbox}
一个基于网络的政府只能在被统治者的同意下去运行。因此,任何网络政府如果想长治久安,必须为其公民提供真正的利益。这些利益可能不仅是为个人提供的商品或服务,还有管理制度能带来的更多积极的东西:一个干净透明的市场,有着明确的规则与行为后果;或者一个受到监督的社区,孩子们可以信任陌生人,个人隐私能得到保护。
\begin{flushright}
—— 艾斯特·戴森(Esther Dyson)
\end{flushright}
\end{tcolorbox}

\subsection{信息时代}
信息时代是西方暴力竞争史演进的第五个阶段;它不在莱恩的预测之中。这个阶段涉及网络空间的竞争,这是一个不受任何“暴力组织”垄断的领域。它之所以不受垄断,因为它根本不是一块领土。

关于战后民族国家的势不可挡,在莱恩的观点中,尽管有一些比较传统的假设,但他认识到一点,这一点对我们今天理解未来,比四、五十年前看起来更为关键。

那就是政府从来没有在公海上建立过稳定的强制性垄断。想想是不是?没有一个政府的法律在公海上是可以排他的。这个问题极为重要,它有助于我们想象,当经济迁移到网络空间后,暴力与保护的组织将如何演进,因为在网络上根本没有实体存在。没有政府可以在公海上垄断暴力,出于莱恩这个观察所揭示出来的道理,政府更不可能成功垄断没有边界的无限网络空间。

\section{不会陷入无政府状态的竞争}
在以往的大政治条件下,当任何一个暴力组织都无法建立垄断地位时,社会将陷入无政府状态,充斥着掠夺。但信息时代已经深刻地改变了组织暴力的技术条件。

在过去,当某个地区无法形成保护垄断,往往意味着更高的军事成本与更低的经济回报。但信息时代恰恰相反,政府无法垄断网络空间,意味着更低的军事成本与更高的经济回报。这是因为信息技术打开了财产保护的新维度。有史以来第一次,利用信息技术,人们可以在新的疆域内创造并保护财富,令所有在领土上垄断暴力的单个政府都鞭长莫及。

\begin{tcolorbox}
在那些存在多重政治权力和治理单位的国家,如果缺乏中央的、稳定的、不受质疑的权力监督源,那它们就要想方设法去处理边界带来的各种问题。
\begin{flushright}
—— 里斯·戴维斯(Rees Davies)
\end{flushright}
\end{tcolorbox}

\subsection{与边疆的类比}

在某种意义上,网络空间属于受到技术保护的边区,相当于中世纪时期的那些边疆地区。在过去,当领主和国王的实力薄弱,一个或多个领主的权力主张在边境重叠时,就会存在近似于竞争性统治的情况。所以,研究边区的运作方式,就能了解边区法或类似的法律将如何移植到网络世界。

安道尔堪称边区的活化石,存在于法国和西班牙之间;它是一种大政治环境的产物,这块 190 平方英里的土地,处于比利牛斯山脉寒冷且几乎无法进入的区域,使得两个王国谁也无法在这里主宰对方。1278 年,双方达成了一项协议,将安道尔的宗主权分给了当地的法国和西班牙领主,即法国的富瓦(Foix)伯爵和西班牙的乌赫尔(Urgel)主教。他们分别认命一名“代理人”,在安道尔低调地行使最小程度的政府权力,主要就是指挥微不足道的地方民兵,现在是一支警队。

伯爵的角色早已成为了历史,如今在巴黎的法国政府为他代班;它的职责之一就是接受安道尔每年朝贡中的一半,其数额还不够支付一个跳蚤公寓的月租。乌赫尔的主教则继续接受他的那一份贡品,就像他在中世纪的那些前任们一样。

均分的贡品也意味着,在安道尔一直有两个“监督权力”的来源,而不是一个。

传统上,安道尔民事诉讼的上诉,是向乌赫尔圣公会学院或巴黎的最高上诉法院提出。

安道尔模糊不清的政治地位带来的一个结果就是,那里几乎没有颁布过什么法律。七百多年来,安道尔一直享受着极小的政府和无税的待遇。这使它在今天作为一个避税天堂而炙手可热。但仅仅一代人之前,安道尔还是出了名地贫穷。这里曾经树木茂盛,几个世纪以来,居民们为了在寒冬中取暖砍伐了森林。每年11 月至次年 4 月,安道尔全境都会被大雪封住。即使在夏天,这里也很冷,农作物只能在向阳的南坡上生长。通过我们的描述,如果你觉得安道尔毫无吸引力,那么你就抓住了它成功的秘密。在民族国家的时代,安道尔能作为一块封建飞地幸存下来,正是因为它位置偏远,土地贫瘠。

曾几何时,存在许多中世纪的边疆或边区,国家主权在那里交错重叠。这些暴力的边区在欧洲的边境线上持续了几十年,有些甚至是几个世纪;其中大部分都很贫穷。在前面我们提到过,在凯尔特人和英特兰人控制的爱尔兰地区之间,在威尔士和英格兰、苏格兰和英格兰、意大利和法国、法国和西班牙、德国和中欧的斯拉夫人边境之间,以及在西班牙的基督教王国和格拉纳达的伊斯兰王国之间,都存在边区。像安道尔一样,这些边区都形成了独特的制度和法律。在下一个千年,我们有望再次看到这种主权形式。

由于有两个权力机构在互相竞争,各自都比较弱小,统治者有时候会在臣民中征集志愿者到边区定居,以扩大其权力的影响。顺理成章,这些臣民受到免税的诱惑,也会在边区常住下去。鉴于双方在边区竞争的微妙关系,任何一方的当局如果试图征税,都会使他的追随者更难维持生计,从而可能与敌人结盟。因此,在决定要服从谁的法律时,边区居民总是有选择的余地。这种选择是基于竞争性当局的弱点,而不是某种意识形态的姿态。

尽管如此,还是有一些实际可能需要解决。封建制度下,边界往往只是名义上的,在边界两边都拥有财产的地主往往会面临严重的责任冲突。例如,位于苏格兰和英格兰边境的领主,如果在两个王国都拥有财产,在发生战争时,理论上他就对两个王国都负有兵役义务。为了解决这种义务矛盾,在封建制度的上上下下,几乎每个人都可以通过一种叫作“声明”(avowal)的法律程序,宣布决定遵守哪个王国的法律。

信息技术为经济活动在决定自己的住所方面,创造了同样的选择机会;但与边区也有很大的区别。其中之一就是,与中世纪的边区社会不同,网络空间有望在未来成为最富裕的经济领域。它会成为一个不断增长而非日益衰退的边疆。在中世纪的核心地区,如果没有很强的诱因,包括宗教性的强制,很少有人愿意搬迁到边区,因为那里又穷又乱。因此,边区无法将资源从当局手中吸引出来,像磁铁一样;而网络空间会实现这样的效果。

其次,在新的边疆,不会出现双头垄断。双头垄断会招致双方当局互相勾结,在边区治理上沆瀣一气。在中世纪时期,这种妥协效果不大,原因有两个:一是敌对当局之间经常存在着尖锐的文化隔阂;更重要的是,他们在当地的军事力量不足,缺乏推进及落实谈判协议的实际能力。在民族国家时代,当国家当局确实有足够的军事力量强行解决问题时,大多数边区和模糊的边界都消失了。固定的边界成为常态。暴力的双头垄断者如果必须在临界区域划分权力,固定边界就是一个很好的解决方式。但是,对网络经济中住所的竞争,不是发生在两个政府之间,而是在全球数百个政府之间。对领土国家而言,想在网上建立有效的垄断来保持很高的税率,是完全不可能的。同理,几百个竞争者想串通起来在市场上享受垄断价格,也是行不通的。

以塞舌尔为证,这个印度洋上的小国,最近颁布了新的投资法案,被美国政府官员描述为“开门揖盗”。根据这项法律,任何在塞舌尔投资 1000 万美元的人,不仅得到保证不被引渡,还将获得一本外交护照。然而,与美国政府的断言相反,有意向的受益人并不是毒贩,而是那些被政治不正确的独立企业家;毒贩不管在什么时候,通常都会受到更强大政府的保护。

塞舌尔法律第一个可能的受益者是一位南非白人,在南非前种族隔离的政权下,他通过规避经济制裁而变得富有。现在他面临着南非新政府的报复,因而愿意向塞舌尔支付保护费。

不管每个个案是什么情况,这个例子说明了,政府想要继续保持那种在地面上的垄断,注定将一败涂地。与中世纪的边界不同,那时候的竞争只发生在两个政府之间;网络经济的边界将涉及到数百个司法管辖区,可能很快会上升到数千个。

在虚拟公司时代,个人将选择一个能以最低成本提供最佳服务的司法管辖区作为公司住所,进行经济创收。换言之,主权将被商业化。中世纪的边区,大部分又穷又乱;而网络时代的绝非如此。信息技术促使政府进入的竞争,不是军事方面的,而是关于经济服务的质量与价格,关于真正的保护服务。简而言之,在信息时代,政府将不得不为客户提供他们想要的东西。

\subsection{暴力效用的减弱}
当然,这并不是说政府会放弃使用暴力。远非如此。我们要说的是,暴力正在失去它强大的杠杆作用。政府对此的反应,可能是加强在当地的暴力运用,以弥补它在全球范围日益下降的重要性。不过,无论政府怎么折腾,它们都不可能将暴力充满整个网络空间,就像在现代世界,在它们垄断的领土上只手遮天。不管有多少家政府进入到网络空间,它们在这里不会比其他竞争者更优越。

讽刺的是,有些民族国家可能会发动“信息战争”,以支配或阻挠人们对网络空间的访问,而这只会加速它们自己的灭亡。由于规模经济效应的下降,以及把日益分裂的社会凝聚在一起的成本上升,大型系统的权力下放势在必行。信息战争的讽刺性在于,它对工业时代遗留下来的脆弱系统本身的冲击,可能比对新兴信息经济的冲击还要大。

只要基本的信息技术持续运行,网络商务就能在与信息战的斗争中同步发展;而这在领土战争中是不可能的。在二十世纪的战争中,你无法想象,前线一边打战还一边在进行数百万项商业交易。但虚拟战争不会耗尽网络空间承载多重活动的能力。而且虚拟现实并不存在,没有近距离交火的危险,也不会被爆炸的虚拟弹片所击中。

\subsection{大规模系统的脆弱性}
信息战的危险,主要集中在那些中央指挥控制下的大规模工业系统。美国和其他主要民族国家的军事领导部门,都很担心并在计划应对信息破坏活动;因为一旦大型系统被瘫痪掉,后果不堪设想。通过网络战,可以关停一个电话交换站,搅乱空中交通管制,或破坏掉调节城市用水的抽水系统。一个病毒程序可以关闭普通发电机甚至核发电机,断掉部分电网。所谓的逻辑炸弹可以扰乱各种信息,最容易受到影响的,就是还在使用从工业时代继承下来的、脆弱的大规模系统的中央控制系统。除非全面摧毁所有的信息技术,使世界经济完全陷入停滞,否则,任何政府都不可能扼制网络经济和虚拟现实的发展,更用说垄断了。

信息技术确实也有明显的缺陷,就是它的存储系统,特别容易被腐蚀或破坏;不过,这一点已经被新的存档技术所解决。一种被称为“高密度只读存储器”或“HD-ROM”的新系统,使用了某种与计算机辅助制造系统中相类似的离子磨,可以在真空中创建档案。它现在的存储容量已高达每平方英寸 25,000 兆字节。

先前的存储系统,很容易发生早期衰变,也容易在冲击下被破坏;而存储在HD-ROM 上的数据,则有望长久保留。HD-ROM 的开发者之一布鲁斯·拉马丁说:“它几乎不受时间、热、机械冲击,及其他对存储介质有破坏性的电磁场的影响。”即使恐怖分子引爆核弹,也不一定能破坏或抹掉上面的重要信息,如数字货币的密码,而网络经济能否顺利运行就有赖于此。

\begin{tcolorbox}
现代军队严重依赖于信息,无需通过常规意义上的战争手段,就可以将其致聋和致哑,从而获得胜利。
\begin{flushright}
—— 阿兰·坎彭(COL. ALAN CAMPEN, U.S.A.F. (Ret.))
\end{flushright}
\end{tcolorbox}


\section{虚拟战争中的超级力量}
随着信息在战争中日益重要,民族国家间相互交战,在大政治层面越来越没有意义。因为网络空间没有物理存在,它的战略重要性,不是我们在物理世界熟悉的那种量级。有多少个程序员确保命令的执行并不重要。一切只在于这个程序能否发挥作用。在网络的世界里,主权个人的地位可能不亚于一个国家,他在联合国有自己的席位,有自己的旗帜,有部署在地面的军队。从纯经济的角度看,一些主权个人每年有数亿美元的可投资收入,超过某些破产的民族国家可支配的消费能力。但这还不是全部。在操纵信息发动虚拟战争方面,一些个人可能和世界上的许多国家一样强大,甚至更强。在网络战争中,一个天才怪客,利用数字奴仆,理论上可以达到与民族国家同样的战斗力。比尔·盖茨肯定可以。

从这个意义上说,主权个人的时代绝不仅仅是一个口号。五角大楼网络战争部队努力所做的一切,一个黑客、一小群数学家,更不用说像微软这样的公司,或者其他任何软件公司,原则上都可以做到。在硅谷和其他地方,有成百上千家公司,他们所拥有的网络战争的能力,超过现在 90\%的民族国家。

1998 年,美国总统和他高级助理的声明,证实了这一点。他们认为,美国的主要敌人不是另一个民族国家,而是流亡的沙特亿万富翁,乌萨马·本·拉登。本·拉登作为一个个人,居然对工业时代军力最强的国家构成了巨大威胁;美国以一连串的巡航导弹支持了这一说法。我们并不完全相信这个说法,本·拉登可能只是被选出来扮演“内罗毕爆炸案中的理查德·朱尔”(一个借口,译注)的角色。

尽管如此,当所有人都相信,本·拉登作为恐怖分子,可以成为美国的可怕敌人;那么,拉登或者其他亿万富翁,如果作为网络恐怖分子,会更有说服力,因为信息技术使个人在面对大型团体时的劣势,要远小于他在炸药和导弹方面的劣势。

当新的保护方式从四面八方涌现,民族国家还想保持它们在地面上的垄断,将变得不合时宜。更有可能出现的结果是,民族国家必须进行重组,以降低它们对计算机病毒、逻辑炸弹、被感染的网络以及可能由美国国家安全局或一些青少年黑客布置的木马程序的脆弱程度。

网络空间的大政治逻辑表明,目前主宰着全球大规模基础设施的中央指挥控制系统,将不得不被具有分布式运作能力的多中心安全模式所取代,因为它们不会轻易地被计算机病毒控制或封锁。新型的软件,即 agoric 开放系统,将取代从工业时代的指挥控制系统。以前的软件,根据僵硬的优先次序分配计算能力,就像前苏联国家计划委员会的老干部,根据死板的规定为货物分配厢式货车一样。新的系统由模拟市场机制的算法控制,通过一种模仿大脑运转的内部竞价进程,从而更高效地分配资源。在新的千年里,计算机不再是巨型并垄断着指挥控制重要职能的,而是去中心化的。

数字设备公司在帕洛阿尔托实验室的研究,完美地证明了分布式网络相对于指挥控制系统的复原能力。根据凯文·凯利的叙述,一位工程师打开壁橱的门,里面是他们自己公司的计算机网络,然后,这个工程师非常戏剧性地“从里面拽掉一个网线,而公司网络绕过了这一破坏,没有受到任何影响。”信息时代将促进网络空间的竞争,但不会出现无政府状态;而工业化时期遗留下来的重要系统,必须被重新配置。这种重置至关重要,可以使它们没那么容易受到各种恶意的攻击。中世纪遗留下来的机构,如学校和大学,在工业时代不可避免地遭到了重置;同样,工业时代的机构也将被分解成微型的形式,反映微技术的逻辑。

为了防止信息时代高速公路上的盗贼,公钥-私钥的加密算法需要被广泛采用。

这些算法可以使任何个人电脑用户对任何信息进行加密,比五角大楼在一代人之前保存其发射密码的方法更安全。这种强大的、牢不可破的加密方式,对确保金融交易免受黑客与小偷的攻击是完全必要的。

这种必要性还源于另外一个原因。当私人金融机构和中央银行意识到,美国政府——可能不止它一家——有能力穿透目前的银行软件和计算机系统,把一个国家搞破产,或者清空任何人的银行账户,不管他在哪里,他们就会使用这种无法破解的加密算法。任何个人或国家都没有理由,在技术上把他们的存款或金融交易,置于美国国家安全局或克格勃的继承者或其他任何合法、非法组织的摆布之下。

政府无法穿透的加密算法并不是白日梦。它们已经在网上作为共享软件使用。当低轨道卫星系统可以运行时,利用先进的个人电脑和比便携电话还小的天线,个人可以在全球任何地方进行通信,甚至不需要连接电话系统。政府不可能垄断完全没有物理存在的网络空间,就像中世纪的骑士,不可能骑着笨重的军马去控制工业时代的交易。

\subsection{隐蔽的保护}
信息社会将使大量的资源超越于掠夺之外。当网络空间越来越多地成为金融交易和其他商业的活动场所,其中的资源就或多或少地对普通的敲诈和盗窃产生了免疫。掠夺者无法再像今天和 20 世纪的大部分时间里那样,控制大量的资源。

这样一来,对于世界上的大部分财富,政府的保护都将沦为多余。对于保护网络银行的账户余额,政府不会比你做得更好。政府越来越没有存在的必要,这一点就会使它的保护价格相应地降低。当然还有其他原因。

在新的千年里,随着发生在网络空间的金融交易的份额越来越大,个人将可以自由选择交易的司法管辖区。这会导致强烈的竞争,人们将开始在非垄断的基础上为政府的服务定价(它收的税)。这是革命性的。正如乔治·梅卢安(Gorge Melloan)在《华尔街日报》上所说,抵抗全球性竞争最成功的一种组织就是福利国家。“沃顿商学院和澳大利亚国立大学的研究人员,进行了一项研究,探讨了未来会影响到收入转移的力量。杰弗里·加勒特和德波拉·米切尔的结论是:几乎没有证据表明,市场一体化的加强给它们最基本的福利项目带来了下行压力。相反,他们写到,‘这些政府无一例外地通过增加收入转移,应对与国际市场的日益融合。’” 终于,网络经济的出现,总算将福利国家暴露在了真正的竞争之中。它将改写主权的性质,变革经济的形式,因为保护与勒索的天平,将前所未有地彻底倒在保护的一边。