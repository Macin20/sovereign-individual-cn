\chapter[政治最后的日子]{政治最后的日子:\\ 圣母教会与保姆国家衰落的相似性}

\begin{tcolorbox}
\kaishu 我相信并且希望,政治和经济在未来不再像在过去那样重要。中世纪最敏锐的头脑曾经把精力耗费在神学辩论中,如今看来已是微不足道或毫无意义;我们现在关于政治和经济的大部分争论,当有一天看起来和中世纪的神学辩论一样,那个未来就来临了。
\begin{flushright}
—— 阿瑟·C·克拉克
\end{flushright}
\end{tcolorbox}

如果说政治即将消亡,肯定会被认为荒谬或者乐观,这取决于每个人的态度。但这就是信息革命可能导致的后果。20 世纪是一个政治氛围浓厚的世纪,对于在该世纪中长大的读者来说,认为生活中可以没有政治,简直就是异想天开,这就好像说一个人只要用空气中吸收营养就能活下去。然而,现代意义上的政治,作为控制国家权力并将其合理化的一套思想,它主要是一种现代发明。我们相信它也会随着现代社会一起结束;就像中世纪的人们纠葛的封建责任和义务,随着中世纪一起消亡一样。就像历史学家马丁·凡·克瑞福德\footnote{当代著名的军事历史学家和军事思想家,1946年出生于荷兰鹿特丹的一个犹太家庭,曾长期在耶路撒冷希伯来大学历史系执教,也是美国海军战争学院的客座教授。克瑞福德教授著作等身,发表过30余部专著,其中颇为著名的包括:《战争后勤》(1977年)、《战争指挥》(1985年)、《科技与战争》(1989年)、《战争的转变》(1991年)等。}(Martin van Creveld)所指出的,在封建时期,“政治并不存在”。我们现在所了解的政治在现代时期之前并不存在,这种说法可能会令人感到讶异;毕竟,在亚历山大大帝的时代,亚里士多德就写过一篇以此为题的文章。但是仔细想一下,古代文本中的词语并不一定等同于当代的概念。亚里士多德也写过一篇题为《辩谬篇》\footnote{《辩谬篇》(拉丁文:De Sophisticis Elenchis,英文:On Sophistical Refutations)是古希腊哲学家斯吉塔拉人亚里士多德创作的哲学著作,收录于其逻辑学著作集《工具论》中。该书另包含《范畴篇》《解释篇》《前分析篇》《后分析篇》《论题篇》《辨谬篇》等六篇。}的文章,这个词在今天就好像“政治”在中世纪一样毫无意义,当时还没有人使用它。它最早在英语中出现是 1529 年,而且在那是它好像是个贬义词,来自于古法语的一个单词 politique,用来描述“机会主义者和见风使舵者”。

近两千年后,亚里士多德的概念才出现了我们现在所知的内涵。原因何在?因为在现代社会将亚里士多德的词语赋予实质的用途之前,需要满足相应的大政治条件,这些条件可以极大地提高暴力的回报。我们在《大清算》中分析过,这就是火药革命,它使暴力的回报率远远超过了以往。在这种情况下,国家控制在谁的手里,就变得前所未有地重要。从逻辑上讲,运用权力可以获得的战利品急剧增加,在控制这些战利品的斗争中,不可避免地就产生了政治。

政治开始于五个世纪之前的工业化早期,如今它正在消亡。对政治和政客的普遍反感正席卷全球。这一点,你可以从对白水公司的报道以及对隐藏细节的猜测中看到,可以从文森特·福特谋杀案的拙劣掩盖中看到。从比尔·克林顿海量的丑闻中;从主要国会议员挪用众议院邮局公款的报道中;从导致约翰·梅杰圈内人辞职的丑闻中;法国最近的两任总理,爱德华·巴拉迪尔和阿兰·朱佩,也爆发了类似的丑闻。意大利,丑闻更大,做了七任总理的吉奥·安得利奥迪被送上被告席接受审判,罪名包括与黑手党有染,还下令谋杀调查记者米诺·佩克雷利。

西班牙首相菲利普·冈萨雷斯,也因为一些丑闻名誉受损。在 20 世纪 90 年代的前 5 年,四位日本首相因为腐败指控而下台。加拿大司法部在给瑞士当局的一封信中指称,其前总理布莱恩·穆罗尼在加拿大航空公司 18 亿加元的空客合同中吃了回扣。甚至在瑞典,副总理和准总理莫娜·萨赫林,因为被指控使用政府信用卡购买尿布和其他家庭用品而被迫辞职。在那些拥有成熟福利的国家,在几乎所有治理良好的国家,无论你走到哪儿,人们都讨厌他们的政治领导人。

\subsection{蔑视是一种先行指标}
对腐败领导人的道德愤怒并不是孤立的历史现象,而是一种常见的变革先兆。当一个时代让位给另一个时代时,这种现象就会一再发生。每当技术的进步导致旧体制与新的经济驱动力脱离,社会的道德标准就相应地发生改变,人们就会对那些旧制度的掌控者越来越不屑。而且,在新的、系统性的革命意识形态形成之前,这种普遍的反感就已经出现了。在我们写这段文字的时候,还没有明确的证据表明人们对政治的抗拒。它会在后面发生。你们同时代的大多数人还想象不到,没有政治的生活是可能的。在二十世纪的最后几年,我们所感受到的还只是无言的蔑视。

类似的情景也发生在 15 世纪末,不同的是,当时处于衰退之中的,是宗教而不是政治。尽管人们还相信“圣职者的神圣性”,但无论高级的还是低级的神职人员,都遭到了极大的蔑视,与今天人们对待政客和官僚的态度并无二致。当时的人们普遍认为,上层神职人员腐败、世俗、贪婪。这不是没有原因的,15 世纪的几个教皇都公开拥有私生子。下层神职人员的地位更低,他们在乡下和城里泛滥成灾,乞讨施舍,并向任何愿意给钱的人兜售上帝的恩典和罪孽的宽恕。

在“表明虔诚的外壳”之下,是一个腐败的、日益失灵的系统。在有人敢于指出这一点之前,人们已经不再尊重它的管理者。宗教充斥着生活的方方面面,不分精神和世俗,它的可能性已经耗尽了。早在马丁·路德将他的《九十五条论纲》钉在维登堡教堂的门上之前\footnote{1517年10月,德国维登堡教堂的大门口张贴了马丁·路德批判教会的文章。文章中说“只要真诚忏悔,不购买赎罪券也可以免罪”。这就像火花落进了火药桶,整个德国都爆炸了。},它的结局就已经注定了。

\section{世俗改革}
我们相信,政治的饱和也会导致同样的变革。苏联的灭亡和对社会主义的否定,就是席卷全球的去政治化格局中的一部分。这一点在人们对各国政府领导人的日益蔑视上,体现得最为明显。部分原因是,人们认识到政客是腐败的,他们经常通过出售政治“赎罪券”以换取竞选资金,或者在商品交易中提供特殊服务,以充实自己的财政。

另外一部分原因是,人们越来越认识到,政客花费巨大代价所做的一切都是无益的。就像在 15 世纪末期,组织忏悔者再一次朝圣,在雪地里赤脚前行;或者再成立一个托钵修道会一样,都是徒劳的,对提高生产力或者缓解生活压力,毫无帮助。

\subsection{圣母教会最后的日子}
到中世纪末期,一元化的教会机构已经衰老不堪,并且严重影响生产力的发展。

这与它在五个世纪前对经济的积极贡献形成鲜明的对比。在上一章中我们探讨过,在 10 世纪末,在教会主导下,城乡秩序得以恢复,经济得以发展,整个社会从无政府状态中走了出来,这也标志着黑暗时代的结束。在当时,对于构成西欧大部分人口的自由民和农奴来说,教会的存在是不可或缺的。而到 15 世纪末,教会已经成为了生产力的严重阻碍,它强加的负担大大降低了人们的生活水准。

今天的民族国家也是如此。五个世纪前的火药革命创造了新的大政治条件,民族国家是对此的必要适应。民族国家取代了支离破碎的地方势力,扩大了市场范围,更大的贸易区带来了更丰厚的回报。在整个欧洲,当中央的君主在巩固权威时,几乎所有的商人都会自发地与他结盟。这一事实本身就足以说明,民族国家在早期的形式下,对商业发展是大有裨益的,它有助于减轻封建领主和地方豪强对商人的压迫。

在暴力回报很高且不断上升的世界里,民族国家是一个很有用的机构。但是 500年过去了,在这个千年的末尾,大政治的环境以及发生了改变。使用暴力的回报率正在下降,民族国家成为了经济增长和生产力发展的阻力,就像中世纪暮年的教会一样,它已经不合时宜了。

和当年的教会一样,今天的民族国家也穷尽了它的可能性。它已经到了衰老的顶点,是一种破产的机构。和当年的教会一样,作为社会组织的主导形式,民族国家也已经存在了五个世纪。促使它诞生的条件已经不复存在,它已经熟透可以坠落了。而它也必将坠落。技术发展正在酝酿一场新的权力革命,毋庸置疑,它必将摧毁民族国家,就像火药武器和印刷术摧毁中世纪教会的垄断一样。

如果我们的推理无误,\uline{民族国家将被新的主权形式所取代}。有些是历史上前所未有的,有些则会让人联想到前现代社会的城邦国家和中世纪的商人共和国。到2000 年之后,旧的将变成新的;难以想象的将变成习以为常的。随着技术规模的急剧下降,政府会发现它们将不得不像公司一样为收入而竞争,它们收入的服务费不会再超过它们所能提供的价值。这种变革的深远影响,现在难以估量。

\section{过去和现在}
五百年前,15 世纪之交,可能有人说过类似的话。当时和现在一样,西方文明站在一个重大变革的门槛上。虽然人们还没有认识到,但中世纪社会正在消亡。

他们既没有预见,也理解不了。普遍的情绪是一种深深的忧郁,这在一个时代结束时很常见,因为传统的思想家感觉到事态正在崩溃,“猎鹰听不到猎鹰人的哨声”。但是他们的思维惯性太大,无法理解正在出现的权力结构的潜在影响。

中世纪历史学家约翰·惠泽加,在谈到中世纪的衰落时写道:“15 世纪的编年史学家们,几乎全都被其时代的误解而蒙骗了,社会演进的真正动力,他们都没有注意到。”

\subsection{被背叛的神话}
当驱动权力的基本动力发生重大的变化,传统的思想家往往会迷惑不解,因为它揭露出,使旧秩序合理化的神话,其实缺乏真正的解释力。中世纪末和今天一样,在人们普遍接受的神话和现实之间存在巨大的差距。惠泽加在谈到 15 世纪末的欧洲人时说:“他们的整个思想体系都充斥这一种虚构,认为是骑士精神统治着世界。”在当今世界,有一种假想与此异曲同工,那就是认为世界是由选票和支持率所统治。这两种观点都经不起仔细推敲。事实上,认为历史的进程是由民主的意愿所决定的,这种想法和中世纪的观念一样愚蠢;那时候的人们认为,历史是由一套精心设计的礼仪规范决定的,也就是骑士精神。

这样的说法近乎异端,这也表明了,传统思维与工业社会晚期权力动态的现实之间是多么的脱节。在我们看来,投票是导致现代民族国家产生大政治条件所带来的一个结果,而不是原因。大众民主和公民概念,随着民族国家的发展而日益丰富,也必将随之一起衰落。华盛顿将面临的不安和骚动,就像 500 年前骑士精神失格后,勃艮第公爵的宫廷中所遭遇的一样。

\section{骑士精神与公民身份的相似之处}
如果你能理解骑士精神誓言的重要性,以及为什么随着社会向工业组织的过渡,它会逐渐消失;那么,你就能更好地理解,我们今天所熟知的公民身份,将如何在信息时代逐渐退出历史舞台。这两者有着同样的功能,在两套完全不同的大政治环境中促进了权力的行使。

在封建誓约盛行的时代,防御性的技术是最重要的。那时候主权很分散,私人和法人团体都凭自身的实力行使军事权力。在火药革命之前,战争通常都是小规模的武装人员进行的。即使是最强大的君主,也没有长期的军事力量,即常备军。

他需要从他的附庸——大领主那里得到支持,大领主又要从他的附庸——小领主那里寻求支持,小领主又要从他的附庸——骑士那里获得支持。整个效忠的链条沿着等级制度向下延伸,直到被认为有资格携带武器的、社会地位最卑微的人为止。

\subsection{制服还是区分?}
与现代军队不同,在公民身份兴起之前的中世纪,军队在战场并没有穿着统一的制服。相反,每个家臣或随从,每个骑士、男爵或不同身份的领主,都有自己独特的装束,用以反映他在等级制度中的地位。与其说是服装,不如说是对社会垂直结构的强调,在这个结构中,每个位置都是不同的。惠泽加指出,中世纪的士兵主要通过以下方式做区分,“外在的分别标志主要有:着装、颜色、旗帜以及口号”。

战争也不仅仅发生在政府或国家之间。马丁·凡·克利福德认为,卡尔·冯·克劳塞维茨等战略家所塑造的现代战争概念,扭曲了前现代时期军事冲突的性质。

克利福德写道:

\begin{tcolorbox}
\kaishu 在罗马灭亡后的一千年里,武装冲突可以由不同类型的社会实体发动。其中包括野蛮人的部落、教会、各种等级的封建男爵、自由的城邦,甚至是私人。这一时期的“军队”与我们今天所知的军队也不一样,实际上,很难找到一个合适的词描述它。那时的战争,往往是由一群披上战装、跟随领主的家臣们发动的。
\end{tcolorbox}

在这样的情况下,对领主来说,他的家臣能真的“披上战装,跟在后面”显然是至关重要的。因此,骑士精神的誓言就受到高度的推崇。中世纪骑士的荣誉和今天服兵役的义务具有类似的功能。中世纪的人,被他向别人和教会许下的誓言所约束,就像今天的人被民族国家的公民身份所约束一样。在中世纪违反誓言,就相当于今天的叛国罪。中世纪晚期的人们,为了不打破自己的誓言不惜走向极端,就像在世界大战期间,数百万的现代人为了履行公民义务而冲进机枪窝里。

任何头脑未受过灌输的人,都不会愿意投身到战场,并且在形势恶劣的情况下还留在那里。而骑士精神和公民身份为这种简单的盘算,增加了一个维度,提高了计算难度。骑士精神和公民意识都引导人们去杀人、去冒死。能够发挥这种作用的价值观,只能是经由统治机构大力强化及高度夸张过的。

\subsection{规避成本效益分析}
任何制度的成功与生存,都取决于它在危机时分调集军事力量的能力。很显然,中世纪的骑士或一战壕沟里的士兵,他们决定在战斗中冒生命危险,不可能经过冷静的成本计算。很少有战争是这么容易打赢的,对那些愿意冲锋在先的士兵,战争所给予的回报,也很少能够远超其所付出的代价。不然的话,统治者就可以招募一支经济最优化的部队,送到前线去冲锋陷阵。几乎每一场战争,或者大多数战争,都有一个瞬间扭转局面的时刻。研究军事史的学生都清楚,战斗的胜利和失败往往一线之隔,就取决于个别士兵的英勇、果敢和凶猛。一块阵地,不过是一块在战斗结束后一文不值的土地,如果打仗的人不愿意为这样一块地而牺牲,他们就很可能战胜不了原本势均力敌的对手。

这一点,意义深刻。主权国家在限制叛逃和鼓励牺牲方面做得越成功,就越有可能在军事上取得胜利。在战争中,最有效的价值体系会诱导人们从事某些行为,而这些行为只要稍加理性思考就不可能被接受。被送到战场上的士兵,如果可以自由地计算对自己最有利的选择是什么,并以此决定是打是逃,那么任何组织都无法有效地调动军力。\uline{能够理性思考的人,就永远不会参与战争}。一个理性的人,如果分析过短期的成本和收益,还决定参加一场致命的战争,那只可能是在最有利的条件下,或者在最绝望的形势下。在一个阳光灿烂的日子,当己方的军力占据压倒性的优势,敌人不堪一击,而战利品极其丰厚的情况下,一个理性经济人可能会加入战斗。也许吧。而如果他被食人族逼到墙角的话,应该也会奋起抗争。

这些都是极端的情况,更常见的战争呢?往往既没有诱人的战利品,也无法通过成本和效益进行分析,也没有绝望到无路可走的地步。正是在这里,骑士精神和公民身份等概念,为发动军事力量提供了强力的支撑。早在战争发生之前,统治集团就必须让个人相信,坚守对领主或民族国家的某些职责,比生命本身更重要。

在社会上编织神话,鼓励到战场上冒险与牺牲,将其合理化,是统治者军事力量的一个关键组成部分。这些神话要想有效,就必须根据当时的大政治条件进行调整。骑士精神统治世界的虚构在今天毫无意义,特别是在纽约这样的城市。但它是封建主义顶礼膜拜的神话。它使义务的纽带合理化,这个纽带将每个人都束缚在教会及好战的贵族的统治之下。在那个贪恋财富、私人战争司空见惯的时代,权力的行使与个体的生存,都取决于他人是否愿意在威胁之下履行兵役的承诺。这些承诺是否可靠,显然是生死攸关的。

\subsection{在国籍出现之前}
与今天不同的是,在中世纪,国籍的概念对于确立主权几乎没有任何作用。在那个时代,君主、红衣主教以及有权势的领主们,以私人权利的形式拥有领土。他们能以现代人无法比拟的方式,出售或赠送领地,或者通过转让、联姻及征服获得新的领土。今天你很难想象,一个不会讲英语的葡萄牙总统碰巧娶了美国前总统的女儿,美国就会归于葡萄牙的主权之下。但这样的事情在中世纪是很常见的。

权力通过血统世袭传承,城市和国家更换主权,就好像古董更换主人一样。很多时候,君主并不是他财产所在地的本地人,他不会讲当地的语言,或者讲得很糟糕,口音很重。但是,不管雅典的国王是西班牙人,还是西班牙的国王是奥地利人,这对个人所承受的义务都没有任何影响。

\subsection{社团主权}
圣殿骑士团、圣约翰骑士团、条顿骑士团和马耳他骑士团等宗教团体也可以行使主权。这种混合型的机构在今天找不到对应的产物;虽然马耳他骑士团还依然存在,而且在我们写这本书的时候,他们正准备恢复对马耳他圣安杰洛堡的主权。

这些机构将宗教、社会、司法和金融活动与对地方的主权结合了起来。虽然他们也行使领土管辖权,但与今天的政府完全相反,他们赢得支持及推进管理计划时,完全不是依靠国籍身份。这些宗教团体的成员和管理者,来自欧洲所有信奉基督教的地区,即“基督教世界”。

没有人认为应该或者必须从当地居民中选拔统治者。在支离破碎的中世纪主权格局中,要动员支持者,不像现代那样取决于民族身份或对国家的责任,而是取决于个人的忠诚和习惯性的联系,这些东西必须作为个人荣誉去维护。来自任何地方的任何人,都可以宣誓遵从这些义务,只要他的社会地位满足相应的资格。

\subsection{宣誓}
骑士的誓言将人们联结到了一起,并以当事人的名誉起誓。就像惠泽加所写的,“立下誓言,人们就对自己施加了一些束缚,作为履行承诺的动力。”当时的人对遵守誓言看得非常重,为了避免失信于人,他们经常冒着生命的危险或者遭受严重的后果。遵守誓言被视为一种个人荣誉,约束着人的行为。而其中某些行为,在本书的读者看来,可能是很滑稽的。

例如,星辰骑士团曾经发誓,绝不“从战场上撤退超过 4 英里。而不久之后,他们中的九十多人就因为这一誓言而送命。”作为一种军事战略,连战术性的撤退都禁止可能是不合理的,但所有的骑士誓言都有这样的要求。在阿金库尔战役之前,英格兰国王发布了一道命令,要求巡逻的骑士们脱掉盔甲;因为如果他们穿着斗篷盔甲从敌阵上撤退,那就太有失荣耀了。而碰巧的是,国王本人迷路了,路过他的先锋部队夜宿的村庄。穿着盔甲的国王,在意识到自己的错误后,他的骑士荣誉感使他无颜走进那个村庄,只好暴露在危险之中,在外面度过了一夜。

这个例子尽管看起来很傻,但亨利国王的盘算可能并没有错。相比在敌人后方睡觉的危险,撤退会辱没自己的荣誉,并为整个军队树立榜样,打击士气,这种损失要大得多。

中世纪的历史上充满了各种事例,都是杰出的人物在履行那些很荒谬的誓言。在很多情况下,他们做的事并不能带来任何客观上的利益,只是生动地表现了宣誓人对誓言本身的重视。在这些誓言中,常见的情况有:如果他们加入了只有一个人的锁链囚徒帮(one-person chain gang),就要遮住一只眼睛,只在站立的时候吃喝,并且要把自己搞残废,变成瘸子。当时有一种普遍的习俗,就是戴着痛苦的脚镣行走。今天如果你看到有人戴着沉重的脚镣在大街上挣扎,你肯定会觉得他疯了,而不会认为他是个大德之人。但在骑士精神的背景下,心甘情愿地戴上这样的装置是一种荣誉的象征。还有很多类似的习俗,在今天看来都很可笑。根据惠泽加的描述,许多人发誓“周五不吃肉,周六不睡床,一个又一个禁欲主义的行为变本加厉。一个贵族承诺:永不穿盔甲,每周有一天不喝酒,不在床上睡觉,不站着吃饭,并穿着刚毛衬衣。”相比这些自我折磨,斋戒显得都温和多了。

很多狂热的誓言分子成立了一些教团,对他们的成员施加严厉的苦修,作为一种荣誉的考验。例如,克拉洛瓦和加洛伊奇骑士团,在夏季“穿着毛皮和毛皮衬里的帽子,还在壁炉里烧着火。在冬天,则只允许穿一件没有毛皮的单薄外套,不能穿斗篷,也不能戴帽子和手套,床单也很薄。”正如惠泽加所说,“大量的成员被冻死也就不足为奇了。”

\begin{tcolorbox}
\kaishu 中世纪的自虐是一种严酷的折磨。人们对自己施加这种折磨,是希望审判和惩罚的上帝能收起棍子,宽恕他们的罪行,使他们免受在今生和来世都应得的更大惩罚。
\begin{flushright}
—— 诺曼·库恩(NORMAN COHN)
\end{flushright}
\end{tcolorbox}


\subsection{鞭刑,过去和现在}
从施加危险和清苦的誓言,到考验、朝圣、折磨、不适,甚至是故意的自残,这中间只是很短的一步。在中世纪,这些行为都被认为是非常有益的,值得高度赞扬。它们象征着宣誓的严肃性,这种逻辑,在今天的兄弟会或联谊会的入会仪式中依然适用。

夏天闷热,冬天冰冷,或者在雪地里赤脚朝圣,这样的折磨,比起自我鞭挞的严酷,都是小巫见大巫。自我鞭挞是中世纪很特别的一种忏悔方式,它与封建主义差不多是同时出现,最早是“在 11 世纪初被卡马尔多利和阿维拉纳修道院的隐士们所采用。”鞭笞者不只是在寒冷的天气里赤脚行走,他们组织游行,日夜不停的赶路,从一个城镇到另一个城镇。“每当他们到达一个城镇,就会在教堂前排成一排,连续鞭打自己几个小时。”我们相信,当未来的人们回顾民族国家的时代,会发现 20 世纪的人以公民的名义所做的事情,和我们看待自我鞭刑一样可笑。从信息社会的角度看,现代时期的士兵出于对民族国家的忠诚而穿越半个地球去赴死,是一种怪诞又愚蠢的景象。它和那些异常夸张的骑士仪式并没有太大的区别,比如戴着脚镣走来走去;而在封建时期,即使是那些理智的人,也以此为荣。

\subsection{骑士精神让位于公民责任}
当大政治的条件发生了改变,宣誓效忠领主的军事意义已经过时,骑士精神就逐渐式微了,被公民责任所取代。在火药武器和工业化军队的世界里,打仗的人和他们的指挥官之间的关系就非常不同了。公民身份出现在暴力回报率很高且不断攀升的时期,相比中世纪发动战争的社会实体,这时候的国家拥有更多的资源。

民族国家以其强大的实力和财富,可以直接与穿着制服作战的普通士兵进行交易。

历史证明,这种交易对国家来说要便宜得多,而且,相比通过与有权有势的领主和地方豪强谈判以组建军队,也要省事得多;因为这些人每一个都有能力拒绝与自身利益相悖的要求,而民族国家的个体公民则做不到这一点。

公民身份的关键取决于一个事实,那就是没有任何个人或小的团体,在大政治层面,具备独立行使军权的能力。在后面我们会深入探讨这其中的原因。

随着信息技术改变打仗的逻辑,它必将使公民身份的神话破产;就像火药使中世纪的骑士精神过时一样。

\subsection{马背上的地狱天使}
在西欧主宰了几个世纪的骑兵贵族,可算不上是他们后代那样的绅士。他们是粗俗的、暴力的;用今天的话说,可以把他们理解为中世纪的机车帮派。礼仪规则和骑士精神的伪装,更多是用来约束他们的过激行为,而不是描述他们真正的作风。即使对骑士精神的规则和义务进行百科全书式地描述,也很少或根本无法揭示出贵族权力的来源。

\subsection{完美是枯竭的同义词}
十五世纪末,就在武装骑士刚刚把他们的战斗技艺提高到臻于完美的境界时,有效的火药武器出现了,在他们的脚底下引发了一场强烈的爆炸。在那时,经过精心培育,终于产出一匹高达 16 手宽的战马,这批骏马的身材可以舒适地承载一个全副武装的骑士。然而,正如诺斯科特·帕金斯(C. Northcote Parkinson)睿智地指出,“只有处于崩溃边缘的制度才会达到完美。”完美的新战马诞生了,新式武器也被部署到了战场上,可以轰炸战马和骑士。这种火药武器可以由普通人使用,不需要什么特殊的技巧;但要大量采购,则非常昂贵。与作为封建经济基础的农业相比,商业的重要性随着新武器的扩散逐步提升。

\subsection{更大规模的战争}
火药武器是怎么促成这种转变的?首先,它提高了战斗的规模,这使得发动战争的代价很快就远超中世纪时期。在火药革命之前,战争往往都是在很小的团体之间展开的,他们甚至可以在一块贫瘠狭小的领地上征战。火药为更大规模的战争提供了有利条件。在这种新的条件下,只有那些拥有富裕臣民的统治者,才有能力派遣出强大的军队。而最能适应商业发展的统治者,往往是那些与城市商人结盟的君主,他们会发现自己在战场上享有竞争优势。用克利福德的话说,“部分原因就在于,他们拥有更优越的财政资源,可以购买更多的大炮,从而把敌人轰得粉碎。”火药武器的逻辑全部被演绎出来,是在几个世纪之后,法国大革命中的平民军队。

但火药对战争的改造,早在文艺复兴时期对军服的采用上,就显示出来了。军装恰如其分地象征了战士与民族国家之间的新型关系,这种发展与骑士精神到公民身份的转变相辅相成。在封建主义时期,君主或者教皇要与一长串的诸侯分别达成不同的交易,而新兴的民族国家,则可以与它的公民达成“统一”(也是制服的意思)的交易。在旧制度下,每个人都在等级结构中有一个不同的位置,每个人都可以达成一个特殊的交易,就像他的纹章或者悬挂的彩旗一样。

\subsection{降低了富人的机会成本}
火药武器还以另外一种方式,从根本上改变了社会的性质。它将权力的行使与体力区分开来,从而降低了商业活动的机会成本。富有的商人不需要再靠技巧和力量进行徒手搏斗,也无需再依靠忠诚度可疑的雇佣兵保护自己;他们可以寻求最高统治者更大规模的、新式军队的保护。正如威廉·普莱费尔在谈到中世纪时所说,“在敌对的情况下,人力是激怒别人的力量,……所以,在当时,一个人不可能长期保持同时富有和强大。”而当火药出现以后,不富有就不可能强大。

\subsection{社会地位与静态思维}
商业对塑造现代社会起到了无与伦比的作用。然而,就像今天的大多数人还不能预见到信息社会的新动态一样,中世纪时期的主流思想家们,也无法预测或理解商业的兴起。五个世纪之前的大多数人,都以静态的眼光看待他们不断变化的社会。惠泽加说,“从现代意义上说,(中世纪)很少有财产是流动的。权力还没有与金钱紧密地联系在一起;它仍然是与生俱来的,取决于掌权者所激发的某种宗教敬畏,它以奢侈浮华的排场及大量忠实的追随者来彰显自我。通过可见的标志,封建或等级思想表现为那种追求显赫的想法。”中世纪晚期的人们最关心的是社会地位,他们还想不到商人能对王国的生活做出什么突出的贡献。因为商人都是平民,几乎毫无例外;他们处于三层身份的最底层,低于贵族和神职人员。

在中世纪的暮年,即使是更加有洞察力的思想家,也认识不到,在农业之外,商业和其他形式的社团对于积累财富的重要性。对他们来说,贫穷是一种使徒般的美德;在他们看来,有钱的银行家和乞丐没有区别。用惠泽加的话说,“在第三阶层中,原则上不区分富人和穷人,也不区分城镇居民和乡村居民。”在当时的体制下,职业和财富都不重要,重要的是骑士身份。

这种对生活中经济层面的盲视,又被教会人士强化了,这些人是中世纪意识形态的守护者。教士对商业重要性的认识严重不足,以致于在 15 世纪,推出了一项广受赞誉的改革方案,要求所有非贵族身份的人都要专门从事手工业或农业劳动。完全没有考虑到商业的任何作用。


\begin{tcolorbox}
1492 年,被习惯性地用来区分中世纪和现代时期,其实也可以作为其他一切事态的分界点。因为从世界史的角度看,哥伦布的大航海,象征着西欧与全球其他地区之间崭新关系的开始。
\begin{flushright}
—— 弗雷德里克·莱恩
\end{flushright}
\end{tcolorbox}


\section{工业社会的诞生}
十五世纪众多最敏锐的头脑,也完全没有捕捉到一个重大的历史进程,而该进程就发生在他们眼皮底下。封建主义的消亡,标志着西方主导全球的伟大的现代时期开始了。在这一时期,暴力的回报率不断上升,企业的规模越来越大。在过去的两个半世纪里,现代经济为能够享受到其最大利益的那部分人,带来了生活水平的空前提高。这些变化的催化剂是新技术,从火药武器到印刷术,以很少有人能够掌控的方式改变了生活的边界。

到十五世纪的最后 10 年,哥伦布等探险家才打开了通往广阔未知大陆的通道。

在人类生存的漫长岁月中,第一次足迹囊括了整个世界。西班牙大帆船,是在地中海帆船的基础上打造的新型高桅帆船,它环绕地球,绘制出了海上的通道,这些海图成为了后来商业贸易、疾病传播和军事征服的指南。在海上,在岸上,征服者靠着他们的新式青铜大炮,开疆拓土。他们发现了代表财富的黄金和香料;种下了新的经济作物,从烟草到土豆;还为他们的牛群开辟了新的牧场。

\subsection{第一种工业技术}
就像火炮打开了新的经济领域,印刷术开拓了新的知识领域。它是第一种大规模生产的机器,也是工业主义萌芽的标志性技术。说到这里,我们同意亚当·斯密在《国富论》中的观点,即工业革命早在他写这本书之前就已经开始了;虽然它肯定还没有达到成熟的阶段,但规模化生产和工厂系统的原则已经确立。他所举例的大头针制造商,就是一个著名的例证。斯密解释了怎么用 18 个工人分开操作来生产大头针。得益于专门的技术和劳动分工,每个员工在一天内生产的大头针,是他自己单独制造数量的 4800 倍。

亚当·斯密的例子还突出了一个事实:工业革命的开端,比传统历史学家认为的要早几个世纪。大多数教科书都把它的起源定于 18 世纪中叶,这也不是不合理,因为那是人类生活水平起飞的时期。但在大政治层面,封建主义和工业主义之间的过渡,实际上要早得多,是在 15 世纪末。几乎是发生的同一时间,它的影响就在统治机构的变革上体现出来了,特别是中世纪教会的消亡。

把工业革命的开端定在较晚时期的历史学家,是在衡量另外一件事,就是由发动机驱动的大规模生产带来的生活质量的飞跃。这提高了非熟练劳动力的价值,并导致各种消费品的价格下降。事实上,不同国家的生活水平在不同时期开始急剧上升,这就提示出,人们是以大政治转型以外的因素在衡量工业革命。在《剑桥欧洲经济史》中,谈到“工业革命”时用的是复数,并明确地将它与国民收入的持续增长联系在一起。在日本和俄国,这种收入的猛涨一直推迟到 19 世纪末。

在亚洲其他国家和非洲的一些地区,生活水平的提高和国民收入的增长是 20 世纪才发生的景象。非洲另外一些地区,持续增长至今还是一个梦想。但这并不意味着它们不是生活在现代时期。

\subsection{转型期收入的下降}
工业主义的出现,并不是收入增长的同义词。向工业社会的转变是个大政治事件,并不能直接用收入统计来衡量。事实上,在工业时代的头两个世纪里,大多数欧洲人的实际收入都在下降;到了 18 世纪初的某个时候才开始上升,直到 1750年左右才恢复到 1250 年的水平。我们把工业时代的启动时间定在 15 世纪末。因为正是早期现代技术的工业特征,包括以化学为动力的武器和印刷机,导致了封建主义的崩溃。

\subsection{知识成本的降低}
大规模生产书籍的能力,对中世纪的机构形成了不可思议的颠覆;就像微处理技术将颠覆现代民族国家一样。印刷术很快就破坏了教会对神的话语的垄断,甚至为异端邪说创造了一个新的市场。与封建社会格格不入的思想迅速传播,到 15世纪的最后 10 年,共出版了 1000 万本书籍。由于教会压制印刷业,大部分新书都是在欧洲那些当局权威最弱的地区出版的。当今天的美国政府试图压制加密技术时,很可能发生类似的情况。教会发现,审查制度并没有抑制住颠覆性技术的传播,反而把它推倒了最能发挥颠覆作用的地方。

\subsection{修道院地位被降低}
印刷机原本是人畜无害的,因为内容而具有了颠覆性。无畏的冒险家和商人赚取财富的知识被四散传播,这本身就是一种强大的溶剂,可以溶解封建义务的约束。

新型市场的诱惑,大规模资助军队和海军的机会,使货币产生了封建时期所不具备的价值。有了新的投资渠道,加上强大的武器,提高了暴力的收益,要求地方领主和城市商人向教会捐献资本的成本越来越高。因此,仅仅是在土地占有之外创造出的投资机会,就破坏了封建制度的稳定,削弱了它的意识形态。

印刷机的另一个颠覆性后果是大大降低了信息复制的成本。在中世纪,识字率极低,经济极不发达,一个重要原因是手工复制书稿的成本太高。正如我们讨论过的,在罗马沦陷以后,教会承担的主要生产职能之一,就是在本笃会修道院复制书籍和手稿。这是一项成本极高的工作。而印刷机极大地降低了手稿复制的价值,修士们日复一日、月复一月地辛勤劳作,制作出来的手稿,用印刷机只需要几个小时就够了。新技术使本笃会的缮写室成为了昂贵的、过时的知识复制手段,而反过来说,这就降低了继续使用抄写员的宗教团体和教会在经济上的重要性。

书籍的大规模印刷结束了教会对圣经及其他信息的垄断。书本的广泛普及也降低了识字成本,从而使得有能力就重要课题——特别是神学问题——提出自己见解的思想家成倍增加。神学史家尤安·卡梅伦(Euan Cameron)指出,16 世纪头20 年的“一系列的出版里程碑”为应用“现代文本批评经文”奠定了基础。这“威胁到了教会的垄断”,因为“人们开始质疑为了支撑传统教条而对经文进行的扭曲式解读。”新知识鼓励了教会竞争,新教出现了,他们要制定自己对《圣经》的解释。总之,书籍的大规模生产降低了异端和异教传播的成本,使他们也获得了大量的读者。

出版也帮助摧毁了中世纪的世界观。信息的成本更低,更加容易获得,使得人们对世界的认识从象征主义转到了因果关系。“象征主义的世界图景,特点就是无懈可击的秩序、建筑结构、等级从属关系。每一种象征性的联系都意味不同的等级和神圣性……核桃象征着耶稣,甜美的内核是他的神性,青涩多肉的外皮是他的人性,中间的木壳是十字架。因此,一切事物都在使人的思想上升到永恒。”符号化(即象征主义)的思维方式不仅是对社会等级结构的补充,它也更适合教育文盲。木刻画中的符号所传达的思想,可以被文盲所接受。而相比之下,现代时期印刷术的出现,使得识字的人口可以运用科学方法,发展因果关系。

\section{与今天的对比}
在 15 世纪中叶,中世纪社会的信仰基础似乎是无与伦比的安全和稳定,但却很快就被改变了。它的主要机构,也就是教会,眼睁睁看着自己的垄断地位受到挑战并被打破。几个世纪以来都未曾受到过质疑的权力,突然陷入了争议。那种神圣的信仰和忠诚,超出今天任何民族国家对公民的束缚,而在短短几年的时间内,被人们重新认知并放弃。这一切的缘由,就是十五世纪最后 10 年出现的一场技术革命。

我们相信,像 500 年前那样的剧变将会再次上演。信息革命将摧毁民族国家对权力的垄断,就像火药革命摧毁教会的垄断一样,毋庸置疑。十五世纪末的情形与今天的状况惊人地相似;当时的生活完全被有组织的宗教所淹没,而今天的世界则充满了政治。当时的教会和今天的民族国家,都是已经衰老到极限的组织机构。

和中世纪暮年的教会一样,20 世纪末的民族国家也是负债累累,无力再支撑自身的存续。对于那些不久前还坚定支持民族国家的人来说,它的存在对他们的繁荣幸福越来越无关紧要,甚至还起反作用。

\subsection{“贫穷、贪婪、奢侈”}
今天的政府收取了钱财,但他们提供的价值则是极其糟糕的。15 世纪末的教会也是一样。教会历史学家尤安·卡梅伦说,“贫穷的地方神职人员,对他们收取的财务管理极差。他们所征收到的财物,实际上都消失在了修道院或神秘的高层那里。尽管教会的某些部门收到了大量的礼物,但整个机构却可以同时显得贫穷、贪婪和奢侈。”你很否认 20 世纪末的政府也是如此。

十五世纪末的宗教活动层出不穷,就像今天福利国家搞的项目一样。不仅仅有特殊的祝祷节庆,就连圣人和圣骨的供应都是无止境的;每年都有更多的教堂、更多的修道院、更多的修会、更多的忏悔师(家庭常驻牧师)、更多的传教士、更多的大教堂分会、更多的捐赠圣堂、更多的遗迹崇拜、更多的宗教联合团体、更多的宗教节日和新圣日。宗教仪式越来越长,祈祷和赞美诗越来越复杂。新的募缘修道团一个接一个地出现,到处乞求施舍。最终的结果是超出了制度的负荷极限,和今天高度政治化的社会一样。

宗教节日和各种节庆全面激增,礼拜仪式越来越多,圣母玛利亚、她的姐妹和耶稣家谱中的所有圣人,都有专门的纪念日。对信徒来说,履行宗教义务的成本和负担越来越沉重,就像今天遵守法律的成本越来越高。

\subsection{无辜者付出代价}
过去和现在一样,是生产者承担着收入再分配的成本,而他们肩上的担子越来越重。由于资本利用方式的转变,这些成本的增长幅度超出了所有当权者的认知。

与货币资本相比,持有土地的相对优势正在下降。但中世纪的人们脑子里想的依然是地位和身份;社会地位取决于你是谁,而不是你有效配置资本的能力。人们很少或者根本没有考虑过,举办浮夸宗教仪式的机会成本已大大增加。这些成本都沉重地砸到了勤劳工作的农民、市镇居民和自耕农身上,相比与贵族阶级,他们需要更加高效地利用自己的资本。他们不得不支付不成比例的费用,为无休止的盛宴、圣日餐饮以及奢侈的教会官僚买单。

\subsection{适得其反的监管}
在 15 世纪末,教会在很大程度上控制了后来由政府掌握的监管权力。教会主导着重要的法律领域,记录契约、登记婚姻、认证遗嘱、许可贸易、规定土地所有权以及商业的条款和条件。教会法对生活细节的规范几乎和今天的官僚机构一样彻底,而且目的也大同小异。和今天的政治法规一样,500 年前的教会法也充满了混乱和矛盾。这些规章制度往往搞得很复杂,以压制商业进步,这表明监管者根本就没想过促进生产力的发展。

例如,不管最近一年的 12 月 28 日是星期几,在一整年当中,每周的这一天都不能做生意。因此,如果某年的 12 月 28 日是星期二,在全年的所有星期二都不能合法做生意。这是为了纪念“诸圣婴孩殉道日”必须表达的一种虔诚。如果 12月 28 日落在周日以外的任何一天,那么这个禁令就限制了很多种商业的经营潜力,延迟或阻碍了正常的交易,从而增加了成本。

\subsection{垄断定价}
教会法的实施也是为了加强垄断价格。教会在意大利托尔法的地产中开采明矾,通过销售明矾获得了大量的收入。当纺织业的客户想要选择从土耳其进口的更便宜的明矾时,梵蒂冈通过教会法宣布使用较低成本的明矾是有罪的,以维持其对价格的垄断。坚持购买便宜的土耳其明矾的商人将被逐出教会。星期五禁止吃肉的著名禁令,也是出于同样的考虑。教会不仅是最大的封建地产所有者,它还控制着主要的渔业。教父们发现虔诚的教徒对于吃鱼有神学上的需求,在当时的交通和卫生条件还不利于鱼肉消费的情况下,教会能够满足吃鱼的需求,绝不是偶然的。

和今天的民族国家一样,中世纪晚期的教会,不仅对某些特定行业进行限制以保证自己的利益;它还充分利用监管的权力进行创收。教士们煞费苦心地颁布了很多难以被遵守的法规和禁令。例如,对乱伦的定义特别宽泛,即使是远房表情或者只有姻亲关系的人,也需要得到教会的特别许可才能结婚。在现代旅行出现之前,这几乎包含了很多欧洲小村庄里的每个人。于是,为乱伦婚姻出售豁免权成为教会丰厚的收入来源。即使是婚内的性行为,也受到教会法的严格限制。夫妻间的性活动,在周日、周三和周五以及复活节和圣诞节前的 40 天之内都是非法的。此外,夫妻在接受圣餐的前 3 天也要禁欲。换句话说,如果没有赎罪券,在一年中至少 55\%的日子里,夫妻双方不能享受性生活。在《主教的妓院》(TheBishop's Brothels)一书中,历史学家 E.J.伯福德(E. J. Burford)认为,这些“愚蠢的”婚姻规定,刺激了中世纪卖淫业的发展,教会也从中获得了巨大的利益。

伯福德指出,在长达几个世纪的时间里,温彻斯特大教堂的主教都是位于南华克的伦敦河岸妓院的主事人。而且,教会从卖淫中谋取暴力绝不仅限于英国:。

\begin{tcolorbox}
教皇西克斯图斯四世(约 1471 年),传闻从众多情妇中的一个感染了梅毒;他是第一个向妓女发放许可证并对她们征税的教皇,并因此而获利颇丰。他的继任者,教皇里奥十世,靠出售卖淫许可证据说赚了大约 2.2 万金币,是他在德国出售赎罪券所得的 4 倍。
\end{tcolorbox}


即使是对牧师实行的著名的独身主义规则,也是中世纪教会有利可图的收入来源。伯福德说,教会采取了一种被称为“情妇税”(cullagium)的敲诈手段,对包养情妇的牧师进行征税。这种税的利润无比丰厚,以致于法国和德国的主教对所有的牧师统一征收;尽管在 1215 年,拉特兰会议谴责了“这种可耻的交易,教长是在出售罪恶的许可。”而这不过是违反教会法规出售许可证以谋取利益的众多方式之一,这种交易背后的动机与逻辑,和贪婪的政客寻求对商业的任意监管权是一致的。

\subsection{赎罪券}
任意监管的权力也是出售豁免的权力,豁免监管可能导致的伤害。教会出售许可证或“赎罪券”,豁免范围无所不包,从减轻商业上的琐碎负担到允许在斋戒期间吃乳制品。这些“赎罪券”不仅高价卖给贵族阶级和富有的乡民;还被包装成彩票式的奖品,就像今天的政府经营彩票一样,用来吸引穷人的钱财。随着教会的支出远超收入,赎罪券的买卖也不断膨胀。很多人开始看到那个显而易见的事实,制度化的教会利用它的权力主要是为了谋利。就像一位当代的批评家所说,“制定教会法完全是为了赚大钱,谁要想成为基督徒,就得为它的规定而买单。”
\subsection{沉重的官僚主义}
到 15 世纪末,制度化宗教的成本已经达到了历史的极点,就像今天支撑政府运作的成本到达极限一样。宗教充斥着人们的生活,它的饱和度越高,教会的成本就越昂贵,也愈发地官僚化。用卡梅伦的话说,“在中世纪,找人填补大量增加的教会职位,比找到钱支付这些职位要容易得多。”今天破产的政府用反动的方式搜刮钱财,500 年前的教会也是如此。而且,教会所使用的一些掠夺性手段,就是今天的政客们在玩弄的。

500 年前的中世纪教会,就像今天的民族国家,它们所消耗的社会资源,是空前的,也是绝后的。当时的教会虽然达到了创纪录的收入,但似乎已经失去了正常运行和存续的能力,今天的国家也是如此。国家主导了工业社会晚期的经济,在一些西欧国家,政府的支持超过全部收入的一半;同样,在封建社会后期,教会主导着经济,它抽干了社会的资源,阻碍了生产力的增长与进步。

\section{十五世纪的财政赤字}
教会运用了一切可以想象得到的权宜之计,从它征收的费用中挤出更多的钱,以便养活臃肿的官僚机构。在有直接管辖权的地区,教会索取越来越高的税负;在没有直接征税权的外省或王国,梵蒂冈征收一种由当地君主直接支付的“年金”,代替直接的教会税。

和今天的国家一样,教会也搜刮自己的国库,把用于指定用途的捐赠资金,转用于支付一般的管理费用。有俸的圣职和宗教职位被公开售卖,就像什一税的收入流一样。实际上,什一税就是神职人员发行的债券,而现代政府为支撑长期赤字而发行的债券是一回事。

在意识形态上,教会是封建主义的捍卫者,是商业和资本主义的批判者。但是,和今天的民族国家一样,教会利用一切可能的营销手段优化自己的收入。教会经营的圣物销售生意极为兴隆,包括圣蜡,在圣枝主日受到祝福的圣枝,还有“圣母升天节受到祝福的草药,特别是各种圣水。”今天的选民,如果拒绝支付更高的税款,政客就会威胁减少垃圾收集,或者采取其他侮辱尊严的手段或措施;15 世纪的教会也经常中断宗教服务,随心所欲地向教众勒索罚款。罚款往往是因为几个人的轻微违法行为,而这些人甚至不一定是有关教区的成员。例如,1436 年,雅克·杜·夏特利埃主教,“一个非常浮夸、贪婪的人”,将巴黎的圣公会教堂关闭了 22 天,停止了所有的宗教活动,就为了要两个乞丐支付一笔不可能的巨额罚款。因为这两个人在教堂里争吵,还流了几滴血。主教认为这玷污了教堂的神圣性;他不允许任何人使用教堂举行婚礼、葬礼或日常的历法圣事,直到他拿到了罚款。

\begin{tcolorbox}
一个意大利人的妓院(为了讨教皇的欢喜)每年付给他两万杜克特金币他们给牧师的报酬则是一个鸡的利润,或者两个三个鸡我想一定是邪恶的神灵才会与妓院有这样的关系。
\begin{flushright}
—— 15 世纪的英国民谣
\end{flushright}
\end{tcolorbox}

\section{对宗教领袖的憎恨}
也难怪 15 世纪末的人们,普遍鄙视上层和下层的神职人员;就像高度政治化的今天,人们都看不惯官僚机构和政客一样。惠泽加说,“憎恨这个词用在这里是很恰当的,因为憎恨是潜在的,但却是普遍且持久的。听到神职人员的恶行被审判,人们总是津津乐道。”人们普遍相信教会“贪婪又挥霍无度”,部分原因在于这是事实。“神职人员高层的世俗化和底层的堕落”实在是太明显了,有眼睛的人都能看得到。从教区牧师到教皇本人,神职人员腐败的普遍性,只有占据社会统治地位的机构才能做得到。

比起 500 年前的教皇亚历山大六世(Alexander VI),朱里奥·安得利奥迪和比尔·克林顿都堪称是正直的典范。亚历山大六世以举办狂野的派对而闻名。作为锡耶纳的红衣主教,他举办过一场著名的狂欢,只有“锡耶纳最漂亮的年轻女子被邀参加,但她们的‘丈夫、父亲和兄弟’都被排除在外。”锡耶纳的派对虽然很有名,但与亚历山大成为教皇之后的疯狂相比,就显得平淡无奇了。其中最淫乱的当属所谓的“栗子芭蕾舞”,它是一场交配大赛,参与者有罗马“最性感的50 名妓女”、教会神父以及罗马有头有脸的人。威廉·曼彻斯特对此描述道:“仆人们对每个男人的高潮表现进行计分,因为教皇非常欣赏阳刚之气……在所有人都精疲力尽之后,教皇会颁发奖品,有斗篷、帽子、靴子和精美的丝质外衣。

根据记载,获奖者是那些与妓女做爱次数最多的人。”亚历山大至少有七个或八个私生子。有一个叫乔瓦尼的(Giovanni),就是所谓的“罗马之子”,显然是他的儿子。乔瓦尼是亚历山大的私生女卢克雷齐娅·博尔吉亚(Lucrezia Borgia)在 18 岁是生下的。在一份秘密的教皇诏书中,亚历山大承认他是乔瓦尼的父亲。如果不是父亲,那他也肯定是双方的祖父。教皇与卢克雷齐娅发生了三方乱伦关系;卢克雷齐娅也是亚历山大最年长的私生子甘迪亚公爵——胡安的情妇,她还是亚历山大另外一个私生子切萨雷·博尔吉亚的情妇。

切萨雷是红衣主教(Cardinal Cesare Borgia),他就是尼科洛·马基雅维利(NiccolòMachia-velli)创作《君主论》的灵感来源。和教皇一样,切萨雷也是个杀手,据说他曾经谋划过几起杀人案。他们两个中的一个显然对胡安产生了嫉妒,后者的尸体于 1497 年 6 月 15 日在台伯河被捞出。

中世纪晚期教会的领导层,与今天民族国家的管理层是一样的腐败。

\begin{tcolorbox}
今天我已经两次成为父亲。上帝保佑。”-鲁道夫·阿格里科拉,在他当选修道院院长的当天,听到他的情妇生了一个儿子。 
\end{tcolorbox}

\section{伪君子}
在“表明的虔诚”之下,中世纪晚期的社会堕落放荡,是非常不敬神的。教堂是男女青年最喜欢的幽会场所,也是妓女和淫秽图片小贩经常聚集的地方。历史学家报告说,“日常宗教活动中的不敬行为毫无节制。”受雇为亡灵超度的唱诗班,经常在弥撒中使用渎神的字眼。守夜和列队祈祷在中世纪的宗教活动中的重要性,比在今天大得多,但据中世纪晚期欧洲最重要的神学权威——加尔都西会的德尼(Denis the Carthusian)说,它们还是被“黄段子、讽刺的模仿秀与喝酒所玷污。”有人可能会说,这些指责是古板的道学家在发牢骚,但这只是众多同类描述中的一种。有充分的理由相信,在中世纪,猥亵和神圣往往是亲密无间的伙伴。例如,朝圣活动就经常沦为骚乱和淫乱,以致于一些品德高尚的改革者主张禁止朝圣,但没有成功。地方性的宗教游行,也为暴民提供了破坏和抢劫的机会,以及各种能激起他们兴趣的醉酒闹剧。即使是静静地坐在教堂听弥散,人民也普遍处于不清醒的状态。教堂里喝掉的酒是惊人的,特别是在节日的晚上。斯特拉斯堡理事会的记录显示,在圣阿道夫之夜,那些在“守望中祈祷”的人,喝光了教会为纪念圣人而准备的 1000 升葡萄酒。

15 世纪著名的神学家让·格森(Jean Gerson)记述道,“即使最神圣的节日,甚至圣诞夜,都是在打牌、讲荤段子、骂脏话、淫乱放荡之中”度过的。当平民百姓因为这些行为受到惩戒时,他们就会举出“贵族和神职人员”为榜样,因为这些人做同样的事却不会受到惩罚。

\subsection{虔诚与同情}
在中世纪晚期,虔诚合理化了充斥社会的有组织宗教,今天为政治生活而辩护的“同情心”,也是出于同样的目的。为了满足毫无道德的虔诚渴望而出售赎罪券,与支出奢侈的福利以满足假模假式的、没有善行的同情心一样。赎罪券是否产生了提升道德或拯救灵魂的实际效果,福利项目是否真正改善了它计划帮助对象的生活品质,根本不重要。“虔诚”和“同情”一样,是一种近乎迷信的召唤。

在一个基本不了解因果关系的时代,教会的仪式和圣礼渗透到生活的各个阶段。

“……一次旅行,一项任务,一次访问,都有无数同样的手续:祈祷、仪式、规则。”刻在羊皮纸上的祈祷词像项链一样,串在发烧者的身上。营养不良的女孩,将一缕缕的头发挂在圣乌尔班的图像前,以防止进一步的脱发。在干旱的季节,纳瓦拉的农民抬着圣彼得像列队游行以求雨。人们热衷于这些及其他“无效的技术,在没有什么好办法的情况下来减轻焦虑。”

\subsection{两个错构成一个善}
人们对圣人遗物的神奇力量深信不疑,当任何有影响力的虔信者去世之后,都会被疯狂地分尸。托马斯·阿奎那在福萨努瓦修道院逝世后,那里的修士将他的尸体斩首并煮沸,以便得到他的骨头。匈牙利的圣伊丽莎白在下葬前供公众瞻仰时,“一群崇拜者冲到这里,剪断或扯走了裹在她脸上的亚麻布条,还剪掉了她的头发、指甲,甚至乳头。”

\subsection{没有美德的虔诚}
中世纪的人把圣人和他们的遗物看作是信仰武器的一部分;在那样一个世界里,冬天之冷、夜晚之黑、疾病面前之绝望,远超本书读者的想象。与现代人相比,中世纪人更相信魔鬼是真实存在的,相信上帝会积极地干预世界,而祈祷、忏悔和朝圣可以赢得神的眷顾。

如果只是说人们相信上帝,既不能表达他们信仰的强度,也不能表达中世纪人把虔诚与罪恶搞在一起的容易程度。对仪式、典礼和圣事的功效的信仰无处不在,这不可避免地削弱了良善行事的紧要性。对于任何罪恶或精神缺陷,都有一种补救措施,一种可以洗刷污点的忏悔,这就是所谓的“救赎数学”。当宗教变得无孔不入,它的诚信必然开始削弱。对此,惠泽加说:“宗教渗入到了生活的一切关系之中,意味着圣洁的与亵渎的思想领域不断融合。神圣的东西变得太过普通,无法再带给人们深刻的感受”。事实确实如此。

\section{教会规模的缩小}
到 15 世纪末,教会不仅与今天的民族国家同样腐败,也是经济增长的主要阻力。

教会吸走了大量的资本,但不是用于生产;它还强加了各种负担,限制社会产出,压制商业。这些负担与今天的民族国家所施加的一样,多如牛毛。我们都知道,火药革命给有组织的宗教带来了什么:它制造了强烈的改革动机,促使教会缩小规模并降低成本。当传统的教会拒绝这么做的时候,新教教派加入竞争,抓住了机会。他们动用了几乎所有能想到的手段去降低虔信生活的成本:

\begin{itemize}
    \item 他们建造了简朴的新教堂,有时候还拆除旧教堂的祭坛,腾出资金另做他用。
    \item 他们修订了基督教教义,以降低信仰的成本,并强调信仰而非善行是得救的关键。
    \item 他们制定了新的、简洁的礼仪,缩减或取消了众多节日,并废除了很多圣礼。
    \item 他们关闭了修道院和修女院,并停止对修行者的施舍。贫穷从一种使徒的美德,变成了一个不受欢迎的、值得谴责的社会问题。
\end{itemize}

要理解教会的缩小怎么解放了生产力,就必须回顾垄断被打破之前,教会有哪些阻碍发展的表现。就像当今的民族国家,教会在 15 世纪末强加给社会的超额成本,负担之沉重是令人难以置信的:

\begin{enumerate}
    \item 诸如什一税、教会税和各种费用等直接成本,养活了极度臃肿的教会官僚机构。在取代中世纪“圣母教会”的新教教堂中,什一税也很常见,但在城市地区一般征收不到。事实上,教会垄断的结束使商业最发达地区的边际税率不断下降。
    \item 宗教教义很不利于储蓄。中世纪教会中的大反派就是“守财奴”,即不惜出卖灵魂去储蓄黄金的人。教义要求信徒资助“善行”,于是他们要向教会捐赠高昂的费用。“赎罪”的教义迫使那些想得到救赎的人捐助弥撒或圣堂,以避免落入炼狱。对于这一点,路德在他的 95 条论纲的第 8 条和第 13 条进行了直接的抨击。他写到:“垂死之人以其死亡偿还了所有的债务。”换句话说,新教信徒的财产是可以传给他的继承人的。根据新教教义,没有必要重复捐赠做追思弥撒的小教堂;通常是 30 内,对于特别富有的人,则是永久的。
    \item 中世纪教会的意识形态还鼓励将资本用于购买圣物。圣物崇拜俘虏了大量的资金,用以获取与基督或其他圣人相关的物品。非常有钱的人甚至建立自己的圣物收藏。例如,萨克森的选帝候弗雷德里克就积累了 19000 件圣物,有一些是在 1493 年去耶路撒冷朝圣时获得的。在他的收藏品中,包含他认为的“一个圣婴的尸体、玛利亚的乳汁和耶稣诞生地马厩的稻草。”根据推测,投资这些圣物的资本回报率是很低的。随着宗教改革强调信仰及选民的概念(上帝拣选),购买圣物作为护身符的重要性就大大降低了。这也鼓励资金去寻找效益更高的投资渠道,可以支付君主索要的回报。
    \item 新教教派的出现,打破了中世纪教会的经济垄断,并大幅削弱了教会监管的权力。我们前面谈论过,教会法经常被扭曲,用来支持教会的垄断和商业利益。由于新教教派需要保护和加强的经济利益比较少,他们的教义往往会促成一个更自由的体系,对商业的压制也更少。
    \item 新教革命废除了中世纪教会的许多仪式和典礼,它们给信徒的时间带来了沉重的负担。到 15 世纪末,经过精心的设计,仪式、圣事和圣日,几乎在日历上排满了。这种超负荷的仪式安排,是教会坚持“一个人可以随意增加祈祷或敬拜的次数,还可以从中获得好处”的必然结果;它们确实在增加。更加冗长、复杂的仪式,在忏悔中重复背诵祈祷词的要求,不能工作的圣日不断增加,这些都影响了生产力的发展。繁多的规章制度和仪式贯穿在一天和四季之中,大大减少了可以从事生产的时间。在中世纪 90\%的人口在从事农业。农业的节奏被打断,影响还不大。在耕作的季节里,很多时候不需要每天都下地干活。在当时的条件下,农作物的产量更多是受天气变化和不可控的病虫害的影响;而不是教会日历所能允许的最低限度以外的边际劳动力的增加。\\ 但在农业之外的其他领域,生产力的损失就是一个大问题。在手工业、制造业、运输业、商业,以及其他由劳动量决定产出和利润的事业,就会与教会对时间的要求产生冲突。\\ 15 世纪末的大转型,发生在地租上涨和农民实际收入下降的时候,这应该不是巧合。人口压力的增加,减少了公共土地的产量。这些土地一般位于河流或溪流的周围,农民靠这些土地放牧,有时候也从中获取鱼和柴火。生活水平的下降,让农民感受到了越来越迫切的生活压力,他们需要寻找其他收入来源。因此,“越来越多的农村人口开始转向小规模的制造市场,最早就是纺织业,这个过程被称为‘原工业化’。”教会强加的负担抑制了在新经济方向上的时间投入,它阻碍了更有进取心的农民,去从事手工业以补充农业收入。\\ 新教教派做出的一大贡献,就是取消了 40 个节庆日。这不仅节省了大量的费用,如活动时需要的饮食;还释放了大量的宝贵时间。停止敬拜这 40 个节日,还意味着每个人可以在一年中增加 300 个工时或更多。简而言之,仅仅通过取消中世纪教会超负荷的仪式活动,释放出本来可能会在商业上损失的时间,就为社会产出的显著提高打开了局面。
    \item 垄断地位被打破后,教会吐出了大量的资产。这些资产在教会的管理下,回报极低,这种情况与 20 世纪末国有资产的运用有明显的相似之处。教会显然是最大的封建地主;它对土地的控制力,与高度政治化的今天,政府的控制力不相上下。在欧洲一些国家,如波西米亚,教会控制了超过总量 50\%的土地。根据教会法,一项财产一旦归教会所有,就不能被转让出去。因此,随着教会从信众那里收到的遗赠越来越多,教会的土地持有量也不断增加;而那些遗赠原本是用于资助各种社会福利服务、建造圣堂及其他活动的。\\ 虽然很难精确衡量教会持有土地的相对生产率,但它在中世纪末期的水平肯定远远低于早期。到了 14 世纪,人们越来越重视市场生产,而不再是自给自足的农业,这使得大多数非宗教领主,从不识字的工头变成了职业经理人,以便优化他们的土地产出。他们所采取的激励措施,可能使他们的土地产出很快就超越了教会;毕竟在理论上,教会的财产不会给任何人带来私利。毫无疑问的是,一些更世俗的红衣主教,也使用普通领主的方式管理财产。但是,其他教会财产的生产力肯定会受到失败管理的影响,毕竟它们是在一个大而无当且远在天边的机构手里。这其中的弊端,与今天的国家所有制和集体所有制的弊端是差不多的。另外很明显的一点是,对修道院的查封也重新调配了资源,在印刷机出现之后,这些资源无需再用于复制手稿和书籍。
    \item 在《大清算》一书中,我们详细阐述过,火药革命发生后,一些新教教派迅速做出了反应,他们修改了教义,以鼓励商业的发展,例如取消了对高利贷及贷款利息的禁令。中世纪的教会在意识形态上是反对资本主义的,它是经济增长的累赘。教会教义的主旨是加强封建主义,因为教会是最大的封建土地所有者,在其中有巨大的利益。无论是否是故意使然,教会都倾向于把自己的经济利益作为宗教美德,同时反对通过制造业和独立商业积累财富,因为它们必然会破坏封建制度的稳定。例如,教会下令反对“贪婪”,但这主要适用于商业贸易,而不是封建税收,也从来没有用在赎罪券的销售上。教会还无耻地为商业项目制定“公正的价格”,这其实抑制了那些非教会生产的商品和服务的经济回报。教会对商业创新的抵制,禁止“高利贷”就是一个明显的例子。银行和信贷是工商业大规模发展的关键。教会限制信贷供应,延缓了商业发展。
    \item 更加微妙的变化是,新的教派对《圣经》文本的专注,摧毁了中世纪教会的思维模式和意识形态;这两者都为社会进步设置了障碍。中世纪晚期的文化编程,鼓励人们用象征性的模拟去看待世界,而不是通过因果关系去理解。\\ 这削弱了人们的逻辑推理能力,也远离了商业主义的思想。用符号等价物的方式思考,很难转入到市场价值的思维模式。杰出的中世纪史学家约翰·惠泽加举例说;“三个庄园代表着圣女的品质;七个选帝候象征着美德;1477年仍然忠于勃艮第家族的阿图瓦和埃诺的五个城镇,是五名聪慧的处女……同理,鞋子意味着关心和勤奋,袜子意味着坚韧,吊袜带意味着决心,等等。”这些例子表明,人们的思维被宗教教条、僵化的符号和寓言所支配,生活的方方面面被等级隶属关系捆在一起。每一种职业、每一个部门、每一种颜色、每一个数字,甚至每一个语法元素,都被绑在一个宏大的宗教概念的体系之中。\\ 因此,日常生活中的点点滴滴不是用因果关系来解释,而是通过静态的符号和寓言。美德和恶习有时会被人格化,每样东西都代表着别的东西,后者又代表另外一些东西,这不仅没有理清因果关系,还搞得更加混乱。为了进一步混淆是非,事物间的各种联系被数字系统任意地套在一起;数字 7 好像特别重要。有七种美德、七宗罪、主祷文中有七个祈愿、圣灵有七种恩赐、激情有七种时刻、还有七种至福和七种圣礼,以及“由七种动物代表,后面跟着七种疾病。”
\end{enumerate}

\subsection{十五世纪的新闻报道}

十五世纪如果有新闻业的话,那么它写出来的故事,除了间接性的、象征性的寓言,不会回答任何报道事实的经典问题。下面是一篇十五世纪的私人日记,请看它关于勃艮第谋杀案的记述:


\begin{tcolorbox}
这时,住在邪恶之塔的不和女神现身了,她唤醒了愤怒女神、疯女人、贪婪女神、狂暴女神和复仇女神,他们拿起各种武器,把上帝的理性、正义和教诲全抛在了脑后,最可耻的是,也扔掉了节制。怀着满腔的暴怒,他们开始了疯狂的谋杀、屠杀、砍伐、处死,把在监狱里发现的所有人,统统杀光……贪婪女神把她的裙子塞进裤腰带,带着她的女儿——掠夺,还有她的儿子——盗窃……完事之后,上述众人在他们的女神——也就是愤怒、贪婪和复仇的带领下,杀遍了巴黎所有的公共监狱。
\end{tcolorbox}

摆脱中世纪的思维范式,使人们开始用因果关系等“现代”术语理解世界,而不再依靠象征性的联系和拟人化的寓言。

中世纪晚期教会的教义和思维模式是虚伪的,这一点没必要再争论;它们更倾向于契合农业封建主义的需求,而不愿意留给商业发展的空间,更不用说工业。准确地说,教会作为当时社会的主导机构,它的所作所为,塑造了道德、文化和法律方面的限制,都是为了保证封建主义的生存与发展。正因为如此,它不符合工业社会的需要;就像现代民族国家在道德、文化和法律上的约束,无益于信息时代的商业一样。我们相信,国家将步教会的后尘遭到革命,使新的发展潜力得以实现。

新教的教义认为,只靠信仰本身就可以升入天堂,而不再需要对亡灵祈福的活动进行捐赠。这被人们视为一个神学问题,但它更是一种匹配新时代经济现实的神学。在当时,把额外的资金投入到臃肿的教会官僚机构中,它的机会成本陡然上升了;很明显,人们需要一种成本效益更高的救赎途径,新教的改革满足了这一点。在没有其他投资渠道的时候,人们还不太介意把钱交给教会。但是,他们看到,资助一个前往东方的香料船队,有机会获得百倍的回报;资助国王的一个军营,回报虽然少一点,一年也有望得到 40\%的利息;在这种情况下,他们于自身利益所在之处,寻求上帝的恩典,也是可以理解的。

不久之后,相比他们在封建制度下的祖先,很多商人和平民变得富裕多了。近代早期,商人和小制造业主的生活水平急剧提高;在那些收入和生活方式随着封建主义垮台而崩溃的人们当中,他们普遍不受欢迎。教会垄断地位的削弱和富人大政治权力的增强,导致可以再分配的收入急剧减少。没能成为新制度直接受益者的农民和城市贫民,对那些受益者充满了嫉妒。惠泽加描述了当时的主流态度,“对于富人,尤其是当时出现的大量新富者,人们的憎恨是普遍的。”信息革命将与此极为相似。

同样惊人相似的是,犯罪率的大幅上升。旧秩序崩溃的后果,即使达不到封建革命时期彻底的无政府状态——对此我们在上一章探讨过,也总是会导致犯罪活动的激增。在中世纪末,随着旧的社会控制体系的瓦解,犯罪率像火箭一般蹿升。

用惠泽加的话说,“犯罪开始成为对社会秩序的极大威胁。”在未来,它也具有同等的威胁。

现代世界,是在新技术、新思想和黑火药臭气的混合之中诞生的。火药武器和进步的航海技术,破坏了封建主义的军事基础;甚至成为一种新的通讯方式,瓦解了它的意识形态。而印刷术揭示出了教会的腐败,使得它的统治阶级和普通成员,在一个以宗教为中心的社会里,很矛盾地被人们普遍鄙视。这显然是一个可与当代相对照的悖论,在一个政治为王的时代,人们对政客和官僚已不再抱有任何幻想。

十五世纪的暮年,是一个幻灭、混乱、悲观和绝望的时代。一如今日。