\chapter[平等主义经济学的终结]{平等主义经济学的终结:\\ 无工作世界里的致富革命}

\begin{tcolorbox}
神是不轻慢人的,人种的是什么,收的也是什么。
\begin{flushright}
——《加拉太书》6:7 
\end{flushright} 
\end{tcolorbox}

以生产或防御为主导的发展模式,一旦发生巨大的变革,将会改变社会的结构,以及财富和权力在不同群体间的分配比例。信息时代的意义,不仅在于强力计算机的广泛应用,它还意味着一场生活方式、机构组织和资源分配的全面革命。隐蔽的暴力对资源控制的能力将急剧衰减,一种新的财富配置模式将会兴起,20世纪那种强迫性的政府中介将被排除在外。因为在信息社会中,地理位置将变得无关紧要,所有在地里边界以内而不是以外运作的组织,在未来的作用都将被削弱。政客、工会、职业管制、说客乃至政府本身都将日落西山;因为费尽心机从政府那里争取到的商贸利益,其价值将大打折扣;浪费到游说和反游说方面的资源也会大幅减少。

利用强制力和地方优势进行收入再分配的人,势必将丧失大部分权力。资源的控制权将随之转移。由私人创造的财富,迄今为止一直遭到民族国家的强占,未来将由创造者本人所控制。越来越多的财富,会流入全世界最优秀的企业家和风险资本家手中。全球化,加上信息经济的其他特点,会使每个领域内最有才华的人收入大增。优异表现所产生的边际价值如此巨大,它将使全球经济中盈利能力的分布,呈现出类似今天的竞技体育和歌剧等表演行业的形式。

\section{超越帕累托法则}

帕累托法则指出,80\%的利益取决于或归属于 20\%的人。这大概是真的;不过,更惊人的是,1\%的美国人支付了 28.7\%的所得税。这也表明了,相对维尔弗雷多·帕累托在上个世纪末的观察,随着社会进入信息时代,人们的收入与能力分配曲线会更加倾斜。现在的人已经很习惯财富上的巨大不平等。在 1828 年,4\%的纽约人被认为拥有该市全部财富的 62\%。到 1845 年,前 4\%的人拥有纽约市所有公司和非公司财富的 81\%。在更大范围内,到 1860 年,前 10\%的人拥有整个美国约 40\%的财富。而到了 1890 年,记录显示,当时最富有的 12\%的人拥有美国财富的 86\%左右。

1890 年的数字很接近帕累托的法则。但它之所以不是二八比率,主要是因为在19 世纪末,大量身无分文的移民涌入到了美国。移民在总财富中所占的份额可以忽略不计,因此,他们的加入自然使财富总量的分配更加不平等。这个例子其实也很好地说明了,在任何时候,当真正的机会爆发出来,总是不可避免地加剧社会的不平等,至少在一段时间内。到 1890 年,移民约占美国总人口的 15\%,但在东北部的一些州则超过了 40\%,大部分的收入和财富都是在那里创造的。把财富的占比根据移民激增的情况调整之后,19 世纪末的美国,就像帕累托生活的 19 世纪末的瑞士,一样符合他的公式。

信息时代已经改变了财富分配的法则,特别是在美国,并且成为了美国现代政治的痛点之一,对此我们会在下一章深入探讨。在信息时代要取得经济成功,需要很强的读写能力和计算能力。美国教育部进行过一项大规模的“美国成人识字率”调查,结果显示,多达 9000 万名 15 岁以上的美国人能力极差。对此,美国侨民比尔·布莱森的说法更加生动:“他们就像猪一样蠢。”具体来说,可以判定,9000 万美国成年人不会写信,不能理解公共汽车时刻表,不会做加减法,即使借助计算器的帮助。可想而知,连普通公交时刻表都看不懂的人,不可能在信息高速公路上有所作为。这 1/3 还没准备好进入电子信息世界的美国人,正在酝酿着一个愤怒的下层社会。而在社会顶层的一小群人,也许只占总人口的 5\%,是受过高等教育的信息工作者或资本所有者;他们在信息时代的地位,相当于封建时代的土地贵族。二者的关键区别是,信息时代的经营是创富专家,而不是暴力专家。

\subsection{创新的大政治学}
20 世纪的大多数社会学家都认为,技术进步会自然而然地促成越来越平等的社会,但他们并没有令人信服的理由。在 1750 年之前,就不是这样。大概也是从那时起,工业技术的创新开始为非技术工人带来了工作机会,并扩大了企业的规模。工厂的新技术提高了穷人的实际收入,并不需要他们付出额外的努力;不仅如此,它还加强了政治系统的力量,使其更有能力重新分配收入,以及抵御社会动乱。长远来看,技术并不天生倾向于抑制人类在能力和动机方面的差异,也可能增强它。有些技术是相对平等的,众多独立的工人会做出大致相同的产出;但其他技术则会将权力和财富集中到少数能者手里,剩余的大多数人不过是农奴而已。历史和技术都以不同的方式,塑造了不同的国家。工业时代是一种形态,而信息时代正在孕育另一种形态,相比被它取代的旧制度,它将更少暴力,因而更加精英化,也更加不平等。

\section{阿蒙的萝卜}
19 世纪末,一些经济学家——其中威廉·斯坦利·杰文斯(WilliamStanley Jevons)是英国最杰出的一位,开始发展数学经济学。最早将概率理论应用于重大社会问题的,是德国的经济学家奥托·阿蒙(Otto Ammon),他的作品在 1899 年被卡洛斯·克劳森(Carlos C. Closson)首次译成英文,发表在《政治经济学杂志》(Journal ofPolitical Economy)上。这篇文章的题目是“概率学说的一些社会应用”。今天的人们可能会觉得,这样的文章只有考古者才感兴趣;事实上,它所研究的经济问题已再次出现,而它的处理方式依然深具启发。

奥托·阿蒙感兴趣的是,社会中的能力分布以及它与收入和地位的关系。他以四个骰子为研究出发点,每个骰子有六个面,他统计了可能出现的总点数。在 1296种可能的点数组合中,有些总数出现的频率要高得多。

从上表(表略)可以明显看出,高总数和低总数都很少出现。有两种可能的总数,但其中前 4 大的总数,在 1296 种组合中只可能出现 35 次,后 4 名也只出现 35次。中间七组的总数则可能出现 884 次,中间三分之一总数出现的概率则超过其余三分之二的投掷结果。这就产生了概率论中心拥挤的特征。

奥托·阿蒙认为,掷骰子的随机分布与人类能力的分布状态是一致的。在他写作的时候,还没有智力测试和智商的概念,他靠的是弗朗西斯·高尔顿(FrancisGalton)以前关于智力的研究。阿蒙并不认为,整个社会的效用或个人生活的成功仅仅取决于智力。他列举了“三组心理特征,它们在很大程度上决定了一个人在生活中能够达到的高度。”这三组特征是:

1. 智力水平:其中包含了人类所有理性方面的能力,如理解力、记忆力、判断力、创造力,以及其他任何属于这一领域的素质。

2. 道德能力:包括自控力、意志力、勤奋、毅力、节制、顾家、诚实等等。

3. 经济能力:如商业才能、组织才能、技术能力、处世谨慎、精于计算、富有远见、勤俭节约等等。 

在这些精神素质之外,他还增加了第 4 组:

4. 身体条件:工作能力、耐力、抗压力、抵抗各种刺激的能力、活力、健康等等。 在奥托-阿蒙看来,这些智力、性格、才能和身体素质的分布概率,与骰子上的总分数类似。他还进一步指出,如果存在四个以上的变量,而且这些变量的变化程度超过了六度。比如,如果不是四颗骰子,而是扔八颗,那么就有不少于1,679,616 次可能的结果,但最高分 48 分,预期出现的概率仍然只有一次。

在决定人生成就的所有因素中,得分都很高的男人或女人,出现的概率比扔出四个六要小得多,可能跟扔出八个六差不多。然而,阿蒙指出,如果在这些素质中得分有高有低,可能会产生“天赋不平衡、不和谐的人,尽管他们在某些方面非常出色,但无法成功应对生活的考验。”。

\begin{tcolorbox}
像一座孤独的山峰,或者大教堂的尖顶,天赋异禀之人会从广大的平庸之辈中升起……天才的数量是如此稀少;要说因为社会制度不完备,导致‘众多’天才被耽误在下层阶级,几乎是不可能的。
\begin{flushright}
—— 奥托·阿蒙
\end{flushright}
\end{tcolorbox}


\subsection{素质与收入}
阿蒙接着谈到了收入分配的问题。当然,19 世纪 90 年代的统计数字远不如现在充分,但当时德国的官僚机构已经非常发达。在萨克森、普鲁士、巴登等德国其他州,奥托·阿蒙发现了一种收入曲线,既类似于他认为的人类能力分布状态,也和骰子点数出现的概率很接近。他还在查尔斯·布斯(Charles Booth)的《伦敦人的生活与劳动》(1892 年)一书中,找到了类似的数字。确实,布斯的社会分布,与阿蒙的概率论推出的结果基本如出一辙。布斯发现,在伦敦,25\%的人生活贫穷或者更差,51.5\%的人感觉舒适,15\%的人过着富裕或更好的生活。

如果把布斯分类中最差的两类人加起来,他们的比例就是 9.5\%。而在 20 世纪福利国家出现之前,人们通常把那些最穷的人说成是“被淹没的十分之一”。另外,在布斯分类中生活最好的两类人,加起来的比例则是 7\%。

从这些数据汇总,奥托·阿蒙得出了很多有价值的结论。他认为,广泛意义上的人类能力,决定了一个人在社会中的地位和收入。能力强,收入和社会地位自然就会上升。“就像一座孤独的山峰,或者大教堂的尖顶,天赋异禀的人会从广大的平庸之辈中升起。”他还认为,“所谓的社会金字塔结构,它真正的形式其实是一个扁平的洋葱或萝卜。”这种萝卜最上面的茎和最下面的根都很细。用萝卜来比喻社会金字塔是说得通的,因为像在现代工业社会中,它的多数群体是处于中间位置,而金字塔的多数则是在底层。

\subsection{萝卜状的社会}
现代工业社会确实都是萝卜状的,顶层是一小撮富人和高级职业群体,中间规模庞大,然后少数穷人处于底层。相对于中间,两头都很小。如果华盛顿不是这样的话,那么在现代的伦敦,百万富翁肯定比无家可归者要多。

所有这些都极具启发,但阿蒙的工作带来的最直接的好处,是使我们认识到,人类正在经历一场长期的重大转变,发生在社会上层和中层之间的政治与金融关系中。工业时代所需要的技能,正在成为过去时,而且毫无疑问,它不符合信息时代的要求。大部分人都能操作 20 世纪中期的机器,但这些工作现在已经被智能机器所取代;实际上,这些机器控制着人。整个需要中低技能的就业领域已经消失了。如果我们没错的话,这就是大部分就业岗位的消失以及现货市场工作重新配置的前奏。


\begin{tcolorbox}
然而,大多数失业的年轻人都不满足任何工作的能力要求,这是官方私下承认的事实。
\begin{flushright}
—— CLIVE JENKINS AND BARRIE SHERMAN
\end{flushright}
\end{tcolorbox}

\section{能者多劳}
我们就用 4 个骰子来简单比拟人类的能力分布。假设在工厂时代,人们的得分可以达到 $ 4 \times 2 $ 或者更高,那就意味着,95\%以上的人都在查尔斯·布斯所说“积极社会效用的最低限度”之上。事实也确实如此,在 20 世纪 40 年代和 50 年代,3\%就是当时设定的充分就业标准(即失业率)。假设信息时代要求的分数上升到 $ 4 \times 3 $,也就是最低分从 8 分涨到 12 分,那就意味着,近 24\%的人将低于“社会效用”的极限。

在社会的顶端也是如此。工厂时代对高水平的能力要求也许是$ 4 \times 4 $,而在信息时代,它可能上升到$ 4 \times 5 $,这样的话,有资格从事顶级工作(即收入最高的工作)的人,比例将从 34\%锐降到只有 5\%。

这些数字纯粹是基于假设。当然,我们不知道技能要求的变化会去到什么程度——或者说已经达到了什么程度,但上升是肯定的。鉴于萝卜的形状,对最低技能要求很小幅度的提高,就会导致大量的人丧失工作机会。同样,较高技能要求的小幅升高,也会使符合高级职位的人数大大减少。这样的转变正在发生,我们还不知道它的规模和程度会有多大。

实际上,确实有社会和政治方面的证据表明,这种转变正在所有先进的工业社会中展开,其速度正在加快,其规模已经很明显。对稀缺技能的奖励已经增加,且还在持续增加。传统型思想家也注意到了这一点,并表示非常不满。例如,罗伯特·H·弗兰克和菲利普·J·库克(Robert H. Frank and Philip J. Cook)所著的《赢家通吃的社会》(The Winner-Take-All Society)一书,就指出了在美国社会的众多领域中,呈现出最具才华的竞争者获得超高收入的趋势。与此同时,中等技能者的机会在减少;而相当多的低技能者,虽然还能在小规模的服务业中找到一份活计,但已经掉出了舒适生活的社会圈层。

如果信息社会在高低两端都提出更高的技能要求,那么除了前 5\%的人,其他人都会相对地处于劣势。前 5\%将会获得超额收益,他们不仅赚得更多,而且能保留的比例也更大。不过,他们将完成的工作,在世界上也是前所未有的。主权个人将从他们中间崛起。在信息时代,收入分配的萝卜形状会更像 1750 年的,而不是 1950 年的。

那些中低技能的人群,在社会的灌输之下,期待更平等的收入与更高的消费水平,将会陷入挫折与焦虑。随着更多国家的经济更深入地吸纳信息技术,人们将会看到,一个基本无力就业的下层阶级将会出现——在北美已经很明显。这就是正在发生的现实。它将导致民族主义与反技术偏见的高涨,对此我们将在下一章中详述。

工业时代可能是一段特殊的历史时期,在其此间,半愚蠢的机器为无技能者留下了赚钱的空间。今天的机器已经可以自我管理了;奥托·阿蒙的萝卜顶端的 5\%,将会得到信息时代的特别礼遇。对于前 10\%的认知精英来说,信息时代已是无与伦比;而对于这前 10\%中的前 10\%,认知精英中的精英,这是最好的时代。

在封建时代,100 个半熟练的农民才能养活一个骑马的高技能军阀(或骑士)。

信息经济中的主权个人不是军阀,而是掌握了创业、投资等特殊技能的大师。但封建社会中 100:1 的比率或许将回归。好坏姑且不论,21 世纪的社会可能比我们生活的 20 世纪更加不平等。

\section{大部分人将从政治的死亡中获益}
平等主义经济及其所支撑的国家,不可能不经历危机就自动消失。根据字面定义,“危机”往往只持续很短的时间,但我们可以想象,国家灭亡的创伤会回荡很多年。这种创伤当然不容忽视,稍后我们会对它的各个层面进行详细分析;但更重要的是不要忘记,在全球的许多地区,向信息时代的转型会促使产出激增,周边的相关收入都会提高。事实上,在那些从未充分享受过工业化的好处但现在向自由市场开放的地区,各阶层的收入都在上升或即将上升。

强迫性作为经济生活中的一个主要特征,在信息时代将被削弱,生产者将可以保留在过去一直遭到征收和再分配的财产。再分配通常意味着资产被拖入了低价值的用途,从而降低了资本的生产力。政客从最擅长投资管理的人那里攫取不成比例的财富,然后重新分配给那些不擅于投资的人。普遍来说,重新分配的收入都被投入到了低级的经济活动。将资源从系统的强制中解放出来,其效应在不同的国家会大不相同。这种资源的解封会导致福利国家的破产,会增强规模的不经济性,进而破坏大政府以及由大政府补贴的所有机构。而另一方面,在某些主权国家的统治下,一直饱受无法大规模组织之苦的人们,转入网络经济将减轻他们所承受的经济弊端。

\begin{tcolorbox}
如果世界以一个大市场来运作,每个雇员都将与世界各地具有同等工作能力的人竞争。那可是人数众多,而且其中很多人都充满渴望。
\begin{flushright}
—— 安德鲁·格鲁夫 英特尔总裁
\end{flushright}
\end{tcolorbox}


\section{地理优势将不复存在}
由于暴力的回报率不再上升,继续生活在一个可以掠夺他们的政府之下,对人们来说没有任何的好处。曾经很能干的政府,不再是积累财富的朋友,而只是敌人。

高额税收、繁重的监管成本,以及雄心勃勃的收入再分配计划,使得政府控制下的领土,将成为人们避而远之的商业环境。

在工业时期一直贫穷落后的地区,当经济从地理的限制中被解放以后,这里的人们将获得最大的利好。这与你听到的说法会正好相反。围绕着信息经济的到来与主权个人的崛起,主要的争议将集中在政治的死亡对“公平”产生的所谓不利影响。毋庸讳言,全球信息经济的出现,将给大规模收入再分配的制度造成沉重的打击。在工业时代,收入再分配的主要受益者是富裕国家的居民,这些国家的消费水平是世界平均水平的 20 倍。也只是在那些经合组织国家,收入再分配在提高非技术工人的收入方面,产生了明显的效果。

最大的收入不平等发生在各个国家之间;收入再分配对缓解这种不平等毫无作用。事实上,我们认为,外国援助和国际发展计划通过补贴不称职的政府,产生了降低穷国穷人实际收入的反作用。在我们分析信息革命对道德造成的冲击时,这是一个更需要深入考虑的问题。

\subsection{收入不平等加剧的一个世纪}
在工业时期,决定普通人一生收入高低的最大因素,是他碰巧出生和居住的政治管辖区。与今天富裕国家人们的普遍印象相反,收入不平等在工业时期其实是迅速加剧的。世界银行引用的一项估计表明,最富裕国家的人均收入,在 1870 年是最贫穷国家人均收入的 11 倍;到 1985 年已经猛增到了 52 倍。在全球层面,收入不平等在急剧扩大,但对于那些生活在富裕工业国家的人,情况并非如此。

所以,相比于国家内部,收入不平等在各国家之间上升得更严重。

在那些政府能力稀松而又可以大规模行使权力的国家,工业技术本身的特点,有助于削弱这些国家内部的收入差距,其原因我们在前面分析过。当暴力的回报率上升时,例如在工业时期,大规模运作的政府往往控制在其雇员手里。在这种情况下,要控制政府对资源的索取几乎是不可能的。只要权力的规模优于效率,政府对资源不受遏制的控制就会获得重大的军事优势。而政府由雇员控制的一个必然结果,就是收入再分配的急剧加速。几乎每个社会都有关于收入再分配的法规,至少也有一些临时性的、适用于某些特殊情况的。然而,如果仔细阅读这些向穷人提供援助的历史,你就会发现:当社会的贫困程度最低时,政府的“福利”往往更慷慨;而当大批人收入减少时,收入再分配更可能被削减。在 20 世纪后半叶,富裕工业社会的条件对实行收入再分配几乎是完美的。这就导致了,在那些条件有利的国家,非熟练工获得了更高的报酬。不久之后,甚至是那些根本不工作的人,也获得了高水平的消费。

\subsection{工业财富的悖论}

讽刺的是,也正是在这些国家,更多的人变得富有起来。不过,一旦你理解了前几章探讨的大政治动态,这个明显的悖论就顺理成章了。工业经济的主要部门需要大规模地维持秩序,以实现最佳的运作效果;这使得它们特别容易受到工会和政府的勒索——这些暴力组织急于使其支配的人数最大化。但普遍实施的收入再分配,并没有完全扼杀工业经济的发展能力。所以,任何有幸在工业化高峰之前出生在西欧、前英国殖民地和日本的人,都可能比生活在南美、东欧、前苏联、非洲及亚洲大地上具有同等技能的人要富裕得多。由于种种障碍,世界上的大多数人口,在现代时期的大部分时间里,没有享受到自由市场的各种好处。而信息技术强大的正向作用,就包括克服这些障碍。


\begin{tcolorbox}
穷国的地方性特色,明显不适合大规模组织的有效开展,特别是那些必须在广大地理区域内运作的大规模组织(如政府)。
\begin{flushright}
—— 曼瑟·奥尔森(MANCOR OLSON)  
\end{flushright}
\end{tcolorbox}

\section{规模不经济与增长迟缓}
正如曼瑟·奥尔森所证明的,20 世纪落后的国家,本身并不是因为缺乏资本或专门的技能。在 1987 年,即柏林墙倒塌前两年,奥尔森发表了一篇文章《规模不经济与发展问题》(Diseconomies of Scale and Development)。文中写道:“如果资本供应在穷国确实是不足的,那么它的‘边际生产力’和利润率就应该比富国的高。但许多接受了大量外国援助的国家,经济增长率依然很低;在穷国建造的现代化工厂,生产率同样很低;这些事实都削弱了‘资本稀缺’导致不发达的说法的可信程度。”的确如此。如果资本或技术稀缺是主要的不足,那么在穷国的管辖范围内,这两者所获得的回报应该比发达国家高。熟练工人和资本会涌入到这些地区,直到收益区域平稳。而事实却恰恰相反。受过教育的人会从落后的国家大量外流。少数幸运儿想方设法在这里积累了资本,也会尽快转移到瑞士和其他先进国家。

\subsection{好政府无法被进口}
我们同意奥尔森的一个观点,他认为落后国家真正的发展障碍,是一个无法从国外简单借入或进口的生产要素,那就是政府。随着 20 世纪的发展,这个问题更加严重。在 1900 年,英国、法国以及欧洲其他一些国家,都在向那些本土势力无法有效大规模运作的地区,输出合格的政府。但是,20 世纪大政治条件的变迁,提高了这种活动的成本,降低了其回报。殖民主义,或者说帝国主义,到处都不受人待见,也不再是一件划算的买卖。技术的转变,提高了将权力从中心推行到外围的成本,并降低了对其进行军事抵抗的成本。因此,帝国主义列强纷纷退出,或仅保留着百慕大及开曼群岛等小块飞地。

\begin{tcolorbox}
如果说后殖民时代的民族国家已经成为了进步的桎梏——在 1980 年代末,非洲越来越的批评家认同这一观点,这其中的主要原因是非常明显的。不管国家如何宣传,它们并没有解放和保护公民;相反,其总效用是限制和剥削,或者说,它们没有任何社会意义上的正向价值。
\begin{flushright}
—— 巴泽尔·戴维森(BASIL DAVISON)    
\end{flushright}
\end{tcolorbox}

在没有欧洲人定居的国家,取代殖民统治的本土政府,其领导人和行政人员,都来自于没有任何管理大型组织经验或技能的人口。在很多情况下,特别是在非洲,殖民国家离开以后,当地继承下来的基础设施,很快就遭到抢夺、破坏,或者任其失修。电话线被拾荒者拆毁,并砸成手镯。道路得不到维护。在没有被欧洲人定居的国家,取代殖民统治的土著政府,其领导人和行政人员都来自于没有什么经验或技能来管理任何类型的大型企业的人口。在许多情况下,特别是在非洲,从离开的殖民国家继承下来的基础设施被迅速掠夺、破坏,或任由其失修。电话线被拾荒者拆毁,并被锤成手镯。道路不再被维护。由于路基崩塌和机车故障,铁路线变得毫无用处。在扎伊尔(非洲中西部国家),比利时人精心设计和建造的交通基础设施,到 1990 年几乎全部消失。只有几艘吱吱作响的河船还在继续飘荡,其中一艘被独裁者接管,改成了浮动的宫殿。

破败的通讯和交通,反映出落后民族国家在维持秩序方面的无能。它们使物价居高不下,又将世界上大多数人口的发展机会丧失殆尽。对此,奥尔森特别强调:


\begin{tcolorbox}
首先,交通和通讯的不畅,往往迫使企业主要依靠当地的生产要素。当一个企业的规模扩大时,它不得不到更远的地方去获取生产要素,而交通和通讯系统越差,这些要素的成本随着产量增加而上升的速度就越快。其次,糟糕的交通和通讯系统之所以不利于大型企业,更重要的原因是,它们使有效协调这些企业的难度大大增加。 
\end{tcolorbox}

\subsection{减轻不良政府带来的负担}
当信息技术将收入能力从人们居住的地理位置解脱出来,全世界富有进取心的穷人,将是最大的受益者。像数字移动电话这样的新技术,可以独立地发挥通信功能,不再依赖于地方警察能否保护好辖区内的电线杆,使铜线不被人偷走的能力。

随着无线传真和互联网的普及,穷困潦倒的邮政员工是否会为了一张邮票而偷藏信件,也不再那么重要了。

在很多情况下,有效的通讯甚至可以取代货物和服务的物理运输。良好的通讯系统,以及大幅提升的计算能力,不仅可以使人们更便宜高效地协调复杂的活动,还将降低规模经济的效益,消解大型组织。这些变化都有利于落后国家的人民,可以减轻他们生活在无能政府下所遭受的伤害。信息革命将使政府是否称职变得无足轻重。因此,生活在传统贫穷国家的人们,将更容易克服其政府对经济增长造成的阻碍。

\subsection{信息时代的机会平等}
在信息时代,人们熟悉的地理优势将很快被技术所改写。具有同等技能的人,不管生活在什么地方,他们的收入能力将变得更加平等。这种现象正在发生。那些利用强制力和地方优势进行收入再分配的机构,将逐渐式微,国家内部的收入不平等会不断扩大。全球性的竞争,会使各个领域内最有才华的人收入大涨,就像现在的职业运动员一样,不管他住在哪里。在全球市场上,卓越表现产生的边际价值是难以置信的。

公众辩论的焦点,虽然还集中在经合组织国家日益严重的“不平等”,但世界各地的个体,将会享有更多平等的机会。人们无需再生活于一个大规模运作良好的国家,才能够获得成功。在一个更加平等的竞争环境里,先天的能力以及发展这些能力的意愿,会更加被看重。工业时代的管辖优势,曾导致贫富之间的差距不断扩大,在信息时代它将发生重大的变化。

\subsection{贫穷地区将获得更高回报}
随着网络经济的发展,贫穷地区的政府对自由市场形成的阻碍,将被大面积瓦解。

相应地,在这些地区,稀缺的资本和技能会得到更高的回报;正如 1950 年代的发展理论家的设想。而且,外来的资本和技术会很容易获得。新兴经济体将不再像工业时代那样,严重依赖本地的生产要素;它们更有能力利用远方的资本和专业技术,从而获得更高的经济增长。不管今天无能的政府是否会变得更加诚信,或者更好地保护产权,这样的改变都将发生。网络空间鞭长莫及,腐败政府将无法阻止其治下人民追求经济自由和美好生活。

\subsection{正向强化}
在新型的网络经济中,信息技术几乎完全可以移植,这会遏制工业时代兴起的集中管辖优势的做法。当越来越多的国家之间竞争愈加强烈,地方优势会演变出新的形式。国家主权将被商业化,而不再是掠夺性的。在竞争的压力下,各国政府不得不制定政策,去吸引能对经济发展做出最大贡献的客户,而不是懒汉和寄生虫。

这与 20 世纪的普遍做法有云泥之别。按照民族国家的意识形态,生活可以而且应该被积极地调节,通过补贴不理想的结果,惩罚理想的结果。成为穷人是不好的,因此要对穷人进行补贴;成为富人是好的,因此要对富人征收惩罚性税收,以此使社会更加“公平”。

这整套的方针政策根植于一种大政治基础,它经得起各种反对意见,所以,补贴所导致的不良后果根本不重要。那些被重新分配的财富,其中所包含的技能、勤劳和聪明才智,也没得到多少应有的解释。20 世纪的政治观点认为,“公平”的结果必须是平等的。

\subsection{新样板}
在 21 世纪新的大政治条件下,可以通过市场测试去调节过去由政治主导的领域。

市场模式假定,奖励理想的结果,惩罚不理想的结果,就可以更好地进行调节。

贫穷不可取,富裕是可取的。因此,激励机制应该奖励致富,并鼓励人们为消耗的资源付费。当人们能够留住更多自己创造的收入时,生活就更加“公平”。

相比正在结束的这个世纪,这种观点在新的千年里会非常流行。而且,它被接受的程度,也将是前所未见的,因为它具备深厚的大政治基础。在信息时代,资本的流动性越来越强。赚大钱的能力不再受制于特定的地理位置,不再像过去那样,只有通过操纵自然资源才能创造大量的财富。随着时间发展,利用高度便携的信息技术,人们创造资产会越来越容易;而且,这些资产受暴力杠杆的影响,也远远小于以往任何形式的财富。

武断的政策法规,如果创造的市场利益不能抵消其增加的成本,很快就会行不通。

强大的竞争,正在使全球范围内的商品、服务、劳动力和资本的价格趋于平等。

政府任意实施各种政策的自由,不会再像它们习以为常的毫无制约。任何一个政府,如果它对某项经济活动施加的规定,比其他主权国家更繁琐;结果很简单,这项活动将从其统治下流走。当然,在某些情况下,把不受欢迎的活动驱逐出去,会增加市场的满意度,提升该地区的吸引力,促进繁荣。从这个意义上说,这种规定类似于连锁酒店在内部实行的“堂训”。酒店老板如果禁止顾客在大堂内光脚或吸烟,肯定会失去一些顾客;但拒绝某些素质差的人,不会在整体上减少酒店的客人,甚至也不会影响其收入。衣着整洁的不吸烟者,可能会支付更高的费用,因为酒店把光脚吸烟客排除在了外面。同样,如果某地区的法规下使渲染厂的经营成本很高,甚至无法盈利,渲染厂就可能会搬到其他地区,但这并不使该地区的总体收入减少。

这些例子表明,在少数情况下,法律法规可能具有积极而非消极的市场价值,特别是在未来的世界,管辖区数量将迅速增加的情况下。在很多地方,制定高标准的规则,以维护公共卫生、清洁的空气和水源,会受到高度重视。在某些领域,也许还会出现具有异国情调的法规和另类的契约,是房地产开发商或酒店为了迎合特定的细分市场而实施的。

\subsection{网络空间没有海关大楼}
我们预计,主权的商业化,很快将导致许多重大的领土主权被下方。如边境管制,至今还在制约着制成品和农产品的贸易,信息技术不会受它的管制。这样的发展影响深远。它意味着随时间推移,贸易保护主义将逐渐失效,因为相比实物产品,信息交易对于创造财富更加重要。它还意味着,小地方想要进入市场赚取收入,将越来越不依赖于大规模的政治管辖。

在服务部门工作的人,以前一直受到保护;信息技术将把他们暴露在全世界的竞争当中。20 年前,如果多伦多的一家公司想要雇用一名会计,这个人必须在多伦多,在附近的社区,在通勤范围内。在信息时代,位于布达佩斯或印度班加罗尔的会计,都可以胜任这项工作,而且是以加密的方式,通过互联网下载所需要的材料。有了卫星链路的即时通讯,通过调制解调器和传真,与世界任何地方的沟通,都只在弹指一挥间。一个人如果需要股票分析师,以华尔街请 1 个的价格,他可以在印度请 27 个。根据摩尔定律,信息技术每 18 个月可以提高一个量级,甚至更高;在这种趋势下,越来越多的服务人员将面临价格竞争,政客们恐怕也无力阻止。这种竞争不仅适用于会计,最终它将全面适用于高学识的职业;数字律师和网络医生会在信息时代大量涌现。

\subsection{民族国家的死亡警报}
民族国家,随着过去在其领土范围内攫取的经济利益的丧失,最终会在自身沉重的债务下崩溃。但是,所有民族国家都处于死亡警戒状态,并不等同于它们注定要在同一时刻消亡。远不是这回事。在那些多数人口的收入都停滞不前甚至下降的大型政治实体中,要求权力下放的压力往往最为激烈。而在拉丁美洲和亚洲,那些人均收入迅速增加的国家,可能还会存续几代人的时间,或者直到那里的终身收入前景相当于以前富裕的工业国家。到那个时候,就很难再找到成本替代型的便宜果实,经济增长将对政治提出更强的挑战。

我们也不太相信,只有一个大都市的民族国家,会比拥有几个大城市的存续更长时间,因为后者拥有多个利益中心和可回旋的腹地。

另一个刺激刺激权力下放的因素,是中央银行的高负债率。加拿大、比利时和意大利,相对负债率最高的三个富裕工业国家,也是独立主义运动高涨的国家,这绝不是巧合。这三个国家都拥有长期的预算赤字,当前的国债已经超过了 GDP的 100\%。随着国债的增加,独立运动的吸引力也在增加。在意大利,北方联盟已经成为一个充满活力并广受广受的地区政治运动。它的纲领基于一个简单的数学计算:对于意大利北部或“帕多尼亚”地区来说,如果它们的大部分收入不被抽走去补贴罗马和贫穷的南部,那么它们将比瑞士还要富有。北方联盟提出了一个显而易见的解决方案:从意大利独立,从而摆脱复利损失的可能后果。同样,在比利时,国债超过 GDP 的 130\%。弗拉芒人和瓦隆人,就像离婚前充满敌意的夫妻一样,上演这各种算计和缠斗。弗拉芒人中的少数派认为,他们在不公平地补贴瓦垄人,把比利时一分为二,会大大改善他们的经济状况。持这种观点的弗拉芒人正越来越多。

加拿大的情况在细节上有所不同。现在要求独立的地区主要是法属加拿大,在历史上一直受英属加拿大补贴。随着联邦债务和财政赤字的增加,魁北克省逐渐意识到,这种形式的收入再分配将会衰减。因此,魁北克开始玩弄其它十年前尚欠缺的吸引力——承诺取消支付加拿大联邦税来提高其境内居民的税后收入。独立派领导人还提出,魁北克不应承担自己那一份联邦债务,应该从加拿大分离出去。

英属加拿大地区的人们抵制这种主张,并且反感它造成的不良影响,因为他们深知,多年以来向魁北克转移了大量的财富。尽管如此,魁北克党的号召力依然很强,公投脱离加拿大,看上去只是时间问题。当其他民族国家的财政状况恶化时,等待它们的将是类似的命运。

另一个不利于加拿大联邦长期存续的因素是,加拿大人口稀少,但又庞大的工业基础设施需要维护。向信息时代的转型必然使这些物理基础设施贬值。随着远程办公取代工厂雇员和办公室人员,高速公路和其他交通要道能否得到重建和妥善维护,就没有那么重要了。当财政危机全面爆发,会有越来越多的加拿大邦,回归到亚当·斯密在 18 世纪提出的排他性公共产品融资观念。他在《国富论》中写道:“伦敦街道的照明和铺设,如果由国库出资的,能造得像现在这么妥帖,同时价钱还这么低吗?况且,这笔费用如果不是来自于伦敦特定街区、教区或社区缴纳的地方税,那势必要从国家一般收入项下开支;这样的话,国内不能受到这街道利益的大部分居民,就要无端分担这个负担了。” 如果把伦敦换成多伦多,你就会发现,阿尔伯塔省和卑诗省等很多地方的人,都会在脑子里算这个账。权力下放的逻辑是可以传染人的。

加拿大解体以后,美国西北太平洋地区的独立活动将会明显增加。阿拉斯加州、华盛顿州、俄勒冈州、爱达荷州和蒙大拿州的居民会发现,与成为独立主权的阿尔伯塔省和卑诗省进行竞争,他们处于明显的劣势。

\section{民族国家之后}
取代民族国家的,首先是省一级较小的管辖区,最终是小型的主权形式,各种飞地,类似中世纪被其腹地围绕的城邦。在那些头脑被灌输、认为政治是头等大事的人看来,这种现象可能很奇怪。但一般而言,这种新型迷你国家的政策,更多是基于企业式的定位,而不是政治斗争。这些新式的、碎片化的主权国家,就像酒店和餐馆,将迎合不同口味的需求,在其公共空间内执行具体的规定,以吸引其细分市场内的顾客。这当然不等于说,这种游牧形式的组织,在保护方面,就没有自己的问题。下一章我们会对此进行讨论。

\begin{tcolorbox}
城镇空气带来自由。
\begin{flushright}
—— 中世纪俗语
\end{flushright}
\end{tcolorbox}

\subsection{苍白之地的非公民们}
尽管有种种困难,但以人类的聪明才智,总会找到方法去创建机构,以抓住繁荣昌盛的机会,即使此种需求是来自支付能力低下之人。可想而知,当潜在的客户是地球上最富有的群体时,这种趋势会多么明显。当过时的产品、组织甚至政府失去吸引力,并且看不到将有所改善的前景时,退出或“用脚投票”终归是一种选择。例如,想一想中世纪城镇的发展,它们就是农奴逃避封建奴役的避难所。

在即将到来的退出民族国家的变迁中,它们将扮演顺应变化的新型管辖区的角色。在中世纪,接受“苍白之地的公民”(the citizens of the Pale),即逃离某些领主的外地人,是违反封建法律和主教权威的普遍惯例的。尽管如此,对于那些逃离的人来说,往往是一种成功的选择,并为削弱封建主义的控制做出了重大的贡献。正如中世纪历史学家弗里茨·罗里格(Fritz Rorig)所说,逃离世俗领主的农奴“在一年零一天后,就会成为城镇的自由市民。”我们有理由期待,未来会冒出新的庇护机构,根据“新的法律原则”,为民族国家的公民提供财务庇护;就像中世纪的城镇,庇护生活在其屋檐下的封建居民一样。

经济学家阿尔伯特·赫希曼(Albert O. Hirschman),在 1969 年首次出版的《退出、发声与忠诚》(Exit, Voice, and Loyalty)一书中,探讨了“用脚投票”理论的微妙之处。他预见到,技术发展将增加退出的可能,作为应对国家衰退的一种策略。他写道:“只有当各国因为通信和全面现代化的进步,而开始变得相似时,过早或过度退出的危机才会出现……”而这正是当前发生的情景。信息技术在极速削减各国之间的种种差异,使退出的选择更具吸引力。当然,赫希曼的“过早和过度退出”的措辞,是出于被抛弃的国家的理想角度。毫无疑问,中世纪的领主们会认为,他们的农奴“过早或过度退出”,逃到可成为自由民的城镇,给他们造成了极大的伤害。

回到我们前面的例子,假如出现一些迷你国,为来自垂死民族国家的流亡者提供庇护,并非天方夜谭。这些国家甚至会在庇护的条件上进行竞争。有些国家,也许位于北美的西海岸,会更努力争取那些不吸烟的人,以及难以容忍二手烟的人。

这种制度显然不受吸烟者的青睐;对他们来说,禁止吸烟的规则是一种武断的强制。

在大众政治的工业时代,这种意见分歧会导致互相斗争的政治运动,最终迫使一个或另一个群体遵守更强大的意愿。其实,当人们的选择互相排斥时,并不是非要压制很多人的偏好,才能解决相关的争论。

就像有人喜欢吃鹅肝,有人喜欢热狗,还有人喜欢吃豆腐。通常情况下,他们不会为自己的饮食偏好而争论不休,因为他们的饮食选择并没有捆绑在一起。没人强迫所有人吃同样的食物。然而,工业时代的大政治条件,确实迫使大众共同消费政府提供的各种公共甚至私人用品。原因何在呢?因为大规模经营有利于获得巨大的经济利益。因此,在工业时代,把庞大的管辖区划分为飞地,让每个人都能各得其所,即使重要事项也可以自选自决,是不切实际的。只有当管辖区的数量达到十倍或百倍之多时,亚当·斯密的排他性公共用品供给方式,才更容易被接纳。在信息时代,越来越多的主权国家会是小型飞地,而不是大陆帝国。有些可能是北美的印第安部落,他们将有权自己的保留地和自然保护区进行税收管辖,就像现在有权经营赌场,或不受限制地捕鱼一样。

信息技术可以减少贸易区衰解的诸多弊端,因此,新型的主权国家,会更多以俱乐部或亲和团体的原则去运作,而不再依照民族国家管理领土的方式。就像在今天,每个潜在的客户,是否具有同样的服装品位,或是否看同样的电视节目,对商家来说并不重要;在未来,是否每个人都符合同样的亲和喜好,对于界定分散型主权国家的治理风格,也没有想象得那么重要。

各种各样的品位选择,会使分散型主权的风格大相径庭,就像服装风格或电视节目,选择越来愈多。一些微型国家,可能会像特许经营的酒店集团一样联结在一起,或者共同经营,在警察服务或其他政府残余职能上,实现竞争优势。喜欢干净的街道、反感在桌板下找到口香糖的人,可能会对新加坡情有独钟,而“瘪四和大头蛋”(Beavis and Butthead,MTV 卡通剧)的粉丝肯定就不会。喜欢狂野夜生活的人,会选择澳门或巴拿马,或其他类似的地方。在一个管辖区感觉不舒服的客户,其他地方会很欢迎他搬过去。盐湖城也许是禁止吸烟的,但哈瓦那的新城邦——可能改名为基督山,则会笼罩在雪茄的烟云之中。

\begin{tcolorbox}
这意味着,随着智能被不断分发到所有网络的边缘,工业社会所有的垄断、等级制度、金字塔结构和权力格局,都将在这种持续的压力之下土崩瓦解。最重要的是,摩尔定律将推翻关键的中心,也是当今美国力量的关键物理集合,那就是大城市。那一大批工业城市,正依赖着生命支持系统在运转,每年吸走我们所有人大约 3600 亿的直接补贴。大城市,是工业时代留下来的包袱。
\begin{flushright}
—— 乔治·吉尔德(GEORGE GILDER)  
\end{flushright}
\end{tcolorbox}

特别讽刺的是,微型主权或“城邦”国家的重新出现,可能会伴随着城市的空洞化。在很大程度上,大城市是西方工业主义的产物。它因工厂系统而崛起,目的是在制造高自然资源含量的产品时,获得规模经济的优势。

在 19 世纪的开端,10 万人以上的城市就属于超级城市;在亚洲以外,没有超过百万人口的城市,而亚洲的人口统计值得怀疑。1800 年,美国最大的城市是费城,人口为 69403 人;纽约只有 60489 人;巴尔的摩是美国第三大城市,有 26114名居民。大多数后来成为欧洲大都市的城市,以 20 世纪的标准看,当时的人口都是很少的。伦敦的人口为 864,845 人,可能是世界上最大的城市。1801 年,巴黎的人口为 547756 人,是欧洲唯一一个人口超过 50 万的城市。里斯本的人口为35 万。维也纳的人口为 25.2 万。柏林到 1819 年才勉强超过 20 万人。马德里的居民为 156,670 人。1802 年,布鲁塞尔的人口为 66,297 人。布达佩斯的人口只有 61,000 人。

显然,人们很容易认为,大城市的发展是人口增长的直接结果。事实上,并不一定。如果地球上的所有人,都挤到德克萨斯州,每家每户都有独立的房屋,而且带院子,也填不满德克萨斯。在经典的研究报告《论 19 世纪城市的成长》(TheGrowth of Cities in the Nineteenth Century)中,阿德娜·韦伯(Adna Weber)论述到,仅凭人口增长并不足以解释,为什么人们都生活在城市环境中,而不是分散在乡村。1890 年,孟加拉的人口密度和英国差不多;然而,孟加拉的城市人口仅占 4.8\%,而英国则是 61\%。

历史上,城市都用围墙与农村隔离开,以防止掠夺者和下层阶级进入。在 19 世纪和 20 世纪,工业就业的增长,创造了大城市。现在,随着工业主义开始消退,大城市已经脆弱不堪。这种变迁的最佳参照,就是 20 世纪中期的工业城市底特律。曾几何时,世界工业产值的很大一块都要经过底特律。如今,它已经被掏空,只剩下一个外壳,充满了犯罪和混乱。在底特律市中心的许多街区,一栋栋废弃的建筑被烧毁或拆掉,给人的印象是,这座城市好像经历了二战轰炸机的一些列空袭。

底特律矗立在那里,提醒着世人,工业城市已经时日不多。随着价值更多地来自于信息和思想,而不再是自然资源,它们都将崩溃消失。很多城市已经过于庞大,无法支撑自身的重量。要维系一个大都市的运转,需要大量的支持系统,能够有效地大规模运行。数百万人挤在一起,意味着更多犯罪、破坏及随机暴力的威胁,城市的脆弱程度大幅增加。在工业时代,对这些风险的治安管理,其代价是由大规模的经济生产来偿还的。

在信息时代,只有能提供高质量的生活,并以此支付其维护成本的城市,才能够保持活力。住在远方的人,不再承担对其进行补贴的义务。要衡量一个城市的生存能力,一个很好的标志是:生活在城市核心区的人,是否比生活在外围的人更加富有。即使南本德(美国印第安纳州的城市)、路易斯维尔和费城的最后一家好餐厅关了门,布宜诺斯艾利斯、伦敦和巴黎依然是众人向往的生活和经商胜地。

\subsection{乡村国家}
一些城邦可能仅仅是一块飞地,并没有附属的城市。所以,把它们看作是村庄国家或者乡村国家,可能更加合适。自然资源的天赋价值,将以新的方式体现出来。

如果你可以在任何地方做生意,那你很可能会选择风景优美的地方;在哪里你可以尽情地深呼吸,而不用担心致癌污染物。通讯技术将语言沟通的困难降到了最低,搬到任何环境宜人的地方,你都可以很容易在那里生活下去。人口稀少、气候温和、人均耕地资源丰富的地区,如新西兰和阿根廷,会比较有优势;因为它们的公共卫生标准很高,并且可生产低成本的食品与可再生产品。随着东亚和拉丁美洲数十亿人生活水平的提高,对这些产品需求的增加,会使这两个国家进一步受益。

\subsection{不等价定理}
经济学家关于人类行为的许多假设,都根植于地方的专制。一个明显的例子是李嘉图的“等价定理”,该定理认为,在一个存在巨额赤字的国家,公民会调整自己的预期,以备将来出现更高的税率去偿还债务。从这个意义上说,通过税收和通过债务为支出融资,二者具有“等价性”。至少在 19 世纪李嘉图写作时,具有这样的等价性。不过,在信息时代,面对为赤字融资而加税的前景,理性的人不会存更多的钱,他会转移自己的住所,或者把交易安排在其他地方。就像生产商会将供应商分类,以寻求最低的成本;同样的道理,这些人更有动力去寻找其他的保护供应商。这样做的好处,比起更换一个塑料管供应商获得的利润提升,简直是天上地下。可想而知,主权个人和其他具有理性的人,会从背负着巨额无准备金债务的国家中逃离。

在信息时代,负债少、要求客户支付的成本少,这样的廉价政府,将成为创造财富的首选之地。也就是说,在负债率低、政府已经重组过的地区,开展业务的商业前景更加有吸引力,如新西兰、阿根廷、智利、秘鲁、新加坡以及亚洲和拉丁美洲其他地区。与北美和西欧尚未改革、成本高企的经济体相比,这些地方是创业致富的优越平台。

\subsection{地方性价格异常将遭到侵蚀}
信息成本的大幅下降,会使大部分的地方价格优势荡然无存。买家不仅可以搜索对比大量的商家,以寻找最低的价格;他们还可以通过远程服务,跨越国界去购物。即使那些很难分析的产品,人们也可以轻易地比较它们的特点,如保险。而且,它还可以规避当地许可程序对贸易的限制。因此,在任何的领域,只要它地方性的价格异常,可能被外来的信息和竞争所侵蚀,那么它的利润率都会下降。

\section{新的组织要件}
在参与者之间的互动方式上,信息经济将与工业经济截然不同。企业利用高昂的交易和信息成本,而形成的长期组织优势,很多都将被信息技术所消解。信息时代将是“虚拟公司”的时代。

很多分析家,比我们更了解信息技术,但他们还完全没有看到,信息技术注定要变革经济组织的逻辑。这种新技术,不仅打破了边界和壁垒,它还彻底改变了计算的“内部”成本。少数企业,没有暴露在大规模的跨境竞争中,因而受到的影响较小。但即便如此,它们也会因为信息和通信技术的改进,而面临新的组织要求。迅速下降的信息和交易成本,将决定性地降低规模经济的效益;工业时代催生出长寿企业和终身雇佣制的激励机制,也将因此而失效。

\subsection{为什么开公司?}
亚当·斯密等古典经济学家,在公司规模的问题上,几乎都保持着沉默。他们没有讨论过,什么影响了公司的最佳规模,为什么公司采取这样的形式,乃至公司为什么会存在。为什么企业家要雇佣员工,而不是把每一项工作任务放到拍卖市场,在独立承包商之间进行招标?诺贝尔经济学获奖者罗纳德·科斯(RonaldCoase),通过追问这些重要的问题,帮助开创了经济学的新方向。他提出的答案,暗示了信息技术对商业结构的革命性后果。科斯认为,公司是克服信息不足和交易成本高昂的有效途径。

\subsection{信息和交易成本}
要理解这其中的原因,请想一下,在工业时代,你要运营一条流水线,如果没有公司协调各种活动的话,会面临都是障碍。原则上,不在单一公司的监制下集中生产,汽车也是造出来的。经济学家奥利弗·威廉姆森(Oliver Williamson),与科斯相并肩,是创新公司理论的另外一位先驱。威廉姆森定义了六种不同的公司经营与控制方式。其中一种是“企业家模式”,在这种模式下,每个工作站都由专家控制和运作。还有一种,威廉姆森称之为“联合工作站”,在这种模式下,“各种中间产品由每个工人跨越不同的阶段进行转移。”没有什么实际的理由可以说明,为什么成千上万的雇员不能被一群独立的承包商所取代,然后承包商在工厂里租下车间,竞标零件或其他服务,如组装车桥,或者把挡泥板焊接到底盘上。但是,如果你想找一家由独立承包商组织和经营的工业主义汽车厂,你肯定找不到。

\subsection{协调问题}
如果没有一家单独的公司进行协调,就开展大规模工业生产的话,会使经营所得的大部分经济利益消耗殆尽。要在许多家拼凑起来的小公司之间进行协调,会产生大量的交易问题,致使生产线失灵和停摆。这样一个系统要正常运作,各承包商之间必须不停地进行谈判。如此一来,大量的承包商和企业家将无心专注生产,不得不把时间和注意力浪费到确定零部件的价格,以及制定朝三暮四的交易条件上。就连监督生产这样简单的事,也会成为一个难题。

\subsection{行动的批准}
这样一伙互相独立的组织,要是由他们来组装一台车,开发或重新设计汽车模型的话,只能是一场噩梦。可想而知,当设计师推出一种新的造型,他要说服数百家独立承包商接受零部件上的改变,要费多大的气力。在实践中,这基本需要全体一致的同意。任何人提出或反对修改产品规格,都会阻碍车型的改进,或提高车型导入成本,进而损害大规模经营的收益。

\subsection{不必要的谈判}
由独立承包商租用(或单独拥有)的装配线有很多弱点,而通过一家公司来进行运作,则可以有效地避免。数千人聚集在同一屋檐下,互相合作去制造某款产品,在这个过程中,个体承包商发生死亡、疾病或财务失败,都是司空见惯的事情。

当然,通过市场竞标,可以找到替代的承包商。但是,每一次替代都需要进行协商,例如,继任者要买断前任的业务等。此外,还要谈判工厂车间的租金并达成协议,或许还需要重新租赁焊接机或用于冲压尾灯插座的压力机。这些工作繁琐至极。

\subsection{激励陷阱}
在工业时代的生产条件下,独立承包商组成的装配线,还面临另外一个致命的难题,那就是各承包商的资本需求差别巨大。例如,生产仪表盘开关的塑料模具,就很便宜;而铸造发动机缸体或冲压金属挡泥挡的设备,则可能需要数百万美元。

在上一章我们分析过,流水线生产的资源含量占比高,而且特别强调顺序性,这不可避免会给高资本成本带来问题。承担资本密集型任务的承包商,基本需要依赖他方的合作来摊销投资成本。高资本要求的承包商能否筹集到资金,投入运营并获得盈利,主要取决于他们能否确保,会得到低资本要求参与者的合作。而在很多时候,他们得不到合作。

小公司有很强的动机去敲诈大公司。某些在流水线上操作特殊功能的承包商,投入的资金可能很少;在关键的时候,他们就会采取不合作的态度以谋取利益。就像罢工的工人一样,他们会以这样或那样的借口关闭生产线;自己付出的代价很小,但会给那些高资本投入者制造很大的痛苦。在整个生产过程中,会不断地发生这样的博弈,小本承包商利用阻挠产出的能力,向高资本承包商勒索赎金。小承包商从大承包商那里榨取附加费用的首发,会大大降低系统的效率。

\subsection{公司的解决方案}
总而言之,在工业时代,如果把生产分解给众多的个体承包商,那么,通过大规模的流水线经营所能收获的经济利益,大部分都会被消耗掉。单一的大企业可以有效克服这些弱点,尽管它也有自身的局限性。大企业充满官僚主义。但在某种程度上,官僚主义和等级制度正是工业时代所需要的。行政和管理团队负责生产监督与协调工作;大量的中层管理人员,将上级的命令向下传递,将其他信息向指挥系统反馈。公司的官僚体制还产生了簿记员和会计控制,并最大限度减少委托代理的问题——即员工未按公司的最佳利益行事。要在工业时代的条件下实现复杂的会计系统,需要很多人的工作。这样一套行政官僚机构,成本是惊人的;而且,无论生产是红火还是低迷,这个费用都要支付。所以,由于这些行政人员掌握着公司经营所需的关键知识,他们所获得的报酬,往往高于其技能在现货市场上的价格。

\subsection{“组织性松懈”}
大量的职业经理人和行政人员,都有一个问题:他们总想“俘虏”公司,以自己的利益为出发去经营公司,而不是为了股东的利益。例如,在工业时代,公司在置办高档办公家具、入会高级俱乐部等福利上,大手大脚的现象并不少见;这些福利往往由管理层享受,却未必能给投资者带去什么直接的回报。在一个复杂的企业中,很难通过外部监督发现,哪些间接开支属于必要,哪些属于员工的放纵和享受。此外,也很难防止相当一部分企业员工推卸责任。正是由于很难在技术上监督员工的表现,才产生了庞大的中层管理人员;但与此同时,监督监督者又成了一个难题。

这些问题造成了所谓的“组织性松懈”。这个词是理查德·赛特和詹姆斯·马奇(Richard Cyert and James March)发明的,出自 1963 年出版的《公司的行为理论》(A Behavioral Theory of the Firm)一书。

\begin{tcolorbox}
不管你有没有产出,报酬都是一样的。\\
不管你卖不卖力,报酬都是一样的。\\
不管你上不上心,报酬都是样的。
\begin{flushright}
—— 克里斯·德雷(CHRIS DRAY)
\end{flushright} 
\end{tcolorbox}

\subsection{“那不是我的工作”}
作为一种渴望永续经营的实体,大型工业企业有一个缺点,我们在前面讨论过,那就是容易受到工会的敲诈。它还有一些官僚主义的特征,虽然不像政府办公室里的那么夸张。在这些企业里,命令从高层向下流转,工作任务刻板陈腐并分门别类,每种任务都有严格的定义。各个工作种类之间的界限分明,简直就像管理高学识职业的卡特尔(垄断组织)所划定的。在工业时代,指望一个簿记员更换他桌上烧坏的灯泡,在很多人看来,就好像请律师帮你治感冒一样奇怪。人们不期望,在很多情况下甚至也不允许,员工跨越严格划定的职能界限。

“那不是我的工作”,是一句广为流传的口号,它凸显了工业时代的“组织松懈”。

每个人的工作,都被精确为一种刻板陈规,绝不容僭越;即使这种僭越可能大幅提高生产效率。公司官僚机构中的每个雇员,都是根据一种“资格”被聘用的;据说这种“资格”可以预测员工特定的职能表现。除了少数人例外,员工的工资标准一般基于工作类别而制定,在整个组织里差不多都是统一的。而在大企业的行政管理阶层,工作业绩往往无法具体衡量,所以,他们的工作节奏非常休闲,和国家官僚机构一样。总之,企业确实获得了大规模生产的经济效益,但它在其他方面也付出了低效率的代价。

\begin{tcolorbox}
在市场中,你采取某些行动,不会是因为某些人告诉你要这样做,或者是因为战略计划书第 30 页上这么说。市场中是没有工作边界的……没有命令,也没有来自高层的信号翻译,没有人把工作分门别类,拆成几块。在市场中,你有的就是顾客;而供应商和顾客之间的关系,从根本上说是非组织性的,因为它发生在两个独立的实体之间。
\begin{flushright}
—— 威廉·布里奇斯(WILLIAM BRIDGES) 
\end{flushright}
\end{tcolorbox}

\subsection{新的组织形式}
信息时代的大政治条件,将彻底地改变企业组织的逻辑。这其中的一部分是显而易见的。即使不论其他方面的作用,信息技术也将大幅降低信息处理、计算和分析的成本。它带来的影响之一,就是将降低雇佣大量中层管理人员进行生产监督的必要性。而事实上,先进的计算能力,会在很多方面,使自动化的机器工具取代小时工。即使在那些还离不了人的生产阶段,控制与协调的操作也基本实现了自动化。比起管理人员,安装了微处理器的设备能更好地监督流水线的作业。这些新型设备,不仅可以测量人们工作的速度和准确度,还可以自动编制账目,在零部件从库存中取出的那一刻,就可以同时下单订购。今天最小规模的企业,也用得起财务控制软件;这些软件核算财务的速度和精确程度,是那些几十年的最大的企业,倾尽其等级制度的全力也难以达到的水平。

我们在前面讨论过,信息技术减少了产品中的自然资源含量,并有利于分散的、非顺序性的生产输出,这将大大降低企业面临敲诈博弈时的脆弱程度。信息技术的这些特点,会使企业更乐意将以前雇员的职能,给外包出去。除此之外,资本成本更低,产品周期更短,独立承包商本身——包括一人公司,都拥有自己可支配的超级信息网络,都将加强这种趋势。不用多久,他们将使用一系列数字助理来执行办公职能,从接电话到秘书服务。数字仆人将扮演秘书、广告代理人、旅行社助理、银行出纳及政府官僚的角色。

\subsection{好工作的消失}
在信息时代,能够创造重大经济价值的人,越来越有能力保留自己创造的大部分财富。大多数从事辅助支持工作的雇员,过去吸收了企业中主要收入创造者贡献的大部分收入,将来会被低成本的自动代理和信息系统所取代。这意味着,一个企业不在内部保留某些职能,而是将其外包出去,更能够保证他们提供的服务质量是最好的。因为在企业内部,要对某些表现出色的员工进行奖励,相对来说更加困难。通过消除组织,虚拟公司将消除大部分的“组织性松懈”。

“好工作”将成为明日黄花。所谓的“好工作”,就像普林斯顿大学的经济学家奥利·阿森费尔特(Orly Ashenfelter)所说,就是报酬高于价值的工作。在工业时代,大量“好工作”的存在,是因为信息与交易的成本都很高。公司越来越大,其职能范围也越来越广,因为这样有利于获得规模经济效益。公司的膨胀,也得到了税法的补贴。在工业时代后期,高税收占据了主导地位,这人为地放大了长寿公司以及终身雇员制的优势。在大多数国家,税法以及相关的规章制度,大大提高了以项目为基础去组建和解散公司的成本。它们还往往迫使企业家,把独立承包商也归为企业的雇员。法律的干预,使得解雇一名员工的成本很高,也很难,不管他对企业的贡献多么微不足道;这进一步暂时性地扩大了“好工作”的供应。

事所必然,因为工业时代企业组织的特性,那些最具才华和能力的员工,虽然为组织创造了超额的增值收入,但他们所获得的报酬比例,肯定远低于其贡献的价值。这种情况在信息时代将发生改变。

微处理革命,正在急剧增加信息的可获得性,并大幅降低交易成本。这将使企业的职能分解。公司不再是长久性的官僚机构,而是将围绕着项目进行组织,就像电影公司在运用的方式。以前属于公司“内部”的大部分职能,都将外包给独立承包商。工业时代那些占据着“好工作”职位,但毫无建树,主要靠同事“掩护”的员工,很快就会发现,他们要在现货市场上进行工作竞标。很多忠诚、勤奋的员工也面临同样的命运。好工作将会过时,因为工作本身都将成为不合时宜的东西。

日本的大公司有一些极端的现象,员工会得到一份终身的工作。即使没有什么生产任务,他们也会被留用,有时候只是到“工厂角落里一张光秃秃的桌子边”坐着。不过,即使在日本,臃肿的白领劳动力也正在被裁减。《国际先驱导报》一篇报道的标题,点出了这一点:酸楚悲哀的分离:日本终身职位的文化在痛苦中到头了。

在后工业时代,工作就是你做的任务,而不是你“拥有”的什么东西。在工业时代以前,长久职业是闻所未闻的。就像威廉·布里奇斯所说:“1800 年以前,在很多情况下,工作指的都是一些特定的任务或项目,从来不是指一个组织中的角色或职位……在 1700 年到 1890 年之间,《牛津英语词典》中有很多术语的用法,像车夫工作(JOB)、医生工作、花匠工作,都是指一次性雇佣的人。而另一个经常使用的词汇‘工作’(WORK),是指偶尔的工作,也不是固定的就业。”所以,以前的大部分任务,变成了企业内部的职位,作为一种降低信息和交易成本的权宜之计;到信息时代,这些任务将迁移回现货市场。由于信息技术的发展,“及时地”控制库存和外包,是完全切实可行的。这是工作岗位走向死亡的起始步骤。像 AT\&T 这样的大公司,已经取消了所有固定的工作种类,它的很多职位现在都是临时性的。用布里奇斯的话说,“就业又变得临时性了,工作因情景而定,工种间的界限正在消失。”在网络经济的新天地,“独立承包人”将跨越各大洲远程办公,在信息时代的流水线上协同工作。

\subsection{好莱坞模式接管}
在未来的信息经济中,企业组织的典范可能是电影制作公司。这种公司会非常复杂,运营预算高达数亿美元。虽然它们操作的都是大型业务,但其实它们也是临时性的组织。一家电影公司制作一部上亿美元的电影,整个团队也许只在一起工作一年的时间,然后就解散了。参与制作的尽管都才华横溢,但他们不会认为,参与项目就等于找到了一份“长期工作”。项目结束后,灯光师、摄像师、音响师、服装师将各奔东西。他们也许会在下一个项目中重逢,也许不会。

随着规模经济效应的下降,很多信息密集型的商业活动,所要求的运营资本也同时下降,如此一来,企业会有强烈动机进行职能分解。公司运营会变得更加专项性与暂时性;更多企业会趋向于短期存在。为特定的目的把人才集结到一起的虚拟公司,会比长期存续的公司更有效率。随着加密技术的普及,资本税在竞争下被迫降低,维持公司“长久”存续的人工规模经济将不复存在。不管减税的速度是快还是慢,这种趋势无可阻挡。如果减得快,以项目为基础的运作方式的人为成本也会很快消失;如果减得慢,已经不合时宜的高额税负,将主要落在现有企业身上。而新型企业将以虚拟公司的形式运作,它们可以更好地逃避,垂死民族国家所施加的沉重负担。

在信息经济中,特殊的技能与天赋将前所未有地重要,但是,大部分职业间的人为界限也将被打破。在先进的信息检索和存储技术下,法律、医学和会计等行业的商业秘密和专业信息,将为所有人所用。记忆作为一种技能的经济价值将下降,而信息的综合与创造性应用将更加重要。

这种变革的全部影响,会被陈腐的监管所延缓。但长久来看,政府对网络经济的监管将不断枯萎,直至消失。任何人为的职业垄断和监管,提高了经营成本又未能创造市场价值,最后会遭到人们的唾弃。

信息经济的变革还有其他诸多层面的影响:

\begin{itemize}
    \item 施加了过高成本的地方性法规,将会转变得更加立足于市场。
    \item 高附加值的商业活动,原则上可以在任何地方开展;这会使各管辖区之间展开激烈的竞争。更好的落脚点,总会是下一个。
    \item 商业关系的开展,会更依赖于“信任圈”。有了加密技术,个人有能力在不被发现的情况下进行偷窃,所以,诚实会成为商业伙伴倍加重视的个人品质。
    \item 由于某些信息更加容易获得,专利和版权制度会发生革新。
    \item 保护会越来越倚重于技术,而不是司法。下层阶级会被隔离在围墙之外,封闭式社区的兴起将不可避免。在中央权威薄弱的时代,把麻烦制造者挡在墙外,是一种传统的也是有效的保护方式,可以尽量降低暴力犯罪的威胁。
    \item 大宗商品会像中世纪那样,被课以重税,并在当地内运输;而奢侈品的税收则很轻,并可以运输到很远的地方。
    \item 警察的职能将越来越多地由私人安保团队承担,后者往往会与商人协会建立联系。
    \item 私营公司比上市公司更具转型优势,因为在逃避政府强制征收方面,私营公司享有更大的回旋余地。
    \item “工作”会越来越多地成为任务或“计件工作”,而不再是组织内一个职位;相应地,终身雇佣的现象将消失。
    \item 经济资源的控制权,会从国家转移到能力超群的人手里;因为在产品种注入知识,会使创造财富越来越容易。
    \item 许多高学识职业的从业者,将被交互式信息检索系统所代替。
    \item 智力较低的人会慢慢发展出新的生存策略,包括更加集中发展休闲技能、体育技能和犯罪技能,以及为越来越多的主权个人进行服务,因为各管辖区内部的收入不平等会愈加严重。
\end{itemize}

在暴力回报率不断上升时期成长起来的政治体系,必将经历痛苦的调整。在今天,对于一个系统,相比它所掌握的权力的规模,效率越来越重要。小规模、高效率的主权国家,能以较低的成本为客户提供更多的保护,将更具有可持续性。

和中世纪时期一样,暴力组织中的经济不规模现象,正再次加剧。在共产主义垮台后,主权实体的数量不断增加,已经证明了这一台。我们预计,随着信息时代的逻辑被经验所证实,世界上的主权数量会暴涨。

权力将再次回归小规模运行。飞地和省区会发现,在为“客户”提供主权服务方面,它们甚至比那些跨越大洲的国家更具竞争优势。这种情形与已经时日不多的现代时期截然不同。在现代时期,任何实体,除非它掌握了足以控制一个王国的军事力量,否则绝不可能生存。在过去,当权力的形式存在规模不经济时,那些从保护中获益最多的人,如中世纪晚期城邦中的富商,确实控制了政府。在我们看来,你可以再次找到这样的形式。随着掠夺性税负的降低,加上资源的高效配置,确实由客户控制地方主权的地区数量,应该会迅速增加。

接下来我们将会探讨,面对失败团体的反对,这些变革与发展,能否或是否应该进行下去。这将是信息时代最大的争议之一。