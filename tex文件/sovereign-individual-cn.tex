\documentclass[11pt,a4paper]{ctexbook}%
% 注意宏包顺序,有可能会报错
% 采用LuaLaTeX编译
\usepackage{ctex}% 中文支持
\setCJKmainfont{FZSSK.TTF}[BoldFont={FZXBSK.TTF},ItalicFont={FZKTK.TTF}]
\setCJKsansfont{FZHTK.TTF}
\setCJKmonofont{FZFSK.TTF}
\usepackage{graphics}% 图形支持
\usepackage{parallel}% 提供双栏排版支持
\usepackage[english]{babel}% 载入美式英语断字模板
\usepackage{fancyhdr}
\usepackage{hyperref}
\usepackage{geometry}

\hypersetup{hidelinks}
%\geometry{left = 1.5cm,right = 1.5cm,top = 2.54cm,bottom = 2.54cm}%

\setlength{\parindent}{2em}% 缩进为两个字符宽度
\setmainfont{Book Antiqua}% 设置全局英文字体
\setcounter{tocdepth}{1}  %设置目录深度
\setcounter{secnumdepth}{2} %设置标题深度

%---------------------用另外一种分栏工具测试下,可以兼容列表和脚注。
\usepackage{paracol}
\columnratio{0.35}           % 栏宽比例
\setlength{\columnsep}{2em}  % 栏间距

\ctexset { %定制章节样式
chapter = {
beforeskip = 0pt,
fixskip = true,
format = \Huge\bfseries,
nameformat = \rule{\linewidth}{1bp}\par\bigskip\hfill\chapternamebox,
number = \arabic{chapter},
aftername = \par\medskip,
aftertitle = \par\bigskip\nointerlineskip\rule{\linewidth}{2bp}\par
}
}
\newcommand\chapternamebox[1]{%
\parbox{\ccwd}{\linespread{1}\selectfont\centering #1}}


% ------------------ 开始 -------------------

%
\begin{document}

\title{\Huge{\textbf{主权个人}}\\ \large{掌握信息时代的变革}}
\author{James Dale Davidson
	\and Lord William Rees-Mogg}
\date{1997年初版,2020年再版}


% ------------------ 正文 -------------------

\maketitle
\thispagestyle{empty} 	%%封面页不要编页码

\include{preface} %序言部分 preface.tex
\pagestyle{empty}
\begin{center}
    \Huge\textbf{译者推荐}
\end{center}

今天,我们生活在一个民族国家的世界里,联合国承认195个“国家”,地球上除南极洲以外的几乎所有陆地领土都属于其中一个国家。我们认为这很正常。但实际上,我们现在的时代是一个历史上的异类:在过去2000年的大部分时间里,世界上的大部分领土都是完全不受统治的。为什么今天不一样了?因为在过去500年的大部分时间里,现有的技术创造了一套激励机制,让民族国家是有意义的。用《主权个人》的作者詹姆斯·戴尔·戴维森\footnote{James Dale Davidson 毕业于牛津大学,获得了本科学位,是美国作家和私人投资者,专门研究经济和金融领域。目前担任班揚山谷出版社战略投资部门的联合编辑,他的职业生涯中,探讨政府权力过度已经占据了他大部分的时间。同时,他以其经济学与金融预测的能力而闻名,据称他预测了近三十年以来的每一次重大金融事件。}的话来说,暴力的逻辑倾向于民族国家,政府通过征服尽可能多的领土会获得很多好处,而且他们这么做很容易,他们确实这么做了。

但事实并非总是如此。一个典型的例子是,在“黑暗时代”欧洲在地理上极为分散,很少有中央政府掌权。这部分是因为盔甲的发明。如果你有一套盔甲,一把剑和一匹马,那时候没有任何武器能真正阻止你。这意味着大国家很难形成,因为国家的运作依赖于隐性的暴力威胁。但当火药在欧洲盛行时,暴力的逻辑被颠覆了。在一个有枪的世界里,集结大量军队和征服大量领土要便宜得多。詹姆斯·戴尔·戴维森认为,\textbf{互联网是另一项改变世界的发明,最终它会让民族国家过时,这是因为互联网能在本质上能使资本流动。}

民族国家依赖富人的掠夺性税收(远远超过政府提供的商品和服务价值的税收),以支持大型军队和福利项目。 但是戴维森说,互联网使掠夺性税收变得不可能,因为当资本流动时,人们可以选择生活在世界上的任何地方。他们不会被自己的工作、拥有的土地或工厂的位置所束缚。他们可以去待遇最好的地方。

所以戴维森说,在未来的一百年左右,民族国家将开始崩溃,因为他们根本没有足够的钱来维持今天的运作方式。相反,他想象的是一个由小城邦组成的世界,类似于19世纪前的意大利,或今天的新加坡。我不知道这一切是否会成真,或者何时会成真,但我们已经看到了一些趋势。疫情证明大规模远程工作是可行的,这意味着工人们可以不再被自己的位置所束缚。许多城市和国家正在为远程工作者提供激励措施,让他们回到自己的境内生活。 此外,越来越多的人以高税收为由放弃了他们的美国公民身份,许多西方商人现在住在避税天堂。纳西姆·塔勒布让我相信未来的世界可能会有不确定性,而《主权个人》让我相信,未来的世界会有所不同。在未来的几十年里,像美国这样的大国的影响力可能会下降,而像葡萄牙、新加坡和土耳其这样的小国的影响力会变得更重要。人们可能会集体逃离西方,特别是如果目前这种低自由度的趋势继续下去的话。

因此,指望过一种“传统”的生活(上大学、在大公司找工作、努力工作、在公司里步步高升)可能没有意义。相反,现在的年轻人应该尽可能多地学习,培养一套灵活的技能,这样他们才可以重新适应未来的任何情况。

\vspace{1em}

\begin{flushright}
    \begin{tabular}{c}
        2023年5月\\
        \url{https://macin.org}
    \end{tabular}
\end{flushright} %译者推荐 mycomment.tex


\renewcommand\contentsname{目录} %为了让中文显示“目录”
\tableofcontents %%目录页
\pagestyle{empty} %%页眉页脚为空

\newpage
\mainmatter
\pagestyle{headings} %设置当前及后续页面风格
\include{chap1}  %导入第1章内容
\include{chap2}  %导入第2章内容

\end{document}