\documentclass[fontset=fandol,11pt,a4paper]{ctexbook}%
%
% 注意宏包顺序,有可能会报错
% 采用LuaLaTeX编译
\usepackage{ctex}% 中文支持

\usepackage{graphics}% 图形支持
\usepackage{parallel}% 提供双栏排版支持
\usepackage[english]{babel}% 载入美式英语断字模板
\usepackage{fancyhdr}
\usepackage{hyperref}


\hypersetup{hidelinks}
\geometry
{
  a4paper,%% 设置为A4纸,并设置边距
  left = 1.5cm,%
  right = 1.5cm,%
  top = 2.54cm,%
  bottom = 2.54cm%
}%

\setlength{\parindent}{2em}% 缩进为两个字符宽度
\setmainfont{Book Antiqua}% 设置全局英文字体


\setcounter{tocdepth}{1}  %设置目录深度
\setcounter{secnumdepth}{2} %设置标题深度


\ctexset { %定制章节样式
chapter = {
beforeskip = 0pt,
fixskip = true,
format = \Huge\bfseries,
nameformat = \rule{\linewidth}{1bp}\par\bigskip\hfill\chapternamebox,
number = \arabic{chapter},
aftername = \par\medskip,
aftertitle = \par\bigskip\nointerlineskip\rule{\linewidth}{2bp}\par
}
}
\newcommand\chapternamebox[1]{%
\parbox{\ccwd}{\linespread{1}\selectfont\centering #1}}


% ------------------ 开始 -------------------

%
\begin{document}

\title{\Huge{\textbf{主权个人}}\\ \large{掌握信息时代的变革}}
\author{James Dale Davidson
	\and Lord William Rees-Mogg}
\date{1997年初版,2020年再版}


% ------------------ 正文 -------------------

\maketitle
\thispagestyle{empty} 	%%封面页不要编页码

\begin{center}
    \Huge\textbf{序言}
\end{center}

中世纪的人对意志没有信心,认为人类是容易受伤的、极其脆弱的,但他们尊重智力。他们认为只要认真思考,即使是人,也有能力回答关于上帝和宇宙的最深奥的问题。现代人崇拜意志,但他们对智力感到绝望。乌合之众,随机粒子的偏转,无意识偏见的影响:所有这些当代的陈词滥调,都在谈论智力的弱点,或者说也在谈论我们自己。

威廉·里斯·莫格勋爵和詹姆斯·戴尔·戴维森,并没有承诺也没有给出任何关于上帝和宇宙的答案。但是,他们对“大政治”的研究,对历史上各种力量的剖析,以及对不久的将来的一系列预测,是非比寻常的,甚至是反文化的,因为他们运用人类的理性,去思索那些我们被教导为“机遇”或“命运”的事情。在《主权个人》首次出版近四分之一个世纪之后,回顾过去,最容易做、也是我们周围的文化最喜欢做的事,就是挑剔他们的错误,这也算是一种自我安慰:那么费心去思考未来有什么意义呢?

当然,有一些事情他们没有想到:首先就是中国的崛起。在共产党的领导下,21世纪的中国创造了自己的信息时代,具有明显的民族主义、种族同化和深刻的国家主义特征。这可能是该书出版以来最大的“大政治”现象。仅举一个关键的例子,共产中国已经粉碎了香港这个城邦(城市国家),而里斯·莫格和戴维森曾将香港描述为“一种心智模式,一种会在信息时代繁荣昌盛的管辖区模式”。

从某个角度看,这是作者的一个盲点。从另一个角度看,中国的政治局委员一定是《主权个人》的热心读者。在不断重温列宁斯大林主义的同时,也积极地展望信息时代,只有这种特有的长期的警惕意识,才使得党的领导人能在本书分析的趋势中获得胜利。

这些趋势在今天依然适用:赢家通吃的经济、管辖权的竞争、大规模生产的转移,以及国家间的战争可能会过时。中国的崛起,与其说是对里斯·莫格和戴维森的反驳,倒更像是对他们所描述的利害关系的剧烈提升。

事实上,未来大政治的巨大冲突才刚刚开始。在技术层面上,\textbf{这场冲突的两极是:人工智能和加密技术}。人工智能展现出一种前景,能够最终解决经济学家所说“计算问题”(计划经济的关键)。理论上,它使集中控制整个经济成为可能。CCP最喜欢的技术,就是人工智能,这绝不是巧合。强加密技术在另外一极,它带来的远景是一个去中心化和个性化的世界。如果说人工智能是共产主义的,那么加密技术就是自由主义的。

未来可能就落在这两极之间。而要知道,我们今天采取的行动,会决定日后全局性的结果。在2020年,阅读《主权个人》,有助你认真思考,自己的行动将塑造什么样的未来;这是一次不容浪费的学习机会。
\\

\rightline{\emph{彼得·蒂尔}}
\rightline{\emph{2020年1月6日,洛杉矶}}
 %序言部分 preface.tex
\pagestyle{empty}
\begin{center}
    \Huge\textbf{译者推荐}
\end{center}

今天,我们生活在一个民族国家的世界里,联合国承认195个“国家”,地球上除南极洲以外的几乎所有陆地领土都属于其中一个国家。我们认为这很正常。但实际上,我们现在的时代是一个历史上的异类:在过去2000年的大部分时间里,世界上的大部分领土都是完全不受统治的。为什么今天不一样了?因为在过去500年的大部分时间里,现有的技术创造了一套激励机制,让民族国家是有意义的。用《主权个人》的作者詹姆斯·戴尔·戴维森\footnote{James Dale Davidson 毕业于牛津大学,获得了本科学位,是美国作家和私人投资者,专门研究经济和金融领域。目前担任班揚山谷出版社战略投资部门的联合编辑,他的职业生涯中,探讨政府权力过度已经占据了他大部分的时间。同时,他以其经济学与金融预测的能力而闻名,据称他预测了近三十年以来的每一次重大金融事件。}的话来说,暴力的逻辑倾向于民族国家,政府通过征服尽可能多的领土会获得很多好处,而且他们这么做很容易,他们确实这么做了。

但事实并非总是如此。一个典型的例子是,在“黑暗时代”欧洲在地理上极为分散,很少有中央政府掌权。这部分是因为盔甲的发明。如果你有一套盔甲,一把剑和一匹马,那时候没有任何武器能真正阻止你。这意味着大国家很难形成,因为国家的运作依赖于隐性的暴力威胁。但当火药在欧洲盛行时,暴力的逻辑被颠覆了。在一个有枪的世界里,集结大量军队和征服大量领土要便宜得多。詹姆斯·戴尔·戴维森认为,\textbf{互联网是另一项改变世界的发明,最终它会让民族国家过时,这是因为互联网能在本质上能使资本流动。}

民族国家依赖富人的掠夺性税收(远远超过政府提供的商品和服务价值的税收),以支持大型军队和福利项目。 但是戴维森说,互联网使掠夺性税收变得不可能,因为当资本流动时,人们可以选择生活在世界上的任何地方。他们不会被自己的工作、拥有的土地或工厂的位置所束缚。他们可以去待遇最好的地方。

所以戴维森说,在未来的一百年左右,民族国家将开始崩溃,因为他们根本没有足够的钱来维持今天的运作方式。相反,他想象的是一个由小城邦组成的世界,类似于19世纪前的意大利,或今天的新加坡。我不知道这一切是否会成真,或者何时会成真,但我们已经看到了一些趋势。疫情证明大规模远程工作是可行的,这意味着工人们可以不再被自己的位置所束缚。许多城市和国家正在为远程工作者提供激励措施,让他们回到自己的境内生活。 此外,越来越多的人以高税收为由放弃了他们的美国公民身份,许多西方商人现在住在避税天堂。纳西姆·塔勒布让我相信未来的世界可能会有不确定性,而《主权个人》让我相信,未来的世界会有所不同。在未来的几十年里,像美国这样的大国的影响力可能会下降,而像葡萄牙、新加坡和土耳其这样的小国的影响力会变得更重要。人们可能会集体逃离西方,特别是如果目前这种低自由度的趋势继续下去的话。

因此,指望过一种“传统”的生活(上大学、在大公司找工作、努力工作、在公司里步步高升)可能没有意义。相反,现在的年轻人应该尽可能多地学习,培养一套灵活的技能,这样他们才可以重新适应未来的任何情况。

\vspace{1em}

\begin{flushright}
    \begin{tabular}{c}
        2023年5月\\
        \url{https://macin.org}
    \end{tabular}
\end{flushright} %译者推荐 mycomment.tex


\renewcommand\contentsname{目录} %为了让中文显示“目录”
\tableofcontents %%目录页
\pagestyle{empty} %%页眉页脚为空

\newpage
\mainmatter
\pagestyle{headings} %设置当前及后续页面风格
\chapter[人类社会第四个阶段]{2000年转折点:\\人类社会第四个阶段}
% 选项为页眉页脚的简写

\section{预言}

\begin{paracol}{2}[]
在耶稣降临后的第一个千年之交,世界并未像传说的那样毁灭。其后,在过去的一千年里,公元2000年的到来一直困扰着西方人的想象。神学家、传教士、诗人和预言家,都在张望着本世纪最后十年的结束,期待着历史性事件的发生。甚至如艾萨克·牛顿这样的权威人士都猜测世界将在公元2000年结束。米歇尔·德·诺斯特拉达姆斯的预言自1568年发表以来,每一代人都会阅读,他预言第三位敌基督将在1999年7月到来。瑞士心理学家卡尔·荣格是“集体无意识”的鉴赏家,他预见到新时代将在1997年诞生。这些预测可以轻易被嘲笑。那么像德意志银行证券(Deutsche Bank Securities)的爱德华·牙尔德尼博士这样的经济学家所做的理智预测也会被嘲笑,他认为公元2000年的午夜计算机故障将“打乱整个全球经济格局”。但无论你是否把Y2K计算机问题视为计算机程序员和信息技术顾问们搞起的无稽之谈,还是技术与预言想象的神秘示例之一,都无法否认千禧年前夕的情况比通常关于世界何去何从的病态怀疑更加令人兴奋。
  
\switchcolumn
The coming of the year 2000 has haunted the Western imagination for the past thousand years. Ever since the world failed to end at the turn of the first millennium after Christ, theologians, evangelists, poets, seers, and now, even computer programmers have looked to the end of this decade with an expectation that it would bring something momentous. No less an authority than Isaac Newton speculated that the world would end with the year 2000. Michel de Nostradamus, whose prophecies have been read by every generation since they were first published in 1568, forecast the coming of the Third Antichrist in July 1999. Swiss psychologist Carl Jung, connoisseur of the “collective unconscious,” envisioned the birth of a New Age in 1997. Such forecasts may easily be ridiculed. And so can the sober forecasts of economists, such as Dr. Edward Yardeni of Deutsche Bank Securities, who expects computer malfunctions on the millennial midnight to “disrupt the entire global economy.” But whether you view the Y2K computer problem as groundless hysteria ginned up by computer programmers and Information Technology consultants to stir up business, or as a mysterious instance of technology unfolding in concert with the prophetic imagination, there is no denying that circumstances at the eve of the millennium excite more than the usual morbid doubt about where the world is tending. 

\switchcolumn*
在过去的250年里,一种对未来的不安,给西方社会特有的乐观主义染上了阴影。各地的人们都犹豫不决,忧心忡忡。你可以从他们的脸上看到,从他们的谈话中听到;它反映在民意调查中,登记在选票箱中。就像在乌云密布、闪电到来之前,大气中看不见的离子的物理变化,已经预示了雷雨即将降临。如今,在千禧年的黄昏,空气中弥漫着变革的预感。一个又一个人,在一种行将结束的生活方式下,感受到时间就要燃尽。随着最后十年的过去,一个肃杀的世纪,同时也是人类成就辉煌的一千年,就此告以终章。所有的一切,都将因2000年的到来而画上句号。

\switchcolumn
A sense of disquiet about the future has begun to color the optimism so characteristic of Western societies for the past 250 years. People everywhere are hesitant and worried. You see it in their faces. Hear it in their conversation. See it reflected in polls and registered in the ballot box. Just as an invisible, physical change of ions in the atmosphere signals that a thunderstorm is imminent even before the clouds darken and lightning strikes, so now, in the twilight of the millennium, premonitions of change are in the air. One person after another, each in his own way, senses that time is running out on a dying way of life. As the decade expires, a murderous century expires with it, and also a glorious millennium of human accomplishment. All draw to a close with the year 2000. 

\switchcolumn*
我们相信,西方文明的现代阶段将以此结束。这本书讲述了为什么。像许多早期的著作一样,它试图暗示未来的模糊形状和尺寸。在这个意义上,我们的工作是启示性的,这是该词的原始含义。Apokalypsis 在希腊语中的意思是“揭示”。我们相信,历史的一个新阶段——信息时代——即将“揭示”。
\switchcolumn
We believe that the modern phase of Western civilization will end with it. This book tells why. Like many earlier works, it is all attempt to see into a glass darkly, to sketch out the vague shapes and dimensions of a future that is still to be. In that sense, we mean our work to be apocalyptic-in the original meaning of the word. Apokalypsis means “unveiling” in Greek. We believe that a new stage in history -- the Information Age -- is about to be “unveiled.” 
\end{paracol}

\section{人类社会的第四阶段}
\begin{paracol}{2}[]
本书的主旨是探讨一场新的权力革命,它将以20世纪民族国家的毁灭为代价,解放出个体。创新,以前所未有的方式改变了暴力的逻辑,并且正在革新未来的边界。如果我们的推断正确,你站在历史上最彻底的革命的门槛上。微处理将会以比大多数人想象的更快的速度颠覆和破坏民族国家,从而在这个过程中创造新的社会组织形式。这将远非一件容易的转变。
\switchcolumn
The theme of this book is the new revolution of power which is liberating individuals at the expense of the twentieth-century nation-state. Innovations that alter the logic of violence in unprecedented ways are transforming the boundaries within which the future must lie. If our deductions are correct, you stand at the threshold of the most sweeping revolution in history. Faster than all but a few now imagine, microprocessing will subvert and destroy the nation-state, creating new forms of social organization in the process. This will be far from an easy transformation.
\switchcolumn*
它将提出巨大的挑战,因为与过去相比,它将以令人难以置信的速度发生。从人类历史的最初起源到现在,经济生活只有三个基本阶段:(1)狩猎采集社会;(2)农业社会;(3)工业社会。现在,正在地平线上出现一些全新的东西,即第四个社会组织阶段:信息社会。
\switchcolumn
The challenge it will pose will be all the greater because it will happen with incredible speed compared with anything seen in the past. Through all of human history from its earliest beginnings until now, there have been only three basic stages of economic life: (1) hunting-and-gathering societies; (2) agricultural societies; and (3) industrial societies. Now, looming over the horizon, is something entirely new, the fourth stage of social organization: information societies.
\switchcolumn*
之前的每个社会阶段都与暴力的演化和控制的截然不同的阶段相对应。如我们详细解释的那样,信息社会有望大大降低暴力带来的回报,部分原因在于它们超越了地域。作家威廉·吉布森所说的虚拟现实是一种“共识性幻觉”,将远远超出恶棍们的掌控范围,就像想象力一样。在新千年,大规模控制暴力的优势将比法国大革命之前任何时候都低得多。这将产生深远的影响。其中之一将是犯罪率上升。当大规模组织暴力的回报下降时,小规模暴力的回报很可能会上升。暴力将变得更加随机和本地化。有组织的犯罪将规模扩大。我们解释了为什么会这样。
\switchcolumn
Each of the previous stages of society has corresponded with distinctly different phases in the evolution and control of violence. As we explain in detail, information societies promise to dramatically reduce the returns to violence, in part because they transcend locality. The virtual reality of cyberspace, what novelist William Gibson characterized as a “consensual hallucination,” will be as far beyond the reach of bullies as imagination can take it. In the new millennium, the advantage of controlling violence on a large scale will be far lower than it has been at any time since before the French Revolution. This will have profound consequences. One of these will be rising crime. When the payoff for organizing violence at a large scale tumbles, the payoff from violence at a smaller scale is likely to jump. Violence will become more random and localized. Organized crime will grow in scope. We explain why.
\switchcolumn*
暴力回报逐渐减少的另一个逻辑含义是政治的日渐式微,而正是政治成了最大罪犯活动的舞台。有很多证据显示,20世纪民族国家的公民信仰已经迅速削弱了。共产主义的垮台只是其中最为显著的例子而已。正如我们将详细探讨的那样,西方政府领导层道德崩溃、腐败蔓延的现象并非偶然事件,而是民族国家潜能已经枯竭的证明。即使是其中的领袖们也不再相信他们所说的那些陈词滥调,更别说其他人信了。
\switchcolumn
Another logical implication of falling returns to violence is the eclipse of politics, which is the stage for crime on the largest scale. There is much evidence that adherence to the civic myths of the twentieth-century nation-state is rapidly eroding. The death of Communism is merely the most striking example. As we explore in detail, the collapse of morality and growing corruption among leaders of Western governments are not random developments. They are evidence that the potential of the nation-state is exhausted. Even many of its leaders no longer believe the platitudes they mouth. Nor are they believed by others.
\end{paracol}

\subsection{历史将重演} 
\begin{paracol}{2}[]
这是一个与过去有惊人相似之处的情况。每当技术变革将旧形态与新经济动力分离时,道德标准会转变,人们开始对那些掌控旧机构的人产生越来越大的鄙视。在人们形成新的一致变革意识形态之前,这种普遍的反感往往会显露出来。这就像在15世纪晚期,当时的中世纪教会是封建体制中的主要机构一样。尽管人们普遍相信“神职职务的神圣性”,但无论是高层还是低层神职人员都备受蔑视,与今天人们对政治家和官僚的普遍态度不相上下。
\switchcolumn
This is a situation with striking parallels in the past. Whenever technological change has divorced the old forms from the new moving forces of the economy, moral standards shift, and people begin to treat those in command of the old institutions with growing disdain. This widespread revulsion often comes into evidence well before people develop a new coherent ideology of change. So it was in the late fifteenth century, when the medieval Church was the predominant institution of feudalism. Notwithstanding popular belief in “the sacredness of the sacerdotal office,” both the higher and lower ranks of clergy were held in the utmost contempt-not unlike the popular attitude toward politicians and bureaucrats today.
\switchcolumn*
我们认为,在15世纪末,当生活已经被有组织的宗教彻底浸透时,以及今天世界已经充满政治时,可以通过类比学习很多东西。在15世纪末,支持制度化宗教的成本已经达到了历史上的极限,就像今天支持政府的成本已经达到了一种丧失理智的极端。
\switchcolumn
We believe that much can be learned by analogy between the situation at the end of the fifteenth century, when life had become thoroughly saturated by organized religion, and the situation today, when the world has become saturated with politics. The costs of supporting institutionalized religion at the end of the fifteenth century had reached a historic extreme, much as the costs of supporting government have reached a senile extreme today.
\end{paracol}

\subsection{信息革命}
\begin{paracol}{2}[]
随着大型系统的崩溃而加速,系统性强制将逐渐减少,不再是塑造经济生活和收入分配的因素。效率将比权力的指令在社会制度组织中更加重要。这意味着,在信息时代,能够有效维护产权并提供司法管理,同时消耗较少资源的州和城市,将成为可行的主权国家,而在过去五个世纪中普遍不是如此。一个完全不受物理暴力威胁的经济活动领域将在网络空间中出现。最明显的好处将流向“认知精英”,他们将越来越在政治边界之外活动,并且在法兰克福、伦敦、纽约、布宜诺斯艾利斯、洛杉矶、东京和香港都感到自在。收入将在各个司法辖区内变得更加不平等,而在各个司法辖区之间则变得更加平等。
\switchcolumn
As the breakdown of large systems accelerates, systematic compulsion will recede as a factor shaping economic life and the distribution of income. Efficiency will become more important than the dictates of power in the organization of social institutions. This means that provinces and even cities that can effectively uphold property rights and provide for the administration of justice, while consuming few resources, will be viable sovereignties in the Information Age, as they generally have not been during the last five centuries. An entirely new realm of economic activity that is not hostage to physical violence will emerge in cyberspace. The most obvious benefits will flow to the “cognitive elite,” who will increasingly operate outside political boundaries. They are already equally at home in Frankfurt, London, New York, Buenos Aires, Los Angeles, Tokyo, and Hong Kong. Incomes will become more unequal within jurisdictions and more equal between them.
\switchcolumn*
《主权个人》探讨了这种革命性变化的社会和财务后果。我们的愿望是帮助您利用新时代的机遇,避免被其影响摧毁。如果我们所期望看到的一半成真,您将面临着几乎史无前例的巨大变革。
\switchcolumn
\emph{The Sovereign Individual} explores the social and financial consequences of this revolutionary change. Our desire is to help you to take advantage of the opportunities of the new age and avoid being destroyed by its impact. If only half of what we expect to see happens, you face change of a magnitude with few precedents in history.
\switchcolumn*
2000年的转型不仅将彻底改变世界经济的性质,而且速度将比以往任何时期的变革都更快。不像农业革命,信息革命不需要千年才能完成工作。不像工业革命,其影响不会持续几个世纪。信息革命将在一个人的寿命内发生。
\switchcolumn
The transformation of the year 2000 will not only revolutionize the character of the world economy, it will do so more rapidly than any previous phase change. Unlike the Agricultural Revolution, the Information Revolution will not take millennia to do its work. Unlike the Industrial Revolution, its impact will not be spread over centuries. The Information Revolution will happen within a lifetime. 
\switchcolumn*
此外,它几乎将同时发生在世界的各个角落。技术和经济创新将不再局限于全球小部分地区。这种转变将是普遍的。并且,它将涉及与过去断裂的程度如此深刻,以至于几乎会使类似古希腊人的早期农业人民所想象的神神秘领域“复活”。在信息社会成型之后,保护许多现代机构在新千年将会变得非常困难甚至不可能。当信息社会成形时,它们与工业社会的差异将会像Aeschylus的希腊世界与穴居人世界一样不同。
\switchcolumn
What is more, it will happen almost everywhere at once. Technical and economic innovations will no longer be confined to small portions of the globe. The transformation will be all but universal. And it will involve a break with the past so profound that it will almost bring to life the magical domain of the gods as imagined by the early agricultural peoples like the ancient Greeks. To a greater degree than most would now be willing to concede, it will prove difficult or impossible to preserve many contemporary institutions in the new millennium. When information societies take shape they will be as different from industrial societies as the Greece of Aeschylus was from the world of the cave dwellers.
\end{paracol}

\section{解放普罗米修斯:主权个体的崛起}
\begin{paracol}{2}[]
即将到来的转变既有好消息也有坏消息。好消息是信息革命将像从未有过的那样解放个人。首次,那些能够自我教育和激励自己的人将几乎完全自由地发明自己的工作,并实现自己生产力的全部好处。天才将被释放,摆脱了政府的压迫和种族和民族偏见的限制。在信息社会中,没有真正有才华的人会被他人没见过的肤浅见解所阻拦。你的种族、外貌、年龄、性取向或发型方式将无关紧要。在网络经济中,别人永远看不见你。在新的网络空间上,丑陋、肥胖、老年和残疾的人将与年轻和美丽的人平等竞争,实现完全无色觉偏见的匿名。
\switchcolumn
The coming transformation is both good news and bad. The good news is that the Information Revolution will liberate individuals as never before. For the first time, those who can educate and motivate themselves will be almost entirely free to invent their own work and realize the full benefits of their own productivity. Genius will be unleashed, freed from both the oppression of government and the drags of racial and ethnic prejudice. In the Information Society, no one who is truly able will be detained by the ill-formed opinions of others. It will not matter what most of the people on earth might think of your race, your looks, your age, your sexual proclivities, or the way you wear your hair. In the cybereconomy, they will never see you. The ugly, the fat, the old, the disabled will vie with the young and beautiful on equal terms in utterly color-blind anonymity on the new frontiers of cyberspace.
\end{paracol}

\subsection{思想成为财富}
\begin{paracol}{2}[]
无论何时何地,凡是有卓越思想的人都将得到前所未有的奖励。在一个最大的财富资源是你脑中的思想而不仅仅是物质资本的环境中,任何能够清晰思考的人都可能富有。信息时代将是流动性的年代。它将为数十亿生活在未曾共享工业社会繁荣的地区的人提供更多平等机会。这些人中最聪明、最成功和最有抱负的人将成为真正的独立个体。
\switchcolumn
Merit, wherever it arises, will be rewarded as never before. In an environment where the greatest source of wealth will be the ideas you have in your head rather than physical capital alone, anyone who thinks clearly will potentially be rich. The Information Age will be the age of upward mobility. It will afford far more equal opportunity for the billions of humans in parts of the world that never shared fully in the prosperity of industrial society. The brightest, most successful and ambitious of these will emerge as truly Sovereign Individuals. 
\switchcolumn*
一开始,只有少数人能实现完全的财务主权。但这并不否定财务独立的优势。并不是每个人都能获得同样巨大的财富,这并不意味着成为富有是毫无意义的。每个亿万富翁都有25000个百万富翁。如果你是百万富翁而不是亿万富翁,那也不代表你是穷人。同样,在未来,衡量你的财务成功的一个里程碑不仅仅是你的净值上有多少个零,而是你能否以一种可以实现个人完全自治和独立的方式来构建你的事务。越聪明的你,就越不需要推动力来实现财务逃逸速度。即使是非常普通的人也可以在全球政治重力对全球经济的影响减弱之际腾飞。在你或你的子孙一生中,无先例的财务独立将成为一个可以实现的目标。
\switchcolumn
At first, only a handful will achieve full financial sovereignty. But this does not negate the advantages of financial independence. The fact that not everyone attains an equally vast fortune does not mean that it is futile or meaningless to become rich. There are 25,000 millionaires for every billionaire. If you are a millionaire and not a billionaire, that does not make you poor. Equally, in the future, one of the milestones by which you measure your financial success will be not just now many zeroes you can add to your net worth, but whether you can structure your affairs in a way that enables you to realize full individual autonomy and independence. The more clever you are, the less propulsion you will require to achieve financial escape velocity. Persons of even quite modest means will soar as the gravitational pull of politics on the global economy weakens. Unprecedented financial independence will be a reachable goal in your lifetime or that of your children.
\switchcolumn*
在生产力的最高高原上,这些主权个体将在类似于希腊神话中神之间的关系的条件下竞争和互动。下一个千禧年的神山将是虚拟空间——一个没有实体存在的领域,但它发展成为新千年二十年代世界上最大的经济体之一。到2025年,虚拟经济将有很多百万参与者。其中一些人将像比尔·盖茨一样富有,价值数百亿美元。虚拟贫穷可能是年收入不到20万美元的人。没有虚拟福利。没有虚拟税收,也没有虚拟政府。虚拟经济可能成为未来30年最伟大的经济现象,而不是中国。
\switchcolumn
At the highest plateau of productivity, these Sovereign Individuals will compete and interact on terms that echo the relations among the gods in Greek myth. The elusive Mount Olympus of the next millennium will be in cyberspace -- a realm without physical existence that will nonetheless develop what promises to be the world's largest economy by the second decade of the new millennium. By 2025, the cybereconomy will have many millions of participants. Some of them will be as rich as Bill Gates, worth tens of billions of dollars each. The cyberpoor may be those with an income of less than \$200,000 a year. There will be no cyberwelfare. No cybertaxes and no cybergovernment. The cybereconomy, rather than China, could well be the greatest economic phenomenon of the next thirty years.
\switchcolumn*
好消息是政治家将不再能够在这个新领域中支配、压制和规范大部分商业活动,就像古希腊城邦的立法者不能修剪宙斯的胡须一样。这对富人来说是好消息。对于不那么富裕的人来说更是好消息。政治施加的障碍和负担对于成为富人来说是更多的障碍,而对于已经富足的人来说则更少。暴力收益递减和权力下放的益处将为每个有活力和雄心壮志的人创造发挥,从而从政治的消亡中受益。即使是政府服务的消费者也会受益,因为企业家会扩大竞争的好处。迄今为止,司法管辖区之间的竞争通常意味着通过暴力竞争来强制执行主导群体的规则。因此,许多跨领土竞争的独创性都集中在军事事业上。但是,网络经济的出现将为主权服务的供给带来新的竞争条件。司法管辖区的繁殖将意味着在新的执行合同和保障人身和财产安全的方式方面的多种多样的实验。全球经济的大部分解放出了政治控制,这将迫使我们所知道的政府在更接近市场原则的条件下运作。政府最终将别无选择,只能把他们服务的地区人口视为顾客,而不是像有组织的犯罪分子对待勒索诈骗受害者一样。
\switchcolumn
The good news is that politicians will no more be able to dominate, suppress, and regulate the greater part of commerce in this new realm than the legislators of the ancient Greek city-states could have trimmed the beard of Zeus. That is good news for the rich. And even better news for the not so rich. The obstacles and burdens that politics imposes are more obstacles to becoming rich than to being rich.    The benefits of declining returns to violence and devolving jurisdictions will create scope for every energetic and ambitious person to benefit from the death of politics. Even the consumers of government services will benefit as entrepreneurs extend the benefits of competition. Heretofore, competition between jurisdictions has usually meant competition by means of violence to enforce the rule of a predominant group. Consequently, much of the ingenuity of interjurisdictional competition was channeled into military endeavor. But the advent of the cybereconomy will bring competition on new terms to provision of sovereignty services. A proliferation of jurisdictions will mean proliferating experimentation in new ways of enforcing contracts and otherwise securing the safety of persons and property. The liberation of a large part of the global economy from political control will oblige whatever remains of government as we have known it to operate on more nearly market terms. Governments will ultimately have little choice but to treat populations in territories they serve more like customers, and less in the way that organized criminals treat the victims of a shakedown racket.
\end{paracol}

\begin{Parallel}{97mm}{45mm}

  
\subsection{超越政治}
  
  \ParallelLText
  {What mythology described as the province of the gods will become a viable option for the individual -- a life outside the reach of kings and councils. First in scores, then in hundreds, and ultimately in the millions, individuals will escape the shackles of politics. As they do, they will transform the character of governments, shrinking the realm of compulsion and widening the scope of private control over resources.}

  \ParallelRText
  {\small 神话描述的神的领域将成为个人的可行选项 - 生活在国王和议会无法触及的生活之外。从成百上千开始,最终达到数百万,个人将逃脱政治的枷锁。他们这样做,将改变政府的性质,缩小强制的范围,扩大对资源的私人控制范围。}

\ParallelPar 
\ParallelLText
  {The emergence of the sovereign individual will demonstrate yet again the strange prophetic power of myth. Conceiving little of the laws of nature, the early agricultural peoples imagined that “powers we should call supernatural” were widely distributed. These powers were sometimes employed by men, sometimes by “incarnate human gods” who looked like men and interacted with them in what Sir James George Frazer described in \emph{The Golden Bough} as “a great democracy.”  }
  
  \ParallelRText
  {\small 个人主权的出现将再次证明神话的奇异预言能力。早期的农业民族很少了解自然法则,他们认为“我们应该称之为超自然的力量”是广泛分布的。这些力量有时被人类利用,有时由“人格化的人类神”利用,他们看起来像人类,并与他们互动在詹姆斯·乔治·弗雷泽在《金枝》中所描述的“一个伟大的民主制”中。}

  \ParallelPar  
  \ParallelLText
  {When the ancients imagined the children of Zeus living among them they were inspired by a deep belief in magic. They shared with other primitive agricultural peoples an awe of nature, and a superstitious conviction that nature's works were set in motion by individual volition, by magic. In that sense, there was nothing self-consciously prophetic about their view of nature and their gods. They were far from anticipating microtechnology. They could not have imagined its impact in altering the marginal productivity of individuals thousands of years later. They certainly could not have foreseen how it would shift the balance between power and efficiency and thus revolutionize the way that assets are created and protected. Yet what they imagined as they spun their myths has a strange resonance with the world you are likely to see. }
  
  \ParallelRText
  {\small 当古人想象宙斯的子女与他们一起生活时,他们受到了对魔法的深刻信仰的启发。他们与其他原始农业民族分享对自然的敬畏,以及通过个体意志的魔法来控制自然力量的迷信信念。从这个意义上说,他们对自然和他们的神并没有什么自觉的预言性。他们远未预见到微技术的到来。数千年后,他们也无法想象它对个人边际生产力的改变会对生产资产和保护方式的变革带来多大的影响。然而,当他们编织神话时所想象的,却与你可能看到的世界有着奇怪的共鸣。 }
  \ParallelPar

\subsection{新的Abracadabra咒语}
 
  \ParallelLText
  {The “abracadabra” of the magic invocation, for example, bears a curious similarity to the password employed to access a computer. In some respects, high-speed computation has already made it possible to mimic the magic of the genie. Early generations of “digital servants” already obey the commands of those who control the computers in which they are sealed much as genies were sealed in magic lamps. The virtual reality of information technology will widen the realm of human wishes to make almost anything that can be imagined seem real. Telepresence will give living individuals the same capacity to span distance at supernatural speed and monitor events from afar that the Greeks supposed was enjoyed by Hermes and Apollo. The Sovereign Individuals of the Information Age, like the gods of ancient and primitive myths, will in due course enjoy a kind of “diplomatic immunity” from most of the political woes that have beset mortal human beings in most times and places. }  
  \ParallelRText
  {\small 魔法咒语中的“阿布拉卡达布拉”与访问电脑的密码惊人地相似。高速计算在某些方面已经让模仿神灵魔法成为可能。早期的“数字仆人”就像法器中被封印的神灵一样服从主人的命令。信息技术的虚拟现实将扩大人类的愿望范围,使几乎任何想象得到的事情都变得真实。远程存在将赋予生命体在超自然速度下跨越距离和远程监控事件的能力,就像希腊神话中的赫尔墨斯和阿波罗一样。信息时代的主权个体,像古代和原始神话中的神灵一样,最终将享有一种“外交豁免权”,使其免于大多数时代和地方困扰凡人的政治问题。}

  \ParallelPar
  \ParallelLText
  {The new Sovereign Individual will operate like the gods of myth in the same physical environment as the ordinary, subject citizen, but in a separate realm politically. Commanding vastly greater resources and beyond the reach of many forms of compulsion, the Sovereign Individual will redesign governments and reconfigure economies in the new millennium. The full implications of this change are all but unimaginable.}

  \ParallelRText
  {\small 新的主权个体将在同一物理环境中与普通公民生活,但在政治上处于单独的领域。拥有非常庞大的资源,超出多种形式约束的范围,主权个体将重新设计政府和经济,进入新的千年。这种变化的全部影响几乎无法想象。”}

  \ParallelPar

\subsection{天才与天惩}  

  \ParallelLText
  {For anyone who loves human aspiration and success, the Information Age will provide a bounty. That is surely the best news in many generations. But it is bad news as well. The new organization of society implied by the triumph of individual autonomy and the true equalization of opportunity based upon merit will lead to very great rewards for merit and great individual autonomy. This will leave individuals far more responsible for themselves than they have been accustomed to being during the industrial period. It will also precipitate transition crises, including a possibly severe economic depression that will reduce the unearned advantage in living standards that has been enjoyed by residents of advanced industrial societies throughout the twentieth century.  As we write, the top 15 percent of the world's population have an average per-capita income of \$21,000 annually. The remaining 85 percent of the world have an average income of just \$1,000. That huge, hoarded advantage from the past is bound to dissipate under the new conditions of the Information Age.  }
  
  \ParallelRText
  {\small 对于任何一个追逐理想和成功的人来说,信息时代的回报将无与伦比。这无疑是几代人以来最好的消息,但也是一个坏消息。基于个人自治的新型社会组织,以及建立在能力之上的、真正的机会均等,会使才能出众者,得到超级的回报和个人自主性。但是,个人要对自己担负的责任,也会远远超过他们在工业时期所习惯的。此外,在整个20世纪,先进工业社会的居民,享受了不劳而获的优越生活,这种优势也将被削弱。在我们写这本书的时候(1997年之前),世界上前15\%的人口,人均年收入为21000美元;其余85\%的人,平均年收入只有1000美元。在信息时代的新环境下,过去囤积起来的巨大优势,必将烟消云散。 }


  \ParallelPar  

  \ParallelLText
  {\small As it does, the capacity of nation-states to redistribute income on a large scale will collapse. Information technology facilitates dramatically increased competition between jurisdictions. When technology is mobile, and transactions occur in cyberspace, as they increasingly will do, governments will no longer be able to charge more for their services than they are worth to the people who pay for them. Anyone with a portable computer and a satellite link will be able to conduct almost any information business anywhere, and that includes almost the whole of the world's multitrillion-dollar financial transactions.  }
  
  \ParallelRText
  {\small 随着它的消散,民族国家大规模重新分配收入的能力将崩溃。信息技术极大地促进了辖区之间的竞争。信息技术加剧了各管辖区之间的竞争。技术是流动的,交易是在网络空间进行的。任何人只要有一台便携式电脑,和一条卫星网络,就可以在任何地方,从事几乎任何信息业务,包括世界上数以万亿美元的金融交易。 }


  \ParallelPar  
  \ParallelLText
  {This means that you will no longer be obliged to live in a high-tax jurisdiction in order to earn high income. In the future, when most wealth can be earned anywhere, and even spent anywhere, governments that attempt to charge too much as the price of domicile will merely drive away their best customers. If our reasoning is correct, and we believe it is, the nation-state as we know it will not endure in anything like its present form.  }
  
  \ParallelRText
  {\small 这意味着,你不再需要为了高收入,而不得不生活在高税率的国家和地区。在未来,大多数财富可以在任何地方赚取,甚至可以在任何地方消费。到那时,政府试图对它的永久居民收取高额的服务费,只会丢掉它们最好的客户。如果我们的推理是正确的,我们相信它是正确的,那么,大家所知道的民族国家,将不会再以任何类似现在的形式而存在。 }
  \ParallelPar

\section{国家的终结} 
  \ParallelLText
  {Changes that diminish the power of predominant institutions are both unsettling and dangerous. Just as monarchs, lords, popes, and potentates fought ruthlessly to preserve their accustomed privileges in the early stages of the modem period, so today's governments will employ violence, often of a covert and arbitrary kind, in the attempt to hold back the clock. Weakened by the challenge from technology, the state will treat increasingly autonomous individuals, its former citizens, with the same range of ruthlessness and diplomacy it has heretofore displayed in its dealing with other governments. The advent of this new stage in history was punctuated with a bang on August 20, 1998, when the United States fired about \$200 million worth of Tomahawk BGM-109 sea-launched cruise missiles at targets allegedly associated with an exiled Saudi millionaire, Osama bin Laden. Bin Laden became the first person in history to have his satellite phone targeted for attack by cruise missiles. Simultaneously, the United States destroyed a pharmaceutical plant in Khartoum, Sudan, in Bin Laden's honor. The emergence of Bin Laden as the enemy-in-chief of the United States reflects a momentous change in the nature of warfare. A single individual, albeit one with hundreds of millions of dollars, can now be depicted as a plausible threat to the greatest military power of the Industrial era. In statements reminiscent of propaganda employed during the Cold War about the Soviet Union, the United States president and his national security aides portrayed Bin Laden, a private individual, as a transnational terrorist and leading enemy of the United States. }
  
  \ParallelRText
  {\small 削弱了主导机构权力的变化既令人不安,又危险。正如君主、贵族、教皇和有权势的人在现代时期的早期阶段为了维护惯有特权而进行的残酷斗争一样,今天的政府也会用暴力,通常是隐蔽和任意的,试图阻止时钟倒转。受技术挑战削弱的国家将像以前对待其他政府一样,用同样的无情和外交手段处理日益自治的个人——它的前公民。这个历史新阶段的出现在1998年8月20日响起;当时,美国向据称与被流放的沙特亿万富翁奥萨马·本·拉登有关的目标发射了价值约2亿美元的海基巡航导弹。本·拉登成为历史上第一个被巡航导弹攻击卫星电话的人。与此同时,美国在苏丹的喀土穆摧毁了一家制药厂,以表彰本·拉登。本·拉登成为美国最大军事力量的可信威胁。单个人,虽然拥有数百万美元,现在也可以被描绘为对工业时代最大的军事力量构成可信威胁的人。美国总统和他的国家安全助手发表的声明,类似于冷战期间有关苏联的宣传,将本·拉登描述为跨国恐怖分子和美国的头号敌人。}


  \ParallelPar  

  \ParallelLText
  {The same military logic that has seen Osama bin Laden elevated to a position as the chief enemy of the United States will assert itself in governments' internal relations with their subjects. Increasingly harsh techniques of exaction will be a logical corollary of the emergence of a new type of bargaining between governments and individuals. Technology will make individuals more nearly sovereign than ever before. And they will be treated that way. Sometimes violently, as enemies, sometimes as equal parties in negotiation, sometimes as allies. But however ruthlessly governments behave, particularly in the transition period, wedding the IRS with the CIA will avail them little. They will be increasingly required by the press of necessity to bargain with autonomous individuals whose resources will no longer be so easily controlled.  }
  
  \ParallelRText
  {\small 相同的军事逻辑已将奥萨马·本·拉登提升为美国的首要敌人,这种逻辑也将在政府与国民的内部关系中得到体现。越来越严厉的敛财手段将成为政府与个人谈判出现的逻辑必然结果。技术将让个人比以往任何时候都更接近主权。他们也将被当作如此对待,有时会被视作敌人,有时会被视作平等的谈判方,有时会被视作盟友。然而,无论政府的行为多么无情,尤其是在过渡期间,将联邦税务局与中央情报局(CIA)捆绑在一起是没有什么用处的。由于自主个体的资源不再轻易被控制,政府将越来越需要与自主个体进行谈判来适应这种变化。}


  \ParallelPar  

  \ParallelLText
  {The changes implied by the Information Revolution will not only create a fiscal crisis for governments, they will tend to disintegrate all large structures. Fourteen empires have disappeared already in the twentieth century. The breakdown of empires is part of a process that will dissolve the nation-state itself. Government will have to adapt to the growing autonomy of the individual. Taxing capacity will plunge by 50-70 percent. This will tend to make smaller jurisdictions more successful. The challenge of setting competitive terms to attract able individuals and their capital will be more easily undertaken in enclaves than across continents.}
  
  \ParallelRText
  {\small 信息革命所带来的变化不仅会为政府创造财政危机,还将倾向于分解所有的大型结构。20世纪已经有14个帝国消失了。帝国的崩溃是一个过程的一部分,该过程将消解民族国家本身。政府将不得不适应个人日益增长的自治。税收收入能力将下降50-70\%。这将倾向于使较小的司法管辖区更为成功。面对吸引有才华的个体和他们的资本的竞争性条款的挑战,将更容易在飞地中而不是跨越大陆进行。 }


  \ParallelPar  

  \ParallelLText
  {We believe that as the modern nation-state decomposes, latter-day barbarians will increasingly come to exercise power behind the scenes. Groups like the Russian mafiya, which picks the bones of the former Soviet Union, other ethnic criminal gangs, nomenklaturas, drug lords, and renegade covert agencies will be laws unto themselves. They already are. Far more than is widely understood, the modern barbarians have already infiltrated the forms of the nation-state without greatly changing its appearances. They are microparasites feeding on a dying system. As violent and unscrupulous as a state at war, these groups employ the techniques of the state on a smaller scale. Their growing influence and power are part of the downsizing of politics. Microprocessing reduces the size that groups must attain in order to be effective in the use and control of violence. As this technological revolution unfolds, predatory violence will be organized more and more outside of central control. Efforts to contain violence will also devolve in ways that depend more upon efficiency than magnitude of power. }
  
  \ParallelRText
  {\small 我们相信,随着现代民族国家的解体,后期野蛮人将越来越多地在幕后行使权力。像俄罗斯黑手党、在前苏联领土上乱捡残羹剩饭的其他族裔犯罪团伙、官僚特权阶层、毒品贩子和叛逆的秘密机构这样的团体将成为自己的法律。他们已经是了。现代野蛮人已经在不大改变国家的形象的情况下,渗透到国家形式之中,远比人们所理解的要多得多。他们是在死亡的系统上寄生的微小寄生虫。这些团体和处于战争状态下的政治机构一样具有暴力和不择手段,他们运用国家的技术进行小规模的实施和控制。他们日益增长的影响力和权力是政治下降的一部分。微处理降低了团体必须达到的规模才能在使用和控制暴力方面发挥有效作用的规模。随着这一技术革命的展开,掠夺性暴力将越来越多地组织在中央控制之外。遏制暴力的努力也将以效率而不是权力大小的方式演化。 }
  \ParallelPar

\subsection{倒退的历史}
 
  \ParallelLText
  {The process by which the nation-state grew over the past five centuries will be put into reverse by the new logic of the Information Age. Local centers of power will reassert themselves as the state devolves into fragmented, overlapping sovereignties. The growing power of organized crime is merely one reflection of this tendency. Multinational companies are already having to subcontract all but essential work. Some conglomerates, such as AT\&T, Unisys, and ITT, have split themselves into several firms in order to function more profitably. The nation-state will devolve like an unwieldy conglomerate, but probably not before it is forced to do so by financial crises.   }
  
  \ParallelRText
  {\small 过去五个世纪民族国家成长的过程将被信息时代的新逻辑逆转。当国家分化成交叉重叠的主权时,地方权力中心将重新确立自己的地位。有组织犯罪的不断壮大仅仅是这种趋势的一个反映。跨国公司已经不得不外包除了必要的工作之外的所有工作。一些企业集团,如AT\&T、Unisys和ITT,已经分裂成几个公司,以更赚钱的方式运作。民族国家将像一个难以管理的企业集团一样分化,但可能不会在金融危机迫使它这样做之前。}


  \ParallelPar  

  \ParallelLText
  {Not only is power in the world changing, but the work of the world is changing as well. This means that the way business operates will inevitably change. The “virtual corporation” is evidence of a sweeping transformation in the nature of the firm, facilitated by the drop in information and transaction costs. We explore the implications of the Information Revolution for dissolving corporations and doing away with the “good job.” In the Information Age, a “job” will be a task to do, not a position you “have.” Microprocessing has created entirely new horizons of economic activity that transcend territorial boundaries. This transcendence of frontiers and territories is perhaps the most revolutionary development since Adam and Eve straggled out of paradise under the sentence of their Maker: “In the sweat of thy face shalt thou eat bread.” As technology revolutionizes the tools we use, it also antiquates our laws, reshapes our morals, and alters our perceptions. This book explains how.   }
  
  \ParallelRText
  {\small 不仅世界权力在改变,世界工作也在改变。这意味着商业运作的方式不可避免地会发生变化。“虚拟企业”是企业性质发生巨变的证据,这一变化得益于信息和交易成本的降低。我们探讨了信息革命对于溶解公司和消除“好工作”的影响。在信息时代,“工作”将是一项任务,而不是一个你“拥有”的职位。微处理技术已经创造了超越领土界限的全新经济活动领域。这种超越国界和领土的能力或许是自亚当和夏娃受造之后最具革命性的进展:“你必须流着汗水才能吃到面包。”随着技术革新所带来的工具倒退了我们的法律,重塑了我们的道德,改变了我们的感知。本书解释了这一点。}


  \ParallelPar  

  \ParallelLText
  {Microprocessing and rapidly improving communications already make it possible for the individual to choose where to work. Transactions on the Internet or the World Wide Web can be encrypted and will soon be almost impossible for tax collectors to capture. Tax-free money already compounds far faster offshore than onshore funds still subject to the high tax burden imposed by the twentieth-century nation-state. After the tum of the millennium, much of the world's commerce will migrate into the new realm of cyberspace, a region where governments will have no more dominion than they exercise over the bottom of the sea or the outer planets. In cyberspace, the threats of physical violence that have been the alpha and omega of politics since time immemorial will vanish. In cyberspace, the meek and the mighty will meet on equal terms. Cyberspace is the ultimate offshore jurisdiction. An economy with no taxes. Bermuda in the sky with diamonds.  }
  
  \ParallelRText
  {\small 微处理技术和迅速改进的通信技术已经使个人有选择工作地点的可能性。在互联网或万维网上进行的交易可以进行加密,并且很快就几乎不可能被税务部门发现。免税的资金已经在海外比在岸上完全受到二十世纪民族国家高税负重担的资金更快地复利增长。在千禧年之后,世界上的大部分商业将迁移到新的网络空间,这是一个政府不再能够支配的领域,就像他们对待海底和外层行星一样。在网络空间,一直是政治的尧舜大禹的物理暴力威胁将消失。在网络空间,弱者和强者将在平等的条件下相遇。网络空间是终极的离岸司法管辖区。一个没有税收的经济。天空中的百慕大岛和钻石。}


  \ParallelPar  

  \ParallelLText
  {When this greatest tax haven of them all is fully open for business, all funds will essentially be offshore funds at the discretion of their owner. This will have cascading consequences. The state has grown used to treating its taxpayers as a farmer treats his cows, keeping them in a field to be milked. Soon, the cows will have wings.  }
  
  \ParallelRText
  {\small 当这个最大的避税天堂完全开放营业时,所有资金实际上将成为业主自行决定的离岸基金。这将产生连锁反应。国家已经习惯了像农民对待奶牛一样对待纳税人,将他们留在田地里挤奶。很快,奶牛将有翅膀。}
  \ParallelPar

\subsection{国家的复仇}
 
  \ParallelLText
  { Like an angry farmer, the state will no doubt take desperate measures at first to tether and hobble its escaping herd. It will employ covert and even violent means to restrict access to liberating technologies. Such expedients will work only temporarily, if at all. The twentieth-century nation-state, with all its pretensions, will starve to death as its tax revenues decline.  }
  
  \ParallelRText
  {\small 像一位愤怒的农民一样,国家无疑会首先采取绝望的措施来束缚和牵制其逃离的群体。它将采用隐秘甚至暴力手段来限制对解放性技术的接触。这些方法只能在短时间内起作用,如果有的话。二十世纪的民族国家,带着所有的自负,将因为税收下降而面临资金不足的困境。 }


  \ParallelPar  

  \ParallelLText
  { When the state finds itself unable to meet its committed expenditure by raising tax revenues, it will resort to other, more desperate measures. Among them is printing money. Governments have grown used to enjoying a monopoly over currency that they could depreciate at will. This arbitrary inflation has been a prominent feature of the monetary policy of all twentieth-century states. Even the best national currency of the postwar period, the German mark, lost 71 percent of its value from January 1, 1949, through the end of June 1995. In the same period, the U.S. dollar lost 84 percent of its value. This inflation had the same effect as a tax on all who hold the currency. As we explore later, inflation as revenue option will be largely foreclosed by the emergence of cybermoney. New technologies will. allow the holders of wealth to bypass the national monopolies that have issued and regulated money in the modern period. Indeed, the credit crises that swept through Asia, Russia, and other emerging economies in 1997 and 1998 attest to the fact that national currencies and national credit ratings are anachronisms inimical to the smooth operation of the global economy. It is precisely the fact that the demands of sovereignty require all transactions within a jurisdiction to be denominated in a national currency that creates the vulnerability to mistakes by central bankers and attacks by speculators which precipitated deflationary crises in one jurisdiction after another. In the Information Age, individuals will be able to use cybercurrencies and thus declare their monetary independence. When individuals can conduct their own monetary policies over the World Wide Web it will matter less or not at all that the state continues to control the industrial-era printing presses. Their importance for controlling the world's wealth will be transcended by mathematical algorithms that have no physical existence. In the new millennium, cybermoney controlled by private markets will supersede fiat money issued by governments. Only the poor will be victims of inflation and ensuing collapses into deflation that are consequences of the artificial leverage which fiat money injects into the economy. }
  
  \ParallelRText
  {\small 当国家发现自己无法通过增加税收来满足支出时,就会采取其他更绝望的措施。其中之一是印钞票。政府已经习惯了享有货币垄断权,可以随意贬值,这种任意通胀一直是所有二十世纪国家的货币政策的一个突出特征。甚至二战后期最好的国家货币德国马克在1949年1月1日至1995年6月底期间的价值下降了71\%。在同一时期,美元贬值了84\%。这种通胀对持有货币的人的影响与税收类似。随着我们之后的探讨,通货膨胀作为一种收入选择在信息时代将被广泛淘汰。新技术将允许财富的持有者绕过现代时期发行和管理货币的国家垄断。事实上,1997年和1998年席卷亚洲、俄罗斯和其他新兴经济体的信贷危机表明,国家货币和国家信用评级是不利于全球经济运作的陈旧思维。正是主权要求在一个管辖范围内的所有交易必须以国家货币计价的事实,才造成了央行家错误和投机者攻击的漏洞,从而引发了一个又一个的通货紧缩危机。在信息时代,个人将能够使用网络货币,因此宣布他们的货币独立。当个人能够通过万维网进行自己的货币政策时,国家继续控制工业时代的印刷机就会变得不那么重要,甚至不重要。它们为掌控世界财富的重要性,将被没有物理存在的数学算法所超越。在新千年,由私人市场控制的网络货币将取代政府发行的法定货币。只有穷人将成为通货膨胀和随后崩溃的受害者,这是法定货币注入经济的人工杠杆的后果。}

  \ParallelPar  

  \ParallelLText
  {Lacking their accustomed scope to tax and inflate, governments, even in traditionally civil countries, will turn nasty. As income tax becomes uncollectible, older and more arbitrary methods of exaction will resurface. The ultimate form of withholding tax --de facto or even overt hostage-taking -- will be introduced by governments desperate to prevent wealth from escaping beyond their reach. Unlucky individuals will find themselves singled out and held to ransom in an almost medieval fashion. Businesses that offer services that facilitate the realization of autonomy by individuals will be subject to infiltration, sabotage, and disruption. Arbitrary forfeiture of property, already commonplace in the United States, where it occurs five thousand times a week, will become even more pervasive. Governments will violate human rights, censor the free flow of information, sabotage useful technologies, and worse. For the same reasons that the late, departed Soviet Union tried in vain to suppress access to personal computers and Xerox machines, Western governments will seek to suppress the cybereconomy by totalitarian means. }
  
  \ParallelRText
  {\small 在没有了习惯性的征收所得税的情况下,即使在传统上文明的国家,政府也将变得残忍。更老、更武断的征税方式将重新出现。政府迫切希望阻止财富逃离其管辖范围,将引入最终形式的代扣税——实际上甚至是公开的劫持人质。不幸的个人将发现自己被单独挑选并以几乎中世纪的方式被绑架和赎金。为个人实现自治的服务的企业将受到渗透、破坏和破坏。在美国,已经普遍存在的任意没收财产行为,每周发生五千次的情况将变得更为普遍。政府将侵犯人权,审查信息的自由流动,破坏有用的技术等等。由于晚已逝去的苏联试图无效地压制个人电脑和复印机的使用,西方政府也将通过极权主义手段试图抑制网络经济。 }
  \ParallelPar

\section{卢德派的回归}   
  \ParallelLText
  {Such methods may prove popular among some population segments. The good news about individual liberation and autonomy will seem to be bad news to many who are frightened by the transition crisis and do not expect to be winners in the new configuration of society. The apparent popularity of the draconian capital controls imposed in 1998 by Malaysian prime minister Mahathir Mohamad in the wake of the Asian meltdown testifies to residual enthusiasm among many for the old-fashioned closed economy dominated by the nation-state. This nostalgia for the past will be fed by resentments inflamed by the inevitable transition crisis. The greatest resentment is likely to be centered among those of middle talent in currently rich countries. They particularly may come to feel that information technology poses a threat to their way of life. The beneficiaries of organized compulsion, including millions receiving income redistributed by governments, may resent the new freedom realized by Sovereign Individuals. Their upset will illustrate the truism that “where you stand is determined by where you sit.” }
  
  \ParallelRText
  {\small 这些方法可能会在某些人群中流行。个人解放和自治的好消息可能会给许多人带来打击,他们对转型危机感到恐惧,并且不期望在社会新形态的赢家中。马来西亚总理马哈蒂尔·莫哈末于1998年在亚洲经济危机后实施的严厉的资本管制政策表明,仍有很多人倾向于传统的以国家为主导的封闭经济。这种对过去的怀旧情结将因转型危机必然带来的愤怒而得到滋养。最大的憎恨可能集中在目前富裕国家中的中等才能人群上。他们特别可能会感到信息技术对他们的生活方式构成威胁。有组织强制的受益者,包括数百万接受政府收入再分配的人,可能会对主权个人实现的新自由感到不满。他们的不满将说明“你所站的位置取决于你的处境”的真理。}
\ParallelPar  

  \ParallelLText
  {It would be misleading, however, to attribute all the bad feelings that will be generated in the coming transition crisis to the bald desire to live at someone else's expense. More will be involved. The very character of human society suggests that there is bound to be a misguided moral dimension to the coming Luddite reaction. Think of it as a bald desire fitted with a moral toupee. We explore the moral and moralistic dimensions of the transition crisis. Self-interested grasping of a conscious kind has far less power to motivate actions than does self-righteous fury. While adherence to the civic myths of the twentieth century is rapidly falling away, they are not without their true believers. Many humans, as the passage quoted from Craig Lambert attests, are belongers, who place importance on being members of a group. The same need to identify that motivates fans of organized sports makes some partisans of nations. Everyone who came of age in the twentieth century has been inculcated in the duties and obligations of the twentieth-century citizen. The residual moral imperatives from industrial society will stimulate at least some neo-Luddite attacks on information technologies.  }
  
  \ParallelRText
  {\small 然而,将即将到来的过渡危机中将产生的所有不良情绪归咎于光秃秃地想生活在别人的代价之下是具有误导性的。它将涉及更多方面。人类社会的性质表明,即将到来的勒德派反应必然会有一个错误的道德维度。把它想象成一个带有道德假发的光秃秃的愿望。我们探讨了过渡期的道德和道德主义维度。有意识的自我利益追求远不如自以为是的愤怒来推动行为。虽然对于20世纪公民传统的遵循正在迅速减少,但它们并不是没有真正的信徒。正如从Craig Lambert引用的文章所证明的那样,许多人是“属于者”,他们认为成为团队成员很重要。同样的认同需求使一些国家倾向于某些主义。在20世纪成年的每个人都受到20世纪公民责任和义务的熏陶。工业社会的残留道德义务将激发至少一些基于信息技术的新勒德派攻击。 }

\ParallelPar  

  \ParallelLText
  {In this sense, this violence to come will be at least partially an expression of what we call “moral anachronism,” the application of moral strictures drawn from one stage of economic life to the circumstances of another. Every stage of society requires its own moral rules to help individuals overcome incentive traps peculiar to the choices they face in that particular way of life. Just as a farming society could not live by the moral rules of a migratory Eskimo band, so the Information Society cannot satisfy moral imperatives that emerged to facilitate the success of a militant twentieth-century industrial state. We explain why. }
  
  \ParallelRText
  {\small 在这个意义上,未来的暴力至少部分上将是我们所谓的“道德时代落后”的表达,即将道德准则应用于另一种经济生活方式的情况。每个社会阶段都需要其自身的道德准则来帮助个人克服该特定生活方式下他们面临的激励陷阱。正如一个农业社会不能按照一个流浪爱斯基摩人团队的道德规则生活一样,信息社会也无法满足于对二十世纪激进的工业国家成功所产生的道德要求。我们会解释原因。 }

\ParallelPar 
 
  \ParallelLText
  {In the next few years, moral anachronism will be in evidence at the core countries of the West in much the way that it has been witnessed at the periphery over the past five centuries. Western colonists and military expeditions stimulated such crises when they encountered indigenous hunting-and-gathering bands, as well as peoples whose societies were still organized for farming. The introduction of new technologies into anachronistic settings caused confusion and moral crises. The success of Christian missionaries in converting millions of indigenous peoples can be laid in large measure to the local crises caused by the sudden introduction of new power arrangements from the outside. Such encounters recurred over and over, from the sixteenth century through the early decades of the twentieth century. We expect similar clashes early in the new millennium as Information Societies supplant those organized along industrial lines. }
  
  \ParallelRText
  {\small 在接下来的几年中,这种道德时代落后将在西方核心国家的许多领域中得到体现,就像在过去的五个世纪里在边缘地区所见到的一样。当西方殖民者和军事远征队遭遇土著狩猎采集部落以及那些仍以耕种种植为生的人时,就会出现这样的危机。新技术的引入到这些时代落后的环境中会导致混乱和道德危机。基督教传教士成功地转化了数百万土著民族,这在很大程度上归功于由外部引入的新能源系统带来的本地危机。从16世纪到20世纪初,这样的冲突一再发生。我们预计,在信息社会取代沿工业线组织的社会时期早期,将会有类似的冲突出现。 }
  \ParallelPar

\subsection{对强制的怀旧情感}

  \ParallelLText
  {The rise of the Information Society will not be wholly welcomed as a promising new phase of history, even among those who benefit from it most. Everyone will feel some misgivings. And many will despise innovations that undermine the territorial nation-state. It is a fact of human nature that radical change of any kind is almost always seen as a dramatic turn for the worse. Five hundred years ago, the courtiers gathered around the duke of Burgundy would have said that unfolding innovations that undermined feudalism were evil. They thought the world was rapidly spiraling downhill at the very time that later historians saw an explosion of human potential in the Renaissance. Likewise, what may someday be seen as a new Renaissance from the perspective of the next millennium will look frightening to tired twentieth-century eyes. }
  
  \ParallelRText
  {\small 信息社会的崛起并不是所有人都欣然接受的一段有前途的历史阶段,即使是那些从中受益最多的人也会感到一些疑虑。每个人都会感到某些不安。许多人会鄙视破坏领土民族国家的创新。这是人性的一个事实,任何一种激进的变革几乎总被视为一个戏剧性的倒退。五百年前,围绕勃艮第公爵的宫廷人士会说,破坏封建制度的展开中的创新是邪恶的。他们认为世界正在迅速地走下坡路,而后来的历史学家则在文艺复兴时期看到了人类潜力的爆发。同样,从下一个千年的角度来看,有一天可能会被看作是新文艺复兴,但它会让疲惫的20世纪眼睛感到恐惧。 }
\ParallelPar  

  \ParallelLText
  {There is a high probability that some who are offended by the new ways, as well as many who are disadvantaged by them, will react unpleasantly. Their nostalgia for compulsion will probably turn violent. Encounters with these new “Luddites” will make the transition to radical new forms of social organization at least a measure of bad news for everyone. Get ready to duck. With the speed of change outracing the moral and economic capacity of many in living generations to adapt, you can expect to see a fierce and indignant resistance to the Information Revolution, notwithstanding its great promise to liberate the future. 
  }
  
  \ParallelRText
  {\small 有很大可能会有一些被新方式冒犯的人,以及许多受其不利影响的人,会做出令人不愉快的反应。他们对强迫的怀旧情结可能会变得暴力。与这些新的“卢德派”相遇将使向激进的新社会组织形式的转变对每个人都至少有一些不好的消息。准备好躲避吧。随着变化的速度超过生活中许多人适应的道德和经济能力,你可以预料到对信息革命的凶猛反抗,尽管它有解放未来的巨大承诺。
  }

\ParallelPar  

  \ParallelLText
  {You must understand and prepare for such unpleasantness. A series of transition crises lies ahead. Deflationary tribulations, such as the Asian contagion that swept through the Far East to Russia and other emerging economies in 1997 and 1998, will erupt sporadically as the dated national and international institutions left over from the Industrial Era prove inadequate to the challenges of the new, dispersed, transnational economy. The new information and communication technologies are more subversive of the modern state than any political threat to its predominance since Columbus sailed. This is important because those in power have seldom reacted peacefully to developments that undermined their authority. They are not likely to now. 
  }
  
  \ParallelRText
  {\small 你必须理解并为这样的不愉快情况做好准备。一系列的转型危机将接踵而至。通货紧缩的磨难,例如 1997 年和 1998 年席卷远东到俄罗斯和其他新兴经济体的亚洲瘟疫,将会间歇性地爆发,因为那些过时的国内外机构已经证明无法应对新的、分散的跨国经济的挑战。新的信息和通讯技术比哥伦布航海后的任何政治威胁都要更具颠覆性,对于现代国家的主导地位更加具有威胁性。这一点很重要,因为那些在权力中的人很少会对破坏他们权威的发展做出和平反应。他们现在也不太可能这么做。
  }
\ParallelPar  

  \ParallelLText
  {The clash between the new and the old will shape the early years of the new millennium. We expect it to be a time of great danger and great reward, and a time of much diminished civility in some realms and unprecedented scope in others. Increasingly autonomous individuals and bankrupt, desperate governments will confront one another across a new divide. We expect to see a radical restructuring of the nature of sovereignty and the virtual death of politics before the transition is over. Instead of state domination and control of resources, you are destined to see the privatization of almost all services governments now provide. For inescapable reasons that we explore in this book, information technology will destroy the capacity of the state to charge more for its services than they are worth to you and other people who pay for them.  
  }
  
  \ParallelRText
  {\small 新旧之间的冲突将塑造新千年的早期年份。我们预计这将是一个充满危险和奖励的时代,在某些领域中,文明的减弱将是空前的,而在其他领域中,范围将是前所未有的。越来越自主的个人和破产、绝望的政府将在新的分界线上相互对抗。我们预计,在过渡结束之前,主权的性质将发生根本性的重组,政治几乎完全死亡。与其主导和控制资源,你注定会看到几乎所有政府现在提供的服务的私有化。出于我们在本书中探讨的无法逃避的原因,信息技术将摧毁国家为其服务所收费比其价值和其他为其支付的人们的贡献更高的状态的能力。
  }

  \ParallelPar

\subsection{市场赋予的主权}

\ParallelLText
  {To an extent that few would have imagined only a decade ago, individuals will achieve increasing autonomy over territorial nation-states through market mechanisms. All nation-states face bankruptcy and the rapid erosion of their authority. Mighty as they are, the power they retain is the power to obliterate, not to command. Their intercontinental missiles and aircraft carriers are already artifacts, as imposing and useless as the last warhorse of feudalism. }
  
  \ParallelRText
  {\small 仅仅十年前,大多数人都无法想象,通过市场机制,个人将获得对领土国家越来越多的自治权。所有国家都面临破产和权威的迅速侵蚀。尽管它们强大,但它们所保留的权力只是毁灭而非统治的权力。它们的洲际导弹和航空母舰已经成为历史,就像封建主义时代的最后一匹战马一样具有威严和无用。}

  \ParallelPar 
 
  \ParallelLText
  {Information technology makes possible a dramatic extension of markets by altering the way that assets are created and protected. This is revolutionary. Indeed, it promises to be more revolutionary for industrial society than the advent of gunpowder proved to be for feudal agriculture. The transformation of the year 2000 implies the commercialization of sovereignty and the death of politics, no less than guns implied the demise of oath-based feudalism. Citizenship will go the way of chivalry.  }
  
  \ParallelRText
  {\small 信息技术通过改变资产的创造和保护方式,使得市场得以大幅度扩展。这是一场革命。实际上,对于工业社会而言,它的革命性可能比火药对封建农业的影响还要深远。2000年的转型意味着主权的商业化和政治的消亡,正如火器对宣誓效忠的封建制度的终结一样。公民身份将逐渐成为历史。}

  \ParallelPar  

  \ParallelLText
  {We believe that the age of individual economic sovereignty is coming. Just as steel mills, telephone companies, mines, and railways that were once “nationalized” have been rapidly privatized throughout the world, you will soon see the ultimate form of privatization --the sweeping denationalization of the individual. The Sovereign Individual of the new millennium will no longer be an asset of the state, a de facto item on the treasury's balance sheet. After the transition of the year 2000, denationalized citizens will no longer be citizens as we know them, but customers.}
  
  \ParallelRText
  {\small 我们相信,个人经济主权时代即将到来。正如曾被国有化的钢铁厂、电话公司、矿山和铁路在全球范围内被迅速私有化一样,你将很快见证终极私有化的形式——个人的彻底非国有化。新千年的主权个体将不再是国家资产,不再是国库资产负债表上的一个实际项目。在2000年过渡后,非国有化公民将不再是我们所知道的公民,而是顾客。}
 \ParallelPar


\section{带宽胜过边界}

  \ParallelLText
  {The commercialization of sovereignty will make the terms and conditions of citizenship in the nation-state as dated as chivalric oaths seemed after the collapse of feudalism. Instead of relating to a powerful state as citizens to be taxed, the Sovereign Individuals of the twenty-first century will be customers of governments operating from a “new logical space.” They will bargain for whatever minimal government they need and pay for it according to contract. The governments of the Information Age will be organized along different principles than those which the world has come to expect over the past several centuries. Some jurisdictions and sovereignty services will be formed through “assortive matching,” a system by which affinities, including commercial affinities, are the basis upon which virtual jurisdictions earn allegiance. In rare cases, the new sovereignties may be holdovers of medieval organizations, like the 900-year-old Sovereign Military Hospitaller Order of St. John of Jerusalem, of Rhodes and of Malta. More commonly known as the Knights of Malta, the order is an affinity group for rich Catholics, with 10,000 current members and an annual income of several billions. The Knights of Malta issues its own passports, stamps, and money, and carries on full diplomatic relations with seventy countries. As we write it is negotiating with the Republic of Malta to reassume possession of Fort St. Angelo. Taking possession of the castle would give the Knights the missing ingredient of territoriality that will enable it to be recognized as a sovereignty. The Knights of Malta could once again become a sovereign microstate, instantly legitimized by a long history. It was from Fort St. Angelo that the Knights of Malta turned back the Turks in the Great Siege of 1565. Indeed, they ruled Malta for many years thereafter, until they were expelled by Napoleon in 1798. If the Knights of Malta were to return in the next few years, there could be no clearer evidence that the modern nation-state system, ushered in after the French Revolution, was merely an interlude in the longer sweep of history in which it has been the norm for many kinds of sovereignties to exist at the same time. }
  
  \ParallelRText
  {\small 主权商业化将使国家公民身份的条款和条件显得像封建主义崩溃后的骑士誓言一样过时。21世纪的主权个人将作为政府的客户从“新的逻辑空间”运营,而不是像纳税的公民那样与强大的国家有关。他们将按照合同谈判并付款以获得他们需要的最小政府。信息时代的政府将按照不同的原则组织,这些原则不同于过去几个世纪世界所期望的原则。一些司法管辖区和主权服务将通过“交配匹配”形成,这是一种基于亲和力(包括商业亲和力)而非地理位置的虚拟管辖权挣得忠诚的基础。在罕见情况下,新的主权可能仍然是中世纪组织(如有着900年历史的圣约翰医院长,罗得和马耳他的邦联)。这个亲密团体是一个富有的天主教徒的可以拥有自己护照,邮票和货币的组织,年收入几十亿。圣约翰骑士团发行自己的护照,邮票和货币,并与70个国家保持着完整的外交关系。就在我们写作的时候,它正在与马耳他共和国进行谈判,以重新占据安琪洛堡。占领城堡将成为这个组织具有领土性的缺失要素,并使其能够被认可为主权。圣约翰骑士团可能再次成为一个主权微型国家,并因悠久的历史而被立即合法化。正是从安琪洛堡,圣约翰骑士团在1565年的大围攻中击退了土耳其人。他们统治了马耳他多年,直到1798年被拿破仑驱逐。如果圣约翰骑士团在未来几年内返回,那么现代国家体系,即法国革命后出现的国家体系,只是历史长河中的一个片段,历史长河中许多种类的主权同时存在是正常的。}


  \ParallelPar
  \ParallelLText
  {Still another and very different model for a postmodern sovereignty based on assortive matching is the Iridium satellite telephone network. At first glance, you may think it odd to treat a cellular telephone service as a kind of sovereignty. Yet Iridium has already received recognition as a virtual sovereignty by international authorities. As you may know, Iridium is a global cellular phone service that allows subscribers to receive calls on a single number, wherever they find themselves on the planet, from Featherston, New Zealand, to the Bolivian Chaco. To allow calls to be routed to Iridium subscribers anywhere on the globe, given the architecture of global telecoms, international telecom authorities had to agree to recognize Iridium as a virtual country, with its own country code: 8816. It is a short step logically from a virtual country comprising satellite telephone subscribers to sovereignty for more coherent virtual communities on the World Wide Web that span borders. Bandwidth, or the carrying capacity of a communications medium, has been expanding faster than computational capacity multiplied after the invention of transistors. If this trend to greater bandwidth continues, as we believe likely, it is only a matter of a few years, soon after the turn of the millennium, until bandwidth becomes sufficiently capacious to make technically possible the “metaverse,” the alternative, cyberspace world imagined by the science fiction novelist Neal Stephenson. Stephenson's “metaverse” is a dense virtual community with its own laws. We believe it is inevitable that, as the cybereconomy becomes richer, its participants will seek and obtain exemption from the anachronistic laws of nation-states. The new cybercommunities will be at least as wealthy and competent at advancing their interests as the Sovereign Military Hospitaller Order of St. John of Jerusalem, of Rhodes and of Malta. Indeed, they will be more capable of asserting themselves because of far-reaching communications and information warfare capabilities. We explore still other models of fragmented sovereignty in which small groups can effectively lease the sovereignty of weak nation-states, and operate their own economic havens much as free ports and free trade zones are licensed to do today.}
  
  \ParallelRText
  {\small 基于匹配交配的后现代主权的另一个非常不同的模型是铱星手机网络。乍一看,你可能认为将一个蜂窝电话服务视为一种主权奇怪。然而,铱星已经被国际当局承认为虚拟主权国家。正如您所知,铱星是一种全球蜂窝电话服务,允许用户在全球任何一个地方接收来电,从新西兰的菲瑟斯顿到玻利维亚查科​​。为了允许呼叫路由到全球任何地方的铱星用户,考虑到全球电信的架构,国际电信当局不得不同意将铱星作为一个虚拟国家来承认,其拥有自己的国家代码:8816。从卫星电话订户组成的虚拟国家到跨越边界的更一致的虚拟社区的主权,这仅是逻辑上的简短步骤。带宽,即通信媒介的承载能力,正在以晶体管发明后计算能力的增长速度加快。如果这种向更大带宽的趋势继续下去,正如我们所相信的那样,那么仅仅几年,在新千年之后不久,带宽就会变得足够宽敞,以使“元宇宙”成为可能,即科幻小说家尼尔·斯蒂芬森想象的另一种虚拟社区。斯蒂芬森的“元宇宙”是一个拥有自己的法律的密集虚拟社区。我们相信,随着网络经济变得更加富裕,其参与者将寻求并获得豁免过时的国家法律的豁免权。由于具有广泛的通信和信息战争能力,新的网络社区将至少与圣约翰马耳他医院长的邦联一样富裕和有能力推进自己的利益。实际上,他们将更有能力表现自己。因为我们认为越来越多的小组可以有效地租用虚弱国家的主权,并运营自己的经济避风港,就像自由港和自由贸易区今天被许可做一样,我们探索了其他碎片化主权的模型。}

  \ParallelPar
  \ParallelLText
  {A new moral vocabulary will be required to describe the relations of Sovereign Individuals with one another and what remains of government. We suspect that as the terms of these new relations come into focus, they will offend many people who came of age as “citizens” of twentieth-century nation-states. The end of nations and the “denationalization of the individual” will deflate some warmly held notions, such as “equal protection under the law,” that presuppose power relations that are soon to be obsolete. As virtual communities gain coherence, they will insist that their members be held accountable according to their own laws, rather than those of the former nation-states in which they happen to reside. Multiple systems of law will again coexist over the same geographic area, as they did in ancient and medieval times.}
  
  \ParallelRText
  {\small 需要新的道德词汇来描述主权个体之间以及政府剩余部分的关系。我们怀疑随着这些新关系的术语逐渐清晰,许多成长于二十世纪民族国家的“公民”会受到冒犯。国家的终结和“个体非国籍化”将破坏一些温暖的理念,如“法律平等保护”,这些理念预设的权力关系很快将过时。随着虚拟社区的凝聚,他们将坚持按照自己的法律来追究成员的责任,而不是根据他们恰巧居住的前民族国家的法律。在同一地理区域中,多个法律系统将再次并存,就像古代和中世纪时期那样。}

  \ParallelPar
  \ParallelLText
  {Just as attempts to preserve the power of knights in armor were doomed to fail in the face of gunpowder weapons, so the modern notions of nationalism and citizenship are destined to be short-circuited by microtechnology. Indeed, they will eventually become comic in much the way that the sacred principles of fifteenth-century feudalism fell to ridicule in the sixteenth century. The cherished civic notions of the twentieth century will be comic anachronisms to new generations after the transformation of the year 2000. The Don Quixote of the twenty-first century will not be a knight-errant struggling to revive the glories o f feudalism but a bureaucrat in a brown suit, a tax collector yearning for a citizen to audit. }
  
  \ParallelRText
  {\small 正如试图保留穿甲骑士的权力注定要失败一样,现代民族主义和公民身份的概念注定会在微技术面前被短路。事实上,它们最终将变得滑稽可笑,就像十五世纪封建主义的神圣原则在十六世纪的嘲笑中失败一样。二十世纪所珍视的公民观念将成为二千年后的新一代的滑稽时代语汇。21世纪的唐吉柯德将不是一个挣扎恢复封建主义辉煌的骑士,而是一名穿着棕色衣服的官僚,一名渴望审计公民的税务员。}
  \ParallelPar


\section{边区法律的复兴}
  
  \ParallelLText
  {We seldom think of governments as competitive entitles, except in the broadest sense, so the modern intuition about the range and possibilities of sovereignty has atrophied. In the past, when the power equation made it more difficult for groups to assert a stable monopoly of coercion, power was frequently fragmented, jurisdictions overlapped, and entities of many different kinds exercised one or more of the attributes of sovereignty. Not infrequently, the nominal overlord actually enjoyed scant power on the ground. Governments weaker than the nation-states are now faced with sustained competition in their ability to impose a monopoly of coercion over a local territory. This competition gave rise to adaptations in controlling violence and attracting allegiance that will soon be new again.   When the reach of lords and kings was weak, and the claims of one or more groups overlapped at a frontier, it frequently happened that neither could decisively dominate the other. In the Middle Ages, there were numerous frontier or “march” regions where sovereignties blended together. These violent frontiers persisted for decades or even centuries in the border areas of Europe. There were marches between areas of Celtic and English control in Ireland; between Wales and England, Scotland and England, Italy and France, France and Spain, Germany and the Slav frontiers of Central Europe, and between the Christian kingdoms of Spain and the Islamic kingdom of Granada. Such march regions developed distinct institutional and legal forms of a kind that we are likely to see again in the next millennium. Because of the competitive position of the two authorities, residents of march regions seldom paid tax. What is more, they usually had a choice in deciding whose laws they were to obey, a choice that was exercised through such legal concepts as “avowal” and “distraint” that have now all but vanished. We expect such concepts to become a prominent feature of the law of Information Societies. }  
  \ParallelRText
  {\small 我们很少将政府视为竞争实体,除了在最广泛的意义上,因而现代人对主权的范围及其可能性的直觉已经萎缩了。 在过去,权力往往是分散的,管辖是重叠的,不同类型的实体行使着主权的一种或多种属性;在这种权力等式中,很难有某个集团能稳定地保持垄断地位。名誉上的最高统治者,在下面并没有多少权力,这种情况在历史上并不少见。现在,比民族国家弱小的政府,它们在地方施加权力的垄断地位,就面临着持续的竞争。这些竞争,曾经改变了控制暴力和吸引效忠的形式,而新的改变很快就会出现。当领主和国王们的势力单薄,往往就会出现一种现象:对同一块边境地区,有一个或多个团体主张权力,而任何一方都无法占据决定性的支配地位。在中世纪,有很多的边疆或“边区”(March)。在这些地方,主权重叠,暴力丛生。边区在欧洲存在了几十年甚至几个世纪,广泛存在于凯尔特人和英特兰人控制的爱尔兰地区之间,在威尔士和英格兰、苏格兰和英格兰、意大利和法国、法国和西班牙、德国和中欧的斯拉夫人边境之间,以及在西班牙的基督教王国和格拉纳达的伊斯兰王国之间。边区形成了独特的制度和法律,在下一个千年,我们很可能会重温它们。在边区,由于存在两个相互竞争的当局,住在这里的人很少交税。更重要的是,他们往往可以选择遵循谁的法律,通过“宣誓”或“封租”等法律方式。这些法律概念和方式现在都不复存在了;我们认为,它们将会成为信息社会法律的明显特征。}
  \ParallelPar

\subsection{超越国籍}

  \ParallelLText
  {Before the nation-state, it was difficult to enumerate precisely the number of sovereignties that existed in the world because they overlapped in complex ways and many varied forms of organization exercised power. They will do so again. The dividing lines between territories tended to become clearly demarcated and fixed as borders in the nation-state system. They will become hazy again in the Information Age. In the new millennium, sovereignty will be fragmented once more. New entities will emerge exercising some but not all of the characteristics we have come to associate with governments. }  
  \ParallelRText
  {\small 在国民国家之前,很难准确地列举出世界上存在的主权数量,因为它们以复杂的方式重叠,许多不同形式的组织行使权力。它们将再次这样做。在国家体系中,领土之间的分界线倾向于变得清晰明确并固定为边界。在信息时代中,它们将再次变得模糊。在新千年中,主权将再次分裂。新实体将出现,行使我们已经习惯了与政府相关的某些特征,但并非全部。}
  \ParallelPar

  \ParallelLText
  {Some of these new entities, like the Knights Templar and other religious military orders of the Middle Ages, may control considerable wealth and military power without controlling any fixed territory. They will be organized on principles that bear no relation to nationality at all. Members and leaders of religious corporations that exercised sovereign authority in parts of Europe in the Middle Ages in no sense derived their authority from national identity. They were of all ethnic backgrounds and professed to owe their allegiance to God, and not to any affinities that members of a nationality are supposed to share in common.}  
  \ParallelRText
  {\small 这些新实体中,像圣殿骑士团和其他中世纪的宗教军团一样,可能会控制相当大的财富和军事力量,而不控制任何固定的领土。它们将按照与国籍无关的原则组织起来。在中世纪欧洲某些地区行使主权权力的宗教公司的成员和领导者,绝不是从国家身份中获得他们的权威。他们来自不同的族裔背景,并自称效忠于上帝,而不是效忠于成员国之间被认为有共同利益的亲缘关系。}
  \ParallelPar

\subsection{赛博空间的商业共和国}

  \ParallelLText
  {You will also see the re-emergence o f associations of merchants and wealthy individuals with semisovereign powers, like the Hanse (confederation of merchants) in the Middle Ages. The Hanse that operated in the French and Flemish fairs grew to encompass the merchants of sixty cities. The “Hanseatic League,” as it is redundantly known in English (the literal translation is “Leaguely League”), was an organization of Germanic merchant guilds that provided protection to members and negotiated trade treaties. It came to exercise semisovereign powers in a number of Northern European and Baltic cities. Such entities will re-emerge in place of the dying nation-state in the new millennium, providing protection and helping to enforce contracts in an unsafe world.}  
  \ParallelRText
  {\small 您还将看到一些富有的商人和个人组成的半主权团体重新出现,例如中世纪的汉萨(商人联盟)。在法国和佛兰德的博览会上运作的汉萨发展到包括六十个城市的商人。被冗余地称为“汉萨同盟”的它是德语商业协会的组织,为成员提供保护,并谈判贸易协议。在一些北欧和波罗的海城市,它开始行使半主权的力量。在新千年代,这样的实体将出现,取代正在衰亡的民族国家,在一个不安全的世界中提供保护,并帮助执行合同。}
  \ParallelPar

  \ParallelLText
  {In short, the future is likely to confound the expectations of those who have absorbed the civic myths of twentieth-century industrial society. Among them are the illusions of social democracy that once thrilled and motivated the most gifted minds. They presuppose that societies evolve in whatever way governments wish them to-preferably in response to opinion polls and scrupulously counted votes. This was never as true as it seemed fifty years ago. Now it is an anachronism, as much an artifact of industrialism as a rusting smokestack. The civic myths reflect not only a mindset that sees society's problems as susceptible to engineering solutions; they also reflect a false confidence that resources and individuals will remain as vulnerable to political compulsion in the future as they have been in the twentieth century. We doubt it. Market forces, not political majorities, will compel societies to reconfigure themselves in ways that public opinion will neither comprehend nor welcome. As they do, the naive view that history is what people wish it to be will prove wildly misleading.  }  
  \ParallelRText
  {\small 简而言之,未来很可能会使那些吸收了20世纪工业社会公民神话的人感到困惑。其中包括曾经激发并激励最有才华的人的社会民主主义幻想。它们预设社会会以政府所希望的方式演进,最好是响应民意调查和严格计数的选票。这在50年前就不如它看起来那么正确。现在,它已经过时了,就像生锈的烟囱一样是工业主义的产物。这些公民神话不仅反映了一种认为社会问题容易被工程解决的心态,它们还反映了一种错误的信心,即资源和个人在将来仍将像在20世纪一样容易受到政治压迫。我们对此表示怀疑。市场力量而不是政治多数将迫使社会以公众意见既无法理解也无法欢迎的方式进行重构。随着这种重构,天真地认为历史是人们希望它成为的观念将被证明是极其误导的。}
  \ParallelPar


  \ParallelLText
  {It will therefore be crucial that you see the world anew. That means looking from the outside in to reanalyze much that you have probably taken for granted. This will enable you to come to a new understanding. If you fail to transcend conventional thinking at a time when conventional thinking is losing touch with reality, then you will be more likely to fall prey to an epidemic of disorientation that lies ahead. Disorientation breeds mistakes that could threaten your business, your investments, and your way of life. }  
  \ParallelRText
  {\small 因此,关键是你需要重新审视世界。这意味着从外部重新分析你可能认为理所当然的许多事情。这将使你能够得出新的理解。如果你在传统思维失去与现实接触的时候不能超越传统思维,那么你更有可能陷入即将到来的失序病毒的困扰中。失序会导致错误,可能会威胁到你的业务、投资和生活方式。}
  \ParallelPar

\subsection{Seeing Anew}

  \ParallelLText
  {To prepare yourself for the world that is coming you must understand why it will be different from what most experts tell you. That involves looking closely at the hidden causes of change. We have attempted to do this with an unorthodox analysis we call the study of megapolitics. In two previous volumes, \emph{Blood in the Streets} and \emph{The Great Reckoning}, we argued that the most important causes of change are not to be found in political manifestos or in the pronouncements of dead economists, but in the hidden factors that alter the boundaries where power is exercised. Often, subtle changes in climate, topography, microbes, and technology alter the logic of violence. They transform the way people organize their livelihoods and defend themselves.   }  
  \ParallelRText
  {\small 为了为即将到来的世界做好准备,您必须理解为什么它会与大多数专家告诉您的不同。这涉及密切关注变化的隐藏原因。我们尝试用一种非正统的分析,称为“大政治”的研究来做到这一点。在两个之前的卷册中,《街头流血》和《大清算》中,我们认为,变化的最重要的原因不在于政治宣言或死亡经济学家的声明,而是在于改变行使权力的边界的隐藏因素。通常,气候、地形、微生物和技术上的微妙变化都会改变暴力的逻辑。它们改变了人们组织生计和自我防卫的方式。}
  \ParallelPar


  \ParallelLText
  {Notice that our approach to understanding how the world changes is very different from that of most forecasters. We are not experts in anything, in the sense that we pretend to know a great deal more about certain “subjects” than those who have spent their entire careers cultivating highly specialized knowledge. To the contrary, we look from the outside in. We are knowledgeable around the subjects about which we make forecasts. Most of all, this involves seeing where the boundaries of necessity are drawn. When they change, society necessarily changes, no matter what people may wish to the contrary. }  
  \ParallelRText
  {\small 请注意,我们理解世界变化的方法与大多数预测者所采用的方法非常不同。我们不是某个领域的专家,也不是假装比那些一生都在培养高度专业知识的人了解更多关于某些“主题”的人。相反,我们从外部看。我们在我们预测的主题领域有知识。最重要的是,这涉及到看到必要性的边界在哪里被画出。当它们改变时,社会必然改变,无论人们想不想改变。}
  \ParallelPar


  \ParallelLText
  {In our view, the key to understanding how societies evolve is to understand factors that determine the costs and rewards of employing violence. Every human society, from the hunting band to the empire, has been informed by the interactions of megapolitical factors that set the prevailing version of the “laws of nature.” Life is always and everywhere complex. The lamb and the lion keep a delicate balance, interacting at the margin. If lions were suddenly more swift, they would catch prey that now escape. If lambs suddenly grew wings, lions would starve. The capacity to utilize and defend against violence is the crucial variable that alters life at the margin. }  
  \ParallelRText
  {\small 在我们看来,理解社会如何演变的关键在于理解决定使用暴力的成本和回报的因素。从狩猎队到帝国,每个人类社会都受到“大政治”因素相互作用的影响,这些因素确定了“自然法则”的普遍版本。生活始终是复杂的。羊和狮子在边缘相互作用,保持着微妙的平衡。如果狮子突然更敏捷,他们会捕捉到现在逃脱的猎物。如果羊突然长出翅膀,狮子会挨饿。利用和防御暴力的能力是改变生命边缘的关键变量。}
  \ParallelPar


  \ParallelLText
  {We put violence at the center of our theory of megapolitics for good reason. The control of violence is the most important dilemma every society faces. As we wrote in \emph{The Great Reckoning}:   }  
  \ParallelRText
  {\small 我们把暴力放在我们的“大都市政治”理论的中心,这是有充分理由的。控制暴力是每个社会面临的最重要的问题。正如我们在《大清算》中所写的那样:}
  \ParallelPar


  \ParallelLText
  {\emph{The reason that people resort to violence is that it often pays. In some ways, the simplest thing a man can do if he wants money is to take it. That is no less true for an army of men seizing an oil field than it is for a single thug taking a wallet. Power, as William Playfair wrote, “has always sought the readiest road to wealth, by attacking those who were in possession of it.” }}
  \ParallelRText
  {\small \emph{人们诉诸暴力的原因在于它往往有利可图。在某些方面,如果一个人想要钱,他最简单的做法就是拿走它。这同样适用于抢夺油田的一支军队和抢夺钱包的单个恶棍。威廉·普雷费尔写道,权力“总是通过攻击那些拥有财富的人来寻求最便捷的致富之路”。}}
  \ParallelPar

  \ParallelLText
  {\emph{The challenge to prosperity is precisely that predatory violence does pay well in some circumstances. War does change things. It changes the rules. It changes the distribution of assets and income. It even determines who lives and who dies. It is precisely the fact that violence does pay that makes it hard to control. }}
  \ParallelRText
  {\small \emph{繁荣的挑战恰恰在于掠夺性暴力在某些情况下确实有好处。战争改变了一切,改变了规则,改变了资产和收入的分配,甚至决定了谁生谁死。正是暴力有利可图的事实使其难以控制。}}
  \ParallelPar

  \ParallelLText
  {Thinking in these terms has helped us foresee a number of developments that better-informed experts insisted could never happen. For example, \emph{Blood in the Streets}, published in early 1987, was our attempt to survey the first stages of the great megapolitical revolution now under way. We argued then that technological change was destabilizing the power equation in the world. Among our principal points:  }  
  \ParallelRText
  {\small 从这些方面思考有助于我们预见到一些明眼人认为永远不可能发生的事情。例如,早在1987年初发表的《街头流血》是我们试图调查当前正在进行的巨型政治革命的最初阶段。我们当时认为,技术变革正在动摇世界上的权力平衡。我们的主要观点之一是:}
  \ParallelPar

\end{Parallel}

\begin{itemize}
\item We said that American predominance was in decline, which would lead to economic imbalances and distress, including another 1929-style stock market crash. Experts were all but unanimous in dfnying that such a thing could happen. Yet within six months, in October 1987, world markets were convulsed by the most violent sell-off of the century.
\item \small 我们曾说过,美国的主导地位正在衰落,这将导致经济失衡和困境,包括另一次1929年式的股市崩盘。专家们几乎一致地否认这种事情可能发生。然而,在六个月内,即1987年10月,世界市场被20世纪最激烈的抛售所动摇。
\end{itemize}

%\vspace{2pt}

\begin{itemize}
\item We told readers to expect the collapse of Communism. Again, experts laughed. Yet 1989 brought the events that “no one could have predicted.” The Berlin Wall fell, as revolutions swept away Communist regimes from the Baltic to Bucharest. 
\item \small 我们告诉读者要预期共产主义的崩溃。同样,专家们嗤之以鼻。然而,在1989年,一些“不可预测”的事件发生了。柏林墙倒塌,革命席卷了从波罗的海到布加勒斯特的共产主义政权。
\end{itemize}


%\vspace{2pt}

\begin{itemize}
\item We explained why the mu1tiethnic empire the Bolshevik nomenklatura inherited from the tsars would “inevitably crack apart.” At the end of December 1991, the hammer-and-sickle banner was lowered over the Kremlin for the last time as the Soviet Union ceased to exist. 
\item \small 我们解释了为什么来自沙皇的多民族帝国会“不可避免地解体”。 1991年12月底,苏联停止存在,镇魂曲被最后一次放下。
\end{itemize}


%\vspace{2pt}

\begin{itemize}
\item During the height of the Reagan arms buildup, we argued that the world stood at the threshold of sweeping disarmament. This, too, was considered unlikely, if not preposterous. Yet the following seven years brought the most sweeping disarmament since the close of World War I. 
\item \small 在里根军备竞赛的高峰期,我们认为世界处于全面裁军的门槛。这也被认为是不可能的,如果不是荒谬的话。然而,接下来的七年带来了自第一次世界大战结束以来最全面的裁军。
\end{itemize}


%\vspace{2pt}

\begin{itemize}
\item At a time when experts in North America and Europe were pointing to Japan for support of the view that governments can successfully rig markets, we said otherwise. We forecast that the Japanese financial assets boom would end in a bust. Soon after the fall of the Berlin Wall, the Japanese stock market crashed, losing almost half its value. We continue to believe that its ultimate low could match or exceed the 89 percent loss that Wall Street suffered at the bottom after 1929. 
\item \small 在北美和欧洲的专家指向日本支持政府可以成功操纵市场的观点的时候,我们持有不同的态度。我们预测,日本的金融资产繁荣将以失败告终。在柏林墙倒塌后不久,日本股市崩盘,几乎损失了一半的价值。我们继续认为,它的最终低点可能会匹配或超过1929年后华尔街跌至底部的89\%的损失。
\end{itemize}


%\vspace{1pt}

\begin{itemize}
\item At a point when almost everyone, from the middle-class family to the world's largest real estate investors, appeared to believe that property markets could only rise and not fall, we warned that a real estate bust was in the offing. Within four years, real estate investors throughout the world lost more than \$1 trillion as property values dropped. 
\item \small 在几乎每个人,从中产阶级家庭到全球最大的房地产投资者,似乎都相信房地产市场只能上涨而不能下跌的时候,我们警告道,房地产危机即将来临。在四年内,全球房地产投资者因房价下跌而损失了1万亿美元以上。
\end{itemize}

%\vspace{1pt}

\begin{itemize}
\item Long before it was obvious to the experts, we explained in \emph{Blood in the Streets} that the income of blue-collar workers had decreased and was destined to continue falling on a long-term basis. As we write today, almost a decade later, it has at last begun to dawn on a sleepy world that this is true. Average hourly wages in the United States have fallen below those achieved in the second Eisenhower administration. In 1993, average annualized hourly wages in constant dollars were \$18,808. In 1957, when Eisenhower was sworn in for his second term, U.S. annualized average hourly wages were \$18,903.
\item \small 长期以来,早在专家们看得出来之前,我们就在《血腥街头》一书中解释,蓝领工人的收入已经下降,并且将在长期基础上继续下降。如今几乎过了十年,这一点终于开始对瞌睡世界有所意识。美国平均小时工资已经下降至艾森豪威尔政府的第二个任期所实现的水平以下。在1993年,美国的年平均工时工资是18808美元。在1957年艾森豪威尔宣誓就职第二个任期时,美国年平均工时工资为18903美元。 
\end{itemize}


\begin{Parallel}{97mm}{45mm}

  \ParallelLText
  {While the main themes of \emph{Blood in the Streets} have proven remarkably accurate with the benefit of hindsight, only a few years ago they were considered rank nonsense by the guardians of conventional thinking. A reviewer in \emph{Newsweek} in 1987 reflected the closed mental climate of late industrial society when he dismissed our analysis as “an unthinking attack on reason.”}  
  \ParallelRText
  {\small 尽管《血腥街头》的主要主题在回顾中表现出了惊人的准确性,但仅仅几年前,它们被传统思维的守护者认为是无稽之谈。1987年《新闻周刊》的一位评论家反映了后期工业社会的封闭思维气氛,当他把我们的分析视为“对理性的无思考攻击”时。}
  \ParallelPar


  \ParallelLText
  {You might imagine that \emph{Newsweek} and similar publications would have recognized with the passage of time that our line of analysis had revealed something useful about the way the world was changing. Not a bit. The first edition of \emph{The Great Reckoning} was greeted with the same sniggering hostility that welcomed \emph{Blood in the Streets}. No less an authority than the \emph{Wall Street Journal} categorically dismissed our analysis as the nattering of “your dopey aunt.”  }  
  \ParallelRText
  {\small 你可能会想象,《新闻周刊》和类似的出版物随着时间的推移应该已经认识到,我们的分析线路揭示了关于世界正在发生变化的有用信息。可惜并没有。 《大清算》的第一版受到了与《血腥街头》相同的嘲笑和敌视。 华尔街日报等权威机构也毫不客气地将我们的分析称为“你迟钝的姑妄言”之类的话语。}
  \ParallelPar


  \ParallelLText
  {This chuckling aside, the themes of \emph{The Great Reckoning} proved less ludicrous than the guardians of orthodoxy pretended.   }  
  \ParallelRText
  {\small 然而,这些调笑话并没有能够掩盖《大清算》所阐述的主题,它们仍然被坚守正统的守护者所忽视。}
  \ParallelPar

\end{Parallel}

\begin{itemize}
\item We extended our forecast of the death of the Soviet Union, exploring why Russia and the other former Soviet republics faced a future of growing civil disorder, hyperinflation, and falling living standards. 
\item \small 我们扩大了我们对苏联解体的预测,并探讨了为什么俄罗斯和其他前苏联共和国面临着日益增长的内部冲突、高通货膨胀和生活水平下降的未来。
\end{itemize}


\begin{itemize}
\item We explained why the 1990s would be a decade of downsizing, including for the first time a worldwide downsizing of governments as well as business entities. 
\item \small 我们解释了为什么20世纪90年代将是一个裁员的十年,包括首次全球范围内的政府和企业实体裁员。
\end{itemize}


\begin{itemize}
\item We also forecast that there would be a major redefinition of terms of income redistribution, with sharp cutbacks in the level of benefits. Hints of fiscal crisis appeared from Canada to Sweden, and American politicians began to talk of “ending welfare as we know it.”
\item \small 我们还预测,将会有一个重大的收入再分配条款重新定义,福利水平将大幅削减。从加拿大到瑞典,财政危机的迹象开始显现,美国政客开始谈论“改变我們熟知的福利制度”。
\end{itemize}


\begin{itemize}
\item We anticipated and explained why the “new world order” would prove to be a “new world disorder.” Well before the atrocities in Bosnia engrossed the headlines, we warned that Yugoslavia would collapse into civil war. 
\item \small 我们预见并解释了为什么“新秩序”将证明是“新混乱”。早在波斯尼亚的暴行引起头条新闻之前,我们就警告说南斯拉夫将陷入内战。
\end{itemize}


\begin{itemize}
\item Before Somalia slid into anarchy, we explained why the pending collapse of governments in Africa would lead some countries there to be effectively placed into receivership. 
\item \small 在索马里陷入无政府状态之前,我们解释了为什么非洲政府的垮台将导致一些国家被有效地接管。
\end{itemize}


\begin{itemize}
\item We forecast and explained why militant Islam would displace Marxism as the principal ideology of confrontation with the West.
\item \small 我们预测并解释了为什么激进伊斯兰教会取代马克思主义成为与西方对抗的主要意识形态。
\end{itemize}


\begin{itemize}
\item Years before the Oklahoma bombing and the attempt to blow up the World Trade Center, we explained why the United States faced an upsurge in terrorism. 
\item \small 在奥克拉荷马城爆炸和企图炸毁世贸中心之前多年,我们就已经解释了为什么美国面临恐怖主义激增。
\end{itemize}


\begin{itemize}
\item Before the headlines that told of the rioting that swept Los Angeles, Toronto, and other cities, we explained why the emergence of criminal subcultures among urban minorities was setting the stage for widespread criminal violence. 
\item \small 在洛杉矶、多伦多和其他城市爆发骚乱之前,我们就已经解释了为什么城市少数族裔中的犯罪亚文化的出现为广泛的犯罪暴力局势铺平了道路。
\end{itemize}


\begin{itemize}
\item We also anticipated “the final depression of the twentieth century,” which began in Asia in 1989 and has been spreading back from the periphery toward the center of the global system. We said that the Japanese stock market would follow Wall Street's path after 1929, and that this would lead to credit collapse and depression. Although massive government intervention in Japan and elsewhere temporarily prevented markets from fully reflecting the deterioration of credit conditions, this only displaced and compounded economic distress, building pressures for competitive devaluations and a systemic credit collapse of the kind that imploded economies worldwide in the 1930s. 
\item \small 我们还预见到了“二十世纪最终的萧条”,它始于1989年的亚洲,并从边缘向全球体系中心扩散。我们说日本股市将沿着1929年华尔街的轨迹走向,这将导致信贷崩溃和经济衰退。尽管日本和其他地方的大规模政府干预暂时阻止了市场充分反映信贷条件恶化的情况,但这只是转移和加剧了经济困境,建立了竞争性货币贬值和类似于20世纪30年代全球性信贷崩溃的系统性信贷崩溃的压力。
\end{itemize}


\begin{Parallel}{97mm}{45mm}


  \ParallelLText
  {\emph{The Great Reckoning} also spelled out a number of controversial theses that have not yet been confirmed, or have not reached the level of development that we forecast:   }  
  \ParallelRText
  {\small 《大清算》也提出了许多有争议的论点,这些观点尚未得到证实,或者没有达到我们预测的水平:}
  \ParallelPar

\end{Parallel}


\begin{itemize}
\item We said that the Japanese stock market would follow Wall Street's path after 1929, and that this would lead to credit collapse and depression. Although unemployment rates in Spain, Finland, and a few other countries exceeded those of the 1930s, and a number of countries, including Japan, did experience local depressions, there has not yet been a systemic credit collapse of the kind that imploded economies worldwide in the 1930s. 
\item \small 我们曾说过,日本股市将在1929年后追随华尔街的道路,这将导致信贷崩溃和经济萧条。尽管西班牙、芬兰和其他一些国家的失业率超过了20世纪30年代的水平,并且包括日本在内的一些国家的确经历了局部经济萧条,但还没有像20世纪30年代那样导致全球经济崩溃的系统性信贷崩溃。
\end{itemize}


\begin{itemize}
\item We argued that the breakdown of the command-and-control system in the former Soviet Union would lead to the spread of nuclear weapons into the hands of ministates, terrorists, and criminal gangs. To the world's good fortune, this has not come to pass, at least not to the degree that we feared. Press reports indicate that Iran purchased several tactical nuclear weapons on the black market; more worryingly, the \emph{Times} of London reported on October 7, 1998, that “Osama bin Laden, the exiled millionaire Saudi terrorist leader, has acquired tactical nuclear weapons from the former Soviet Central Asian states, according to a leading Arab newspaper.” That said, there has been no officially confirmed deployment or use of nuclear weapons from the arsenals of the former Soviet Union.   
\item \small 我们认为,前苏联指挥与控制系统的崩溃将导致核武器传播到小国、恐怖分子和犯罪团伙手中。对于全球幸运的是,至少没有达到我们担心的程度。媒体报道表明,伊朗在黑市上购买了几个战术核武器;更令人担忧的是,《泰晤士报》于1998年10月7日报道称,“流亡的千万富翁沙特恐怖分子领袖本·拉登已从前苏联中亚国家购买了战术核武器,据一家领先的阿拉伯报纸报道。”尽管如此,前苏联的军火库还没有正式确认部署或使用过核武器。
\end{itemize}


\begin{itemize}
\item We explained why the “War on Drugs” was a recipe for subverting the police and judicial systems of countries where drug use is widespread, particularly the United States. With tens of billions of dollars in hidden monopoly profits piling up each year, drug dealers have the means as well as the incentive to corrupt even apparently stable countries. While the world media have carried occasional stories hinting at high-level penetration of the U.S. political system by drug money, the full story has not yet been told.  
\item \small 我们解释了为什么“毒品战争”是破坏警察和司法系统的谋略,特别是在毒品泛滥的国家,尤其是美国。由于每年隐蔽的垄断利润累积数十亿美元,毒品贩子具有腐败甚至是看似稳定国家的手段和动机。尽管全球媒体偶尔报道高层渗透美国政治体系的毒品资金,但完整的故事尚未被揭示。
\end{itemize}


\subsection{看得比别人更远}

\begin{Parallel}{97mm}{45mm}

 \ParallelLText
  {Notwithstanding the points where our forecasts were mistaken or seem mistaken in light of what is now known, the record stands to scrutiny. Much of what is likely to figure in future economic histories of the 1990s was forecast or anticipated and explained in \emph{The Great Reckoning}. Many of our predictions were not simple extrapolations or extensions of trends, but forecasts of major departures from what has been considered normal since World War II. We warned that the 1990s would be dramatically different from the previous five decades. Reading the news of 1991 through 1998, we see that the themes of \emph{The Great Reckoning} were borne out almost daily. }  
  \ParallelRText
  {\small 尽管我们的预测在已知信息的光下似乎是错误的,但记录经得起审查。在《大清算》一书中,许多将会成为未来经济史的内容都被预测或预见,并加以解释。我们的许多预测不是趋势的简单推展或延伸,而是对自第二次世界大战以来被认为是常态的重大变革的预测。我们警告说,1990年代将与前五十年截然不同。阅读1991年至1998年的新闻,我们发现《大清算》的主题几乎每天都在得到证实。}
  \ParallelPar



  \ParallelLText
  {We see these developments not as examples of isolated difficulties, trouble here, trouble there, but as shocks and tremors that run along the same fault line. The old order is being toppled by a megapolitical earthquake that will revolutionize institutions and alter the way thinking people see the world.   }  
  \ParallelRText
  {\small 我们认为这些发展不是孤立的困难例子,这里有麻烦,那里也有麻烦,而是沿着同一断层线发生的震荡和颤动。旧秩序正在被一个巨大的政治地震所推翻,这将彻底改变制度并改变思考人士看待世界的方式。}
  \ParallelPar



  \ParallelLText
  {In spite of the central role of violence in determining the way the world works, it attracts surprisingly little serious attention. Most political analysts and economists write as if violence were a minor irritant, like a fly buzzing around a cake, and not the chef who baked it.  }  
  \ParallelRText
  {\small 尽管暴力在确定世界运作方式中发挥着核心作用,但它吸引的严肃关注却出奇的少。大多数政治分析家和经济学家的写作都表现得好像暴力只是一个小烦恼,就像苍蝇围绕着蛋糕,而不是制作它的厨师。}
  \ParallelPar

\subsection{另一个超级政治先驱}


  \ParallelLText
  {In fact, there has been so little clear thinking about the role of violence in history that a bibliography of megapolitical analysis could be written on a single sheet of paper. In \emph{The Great Reckoning}, we drew upon and elaborated arguments of an almost entirely forgotten classic of megapolitical analysis, William Playfair's \emph{An Enquiry into the Permanent Causes of the Decline and Fall of Powerful and Wealthy Nations}, published in 1805. Here one of our departure points is the work of Frederic C. Lane. Lane was a medieval historian who wrote several penetrating essays on the role of violence in history during the 1940s and 1950s. Perhaps the most comprehensive of these was “Economic Consequences of Organized Violence,” which appeared in the \emph{Journal of Economic History} in 1958. Few people other than professional economists and historians have read it, and most of them seem not to have recognized its significance. Like Playfair, Lane wrote for an audience that did not yet exist.  }  
  \ParallelRText
  {\small 实际上,历史上暴力的角色所涉及的清晰思考如此之少,以至于关于巨型政治分析的书目可以写在一张纸上。在《大清算》中,我们借鉴了并详细阐述了威廉·普莱费尔的巨型政治分析经典作品《对富国强兵的永久性原因的探讨》(1805年出版)的论点,该作品几乎已被人遗忘。其中之一的出发点是弗雷德里克·C·莱恩的工作。莱恩是一位中世纪历史学家,在1940年代和1950年代写了几篇关于历史中暴力角色的深入研究论文。其中最全面的是《有组织暴力的经济后果》,发表于1958年的《经济史学杂志》上。除了专业经济学家和历史学家外,很少有其他人读过它,而且大多数人似乎没有意识到其重要性。像普莱费尔一样,莱恩为一个尚未存在的观众写作。 眼下,对我们最有用的莱恩和普莱费尔的主题是暴力与经济增长之间的关系,这是我们将要探讨的。 }
  \ParallelPar


\subsection{信息时代的洞见}

  \ParallelLText
  {Lane published his work on violence and the economic meaning of war well before the advent of the Information Age. He certainly was not writing in anticipation of microprocessing or the other technological revolutions now unfolding. Yet his insights into violence established a framework for understanding how society will be reconfigured in the Information Revolution.   }  
  \ParallelRText
  {\small 莱恩在信息时代出现之前就发表了关于暴力和战争经济意义的研究工作。他的著述并不是在预测微处理或其他科技革命的到来。然而,他对暴力的洞察建立了一个框架,可以理解信息革命时期,社会将如何重新构建。}
  \ParallelPar


  \ParallelLText
  {The window Lane opened into the future was one through which he peered into the past. He was a medieval historian, and particularly a historian of a trading city, Venice, whose fortunes surged and sagged in a violent world. In thinking about how Venice rose and fell, his attention was attracted to issues that can help you understand the future. He saw the fact that how violence is organized and controlled plays a large role in determining “what uses are made of scarce resources.”  }  
  \ParallelRText
  {\small 莱恩打开的未来之窗是一个窥视过去的窗口。他是一名中世纪历史学家,特别是威尼斯这座交易城市的历史学家。在思考威尼斯的兴衰时,他的注意力被吸引到一些可以帮助我们理解未来的问题上。他认为,暴力的组织和控制方式在决定“稀缺资源的使用方式”方面发挥了重要作用。}
  \ParallelPar



  \ParallelLText
  {We believe that Lane's analyses of the competitive uses of violence has much to tell us about how life is likely to change in the Information Age. But don't expect most people to notice, much less follow, so unfashionably abstract an argument. While the attention of the world is riveted on dishonest debates and wayward personalities, the meanderings of megapolitics continue almost unnoted. The average North American has probably lavished one hundred times more attention on O.J.Simpson and Monica Lewinsky than he has on the new micro technologies that are poised to antiquate his job and subvert the political system he depends on for unemployment compensation.}  
  \ParallelRText
  {\small 我们认为,莱恩关于暴力竞争使用的分析可以告诉我们,在信息时代生活将如何改变。但是不要期望大多数人会注意这么不太流行的抽象论证,更不要期望他们会跟随这个论证。当全世界的注意力都被卡戴珊和莫妮卡·莱温斯基之类的有争议的人物所吸引时,超级政治的波折几乎没有引起注意。普通北美人可能在O·J·辛普森和莫妮卡·莱温斯基方面花费的关注度是他们在新的微技术上花费关注度的百倍,而后者正准备使他们的工作过时,颠覆他们依赖于失业赔偿的政治系统。}
  \ParallelPar

\section{愿望的虚荣}


  \ParallelLText
  {The tendency to overlook what is fundamentally important is not confined solely to the couch dweller watching television. Conventional thinkers of all shapes and sizes observe one of the pretenses of the democratic nation-state --that the views people hold determine the way the world changes. Apparently sophisticated analysts lapse into explanations and forecasts that interpret major historical developments as if they were determined in a wishful way. A striking example of this type of reasoning appeared on the editorial page of the \emph{New York Times} just as we were writing: “Goodbye, Nation-State, Hello...What?,” by Nicholas Colchester. Not only was the topic, the death of the nation-state, the very topic we are addressing, but its author presents himself as an excellent marker to illustrate how far removed our way of thinking is from the norm. Colchester is no simpleton. He wrote as editorial director of the \emph{Economist} Intelligence Unit. If anyone should form a realistic view of the world it should be he. Yet his article clearly indicates in several places that “the coming of international government” is “now logically unstoppable.”}  
  \ParallelRText
  {\small 忽视根本重要的事情的倾向并不仅限于躺在沙发上看电视的人。各种形状和大小的传统思想家观察到民主国家的伪装之一 -- 人们持有的观点决定了世界变化的方式。明显高级的分析师们陷入解释和预测之中,将主要历史发展解释为如愿以偿的方式。刚好在我们写作的时候,在《纽约时报》的编辑专栏上出现了一个醒目的例子:“再见,民族国家,你好……什么?” ,由尼古拉斯·科尔切斯特(Nicholas Colchester)撰写。不仅是话题,民族国家的死亡,就是我们正在处理的话题,而且作者表现出自己是一个很好的标志,用以说明我们的思维方式与常规思维方式有多么不同。科尔切斯特不是傻瓜。他是《经济学家》情报部门的编辑总监。如果应该有人形成现实的世界观,那应该是他。然而,他的文章在若干地方明确指出,“国际政府的到来”现在是“逻辑上不可阻挡的”。}
  \ParallelPar


  \ParallelLText
  {Why? Because the nation-state is faltering and can no longer control economic forces.  In our view, this assumption verges on the absurd. To suppose that some specific new form of governance will emerge simply because another has failed is a fallacy. By that reasoning, Haiti and the Congo would long ago have had better government simply because what they had was so luminously inadequate.   }  
  \ParallelRText
  {\small 为什么?因为民族国家正在衰败,无法再控制经济力量。在我们看来,这种假设接近荒谬。假设仅仅因为另一种形式的治理失败了,某种特定的新治理形式就会出现,这是一种谬论。按照这种推理,海地和刚果早就应该有更好的政府了,仅仅因为它们原有的那种政府是如此的不足。}
  \ParallelPar

  \ParallelLText
  {Colchester's point of view, widely shared among the few who think about such things in North America and Europe, utterly fails to take into account the larger megapolitical forces that determine what types of political systems are actually viable. That is the focus of this book. When the technologies that are shaping the new millennium are considered, it is far more likely that we will see not one world government, but microgovernment, or even conditions approaching anarchy.    }  
  \ParallelRText
  {\small 考尔切斯特的观点在北美和欧洲这样的思考之人中间广泛共享,但它完全没有考虑到决定政治体制实际上是可行的更大的超级政治力量。这本书的重点就是这个问题。当考虑到塑造新千年的技术时,我们更有可能看到的不是一个世界政府,而是微型政府,甚至接近无政府状态。}
  \ParallelPar


  \ParallelLText
  {For every serious analysis of the role of violence in determining the rules by which everyone operates, dozens of books have been written about the intricacies of wheat subsidies, and hundreds more about arcane aspects of monetary policy. Much of this shortfall in thinking about the crucial issues that actually determine the course of history probably reflects the relative stability of the power configuration over the past several centuries. The bird that falls asleep on the back of a hippopotamus does not think about losing its perch until the hippo actually moves. Dreams, myths, and fantasies play a much larger role in informing the supposed social sciences than we commonly think.  }  
  \ParallelRText
  {\small 这在经济正义的丰富文献中特别明显。对经济正义和不正义的数百万言论和文字述说,与认真分析暴力如何塑造社会并因此为经济体制设定界限的书页相比,数量要多得多。然而在现代情境下,经济正义的表述预设社会被一种力量所统治,这种力量非常强大,可以夺取和重新分配生活的好东西。这种权力仅在现代时期的短暂几代人存在。现在它正在消逝。}
  \ParallelPar

  \ParallelLText
  {This is particularly evident in the abundant literature of economic justice. Millions of words have been uttered and written about economic justice and injustice for each page devoted to careful analysis of how violence shapes society, and thus sets the boundaries within which economies must function. Yet formulations of economic justice in the modem context presuppose that society is dominated by an instrument of compulsion so powerful that it can take away and redistribute life's good things. Such power has existed for only a few generations of the modem period. Now it is fading away. }  
  \ParallelRText
  {\small 对于决定每个人运作规则的暴力角色的严肃分析,有数十本书被写成关于小麦补贴的错综复杂事宜的数倍,还有几百本有关货币政策的深奥方面的书籍。关于实际上决定历史进程的关键问题思考的缺少大部分可能反映了过去几个世纪权力配置的相对稳定。在河马背上睡着的鸟类直到河马真正移动起来才会考虑失去它的栖息地。梦想、神话和幻想在通知所谓社会科学方面所起的作用比我们通常想象的要大得多。}
  \ParallelPar

\subsection{社会保障的大哥}

  \ParallelLText
  {Industrial technology gave governments greater instruments of control in the twentieth century than ever before. For a time, it seemed inevitable that governments would become so effective at monopolizing violence as to leave little room for individual autonomy. Nobody at mid-century was looking forward to the triumph of the Sovereign Individual.  }  
  \ParallelRText
  {\small 工业技术在20世纪为政府提供了比以往任何时候都更大的控制工具。曾经有一段时间,政府似乎注定会在垄断暴力方面变得如此有效,以至于个体自治的空间很小。半个世纪以来,没有人期待主权个体的胜利。}
  \ParallelPar


  \ParallelLText
{Some of the shrewdest observers of the mid-twentieth century became convinced on the evidence of the day that the tendency of the nation-state to centralize power would lead to totalitarian domination over all aspects of life. In George Orwell's \emph{1984} (1949), Big Brother was watching the individual vainly struggle to maintain a margin of autonomy and self-respect. It appeared to be a losing cause. Friedrich von Hayek's \emph{The Road to Serfdom} (1944) took a more scholarly view in arguing that freedom was being lost to a new form of economic control that left the state as the master of everything. These works were written before the advent of microprocessing, which has incubated a whole range of technologies that enhance the capacity of small groups and even individuals to function independently of central authority.  }  
  \ParallelRText
  {\small 二十世纪中期最精明的观察家之一通过当时的证据确信,民族国家集中权力的趋势将导致对生活各个方面的极权主义控制。在乔治·奥威尔的《1984》(1949)中,大兄弟一直在注视着个体徒劳地努力保持自治权和自尊心。这似乎是一场失败的事业。弗里德里希·冯·哈耶克(Friedrich von Hayek)的《通往奴役之路》(1944)则采取了更加学术的视角,认为自由正在失去,被一种新形式的经济控制所取代,这让国家成为了一切的主宰。这些作品均在微处理器的出现之前编写,而微处理器孕育了一系列增强小团体甚至个人独立于中央权威运作的技术。}
  \ParallelPar


  \ParallelLText
  {As shrewd as observers like Hayek and Orwell were, they were unduly pessimistic. History has unfolded its surprises. Totalitarian Communism barely outlasted the year 1984. A new form of serfdom may yet emerge in the next millennium if governments succeed in suppressing the liberating aspects of microtechnology. But it is far more likely that we will see unprecedented opportunity and autonomy for the individual. What our parents worried about may prove to be no problem at all. What they took for granted as fixed and permanent features of social life now seem destined to disappear. Wherever necessity sets boundaries to human choice, we adjust, and reorganize our lives accordingly. }  
  \ParallelRText
  {\small 正如哈耶克和奥威尔等观察家一样精明,他们过于悲观了。历史已经揭示了它的惊喜。极权主义共产主义仅在1984年之后未能幸存。如果政府成功地压制了微技术的解放方面,未来的新奴隶制形式可能会出现。但更有可能的是,我们将看到个人前所未有的机遇和自治权。我们父母所担心的问题可能根本不是问题,他们认为是社会生活的固定和永久特征的事情现在注定会消失。无论什么时候,当必要性给人类的选择设置边界时,我们都会进行调整并重新组织我们的生活。}
  \ParallelPar

\subsection{预测有风险}


  \ParallelLText
  {No doubt we put our small measure of dignity at risk in attempting to foresee and explain profound changes in the organization of life and the culture that binds it together. Most forecasts are doomed to make silly reading in the fullness of time. And the more dramatic the change they envision, the more embarrassingly wrong they tend to be. The world doesn't end. The ozone doesn't vanish. The coming Ice Age dissolves into global warming. Notwithstanding all the alarms to the contrary, there is still oil in the tank. Mr. Antrobus, the everyman of \emph{The Skin of Our Teeth}, avoids freezing, survives wars and threatened economic calamities, and grows old ignoring the studied alarms of experts.  }  
  \ParallelRText
  {\small 毫无疑问,在试图预见和解释生活组织和文化变革方面,我们将冒险失去一些尊严。大多数预测注定在时间的推移中变得愚蠢。而且,它们设想的变化越剧烈,它们就越容易出错。世界不会终结,臭氧不会消失。即使有所有相反的警报,油箱里还是有油。《我们的牙齿的皮肤》中的平民先生安特罗布斯避免冻结,幸存于战争和潜在的经济灾难中,老化时忽略专家的警报。 }
  \ParallelPar


  \ParallelLText
  {Most attempts to “unveil” the future soon tum out to be comic. Even where self-interest provides a strong incentive to clear thinking, forward vision is often myopic. In 1903, the Mercedes company said that “there would never be as many as 1 million automobiles worldwide. The reason was that it was implausible that as many as 1 million artisans worldwide would be trainable as chauffeurs.”}  
  \ParallelRText
  {\small 大多数揭示未来的尝试很快就变成了喜剧。即使自身利益提供了强烈的清晰思考的动力,前瞻性的愿景也经常是短视的。1903年,梅赛德斯公司说:“全世界永远不会超过100万辆汽车。原因是全球最多只有100万个工匠可供培训为司机。”}
  \ParallelPar


  \ParallelLText
  {Recognizing this should stop our mouths. It doesn't. We are not afraid to stand in line for a due share of ridicule. If we mistake matters greatly, future generations may laugh as heartily as they please, presuming anyone remembers what we said. To dare a thought is to risk being wrong. We are hardly so stiff and useless that we are afraid to err. Far from it. We would rather venture thoughts that might prove useful to you than suppress them out of apprehension that they might prove overblown or embarrassing in retrospect.  }  
  \ParallelRText
  {\small 认识到这一点应该让我们闭嘴。但事实并非如此。我们不怕被取笑,也会排队领取应得的嘲笑。虽然我们可能会大错特错,但后代可能会为我们的言论大笑,如果有人记得我们所说的话的话。敢于思考就冒着被错的风险。我们并非这么僵硬和无用,以至于害怕犯错。恰恰相反,我们宁愿尝试提出可能对您有用的想法,而不会因为担心它们可能被夸大或在回顾中尴尬而压抑它们。}
  \ParallelPar


  \ParallelLText
  {As Arthur C. Clarke shrewdly noted, the two overriding reasons why attempts to anticipate the future usually fall flat are “Failure of Nerve and Failure of Imagination.” Of the two, he wrote, “Failure of Nerve seems to be the more common; it occurs when even given all the relevant facts the would-be prophet cannot see that they point to an inescapable conclusion. Some of these failures are so ludicrous as to be almost unbelievable.”}  
  \ParallelRText
  {\small 正如亚瑟·克拉克所说,预测未来通常失败的两个主要原因是“胆怯的失败”和“想象的失败”。在这两者中,他写道,“胆怯的失败似乎更为常见;即使在提供了所有相关事实的情况下,那些想预言未来的人也看不到它们指向不可避免的结论。其中一些失败是如此荒谬,以至于几乎难以置信。”}
  \ParallelPar


  \ParallelLText
  {Where our exploration of the Information Revolution falls short, as it inevitably will, the cause will be due more to a lack of imagination than to a lack of nerve. Forecasting the future has always been a bold enterprise, one which properly excites skepticism. Perhaps time will prove that our deductions are wildly off the mark. Unlike Nostradamus, we do not pretend to be prophetic personalities. We do not foretell the future by stirring a wand in a bowl of water or by casting horoscopes. Nor do we write in cryptic verse. Our purpose is to provide you with a sober, detached analysis of issues that could prove to be of great importance to you.   }  
  \ParallelRText
  {\small 当我们对信息革命的探索受阻时,这必然更多地是由于缺乏想象力而不是缺乏勇气。预测未来一直是一项大胆的企业,这正是激起怀疑心理的原因。也许时间会证明我们的推断完全错了。与诺斯特拉达姆斯不同,我们不假装有预言性的个性。我们不通过在水的碗里搅拌魔杖或占星术来预测未来。我们也不写含义深奥的诗歌。我们的目的是为您提供一种冷静、客观的对重要问题进行分析的方法。}
  \ParallelPar


  \ParallelLText
  {We feel an obligation to set out our views, even where they seem heretical, precisely because they may not otherwise be heard. In the closed mental atmosphere of late industrial society, ideas do not traffic as freely as they should through the established media.}  
  \ParallelRText
  {\small 即使看似异端邪说,我们也有义务阐述我们的观点,因为它们可能不会被他人听到。在后期工业社会的封闭思维氛围中,思想不像它们应该在传统媒体中那样自由。}
  \ParallelPar


  \ParallelLText
  { This book is written in a constructive spirit. It is the third we have written together, analyzing various stages of the great change now under way. Like \emph{Blood in the Streets} and \emph{The Great Reckoning}, it is a thought exercise. It explores the death of industrial society and its reconfiguration in new forms. We expect to see amazing paradoxes in the years to come. On the one hand, you will witness the realization of a new form of freedom, with the emergence of the Sovereign Individual. You can expect to see almost the complete liberation of productivity. At the same time, we expect to see the death of the modern nation-state. Many of the assurances of equality that Western people have grown to take for granted in the twentieth century are destined to die with it. We expect that representative democracy as it is now known will fade away, to be replaced by the new democracy of choice in the cybermarketplace. If our deductions are correct, the politics of the next century will be much more varied and less important than that to which we have become accustomed. }  
  \ParallelRText
  {\small 这本书是以建设性的精神编写的。这是我们合作编写的第三本书,分析正在发生的重大变革的不同阶段。像《街头大屠杀》和《大清算》一样,它是一种思想锻炼。它探讨了工业社会的死亡以及它在新形式中的重组。我们预计未来几年会出现惊人的悖论。一方面,你将见证一种新形式的自由实现,即主权个人的出现。你可以期待看到生产力的几乎完全解放。同时,我们预计现代主权国家将会消亡。许多西方人在二十世纪习惯于接受的平等保证也将随之消失。我们预计现有的代议制民主将逐渐式微,被网络市场的新民主所取代。如果我们的推论正确,未来世纪的政治将比我们习惯的要更多样化,也更不重要。}
  \ParallelPar


  \ParallelLText
  {We are confident that our argument will be easy to follow, notwithstanding the fact that it leads through some territory that is the intellectual equivalent of the backwoods and bad neighborhoods. If our meaning is not entirely intelligible in places, that is not because we are being cute, or using the time-honored equivocation of those who pretend to foretell the future by making cryptic pronouncements. We are not equivocators. If our arguments are unclear, it is because we have failed the task of writing in a way that makes compelling ideas accessible. Unlike many forecasters, we want you to understand and even duplicate our line of thinking. It is based not upon psychic reveries or the gyrations of planets, but upon old-fashioned, ugly logic. For quite logical reasons, we believe that microprocessing will inevitably subvert and destroy the nation-state, creating new forms of social organization in the process. It is both necessary and possible for you to foresee at least some details of the new way of life that may be here sooner than you think.  }  
  \ParallelRText
  {\small 我们有信心说服您接受我们的论点,尽管有时候会经过某些思维混乱的地方。如果有些地方不太容易理解,那并不是因为我们想卖弄,或者使用掐头去尾的说法来预言未来。我们不是隐晦不明的人。如果我们的论点不清晰,那是因为我们在书写中没有成功地表达好我们的理念。与许多预言家不同的是,我们希望您能够理解并且复制我们的思维方式。我们的论点是根据老派、丑陋的逻辑推导而成,而非凭借心理幻觉或行星运转。由于一些十分合乎逻辑的原因,我们相信微处理将不可避免地颠覆和摧毁民族国家,在此过程中产生新的社会组织形式。至少可以预见到生活的某些新细节即将到来,这一点是必要的和可能的。
}
  \ParallelPar

\subsection{讽刺的未来预言}

  \ParallelLText
  {For centuries, the end of this millennium has been seen as a pregnant moment in history. More than 850 years ago, St. Malachy fixed 2000 as the date of the Last Judgment. American psychic Edgar Cayce said in 1934 that the earth would shift on its axis in the year 2000, causing California to split in two and inundating New York City and Japan. A Japanese rocket scientist, Hideo Itokawa, announced in 1980 that the alignment of the planets in a “Grand Cross” on August 18, 1999, would cause widespread environmental devastation, leading to the end of human life on earth.}  
  \ParallelRText
  {\small 数百年来,千禧年末日被视为历史上一个充满可能性的时刻。850多年前,圣马拉基预言2000年是世界末日的日期。1934年,美国通灵者埃德加·凯西称地球会在2000年转移其轴心,导致加利福尼亚州分裂成两半,淹没纽约市和日本。1980年,日本火箭科学家板川英雄宣布,1999年8月18日行星的“宏伟十字”将引起广泛的环境破坏,导致地球上的人类生命的终结。}
  \ParallelPar


  \ParallelLText
  {Such visions of apocalypse make a plump target for ridicule. After all, the year 2000, while an imposing round number, would appear to be only an arbitrary artifact of the Christian calendar as adopted in the West. Other calendars and dating systems calculate centuries and millennia from different starting points. By the reckoning of the Islamic calendar, for example, A.D. 2000 will be the year 1378. As ordinary-sounding as a year can be. According to the Chinese calendar, which repeats itself every sixty years, A.D. 2000 is just another year of the dragon. It is part of a continuous cycle that extends millennia into the past. Yet there is more than theological investment in the year 2000. Its importance is undergirded not only by Christian tradition, but by the limitations of mid-century information technology. The so-called Y2K or year 2000 computer problem, a potentially devastating logic flaw in billions of lines of computer code, could approximate apocalyptic conditions by closing down essential elements of industrial society on the millennial midnight. Many computers and microprocessors use software preserved and recycled from the earliest days of computers, when memory space, at \$600,000 per megabyte, was more valuable than gold. To save expensive space, the early programmers tracked dates with only the last two numbers of the year. This convention of employing two-digit date fields was carried over into most software employed in mainframe computers, and even found wide use in personal computers and so-called embedded chips, microprocessors that are used to control almost everything, from VCRs to car ignition systems, security systems, telephones, the switching systems that control the telephone network, process and control systems in factories, power plants, oil refineries, chemical plants, pipelines and much more. Thus, abbreviated into a two-digit field, the year 1999 would be “99.” The trouble is what happens when 00 comes up for the year 2000. Many computers will read this as 1900. This may make it impossible for many unremediated computers and other digital devices to recognize the year 2000 in date fields.}  
  \ParallelRText
  {\small 这样的末日预言成为嘲笑的目标。毕竟,2000年虽然是一个庞大的圆整数,但在西方采用的基督教日历中似乎只是牵强附会的结果。其他日历和日期计算系统从不同的起始点计算世纪和千禧年。例如,按照伊斯兰教历,公元2000年将是1378年。根据中国的农历,每60年重复一次的公元2000年只是龙年的又一年。它是一个延续了几千年的持续循环的一部分。然而,2000年的重要性不仅在于基督教传统,还在于20世纪信息技术的局限性。所谓的Y2K或2000年计算机问题是数十亿行电脑代码中可能严重的逻辑漏洞,可以通过关闭千禧时刻的主要元素来近似末日条件。许多计算机和微处理器使用了早期计算机保存和回收的软件,当时每兆字节的存储空间价格高达60万美元,比黄金更贵。为了节省昂贵的空间,早期的程序员只使用年份的最后两个数字跟踪日期。这个使用两位数日期字段的约定传承到大型计算机中使用的大多数软件中,甚至在个人计算机和嵌入式芯片中也得到广泛应用,微处理器用于控制几乎所有东西,从录像机到汽车点火系统、安全系统、电话、控制电话网络的交换系统,在工厂、发电厂、炼油厂、化工厂、管道和其它众多设施的处理和控制系统中使用。因此,缩写成两位数字的1999年将是“99”年。问题在于00年出现后会发生什么。许多计算机将把这个读作1900年。这可能使得许多未进行修复的计算机和其它数字设备无法识别日期字段中的2000年。}
  \ParallelPar


  \ParallelLText
  {The result will be a massive problem of data corruption that will provide an accidental illustration of a new potential for information warfare. In the Information Age, potential adversaries will be able to wreak havoc by detonating “logic bombs” that sabotage the functions of essential systems by corrupting the data upon which their functioning depends. As a military exercise, for example, you would not need to shoot down an airplane, if you could corrupt data crucial to its safe operation. Data corruption can do almost as much as physical weapons can to thwart the function of a modern society. That this has potentially far-reaching consequences should be obvious on reflection. For example, the \emph{Mail of London} reported on December 14, 1997, that airlines around the globe were planning to cancel hundreds of flights on January 1, 2000 out of fear that air traffic control systems could fail.19 Potential problems include not only the air traffic systems, but also date-sensitive functions built into the airplanes themselves. According to Boeing, many airplanes will require Y2K remediation. Many devices may have a problem if they try to log an event on an invalid date. The fly-by-wire computer-controlled systems that operate airplanes may malfunction if they are programmed to conclude that crucial maintenance was last performed in the year 1900. They many even go into an error loop and shut down. }  
  \ParallelRText
  {\small 结果将是大规模的数据损坏问题,将提供一个关于信息战新潜力的意外说明。在信息时代,潜在的对手将能够通过炸毁破坏其功能的必须数据的“逻辑炸弹”来制造混乱。例如,作为一个军事演习,如果你能破坏关键维护的数据,就不需要击落一架飞机了。数据损坏几乎可以像物理武器一样阻扰现代社会的功能。这个潜在的影响应该毋庸置疑。例如,《伦敦邮报》在1997年12月14日报道,全球航空公司计划取消2000年1月1日的数百个航班,因为他们担心空中交通管制系统可能会失效。潜在的问题不仅包括空中交通系统,还包括内置在飞机本身中的日期敏感功能。据波音公司称,许多飞机将需要进行Y2K修复。如果设备试图记录无效日期上的事件可能会出现问题。如果计算机程序被编程为认为关键维护是在1900年进行的,则控制飞机的电脑化飞控系统可能会出现故障。它们可能甚至会进入错误循环并关闭。}
  \ParallelPar


  \ParallelLText
  {The potentially lethal feedback effects of a logic time bomb that closes down noncompliant control systems could make the turn of the millennium a memorable date for unpleasant reasons. Remember, you can be affected by many devices that go into an error loop and shut down even if you are lucky enough not to find yourself in midair when the new millennium begins.  }  
  \ParallelRText
  {\small 潜在致命的反馈效应可能会关闭不符合规定的控制系统,使千年之交成为一个不愉快的纪念日。要记住,即使你在新千年开始时没有置身于空中,你也可能会受到许多设备进入错误循环并关闭的影响。}
  \ParallelPar


  \ParallelLText
  {You would be well advised to avoid an accident arising from non-Y2K-compliant pacemakers, or simply inebriated millennial revelers, because if the pacemakers shut down, the phone system might also, so the ambulance might never come. Unless you live in Brazil or Ukraine, you are used to picking up the telephone or turning on the car phone and automatically getting a dial tone. Happily, you seldom have to concern yourself with the technical details of how the telephone system operates. But it turns out that phone network switches and routers are highly date dependent. All connections are logged to a date and time, which is crucial to calculating call duration for billing. If you happen to make a one-minute call at 11:59:30 on December 31, 1999, and at 12:00:00 the system reads your call as having had a negative duration of more than 99 years, error loops and shutdown are possible. While long-distance companies are spending great sums to upgrade their switches to make them year 2000 compliant, and local service providers presumably are too, if even a few smaller companies fail to comply and go down, the whole network could be affected. You will be lucky to get a dial tone on January 1,2000.   }  
  \ParallelRText
  {\small 你最好避免因非Y2K兼容的起搏器或醉酒的千禧年狂欢者而发生事故,因为如果起搏器关闭,电话系统可能也会关闭,那么救护车可能永远不会来。除非你住在巴西或乌克兰,否则你习惯于拿起电话或打开车载电话自动获得拨号音。令人高兴的是,你很少需要关注电话系统的技术细节。但事实证明,电话网络交换机和路由器高度依赖日期。所有连接都记录在日期和时间上,这对于计算通话持续时间很关键。如果你在1999年12月31日的11:59:30打了一个一分钟电话,而在12:00:00系统读取你的通话时,认为它持续时间为负99年以上,那么错误循环和关闭是可能的。虽然长途公司正在花费大量资金升级其交换机以使其符合2000年标准,地方服务提供商也可能在这样做,但如果即使有一些较小的公司未能遵守规定而垮掉,整个网络都可能受到影响。你很幸运在2000年1月1日能听到拨号音。}
  \ParallelPar


  \ParallelLText
  {In the words of the Y2K expert Peter de Jager, “If we lose the ability to make a phone call, then we lose everything. We lose electronic fund transfers, we lose trading, we lose branch banking.” And the follow-on consequences of Y2K failures could come to more than that.}  
  \ParallelRText
  {\small 正如Y2K专家彼得·德贾格所说:“如果我们失去打电话的能力,那么我们就失去了一切。我们失去了电子资金转移,失去了交易,失去了分支银行。”}
  \ParallelPar

  \ParallelLText
  {Today, no one knows how pervasively crucial systems will crash because of the year 2000 problem. Embedded systems that cannot be reprogrammed but must be replaced if nonfunctional on a date-sensitive basis are found in cars, trucks, and buses built after 1976. (Perhaps you won't be in an accident with vehicles driven by persons with noncompliant pacemakers, because their vehicles might not start.) Embedded systems are also widespread in all types of power plants, water and sewage systems, medical devices, military equipment, aircraft, offshore oil platforms, oil tankers, alarm systems, and elevators. While many assemblies of microprocessors perform no date sensitive functions, they may nonetheless depend upon a clock, which may be Y2K sensitive, for their internal operations.   }  
  \ParallelRText
  {\small 由于Y2K问题的后续影响可能更多,因此目前没有人知道关键系统由于Y2K问题会崩溃到什么程度。不能重新编程但必须在日期敏感的基础上更换的嵌入式系统可以在1976年之后建造的汽车、卡车和公共汽车中找到。(也许你不会与搭载非符合性心脏起搏器的人开的车辆发生交通事故,因为他们的车可能无法启动。)嵌入式系统在各种发电厂、水和污水系统、医疗设备、军用设备、飞机、海上石油平台、油轮、报警系统和电梯中也很常见。虽然许多微处理器组件不执行日期敏感功能,但它们可能仍然依赖一个时钟,该时钟可能对其内部运行敏感于Y2K问题。}
  \ParallelPar

\section{主机和Y2K时间炸弹}

  \ParallelLText
  {The large-scale command and control systems of government and major corporations that involve high transaction volumes on mainframe computers were the original focus of Y2K concern. Because they operate on big machines for which most software is decades old and mostly noncompliant, the original alarms about Y2K, first sounded by Peter de Jager early in the 1990s, have focused mainly on the need to upgrade operating systems for big, multiprocessing mainframes. Mr. de Jager voiced concern that there might not be enough programmers conversant with COBOL, the old mainframe language, to complete the necessary patches and repairs to date sensitive code, even if every company and government agency with a vulnerable system had begun a crash program several years ago. Since this has not happened, and many operators of date-sensitive information systems have only just begun to assess their vulnerability, you can predict with a high degree of confidence that many mainframe systems will not be prepared to operate smoothly into the year 2000.}  
  \ParallelRText
  {\small 政府和大型企业的大规模指挥和控制系统涉及在主机计算机上进行高交易量。这些系统最初是Y2K关注的焦点,因为它们在大型设备上运行,而大多数软件已经数十年了,大部分都不兼容。最初在20世纪90年代由彼得德贾格提出的Y2K警报主要关注于需要升级大型多处理主机的操作系统。De Jager先生对于可能没有足够熟悉COBOL的程序员完成必要的补丁和修复日期敏感代码表示担忧,即使每个存在易受攻击系统的公司和政府机构几年前就开始了紧急计划。由于这种情况从未发生过,许多日期敏感信息系统的操作者直到最近才开始评估其易受攻击性,你可以有很大把握预测许多主机系统将无法平稳地运行到2000年。}
  \ParallelPar


  \ParallelLText
  {This is certainly a major concern because there is really no alternative to computer processing as the economy is now structured. Most businesses that are large enough to require a mainframe to handle their transactions are dependent upon transaction volumes that could not be managed with old fashioned nineteenth-century paperwork systems. If such businesses were forced to revert to shuffling paper they could expect to complete only a fraction of their normal transaction volume. The revenue shock from such a drop-off in business would endanger the survival of all but the most highly capitalized companies.}  
  \ParallelRText
  {\small 这肯定是一个重大问题,因为随着经济结构的变化,计算机处理确实没有替代品。大多数需要大型机来处理交易的企业都依赖于无法通过老式的19世纪的文书系统来管理的交易量。如果这些企业被迫回归手工处理文书,他们只能完成正常交易量的一小部分。这种业务量的下降将导致收入的严重损失,危及大多数只有高度资本化的公司才能存活。}
  \ParallelPar


  \ParallelLText
  {Almost everything related to money --invoicing, purchasing, and payroll systems, plus inventory controls and regulatory compliance-- would be fouled up. Huge quantities of data would be lost as computers crash or spew out false data in response to the Y2K problem. In some cases, it would actually prove a blessing if systems crash immediately rather than corrupting their data on a compounding basis until massive malfunction draws attention to the problem. What happens to files when a backup utility copies files originating on 07/04/99 to an update on 01/04/00? Who knows? Will the computer interpret a payment made on January 4, “1900,” for an insurance policy as a signal that the policy has been in default for a century, resulting in a canceled policy that is stricken from the file? Will banks and finance company computers seek to assess a hundred years of interest on loans that span the shift to the new millennium? Will your banks and brokerage firms retain accurate records of your account balances and give you timely access to your funds? These are just some of the interesting quandaries that you will confront because of the Y2K problem.}  
  \ParallelRText
  {\small 几乎与金钱有关的一切——包括发票、采购和支付系统,加上库存控制和监管合规性——都将混乱不堪。由于计算机崩溃或响应Y2K问题而产生了错误数据,大量数据将会丢失。有些情况下,如果系统立即崩溃而不是在复合的基础上破坏它们的数据,直到大规模故障引起关注,那么它实际上可能会被证明是一个好事。备份工具将07/04/99起源的文件复制到01/04/00更新时,文件会发生什么?谁知道?会不会让计算机将在1900年1月4日支付的保险费解释为一个信号,表明保单已经逾期一个世纪,导致保单被取消并从文件中删除?银行和金融公司的计算机会尝试在跨越新千年的贷款上评估一百年的利息吗?您的银行和经纪公司会保留准确的账户余额信息并及时为您提供资金访问吗?这些只是由于Y2K问题而面临的一些有趣的困境。}
  \ParallelPar


  \ParallelLText
  {Also high on your list of concerns should be what happens if the electricity goes off because of Y2K-related malfunctions. Without electricity, even most systems that are not Y2K-impaired will not function: your refrigerator, your freezer, perhaps even your source of heat. Y2K compliance issues could effect safety-related access and control functions at nuclear power plants. For example, personnel at nuclear facilities wear dosimetry devices that measure the amount of radiation exposure they receive while in the plant. These devices are analyzed regularly, with the data on exposure amounts maintained on a computer system that controls personnel access to the facility. Obviously, if the controlling computers fail, they will make a hash of all the elaborate controls designed to insure safe operation and guarantee proper maintenance. But, more importantly, a Nuclear Regulatory Commission memo notes that many “non-safety-related, but important computer-based systems, primarily databases and data collection necessary for plant operations,” are date sensitive.}  
  \ParallelRText
  {\small 此外,你最担心的事情之一应该是,如果由于Y2K相关的故障而导致电力中断会发生什么。没有电力,即使是没有Y2K问题的大多数系统也将无法运作:你的冰箱、冷冻机,甚至你的供暖系统。Y2K合规问题可能会影响核电站的与安全有关的访问和控制功能。例如,核设施的人员佩戴能够测量工作期间接受的辐射暴露量的剂量计。这些设备经常进行分析,暴露量的数据保存在控制人员进入该设施的计算机系统上。很明显,如果控制计算机出现故障,将破坏所有旨在确保安全操作和保证正确维护的复杂控制。但更重要的是,核能监管委员会备忘录指出,许多“不涉及安全但重要的基于计算机的系统,主要是必须用于电厂运营的数据库和数据收集”,对日期敏感。}
  \ParallelPar


  \ParallelLText
  {The conventional generating plants are not less vulnerable to Y2K disruption. For one thing, coal-powered plants are susceptible to disruptions in the surface transportation system that brings the coal to the boilers. In the 1997-1998 winter heating season, operators of coal-fired electricity generation found themselves forced to reduce output in some instances because of a slowdown in rail deliveries of Western coal arising from the merger of the Southern Pacific and Union Pacific railway systems. The problem arose because of incompatibilities between the computer control and dispatch systems employed by the two railroads. According to a Union Pacific spokesman, integrating the two systems became a “nightmare,” in spite of the fact that Union Pacific Technologies has been considered an industry leader in developing computerized transportation control systems. As a result of the programming difficulties, the railroad was unable to accurately track the movements of its freight cars. The failure of Union Pacific to master the assimilation of Southern Pacific is a bad omen about what could happen when Y2K logic time bombs disrupt transportation, power generation, and other aspects of the economy.}  
  \ParallelRText
  {\small 常规的发电厂同样容易受到Y2K的干扰。首先,燃煤发电厂容易受到表面运输系统中运煤的干扰。在1997-1998年的供暖季节中,一些燃煤发电运营商不得不减少产量,因为由于南太平洋和联合太平洋铁路系统合并而导致的西部煤炭铁路交货减缓。这个问题的出现是由于这两家铁路公司所采用的计算机控制和调度系统之间的不兼容。据联合太平洋一位发言人称,将两个系统整合起来变成了“噩梦”,尽管联合太平洋技术一直被认为是开发计算机化运输控制系统的行业领导者。由于编程困难,该铁路无法准确地跟踪其货车的运动。联合太平洋未能掌握统合南太平洋的失败是一个不祥之兆,表明逻辑时钟炸弹干扰了运输、发电和其他经济方面的后果。}
  \ParallelPar


  \ParallelLText
  {The biggest worry about the electric grid, however, arises from the fact that the whole system is subject to sensitive monitoring and computer control to transfer electricity from areas of surplus generation to those with a deficit. This process must be carefully monitored by computer to prevent power surges and system failures. All the transfers of electricity are logged to time and date for duration, much like a telephone connection. While heavy-duty mechanical relays are used to make the connections, they are controlled by computer systems. These computer controls, essential for load balancing, may fail for the same reasons as the phone networks. In fact, the power load distribution-control systems in North America are networked together through T-1 lines and telephone microwave links. So if the phone network fails, you can expect the electricity to go down as well. And remember, as the experience in Canada in January 1998 confirms, once the electricity shuts down over a wide area, getting the system running again is a challenge. A blackout may last for an inconveniently long time.  }  
  \ParallelRText
  {\small 然而,关于电网的最大担忧,是整个系统都受监控和计算机控制的敏感性影响,以将电力从盈余发电区域传输到不足区域。必须通过计算机仔细监控此过程,以防止电力过载和系统故障。所有电力转移都记录了时间和日期的持续时间,就像电话连接一样。虽然使用重型机械继电器进行连接,但它们由计算机系统控制。这些计算机控制对于负载平衡至关重要,可能因与电话网络相同的原因而失败。事实上,北美的电力负载分配控制系统通过T-1线和电话微波链接在一起。因此,如果电话网络出现故障,您可以预期电力也会停止供应。并且记住,正如加拿大在1998年1月的经验所证实的那样,一旦大范围停电,再使系统运行起来就是一个挑战。停电可能会持续很长时间。}
  \ParallelPar


\section{千年虫问题和核武库}

  \ParallelLText
  {For modern economies to have the electricity turn off in the dead of winter would be disruptive and potentially health threatening, especially for those who depend upon electric heat and medical equipment. Yet the worst case scenario is even worse. According to John Koskinen, who heads President Clinton's Y2K Conversion Council, U.S. military arsenals may cease to function on the stroke of midnight, December 31, 1999. While indicating that he does not wish to touch off undue alarm, Koskinen adds, “It needs to be worried about.” One concern about nuclear missiles “is if the data doesn't function and they actually go off.”}  
  \ParallelRText
  {\small 对于现代经济而言,在寒冬季节断电将会对依赖于电热和医疗设备的人们造成一定的破坏和健康威胁。然而最糟糕的情况更糟。根据约翰·科斯基嫩(John Koskinen)的说法,他是克林顿总统Y2K Conversion Council的负责人,美国军队的武器库可能会在1999年12月31日午夜时刻停止运作。虽然表示他不希望引起不必要的恐慌,但科斯基嫩补充说:“这是应该引起担忧的。” 关于核导弹的一个担忧是“如果数据不正常,它们可能发射。”}
  \ParallelPar


  \ParallelLText
  {Of course, this concern would apply with equal or greater force to Russian nuclear missiles. Russia's bankruptcy has made upgrades for Y2K compliance even more problematic than in the United States. And there is evidence that Russia is not yet taking Y2K conversion seriously. While one would pray that no accidental launches would occur, there should be little doubt that the turn of the year 2000 has a potential for aggravating global insecurity if for no other reason than that military communications systems in many countries may not function normally. As Koskinen puts it, “If you're sitting in a country and suddenly you can't quite figure out exactly what's happening, and your communications don't work as well, you get even more nervous.” So put that on your list of Y2K worries. The logic time bomb could precipitate the launch of genuinely explosive bombs -- a fact that highlights the danger from information warfare to centralized command and control systems.}  
  \ParallelRText
  {\small 当然,这个问题同样适用于俄罗斯的核导弹,甚至可能更加严峻。俄罗斯的破产使Y2K合规性的升级更加困难。此外,有证据表明,俄罗斯并未认真对待Y2K转换。虽然希望不会发生意外发射事件,但毫无疑问,千禧年交替可能导致全球不安全因素加剧,原因是许多国家的军事通讯系统可能无法正常工作。正如科斯基嫩所说:“如果你坐在一个国家里,突然你无法确定发生了什么事情,你的通讯也没有那么顺畅,你会变得更加紧张。” 因此,请将这列入您的Y2K担忧清单中。逻辑时限炸弹可能引发真正爆炸性的炸弹发射,这一事实凸显了信息战对于集中式指挥和控制系统的危险性。}
  \ParallelPar


  \ParallelLText
  {If terrorists wish to strike any centralized system, they may pick December 31, 1999, as the date for action because it will be a time of maximum vulnerability of many systems. Not only will communications be strained at best, with the possibility that electricity may fail, vehicles may not start, police, fire, and ambulance 911 service may not work, and so on, but many other functions you probably take for granted, such as air traffic control, may cease to function. No power means no water from the tap. Sewage systems would fail. Traffic lights could turn off. Within a few hours of a genuine breakdown in the transportation system, food in grocery stores would be shopped out. (Or looted.) On the basis of recent experience in American cities, you could suppose that no power, no water, no heat for many, no light, and fragmented communications with emergency services, including police and fire, all add up to no civilization. While no one can be sure what the impact of the Y2K problem may be, it could extend to looting and rioting in the streets, especially if it becomes known that there could be widespread failures to issue payroll, welfare and pension checks. }  
  \ParallelRText
  {\small 如果恐怖分子想要攻击任何集中式系统,他们可能会选择1999年12月31日作为行动时间,因为这将是许多系统最易受攻击的时候。不仅通讯最多只能保持紧绷状态,有停电的可能,车辆可能无法启动,警察、消防和救护911服务可能不起作用等,而且许多其他您可能认为理所当然的功能,如空中交通管制,可能会停止运作。没有电力就意味着自来水也没有了。污水系统将会失灵。交通信号灯可能会熄灭。在交通系统真正崩溃的几个小时内,超市里的食品就会被全部抢购(或掠夺)。根据美国城市最近的经验,如果没有电力、没有自来水、没有暖气、没有光线和紊乱的紧急服务通讯(包括警察和消防),所有这些因素加起来就意味着没有文明。虽然谁也不能确定Y2K问题的影响会是什么,但它可能会延伸到街头抢劫和暴动,特别是如果人们知道可能会有大量的工资、福利和养老金支票发放失败。}
  \ParallelPar


  \ParallelLText
  {Premonitions of doom about the new millennium do not necessarily rest upon theology tied to the Christian faith, but they do fit within the millennial tradition of Joachim de Fiore whose mediations convinced him that Christ was only “the second hinge of history” and that another was destined to unfold. “21 So argues the philosopher Michael Grosso, who suggests that the Information Revolution is piloting human history toward the realization of the prophetic vision of the Western world. He calls this “technocalypse.” Whether or not the development of technology is somehow informed by millennial visions, the Y2K phenomenon is an artifact of the predominant Western imagination of time. In a strange way, it could complement dreams, reveries and visions, or numerical interpretations of visions, like Newton's gloss on the prophecies of Daniel. These intuitive leaps begin with a perspective that takes the birth of Christ to be the central fact of history. They are compounded by the psychological power of large round numbers, which every trader will recognize as having an arresting quality. The two thousandth year of our epoch cannot help but become a focus for the imagination of intuitive people.}  
  \ParallelRText
  {\small 对于新千年的末日预言并不一定建立在基督教信仰上的神学基础,但它符合Joachim de Fiore的千年传统的思想。他的思考使他确信,基督仅是“历史的第二个转轴”,另一个转轴正注定要展开。所以,哲学家Michael Grosso认为,信息革命正在引领人类历史走向西方世界预言的实现,他称之为“technocalypse”。无论技术的发展是否与千禧年异象有所联系,Y2K现象都是主导西方时间想象的产物。以奇怪的方式,它都能够与梦想、沉思和幻想相辅相成,或像牛顿对但以理预言的诠释那样,对异象进行数字解读。这些直觉跳跃始于一个将基督的诞生视为历史中心事件的角度。它们被大而圆的数字的心理力量所弥漫,任何商人都会认识到这种数字具有引人注目的特性。我们时代的第二千年必须成为具有直觉力人们想象的焦点所在。}
  \ParallelPar


  \ParallelLText
  {A critic could easily make these premonitions seem silly, without even addressing the ambiguous and debatable theological notions of the Apocalypse and the Last Judgment that give these visions so much of their power. Interestingly, however, the Y2K computer glitch trumps the errors of arithmetic that otherwise might seem to devalue the importance of the year 2000 even within the Christian framework. The year 2000 has the potential to become an inflection point for the next stage of history simply because it brings forward the arrival of the new millennium. In strict logic, the next millennium will not begin until 2001. The year 2000 will be only the last year of the twentieth century, the two thousandth year since Christ's birth. Or it would be had Christ been born in the first year of the Christian era. He was not. In 533, when Christ's birth replaced the founding date of Rome as the basis for calculating years according to the Western calendar, the monks who introduced the new convention miscalculated Christ's birth. It is now accepted that he was born in 4 B.C. On that basis, a full two thousand years since his birth were completed sometime in 1997. Hence Carl Jung's apparently odd launch date for the start of a New Age. }  
  \ParallelRText
  {\small 批评家可能轻易地使这些预言看起来很傻,甚至不考虑启示录和末日审判的模糊和有争议的神学概念,这些幻象赋予了这些幻象如此强大的力量。然而有趣的是,Y2K电脑故障可以挑战算术错误,否则可能会在基督教框架内贬低2000年的重要性。2000年有可能成为历史下一阶段的拐点,仅仅因为它提前了新千年的到来。严格来说,下一个千年将不会开始直到2001年。2000年只是二十世纪的最后一年,也是自基督诞生以来的第二千年。如果基督在基督教纪元的第一年出生,那么它将是。但事实并非如此。533年,当基督的诞生取代罗马建立日期作为根据西方日历计算年份的基础时,引入新约定的修道士们计算出了基督的诞生年份。现在认为他在公元前4年出生。基于这一点,从他的诞生到现在已经过去了整整两千年,在1997年的某个时候完成了。因此,卡尔·荣格对新时代开始日期的明显奇怪的启动日期。}
  \ParallelPar


  \ParallelLText
  {Giggle if you will, but we do not despise or dismiss intuitive understandings of history. Although our argument is grounded in logic, not in reveries, we are awed by the prophetic power of human consciousness. Time after time, it redeems the visions of madmen, psychics, and saints. So it may be with the transformation of the year 2000. The date that has long been fixed in the imagination of the West looks to be the inflection point that at least half confirms that history has a destiny. We cannot explain why this should be, but nonetheless we are convinced that it is so.  }  
  \ParallelRText
  {\small 如果你愿意笑,但我们不蔑视或否认对历史的直觉理解。虽然我们的论点基于逻辑而不是幻想,但我们对人类意识的先知力量感到惊叹。一次又一次地,它弥合了疯子、通灵者和圣徒的幻象。关于2000年的转型可能也是如此。在西方想象中早已确定的日期似乎是转折点,至少半数证明了历史有一个命运。我们无法解释为什么会这样,但我们仍然相信它是这样的。}
  \ParallelPar


  \ParallelLText
  {Our intuition is that history has a destiny, and that free will and determinism are two versions of the same phenomenon. The human interactions that form history behave as though they were informed by a kind of destiny. Just as an electron plasma, a dense gas of electrons, behaves as a complex system, so do human beings. The freedom of individual movement by the electrons turns out to be compatible with highly organized collective behavior. As David Bohm said of an electron plasma, human history is “a highly organized system which behaves as a whole.”}  
  \ParallelRText
  {\small 我们的直觉是历史有一个命运,自由意志和决定论是同一个现象的两个版本。形成历史的人类互动表现得好像它们受到某种命运的启示。正如一个电子等离子,一个电子的密集气体的行为是一个复杂的系统,人类也是如此。电子个体的运动自由证明与高度组织的集体行为是兼容的。正如David Bohm所说的电子等离子体,人类历史是“一个高度组织的系统,表现为整体”。}
  \ParallelPar


  \ParallelLText
  {Understanding the way the world works means developing a realistic intuition of the way that human society obeys the mathematics of natural processes. Reality is nonlinear. But most people's expectations are not. To understand the dynamics of change, you have to recognize that human society, like other complex systems in nature, is characterized by cycles and discontinuities. That means certain features of history have a tendency to repeat themselves, and the most important changes, when they occur, may be abrupt rather than gradual. }  
  \ParallelRText
  {\small 了解世界的运作方式意味着发展对人类社会服从自然过程数学规律的现实直觉。现实是非线性的,但大多数人的期望并非如此。为了理解变化的动力学,您必须认识到人类社会像其他自然复杂系统一样具有周期性和不连续性。这意味着历史的某些特征有循环复发的倾向,并且最重要的变化可能是突然而非渐进式的。}
  \ParallelPar


  \ParallelLText
  {Among the cycles that permeate human life, a mysterious five-hundred-year cycle appears to mark major turning points in the history of Western civilization. As the year 2000 approaches, we are haunted by the strange fact that the final decade of each century divisible by five has marked a profound transition in Western civilization, a pattern of death and rebirth that marks new phases of social organization in much the way that death and birth delineate the cycle of human generations. This has been true since at least 500 B.C. when Greek democracy emerged with the constitutional reforms of Cleisthenes in 508 B.C. The following five centuries were a period of growth and intensification of the ancient economy, culminating in the birth of Christ in 4 B.C. This was also the time of the greatest prosperity of the ancient economy, when interest rates reached their lowest level prior to the modern period. }  
  \ParallelRText
  {\small 在渗透人类生活的周期中,一种神秘的五百年周期似乎标志着西方文明历史上的重大转折点。随着2000年的临近,我们受到一个奇怪的事实的困扰,即每个以五为倍数结尾的世纪的最后十年标志着西方文明的深刻转变,这种死亡和重生的模式标志着社会组织的新阶段,就像死亡和出生划分人类世代周期一样。至少从公元前500年希腊民主政治在克里斯泰尼斯的宪法改革中出现以来,这种情况就已经成立。接下来的五个世纪是古代经济增长和强化的时期,最终在公元前4年的基督诞生中达到高峰。这也是古代经济最繁荣的时期,利率在现代期前达到最低水平。}
  \ParallelPar

  \ParallelLText
  {The next five centuries saw a gradual winding down of prosperity, leading to the collapse of the Roman Empire late in the fifth century A.D. William Playfair's summary is worth repeating: “When Rome was at its highest pitch of greatness ... will be seen to be at the birth of Christ, that is, during the reign of Augustus, and by the same means it will be found declining gradually till the year 490.” It was then that the last legions dissolved, and the Western world sank into the Dark Ages. }  
  \ParallelRText
  {\small 接下来的五个世纪见证了繁荣逐渐走向尽头,导致罗马帝国在公元五世纪末崩溃。威廉·普莱费尔的总结值得重复:“当罗马处于最高峰时……就是在基督诞生时期,即在奥古斯都统治期间,随着同样的手段会逐渐衰落,直到公元490年。”就在那时,最后一批军队解散,西方世界沉入黑暗时代。}
  \ParallelPar

  \ParallelLText
  {During the following five centuries, the economy withered, long-distance trade ground to a halt, cities were depopulated, money vanished from circulation, and art and literacy almost disappeared. The disappearance of effective law with the collapse of the Roman Empire in the West led to the emergence of more primitive arrangements for settling disputes. The blood feud began to be significant at the end of the fifth century. The first recorded incident of trial by ordeal occurred precisely in the year 500.  }  
  \ParallelRText
  {\small 在接下来五个世纪里,经济萎缩,远程贸易停滞,城市人口减少,货币从流通中消失,艺术和文化也几乎消失。随着西罗马帝国的崩溃,有效法律的消失导致更为原始的纠纷解决安排的出现。血仇开始在公元五世纪末变得重要,首次记录的禁锢审判事件恰好发生在公元500年。}
  \ParallelPar

  \ParallelLText
  {Once again, a thousand years ago, the final decade of the tenth century witnessed another “tremendous upheaval in social and economic systems.” Perhaps the least known of these transitions, the feudal revolution, began at a time of utter economic and political turmoil. In \emph{The Transformation of the Year One Thousand}, Guy Bois, a professor of medieval history at the University of Paris, claims that this rupture at the end of the tenth century involved the complete collapse of the remnants of ancient institutions, and the emergence of something new out of the anarchy--feudalism. In the words of Raoul Glaber, “It was said that the whole world, with one accord, shook off the tatters of antiquity.” The new system that suddenly emerged accommodated the slow revival of economic growth. The five centuries now known as the Middle Ages saw a rebirth of money and international trade, along with the rediscovery of arithmetic, literacy, and time awareness. }  
  \ParallelRText
  {\small 再一次,一千年前,十世纪的最后十年见证了另一场“社会和经济体制的巨大动荡”。也许,最少为人所知的这些转变之一,封建革命,开始于完全的经济和政治动荡时期。在《一千年的转变》中,Guy Bois教授是巴黎大学的一名中世纪历史教授,他声称,在十世纪末这一断裂涉及古老制度的残留完全崩溃,新的一些东西由混乱之中诞生 -- 封建主义。用Raoul Glaber的话来说,“据说整个世界全都一齐摇掉了古代的破布。”新的系统突然出现,容纳了经济增长的缓慢复苏。现在被称为中世纪的五个世纪见证了货币和国际贸易的复兴,以及算术,读写能力和时间意识的重新发现。}
  \ParallelPar


  \ParallelLText
  {Then, in the final decade of the fifteenth century, there was yet another turning point. It was then that Europe emerged from the demographic deficit caused by the Black Death and almost immediately began to assert dominion over the rest of the globe. The “Gunpowder Revolution,” the “Renaissance,” and the “Reformation” are names given to different aspects of this transition that ushered in the Modern Age. It was announced with a bang when Charles VIII invaded Italy with new bronze cannon. It involved an opening to the world, epitomized by Columbus sailing to America in 1492. This opening to the New World launched a push toward the most dramatic economic growth in the experience of humanity. It involved a transformation of physics and astronomy that led to the creation of modem science. And its ideas were disseminated widely with the new technology of the printing press. }  
  \ParallelRText
  {\small 在15世纪的最后十年,又到了一个转折点。这时欧洲从黑死病造成的人口赤字中走出来,几乎立即开始对全球施加统治。“火药革命”、“文艺复兴”和“宗教改革”是这个转变的不同方面所命名的。当查理八世率军入侵意大利带着新铜炮时,它就以轰鸣声宣告了来临;它涉及了一个打开世界的过程,以哥伦布1492年航行到美洲为代表。这种向新世界的开放启动了人类经历中最为激动人心的经济增长。它涉及了对物理学和天文学的转变,导致了现代科学的诞生。它的思想以新的印刷技术广泛传播。}
  \ParallelPar


  \ParallelLText
  {Now we sit at the threshold of another millennial transformation. The large command and control systems inherited from the Industrial Era may break down like the one-horse shay on the stroke of the millennial midnight. Yet whether or not the Y2K logic bomb precipitates an immediate collapse of industrial society, its days are numbered. We expect the advent of the Information Society to utterly transform the world, in ways that this book is meant to explain. You would be perfectly within your rights to doubt this, since no cycle that repeats itself only twice in a millennium has demonstrated enough iterations to be statistically significant. Indeed, even much shorter cycles have been viewed skeptically by economists demanding more statistically satisfying proof. “Professor Dennis Robertson once wrote that we had better wait a few centuries before being sure” about the existence of four-year and eight- to ten-year trade cycles. By that standard, Professor Robertson would have to suspend judgment for about thirty thousand years to be sure that the five-hundred-year cycle is not a statistical fluke. We are less dogmatic, or more willing to take a hint. We recognize that the patterns of reality are more complex than the static- and linear-equilibrium models of most economists.  }  
  \ParallelRText
  {\small 现在我们正处于另一个千年转型的门槛上,从工业时代继承而来的大型指挥和控制系统可能会在千年之夜的一举手之劳间崩溃。然而,无论Y2K逻辑炸弹是否会导致工业社会的立即崩溃,其日子都已经被数了。我们期待信息社会的出现将彻底改变这个世界,而这本书就是为了解释这些变化。您有权怀疑这一点,因为在一个千年里只重复两次的周期还没有展示足够多的重复次数来证明其统计学意义。事实上,即使是更短的周期,也被经济学家持怀疑态度,要求更多的统计学证明。“丹尼斯·罗伯逊教授曾经写道,我们最好再等几个世纪才能确定”四年和八到十年的贸易周期是否存在。按照这一标准,罗伯逊教授将不得不暂缓判断大约三万年,以确定五百年周期不是一种统计学偶然。我们不那么教条,或者更愿意接受暗示。我们认识到现实模式比大多数经济学家的静态和线性平衡模型更为复杂}
  \ParallelPar

  \ParallelLText
  {We believe that the coming of the year 2000 marks more than another convenient division along an endless continuum of time. We believe it will be an inflection point between the Old World and a New World to come. The Industrial Age is rapidly passing, and its demise may, ironically, be accelerated by the fact that early computer memory was so expensive that it encouraged the widespread adoption of two-digit date fields. When Hallerith punch cards could accommodate only eighty characters each, abbreviating dates seemed a prudent thing to do. Contrary to the expectations of the early programmers, however, their abbreviation of the date field endured four decades until the end of the millennium as an accidental logic bomb that could destroy a large part of industrials society. The U.S. government's Office of Management and Budget described the problem in “Getting Federal Computers Ready for 2000,” a report dated February 7, 1997. The OMB concludes of computers: “Unless they are fixed or replaced, they will fail at the tum of the century in one of three ways: they will reject legitimate entries, or they will compute erroneous results, or they simply will not run.” These three outcomes in combination could cripple Industrial society. Its technology of mass production is destined to be eclipsed by a new technology of miniaturization in any event. A near-term crisis will merely accelerate the process. With the new information technology has come a new science of nonlinear dynamics, one whose startling conclusions are mere strands that have yet to be woven together into a comprehensive worldview. We live in the time of the computer, but our dreams are still spun on the loom. We continue to live by the metaphors and thoughts of industrialism. We don't yet imagine the world in terms of strange attractors. Our politics still straddles the industrial divide between right and left, as mapped by thinkers like Adam Smith and Karl Marx, who died before almost everyone now living was born. The industrial worldview, incorporating the operating principles of industrial science, is still the “commonsense” intuition of educated opinion. It is our thesis that the “common sense” of the Industrial Age will no longer apply to many areas as the world is transformed.}  
  \ParallelRText
  {\small 我们相信,2000年的到来标志着旧世界与即将到来的新世界之间的一种拐点。工业时代正在迅速过去,而它的终结可能会被加速,这是具有讽刺意味的,因为早期计算机内存非常昂贵,这促使广泛采用两位数字日期字段。当哈勃里斯打孔卡只能容纳80个字符时,缩写日期似乎是一件明智的事。然而,与早期程序员的预期相反,他们对日期字段的缩写持续了四十年,直到千年结束,作为一个意外的逻辑炸弹,可能摧毁工业社会的大部分。美国政府的管理和预算办公室在1997年2月7日的一份报告中描述了这个问题。OMB针对计算机的结论是:“除非修复或更换,否则它们将在世纪之交以以下三种方式之一失败: 它们将拒绝合法条目,或者它们将计算错误的结果,或者它们根本无法运行。” 这三个结果的组合可能会瘫痪工业社会。无论如何,其大规模生产的技术注定会被一种微型化的新技术所超越。近期危机只会加速这一过程。随着新信息技术的出现,一个新的非线性动力学科学应运而生,其惊人的结论只是尚未被织在一起形成全面的世界观。我们生活在计算机的时代,但我们的梦想仍然在纺织机上编织。我们继续按照工业主义的隐喻和思想进行生活。我们还没有想象以奇怪的吸引子而非工业主义的方式望去的世界。我们的政治仍然跨越着工业分化,就像亚当·斯密和卡尔·马克思这样的思想家绘制的那样,在右翼和左翼之间。这些思想家在几乎所有现在活着的人出生之前就去世了。工业世界观,融合了工业科学的操作原则,仍然是受过教育的观点的“常识”直觉。我们的论点是,工业时代的“常识”在许多领域将不再适用于随着世界的转变。}
  \ParallelPar

  \ParallelLText
  {More than eighty-five years after the day in 1911 when Oswald Spengler was seized with an intuition of a coming world war and “the decline of the West,” we, too, see “a historical change of phase occurring ... at the point preordained for it hundreds of years ago.” Like Spengler, we see the impending death of Western civilization, and with it the collapse of the world order that has predominated these past five centuries, ever since Columbus sailed west to open contact with the New World. Yet unlike Spengler we see the birth of a new stage in Western civilization in the coming millennium.   }  
  \ParallelRText
  {\small 在1911年,奥斯瓦尔德·施宾格被一种即将来临的世界大战和“西方文明的衰落”的直觉抓住,85年后的今天,我们也看到“历史性转变的阶段正在发生……在数百年前就预定的时刻”。像施宾格一样,我们看到西方文明的迫在眉睫的死亡,随之而来的是这过去五个世纪以来主导的世界秩序的崩溃,自哥伦布西航以来。然而,与施宾格不同,我们看到在新千年即将出现西方文明的新阶段。}
  \ParallelPar

\end{Parallel}  %导入第1章内容
\chapter[历史上的大都市转型]{历史上的大都市转型}
% 选项为页眉页脚的简写

\section{现代世界的衰落}
\begin{paracol}{2}
在我们看来,你们正在见证的不仅仅是现代时代的衰落,更是一种无情但隐藏的逻辑所驱动的发展。在我们共同理解的范畴之外,甚至超越CNN和报纸带给我们的信息,下一个千年将不再是“现代”的。我们之所以这么说,并不是在暗示你们将面临野蛮或落后的未来,虽然这是有可能的,而是强调历史阶段正在打开,而这将与你们出生时的历史阶段在本质上有所不同。
\switchcolumn
In our view, you are witnessing nothing less than the waning of the Modern Age. It is a development driven by a ruthless but hidden logic. More than we commonly understand, more than CNN and the newspapers tell us, the next millennium will no longer be "modern." We say this not to imply that you face a savage or backward future, although that is possible, but to emphasize that the stage of history now opening will be qualitatively different from that into which you were born.

\switchcolumn*
新的东西即将到来。正如农业社会与狩猎采集社会之间有着本质区别,工业社会与封建或家族农业系统之间有着根本不同的区别,新世界的到来将标志着与以往任何时代的根本性分别。
\switchcolumn
Something new is coming. Just as farming societies differed in kind from hunting-and-gathering bands, and industrial societies differed radically from feudal or yeoman agricultural systems, so the New World to come will mark a radical departure from anything seen before.

\switchcolumn*
在新千年中,经济和政治生活将不再像现代世纪那样以国家为主导而组织在巨大的规模上。那个给你带来世界大战、装配线、社会保障、所得税、除臭剂和烤箱的文明正在消亡。除臭剂和烤箱可能会幸存下来,但其他的则不会。像一个古老而曾经强大的人一样,民族国家的未来将只剩下年和日,而不再是世纪和十年。
\switchcolumn
In the new millennium, economic and political life will no longer be organized on a gigantic scale under the domination of the nation-state as it was during the modern centuries. The civilization that brought you world war, the assembly line, social security, income tax, deodorant, and the toaster oven is dying. Deodorant and the toaster oven may survive. The others won't. Like an ancient and once mighty man, the nation-state has a future numbered in years and days, and no longer in centuries and decades.

\switchcolumn*
政府已经失去了大部分的监管和强制力。共产主义的崩溃标志着五个世纪的长周期结束了,在这期间,政府组织中的权力量表现得比效率还要强大。现在,暴力的回报率已经不再高涨。世界历史上已经开始了历史性的相变。事实上,下一个千年记录现代时代沦落和衰落的吉本可能会宣称,当你读这本书时,它已经结束了。回顾过去,他可能会像我们一样说,它已经在1989年的柏林墙倒塌和1991年苏联的解体时期结束了。这两个时间点都可能成为定义文明演化的关键事件,是我们所知的现代时代的结束。
\switchcolumn
Governments have already lost much of their power to regulate and compel. The collapse of Communism marked the end of a long cycle of five centuries during which magnitude of power overwhelmed efficiency in the organization of government. It was a time when the returns to violence were high and rising. They no longer are. A phase transition of world-historic dimensions has already begun. Indeed, the future Gibbon who chronicles the decline and fall of the once-Modern Age in the next millennium may declare that it had already ended by the time you read this book. Looking back, he may say, as we do, that it ended with the fall of the Berlin Wall in 1989. Or with the death of the Soviet Union in 1991. Either date could come to stand as a defining event in the evolution of civilization, the end of what we now know as the Modern Age.

\switchcolumn*
人类发展的第四阶段即将来临,也许它最不可预测的特征是它将被称为新的名字。称之为“后现代”,称之为“网络社会”或“信息时代”。或者想出你自己的名字。没有人知道什么概念性的胶水会将昵称与历史的下一个阶段粘在一起。
\switchcolumn
The fourth stage of human development is coming, and perhaps its least predictable feature is the new name under which it will be known. Call it "Post-Modern." Call it the "Cyber Society" or the "Information Age." Or make up your own name. No one knows what conceptual glue will stick a nickname to the next phase of history.
  
\switchcolumn*
我们甚至不知道刚刚结束的五百年历史是否会继续被视为“现代”。如果未来的历史学家知道词语源流,那么不会是这样的。更加描述性的标题可能是“国家时代”或“暴力时代”。但这样的名字将超出当前定义历史时代的时间范围。牛津英语词典中,“现代”意味着“与现在和最近时期有关,与远古不同... 在历史使用中,通常应用于中世纪之后的时间(与古代和中世纪相矛盾)。”
\switchcolumn
We do not even know that the five-hundred-year stretch of history just ending will continue to be thought of as "modern." If future historians know anything about word derivations, it will not be. A more descriptive title might be "The Age of the State" or "The Age of Violence." But such a name would fall outside the temporal spectrum that currently defines the epochs of history. "Modern," according to the \emph{Oxford English Dictionary}, means "pertaining to the present and recent times, as distinguished from the remote past... In historical use commonly applied (in contradiction to ancient and medieval) to the time subsequent to the MIDDLE AGES."
\switchcolumn*
西方的人只有在意识到中世纪已经结束时,才有意识地认为自己是“现代人”。在1500年之前,没有人认为封建时期是西方文明中的“中间时期”。原因很明显,回想一下就会明白:在一个时代被合理地看作夹在另外两个历史时期之间之前,它必须已经结束了。在封建时期生活的人无法想象自己生活在古代文明和现代文明之间的半路上,直到他们意识到中世纪时期不仅已经结束,而且中世纪文明与黑暗时代或古代文明大相径庭。
\switchcolumn
A Western people consciously thought of themselves as "modern" only when they came to understand that the medieval period was over. Before 1500, no one had ever thought of the feudal centuries as a "middle" period in Western civilization. The reason is obvious upon reflection: before an age can reasonably be seen as sandwiched in the "middle" of two other historic epochs, it must have already come to an end. Those living during the feudal centuries could not have imagined themselves as living in a halfway house between antiquity and modern civilization until it dawned on them not just that the medieval period was over, but also that medieval civilization differed dramatically from that of the Dark Ages or antiquity.

\switchcolumn*
人类文化有盲点。我们没有词汇来描述生活中最大范围的范式变化,尤其是那些发生在我们身边的变化。尽管自摩西时代以来许多戏剧性的变化已经发生,但只有少数异端分子费心思考过文明从一个阶段过渡到另一个阶段的方式。
\switchcolumn
Human cultures have blind spots. We have no vocabulary to describeparadigm changes in the largest boundaries of life, especially those happening around us. Notwithstanding the many dramatic changes that have unfolded since the time of Moses, only a few heretics have bothered to think about how the transitions from one phase of civilization to another actually unfold.

\switchcolumn*
它们是如何被触发的?有什么共同点?有哪些模式可以帮助你告诉它们何时开始知道何时结束?英国或美国什么时候会灭亡?这些问题你很难找到常规答案。
\switchcolumn
How are they triggered? What do they have in common? What patterns can help you tell when they begin and know when they are over? When will Great Britain or the United States come to an end? These are questions for which you would be hard-pressed to find conventional answers.

\end{paracol}

\subsection{对前瞻的禁忌}
\begin{paracol}{2}
看到一个现有的系统“之外”的东西,就像是一个舞台工作人员试图强迫与戏剧中的角色对话一样。这违反了保持系统运作的一个惯例。每个社会秩序都包含了其关键禁忌中的概念,即生活在其中的人不应该想到它将如何结束以及新的取代它的系统中会出现哪些规则。隐含在其中的是无论什么系统存在,它都是最后一个或唯一一个存在的系统。并不是说就是这么直白地陈述。几乎所有读过历史书的人都不会认为这种假设是现实的,如果这是清晰表述的话。尽管如此,这种惯例仍然统治着世界。每个社会系统,无论它紧紧地还是弱弱地掌握着权力,都假装它的规则永远不会被取代。它们是最后的话语。或许只有话语。原始的人们认为他们的组织生活方式是唯一的可能。更经济复杂的系统包含了对历史的感知,通常将自己置于历史的顶点。无论是中国皇帝朝廷中的官员,斯大林的克里姆林宫中的马克思主义者,还是华盛顿众议院的议员,那些掌权的人们要么没有想过历史,要么将自己置于历史顶点,相对于过去的每个人,处于更高的位置,并成为未来任何事物的先锋。
\switchcolumn
To see "outside" an existing system is like being a stagehand trying to force a dialogue with a character in a play. It breaches a convention that helps keep the system functioning. Every social order incorporates among its key taboos the notion that people living in it should not think about how it will end and what rules may prevail in the new system that takes its place. Implicitly, whatever system exists is the last or the only system that will ever exist. Not that this is so baldly stated. Few who have ever read a history book would find such an assumption realistic if it was articulated. Nonetheless, that is the convention that rules the world. Every social system, however strongly or weakly it clings to power, pretends that its rules will never be superseded. They are the last word. Or perhaps the only word. Primitives assume that theirs is the only possible way oforganizing life. More economically complicated systems that incorporate a sense of history usually place themselves at its apex. Whether they are Chinese mandarins in the court of the emperor, the Marxist nomenklatura in Stalin's Kremlin, or members of the House of Representatives in Washington, the powers-that-be either imagine no history at all or place themselves at the pinnacle of history, in a superior position compared to everyone who came before, and the vanguard of anything to come.  

\switchcolumn*
这是因为实际的原因。系统接近结束的情况越明显,人们就越不愿意遵守它的法律。因此,任何社会组织都会倾向于阻止或淡化预测其灭亡的分析。这可以确保历史上的伟大转变很少在其发生时被发现。如果你对未来一无所知,你可以放心,戏剧性的变化既不会受到欢迎,也不会被传统思想家广泛宣传。
\switchcolumn
This is true for practical reasons. The more apparent it is that a system is nearing an end, the more reluctant people will be to adhere to its laws. Any social organization will therefore tend to discourage or play down analyses that anticipate its demise. This alone helps ensure that history's great transitions are seldom spotted as they happen. If you know nothing else about the future, you can rest assured that dramatic changes will be neither welcomed nor advertised by conventional thinkers.
\switchcolumn*
你不能依赖传统的信息来源为你提供关于世界正在如何改变以及为什么改变的客观及时的警告。如果你想了解当前正在进行的伟大转变,你别无选择,只能自己想办法。
\switchcolumn
You cannot depend upon conventional information sources to give you an objective and timely warning about how the world is changing and why. If you wish to understand the great transition now under way, you have little choice but to figure it out for yourself.  
\end{paracol}

\subsection{超越表面现象}
\begin{paracol}{2}
这意味着超越显而易见的东西。纪录显示,即使是回顾起来无可置疑地真实的转变,可能在几十年甚至几个世纪后才被承认。考虑罗马的陨落。这可能是基督教时代的前一千年中最重要的历史发展。然而,在罗马倒塌之后很长一段时间,仍然存在着它仍然生存的虚构,就像列宁的防腐处理过的尸体一样。任何依赖于官员的假装来了解“新闻”的人,直到那些信息不再重要之后很久才会知道罗马的倒塌。
\switchcolumn
This means looking beyond the obvious. The record shows that even transitions that are undeniably real in retrospect may not be acknowledged for decades or even centuries after they happen. Consider the fall of Rome. It was probably the most important historic development in the first millennium of the Christian era. Yet long after Rome's demise, the fiction that it survived was held out to public view, like Lenin's embalmed corpse. No one who depended upon the pretenses of officials for his understanding of the "news" would have learned that Rome had fallen until long after that information ceased to matter.  
\switchcolumn*
这不仅仅是古代通讯方式的不足所导致的。即使CNN在9月476日奇迹般地营业,并播放其录像带,结果也会大致相同。那是当时西部最后的罗马皇帝罗慕路斯·奥古斯图卢斯在拉文纳被俘的时候,他被强制性地退休到坎帕尼亚的一座别墅里,并领取抚恤金。即使沃尔夫·布里策尔与短视频记录在476年报道新闻,也不太可能有人敢说这些事件标志着罗马帝国的终结。 当然,这正是后来的历史学家所说的事情。
\switchcolumn
The reason was not merely the inadequacy of communications in the ancient world. The outcome would have been much the same had CNN miraculously been in business, running its videotape in September 476. That is when the last Roman emperor in the West, Romulus Augustulus, was captured in Ravenna and forcibly retired to a villa in Campania on a pension. Even if Wolfe Blitzer had been there with minicams recording the news in 476, it is unlikely that he or anyone else would have dared to characterize those events as marking the end of the Roman Empire. That, of course, is exactly what latter historians said happened.
\switchcolumn*
CNN编辑们可能不会批准"今晚罗马倒了"这样的头条新闻。那些当权者否认罗马已经倒下。 "新闻"的推销员很少是争议的党派,这些争议可能会破坏自己的利润。他们可能是党派性的。他们甚至可能非常令人震惊。但是,他们很少报道会让订户取消订阅并逃到山上的结论。因此,即使从技术上可能,也很少有人报道罗马的倒塌。专家会站出来说,说罗马倒塌是荒谬的。否则会对商业不利,甚至可能对报道者的健康有害。五世纪晚期的罗马当权者是野蛮人,他们否认罗马已经倒塌。
\switchcolumn
CNN editors probably would not have approved a headline story saying "Rome fell this evening." The powers-that-be denied that Rome had fallen. Peddlers of "news" seldom are partisans of controversy in ways that would undermine their own profits. They may be partisan. They may even be outrageously so. But they seldom report conclusions that would convince subscribers to cancel their subscriptions and head for the hills. Which is why few would have reported the fall of Rome even if it had been technologically possible. Experts would have come forth to say that it was ridiculous to speak of Rome falling. To have said otherwise would have been bad for business and, perhaps, bad for the health of those doing the reporting. The powers in late-fifth-century Rome were barbarians, and they denied that Rome had fallen.
\switchcolumn*
但问题不仅仅在于当局说,“不要报道这个,否则我们会杀了你。”问题的一部分在于,到了第五世纪后期,罗马已经如此堕落,以至于“倒下”真正逃脱了大多数生活于其中的人的注意。事实上,在一代人之后,马塞利努斯伯爵首次提出“西方罗马帝国已经随着奥古斯图卢斯的倒塌而灭亡”。在许多年,也许是几个世纪之后,人们才共同承认西方的罗马帝国已不存在。当然,查理曼也认为自己是公元800年的合法罗马皇帝。
\switchcolumn
But it was not merely a case of authorities' saying, "Don't report this or we will kill you." Part ofthe problem was that Rome was already so degener- ate by the later decades of the fifth century that its "fall" genuinely eluded the notice of most people who lived through it. In fact, it was a generation later before Count Marcellinus first suggested that "The Western Roman Empire perished with this Augustulus." Many more decades passed, perhaps centuries, before there was a common acknowledgment that the Roman Empire in the West no longer existed. Certainly Charlemagne believed that he was a legitimate Roman emperor in the year 800.  
\switchcolumn*
重点不在于查理曼和476年以后所有那些按照传统思考罗马帝国的人都是傻瓜。相反,社会发展的表征经常是模糊的。当主导机构的权力混入其中,加强一个基本上基于虚假的便利结论时,只有一个有坚强意志和强烈观点的人敢于反驳。如果你试图站在五世纪末期罗马人的立场上,很容易想象得出得到一个令人鼓舞的结论——什么都没有改变。这当然是乐观的结论。认为情况不同可能会令人感到恐惧。既然有一个令人放心的结论在手,为什么要得出一个令人恐惧的结论呢?
\switchcolumn
The point is not that Charlemagne and all who thought in conventional terms about the Roman Empire after 476 were fools. To the contrary. The characterization of social developments is frequently ambiguous. When the power of predominant institutions is brought into the bargain to reinforce a convenient conclusion, even one based largely on pretense, only someone of strong character and strong opinions would dare contradict it. If you try to put yourself in the position of a Roman of the late fifth century, it is easy to imagine how tempting it would have been to conclude that nothing had changed. That certainly was the optimistic conclusion. To have thought otherwise might have been frightening. And why come to a frightening conclusion when a reassuring one was at hand? 

\switchcolumn*
毕竟,可以提出一个观点——生意将会照常进行。这在过去是这样。罗马军队,特别是边境的守卫队,已经成为野蛮化的几个世纪了。到了第三个世纪,军队宣布一位新皇帝已经成为一种常规做法。到了第四个世纪,甚至高级军官也已经野蛮化,并经常不识字。在罗穆卢斯·奥古斯都被废黜之前,已经有许多皇帝被暴力推翻。在这个混乱的时期,他的离去可能对他的同时代人来说并没有什么区别。而他被开除,有养老金安排。即使在他被谋杀之前,他获得的养老金的事实也是一个安全的保障,表明体制存活了下来。对于一个乐观主义者来说,罗马帝国奥多阿塞尔推翻罗穆卢斯·奥古斯都,团结起来,而不是被摧毁了。奥多阿塞尔是阿提拉助手埃代孔的儿子,是一个聪明人。他没有宣称自己是皇帝。相反,他召集了参议院,并说服了过于容易受影响的成员,要把皇帝的地位和整个帝国的主权提供给在遥远的拜占庭的东部皇帝泽诺。奥多阿塞尔只是泽诺的执事,统治意大利。
\switchcolumn
After all, a case could have been made that business would continue as usual. It had in the past. The Roman army, and particularly the frontier garrisons, had been barbarized for centuries. By the third century, it had become regular practice for the army to proclaim a new emperor. By the fourth century, even officers were Germanized and frequently illiterate. There had been many violent overthrows of emperors before Romulus Augustulus was removed from the throne. His departure might have seemed no different to his contemporaries than many other upheavals in a chaotic time. And he was sent packing with a pension. The very fact that he received a pension, even for a brief period before he was murdered, was a reassurance that the system survived. To an optimist, Odoacer, who deposed Romulus Augustulus, reunified rather than destroyed the empire. A son of Attila's sidekick Edecon, Odoacer was a clever man. He did not proclaim himself emperor. Instead, he convened the Senate and prevailed upon its too-suggestible members that they offer the emperorship and thus sovereignty over the whole empire to Zeno, the Eastern emperor in faraway Byzantium. Odoacer was merely to be Zeno's patricius to govern Italy.  
\switchcolumn*
正如威尔·杜兰特在《文明的故事》中写的那样,这些变化似乎并不是“罗马的陨落”,而只是“国家形势表面上微不足道的变化”。当罗马崩溃时,奥多阿塞尔说罗马仍然存在。他和几乎所有人都热衷于假装什么都没有改变。他们知道,“罗马的荣耀”比正在取而代之的野蛮更好。即使是野蛮人也这么认为。正如C·W·普雷维特-奥顿在《剑桥中世纪简史》中所写的,五世纪末,当“以野蛮德意志国王取代皇帝时”,这是一个“持久的虚假”。
\switchcolumn
As Will Durant wrote in \emph{The Story of Civilization}, these changes did not appear to be the "fall of Rome" but merely "negligible shifts on the surface of the national scene." When Rome fell, Odoacer said that Rome endured. He, along with almost everyone else, was keen to pretend that nothing had changed. They knew that "the glory that was Rome" was far better than the barbarism that was taking its place. Even the barbarians thought so. As C. W. Previte-Orton wrote in \emph{The Shorter Cambridge Medieval History}, the end of the fifth century, when "the Emperors had been replaced by barbaric German kings," was a time of "persistent make-believe".  
\end{paracol}

\subsection{"Persistent Make-Believe"}
\begin{paracol}{2}
  这种“虚假”包括保留旧体制的外貌,而其实质则被“野蛮化”了。最后一位皇帝被野蛮的“副官”代替时,旧的政府形式保持不变。参议院仍然开会。“执政府和其他高职继续存在,并由杰出的罗马人担任。”每年仍然会提名执政官。“罗马的文职行政管理能力仍然完好无损。”事实上,在某些方面,它一直保持完好无损,直到十世纪末封建主义的诞生。在公共场合,仍然使用旧的皇帝标志。基督教仍然是国教。野蛮人仍然假装向位于君士坦丁堡的东部皇帝和罗马法的传统效忠。事实上,在杜兰特的话中,“在西方,伟大的帝国不存在了”。
  \switchcolumn
  This "make-believe" involved the preservation of the facade of the old system, even as its essence was "deformed by barbarism." The old forms of government remained the same when the last emperor was replaced by a barbarian "lieutenant." The Senate still met. "The praetorian prefecture and other high offices continued, and were held by eminent Romans." Consuls were still nominated for a year. "The Roman civil administration survived intact." Indeed, in some ways it remained intact until the birth of feudalism at the end of the tenth century. On public occasions, the old imperial insignia was still employed. Christianity was still the state religion. The barbarians still pretended to owe fealty to the Eastern emperor in Constantinople, and to the traditions of Roman law. In fact, in Durant's words, "in the West the great Empire was no more." 
\end{paracol}

  
\subsection{所以呢?} 
\begin{paracol}{2}
  罗马帝国的远古衰亡例子对于你思考现今世界的情况而言是相关的。大多数有关未来的书籍实际上都是有关现在的书籍。我们通过使这本书首先成为一本关于过去的书籍来解决这个问题。我们认为,如果我们通过从过去的真实例子中阐明有关暴力逻辑的重要大政治点,你将更有可能得出有关未来的更好视角。历史是一位惊人的老师。它所讲述的故事比我们能够编造的故事更有趣。其中许多有趣的故事与罗马帝国的衰亡有关。它们记录了可以与信息时代的未来相关的重要经验教训。
  \switchcolumn 
  The faraway example of the fall of Rome is relevant for a number of reasons as you contemplate conditions in the world today. Most books about the future are really books about the present. We have sought to remedy that defect by making this book about the future first of all a book about the past. We think that you are likely to draw a better perspective about what the future has in store if we illustrate important megapolitical points about the logic of violence with real examples from the past. History is an amazing teacher. The stories it has to tell are more interesting than any we could make up. And many of the more interesting relate to the fall of Rome. They document important lessons that could be relevant to your future in the Information Age.  
  \switchcolumn*
  首先,罗马帝国的衰亡是历史上更生动的例子之一,展示了在大规模政府崩溃时会发生什么。公元1000年的转型同样牵涉到中央权威的崩溃,而且这种崩溃方式会增加经济活动的复杂性和范围。15世纪末的火药革命带来了机构上的重大变化,这些变化倾向于扩大而不是缩小治理的范围。今天,在西方,大政治条件首次在一千年内削弱和摧毁了政府、企业集团、工会以及许多其他进行大规模运作的机构。
  \switchcolumn
  First of all, the fall of Rome is one of history's more vivid examples of what happened in a major transition when the scale of government was collapsing. The transitions of the year 1000 also jnvolved the collapse of central authority, and did so in a way that increased the complexity and scope of economic activity. The Gunpowder Revolution at the end of the fifteenth century involved major changes in institutions that tended to raise rather than shrink the scale of governance. Today, for the first time in a thousand years, megapolitical conditions in the West are undermining and destroying governments, corporate conglomerates, labor unions, and many other institutions that operate on a large scale.  
  \switchcolumn*
  当然,与现今信息时代的来临时占主导地位的原因有所不同,罗马帝国衰落的原因也是如此。罗马的衰落部分原因在于,其疆界扩张超出了维持暴力经济的规模。驻扎帝国遥远边界的成本超过了古代农业经济能够支持的经济优势。为了筹资军事行动所需的税收和监管负担超过了经济承载能力。腐败变得普遍存在。历史学家Ramsay MacMullen所记录的大部分军事指挥官的努力都是为追求“非法利润而奋斗的”。他们通过向人口敲诈勒索盈利。这就是四世纪观察家Synesius所描述的“和平时期的战争,这几乎比野蛮人战争更糟糕的战争,它源于军队的纪律性和军官的贪婪”。
  \switchcolumn
  Of course, the collapse in the scale of governance at the end of the Roman Empire had very different causes from those prevailing now, at the advent of the Information Age. Part of the reason that Rome fell is simply that it had expanded beyond the scale at which the economies of violence could be maintained. The cost of garrisoning the empire's far-flung borders exceeded the economic advantages that an ancient agricultural economy could support. The burden of taxation and regulation required to finance the military effort rose to exceed the carrying capacity of the economy. Corruption became endemic. A large part .of the effort of military commanders, as historian Ramsay MacMullen has documented, was devoted to pursuit of "illicit profits of their command." This they pursued by shaking down the population, what the fourth-century observer Synesius described as "the peace-time war, one almost worse than the barbarian war and arising from the military's indiscipline and the officers' greed."  
  \switchcolumn*
  罗马文明崩溃的另一个重要因素是安敦尼瘟疫导致的人口赤字。许多地区罗马人口的崩溃显然导致了经济和军事实力的削弱。至少目前为止,没有发生类似的事情。从更长远的角度来看,新“瘟疫”的肆虐可能会加剧新千年技术倒退的挑战。 20世纪人类规模前所未有的增长创造了一个诱人的目标,那就是迅速突变的微生物寄生虫。对埃博拉病毒或类似病毒入侵都市人口的担忧可能是合理的。但现在不是考虑人类与疾病共同进化的地方。尽管这是一个有趣的话题,但是我们在这里的论点并不是关于为什么罗马崩溃了,甚至也不涉及今天世界是否容易受到导致罗马衰落的相同影响的问题。它与不同的事情有关 -- 也就是说,历史上伟大转变的感知方式,或者更确切地说,它们发生时的误解。
  \switchcolumn
  Another important contributing factor to Rome's collapse was a demographic deficit caused by the Antonine plagues. The collapse of the Roman population in many areas obviously contributed to economic and military weakness. Nothing of that kind has happened today, at least not yet. Taking a longer view, perhaps, the scourge of new "plagues" will compound the challenges oftechnological devolution in the new millennium. The unprecedented bulge in human population in the twentieth century creates a tempting target for rapidly mutating microparasites. Fears about the Ebola virus, or something like it, invading metropolitan populations may be well founded. But this is not the place to consider the coevolution of humans and diseases. As interesting a topic as that is, our argument at this juncture is not about why Rome fell, or even about whether the world today is vulnerable to some of the same influences that contributed to Roman decline. It is about something different -- namely, the way that history's great transformations are perceived, or rather, misperceived as they happen.

  \switchcolumn*
  人们总是在某种程度上保守,带有一点“c”,这意味着不愿意考虑消解尊敬的社会传统、推翻被接受的制度,以及违反他们获得指引的法律和价值观。几乎没有人愿意想象,气候、技术或其他一些变量的显然微小的变化可能会在某种程度上导致与他们父辈的世界断绝联系。罗马人不愿意承认正在发生的变化。我们也是如此。
  \switchcolumn
  People are always and everywhere to some degree conservative, with a small "c." That implies a reluctance to think in terms of dissolving venerable social conventions, overturning the accepted institutions, and defying the laws and values from which they drew their bearings. Few are inclined to imagine that apparently minor changes in climate or technology or some other variable can somehow be responsible for severing connections to the world of their fathers. The Romans were reluctant to acknowledge the changes unfolding around them. So are we.
  \switchcolumn*
  但是,认识到或不认识到,我们正在经历历史季节的变化,这种变化将是彻底的,将不可避免地改变整个社会。事实上,这种变化将如此深刻,以至于理解它几乎需要将几乎一切都视为理所当然。几乎每一天都会邀请您相信即将到来的信息社会将非常类似于您成长的工业社会。我们怀疑这一点。微处理将溶解砖中的灰泥。它将如此深刻地改变暴力的逻辑,以至于它必然改变人们组织其生计和自卫的方式。然而,趋势将是淡化这些变化的必然性,或者争论它们的可取性,仿佛工业机构能够决定历史的演变方式。
  \switchcolumn
  Yet recognize it or not, we are living through a change of historical season, a transformation in the way people organize their livelihoods and defend themselves that is so far-reaching that it will inevitably transform the whole of society. The change will be so profound, in fact, that to understand it will require taking almost nothing for granted. You will be invited at almost every tum to believe that the coming Information Societies will be very like the industrial society you grew up in. We doubt it. Microprocessing will dissolve the mortar in the bricks. It will so profoundly alter the logic of violence that it will inevitably change the way people organize their livelihoods and defend themselves. Yet the tendency will be to downplay the inevitability of these changes, or to argue about their desirability as if it were within the fiat of industrial institutions to determine how history evolves.

  \end{paracol}

\subsection{The Grand Illusion}
\begin{paracol}{2}  
很多方面比我们更了解的作者仍然会误导您对未来的思考,因为他们在研究社会运作方面过于肤浅。例如,大卫·克莱因和丹尼尔·伯斯汀写了一本名为《公路勇士:信息高速公路上的梦想与噩梦》的良心力作。它充满了令人钦佩的细节,但这些细节中有很多是在争论一种错觉,即“公民可以共同有意识地塑造周围的经济和自然过程”。虽然这可能不明显,但这相当于说如果每个人都重新致力于骑士精神,封建主义可能会存活下来。15世纪晚期的法院中没有任何人会反对这种情感。实际上,这将是异端邪说。但它也是完全具有误导性的,它是蛇试图将未来适应其旧皮肤的一个例子。
\switchcolumn
Authors who are in many ways better informed than we are will nevertheless lead you astray in thinking about the future because they are far too superficial in examining how societies work. For example, David Kline and Daniel Burstein have written a well-researched volume entitled \emph{Road Warriors: Dreams and Nightmares Along the Information Highway}. It is full of admirable detail, but much of this detail is marshaled in arguing an illusion, the idea "that citizens can act together, consciously, to shape the spontaneous economic and natural processes going on around them." Although it may not be obvious, this is equivalent to saying that feudalism might have survived if everyone had rededicated himself to chivalry. No one in a court of the late fifteenth century would have objected to such a sentiment. Indeed, it would have been heresy to do so. But it also would have been entirely misleading, an example of the snake trying to fit the future into its old skin.
\switchcolumn*
变革的基本原因确切地说是无法受到意识控制的。它们是改变暴力得分条件的因素。实际上,它们与任何明显的操纵手段相距甚远,甚至在一个充斥政治的世界里也不是政治策略的主题。从未有人在示威中喊着:“增加生产过程中的规模经济”。也从未有过要求:“发明一种武器系统,增加步兵的重要性”的横幅。也从未有任何候选人承诺“在保护暴力方面改变效率和规模的平衡”。这些口号是荒谬的,因为它们的目标超出了任何人有意识地影响的能力范围。然而,正如我们将探讨的那样,这些变量在更大程度上决定了世界的运作方式,而不仅仅是任何政治平台。
\switchcolumn
The basic causes of change are precisely those that are not subject to conscious control. They are the factors that alter the conditions under which violence pays. Indeed, they are so remote from any obvious means of manipulation that they are not even subjects of political maneuvering in a world saturated with politics. No one ever marched in a demonstration shouting, "Increase scale economies in the production process." No banner has ever demanded, "Invent a weapons system that increases the importance of the infantry." No candidate ever promised to "alter the balance between efficiency and magnitude in protection against violence." Such slogans would be ridiculous, precisely because their goals are beyond the capacity of anyone to consciously affect. Yet as we will explore, these variables determine how the world works to a far greater degree than any political platform.    
\switchcolumn*
如果你仔细思考,就应该明显,历史上重要的转变很少主要是由人类意愿驱动的。它们并不是因为人们对一种生活方式感到厌倦,突然就转换了。一个瞬间的反思就能说明这一点的原因。如果人们的想法和愿望是发生的唯一决定因素,那么历史上所有突然的变化都必须用无法连接到任何实际生活条件变化的狂躁情绪来解释。实际上,这从来没有发生过。只有在影响少数人的医疗问题的情况下,我们才会看到情绪的任意波动,这些波动似乎完全脱离了任何客观原因。
\switchcolumn
If you think about it carefully, it should be obvious that important transitions in history seldom are driven primarily by human wishes. They do not happen because people get fed up with one way of life and suddenly prefer another. A moment's reflection suggests why. If what people think and desire were the only determinants of what happens, then all the abrupt changes in history would have to be explained by wild mood swings unconnected to any change in the actual conditions of life. In fact, this never happens. Only in cases of medical problems affecting a few people do we see arbitrary fluctuations in mood that appear entirely divorced from any objective cause.
\switchcolumn*
通常情况下,大量的人不会突然一齐决定放弃他们的生活方式,只是因为他们发现这样做很有趣。没有一个采集者会说:“我厌倦了生活在史前时代,我宁愿过一个在农村村庄里的农民的生活。”行为和价值观决策的决定性转变往往是对实际生活条件的变化的反应。在这个意义上,至少人们总是现实的。如果他们的观点突然发生了变化,这可能表明他们已经面临了一些熟悉条件的改变:入侵、瘟疫、突然的气候转变或技术革命,这些都影响了他们的生计或他们保卫自己的能力。
\switchcolumn
As a rule, large numbers of people do not suddenly and all at once decide to abandon their way of life simply because they find it amusing to do so. No forager ever said, "I am tired of living in prehistoric times, I would prefer the life of a peasant in a farming village." Any decisive swing in patterns of behavior and values is invariably a response to an actual change in the conditions of life. In this sense, at least, people are always realistic. If their views do change abruptly, it probably indicates that they have been confronted by some departure from familiar conditions: an invasion, a plague, a sudden climatic shift, or a technological revolution that alters their livelihoods or their ability to defend themselves.
\switchcolumn*
决定性的历史变革远非人性的产物,往往会让大多数人渴望稳定的愿望不成反而陷入困惑。当变革发生时,通常会引起广泛的失望,尤其是那些失去收入或社会地位的人。你徒劳地在民意调查或其他情绪测量中寻找理解即将到来的大政治转变的方法。
\switchcolumn
Far from being the product of human desire, decisive historic changes more often than not confound the wish of most people for stability. When change occurs, it typically causes widespread disorientation, especially among those who lose income or social status. You will look in vain at public opinion polls or other measures of mood for an understanding of how the coming megapolitical transition is likely to unfold.
\end{paracol}

\section{没有远见的生活}

\begin{paracol}{2}
如果我们未能感知到正在我们身边发生的巨大变革,这部分原因在于我们不愿意看见。我们的采集祖先可能也同样顽固,但他们有更好的借口。一万年前没有人能够预见到农业革命的后果,实际上,没有人能够预见未来的很多事情,即使是下一个饭食从哪里得到。农业开始时,并没有过去事件的记录可以借助,以此来预见未来。甚至没有西方的时间概念,将时间划分为有序的单位,如秒、分钟、小时、日等,来测量年份。采集者生活在“永恒的现在”,没有日历,事实上,没有任何书面记录。他们没有科学,也没有任何其他的知识体系来理解因果关系,除了他们自己的直觉。当我们看向未来时,我们的原始祖先是盲目的,以圣经的隐喻为例,他们还没有吃过知识之果。
\switchcolumn
If we fail to perceive the great transition going on around us, it is partly because we do not desire to see. Our foraging forebears may have been just as obdurate, but they had a better excuse. No one ten thousand years ago could have foreseen the consequences of the Agricultural Revolution. Indeed, no one could have foreseen much of anything beyond where to find the next meal. When farming began, there was no record of past events from which to draw perspective on the future. There was not even a Western sense oftime divided into orderly units, like seconds, minutes, hours, days, and so on, to measure out the years. Foragers lived in the "eternal present," without calendars, and indeed, without written records at all. They had no science, and no other intellectual apparatus for understanding cause and effect beyond their own intuitions. When it came to looking ahead, our primeval ancestors were blind. To cite the biblical metaphor, they had not yet eaten of the fruit of knowledge.    
\end{paracol}

\subsection{从过去学习}
\begin{paracol}{2}
幸运的是,我们拥有更好的视角。过去的500代人给予我们比我们的祖先更好的分析能力。科学和数学帮助我们解开了许多自然的秘密,使我们对因果关系的理解接近于魔法,尤其是与早期采集者相比。由于高速计算机的发展,计算机算法对于像人类经济这样的复杂动态系统的运作方面带来了新的见解。政治经济学的艰苦发展,虽然远未达到完美,但已经磨练了对影响人类行为的因素的理解。其中一个重要的内容是认识到,在所有时间和地点,人们都倾向于响应激励。虽然并非总是像经济学家所想象的那样机械化,但他们确实做出了回应,成本和奖励很重要。当外部条件的变化提高某种行为的奖励或降低成本时,其他条件保持不变,将会导致该行为的增加。
\switchcolumn
Luckily, we have a better vantage point. The past five hundred generations have given us analytic capabilities that our forebears lacked. Science and mathematics have helped unlock many of nature's secrets, giving us an understanding of cause and effect that approaches the magical when compared to that of the early foragers. Computational algorithms developed as a result of high-speed computers have shed new insights on the workings of complex, dynamic systems like the human economy. The painstaking development of political economy itself, although it falls well short of perfection, has honed understanding of the factors informing human action. Important among these is the recognition that people at all times and places tend to respond to incentives. Not always as mechanically as economists imagine, but they do respond. Costs and rewards matter. Changes in external conditions that raise the rewards or lower the costs of certain behavior will lead to more of that behavior, other things being equal.
\end{paracol}

\subsection{刺激很重要}
\begin{paracol}{2}
人们往往会对成本和奖励做出反应,这是预测的一个重要因素。例如,你可以非常自信地说,如果你在纽约、墨西哥城或莫斯科的街上掉落了一张一百美元的钞票,很快就会有人来捡起来。这并不是看似微不足道的事情。这表明了那些聪明人说预测是不可能的是错误的。任何能够准确预见激励对行为的影响的预测,可能大体上是正确的。而且,预期成本和奖励的变化越大,预期的预测变化就越不微不足道。
\switchcolumn
The fact that people tend to respond to costs and rewards is an essential element of forecasting. You can say with a high degree of confidence that if you drop a hundred-dollar bill on the street, someone will soon pick it up, whether you are in New York, Mexico City, or Moscow. This is not as trivial as it seems. It shows why the clever people who say that forecasting is impossible are wrong. Any forecast that accurately anticipates the impact of incentives on behavior is likely to be broadly correct. And the greater the anticipated change in costs and rewards, the less trivial the implied forecast is likely to be.    
\switchcolumn*
所有最深远的预测可能都源于认识到政治变量转移的含义。暴力是行为的最终边界力量;因此如果你能理解暴力逻辑将如何变化,你就可以有用地预测未来人们将在哪里弃置或拾起一百美元的等值物品。
\switchcolumn
The most far-reaching forecasts of all are likely to arise from recognizing the implications of shifting megapolitical variables. Violence is the ultimate boundary force on behavior; thus, if you can understand how the logic of violence will change, you can usefully predict where people will be dropping or picking up the equivalent of one-hundred-dollar bills in the future.    
\switchcolumn*
这并不意味着你可以知道不可知的事情。我们不能告诉你如何预测中奖彩票号码或任何真正的随机事件。我们无法知道恐怖分子何时或是否会在曼哈顿引爆原子爆炸。或者是否有一颗小行星会撞击沙特阿拉伯。我们无法预测新的冰河时期、大规模火山爆发或出现新疾病。可能改变历史进程的不可知事件数量很大。但是,知道不可知的事情与推断已知事物的影响是截然不同的。如果你看到远处闪电,你可以非常有信心地预测一声雷鸣即将响起。预测大政治转型的后果涉及更长时间范围和不太确定的联系,但它是一种类似的锻炼。
\switchcolumn
We do not mean by this that you can know the unknowable. We cannot tell you how to forecast winning lottery numbers or any truly random event. We have no way of knowing when or whether a terrorist will detonate an atomic blast in Manhattan. Or if an asteroid will strike Saudi Arabia. We cannot predict the coming of a new Ice Age, a sudpen volcanic eruption, or the emergence of a new disease. The number of unknowable events that could alter the course of history is large. But knowing the unknowable is very different from drawing out the implications of what is already known. If you see a flash of lightning far away, you can forecast with a high degree of confidence that a thunderclap is due. Forecasting the consequences of megapolitical transitions involves much longer time frames, and less certain connections, but it is a similar kind of exercise.    
\switchcolumn*
导致变化的大政治催化剂通常在其后果显现之前就已经出现了。完全意义上的农业革命需要五千年才能浮出水面。从基于制造业和化学制品动力的农业社会向工业社会的转变进展更快。它需要数个世纪的时间。向信息社会的转变将更快地发生,可能在一生内。然而,即使缩短了历史,你也可以期望几十年过去,才能实现现有信息技术的完整大政治影响。
\switchcolumn
Megapolitical catalysts for change usually appear well before their consequences manifest themselves. It took five thousand years for the full implications of the Agricultural Revolution to come to the surface. The transition from an agricultural society to an industrial society based on manufacturing and chemical power unfolded more quickly. It took centuries. The transition to the Information Society will happen more rapidly still, probably within a lifetime. Yet even allowing for the foreshortening of history, you can expect decades to pass before the full megapolitical impact of existing information technology is realized.
\end{paracol}

\subsection{重要和次要的大政治转变}

\begin{paracol}{2}
这一章分析了城市政治转变的一些常见特征。在接下来的章节中,我们会更仔细地看待农业革命和从农场到工厂的转变,即前一次伟大的阶段性变革中的第二个转变。在文明的农业阶段中,有许多次较小的城市政治转变,如罗马帝国的崩溃和公元1000年封建革命。这标志着政府的兴衰和农业战利品从一组人手中流入另一组人手中。罗马帝国广阔庄园的所有者、欧洲黑暗时期的自耕农,以及封建时期的领主和农奴都在同一片田地上种植谷物。他们生活在非常不同的政府统治下,这是由于不同技术、气候波动以及疾病的破坏性影响的累积影响造成的。
\switchcolumn
This chapter analyzes some of the common features of megapolitical transitions. In following chapters we look more closely at the Agricultural Revolu- tion, and the transition from farm to factory, the second of the previous great phase changes. Within the agricultural stage of civilization there were many minor megapolitical transitions such as the fall of Rome and the feudal revolution of the year 1000. These marked the waxing and waning of the power equation as governments rose and fell and the spoils of farming passed from one set of hands to another. The owners of sprawling estates under the Roman Empire, yeoman farmers in the European Dark Ages, and the lords and serfs of the feudal period all ate grain from the same fields. They lived under very different governments because of the cumulative impact of different technologies, fluctuations in climate, and the disruptive influences of disease.
\switchcolumn*
我们的目的并不是彻底解释所有这些变化。虽然我们已经勾勒出了一些变化的方式,但我们并不假装完全做到了这一点。随着城市政治波动降低和提高实行权力的成本,政府的规模不断扩大和缩小。以下是一些摘要观点,您应该将这些观点牢记在心,以便更好地理解信息革命:
\switchcolumn
Our purpose is not to thoroughly explain all of these changes. We do not pretend to do so, although we have sketched out some illustrations of the way that changing megapolitical variables have altered the way that power was exercised in the past. Governments have grown and shrunk as megapolitical fluctuations have lowered and raised the costs of projecting power.Here are some summary points that you should keep in mind as you seek to understand the Information Revolution:
\switchcolumn*
\begin{itemize}
  \item 实际引起权力转变的转型在城市政治基础方面通常会提前很久发生。
  \item 当一个主要的转型开始时,收入通常会下降,往往是因为一个社会由于人口压力而使资源边缘化,从而使自身陷入危机。
  \item 看待系统的“外部”通常是禁忌的。人们经常看不到现有社会暴力逻辑的盲点,因此他们几乎总是盲目地看不到这种逻辑的潜在或明显的变化。大都市政治转变很少在发生之前被认识到。
  \item 主要的转变总是涉及文化革命,并且通常涉及到支持旧价值观和新价值观的人之间的冲突。
  \item 大都市政治转变从来都不受欢迎,因为它们会淘汰艰苦学得的智力资本并迷惑既有的道德准则。它们并非是应市民的要求而进行的,而是对外部条件变化的反应,这些变化改变了本地暴力逻辑。
  \item 转变为新的谋生方式或新的政府类型最初仅限于大都市政治催化剂起作用的地区。
  \item 在农业的早期阶段可能除外,过去的转变总是涉及到社会混乱和暴力加剧的时期,这是由于旧的系统失序和崩溃所导致的。
  \item 腐败、道德沦丧和低效似乎是一个系统的最后阶段的显著特征。
  \item 技术在塑造暴力逻辑方面的日益重要作用加速了历史,使得每个后续转型的适应时间都比以往任何时候都要短。
\end{itemize}
\switchcolumn
\begin{itemize}
  \item A shift in the megapolitical foundations of power normally unfolds far in advance of the actual revolutions in the use of power.
  \item Incomes are usually falling when a major transition begins, often because a society has rendered itself crisis-prone by marginalizing resources due to population pressures.
  \item Seeing "outside" of a system is usually taboo. People are frequently blind to the logic of violence in the existing society; therefore, they are almost always blind to changes in that logic, latent or overt. Megapolitical transitions are seldom recognized before they happen.
  \item Major transitions always involve a cultural revolution, and usually entail clashes between adherents ofthe old and new values.
  \item Megapolitical transitions are never popular, because they antiquate painstakingly acquired intellectual capital and confound established moral imperatives. They are not undertaken by popular demand, but in response to changes in the external conditions that alter the logic of violence in the local setting.
  \item Transitions to new ways of organizing livelihoods or new types of government are initially confined to those areas where the megapolitical catalysts are at work.
  \item With the possible exception ofthe early stages offarming, past transitions have always involved periods of social chaos and heightened violence due to disorientation and breakdown of the old system.
  \item Corruption, moral decline, and inefficiency appear to be signal features of the final stages of a system.
  \item The growing importance of technology in shaping the logic of violence has led to an acceleration of history, leaving each successive transition with less adaptive time than ever before.
\end{itemize}
\end{paracol}

\subsection{历史加速了}
\begin{paracol}{2}
随着事件的发展比之前的转型快很多倍,对于世界将如何改变的早期理解对你来说可能比对你祖先在相同时期的理解要有用得多。即使最初的农民神奇地理解了耕地的全部超级政治影响,这个信息也会实际上是无用的,因为成千上万年要过去,直到社会的新阶段的转型完成。
\switchcolumn
With events unfolding many times faster than during previous transformations, early understanding of how the world will change could turn out to be far more useful to you than it would have been to your ancestors at an equivalent juncture in the past. Even if the first farmers had miraculously understood the full megapolitical implications of tilling the earth, this information would have been practically useless because thousands of years were to pass before the transition to the new phase of society was complete.
\switchcolumn*
而今天则不同了。历史加速了。能正确预测新技术的超级政治影响的预测可能会更有用。如果我们能够像拥有当前知识的人一样深入发展信息社会的当前转型的影响,就像某些过去的转型到农场和工厂的影响一样,那么这个信息现在可能会有多倍的价值。简而言之,超级政治预测的行动地平线已经缩短到最有用的范围,即单个生命的时间跨度。
\switchcolumn
Not so today. History has sped up. Forecasts that correctly anticipate the megapolitical implications of new technology are likely to be far more useful today. If we can develop the implications of the current transition to the Information Society to the same extent that someone with current knowledge could have grasped the implications of past transi~ions to farm and factory, that information should be many times more valuable now. Put simply, the action horizon for megapolitical forecasts has shrunk to its most useful range, within the span of a single lifetime.
\switchcolumn*
我们对城市政治的研究正是试图做到这一点——揭示改变暴力实施边界的因素所带来的影响。这些城市政治因素主要决定了何时、在何地暴力才是有利可图的。它们还有助于说明收入的市场分布情况。正如经济历史学家弗雷德里克·莱恩所明确阐述的那样,暴力如何组织和控制在决定“稀缺资源的使用方式”方面发挥着重要作用。
\switchcolumn
Our study of megapolitics is an attempt to do just that --to draw out the implications of the changing factors that alter the boundaries where violence is exercised. These megapolitic~l factors largely determine when and where violence pays. They also help inform the market distribution of income. As economic historian Frederic Lane so clearly put it, how violence is organized and controlled plays a large role in determining "what uses are made of scarce resources."
\end{paracol}

\section{A CRASH COURSE IN MEGAPOLITICS}
\begin{paracol}{2}
《街头流血》和《大恐慌》中我们探讨了很多重要的隐含的城市政治因素,这些因素决定了历史的演变。揭示城市政治变革的影响的关键在于理解导致暴力使用革命的因素。这些变量可以被任意地分成四类:地形、气候、微生物和技术。
\switchcolumn
The concept of megapolitics is a powerful one. It helps illuminate some of the major mysteries of history: how governments rise and fall and what types of institutions they become; the timing and outcome of wars; patterns of economic prosperity and decline. By raising or lowering the costs and rewards of projecting power, megapolitics governs the ability of people to impose their will on others. This has been true from the earliest human societies onward. It still is. We explored many of the important hidden megapolitical factors that determine the evolution of history in \emph{Blood in the Streets} and \emph{The Great Reckoning}. The key to unlocking the implications of megapolitical change is understanding the factors that precipitate revolutions in the use of violence. These variables can be somewhat arbitrarily grouped into four categories: topography, climate, microbes, and technology.

\switchcolumn*
1.\textbf{地形}是至关重要的因素,如控制海上暴力从未像在陆地上那样垄断。没有政府的法律曾经被独家应用在那里,这对于理解暴力和保护的组织方式随着经济转移到网络空间的演变而演变的方式至关重要。

地形与气候结合在一起,在早期历史中起着重要作用。第一个国家出现在洪水平原上,被沙漠所包围,如在美索不达米亚和埃及,那里有大量用于灌溉的水源,但周围地区太干旱无法支持自给自足的农业。在这种情况下,个体农民面临着维护政治结构失败的极高成本。没有灌溉,只有在大规模下才能提供,庄稼就不会长。没有庄稼意味着饥饿。这些控制沙漠中的水源的人处于优势地位的条件导致了专制和富裕的政府。

正如我们在《大清算》中分析的那样,地形条件也在希腊古代独立农民的繁荣中起着重要作用,使该地区成为西方民主的摇篮。在地中海地区三千年前盛行的原始交通条件下,生活在离海岸线几英里以上的地方几乎不可能竞争古代世界中的高价值作物,如橄榄和葡萄。如果必须将油和酒运输到远距离的内陆地区,搬运成本是如此之高,以至于无法盈利销售。希腊的复杂海岸线意味着希腊的大多数地区距离海岸线不超过二十英里。这使得希腊农民在潜在的内陆竞争对手面前具有决定性的优势。

由于在高价值产品交易方面的优势,希腊农民可以从控制仅有的小块土地中获得高收入。这些高收入使他们能够购买昂贵的盔甲。古希腊著名的重装步兵是农民或地主,他们自费武装自己。希腊的重装步兵既装备精良又具有很强的动力,军事上是令人敬畏的,不能被忽视。地形条件是希腊民主的基础,就像不同种类的地形条件导致了埃及和其他地方的东方专制主义一样。
\switchcolumn
1. \textbf{Topography} is a crucial factor, as evidenced by the fact that control of violence on the open seas has never been monopolized as it has on land. No government's laws have ever exclusively applied there. This is a matter of the utmost importance in understanding how the organization of violence and protection will evolve as the economy migrates into cyberspace.

Topography, in conjunction with climate, had a major role to play in early history. The first states emerged on floodplains, surrounded by desert, such as in Mesopotamia and Egypt, where water for irrigation was plentiful but surrounding regions were too dry to support yeoman farming. Under such conditions, individual farmers faced a very high cost for failing to cooperate in maintaining the political structure. Without irrigation, which could be provided only on a large scale, crops would not grow. No crops meant starvation. The conditions that placed those who controlled the water in a desert in a position of strength made for despotic and rich government.

As we analyzed in \emph{The Great Reckoning}, topographic conditions also played a major role in the prosperity of yeoman farmers in ancient Greece, enabling that region to become the cradle of Western democracy. Given the primitive transportation conditions prevailing in the Mediterranean region three thousand years ago, it was all but impossible for persons living more than a few miles from the sea to compete in the production of high-value crops of the ancient world, olives and grapes. If the oil and the wine had to be transported any distance overland, the portage costs were so great that they could not be sold at a profit. The elaborate shoreline of the Greek littoral meant that most areas of Greece were no more than twenty miles from the sea. This gave a decisive advantage to Greek farmers over their potential competitors in landlocked areas.

Because of this advantage in trading high-value products, Greek farmers earned high incomes from control of only small parcels of land. These high incomes enabled them to purchase costly armor. The famous hoplites of ancient Greece were farmers or landlords who armed themselves at their own expense. Both well armed and well motivated, the Greek hoplites were militarily formidable and could not be ignored. Topographic conditions were the foundation of Greek democracy, just as those of a different kind gave rise to the Oriental despotisms of Eygpt and elsewhere.
\switchcolumn*
2. \textbf{气候}也有助于设置可以行使蛮力的边界。气候变化是从采集到农业的第一个重大转变的催化剂。大约一万三千年前,最后一次冰河时期的结束导致植被发生了根本性的变化。从近东地区开始,随着冰河时期的撤退,温度和降雨量逐渐上升,将森林扩展到以前是草原的地区。特别是,山毛榉森林的快速扩散严重限制了人类的饮食。如苏珊·阿林·格雷格在《采集者和农民》中所说:

\footnotesize{\emph{植被恢复为山毛榉森林对于当地的人、植物和动物种群可能有重大的后果。橡树林的树冠相对较开阔,能够让大量阳光照射到地面。丰富的灌木、草本和草类植被发展,植物多样性支持了各种野生动物。相比之下,山毛榉森林的树冠封闭且森林地面沉重阴暗。除了在叶片出现前的春季一批草本植物,只能看到能耐阴性的莎草、蕨类及少数草本植物。}}

随着时间的推移,茂密的森林蔓延至原本开阔的草原,并向东方草原延伸至整个欧洲。森林减少了大型动物的牧草生长区域,使得人类采集者的种群越来越难以维持生计。

在冰川时期的繁荣期,狩猎采集者的种群已经过度膨胀,无法支持自身在数量上的增长,许多大型哺乳动物物种被猎杀至灭绝。农业转型不是一种偏好的选择,而是在为弥补饮食的缺陷而迫不得已采取的措施。在更远北的地区,向北的趋势没有严重影响到大型哺乳动物的栖息地,因此继续采集和狩猎成为主要生产方式;在热带雨林中,全球变暖趋势并未产生减少食物供应的反效果,采集和狩猎也成为当地人们的主要生产方式。自从农业出现以来,气候变化更多地是由于气温变冷而不是变热引起的。

对过去社会气候变化动态的了解,对于未来气候继续波动时可能会证明有用。如果您知道平均温度下降一摄氏度会使种植季节缩短三至四周,会使作物的最大种植海拔高度减少五百英尺,那么您就了解到这些限制条件对未来行动的影响。您可以利用这些知识预测从谷物价格到土地价值的变化。您甚至可以基于相应的温度下降对实际收入和政治稳定性的影响得出明智的结论。在过去,由于长期几年的农作物歉收导致食品价格上涨和可支配收入下降,曾导致政府被推翻。


例如,不是巧合的是,现代时期最寒冷的17世纪是世界范围内的革命期。这种不幸的情况的一个隐藏的超级政治原因是异常寒冷的天气。事实上,气温如此之低,以至于凡尔赛宫“太阳国王”桌子上的葡萄酒都结冰了。缩短的生长季节导致了农作物收成的减少,破坏了实际收入。由于天气寒冷,繁荣逐渐转向接近1620年开始的全球长期的萧条。这证明了其极其不稳定的趋势。17世纪的经济危机导致了世界范围内的反叛,其中许多聚集在1648年,恰好是另一个更著名的反叛周期的正好两百年之前。在1640年到1650年期间,爆发了爱尔兰、苏格兰、英格兰、葡萄牙、加泰罗尼亚、法国、莫斯科、那不勒斯、西西里、巴西、波希米亚、乌克兰、奥地利、波兰、瑞典、荷兰和土耳其的反叛。甚至中国和日本也因骚动被席卷。

在十七世纪贸易缩减时期主导了重商主义,也许并非巧合。在世纪末发生可怕的饥荒时,经济封闭可能尤为明显。在18世纪,尤其是1750年之后,更温暖的气候和更高的农作物产量已经足以提高西欧的实际收入,以扩大对制成品的需求。采取了更多的自由市场政策。这导致了自我加强的经济增长爆发,因为工业在常常被描述为工业革命的更大规模扩张。技术和制造业产出的不断增长降低了天气对经济周期的影响。

然而,即使在像北美这样的富裕地区,突然变冷的天气降低实际收入的影响也不容小视。当现有机构配置已经耗尽潜力时,社会往往倾向于使自己陷入危机。过去,这种趋势常常表现为人口增长,将土地的承载能力推到极限。这在1000年前后的过渡期和15世纪末再次发生过。由于作物歉收和产量降低引起的实际收入下降,在两种情况下都对主导性的制度的破坏起了重要作用。今天,这种边缘化表现在消费信贷市场上。如果突然变冷的天气降低了作物产量并降低了可支配收入,这将导致债务违约和税收反叛。如过去所示,经济封闭和政治不稳定可能会导致。
\switchcolumn
2. \textbf{Climate} also helps set the boundaries within which brute force can be exercised. A climatic change was the catalyst for the first major transition from foraging to farming. The end of the last Ice Age, about thirteen thousand years ago, led to a radical alteration in vegetation. Beginning in the Near East, where the Ice Age retreated first, a gradual rise in temperature and rainfall spread forests into areas that had previously been grasslands. In particular, the rapid spread of beech forests seriously curtailed the human diet. As Susan Alling Gregg put it in \emph{Foragers and Farmers}:

\footnotesize{\emph{The establishment of beech forests must have had serious consequences for local human, plant and animal populations. The canopy of an oak forest is relatively open and allows large amounts of sunlight to reach the forest floor. An exuberant undergrowth of mixed shrubs, forbs, and grasses develops, and the diversity of plants supports a variety of wildlife. In contrast, the canopy of a beech forest is closed and the forest floor is heavily shaded. Other than a flush of spring annuals prior to the emergence of the leaves, only shade-tolerant sedges, ferns, and a few grasses are found.}}

Over time, dense forests encroached on the open plains, spreading throughout Europe into the Eastern steppes. The forests reduced the grazing area available to support large animals, making it increasingly difficult for the population of human foragers to support themselves.

The population of hunter-gatherers had swollen too greatly during the Ice Age prosperity to support itself on the dwindling herds of large mammals, many species of which were hunted to extinction. The transition to agriculture was not a choice of preference, but an improvisation adopted under duress to make up for shortfalls in the diet. Foraging continued to predominate in those areas farther north, where the warming trend had not adversely affected the habitats of large mammals, and in tropical rainforests, where the global warming trend did not have the perverse effect of reducing food supplies. Since the advent of farming, it has been far more common for changes to be precipitated by the cooling rather than the warming of the climate.

A modest understanding of the dynamics of climatic change in past societies could well prove useful in the event that climates continue to fluctuate. If you know that a drop of one degree Centigrade on average reduces the growing season by three to four weeks and shaves five hundred feet off the maximum elevation at which crops can be grown, then you know something about the boundary conditions that will confine people's action in the future. You can use this knowledge to forecast changes in everything from grain prices to land values. You may even be able to draw informed conclusions about the likely impact of falling temperatures on real incomes and political stability. In the past, governments have been overthrown when crop failures extending over several years raised food prices and shrank disposable incomes.

For example, it is no coincidence that the seventeenth century, the coldest in the modem period, was also a period of revolution worldwide. A hidden megapolitical cause of this unhappiness was sharply colder weather. It was so cold, in fact, that wine froze on the "Sun King's" table at Versailles. Shortened growing seasons produced crop failures and undermined real income. Because of the colder weather, prosperity began to wind down into a long global depression that began around 1620. It proved drastically destabilizing. The economic crisis of the seventeenth century led to the world being overwhelmed by rebellions, many clustering in 1648, exactly two hundred years before another and more famous cycle of rebellions. Between 1640 and 1650, there were rebellions in Ireland, Scotland, England, Portugal, Catalonia, France, Moscow, Naples, Sicily, Brazil, Bohemia, Ukraine, Austria, Poland, Sweden, the Netherlands, and Turkey. Even China and Japan were swept with unrest.

It may also be no coincidence that mercantilism predominated in the seventeenth century during a period of shrinking trade. Economic closure was perhaps most pronounced at the end of the century, "when a terrible famine occurred." By the eighteenth century, especially after 1750, warmer temperatures and higher crop yields had begun to raise real incomes in Western Europe sufficiently to expand demand for manufactured goods. More free-market policies were adopted. This led to a self-reinforcing burst of economic growth as industry expanded to a larger scale in what is commonly described as the Industrial Revolution. The growing importance of technology and manufactured output reduced the impact of the weather on economic cycles.

Even today, however, you should not underestimate the impact of suddenly colder weather in lowering real incomes --even in wealthy regions such as North America. There is a strong tendency for societies to render themselves crisis-prone when the existing configuration of institutions has exhausted its potential. In the past, this tendency has often been manifested by population increases that stretched the carrying capacity of land to the limit. This happened both before the transition of the year 1000 and again at the end of the fifteenth century. The plunge in real income caused by crop failures and lower yields played a significant role in both instances in destroying the predominant institutions. Today the marginalization is manifested in the consumer credit markets. If sharply colder weather reduced crop yields and lowered disposable incomes, this would lead to debt default as well as tax rebellions. If the past is a guide, both economic closure and political instability could result.
    
\switchcolumn*
3.\textbf{微生物}以各种方式传达着可能导致伤害或免疫免受伤害的力量,这在许多情况下决定了权力的行使方式。在我们探讨的《大清算》中,欧洲殖民者来自遍布疾病的定居农业社会,带来了相对免疫于麻疹等儿童感染病的免疫力。他们遇到的印第安人主要生活在人迹稀少的采集族群中。他们没有这样的免疫力,因此被大量歼灭。通常,最大的死亡率出现在白人甚至还没到达之前,因为第一批遇到欧洲人的印第安人带着感染病到达内陆。

此外,微生物存在也限制了权力的行使。在我们探讨了强大疟疾菌株在多个世纪内使热带非洲对白人的入侵无法成功的角色《血在街头》中。在19世纪中叶奎宁被发现之前,白人军队在疟疾地区无法生存,无论他们的武器多么优越。

人类与微生物之间的相互作用还产生了重要的人口统计学影响,改变了暴力的成本和收益。当由于流行病,饥荒或其他原因导致的死亡率波动较大时,战争中的相对死亡风险则会降低。从16世纪以后死亡率爆发的频率不断下降,有助于解释家庭规模的缩小,最终,与过去相比,今天对于战争中突然死亡的容忍度大大降低,这降低了帝国主义的容忍度,提高了在人口出生率低的社会中推行权力的成本。

当代社会由小家庭组成,即使是小规模的战斗死亡也让人难以容忍。相比之下,早期的现代社会对帝国主义所带来的死亡代价更加容忍。在本世纪之前,大多数父母会生下许多子女,其中一些被视为可能会因疾病而随机突然死亡。在早死很常见的年代里,未来的士兵和他们的家庭会更不抵制战场上的危险。
\switchcolumn
3. \textbf{Microbes} convey power to harm or immunity from harm in ways that have often determined how power was exercised. This was certainly the case in the European conquest of the New World, as we explored in \emph{The Great Reckoning}. European settlers, arriving from settled agricultural societies riddled with disease, brought with them relative immunity from childhood infections like measles. The Indians they encountered lived largely in thinly populated foraging bands. They possessed no such immunity and were decimated. Often, the greatest mortality occurred before white people even arrived, as Indians who first encountered Europeans on the coasts traveled inland with infections.

There are also microbiological barriers to the exercise of power. In \emph{Blood in the Streets}, we discussed the role that potent strains of malaria served in making tropical Africa impervious to invasion by white men for many centuries. Before the discovery of quinine in the mid-nineteenth century, white armies could not survive in malarial regions, however superior their weapons might have been.

The interaction between humans and microbes has also produced important demographic effects that altered the costs and rewards of violence. When fluctuations in mortality are high due to epidemic disease, famine, or other causes, the relative risk of mortality in warfare falls. The declining frequency of eruptions in death rates from the sixteenth century onward helps explain smaller family size and, ultimately, the far lower tolerance of sudden death in war today as compared to the past. This has had the effect of lowering the tolerance for imperialism and raising the costs of projecting power in societies with low birthrates.

Contemporary societies, comprising small families, tend to find even small numbers of battle deaths intolerable. By contrast, early modern societies were much more tolerant ofthe mortality costs associated with imperialism. Before this century, most parents gave birth to many children, some of whom were expected to die randomly and suddenly from disease. In an era when early death was commonplace, would-be soldiers and their familiesfaced the dangers of the battlefield with less resistance.

\switchcolumn*
4.\textbf{技术}在决定现代世纪投射力量的成本和回报方面发挥了迄今为止最大的作用。本书的论点认为它将继续发挥作用。技术具有几个关键的维度:
\begin{itemize}
  \item A. 进攻与防御的平衡。随着攻势能力的增强,在支配武器技术的进攻和防御之间能够取得平衡,有助于确定政治组织的规模。此时能够以远距离投射力量的能力占主导地位,管辖区 tend 类往往会整合,政府形成较大规模。在其他时候,比如现在,防御能力正在上升。这使得在核心地区之外投射力量的成本更高。管辖区 tend 类往往会分散,大政府分裂为小政府。
  \item B.平等与步兵的优势。决定市民平等程度的一个关键特征是武器技术的性质。那些相对便宜,可由非专业人员使用,并增强步兵的军事重要性的武器,往往会使力量平等化。当托马斯·杰斐逊写下“所有人被创造平等”时,他说的比几个世纪以前类似的说法要真实得多。手持狩猎步枪的农夫不仅装备像典型的布朗贝斯武装英国士兵一样,而且更好。持枪的农民可以在更远的距离上射击士兵,并比士兵更精准。这与中世纪完全不同,当时一个只拿着草叉的农民——他买不起更多——几乎没法指望抵挡重装骑士的攻击。当时没有人写下“所有人被创造平等”。那时,在最明显的意义上,人们是不平等的。单个骑士的力量远远超过数十个农民的力量之和。
  \item C. 暴力规模的优劣势。决定政府数量是少数大政府还是众多小政府的另一个变量是部署主流武器所需的组织规模。当暴力收益不断增加时,大规模运作政府更具有回报性;因此,政府往往变得更大。当少数人能够指挥有效的方式抵抗大团体的攻击时(中世纪就是这种情况),主权往往会分裂。小型独立机构行使了许多政府职能。正如我们在后面的章节中所探讨的那样,我们相信信息时代将带来网络士兵的曙光,他们将是分权的先驱。网络士兵不仅可以由国家部署,还可以由非常小的组织甚至个人部署。下一千年的战争将包括一些几乎是用计算机进行的无血战斗。
  \item D. 生产规模效益。决定最终权力是地方行使还是由远处行使的另一个重要因素是人们谋取生计的主导企业的规模。当至关重要的企业只有在拥有一个包括经营区域的大规模组织时才能发挥最佳效益,扩展以提供这种环境给企业的政府可以获得足够的额外财富来支付维护大型政治体系的成本。在这种情况下,整个世界经济通常在一个最高的世界强权主导下比其他所有国家都更有效,就像19世纪英国帝国一样。但有时跨度变量相结合会导致规模经济递减。如果维护大型经营区域的经济利益减少,之前从利用包括经营区域在内的优势而繁荣的大型政府可能开始分裂 - 即使进攻和防御之间的武器平衡仍然保持差不多。但有时跨度变量相结合会导致规模经济递减。如果维护大型经营区域的经济利益减少,之前从利用包括经营区域在内的优势而繁荣的大型政府可能开始分裂--即使进攻和防御之间的武器平衡仍然保持差不多。
  \item E. 技术分散。还有另一个因素有助于权力平衡,那就是关键技术的分散程度。当武器或生产工具能够有效地垄断或垄断时,它们往往会集中权力。即使是本质上是防御性的技术,例如机枪,也被证明是有效的攻击武器,这加剧了在这些武器不被广泛传播的时期,政府规模的上升。当欧洲大国在19世纪晚期垄断机枪时,他们能够将这些武器用于对边缘民族的攻击,从而大大扩展殖民帝国。后来,在20世纪,特别是在二战后机枪变得广泛可用后,它们被用于帮助摧毁帝国的力量。其他条件相等时,关键技术分散得越广泛,权力就越分散,政府的最优规模就越小。
\end{itemize}

\switchcolumn
4. \textbf{Technology} has played by far the largest role in determining the costs and rewards ofprojecting power during the modem centuries. The argument of this book presumes it will continue to do so. Technology has several crucial dimensions:
\begin{itemize}
  \item A. \emph{Balance between offense and defense.} The balance between the offense and the defense implied by prevailing weapons technology helps determine the scale of political organization. When offensive capabilities are rising, the ability to project power at a distance predominates, jurisdictions tend to consolidate, and governments form on a larger scale. At other times, like now, defensive capabilities are rising. This makes it more costly to project power outside of core areas. Jurisdictions tend to devolve, and big governments break down into smaller ones.
  \item B. \emph{Equality and the predominance of the infantry.} A key feature determining the degree of equality among citizens is the nature of weapons technology. Weapons that are relatively cheap, can be employed by nonprofessionals, and enhance the military importance of infantry tend to equalize power. When Thomas Jefferson wrote that "all men are created equal," he was saying something that was much more true than a similar statement would have seemed centuries earlier. A farmer with his hunting rifle was not only as well armed as the typical British soldier with his Brown Bess, he was better armed. The farmer with the rifle could shoot at the soldier from a greater distance, and with greater accuracy than the soldier could return fire. This was a distinctly different circumstance from the Middle Ages, when a farmer with a pitchfork --he could not have afforded more --could scarcely have hoped to stand against a heavily armed knight on horseback. No one was writing in 1276 that "all men are created equal." At that time, in the most manifestly important sense, men were not equal. A single knight exercised far more brute force than dozens of peasants put together.
  \item C. \emph{Advantages and disadvantages of scale in violence.} Another variable that helps determine whether there are a few large governments or many small ones is the scale of organization required to deploy the prevailing weapons. When there are increasing returns to violence, it is more rewarding to operate governments at a large scale; therefore governments tend to get bigger. When a small group can command effective means of resisting an assault by a large group, which was the case during the Middle Ages, sovereignty tends to fragment. Small, independent authorities exercise many of the functions of government. As we explore in a latter chapter, we believe that the Information Age will bring the dawn of cybersoldiers, who will be heralds of devolution. Cybersoldiers could be deployed not merely by nation-states but by very small organizations, and even by individuals. Wars of the next millennium will include some almost bloodless battles fought with computers.
  \item D. \emph{Economies ofscale in production.} Another important factor that weighs in the balance in determining whether ultimate power is exercised locally or from a distance is the scale of the predominant enterprises in which people gain their livelihoods. When crucial enterprises can function optimally only when they are organized on a large scale in an encompassing trading area, governments that expand to provide such a setting for enterprises under their protection may rake off enough additional wealth to pay the costs of maintaining a large political system. Under such conditions, the entire world economy usually functions more effectively where one supreme world power dominates all others, as the British Empire did in the nineteenth century. But sometimes megapolitical variables combine to produce falling economies of scale. If the economic benefits of maintaining a large trading area dwindle, larger governments that previously prospered from exploiting the benefits of encompassing trading areas may begin to break apart --even if the balance of weaponry between offense and defense otherwise remains much as it had been.
  \item E. \emph{Dispersal of technology.} Still another factor that contributes to the power equation is the degree of dispersal of key technologies. When weapons or tools of production can be effectively hoarded or monopolized, they tend to centralize power. Even technologies that are essentially defensive in character, like the machine gun, proved to be potent offensive weapons, that contributed to a rising scale of governance during the period when they were not widely dispersed. When the European powers enjoyed a monopoly on machine guns late in the nineteenth century, they were able to use those weapons against peoples at the periphery to dramatically expand colonial empires. Later, in the twentieth century, when machine guns became widely available, especially in the wake of World War II, they were deployed to help destroy the power of empires. Other things being equal, the more widely dispersed key technologies are, the more widely dispersed power will tend to be, and the smaller the optimum scale of government.
\end{itemize}  
\end{paracol}

\section{跨度政治变化的速度}
\begin{paracol}{2}
虽然技术是当今最重要的因素,而且显然越来越重要,但过去四个主要的“超级政治”因素都在决定权力可以在何种规模下行使方面发挥了作用。
\switchcolumn
While technology is by far the most important factor today, and apparently growing more so, all four major megapolitical factors have played a role in determining the scale at which power could be exercised in the past.  
\switchcolumn*
这不这些因素共同决定暴力的回报是否随着暴力规模的扩大而不断上升。这决定了火力强度与资源利用效率之间的重要性。它也强烈影响收入的市场分配。问题是,它们在未来将发挥什么样的作用?估算答案的关键在于认识到这些超级政治变量以戏剧性不同的速度突变。
\switchcolumn
Together, these factors determine whether the returns to violence continue to rise as violence is employed on a larger scale. This determines the importance of magnitude of firepower versus efficiency in employing resources. It also strongly influences the market distribution of income. The question is, What role will they command in the future? A key to estimating an answer lies in recognizing that these megapolitical variables mutate at dramatically different speeds. 
\switchcolumn*
地形几乎在整个记载历史时期都是固定的。除了涉及海港淤积,填海造陆或侵蚀等轻微局部影响外,地球的地形今天几乎与阿当和夏娃走出伊甸园时一样。并且在地球表面再次受到冰川时代的侵袭或其他剧烈事件扰乱之前,它很可能保持不变。在更深刻的层面上,地质时代似乎会在1000万到4000万年的时间内发生变化,可能是为了响应大型陨石撞击。有一天,可能会再次发生改变,从而显著地改变我们星球的地形。如果发生这种情况,您可以放心地假定棒球赛和板球赛季都将取消。
\switchcolumn
Topography has been almost fixed through the whole of recorded history. Except for minor local effects involving the silting of harbors, landfills, or erosion, the topography of the earth is almost the same today as it was when Adam and Eve straggled out of Eden. And it is likely to remain so until another lee Age recarves the landscapes of continents or some other drastic event disturbs the surface of the earth. At a more profound scale, geological ages seem to shift, perhaps in response to large meteorite strikes, over a period of 10 to 40 million years. Someday, there may again be geological upheavals that will alter significantly the topography of our planet. If that happens, you can safely assume that both the baseball and cricket seasons will be canceled.
\switchcolumn*
气候波动比地形波动活跃得多。在过去的一百万年中,气候变化负责地球表面大部分已知变化。在冰河时代期间,冰川挖出新的山谷,改变了河流的走向,通过降低海平面将岛屿从大陆上隔离或将它们连接在一起。气候波动在历史上发挥了重要作用,首先是在最后一次冰河时期结束后促成了农业革命,以及在寒冷和干旱时期破坏政权。
\switchcolumn
Climate fluctuates much more actively than topography. In the last million years, climatic change has been responsible for most of the known variation in the features of the earth's surface. During Ice Ages, glaciers gouged new valleys, altered the course of rivers, severed islands from continents or joined them together by lowering the sea level. Fluctuations in climate have played a significant role in history, first in precipitating the Agricultural Revolution after the close of the last Ice Age, and later in destablizing regimes during periods of colder temperatures and drought. 
\switchcolumn*
最近,人们对“全球变暖”的可能影响表示担忧。这些担忧不能轻率地被排除。然而,从更长远的视角来看,更有可能的风险是转向更冷的气候,而不是更暖和的气候。基于分析从海洋底部取出的核心样本中的氧同位素的温度波动研究表明,当前时期是超过200万年中第二个最热的时期。如果温度像17世纪那样变冷,那可能会导致超级政治动荡。有关全球变暖的当前警报在这个意义上可能是令人放心的。在这个意义上,只要它们是真实的,就可以保证温度将继续在过去三个世纪体验的异常温暖和相对温和的范围内波动。
\switchcolumn
Lately, there have been concerns over the possible impact of "global warming." These concerns cannot be dismissed out of hand. Yet, taking a longer perspective, the more likely risk appears to be a shift toward a colder, not a warmer climate. Study oftemperature fluctuations based upon analysis of oxygen isotopes in core samples taken from the ocean floor show that the current period is the second warmest in more than 2 million years. If temperatures were to turn colder, as they did in the seventeenth century, that might prove megapolitically destabilizing. Current alarms about global warming may in that sense be reassuring. To the extent that they are true, that assures that temperatures will continue to fluctuate within the abnormally warm and relatively benign range experienced for the past three centunes.
\switchcolumn*
微生物影响权力运作的变化率更像是一个谜题。微生物可以迅速发生变异。这在病毒中尤其明显。例如,普通感冒以一种几乎万花筒般的方式变异。尽管这些变异正在快速进行,但它们对于移动权力边界的影响远不及技术变革的突然。为什么?部分原因是自然环境中的平衡通常使感染微生物对宿主人群有益而不是摧毁。太具有致命性的病毒感染会在这个过程中自我灭绝。微生物寄生体的生存取决于它们对入侵的宿主不会过于迅速或一致地致命。
\switchcolumn
The rate of change in the influence of microbes on the exercise of power is more of a puzzle. Microbes can mutate very rapidly. This is especially true of viruses. The common cold, for example, mutates in an almost kaleidoscopic way. Yet although these mutations proceed apace, their impact in shifting the boundaries where power is exercised have been far less abrupt than technological change. Why? Part of the reason is that the normal balance of nature tends to make it beneficial for microbes to infect but not destroy host populations. Virulent infections that kill their hosts too readily tend to eradicate themselves in the process. The survival of microparasites depends upon their not being too rapidly or uniformly fatal to the hosts they invade.
\switchcolumn*
当然,这并不是说不能有致命的疾病暴发会改变权力的平衡。这种情况已经在历史上占据了重要地位。黑死病摧毁了欧亚大陆人口的大量部分,并给14世纪版本的国际经济以沉重打击。
\switchcolumn
That is not to say, of course, that there cannot be deadly eruptions of disease that alter the balance of power. Such episodes have figured prominently in history. The Black Death wiped out large fractions of the population of Eurasia and dealt a crushing blow to the fourteenth-century version of the international economy.
\end{paracol}

\subsection{可能发生的情况}

\begin{paracol}{2}
历史可以通过可能发生的事情以及发生的事情来理解。我们不知道为什么微生物寄生体不能在现代时期继续对人类社会造成伤害。例如,等同于疟疾但更具致命性的微生物障碍可能会阻止西方侵略边缘。第一个翻越非洲水域的葡萄牙冒险家可能感染了一种致命的反转录病毒,一种更具传染性的艾滋病病毒,这会在开辟前往亚洲的新贸易路线之前就阻止了这一过程。哥伦布和新世界的第一波移民也可能遭遇到像本土的人口一样受麻疹和其他西方儿童疾病影响的疾病。然而,没有发生任何这样的事情,这种巧合强调了历史有一个命运的直觉。
\switchcolumn
History can be understood in terms of what might have been as well as what was. We know of no reason that microparasites equid not have continued to play havoc with human society during the modem-period. For example, it is possible that microbiological barriers to the exercise of power, equivalent to malaria but more virulent, could have halted the Western invasion of the periphery in its tracks. The first intrepid Portuguese adventurers who sailed into African waters could have contracted a deadly retrovirus, a more communicable version of AIDS, that would have stopped the opening of the new trade route to Asia before it even began. Columbus, too, and the first waves of settlers in the New World might have encountered diseases that decimated them in the same way that indigenous local populations were affected by measles and other Western childhood diseases. Yet nothing of the kind happened, a coincidence that underlines the intuition that history has a destiny.
\switchcolumn*
微生物并不像现代时期那样阻碍了权力的巩固,而是有利于它。在边缘地区的西方部队和殖民者经常发现,使他们有能力施展影响的技术优势得到了微生物方面的加强。西方人武装了无形的生物武器,在童年时期经常摧毁当地人的疾病上则相对免疫。这给西方的航海者带来了与来自人口稀少地区的对手所缺乏的明显优势。随着事态的发展,疾病传播几乎完全呈单向性——从欧洲向外传播。没有等价的疾病从边缘地区向核心地区传播。
\switchcolumn
Microbes did far less to impede the consolidation of power in the modem period than to facilitate it. Western troops and colonists at the periphery often found that the technological advantages that allowed them to project power were underscored by microbiological ones. Westerners were armed with unseen biological weapons, their relative immunity to childhood diseases that frequently devastated native peoples. This gave voyagers from the West a distinct advantage that their antagonists from less densely settled regions lacked. As events unfolded, the disease transfer was almost entirely in one direction --from Europe outward. There was no equivalent transfer of disease in the other direction, from the periphery to the core. 
\switchcolumn*
作为一个可能的反例,有些人声称西方探险家从新大陆引入了梅毒到欧洲。这是可以争辩的。然而,如果真是这样,它并没有证明它成为行使权力的一个重要障碍。梅毒的主要影响是改变了西方的性道德观念。
\switchcolumn
As a possible counterexample, some have claimed that Western explorers imported syphilis from the New World to Europe. This is arguable. If true, however, it did not prove to be a significant barrier to the exercise of power. The major impact of syphilis was to shift sexual mores in the West.
\switchcolumn*
从十五世纪末到二十世纪最后一个季度,微生物对工业社会的影响越来越温和。尽管肺结核、小儿麻痹症和流感暴发所引起的个人悲剧和不幸,但现代时期没有出现新疾病,甚至不会接近安东宁瘟疫或黑死病的宏观政治影响。改善公共卫生和疫苗及解毒剂的出现,通常减少了现代时期传染性微生物的重要性,从而增加了技术在设定权力边界方面的相对重要性。
\switchcolumn
From the end of the fifteenth century to the last quarter of the twentieth, the impact of microbes on industrial society was ever more benign. Notwithstanding the personal tragedies and unhappiness caused by outbreaks of tuberculosis, polio, and flu, no new diseases emerged in the modern period that even approached the megapolitical impact of the Antonine plagues or the Black Death. Improving public health, and the advent of vaccinations and antidotes, generally reduced the importance of infectious microbes during the modern period, thereby increasing the relative importance oftechnology in setting the boundaries where power was exercised. 
\switchcolumn*
最近的艾滋病爆发和对外来病毒扩散的警报表明,微生物的作用在未来可能不完全是宏观政治上的同情。但新的瘟疫什么时候或是否会感染世界是不可预测的。微生物寄生物的爆发,例如病毒大流行,而不是气候或地形的剧烈变化,更有可能打破技术的宏观政治优势。
\switchcolumn
The recent emergence of AIDS and alarms over the potential spread of exotic viruses are hints that the role of microbes may not be altogether as megapolitically benign in the future as it has been over the past five hundred years. But when or whether a new plague will infect the world is unknowable. An eruption of microparasites, such as a viral pandemic, rather than drastic changes in climate or topography, would more likely disrupt the megapolitical predominance of technology.
\switchcolumn*
我们没有办法监测或预测地球上生命的性质如何彻底改变。我们交叉着手指,假定未来一千年的主要宏观政治变量将是技术而不是微生物学。如果运气继续站在人类一边,技术将继续成为主要的宏观政治变量。然而,对第一次巨大的超大城市转型——农业革命的回顾清晰地表明,情况并非总是如此。
\switchcolumn 
We have no way of monitoring or anticipating drastic departures from the nature of life on earth as we have known it. We cross our fingers and assume that the major megapolitical variables in the next millennium will be technological rather than microbiological. If luck continues to side with humanity, technology will continue to grow in prominence as the leading megapolitical variable. It was not always such, however, as a review ofthe first great megapolitical transformation, the Agricultural Revolution, clearly shows.
\end{paracol}  %导入第2章内容

\end{document}