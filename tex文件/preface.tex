\begin{center}
    \Huge\textbf{序言}
\end{center}

中世纪的人对意志没有信心,认为人类是容易受伤的、极其脆弱的,但他们尊重智力。他们认为只要认真思考,即使是人,也有能力回答关于上帝和宇宙的最深奥的问题。现代人崇拜意志,但他们对智力感到绝望。乌合之众,随机粒子的偏转,无意识偏见的影响:所有这些当代的陈词滥调,都在谈论智力的弱点,或者说也在谈论我们自己。

威廉·里斯·莫格勋爵和詹姆斯·戴尔·戴维森,并没有承诺也没有给出任何关于上帝和宇宙的答案。但是,他们对“大政治”的研究,对历史上各种力量的剖析,以及对不久的将来的一系列预测,是非比寻常的,甚至是反文化的,因为他们运用人类的理性,去思索那些我们被教导为“机遇”或“命运”的事情。在《主权个人》首次出版近四分之一个世纪之后,回顾过去,最容易做、也是我们周围的文化最喜欢做的事,就是挑剔他们的错误,这也算是一种自我安慰:那么费心去思考未来有什么意义呢?

当然,有一些事情他们没有想到:首先就是中国的崛起。在共产党的领导下,21世纪的中国创造了自己的信息时代,具有明显的民族主义、种族同化和深刻的国家主义特征。这可能是该书出版以来最大的“大政治”现象。仅举一个关键的例子,共产中国已经粉碎了香港这个城邦(城市国家),而里斯·莫格和戴维森曾将香港描述为“一种心智模式,一种会在信息时代繁荣昌盛的管辖区模式”。

从某个角度看,这是作者的一个盲点。从另一个角度看,中国的政治局委员一定是《主权个人》的热心读者。在不断重温列宁斯大林主义的同时,也积极地展望信息时代,只有这种特有的长期的警惕意识,才使得党的领导人能在本书分析的趋势中获得胜利。

这些趋势在今天依然适用:赢家通吃的经济、管辖权的竞争、大规模生产的转移,以及国家间的战争可能会过时。中国的崛起,与其说是对里斯·莫格和戴维森的反驳,倒更像是对他们所描述的利害关系的剧烈提升。

事实上,未来大政治的巨大冲突才刚刚开始。在技术层面上,\textbf{这场冲突的两极是:人工智能和加密技术}。人工智能展现出一种前景,能够最终解决经济学家所说“计算问题”(计划经济的关键)。理论上,它使集中控制整个经济成为可能。CCP最喜欢的技术,就是人工智能,这绝不是巧合。强加密技术在另外一极,它带来的远景是一个去中心化和个性化的世界。如果说人工智能是共产主义的,那么加密技术就是自由主义的。

未来可能就落在这两极之间。而要知道,我们今天采取的行动,会决定日后全局性的结果。在2020年,阅读《主权个人》,有助你认真思考,自己的行动将塑造什么样的未来;这是一次不容浪费的学习机会。
\\

\rightline{\emph{彼得·蒂尔}}
\rightline{\emph{2020年1月6日,洛杉矶}}
