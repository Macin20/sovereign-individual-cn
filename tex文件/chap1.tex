\chapter[人类社会第四个阶段]{2000年转折点:\\人类社会第四个阶段}
% 选项为页眉页脚的简写

\section{预言}

\begin{paracol}{2}[]
在耶稣降临后的第一个千年之交,世界并未像传说的那样毁灭。其后,在过去的一千年里,公元2000年的到来一直困扰着西方人的想象。神学家、传教士、诗人和预言家,都在张望着本世纪最后十年的结束,期待着历史性事件的发生。甚至如艾萨克·牛顿这样的权威人士都猜测世界将在公元2000年结束。米歇尔·德·诺斯特拉达姆斯的预言自1568年发表以来,每一代人都会阅读,他预言第三位敌基督将在1999年7月到来。瑞士心理学家卡尔·荣格是“集体无意识”的鉴赏家,他预见到新时代将在1997年诞生。这些预测可以轻易被嘲笑。那么像德意志银行证券(Deutsche Bank Securities)的爱德华·牙尔德尼博士这样的经济学家所做的理智预测也会被嘲笑,他认为公元2000年的午夜计算机故障将“打乱整个全球经济格局”。但无论你是否把Y2K计算机问题视为计算机程序员和信息技术顾问们搞起的无稽之谈,还是技术与预言想象的神秘示例之一,都无法否认千禧年前夕的情况比通常关于世界何去何从的病态怀疑更加令人兴奋。
  
\switchcolumn
The coming of the year 2000 has haunted the Western imagination for the past thousand years. Ever since the world failed to end at the turn of the first millennium after Christ, theologians, evangelists, poets, seers, and now, even computer programmers have looked to the end of this decade with an expectation that it would bring something momentous. No less an authority than Isaac Newton speculated that the world would end with the year 2000. Michel de Nostradamus, whose prophecies have been read by every generation since they were first published in 1568, forecast the coming of the Third Antichrist in July 1999. Swiss psychologist Carl Jung, connoisseur of the “collective unconscious,” envisioned the birth of a New Age in 1997. Such forecasts may easily be ridiculed. And so can the sober forecasts of economists, such as Dr. Edward Yardeni of Deutsche Bank Securities, who expects computer malfunctions on the millennial midnight to “disrupt the entire global economy.” But whether you view the Y2K computer problem as groundless hysteria ginned up by computer programmers and Information Technology consultants to stir up business, or as a mysterious instance of technology unfolding in concert with the prophetic imagination, there is no denying that circumstances at the eve of the millennium excite more than the usual morbid doubt about where the world is tending. 

\switchcolumn*
在过去的250年里,一种对未来的不安,给西方社会特有的乐观主义染上了阴影。各地的人们都犹豫不决,忧心忡忡。你可以从他们的脸上看到,从他们的谈话中听到;它反映在民意调查中,登记在选票箱中。就像在乌云密布、闪电到来之前,大气中看不见的离子的物理变化,已经预示了雷雨即将降临。如今,在千禧年的黄昏,空气中弥漫着变革的预感。一个又一个人,在一种行将结束的生活方式下,感受到时间就要燃尽。随着最后十年的过去,一个肃杀的世纪,同时也是人类成就辉煌的一千年,就此告以终章。所有的一切,都将因2000年的到来而画上句号。

\switchcolumn
A sense of disquiet about the future has begun to color the optimism so characteristic of Western societies for the past 250 years. People everywhere are hesitant and worried. You see it in their faces. Hear it in their conversation. See it reflected in polls and registered in the ballot box. Just as an invisible, physical change of ions in the atmosphere signals that a thunderstorm is imminent even before the clouds darken and lightning strikes, so now, in the twilight of the millennium, premonitions of change are in the air. One person after another, each in his own way, senses that time is running out on a dying way of life. As the decade expires, a murderous century expires with it, and also a glorious millennium of human accomplishment. All draw to a close with the year 2000. 

\switchcolumn*
我们相信,西方文明的现代阶段将以此结束。这本书讲述了为什么。像许多早期的著作一样,它试图暗示未来的模糊形状和尺寸。在这个意义上,我们的工作是启示性的,这是该词的原始含义。Apokalypsis 在希腊语中的意思是“揭示”。我们相信,历史的一个新阶段——信息时代——即将“揭示”。
\switchcolumn
We believe that the modern phase of Western civilization will end with it. This book tells why. Like many earlier works, it is all attempt to see into a glass darkly, to sketch out the vague shapes and dimensions of a future that is still to be. In that sense, we mean our work to be apocalyptic-in the original meaning of the word. Apokalypsis means “unveiling” in Greek. We believe that a new stage in history -- the Information Age -- is about to be “unveiled.” 
\end{paracol}

\section{人类社会的第四阶段}
\begin{paracol}{2}[]
本书的主旨是探讨一场新的权力革命,它将以20世纪民族国家的毁灭为代价,解放出个体。创新,以前所未有的方式改变了暴力的逻辑,并且正在革新未来的边界。如果我们的推断正确,你站在历史上最彻底的革命的门槛上。微处理将会以比大多数人想象的更快的速度颠覆和破坏民族国家,从而在这个过程中创造新的社会组织形式。这将远非一件容易的转变。
\switchcolumn
The theme of this book is the new revolution of power which is liberating individuals at the expense of the twentieth-century nation-state. Innovations that alter the logic of violence in unprecedented ways are transforming the boundaries within which the future must lie. If our deductions are correct, you stand at the threshold of the most sweeping revolution in history. Faster than all but a few now imagine, microprocessing will subvert and destroy the nation-state, creating new forms of social organization in the process. This will be far from an easy transformation.
\switchcolumn*
它将提出巨大的挑战,因为与过去相比,它将以令人难以置信的速度发生。从人类历史的最初起源到现在,经济生活只有三个基本阶段:(1)狩猎采集社会;(2)农业社会;(3)工业社会。现在,正在地平线上出现一些全新的东西,即第四个社会组织阶段:信息社会。
\switchcolumn
The challenge it will pose will be all the greater because it will happen with incredible speed compared with anything seen in the past. Through all of human history from its earliest beginnings until now, there have been only three basic stages of economic life: (1) hunting-and-gathering societies; (2) agricultural societies; and (3) industrial societies. Now, looming over the horizon, is something entirely new, the fourth stage of social organization: information societies.
\switchcolumn*
之前的每个社会阶段都与暴力的演化和控制的截然不同的阶段相对应。如我们详细解释的那样,信息社会有望大大降低暴力带来的回报,部分原因在于它们超越了地域。作家威廉·吉布森所说的虚拟现实是一种“共识性幻觉”,将远远超出恶棍们的掌控范围,就像想象力一样。在新千年,大规模控制暴力的优势将比法国大革命之前任何时候都低得多。这将产生深远的影响。其中之一将是犯罪率上升。当大规模组织暴力的回报下降时,小规模暴力的回报很可能会上升。暴力将变得更加随机和本地化。有组织的犯罪将规模扩大。我们解释了为什么会这样。
\switchcolumn
Each of the previous stages of society has corresponded with distinctly different phases in the evolution and control of violence. As we explain in detail, information societies promise to dramatically reduce the returns to violence, in part because they transcend locality. The virtual reality of cyberspace, what novelist William Gibson characterized as a “consensual hallucination,” will be as far beyond the reach of bullies as imagination can take it. In the new millennium, the advantage of controlling violence on a large scale will be far lower than it has been at any time since before the French Revolution. This will have profound consequences. One of these will be rising crime. When the payoff for organizing violence at a large scale tumbles, the payoff from violence at a smaller scale is likely to jump. Violence will become more random and localized. Organized crime will grow in scope. We explain why.
\switchcolumn*
暴力回报逐渐减少的另一个逻辑含义是政治的日渐式微,而正是政治成了最大罪犯活动的舞台。有很多证据显示,20世纪民族国家的公民信仰已经迅速削弱了。共产主义的垮台只是其中最为显著的例子而已。正如我们将详细探讨的那样,西方政府领导层道德崩溃、腐败蔓延的现象并非偶然事件,而是民族国家潜能已经枯竭的证明。即使是其中的领袖们也不再相信他们所说的那些陈词滥调,更别说其他人信了。
\switchcolumn
Another logical implication of falling returns to violence is the eclipse of politics, which is the stage for crime on the largest scale. There is much evidence that adherence to the civic myths of the twentieth-century nation-state is rapidly eroding. The death of Communism is merely the most striking example. As we explore in detail, the collapse of morality and growing corruption among leaders of Western governments are not random developments. They are evidence that the potential of the nation-state is exhausted. Even many of its leaders no longer believe the platitudes they mouth. Nor are they believed by others.
\end{paracol}

\subsection{历史将重演} 
\begin{paracol}{2}[]
这是一个与过去有惊人相似之处的情况。每当技术变革将旧形态与新经济动力分离时,道德标准会转变,人们开始对那些掌控旧机构的人产生越来越大的鄙视。在人们形成新的一致变革意识形态之前,这种普遍的反感往往会显露出来。这就像在15世纪晚期,当时的中世纪教会是封建体制中的主要机构一样。尽管人们普遍相信“神职职务的神圣性”,但无论是高层还是低层神职人员都备受蔑视,与今天人们对政治家和官僚的普遍态度不相上下。
\switchcolumn
This is a situation with striking parallels in the past. Whenever technological change has divorced the old forms from the new moving forces of the economy, moral standards shift, and people begin to treat those in command of the old institutions with growing disdain. This widespread revulsion often comes into evidence well before people develop a new coherent ideology of change. So it was in the late fifteenth century, when the medieval Church was the predominant institution of feudalism. Notwithstanding popular belief in “the sacredness of the sacerdotal office,” both the higher and lower ranks of clergy were held in the utmost contempt-not unlike the popular attitude toward politicians and bureaucrats today.
\switchcolumn*
我们认为,在15世纪末,当生活已经被有组织的宗教彻底浸透时,以及今天世界已经充满政治时,可以通过类比学习很多东西。在15世纪末,支持制度化宗教的成本已经达到了历史上的极限,就像今天支持政府的成本已经达到了一种丧失理智的极端。
\switchcolumn
We believe that much can be learned by analogy between the situation at the end of the fifteenth century, when life had become thoroughly saturated by organized religion, and the situation today, when the world has become saturated with politics. The costs of supporting institutionalized religion at the end of the fifteenth century had reached a historic extreme, much as the costs of supporting government have reached a senile extreme today.
\end{paracol}

\subsection{信息革命}
\begin{paracol}{2}[]
随着大型系统的崩溃而加速,系统性强制将逐渐减少,不再是塑造经济生活和收入分配的因素。效率将比权力的指令在社会制度组织中更加重要。这意味着,在信息时代,能够有效维护产权并提供司法管理,同时消耗较少资源的州和城市,将成为可行的主权国家,而在过去五个世纪中普遍不是如此。一个完全不受物理暴力威胁的经济活动领域将在网络空间中出现。最明显的好处将流向“认知精英”,他们将越来越在政治边界之外活动,并且在法兰克福、伦敦、纽约、布宜诺斯艾利斯、洛杉矶、东京和香港都感到自在。收入将在各个司法辖区内变得更加不平等,而在各个司法辖区之间则变得更加平等。
\switchcolumn
As the breakdown of large systems accelerates, systematic compulsion will recede as a factor shaping economic life and the distribution of income. Efficiency will become more important than the dictates of power in the organization of social institutions. This means that provinces and even cities that can effectively uphold property rights and provide for the administration of justice, while consuming few resources, will be viable sovereignties in the Information Age, as they generally have not been during the last five centuries. An entirely new realm of economic activity that is not hostage to physical violence will emerge in cyberspace. The most obvious benefits will flow to the “cognitive elite,” who will increasingly operate outside political boundaries. They are already equally at home in Frankfurt, London, New York, Buenos Aires, Los Angeles, Tokyo, and Hong Kong. Incomes will become more unequal within jurisdictions and more equal between them.
\switchcolumn*
《主权个人》探讨了这种革命性变化的社会和财务后果。我们的愿望是帮助您利用新时代的机遇,避免被其影响摧毁。如果我们所期望看到的一半成真,您将面临着几乎史无前例的巨大变革。
\switchcolumn
\emph{The Sovereign Individual} explores the social and financial consequences of this revolutionary change. Our desire is to help you to take advantage of the opportunities of the new age and avoid being destroyed by its impact. If only half of what we expect to see happens, you face change of a magnitude with few precedents in history.
\switchcolumn*
2000年的转型不仅将彻底改变世界经济的性质,而且速度将比以往任何时期的变革都更快。不像农业革命,信息革命不需要千年才能完成工作。不像工业革命,其影响不会持续几个世纪。信息革命将在一个人的寿命内发生。
\switchcolumn
The transformation of the year 2000 will not only revolutionize the character of the world economy, it will do so more rapidly than any previous phase change. Unlike the Agricultural Revolution, the Information Revolution will not take millennia to do its work. Unlike the Industrial Revolution, its impact will not be spread over centuries. The Information Revolution will happen within a lifetime. 
\switchcolumn*
此外,它几乎将同时发生在世界的各个角落。技术和经济创新将不再局限于全球小部分地区。这种转变将是普遍的。并且,它将涉及与过去断裂的程度如此深刻,以至于几乎会使类似古希腊人的早期农业人民所想象的神神秘领域“复活”。在信息社会成型之后,保护许多现代机构在新千年将会变得非常困难甚至不可能。当信息社会成形时,它们与工业社会的差异将会像Aeschylus的希腊世界与穴居人世界一样不同。
\switchcolumn
What is more, it will happen almost everywhere at once. Technical and economic innovations will no longer be confined to small portions of the globe. The transformation will be all but universal. And it will involve a break with the past so profound that it will almost bring to life the magical domain of the gods as imagined by the early agricultural peoples like the ancient Greeks. To a greater degree than most would now be willing to concede, it will prove difficult or impossible to preserve many contemporary institutions in the new millennium. When information societies take shape they will be as different from industrial societies as the Greece of Aeschylus was from the world of the cave dwellers.
\end{paracol}

\section{解放普罗米修斯:主权个体的崛起}
\begin{paracol}{2}[]
即将到来的转变既有好消息也有坏消息。好消息是信息革命将像从未有过的那样解放个人。首次,那些能够自我教育和激励自己的人将几乎完全自由地发明自己的工作,并实现自己生产力的全部好处。天才将被释放,摆脱了政府的压迫和种族和民族偏见的限制。在信息社会中,没有真正有才华的人会被他人没见过的肤浅见解所阻拦。你的种族、外貌、年龄、性取向或发型方式将无关紧要。在网络经济中,别人永远看不见你。在新的网络空间上,丑陋、肥胖、老年和残疾的人将与年轻和美丽的人平等竞争,实现完全无色觉偏见的匿名。
\switchcolumn
The coming transformation is both good news and bad. The good news is that the Information Revolution will liberate individuals as never before. For the first time, those who can educate and motivate themselves will be almost entirely free to invent their own work and realize the full benefits of their own productivity. Genius will be unleashed, freed from both the oppression of government and the drags of racial and ethnic prejudice. In the Information Society, no one who is truly able will be detained by the ill-formed opinions of others. It will not matter what most of the people on earth might think of your race, your looks, your age, your sexual proclivities, or the way you wear your hair. In the cybereconomy, they will never see you. The ugly, the fat, the old, the disabled will vie with the young and beautiful on equal terms in utterly color-blind anonymity on the new frontiers of cyberspace.
\end{paracol}

\subsection{思想成为财富}
\begin{paracol}{2}[]
无论何时何地,凡是有卓越思想的人都将得到前所未有的奖励。在一个最大的财富资源是你脑中的思想而不仅仅是物质资本的环境中,任何能够清晰思考的人都可能富有。信息时代将是流动性的年代。它将为数十亿生活在未曾共享工业社会繁荣的地区的人提供更多平等机会。这些人中最聪明、最成功和最有抱负的人将成为真正的独立个体。
\switchcolumn
Merit, wherever it arises, will be rewarded as never before. In an environment where the greatest source of wealth will be the ideas you have in your head rather than physical capital alone, anyone who thinks clearly will potentially be rich. The Information Age will be the age of upward mobility. It will afford far more equal opportunity for the billions of humans in parts of the world that never shared fully in the prosperity of industrial society. The brightest, most successful and ambitious of these will emerge as truly Sovereign Individuals. 
\switchcolumn*
一开始,只有少数人能实现完全的财务主权。但这并不否定财务独立的优势。并不是每个人都能获得同样巨大的财富,这并不意味着成为富有是毫无意义的。每个亿万富翁都有25000个百万富翁。如果你是百万富翁而不是亿万富翁,那也不代表你是穷人。同样,在未来,衡量你的财务成功的一个里程碑不仅仅是你的净值上有多少个零,而是你能否以一种可以实现个人完全自治和独立的方式来构建你的事务。越聪明的你,就越不需要推动力来实现财务逃逸速度。即使是非常普通的人也可以在全球政治重力对全球经济的影响减弱之际腾飞。在你或你的子孙一生中,无先例的财务独立将成为一个可以实现的目标。
\switchcolumn
At first, only a handful will achieve full financial sovereignty. But this does not negate the advantages of financial independence. The fact that not everyone attains an equally vast fortune does not mean that it is futile or meaningless to become rich. There are 25,000 millionaires for every billionaire. If you are a millionaire and not a billionaire, that does not make you poor. Equally, in the future, one of the milestones by which you measure your financial success will be not just now many zeroes you can add to your net worth, but whether you can structure your affairs in a way that enables you to realize full individual autonomy and independence. The more clever you are, the less propulsion you will require to achieve financial escape velocity. Persons of even quite modest means will soar as the gravitational pull of politics on the global economy weakens. Unprecedented financial independence will be a reachable goal in your lifetime or that of your children.
\switchcolumn*
在生产力的最高高原上,这些主权个体将在类似于希腊神话中神之间的关系的条件下竞争和互动。下一个千禧年的神山将是虚拟空间——一个没有实体存在的领域,但它发展成为新千年二十年代世界上最大的经济体之一。到2025年,虚拟经济将有很多百万参与者。其中一些人将像比尔·盖茨一样富有,价值数百亿美元。虚拟贫穷可能是年收入不到20万美元的人。没有虚拟福利。没有虚拟税收,也没有虚拟政府。虚拟经济可能成为未来30年最伟大的经济现象,而不是中国。
\switchcolumn
At the highest plateau of productivity, these Sovereign Individuals will compete and interact on terms that echo the relations among the gods in Greek myth. The elusive Mount Olympus of the next millennium will be in cyberspace -- a realm without physical existence that will nonetheless develop what promises to be the world's largest economy by the second decade of the new millennium. By 2025, the cybereconomy will have many millions of participants. Some of them will be as rich as Bill Gates, worth tens of billions of dollars each. The cyberpoor may be those with an income of less than \$200,000 a year. There will be no cyberwelfare. No cybertaxes and no cybergovernment. The cybereconomy, rather than China, could well be the greatest economic phenomenon of the next thirty years.
\switchcolumn*
好消息是政治家将不再能够在这个新领域中支配、压制和规范大部分商业活动,就像古希腊城邦的立法者不能修剪宙斯的胡须一样。这对富人来说是好消息。对于不那么富裕的人来说更是好消息。政治施加的障碍和负担对于成为富人来说是更多的障碍,而对于已经富足的人来说则更少。暴力收益递减和权力下放的益处将为每个有活力和雄心壮志的人创造发挥,从而从政治的消亡中受益。即使是政府服务的消费者也会受益,因为企业家会扩大竞争的好处。迄今为止,司法管辖区之间的竞争通常意味着通过暴力竞争来强制执行主导群体的规则。因此,许多跨领土竞争的独创性都集中在军事事业上。但是,网络经济的出现将为主权服务的供给带来新的竞争条件。司法管辖区的繁殖将意味着在新的执行合同和保障人身和财产安全的方式方面的多种多样的实验。全球经济的大部分解放出了政治控制,这将迫使我们所知道的政府在更接近市场原则的条件下运作。政府最终将别无选择,只能把他们服务的地区人口视为顾客,而不是像有组织的犯罪分子对待勒索诈骗受害者一样。
\switchcolumn
The good news is that politicians will no more be able to dominate, suppress, and regulate the greater part of commerce in this new realm than the legislators of the ancient Greek city-states could have trimmed the beard of Zeus. That is good news for the rich. And even better news for the not so rich. The obstacles and burdens that politics imposes are more obstacles to becoming rich than to being rich.    The benefits of declining returns to violence and devolving jurisdictions will create scope for every energetic and ambitious person to benefit from the death of politics. Even the consumers of government services will benefit as entrepreneurs extend the benefits of competition. Heretofore, competition between jurisdictions has usually meant competition by means of violence to enforce the rule of a predominant group. Consequently, much of the ingenuity of interjurisdictional competition was channeled into military endeavor. But the advent of the cybereconomy will bring competition on new terms to provision of sovereignty services. A proliferation of jurisdictions will mean proliferating experimentation in new ways of enforcing contracts and otherwise securing the safety of persons and property. The liberation of a large part of the global economy from political control will oblige whatever remains of government as we have known it to operate on more nearly market terms. Governments will ultimately have little choice but to treat populations in territories they serve more like customers, and less in the way that organized criminals treat the victims of a shakedown racket.
\end{paracol}

\begin{Parallel}{97mm}{45mm}

  
\subsection{超越政治}
  
  \ParallelLText
  {What mythology described as the province of the gods will become a viable option for the individual -- a life outside the reach of kings and councils. First in scores, then in hundreds, and ultimately in the millions, individuals will escape the shackles of politics. As they do, they will transform the character of governments, shrinking the realm of compulsion and widening the scope of private control over resources.}

  \ParallelRText
  {\small 神话描述的神的领域将成为个人的可行选项 - 生活在国王和议会无法触及的生活之外。从成百上千开始,最终达到数百万,个人将逃脱政治的枷锁。他们这样做,将改变政府的性质,缩小强制的范围,扩大对资源的私人控制范围。}

\ParallelPar 
\ParallelLText
  {The emergence of the sovereign individual will demonstrate yet again the strange prophetic power of myth. Conceiving little of the laws of nature, the early agricultural peoples imagined that “powers we should call supernatural” were widely distributed. These powers were sometimes employed by men, sometimes by “incarnate human gods” who looked like men and interacted with them in what Sir James George Frazer described in \emph{The Golden Bough} as “a great democracy.”  }
  
  \ParallelRText
  {\small 个人主权的出现将再次证明神话的奇异预言能力。早期的农业民族很少了解自然法则,他们认为“我们应该称之为超自然的力量”是广泛分布的。这些力量有时被人类利用,有时由“人格化的人类神”利用,他们看起来像人类,并与他们互动在詹姆斯·乔治·弗雷泽在《金枝》中所描述的“一个伟大的民主制”中。}

  \ParallelPar  
  \ParallelLText
  {When the ancients imagined the children of Zeus living among them they were inspired by a deep belief in magic. They shared with other primitive agricultural peoples an awe of nature, and a superstitious conviction that nature's works were set in motion by individual volition, by magic. In that sense, there was nothing self-consciously prophetic about their view of nature and their gods. They were far from anticipating microtechnology. They could not have imagined its impact in altering the marginal productivity of individuals thousands of years later. They certainly could not have foreseen how it would shift the balance between power and efficiency and thus revolutionize the way that assets are created and protected. Yet what they imagined as they spun their myths has a strange resonance with the world you are likely to see. }
  
  \ParallelRText
  {\small 当古人想象宙斯的子女与他们一起生活时,他们受到了对魔法的深刻信仰的启发。他们与其他原始农业民族分享对自然的敬畏,以及通过个体意志的魔法来控制自然力量的迷信信念。从这个意义上说,他们对自然和他们的神并没有什么自觉的预言性。他们远未预见到微技术的到来。数千年后,他们也无法想象它对个人边际生产力的改变会对生产资产和保护方式的变革带来多大的影响。然而,当他们编织神话时所想象的,却与你可能看到的世界有着奇怪的共鸣。 }
  \ParallelPar

\subsection{新的Abracadabra咒语}
 
  \ParallelLText
  {The “abracadabra” of the magic invocation, for example, bears a curious similarity to the password employed to access a computer. In some respects, high-speed computation has already made it possible to mimic the magic of the genie. Early generations of “digital servants” already obey the commands of those who control the computers in which they are sealed much as genies were sealed in magic lamps. The virtual reality of information technology will widen the realm of human wishes to make almost anything that can be imagined seem real. Telepresence will give living individuals the same capacity to span distance at supernatural speed and monitor events from afar that the Greeks supposed was enjoyed by Hermes and Apollo. The Sovereign Individuals of the Information Age, like the gods of ancient and primitive myths, will in due course enjoy a kind of “diplomatic immunity” from most of the political woes that have beset mortal human beings in most times and places. }  
  \ParallelRText
  {\small 魔法咒语中的“阿布拉卡达布拉”与访问电脑的密码惊人地相似。高速计算在某些方面已经让模仿神灵魔法成为可能。早期的“数字仆人”就像法器中被封印的神灵一样服从主人的命令。信息技术的虚拟现实将扩大人类的愿望范围,使几乎任何想象得到的事情都变得真实。远程存在将赋予生命体在超自然速度下跨越距离和远程监控事件的能力,就像希腊神话中的赫尔墨斯和阿波罗一样。信息时代的主权个体,像古代和原始神话中的神灵一样,最终将享有一种“外交豁免权”,使其免于大多数时代和地方困扰凡人的政治问题。}

  \ParallelPar
  \ParallelLText
  {The new Sovereign Individual will operate like the gods of myth in the same physical environment as the ordinary, subject citizen, but in a separate realm politically. Commanding vastly greater resources and beyond the reach of many forms of compulsion, the Sovereign Individual will redesign governments and reconfigure economies in the new millennium. The full implications of this change are all but unimaginable.}

  \ParallelRText
  {\small 新的主权个体将在同一物理环境中与普通公民生活,但在政治上处于单独的领域。拥有非常庞大的资源,超出多种形式约束的范围,主权个体将重新设计政府和经济,进入新的千年。这种变化的全部影响几乎无法想象。”}

  \ParallelPar

\subsection{天才与天惩}  

  \ParallelLText
  {For anyone who loves human aspiration and success, the Information Age will provide a bounty. That is surely the best news in many generations. But it is bad news as well. The new organization of society implied by the triumph of individual autonomy and the true equalization of opportunity based upon merit will lead to very great rewards for merit and great individual autonomy. This will leave individuals far more responsible for themselves than they have been accustomed to being during the industrial period. It will also precipitate transition crises, including a possibly severe economic depression that will reduce the unearned advantage in living standards that has been enjoyed by residents of advanced industrial societies throughout the twentieth century.  As we write, the top 15 percent of the world's population have an average per-capita income of \$21,000 annually. The remaining 85 percent of the world have an average income of just \$1,000. That huge, hoarded advantage from the past is bound to dissipate under the new conditions of the Information Age.  }
  
  \ParallelRText
  {\small 对于任何一个追逐理想和成功的人来说,信息时代的回报将无与伦比。这无疑是几代人以来最好的消息,但也是一个坏消息。基于个人自治的新型社会组织,以及建立在能力之上的、真正的机会均等,会使才能出众者,得到超级的回报和个人自主性。但是,个人要对自己担负的责任,也会远远超过他们在工业时期所习惯的。此外,在整个20世纪,先进工业社会的居民,享受了不劳而获的优越生活,这种优势也将被削弱。在我们写这本书的时候(1997年之前),世界上前15\%的人口,人均年收入为21000美元;其余85\%的人,平均年收入只有1000美元。在信息时代的新环境下,过去囤积起来的巨大优势,必将烟消云散。 }


  \ParallelPar  

  \ParallelLText
  {\small As it does, the capacity of nation-states to redistribute income on a large scale will collapse. Information technology facilitates dramatically increased competition between jurisdictions. When technology is mobile, and transactions occur in cyberspace, as they increasingly will do, governments will no longer be able to charge more for their services than they are worth to the people who pay for them. Anyone with a portable computer and a satellite link will be able to conduct almost any information business anywhere, and that includes almost the whole of the world's multitrillion-dollar financial transactions.  }
  
  \ParallelRText
  {\small 随着它的消散,民族国家大规模重新分配收入的能力将崩溃。信息技术极大地促进了辖区之间的竞争。信息技术加剧了各管辖区之间的竞争。技术是流动的,交易是在网络空间进行的。任何人只要有一台便携式电脑,和一条卫星网络,就可以在任何地方,从事几乎任何信息业务,包括世界上数以万亿美元的金融交易。 }


  \ParallelPar  
  \ParallelLText
  {This means that you will no longer be obliged to live in a high-tax jurisdiction in order to earn high income. In the future, when most wealth can be earned anywhere, and even spent anywhere, governments that attempt to charge too much as the price of domicile will merely drive away their best customers. If our reasoning is correct, and we believe it is, the nation-state as we know it will not endure in anything like its present form.  }
  
  \ParallelRText
  {\small 这意味着,你不再需要为了高收入,而不得不生活在高税率的国家和地区。在未来,大多数财富可以在任何地方赚取,甚至可以在任何地方消费。到那时,政府试图对它的永久居民收取高额的服务费,只会丢掉它们最好的客户。如果我们的推理是正确的,我们相信它是正确的,那么,大家所知道的民族国家,将不会再以任何类似现在的形式而存在。 }
  \ParallelPar

\section{国家的终结} 
  \ParallelLText
  {Changes that diminish the power of predominant institutions are both unsettling and dangerous. Just as monarchs, lords, popes, and potentates fought ruthlessly to preserve their accustomed privileges in the early stages of the modem period, so today's governments will employ violence, often of a covert and arbitrary kind, in the attempt to hold back the clock. Weakened by the challenge from technology, the state will treat increasingly autonomous individuals, its former citizens, with the same range of ruthlessness and diplomacy it has heretofore displayed in its dealing with other governments. The advent of this new stage in history was punctuated with a bang on August 20, 1998, when the United States fired about \$200 million worth of Tomahawk BGM-109 sea-launched cruise missiles at targets allegedly associated with an exiled Saudi millionaire, Osama bin Laden. Bin Laden became the first person in history to have his satellite phone targeted for attack by cruise missiles. Simultaneously, the United States destroyed a pharmaceutical plant in Khartoum, Sudan, in Bin Laden's honor. The emergence of Bin Laden as the enemy-in-chief of the United States reflects a momentous change in the nature of warfare. A single individual, albeit one with hundreds of millions of dollars, can now be depicted as a plausible threat to the greatest military power of the Industrial era. In statements reminiscent of propaganda employed during the Cold War about the Soviet Union, the United States president and his national security aides portrayed Bin Laden, a private individual, as a transnational terrorist and leading enemy of the United States. }
  
  \ParallelRText
  {\small 削弱了主导机构权力的变化既令人不安,又危险。正如君主、贵族、教皇和有权势的人在现代时期的早期阶段为了维护惯有特权而进行的残酷斗争一样,今天的政府也会用暴力,通常是隐蔽和任意的,试图阻止时钟倒转。受技术挑战削弱的国家将像以前对待其他政府一样,用同样的无情和外交手段处理日益自治的个人——它的前公民。这个历史新阶段的出现在1998年8月20日响起;当时,美国向据称与被流放的沙特亿万富翁奥萨马·本·拉登有关的目标发射了价值约2亿美元的海基巡航导弹。本·拉登成为历史上第一个被巡航导弹攻击卫星电话的人。与此同时,美国在苏丹的喀土穆摧毁了一家制药厂,以表彰本·拉登。本·拉登成为美国最大军事力量的可信威胁。单个人,虽然拥有数百万美元,现在也可以被描绘为对工业时代最大的军事力量构成可信威胁的人。美国总统和他的国家安全助手发表的声明,类似于冷战期间有关苏联的宣传,将本·拉登描述为跨国恐怖分子和美国的头号敌人。}


  \ParallelPar  

  \ParallelLText
  {The same military logic that has seen Osama bin Laden elevated to a position as the chief enemy of the United States will assert itself in governments' internal relations with their subjects. Increasingly harsh techniques of exaction will be a logical corollary of the emergence of a new type of bargaining between governments and individuals. Technology will make individuals more nearly sovereign than ever before. And they will be treated that way. Sometimes violently, as enemies, sometimes as equal parties in negotiation, sometimes as allies. But however ruthlessly governments behave, particularly in the transition period, wedding the IRS with the CIA will avail them little. They will be increasingly required by the press of necessity to bargain with autonomous individuals whose resources will no longer be so easily controlled.  }
  
  \ParallelRText
  {\small 相同的军事逻辑已将奥萨马·本·拉登提升为美国的首要敌人,这种逻辑也将在政府与国民的内部关系中得到体现。越来越严厉的敛财手段将成为政府与个人谈判出现的逻辑必然结果。技术将让个人比以往任何时候都更接近主权。他们也将被当作如此对待,有时会被视作敌人,有时会被视作平等的谈判方,有时会被视作盟友。然而,无论政府的行为多么无情,尤其是在过渡期间,将联邦税务局与中央情报局(CIA)捆绑在一起是没有什么用处的。由于自主个体的资源不再轻易被控制,政府将越来越需要与自主个体进行谈判来适应这种变化。}


  \ParallelPar  

  \ParallelLText
  {The changes implied by the Information Revolution will not only create a fiscal crisis for governments, they will tend to disintegrate all large structures. Fourteen empires have disappeared already in the twentieth century. The breakdown of empires is part of a process that will dissolve the nation-state itself. Government will have to adapt to the growing autonomy of the individual. Taxing capacity will plunge by 50-70 percent. This will tend to make smaller jurisdictions more successful. The challenge of setting competitive terms to attract able individuals and their capital will be more easily undertaken in enclaves than across continents.}
  
  \ParallelRText
  {\small 信息革命所带来的变化不仅会为政府创造财政危机,还将倾向于分解所有的大型结构。20世纪已经有14个帝国消失了。帝国的崩溃是一个过程的一部分,该过程将消解民族国家本身。政府将不得不适应个人日益增长的自治。税收收入能力将下降50-70\%。这将倾向于使较小的司法管辖区更为成功。面对吸引有才华的个体和他们的资本的竞争性条款的挑战,将更容易在飞地中而不是跨越大陆进行。 }


  \ParallelPar  

  \ParallelLText
  {We believe that as the modern nation-state decomposes, latter-day barbarians will increasingly come to exercise power behind the scenes. Groups like the Russian mafiya, which picks the bones of the former Soviet Union, other ethnic criminal gangs, nomenklaturas, drug lords, and renegade covert agencies will be laws unto themselves. They already are. Far more than is widely understood, the modern barbarians have already infiltrated the forms of the nation-state without greatly changing its appearances. They are microparasites feeding on a dying system. As violent and unscrupulous as a state at war, these groups employ the techniques of the state on a smaller scale. Their growing influence and power are part of the downsizing of politics. Microprocessing reduces the size that groups must attain in order to be effective in the use and control of violence. As this technological revolution unfolds, predatory violence will be organized more and more outside of central control. Efforts to contain violence will also devolve in ways that depend more upon efficiency than magnitude of power. }
  
  \ParallelRText
  {\small 我们相信,随着现代民族国家的解体,后期野蛮人将越来越多地在幕后行使权力。像俄罗斯黑手党、在前苏联领土上乱捡残羹剩饭的其他族裔犯罪团伙、官僚特权阶层、毒品贩子和叛逆的秘密机构这样的团体将成为自己的法律。他们已经是了。现代野蛮人已经在不大改变国家的形象的情况下,渗透到国家形式之中,远比人们所理解的要多得多。他们是在死亡的系统上寄生的微小寄生虫。这些团体和处于战争状态下的政治机构一样具有暴力和不择手段,他们运用国家的技术进行小规模的实施和控制。他们日益增长的影响力和权力是政治下降的一部分。微处理降低了团体必须达到的规模才能在使用和控制暴力方面发挥有效作用的规模。随着这一技术革命的展开,掠夺性暴力将越来越多地组织在中央控制之外。遏制暴力的努力也将以效率而不是权力大小的方式演化。 }
  \ParallelPar

\subsection{倒退的历史}
 
  \ParallelLText
  {The process by which the nation-state grew over the past five centuries will be put into reverse by the new logic of the Information Age. Local centers of power will reassert themselves as the state devolves into fragmented, overlapping sovereignties. The growing power of organized crime is merely one reflection of this tendency. Multinational companies are already having to subcontract all but essential work. Some conglomerates, such as AT\&T, Unisys, and ITT, have split themselves into several firms in order to function more profitably. The nation-state will devolve like an unwieldy conglomerate, but probably not before it is forced to do so by financial crises.   }
  
  \ParallelRText
  {\small 过去五个世纪民族国家成长的过程将被信息时代的新逻辑逆转。当国家分化成交叉重叠的主权时,地方权力中心将重新确立自己的地位。有组织犯罪的不断壮大仅仅是这种趋势的一个反映。跨国公司已经不得不外包除了必要的工作之外的所有工作。一些企业集团,如AT\&T、Unisys和ITT,已经分裂成几个公司,以更赚钱的方式运作。民族国家将像一个难以管理的企业集团一样分化,但可能不会在金融危机迫使它这样做之前。}


  \ParallelPar  

  \ParallelLText
  {Not only is power in the world changing, but the work of the world is changing as well. This means that the way business operates will inevitably change. The “virtual corporation” is evidence of a sweeping transformation in the nature of the firm, facilitated by the drop in information and transaction costs. We explore the implications of the Information Revolution for dissolving corporations and doing away with the “good job.” In the Information Age, a “job” will be a task to do, not a position you “have.” Microprocessing has created entirely new horizons of economic activity that transcend territorial boundaries. This transcendence of frontiers and territories is perhaps the most revolutionary development since Adam and Eve straggled out of paradise under the sentence of their Maker: “In the sweat of thy face shalt thou eat bread.” As technology revolutionizes the tools we use, it also antiquates our laws, reshapes our morals, and alters our perceptions. This book explains how.   }
  
  \ParallelRText
  {\small 不仅世界权力在改变,世界工作也在改变。这意味着商业运作的方式不可避免地会发生变化。“虚拟企业”是企业性质发生巨变的证据,这一变化得益于信息和交易成本的降低。我们探讨了信息革命对于溶解公司和消除“好工作”的影响。在信息时代,“工作”将是一项任务,而不是一个你“拥有”的职位。微处理技术已经创造了超越领土界限的全新经济活动领域。这种超越国界和领土的能力或许是自亚当和夏娃受造之后最具革命性的进展:“你必须流着汗水才能吃到面包。”随着技术革新所带来的工具倒退了我们的法律,重塑了我们的道德,改变了我们的感知。本书解释了这一点。}


  \ParallelPar  

  \ParallelLText
  {Microprocessing and rapidly improving communications already make it possible for the individual to choose where to work. Transactions on the Internet or the World Wide Web can be encrypted and will soon be almost impossible for tax collectors to capture. Tax-free money already compounds far faster offshore than onshore funds still subject to the high tax burden imposed by the twentieth-century nation-state. After the tum of the millennium, much of the world's commerce will migrate into the new realm of cyberspace, a region where governments will have no more dominion than they exercise over the bottom of the sea or the outer planets. In cyberspace, the threats of physical violence that have been the alpha and omega of politics since time immemorial will vanish. In cyberspace, the meek and the mighty will meet on equal terms. Cyberspace is the ultimate offshore jurisdiction. An economy with no taxes. Bermuda in the sky with diamonds.  }
  
  \ParallelRText
  {\small 微处理技术和迅速改进的通信技术已经使个人有选择工作地点的可能性。在互联网或万维网上进行的交易可以进行加密,并且很快就几乎不可能被税务部门发现。免税的资金已经在海外比在岸上完全受到二十世纪民族国家高税负重担的资金更快地复利增长。在千禧年之后,世界上的大部分商业将迁移到新的网络空间,这是一个政府不再能够支配的领域,就像他们对待海底和外层行星一样。在网络空间,一直是政治的尧舜大禹的物理暴力威胁将消失。在网络空间,弱者和强者将在平等的条件下相遇。网络空间是终极的离岸司法管辖区。一个没有税收的经济。天空中的百慕大岛和钻石。}


  \ParallelPar  

  \ParallelLText
  {When this greatest tax haven of them all is fully open for business, all funds will essentially be offshore funds at the discretion of their owner. This will have cascading consequences. The state has grown used to treating its taxpayers as a farmer treats his cows, keeping them in a field to be milked. Soon, the cows will have wings.  }
  
  \ParallelRText
  {\small 当这个最大的避税天堂完全开放营业时,所有资金实际上将成为业主自行决定的离岸基金。这将产生连锁反应。国家已经习惯了像农民对待奶牛一样对待纳税人,将他们留在田地里挤奶。很快,奶牛将有翅膀。}
  \ParallelPar

\subsection{国家的复仇}
 
  \ParallelLText
  { Like an angry farmer, the state will no doubt take desperate measures at first to tether and hobble its escaping herd. It will employ covert and even violent means to restrict access to liberating technologies. Such expedients will work only temporarily, if at all. The twentieth-century nation-state, with all its pretensions, will starve to death as its tax revenues decline.  }
  
  \ParallelRText
  {\small 像一位愤怒的农民一样,国家无疑会首先采取绝望的措施来束缚和牵制其逃离的群体。它将采用隐秘甚至暴力手段来限制对解放性技术的接触。这些方法只能在短时间内起作用,如果有的话。二十世纪的民族国家,带着所有的自负,将因为税收下降而面临资金不足的困境。 }


  \ParallelPar  

  \ParallelLText
  { When the state finds itself unable to meet its committed expenditure by raising tax revenues, it will resort to other, more desperate measures. Among them is printing money. Governments have grown used to enjoying a monopoly over currency that they could depreciate at will. This arbitrary inflation has been a prominent feature of the monetary policy of all twentieth-century states. Even the best national currency of the postwar period, the German mark, lost 71 percent of its value from January 1, 1949, through the end of June 1995. In the same period, the U.S. dollar lost 84 percent of its value. This inflation had the same effect as a tax on all who hold the currency. As we explore later, inflation as revenue option will be largely foreclosed by the emergence of cybermoney. New technologies will. allow the holders of wealth to bypass the national monopolies that have issued and regulated money in the modern period. Indeed, the credit crises that swept through Asia, Russia, and other emerging economies in 1997 and 1998 attest to the fact that national currencies and national credit ratings are anachronisms inimical to the smooth operation of the global economy. It is precisely the fact that the demands of sovereignty require all transactions within a jurisdiction to be denominated in a national currency that creates the vulnerability to mistakes by central bankers and attacks by speculators which precipitated deflationary crises in one jurisdiction after another. In the Information Age, individuals will be able to use cybercurrencies and thus declare their monetary independence. When individuals can conduct their own monetary policies over the World Wide Web it will matter less or not at all that the state continues to control the industrial-era printing presses. Their importance for controlling the world's wealth will be transcended by mathematical algorithms that have no physical existence. In the new millennium, cybermoney controlled by private markets will supersede fiat money issued by governments. Only the poor will be victims of inflation and ensuing collapses into deflation that are consequences of the artificial leverage which fiat money injects into the economy. }
  
  \ParallelRText
  {\small 当国家发现自己无法通过增加税收来满足支出时,就会采取其他更绝望的措施。其中之一是印钞票。政府已经习惯了享有货币垄断权,可以随意贬值,这种任意通胀一直是所有二十世纪国家的货币政策的一个突出特征。甚至二战后期最好的国家货币德国马克在1949年1月1日至1995年6月底期间的价值下降了71\%。在同一时期,美元贬值了84\%。这种通胀对持有货币的人的影响与税收类似。随着我们之后的探讨,通货膨胀作为一种收入选择在信息时代将被广泛淘汰。新技术将允许财富的持有者绕过现代时期发行和管理货币的国家垄断。事实上,1997年和1998年席卷亚洲、俄罗斯和其他新兴经济体的信贷危机表明,国家货币和国家信用评级是不利于全球经济运作的陈旧思维。正是主权要求在一个管辖范围内的所有交易必须以国家货币计价的事实,才造成了央行家错误和投机者攻击的漏洞,从而引发了一个又一个的通货紧缩危机。在信息时代,个人将能够使用网络货币,因此宣布他们的货币独立。当个人能够通过万维网进行自己的货币政策时,国家继续控制工业时代的印刷机就会变得不那么重要,甚至不重要。它们为掌控世界财富的重要性,将被没有物理存在的数学算法所超越。在新千年,由私人市场控制的网络货币将取代政府发行的法定货币。只有穷人将成为通货膨胀和随后崩溃的受害者,这是法定货币注入经济的人工杠杆的后果。}

  \ParallelPar  

  \ParallelLText
  {Lacking their accustomed scope to tax and inflate, governments, even in traditionally civil countries, will turn nasty. As income tax becomes uncollectible, older and more arbitrary methods of exaction will resurface. The ultimate form of withholding tax --de facto or even overt hostage-taking -- will be introduced by governments desperate to prevent wealth from escaping beyond their reach. Unlucky individuals will find themselves singled out and held to ransom in an almost medieval fashion. Businesses that offer services that facilitate the realization of autonomy by individuals will be subject to infiltration, sabotage, and disruption. Arbitrary forfeiture of property, already commonplace in the United States, where it occurs five thousand times a week, will become even more pervasive. Governments will violate human rights, censor the free flow of information, sabotage useful technologies, and worse. For the same reasons that the late, departed Soviet Union tried in vain to suppress access to personal computers and Xerox machines, Western governments will seek to suppress the cybereconomy by totalitarian means. }
  
  \ParallelRText
  {\small 在没有了习惯性的征收所得税的情况下,即使在传统上文明的国家,政府也将变得残忍。更老、更武断的征税方式将重新出现。政府迫切希望阻止财富逃离其管辖范围,将引入最终形式的代扣税——实际上甚至是公开的劫持人质。不幸的个人将发现自己被单独挑选并以几乎中世纪的方式被绑架和赎金。为个人实现自治的服务的企业将受到渗透、破坏和破坏。在美国,已经普遍存在的任意没收财产行为,每周发生五千次的情况将变得更为普遍。政府将侵犯人权,审查信息的自由流动,破坏有用的技术等等。由于晚已逝去的苏联试图无效地压制个人电脑和复印机的使用,西方政府也将通过极权主义手段试图抑制网络经济。 }
  \ParallelPar

\section{卢德派的回归}   
  \ParallelLText
  {Such methods may prove popular among some population segments. The good news about individual liberation and autonomy will seem to be bad news to many who are frightened by the transition crisis and do not expect to be winners in the new configuration of society. The apparent popularity of the draconian capital controls imposed in 1998 by Malaysian prime minister Mahathir Mohamad in the wake of the Asian meltdown testifies to residual enthusiasm among many for the old-fashioned closed economy dominated by the nation-state. This nostalgia for the past will be fed by resentments inflamed by the inevitable transition crisis. The greatest resentment is likely to be centered among those of middle talent in currently rich countries. They particularly may come to feel that information technology poses a threat to their way of life. The beneficiaries of organized compulsion, including millions receiving income redistributed by governments, may resent the new freedom realized by Sovereign Individuals. Their upset will illustrate the truism that “where you stand is determined by where you sit.” }
  
  \ParallelRText
  {\small 这些方法可能会在某些人群中流行。个人解放和自治的好消息可能会给许多人带来打击,他们对转型危机感到恐惧,并且不期望在社会新形态的赢家中。马来西亚总理马哈蒂尔·莫哈末于1998年在亚洲经济危机后实施的严厉的资本管制政策表明,仍有很多人倾向于传统的以国家为主导的封闭经济。这种对过去的怀旧情结将因转型危机必然带来的愤怒而得到滋养。最大的憎恨可能集中在目前富裕国家中的中等才能人群上。他们特别可能会感到信息技术对他们的生活方式构成威胁。有组织强制的受益者,包括数百万接受政府收入再分配的人,可能会对主权个人实现的新自由感到不满。他们的不满将说明“你所站的位置取决于你的处境”的真理。}
\ParallelPar  

  \ParallelLText
  {It would be misleading, however, to attribute all the bad feelings that will be generated in the coming transition crisis to the bald desire to live at someone else's expense. More will be involved. The very character of human society suggests that there is bound to be a misguided moral dimension to the coming Luddite reaction. Think of it as a bald desire fitted with a moral toupee. We explore the moral and moralistic dimensions of the transition crisis. Self-interested grasping of a conscious kind has far less power to motivate actions than does self-righteous fury. While adherence to the civic myths of the twentieth century is rapidly falling away, they are not without their true believers. Many humans, as the passage quoted from Craig Lambert attests, are belongers, who place importance on being members of a group. The same need to identify that motivates fans of organized sports makes some partisans of nations. Everyone who came of age in the twentieth century has been inculcated in the duties and obligations of the twentieth-century citizen. The residual moral imperatives from industrial society will stimulate at least some neo-Luddite attacks on information technologies.  }
  
  \ParallelRText
  {\small 然而,将即将到来的过渡危机中将产生的所有不良情绪归咎于光秃秃地想生活在别人的代价之下是具有误导性的。它将涉及更多方面。人类社会的性质表明,即将到来的勒德派反应必然会有一个错误的道德维度。把它想象成一个带有道德假发的光秃秃的愿望。我们探讨了过渡期的道德和道德主义维度。有意识的自我利益追求远不如自以为是的愤怒来推动行为。虽然对于20世纪公民传统的遵循正在迅速减少,但它们并不是没有真正的信徒。正如从Craig Lambert引用的文章所证明的那样,许多人是“属于者”,他们认为成为团队成员很重要。同样的认同需求使一些国家倾向于某些主义。在20世纪成年的每个人都受到20世纪公民责任和义务的熏陶。工业社会的残留道德义务将激发至少一些基于信息技术的新勒德派攻击。 }

\ParallelPar  

  \ParallelLText
  {In this sense, this violence to come will be at least partially an expression of what we call “moral anachronism,” the application of moral strictures drawn from one stage of economic life to the circumstances of another. Every stage of society requires its own moral rules to help individuals overcome incentive traps peculiar to the choices they face in that particular way of life. Just as a farming society could not live by the moral rules of a migratory Eskimo band, so the Information Society cannot satisfy moral imperatives that emerged to facilitate the success of a militant twentieth-century industrial state. We explain why. }
  
  \ParallelRText
  {\small 在这个意义上,未来的暴力至少部分上将是我们所谓的“道德时代落后”的表达,即将道德准则应用于另一种经济生活方式的情况。每个社会阶段都需要其自身的道德准则来帮助个人克服该特定生活方式下他们面临的激励陷阱。正如一个农业社会不能按照一个流浪爱斯基摩人团队的道德规则生活一样,信息社会也无法满足于对二十世纪激进的工业国家成功所产生的道德要求。我们会解释原因。 }

\ParallelPar 
 
  \ParallelLText
  {In the next few years, moral anachronism will be in evidence at the core countries of the West in much the way that it has been witnessed at the periphery over the past five centuries. Western colonists and military expeditions stimulated such crises when they encountered indigenous hunting-and-gathering bands, as well as peoples whose societies were still organized for farming. The introduction of new technologies into anachronistic settings caused confusion and moral crises. The success of Christian missionaries in converting millions of indigenous peoples can be laid in large measure to the local crises caused by the sudden introduction of new power arrangements from the outside. Such encounters recurred over and over, from the sixteenth century through the early decades of the twentieth century. We expect similar clashes early in the new millennium as Information Societies supplant those organized along industrial lines. }
  
  \ParallelRText
  {\small 在接下来的几年中,这种道德时代落后将在西方核心国家的许多领域中得到体现,就像在过去的五个世纪里在边缘地区所见到的一样。当西方殖民者和军事远征队遭遇土著狩猎采集部落以及那些仍以耕种种植为生的人时,就会出现这样的危机。新技术的引入到这些时代落后的环境中会导致混乱和道德危机。基督教传教士成功地转化了数百万土著民族,这在很大程度上归功于由外部引入的新能源系统带来的本地危机。从16世纪到20世纪初,这样的冲突一再发生。我们预计,在信息社会取代沿工业线组织的社会时期早期,将会有类似的冲突出现。 }
  \ParallelPar

\subsection{对强制的怀旧情感}

  \ParallelLText
  {The rise of the Information Society will not be wholly welcomed as a promising new phase of history, even among those who benefit from it most. Everyone will feel some misgivings. And many will despise innovations that undermine the territorial nation-state. It is a fact of human nature that radical change of any kind is almost always seen as a dramatic turn for the worse. Five hundred years ago, the courtiers gathered around the duke of Burgundy would have said that unfolding innovations that undermined feudalism were evil. They thought the world was rapidly spiraling downhill at the very time that later historians saw an explosion of human potential in the Renaissance. Likewise, what may someday be seen as a new Renaissance from the perspective of the next millennium will look frightening to tired twentieth-century eyes. }
  
  \ParallelRText
  {\small 信息社会的崛起并不是所有人都欣然接受的一段有前途的历史阶段,即使是那些从中受益最多的人也会感到一些疑虑。每个人都会感到某些不安。许多人会鄙视破坏领土民族国家的创新。这是人性的一个事实,任何一种激进的变革几乎总被视为一个戏剧性的倒退。五百年前,围绕勃艮第公爵的宫廷人士会说,破坏封建制度的展开中的创新是邪恶的。他们认为世界正在迅速地走下坡路,而后来的历史学家则在文艺复兴时期看到了人类潜力的爆发。同样,从下一个千年的角度来看,有一天可能会被看作是新文艺复兴,但它会让疲惫的20世纪眼睛感到恐惧。 }
\ParallelPar  

  \ParallelLText
  {There is a high probability that some who are offended by the new ways, as well as many who are disadvantaged by them, will react unpleasantly. Their nostalgia for compulsion will probably turn violent. Encounters with these new “Luddites” will make the transition to radical new forms of social organization at least a measure of bad news for everyone. Get ready to duck. With the speed of change outracing the moral and economic capacity of many in living generations to adapt, you can expect to see a fierce and indignant resistance to the Information Revolution, notwithstanding its great promise to liberate the future. 
  }
  
  \ParallelRText
  {\small 有很大可能会有一些被新方式冒犯的人,以及许多受其不利影响的人,会做出令人不愉快的反应。他们对强迫的怀旧情结可能会变得暴力。与这些新的“卢德派”相遇将使向激进的新社会组织形式的转变对每个人都至少有一些不好的消息。准备好躲避吧。随着变化的速度超过生活中许多人适应的道德和经济能力,你可以预料到对信息革命的凶猛反抗,尽管它有解放未来的巨大承诺。
  }

\ParallelPar  

  \ParallelLText
  {You must understand and prepare for such unpleasantness. A series of transition crises lies ahead. Deflationary tribulations, such as the Asian contagion that swept through the Far East to Russia and other emerging economies in 1997 and 1998, will erupt sporadically as the dated national and international institutions left over from the Industrial Era prove inadequate to the challenges of the new, dispersed, transnational economy. The new information and communication technologies are more subversive of the modern state than any political threat to its predominance since Columbus sailed. This is important because those in power have seldom reacted peacefully to developments that undermined their authority. They are not likely to now. 
  }
  
  \ParallelRText
  {\small 你必须理解并为这样的不愉快情况做好准备。一系列的转型危机将接踵而至。通货紧缩的磨难,例如 1997 年和 1998 年席卷远东到俄罗斯和其他新兴经济体的亚洲瘟疫,将会间歇性地爆发,因为那些过时的国内外机构已经证明无法应对新的、分散的跨国经济的挑战。新的信息和通讯技术比哥伦布航海后的任何政治威胁都要更具颠覆性,对于现代国家的主导地位更加具有威胁性。这一点很重要,因为那些在权力中的人很少会对破坏他们权威的发展做出和平反应。他们现在也不太可能这么做。
  }
\ParallelPar  

  \ParallelLText
  {The clash between the new and the old will shape the early years of the new millennium. We expect it to be a time of great danger and great reward, and a time of much diminished civility in some realms and unprecedented scope in others. Increasingly autonomous individuals and bankrupt, desperate governments will confront one another across a new divide. We expect to see a radical restructuring of the nature of sovereignty and the virtual death of politics before the transition is over. Instead of state domination and control of resources, you are destined to see the privatization of almost all services governments now provide. For inescapable reasons that we explore in this book, information technology will destroy the capacity of the state to charge more for its services than they are worth to you and other people who pay for them.  
  }
  
  \ParallelRText
  {\small 新旧之间的冲突将塑造新千年的早期年份。我们预计这将是一个充满危险和奖励的时代,在某些领域中,文明的减弱将是空前的,而在其他领域中,范围将是前所未有的。越来越自主的个人和破产、绝望的政府将在新的分界线上相互对抗。我们预计,在过渡结束之前,主权的性质将发生根本性的重组,政治几乎完全死亡。与其主导和控制资源,你注定会看到几乎所有政府现在提供的服务的私有化。出于我们在本书中探讨的无法逃避的原因,信息技术将摧毁国家为其服务所收费比其价值和其他为其支付的人们的贡献更高的状态的能力。
  }

  \ParallelPar

\subsection{市场赋予的主权}

\ParallelLText
  {To an extent that few would have imagined only a decade ago, individuals will achieve increasing autonomy over territorial nation-states through market mechanisms. All nation-states face bankruptcy and the rapid erosion of their authority. Mighty as they are, the power they retain is the power to obliterate, not to command. Their intercontinental missiles and aircraft carriers are already artifacts, as imposing and useless as the last warhorse of feudalism. }
  
  \ParallelRText
  {\small 仅仅十年前,大多数人都无法想象,通过市场机制,个人将获得对领土国家越来越多的自治权。所有国家都面临破产和权威的迅速侵蚀。尽管它们强大,但它们所保留的权力只是毁灭而非统治的权力。它们的洲际导弹和航空母舰已经成为历史,就像封建主义时代的最后一匹战马一样具有威严和无用。}

  \ParallelPar 
 
  \ParallelLText
  {Information technology makes possible a dramatic extension of markets by altering the way that assets are created and protected. This is revolutionary. Indeed, it promises to be more revolutionary for industrial society than the advent of gunpowder proved to be for feudal agriculture. The transformation of the year 2000 implies the commercialization of sovereignty and the death of politics, no less than guns implied the demise of oath-based feudalism. Citizenship will go the way of chivalry.  }
  
  \ParallelRText
  {\small 信息技术通过改变资产的创造和保护方式,使得市场得以大幅度扩展。这是一场革命。实际上,对于工业社会而言,它的革命性可能比火药对封建农业的影响还要深远。2000年的转型意味着主权的商业化和政治的消亡,正如火器对宣誓效忠的封建制度的终结一样。公民身份将逐渐成为历史。}

  \ParallelPar  

  \ParallelLText
  {We believe that the age of individual economic sovereignty is coming. Just as steel mills, telephone companies, mines, and railways that were once “nationalized” have been rapidly privatized throughout the world, you will soon see the ultimate form of privatization --the sweeping denationalization of the individual. The Sovereign Individual of the new millennium will no longer be an asset of the state, a de facto item on the treasury's balance sheet. After the transition of the year 2000, denationalized citizens will no longer be citizens as we know them, but customers.}
  
  \ParallelRText
  {\small 我们相信,个人经济主权时代即将到来。正如曾被国有化的钢铁厂、电话公司、矿山和铁路在全球范围内被迅速私有化一样,你将很快见证终极私有化的形式——个人的彻底非国有化。新千年的主权个体将不再是国家资产,不再是国库资产负债表上的一个实际项目。在2000年过渡后,非国有化公民将不再是我们所知道的公民,而是顾客。}
 \ParallelPar


\section{带宽胜过边界}

  \ParallelLText
  {The commercialization of sovereignty will make the terms and conditions of citizenship in the nation-state as dated as chivalric oaths seemed after the collapse of feudalism. Instead of relating to a powerful state as citizens to be taxed, the Sovereign Individuals of the twenty-first century will be customers of governments operating from a “new logical space.” They will bargain for whatever minimal government they need and pay for it according to contract. The governments of the Information Age will be organized along different principles than those which the world has come to expect over the past several centuries. Some jurisdictions and sovereignty services will be formed through “assortive matching,” a system by which affinities, including commercial affinities, are the basis upon which virtual jurisdictions earn allegiance. In rare cases, the new sovereignties may be holdovers of medieval organizations, like the 900-year-old Sovereign Military Hospitaller Order of St. John of Jerusalem, of Rhodes and of Malta. More commonly known as the Knights of Malta, the order is an affinity group for rich Catholics, with 10,000 current members and an annual income of several billions. The Knights of Malta issues its own passports, stamps, and money, and carries on full diplomatic relations with seventy countries. As we write it is negotiating with the Republic of Malta to reassume possession of Fort St. Angelo. Taking possession of the castle would give the Knights the missing ingredient of territoriality that will enable it to be recognized as a sovereignty. The Knights of Malta could once again become a sovereign microstate, instantly legitimized by a long history. It was from Fort St. Angelo that the Knights of Malta turned back the Turks in the Great Siege of 1565. Indeed, they ruled Malta for many years thereafter, until they were expelled by Napoleon in 1798. If the Knights of Malta were to return in the next few years, there could be no clearer evidence that the modern nation-state system, ushered in after the French Revolution, was merely an interlude in the longer sweep of history in which it has been the norm for many kinds of sovereignties to exist at the same time. }
  
  \ParallelRText
  {\small 主权商业化将使国家公民身份的条款和条件显得像封建主义崩溃后的骑士誓言一样过时。21世纪的主权个人将作为政府的客户从“新的逻辑空间”运营,而不是像纳税的公民那样与强大的国家有关。他们将按照合同谈判并付款以获得他们需要的最小政府。信息时代的政府将按照不同的原则组织,这些原则不同于过去几个世纪世界所期望的原则。一些司法管辖区和主权服务将通过“交配匹配”形成,这是一种基于亲和力(包括商业亲和力)而非地理位置的虚拟管辖权挣得忠诚的基础。在罕见情况下,新的主权可能仍然是中世纪组织(如有着900年历史的圣约翰医院长,罗得和马耳他的邦联)。这个亲密团体是一个富有的天主教徒的可以拥有自己护照,邮票和货币的组织,年收入几十亿。圣约翰骑士团发行自己的护照,邮票和货币,并与70个国家保持着完整的外交关系。就在我们写作的时候,它正在与马耳他共和国进行谈判,以重新占据安琪洛堡。占领城堡将成为这个组织具有领土性的缺失要素,并使其能够被认可为主权。圣约翰骑士团可能再次成为一个主权微型国家,并因悠久的历史而被立即合法化。正是从安琪洛堡,圣约翰骑士团在1565年的大围攻中击退了土耳其人。他们统治了马耳他多年,直到1798年被拿破仑驱逐。如果圣约翰骑士团在未来几年内返回,那么现代国家体系,即法国革命后出现的国家体系,只是历史长河中的一个片段,历史长河中许多种类的主权同时存在是正常的。}


  \ParallelPar
  \ParallelLText
  {Still another and very different model for a postmodern sovereignty based on assortive matching is the Iridium satellite telephone network. At first glance, you may think it odd to treat a cellular telephone service as a kind of sovereignty. Yet Iridium has already received recognition as a virtual sovereignty by international authorities. As you may know, Iridium is a global cellular phone service that allows subscribers to receive calls on a single number, wherever they find themselves on the planet, from Featherston, New Zealand, to the Bolivian Chaco. To allow calls to be routed to Iridium subscribers anywhere on the globe, given the architecture of global telecoms, international telecom authorities had to agree to recognize Iridium as a virtual country, with its own country code: 8816. It is a short step logically from a virtual country comprising satellite telephone subscribers to sovereignty for more coherent virtual communities on the World Wide Web that span borders. Bandwidth, or the carrying capacity of a communications medium, has been expanding faster than computational capacity multiplied after the invention of transistors. If this trend to greater bandwidth continues, as we believe likely, it is only a matter of a few years, soon after the turn of the millennium, until bandwidth becomes sufficiently capacious to make technically possible the “metaverse,” the alternative, cyberspace world imagined by the science fiction novelist Neal Stephenson. Stephenson's “metaverse” is a dense virtual community with its own laws. We believe it is inevitable that, as the cybereconomy becomes richer, its participants will seek and obtain exemption from the anachronistic laws of nation-states. The new cybercommunities will be at least as wealthy and competent at advancing their interests as the Sovereign Military Hospitaller Order of St. John of Jerusalem, of Rhodes and of Malta. Indeed, they will be more capable of asserting themselves because of far-reaching communications and information warfare capabilities. We explore still other models of fragmented sovereignty in which small groups can effectively lease the sovereignty of weak nation-states, and operate their own economic havens much as free ports and free trade zones are licensed to do today.}
  
  \ParallelRText
  {\small 基于匹配交配的后现代主权的另一个非常不同的模型是铱星手机网络。乍一看,你可能认为将一个蜂窝电话服务视为一种主权奇怪。然而,铱星已经被国际当局承认为虚拟主权国家。正如您所知,铱星是一种全球蜂窝电话服务,允许用户在全球任何一个地方接收来电,从新西兰的菲瑟斯顿到玻利维亚查科​​。为了允许呼叫路由到全球任何地方的铱星用户,考虑到全球电信的架构,国际电信当局不得不同意将铱星作为一个虚拟国家来承认,其拥有自己的国家代码:8816。从卫星电话订户组成的虚拟国家到跨越边界的更一致的虚拟社区的主权,这仅是逻辑上的简短步骤。带宽,即通信媒介的承载能力,正在以晶体管发明后计算能力的增长速度加快。如果这种向更大带宽的趋势继续下去,正如我们所相信的那样,那么仅仅几年,在新千年之后不久,带宽就会变得足够宽敞,以使“元宇宙”成为可能,即科幻小说家尼尔·斯蒂芬森想象的另一种虚拟社区。斯蒂芬森的“元宇宙”是一个拥有自己的法律的密集虚拟社区。我们相信,随着网络经济变得更加富裕,其参与者将寻求并获得豁免过时的国家法律的豁免权。由于具有广泛的通信和信息战争能力,新的网络社区将至少与圣约翰马耳他医院长的邦联一样富裕和有能力推进自己的利益。实际上,他们将更有能力表现自己。因为我们认为越来越多的小组可以有效地租用虚弱国家的主权,并运营自己的经济避风港,就像自由港和自由贸易区今天被许可做一样,我们探索了其他碎片化主权的模型。}

  \ParallelPar
  \ParallelLText
  {A new moral vocabulary will be required to describe the relations of Sovereign Individuals with one another and what remains of government. We suspect that as the terms of these new relations come into focus, they will offend many people who came of age as “citizens” of twentieth-century nation-states. The end of nations and the “denationalization of the individual” will deflate some warmly held notions, such as “equal protection under the law,” that presuppose power relations that are soon to be obsolete. As virtual communities gain coherence, they will insist that their members be held accountable according to their own laws, rather than those of the former nation-states in which they happen to reside. Multiple systems of law will again coexist over the same geographic area, as they did in ancient and medieval times.}
  
  \ParallelRText
  {\small 需要新的道德词汇来描述主权个体之间以及政府剩余部分的关系。我们怀疑随着这些新关系的术语逐渐清晰,许多成长于二十世纪民族国家的“公民”会受到冒犯。国家的终结和“个体非国籍化”将破坏一些温暖的理念,如“法律平等保护”,这些理念预设的权力关系很快将过时。随着虚拟社区的凝聚,他们将坚持按照自己的法律来追究成员的责任,而不是根据他们恰巧居住的前民族国家的法律。在同一地理区域中,多个法律系统将再次并存,就像古代和中世纪时期那样。}

  \ParallelPar
  \ParallelLText
  {Just as attempts to preserve the power of knights in armor were doomed to fail in the face of gunpowder weapons, so the modern notions of nationalism and citizenship are destined to be short-circuited by microtechnology. Indeed, they will eventually become comic in much the way that the sacred principles of fifteenth-century feudalism fell to ridicule in the sixteenth century. The cherished civic notions of the twentieth century will be comic anachronisms to new generations after the transformation of the year 2000. The Don Quixote of the twenty-first century will not be a knight-errant struggling to revive the glories o f feudalism but a bureaucrat in a brown suit, a tax collector yearning for a citizen to audit. }
  
  \ParallelRText
  {\small 正如试图保留穿甲骑士的权力注定要失败一样,现代民族主义和公民身份的概念注定会在微技术面前被短路。事实上,它们最终将变得滑稽可笑,就像十五世纪封建主义的神圣原则在十六世纪的嘲笑中失败一样。二十世纪所珍视的公民观念将成为二千年后的新一代的滑稽时代语汇。21世纪的唐吉柯德将不是一个挣扎恢复封建主义辉煌的骑士,而是一名穿着棕色衣服的官僚,一名渴望审计公民的税务员。}
  \ParallelPar


\section{边区法律的复兴}
  
  \ParallelLText
  {We seldom think of governments as competitive entitles, except in the broadest sense, so the modern intuition about the range and possibilities of sovereignty has atrophied. In the past, when the power equation made it more difficult for groups to assert a stable monopoly of coercion, power was frequently fragmented, jurisdictions overlapped, and entities of many different kinds exercised one or more of the attributes of sovereignty. Not infrequently, the nominal overlord actually enjoyed scant power on the ground. Governments weaker than the nation-states are now faced with sustained competition in their ability to impose a monopoly of coercion over a local territory. This competition gave rise to adaptations in controlling violence and attracting allegiance that will soon be new again.   When the reach of lords and kings was weak, and the claims of one or more groups overlapped at a frontier, it frequently happened that neither could decisively dominate the other. In the Middle Ages, there were numerous frontier or “march” regions where sovereignties blended together. These violent frontiers persisted for decades or even centuries in the border areas of Europe. There were marches between areas of Celtic and English control in Ireland; between Wales and England, Scotland and England, Italy and France, France and Spain, Germany and the Slav frontiers of Central Europe, and between the Christian kingdoms of Spain and the Islamic kingdom of Granada. Such march regions developed distinct institutional and legal forms of a kind that we are likely to see again in the next millennium. Because of the competitive position of the two authorities, residents of march regions seldom paid tax. What is more, they usually had a choice in deciding whose laws they were to obey, a choice that was exercised through such legal concepts as “avowal” and “distraint” that have now all but vanished. We expect such concepts to become a prominent feature of the law of Information Societies. }  
  \ParallelRText
  {\small 我们很少将政府视为竞争实体,除了在最广泛的意义上,因而现代人对主权的范围及其可能性的直觉已经萎缩了。 在过去,权力往往是分散的,管辖是重叠的,不同类型的实体行使着主权的一种或多种属性;在这种权力等式中,很难有某个集团能稳定地保持垄断地位。名誉上的最高统治者,在下面并没有多少权力,这种情况在历史上并不少见。现在,比民族国家弱小的政府,它们在地方施加权力的垄断地位,就面临着持续的竞争。这些竞争,曾经改变了控制暴力和吸引效忠的形式,而新的改变很快就会出现。当领主和国王们的势力单薄,往往就会出现一种现象:对同一块边境地区,有一个或多个团体主张权力,而任何一方都无法占据决定性的支配地位。在中世纪,有很多的边疆或“边区”(March)。在这些地方,主权重叠,暴力丛生。边区在欧洲存在了几十年甚至几个世纪,广泛存在于凯尔特人和英特兰人控制的爱尔兰地区之间,在威尔士和英格兰、苏格兰和英格兰、意大利和法国、法国和西班牙、德国和中欧的斯拉夫人边境之间,以及在西班牙的基督教王国和格拉纳达的伊斯兰王国之间。边区形成了独特的制度和法律,在下一个千年,我们很可能会重温它们。在边区,由于存在两个相互竞争的当局,住在这里的人很少交税。更重要的是,他们往往可以选择遵循谁的法律,通过“宣誓”或“封租”等法律方式。这些法律概念和方式现在都不复存在了;我们认为,它们将会成为信息社会法律的明显特征。}
  \ParallelPar

\subsection{超越国籍}

  \ParallelLText
  {Before the nation-state, it was difficult to enumerate precisely the number of sovereignties that existed in the world because they overlapped in complex ways and many varied forms of organization exercised power. They will do so again. The dividing lines between territories tended to become clearly demarcated and fixed as borders in the nation-state system. They will become hazy again in the Information Age. In the new millennium, sovereignty will be fragmented once more. New entities will emerge exercising some but not all of the characteristics we have come to associate with governments. }  
  \ParallelRText
  {\small 在国民国家之前,很难准确地列举出世界上存在的主权数量,因为它们以复杂的方式重叠,许多不同形式的组织行使权力。它们将再次这样做。在国家体系中,领土之间的分界线倾向于变得清晰明确并固定为边界。在信息时代中,它们将再次变得模糊。在新千年中,主权将再次分裂。新实体将出现,行使我们已经习惯了与政府相关的某些特征,但并非全部。}
  \ParallelPar

  \ParallelLText
  {Some of these new entities, like the Knights Templar and other religious military orders of the Middle Ages, may control considerable wealth and military power without controlling any fixed territory. They will be organized on principles that bear no relation to nationality at all. Members and leaders of religious corporations that exercised sovereign authority in parts of Europe in the Middle Ages in no sense derived their authority from national identity. They were of all ethnic backgrounds and professed to owe their allegiance to God, and not to any affinities that members of a nationality are supposed to share in common.}  
  \ParallelRText
  {\small 这些新实体中,像圣殿骑士团和其他中世纪的宗教军团一样,可能会控制相当大的财富和军事力量,而不控制任何固定的领土。它们将按照与国籍无关的原则组织起来。在中世纪欧洲某些地区行使主权权力的宗教公司的成员和领导者,绝不是从国家身份中获得他们的权威。他们来自不同的族裔背景,并自称效忠于上帝,而不是效忠于成员国之间被认为有共同利益的亲缘关系。}
  \ParallelPar

\subsection{赛博空间的商业共和国}

  \ParallelLText
  {You will also see the re-emergence o f associations of merchants and wealthy individuals with semisovereign powers, like the Hanse (confederation of merchants) in the Middle Ages. The Hanse that operated in the French and Flemish fairs grew to encompass the merchants of sixty cities. The “Hanseatic League,” as it is redundantly known in English (the literal translation is “Leaguely League”), was an organization of Germanic merchant guilds that provided protection to members and negotiated trade treaties. It came to exercise semisovereign powers in a number of Northern European and Baltic cities. Such entities will re-emerge in place of the dying nation-state in the new millennium, providing protection and helping to enforce contracts in an unsafe world.}  
  \ParallelRText
  {\small 您还将看到一些富有的商人和个人组成的半主权团体重新出现,例如中世纪的汉萨(商人联盟)。在法国和佛兰德的博览会上运作的汉萨发展到包括六十个城市的商人。被冗余地称为“汉萨同盟”的它是德语商业协会的组织,为成员提供保护,并谈判贸易协议。在一些北欧和波罗的海城市,它开始行使半主权的力量。在新千年代,这样的实体将出现,取代正在衰亡的民族国家,在一个不安全的世界中提供保护,并帮助执行合同。}
  \ParallelPar

  \ParallelLText
  {In short, the future is likely to confound the expectations of those who have absorbed the civic myths of twentieth-century industrial society. Among them are the illusions of social democracy that once thrilled and motivated the most gifted minds. They presuppose that societies evolve in whatever way governments wish them to-preferably in response to opinion polls and scrupulously counted votes. This was never as true as it seemed fifty years ago. Now it is an anachronism, as much an artifact of industrialism as a rusting smokestack. The civic myths reflect not only a mindset that sees society's problems as susceptible to engineering solutions; they also reflect a false confidence that resources and individuals will remain as vulnerable to political compulsion in the future as they have been in the twentieth century. We doubt it. Market forces, not political majorities, will compel societies to reconfigure themselves in ways that public opinion will neither comprehend nor welcome. As they do, the naive view that history is what people wish it to be will prove wildly misleading.  }  
  \ParallelRText
  {\small 简而言之,未来很可能会使那些吸收了20世纪工业社会公民神话的人感到困惑。其中包括曾经激发并激励最有才华的人的社会民主主义幻想。它们预设社会会以政府所希望的方式演进,最好是响应民意调查和严格计数的选票。这在50年前就不如它看起来那么正确。现在,它已经过时了,就像生锈的烟囱一样是工业主义的产物。这些公民神话不仅反映了一种认为社会问题容易被工程解决的心态,它们还反映了一种错误的信心,即资源和个人在将来仍将像在20世纪一样容易受到政治压迫。我们对此表示怀疑。市场力量而不是政治多数将迫使社会以公众意见既无法理解也无法欢迎的方式进行重构。随着这种重构,天真地认为历史是人们希望它成为的观念将被证明是极其误导的。}
  \ParallelPar


  \ParallelLText
  {It will therefore be crucial that you see the world anew. That means looking from the outside in to reanalyze much that you have probably taken for granted. This will enable you to come to a new understanding. If you fail to transcend conventional thinking at a time when conventional thinking is losing touch with reality, then you will be more likely to fall prey to an epidemic of disorientation that lies ahead. Disorientation breeds mistakes that could threaten your business, your investments, and your way of life. }  
  \ParallelRText
  {\small 因此,关键是你需要重新审视世界。这意味着从外部重新分析你可能认为理所当然的许多事情。这将使你能够得出新的理解。如果你在传统思维失去与现实接触的时候不能超越传统思维,那么你更有可能陷入即将到来的失序病毒的困扰中。失序会导致错误,可能会威胁到你的业务、投资和生活方式。}
  \ParallelPar

\subsection{Seeing Anew}

  \ParallelLText
  {To prepare yourself for the world that is coming you must understand why it will be different from what most experts tell you. That involves looking closely at the hidden causes of change. We have attempted to do this with an unorthodox analysis we call the study of megapolitics. In two previous volumes, \emph{Blood in the Streets} and \emph{The Great Reckoning}, we argued that the most important causes of change are not to be found in political manifestos or in the pronouncements of dead economists, but in the hidden factors that alter the boundaries where power is exercised. Often, subtle changes in climate, topography, microbes, and technology alter the logic of violence. They transform the way people organize their livelihoods and defend themselves.   }  
  \ParallelRText
  {\small 为了为即将到来的世界做好准备,您必须理解为什么它会与大多数专家告诉您的不同。这涉及密切关注变化的隐藏原因。我们尝试用一种非正统的分析,称为“大政治”的研究来做到这一点。在两个之前的卷册中,《街头流血》和《大清算》中,我们认为,变化的最重要的原因不在于政治宣言或死亡经济学家的声明,而是在于改变行使权力的边界的隐藏因素。通常,气候、地形、微生物和技术上的微妙变化都会改变暴力的逻辑。它们改变了人们组织生计和自我防卫的方式。}
  \ParallelPar


  \ParallelLText
  {Notice that our approach to understanding how the world changes is very different from that of most forecasters. We are not experts in anything, in the sense that we pretend to know a great deal more about certain “subjects” than those who have spent their entire careers cultivating highly specialized knowledge. To the contrary, we look from the outside in. We are knowledgeable around the subjects about which we make forecasts. Most of all, this involves seeing where the boundaries of necessity are drawn. When they change, society necessarily changes, no matter what people may wish to the contrary. }  
  \ParallelRText
  {\small 请注意,我们理解世界变化的方法与大多数预测者所采用的方法非常不同。我们不是某个领域的专家,也不是假装比那些一生都在培养高度专业知识的人了解更多关于某些“主题”的人。相反,我们从外部看。我们在我们预测的主题领域有知识。最重要的是,这涉及到看到必要性的边界在哪里被画出。当它们改变时,社会必然改变,无论人们想不想改变。}
  \ParallelPar


  \ParallelLText
  {In our view, the key to understanding how societies evolve is to understand factors that determine the costs and rewards of employing violence. Every human society, from the hunting band to the empire, has been informed by the interactions of megapolitical factors that set the prevailing version of the “laws of nature.” Life is always and everywhere complex. The lamb and the lion keep a delicate balance, interacting at the margin. If lions were suddenly more swift, they would catch prey that now escape. If lambs suddenly grew wings, lions would starve. The capacity to utilize and defend against violence is the crucial variable that alters life at the margin. }  
  \ParallelRText
  {\small 在我们看来,理解社会如何演变的关键在于理解决定使用暴力的成本和回报的因素。从狩猎队到帝国,每个人类社会都受到“大政治”因素相互作用的影响,这些因素确定了“自然法则”的普遍版本。生活始终是复杂的。羊和狮子在边缘相互作用,保持着微妙的平衡。如果狮子突然更敏捷,他们会捕捉到现在逃脱的猎物。如果羊突然长出翅膀,狮子会挨饿。利用和防御暴力的能力是改变生命边缘的关键变量。}
  \ParallelPar


  \ParallelLText
  {We put violence at the center of our theory of megapolitics for good reason. The control of violence is the most important dilemma every society faces. As we wrote in \emph{The Great Reckoning}:   }  
  \ParallelRText
  {\small 我们把暴力放在我们的“大都市政治”理论的中心,这是有充分理由的。控制暴力是每个社会面临的最重要的问题。正如我们在《大清算》中所写的那样:}
  \ParallelPar


  \ParallelLText
  {\emph{The reason that people resort to violence is that it often pays. In some ways, the simplest thing a man can do if he wants money is to take it. That is no less true for an army of men seizing an oil field than it is for a single thug taking a wallet. Power, as William Playfair wrote, “has always sought the readiest road to wealth, by attacking those who were in possession of it.” }}
  \ParallelRText
  {\small \emph{人们诉诸暴力的原因在于它往往有利可图。在某些方面,如果一个人想要钱,他最简单的做法就是拿走它。这同样适用于抢夺油田的一支军队和抢夺钱包的单个恶棍。威廉·普雷费尔写道,权力“总是通过攻击那些拥有财富的人来寻求最便捷的致富之路”。}}
  \ParallelPar

  \ParallelLText
  {\emph{The challenge to prosperity is precisely that predatory violence does pay well in some circumstances. War does change things. It changes the rules. It changes the distribution of assets and income. It even determines who lives and who dies. It is precisely the fact that violence does pay that makes it hard to control. }}
  \ParallelRText
  {\small \emph{繁荣的挑战恰恰在于掠夺性暴力在某些情况下确实有好处。战争改变了一切,改变了规则,改变了资产和收入的分配,甚至决定了谁生谁死。正是暴力有利可图的事实使其难以控制。}}
  \ParallelPar

  \ParallelLText
  {Thinking in these terms has helped us foresee a number of developments that better-informed experts insisted could never happen. For example, \emph{Blood in the Streets}, published in early 1987, was our attempt to survey the first stages of the great megapolitical revolution now under way. We argued then that technological change was destabilizing the power equation in the world. Among our principal points:  }  
  \ParallelRText
  {\small 从这些方面思考有助于我们预见到一些明眼人认为永远不可能发生的事情。例如,早在1987年初发表的《街头流血》是我们试图调查当前正在进行的巨型政治革命的最初阶段。我们当时认为,技术变革正在动摇世界上的权力平衡。我们的主要观点之一是:}
  \ParallelPar

\end{Parallel}

\begin{itemize}
\item We said that American predominance was in decline, which would lead to economic imbalances and distress, including another 1929-style stock market crash. Experts were all but unanimous in dfnying that such a thing could happen. Yet within six months, in October 1987, world markets were convulsed by the most violent sell-off of the century.
\item \small 我们曾说过,美国的主导地位正在衰落,这将导致经济失衡和困境,包括另一次1929年式的股市崩盘。专家们几乎一致地否认这种事情可能发生。然而,在六个月内,即1987年10月,世界市场被20世纪最激烈的抛售所动摇。
\end{itemize}

%\vspace{2pt}

\begin{itemize}
\item We told readers to expect the collapse of Communism. Again, experts laughed. Yet 1989 brought the events that “no one could have predicted.” The Berlin Wall fell, as revolutions swept away Communist regimes from the Baltic to Bucharest. 
\item \small 我们告诉读者要预期共产主义的崩溃。同样,专家们嗤之以鼻。然而,在1989年,一些“不可预测”的事件发生了。柏林墙倒塌,革命席卷了从波罗的海到布加勒斯特的共产主义政权。
\end{itemize}


%\vspace{2pt}

\begin{itemize}
\item We explained why the mu1tiethnic empire the Bolshevik nomenklatura inherited from the tsars would “inevitably crack apart.” At the end of December 1991, the hammer-and-sickle banner was lowered over the Kremlin for the last time as the Soviet Union ceased to exist. 
\item \small 我们解释了为什么来自沙皇的多民族帝国会“不可避免地解体”。 1991年12月底,苏联停止存在,镇魂曲被最后一次放下。
\end{itemize}


%\vspace{2pt}

\begin{itemize}
\item During the height of the Reagan arms buildup, we argued that the world stood at the threshold of sweeping disarmament. This, too, was considered unlikely, if not preposterous. Yet the following seven years brought the most sweeping disarmament since the close of World War I. 
\item \small 在里根军备竞赛的高峰期,我们认为世界处于全面裁军的门槛。这也被认为是不可能的,如果不是荒谬的话。然而,接下来的七年带来了自第一次世界大战结束以来最全面的裁军。
\end{itemize}


%\vspace{2pt}

\begin{itemize}
\item At a time when experts in North America and Europe were pointing to Japan for support of the view that governments can successfully rig markets, we said otherwise. We forecast that the Japanese financial assets boom would end in a bust. Soon after the fall of the Berlin Wall, the Japanese stock market crashed, losing almost half its value. We continue to believe that its ultimate low could match or exceed the 89 percent loss that Wall Street suffered at the bottom after 1929. 
\item \small 在北美和欧洲的专家指向日本支持政府可以成功操纵市场的观点的时候,我们持有不同的态度。我们预测,日本的金融资产繁荣将以失败告终。在柏林墙倒塌后不久,日本股市崩盘,几乎损失了一半的价值。我们继续认为,它的最终低点可能会匹配或超过1929年后华尔街跌至底部的89\%的损失。
\end{itemize}


%\vspace{1pt}

\begin{itemize}
\item At a point when almost everyone, from the middle-class family to the world's largest real estate investors, appeared to believe that property markets could only rise and not fall, we warned that a real estate bust was in the offing. Within four years, real estate investors throughout the world lost more than \$1 trillion as property values dropped. 
\item \small 在几乎每个人,从中产阶级家庭到全球最大的房地产投资者,似乎都相信房地产市场只能上涨而不能下跌的时候,我们警告道,房地产危机即将来临。在四年内,全球房地产投资者因房价下跌而损失了1万亿美元以上。
\end{itemize}

%\vspace{1pt}

\begin{itemize}
\item Long before it was obvious to the experts, we explained in \emph{Blood in the Streets} that the income of blue-collar workers had decreased and was destined to continue falling on a long-term basis. As we write today, almost a decade later, it has at last begun to dawn on a sleepy world that this is true. Average hourly wages in the United States have fallen below those achieved in the second Eisenhower administration. In 1993, average annualized hourly wages in constant dollars were \$18,808. In 1957, when Eisenhower was sworn in for his second term, U.S. annualized average hourly wages were \$18,903.
\item \small 长期以来,早在专家们看得出来之前,我们就在《血腥街头》一书中解释,蓝领工人的收入已经下降,并且将在长期基础上继续下降。如今几乎过了十年,这一点终于开始对瞌睡世界有所意识。美国平均小时工资已经下降至艾森豪威尔政府的第二个任期所实现的水平以下。在1993年,美国的年平均工时工资是18808美元。在1957年艾森豪威尔宣誓就职第二个任期时,美国年平均工时工资为18903美元。 
\end{itemize}


\begin{Parallel}{97mm}{45mm}

  \ParallelLText
  {While the main themes of \emph{Blood in the Streets} have proven remarkably accurate with the benefit of hindsight, only a few years ago they were considered rank nonsense by the guardians of conventional thinking. A reviewer in \emph{Newsweek} in 1987 reflected the closed mental climate of late industrial society when he dismissed our analysis as “an unthinking attack on reason.”}  
  \ParallelRText
  {\small 尽管《血腥街头》的主要主题在回顾中表现出了惊人的准确性,但仅仅几年前,它们被传统思维的守护者认为是无稽之谈。1987年《新闻周刊》的一位评论家反映了后期工业社会的封闭思维气氛,当他把我们的分析视为“对理性的无思考攻击”时。}
  \ParallelPar


  \ParallelLText
  {You might imagine that \emph{Newsweek} and similar publications would have recognized with the passage of time that our line of analysis had revealed something useful about the way the world was changing. Not a bit. The first edition of \emph{The Great Reckoning} was greeted with the same sniggering hostility that welcomed \emph{Blood in the Streets}. No less an authority than the \emph{Wall Street Journal} categorically dismissed our analysis as the nattering of “your dopey aunt.”  }  
  \ParallelRText
  {\small 你可能会想象,《新闻周刊》和类似的出版物随着时间的推移应该已经认识到,我们的分析线路揭示了关于世界正在发生变化的有用信息。可惜并没有。 《大清算》的第一版受到了与《血腥街头》相同的嘲笑和敌视。 华尔街日报等权威机构也毫不客气地将我们的分析称为“你迟钝的姑妄言”之类的话语。}
  \ParallelPar


  \ParallelLText
  {This chuckling aside, the themes of \emph{The Great Reckoning} proved less ludicrous than the guardians of orthodoxy pretended.   }  
  \ParallelRText
  {\small 然而,这些调笑话并没有能够掩盖《大清算》所阐述的主题,它们仍然被坚守正统的守护者所忽视。}
  \ParallelPar

\end{Parallel}

\begin{itemize}
\item We extended our forecast of the death of the Soviet Union, exploring why Russia and the other former Soviet republics faced a future of growing civil disorder, hyperinflation, and falling living standards. 
\item \small 我们扩大了我们对苏联解体的预测,并探讨了为什么俄罗斯和其他前苏联共和国面临着日益增长的内部冲突、高通货膨胀和生活水平下降的未来。
\end{itemize}


\begin{itemize}
\item We explained why the 1990s would be a decade of downsizing, including for the first time a worldwide downsizing of governments as well as business entities. 
\item \small 我们解释了为什么20世纪90年代将是一个裁员的十年,包括首次全球范围内的政府和企业实体裁员。
\end{itemize}


\begin{itemize}
\item We also forecast that there would be a major redefinition of terms of income redistribution, with sharp cutbacks in the level of benefits. Hints of fiscal crisis appeared from Canada to Sweden, and American politicians began to talk of “ending welfare as we know it.”
\item \small 我们还预测,将会有一个重大的收入再分配条款重新定义,福利水平将大幅削减。从加拿大到瑞典,财政危机的迹象开始显现,美国政客开始谈论“改变我們熟知的福利制度”。
\end{itemize}


\begin{itemize}
\item We anticipated and explained why the “new world order” would prove to be a “new world disorder.” Well before the atrocities in Bosnia engrossed the headlines, we warned that Yugoslavia would collapse into civil war. 
\item \small 我们预见并解释了为什么“新秩序”将证明是“新混乱”。早在波斯尼亚的暴行引起头条新闻之前,我们就警告说南斯拉夫将陷入内战。
\end{itemize}


\begin{itemize}
\item Before Somalia slid into anarchy, we explained why the pending collapse of governments in Africa would lead some countries there to be effectively placed into receivership. 
\item \small 在索马里陷入无政府状态之前,我们解释了为什么非洲政府的垮台将导致一些国家被有效地接管。
\end{itemize}


\begin{itemize}
\item We forecast and explained why militant Islam would displace Marxism as the principal ideology of confrontation with the West.
\item \small 我们预测并解释了为什么激进伊斯兰教会取代马克思主义成为与西方对抗的主要意识形态。
\end{itemize}


\begin{itemize}
\item Years before the Oklahoma bombing and the attempt to blow up the World Trade Center, we explained why the United States faced an upsurge in terrorism. 
\item \small 在奥克拉荷马城爆炸和企图炸毁世贸中心之前多年,我们就已经解释了为什么美国面临恐怖主义激增。
\end{itemize}


\begin{itemize}
\item Before the headlines that told of the rioting that swept Los Angeles, Toronto, and other cities, we explained why the emergence of criminal subcultures among urban minorities was setting the stage for widespread criminal violence. 
\item \small 在洛杉矶、多伦多和其他城市爆发骚乱之前,我们就已经解释了为什么城市少数族裔中的犯罪亚文化的出现为广泛的犯罪暴力局势铺平了道路。
\end{itemize}


\begin{itemize}
\item We also anticipated “the final depression of the twentieth century,” which began in Asia in 1989 and has been spreading back from the periphery toward the center of the global system. We said that the Japanese stock market would follow Wall Street's path after 1929, and that this would lead to credit collapse and depression. Although massive government intervention in Japan and elsewhere temporarily prevented markets from fully reflecting the deterioration of credit conditions, this only displaced and compounded economic distress, building pressures for competitive devaluations and a systemic credit collapse of the kind that imploded economies worldwide in the 1930s. 
\item \small 我们还预见到了“二十世纪最终的萧条”,它始于1989年的亚洲,并从边缘向全球体系中心扩散。我们说日本股市将沿着1929年华尔街的轨迹走向,这将导致信贷崩溃和经济衰退。尽管日本和其他地方的大规模政府干预暂时阻止了市场充分反映信贷条件恶化的情况,但这只是转移和加剧了经济困境,建立了竞争性货币贬值和类似于20世纪30年代全球性信贷崩溃的系统性信贷崩溃的压力。
\end{itemize}


\begin{Parallel}{97mm}{45mm}


  \ParallelLText
  {\emph{The Great Reckoning} also spelled out a number of controversial theses that have not yet been confirmed, or have not reached the level of development that we forecast:   }  
  \ParallelRText
  {\small 《大清算》也提出了许多有争议的论点,这些观点尚未得到证实,或者没有达到我们预测的水平:}
  \ParallelPar

\end{Parallel}


\begin{itemize}
\item We said that the Japanese stock market would follow Wall Street's path after 1929, and that this would lead to credit collapse and depression. Although unemployment rates in Spain, Finland, and a few other countries exceeded those of the 1930s, and a number of countries, including Japan, did experience local depressions, there has not yet been a systemic credit collapse of the kind that imploded economies worldwide in the 1930s. 
\item \small 我们曾说过,日本股市将在1929年后追随华尔街的道路,这将导致信贷崩溃和经济萧条。尽管西班牙、芬兰和其他一些国家的失业率超过了20世纪30年代的水平,并且包括日本在内的一些国家的确经历了局部经济萧条,但还没有像20世纪30年代那样导致全球经济崩溃的系统性信贷崩溃。
\end{itemize}


\begin{itemize}
\item We argued that the breakdown of the command-and-control system in the former Soviet Union would lead to the spread of nuclear weapons into the hands of ministates, terrorists, and criminal gangs. To the world's good fortune, this has not come to pass, at least not to the degree that we feared. Press reports indicate that Iran purchased several tactical nuclear weapons on the black market; more worryingly, the \emph{Times} of London reported on October 7, 1998, that “Osama bin Laden, the exiled millionaire Saudi terrorist leader, has acquired tactical nuclear weapons from the former Soviet Central Asian states, according to a leading Arab newspaper.” That said, there has been no officially confirmed deployment or use of nuclear weapons from the arsenals of the former Soviet Union.   
\item \small 我们认为,前苏联指挥与控制系统的崩溃将导致核武器传播到小国、恐怖分子和犯罪团伙手中。对于全球幸运的是,至少没有达到我们担心的程度。媒体报道表明,伊朗在黑市上购买了几个战术核武器;更令人担忧的是,《泰晤士报》于1998年10月7日报道称,“流亡的千万富翁沙特恐怖分子领袖本·拉登已从前苏联中亚国家购买了战术核武器,据一家领先的阿拉伯报纸报道。”尽管如此,前苏联的军火库还没有正式确认部署或使用过核武器。
\end{itemize}


\begin{itemize}
\item We explained why the “War on Drugs” was a recipe for subverting the police and judicial systems of countries where drug use is widespread, particularly the United States. With tens of billions of dollars in hidden monopoly profits piling up each year, drug dealers have the means as well as the incentive to corrupt even apparently stable countries. While the world media have carried occasional stories hinting at high-level penetration of the U.S. political system by drug money, the full story has not yet been told.  
\item \small 我们解释了为什么“毒品战争”是破坏警察和司法系统的谋略,特别是在毒品泛滥的国家,尤其是美国。由于每年隐蔽的垄断利润累积数十亿美元,毒品贩子具有腐败甚至是看似稳定国家的手段和动机。尽管全球媒体偶尔报道高层渗透美国政治体系的毒品资金,但完整的故事尚未被揭示。
\end{itemize}


\subsection{看得比别人更远}

\begin{Parallel}{97mm}{45mm}

 \ParallelLText
  {Notwithstanding the points where our forecasts were mistaken or seem mistaken in light of what is now known, the record stands to scrutiny. Much of what is likely to figure in future economic histories of the 1990s was forecast or anticipated and explained in \emph{The Great Reckoning}. Many of our predictions were not simple extrapolations or extensions of trends, but forecasts of major departures from what has been considered normal since World War II. We warned that the 1990s would be dramatically different from the previous five decades. Reading the news of 1991 through 1998, we see that the themes of \emph{The Great Reckoning} were borne out almost daily. }  
  \ParallelRText
  {\small 尽管我们的预测在已知信息的光下似乎是错误的,但记录经得起审查。在《大清算》一书中,许多将会成为未来经济史的内容都被预测或预见,并加以解释。我们的许多预测不是趋势的简单推展或延伸,而是对自第二次世界大战以来被认为是常态的重大变革的预测。我们警告说,1990年代将与前五十年截然不同。阅读1991年至1998年的新闻,我们发现《大清算》的主题几乎每天都在得到证实。}
  \ParallelPar



  \ParallelLText
  {We see these developments not as examples of isolated difficulties, trouble here, trouble there, but as shocks and tremors that run along the same fault line. The old order is being toppled by a megapolitical earthquake that will revolutionize institutions and alter the way thinking people see the world.   }  
  \ParallelRText
  {\small 我们认为这些发展不是孤立的困难例子,这里有麻烦,那里也有麻烦,而是沿着同一断层线发生的震荡和颤动。旧秩序正在被一个巨大的政治地震所推翻,这将彻底改变制度并改变思考人士看待世界的方式。}
  \ParallelPar



  \ParallelLText
  {In spite of the central role of violence in determining the way the world works, it attracts surprisingly little serious attention. Most political analysts and economists write as if violence were a minor irritant, like a fly buzzing around a cake, and not the chef who baked it.  }  
  \ParallelRText
  {\small 尽管暴力在确定世界运作方式中发挥着核心作用,但它吸引的严肃关注却出奇的少。大多数政治分析家和经济学家的写作都表现得好像暴力只是一个小烦恼,就像苍蝇围绕着蛋糕,而不是制作它的厨师。}
  \ParallelPar

\subsection{另一个超级政治先驱}


  \ParallelLText
  {In fact, there has been so little clear thinking about the role of violence in history that a bibliography of megapolitical analysis could be written on a single sheet of paper. In \emph{The Great Reckoning}, we drew upon and elaborated arguments of an almost entirely forgotten classic of megapolitical analysis, William Playfair's \emph{An Enquiry into the Permanent Causes of the Decline and Fall of Powerful and Wealthy Nations}, published in 1805. Here one of our departure points is the work of Frederic C. Lane. Lane was a medieval historian who wrote several penetrating essays on the role of violence in history during the 1940s and 1950s. Perhaps the most comprehensive of these was “Economic Consequences of Organized Violence,” which appeared in the \emph{Journal of Economic History} in 1958. Few people other than professional economists and historians have read it, and most of them seem not to have recognized its significance. Like Playfair, Lane wrote for an audience that did not yet exist.  }  
  \ParallelRText
  {\small 实际上,历史上暴力的角色所涉及的清晰思考如此之少,以至于关于巨型政治分析的书目可以写在一张纸上。在《大清算》中,我们借鉴了并详细阐述了威廉·普莱费尔的巨型政治分析经典作品《对富国强兵的永久性原因的探讨》(1805年出版)的论点,该作品几乎已被人遗忘。其中之一的出发点是弗雷德里克·C·莱恩的工作。莱恩是一位中世纪历史学家,在1940年代和1950年代写了几篇关于历史中暴力角色的深入研究论文。其中最全面的是《有组织暴力的经济后果》,发表于1958年的《经济史学杂志》上。除了专业经济学家和历史学家外,很少有其他人读过它,而且大多数人似乎没有意识到其重要性。像普莱费尔一样,莱恩为一个尚未存在的观众写作。 眼下,对我们最有用的莱恩和普莱费尔的主题是暴力与经济增长之间的关系,这是我们将要探讨的。 }
  \ParallelPar


\subsection{信息时代的洞见}

  \ParallelLText
  {Lane published his work on violence and the economic meaning of war well before the advent of the Information Age. He certainly was not writing in anticipation of microprocessing or the other technological revolutions now unfolding. Yet his insights into violence established a framework for understanding how society will be reconfigured in the Information Revolution.   }  
  \ParallelRText
  {\small 莱恩在信息时代出现之前就发表了关于暴力和战争经济意义的研究工作。他的著述并不是在预测微处理或其他科技革命的到来。然而,他对暴力的洞察建立了一个框架,可以理解信息革命时期,社会将如何重新构建。}
  \ParallelPar


  \ParallelLText
  {The window Lane opened into the future was one through which he peered into the past. He was a medieval historian, and particularly a historian of a trading city, Venice, whose fortunes surged and sagged in a violent world. In thinking about how Venice rose and fell, his attention was attracted to issues that can help you understand the future. He saw the fact that how violence is organized and controlled plays a large role in determining “what uses are made of scarce resources.”  }  
  \ParallelRText
  {\small 莱恩打开的未来之窗是一个窥视过去的窗口。他是一名中世纪历史学家,特别是威尼斯这座交易城市的历史学家。在思考威尼斯的兴衰时,他的注意力被吸引到一些可以帮助我们理解未来的问题上。他认为,暴力的组织和控制方式在决定“稀缺资源的使用方式”方面发挥了重要作用。}
  \ParallelPar



  \ParallelLText
  {We believe that Lane's analyses of the competitive uses of violence has much to tell us about how life is likely to change in the Information Age. But don't expect most people to notice, much less follow, so unfashionably abstract an argument. While the attention of the world is riveted on dishonest debates and wayward personalities, the meanderings of megapolitics continue almost unnoted. The average North American has probably lavished one hundred times more attention on O.J.Simpson and Monica Lewinsky than he has on the new micro technologies that are poised to antiquate his job and subvert the political system he depends on for unemployment compensation.}  
  \ParallelRText
  {\small 我们认为,莱恩关于暴力竞争使用的分析可以告诉我们,在信息时代生活将如何改变。但是不要期望大多数人会注意这么不太流行的抽象论证,更不要期望他们会跟随这个论证。当全世界的注意力都被卡戴珊和莫妮卡·莱温斯基之类的有争议的人物所吸引时,超级政治的波折几乎没有引起注意。普通北美人可能在O·J·辛普森和莫妮卡·莱温斯基方面花费的关注度是他们在新的微技术上花费关注度的百倍,而后者正准备使他们的工作过时,颠覆他们依赖于失业赔偿的政治系统。}
  \ParallelPar

\section{愿望的虚荣}


  \ParallelLText
  {The tendency to overlook what is fundamentally important is not confined solely to the couch dweller watching television. Conventional thinkers of all shapes and sizes observe one of the pretenses of the democratic nation-state --that the views people hold determine the way the world changes. Apparently sophisticated analysts lapse into explanations and forecasts that interpret major historical developments as if they were determined in a wishful way. A striking example of this type of reasoning appeared on the editorial page of the \emph{New York Times} just as we were writing: “Goodbye, Nation-State, Hello...What?,” by Nicholas Colchester. Not only was the topic, the death of the nation-state, the very topic we are addressing, but its author presents himself as an excellent marker to illustrate how far removed our way of thinking is from the norm. Colchester is no simpleton. He wrote as editorial director of the \emph{Economist} Intelligence Unit. If anyone should form a realistic view of the world it should be he. Yet his article clearly indicates in several places that “the coming of international government” is “now logically unstoppable.”}  
  \ParallelRText
  {\small 忽视根本重要的事情的倾向并不仅限于躺在沙发上看电视的人。各种形状和大小的传统思想家观察到民主国家的伪装之一 -- 人们持有的观点决定了世界变化的方式。明显高级的分析师们陷入解释和预测之中,将主要历史发展解释为如愿以偿的方式。刚好在我们写作的时候,在《纽约时报》的编辑专栏上出现了一个醒目的例子:“再见,民族国家,你好……什么?” ,由尼古拉斯·科尔切斯特(Nicholas Colchester)撰写。不仅是话题,民族国家的死亡,就是我们正在处理的话题,而且作者表现出自己是一个很好的标志,用以说明我们的思维方式与常规思维方式有多么不同。科尔切斯特不是傻瓜。他是《经济学家》情报部门的编辑总监。如果应该有人形成现实的世界观,那应该是他。然而,他的文章在若干地方明确指出,“国际政府的到来”现在是“逻辑上不可阻挡的”。}
  \ParallelPar


  \ParallelLText
  {Why? Because the nation-state is faltering and can no longer control economic forces.  In our view, this assumption verges on the absurd. To suppose that some specific new form of governance will emerge simply because another has failed is a fallacy. By that reasoning, Haiti and the Congo would long ago have had better government simply because what they had was so luminously inadequate.   }  
  \ParallelRText
  {\small 为什么?因为民族国家正在衰败,无法再控制经济力量。在我们看来,这种假设接近荒谬。假设仅仅因为另一种形式的治理失败了,某种特定的新治理形式就会出现,这是一种谬论。按照这种推理,海地和刚果早就应该有更好的政府了,仅仅因为它们原有的那种政府是如此的不足。}
  \ParallelPar

  \ParallelLText
  {Colchester's point of view, widely shared among the few who think about such things in North America and Europe, utterly fails to take into account the larger megapolitical forces that determine what types of political systems are actually viable. That is the focus of this book. When the technologies that are shaping the new millennium are considered, it is far more likely that we will see not one world government, but microgovernment, or even conditions approaching anarchy.    }  
  \ParallelRText
  {\small 考尔切斯特的观点在北美和欧洲这样的思考之人中间广泛共享,但它完全没有考虑到决定政治体制实际上是可行的更大的超级政治力量。这本书的重点就是这个问题。当考虑到塑造新千年的技术时,我们更有可能看到的不是一个世界政府,而是微型政府,甚至接近无政府状态。}
  \ParallelPar


  \ParallelLText
  {For every serious analysis of the role of violence in determining the rules by which everyone operates, dozens of books have been written about the intricacies of wheat subsidies, and hundreds more about arcane aspects of monetary policy. Much of this shortfall in thinking about the crucial issues that actually determine the course of history probably reflects the relative stability of the power configuration over the past several centuries. The bird that falls asleep on the back of a hippopotamus does not think about losing its perch until the hippo actually moves. Dreams, myths, and fantasies play a much larger role in informing the supposed social sciences than we commonly think.  }  
  \ParallelRText
  {\small 这在经济正义的丰富文献中特别明显。对经济正义和不正义的数百万言论和文字述说,与认真分析暴力如何塑造社会并因此为经济体制设定界限的书页相比,数量要多得多。然而在现代情境下,经济正义的表述预设社会被一种力量所统治,这种力量非常强大,可以夺取和重新分配生活的好东西。这种权力仅在现代时期的短暂几代人存在。现在它正在消逝。}
  \ParallelPar

  \ParallelLText
  {This is particularly evident in the abundant literature of economic justice. Millions of words have been uttered and written about economic justice and injustice for each page devoted to careful analysis of how violence shapes society, and thus sets the boundaries within which economies must function. Yet formulations of economic justice in the modem context presuppose that society is dominated by an instrument of compulsion so powerful that it can take away and redistribute life's good things. Such power has existed for only a few generations of the modem period. Now it is fading away. }  
  \ParallelRText
  {\small 对于决定每个人运作规则的暴力角色的严肃分析,有数十本书被写成关于小麦补贴的错综复杂事宜的数倍,还有几百本有关货币政策的深奥方面的书籍。关于实际上决定历史进程的关键问题思考的缺少大部分可能反映了过去几个世纪权力配置的相对稳定。在河马背上睡着的鸟类直到河马真正移动起来才会考虑失去它的栖息地。梦想、神话和幻想在通知所谓社会科学方面所起的作用比我们通常想象的要大得多。}
  \ParallelPar

\subsection{社会保障的大哥}

  \ParallelLText
  {Industrial technology gave governments greater instruments of control in the twentieth century than ever before. For a time, it seemed inevitable that governments would become so effective at monopolizing violence as to leave little room for individual autonomy. Nobody at mid-century was looking forward to the triumph of the Sovereign Individual.  }  
  \ParallelRText
  {\small 工业技术在20世纪为政府提供了比以往任何时候都更大的控制工具。曾经有一段时间,政府似乎注定会在垄断暴力方面变得如此有效,以至于个体自治的空间很小。半个世纪以来,没有人期待主权个体的胜利。}
  \ParallelPar


  \ParallelLText
{Some of the shrewdest observers of the mid-twentieth century became convinced on the evidence of the day that the tendency of the nation-state to centralize power would lead to totalitarian domination over all aspects of life. In George Orwell's \emph{1984} (1949), Big Brother was watching the individual vainly struggle to maintain a margin of autonomy and self-respect. It appeared to be a losing cause. Friedrich von Hayek's \emph{The Road to Serfdom} (1944) took a more scholarly view in arguing that freedom was being lost to a new form of economic control that left the state as the master of everything. These works were written before the advent of microprocessing, which has incubated a whole range of technologies that enhance the capacity of small groups and even individuals to function independently of central authority.  }  
  \ParallelRText
  {\small 二十世纪中期最精明的观察家之一通过当时的证据确信,民族国家集中权力的趋势将导致对生活各个方面的极权主义控制。在乔治·奥威尔的《1984》(1949)中,大兄弟一直在注视着个体徒劳地努力保持自治权和自尊心。这似乎是一场失败的事业。弗里德里希·冯·哈耶克(Friedrich von Hayek)的《通往奴役之路》(1944)则采取了更加学术的视角,认为自由正在失去,被一种新形式的经济控制所取代,这让国家成为了一切的主宰。这些作品均在微处理器的出现之前编写,而微处理器孕育了一系列增强小团体甚至个人独立于中央权威运作的技术。}
  \ParallelPar


  \ParallelLText
  {As shrewd as observers like Hayek and Orwell were, they were unduly pessimistic. History has unfolded its surprises. Totalitarian Communism barely outlasted the year 1984. A new form of serfdom may yet emerge in the next millennium if governments succeed in suppressing the liberating aspects of microtechnology. But it is far more likely that we will see unprecedented opportunity and autonomy for the individual. What our parents worried about may prove to be no problem at all. What they took for granted as fixed and permanent features of social life now seem destined to disappear. Wherever necessity sets boundaries to human choice, we adjust, and reorganize our lives accordingly. }  
  \ParallelRText
  {\small 正如哈耶克和奥威尔等观察家一样精明,他们过于悲观了。历史已经揭示了它的惊喜。极权主义共产主义仅在1984年之后未能幸存。如果政府成功地压制了微技术的解放方面,未来的新奴隶制形式可能会出现。但更有可能的是,我们将看到个人前所未有的机遇和自治权。我们父母所担心的问题可能根本不是问题,他们认为是社会生活的固定和永久特征的事情现在注定会消失。无论什么时候,当必要性给人类的选择设置边界时,我们都会进行调整并重新组织我们的生活。}
  \ParallelPar

\subsection{预测有风险}


  \ParallelLText
  {No doubt we put our small measure of dignity at risk in attempting to foresee and explain profound changes in the organization of life and the culture that binds it together. Most forecasts are doomed to make silly reading in the fullness of time. And the more dramatic the change they envision, the more embarrassingly wrong they tend to be. The world doesn't end. The ozone doesn't vanish. The coming Ice Age dissolves into global warming. Notwithstanding all the alarms to the contrary, there is still oil in the tank. Mr. Antrobus, the everyman of \emph{The Skin of Our Teeth}, avoids freezing, survives wars and threatened economic calamities, and grows old ignoring the studied alarms of experts.  }  
  \ParallelRText
  {\small 毫无疑问,在试图预见和解释生活组织和文化变革方面,我们将冒险失去一些尊严。大多数预测注定在时间的推移中变得愚蠢。而且,它们设想的变化越剧烈,它们就越容易出错。世界不会终结,臭氧不会消失。即使有所有相反的警报,油箱里还是有油。《我们的牙齿的皮肤》中的平民先生安特罗布斯避免冻结,幸存于战争和潜在的经济灾难中,老化时忽略专家的警报。 }
  \ParallelPar


  \ParallelLText
  {Most attempts to “unveil” the future soon tum out to be comic. Even where self-interest provides a strong incentive to clear thinking, forward vision is often myopic. In 1903, the Mercedes company said that “there would never be as many as 1 million automobiles worldwide. The reason was that it was implausible that as many as 1 million artisans worldwide would be trainable as chauffeurs.”}  
  \ParallelRText
  {\small 大多数揭示未来的尝试很快就变成了喜剧。即使自身利益提供了强烈的清晰思考的动力,前瞻性的愿景也经常是短视的。1903年,梅赛德斯公司说:“全世界永远不会超过100万辆汽车。原因是全球最多只有100万个工匠可供培训为司机。”}
  \ParallelPar


  \ParallelLText
  {Recognizing this should stop our mouths. It doesn't. We are not afraid to stand in line for a due share of ridicule. If we mistake matters greatly, future generations may laugh as heartily as they please, presuming anyone remembers what we said. To dare a thought is to risk being wrong. We are hardly so stiff and useless that we are afraid to err. Far from it. We would rather venture thoughts that might prove useful to you than suppress them out of apprehension that they might prove overblown or embarrassing in retrospect.  }  
  \ParallelRText
  {\small 认识到这一点应该让我们闭嘴。但事实并非如此。我们不怕被取笑,也会排队领取应得的嘲笑。虽然我们可能会大错特错,但后代可能会为我们的言论大笑,如果有人记得我们所说的话的话。敢于思考就冒着被错的风险。我们并非这么僵硬和无用,以至于害怕犯错。恰恰相反,我们宁愿尝试提出可能对您有用的想法,而不会因为担心它们可能被夸大或在回顾中尴尬而压抑它们。}
  \ParallelPar


  \ParallelLText
  {As Arthur C. Clarke shrewdly noted, the two overriding reasons why attempts to anticipate the future usually fall flat are “Failure of Nerve and Failure of Imagination.” Of the two, he wrote, “Failure of Nerve seems to be the more common; it occurs when even given all the relevant facts the would-be prophet cannot see that they point to an inescapable conclusion. Some of these failures are so ludicrous as to be almost unbelievable.”}  
  \ParallelRText
  {\small 正如亚瑟·克拉克所说,预测未来通常失败的两个主要原因是“胆怯的失败”和“想象的失败”。在这两者中,他写道,“胆怯的失败似乎更为常见;即使在提供了所有相关事实的情况下,那些想预言未来的人也看不到它们指向不可避免的结论。其中一些失败是如此荒谬,以至于几乎难以置信。”}
  \ParallelPar


  \ParallelLText
  {Where our exploration of the Information Revolution falls short, as it inevitably will, the cause will be due more to a lack of imagination than to a lack of nerve. Forecasting the future has always been a bold enterprise, one which properly excites skepticism. Perhaps time will prove that our deductions are wildly off the mark. Unlike Nostradamus, we do not pretend to be prophetic personalities. We do not foretell the future by stirring a wand in a bowl of water or by casting horoscopes. Nor do we write in cryptic verse. Our purpose is to provide you with a sober, detached analysis of issues that could prove to be of great importance to you.   }  
  \ParallelRText
  {\small 当我们对信息革命的探索受阻时,这必然更多地是由于缺乏想象力而不是缺乏勇气。预测未来一直是一项大胆的企业,这正是激起怀疑心理的原因。也许时间会证明我们的推断完全错了。与诺斯特拉达姆斯不同,我们不假装有预言性的个性。我们不通过在水的碗里搅拌魔杖或占星术来预测未来。我们也不写含义深奥的诗歌。我们的目的是为您提供一种冷静、客观的对重要问题进行分析的方法。}
  \ParallelPar


  \ParallelLText
  {We feel an obligation to set out our views, even where they seem heretical, precisely because they may not otherwise be heard. In the closed mental atmosphere of late industrial society, ideas do not traffic as freely as they should through the established media.}  
  \ParallelRText
  {\small 即使看似异端邪说,我们也有义务阐述我们的观点,因为它们可能不会被他人听到。在后期工业社会的封闭思维氛围中,思想不像它们应该在传统媒体中那样自由。}
  \ParallelPar


  \ParallelLText
  { This book is written in a constructive spirit. It is the third we have written together, analyzing various stages of the great change now under way. Like \emph{Blood in the Streets} and \emph{The Great Reckoning}, it is a thought exercise. It explores the death of industrial society and its reconfiguration in new forms. We expect to see amazing paradoxes in the years to come. On the one hand, you will witness the realization of a new form of freedom, with the emergence of the Sovereign Individual. You can expect to see almost the complete liberation of productivity. At the same time, we expect to see the death of the modern nation-state. Many of the assurances of equality that Western people have grown to take for granted in the twentieth century are destined to die with it. We expect that representative democracy as it is now known will fade away, to be replaced by the new democracy of choice in the cybermarketplace. If our deductions are correct, the politics of the next century will be much more varied and less important than that to which we have become accustomed. }  
  \ParallelRText
  {\small 这本书是以建设性的精神编写的。这是我们合作编写的第三本书,分析正在发生的重大变革的不同阶段。像《街头大屠杀》和《大清算》一样,它是一种思想锻炼。它探讨了工业社会的死亡以及它在新形式中的重组。我们预计未来几年会出现惊人的悖论。一方面,你将见证一种新形式的自由实现,即主权个人的出现。你可以期待看到生产力的几乎完全解放。同时,我们预计现代主权国家将会消亡。许多西方人在二十世纪习惯于接受的平等保证也将随之消失。我们预计现有的代议制民主将逐渐式微,被网络市场的新民主所取代。如果我们的推论正确,未来世纪的政治将比我们习惯的要更多样化,也更不重要。}
  \ParallelPar


  \ParallelLText
  {We are confident that our argument will be easy to follow, notwithstanding the fact that it leads through some territory that is the intellectual equivalent of the backwoods and bad neighborhoods. If our meaning is not entirely intelligible in places, that is not because we are being cute, or using the time-honored equivocation of those who pretend to foretell the future by making cryptic pronouncements. We are not equivocators. If our arguments are unclear, it is because we have failed the task of writing in a way that makes compelling ideas accessible. Unlike many forecasters, we want you to understand and even duplicate our line of thinking. It is based not upon psychic reveries or the gyrations of planets, but upon old-fashioned, ugly logic. For quite logical reasons, we believe that microprocessing will inevitably subvert and destroy the nation-state, creating new forms of social organization in the process. It is both necessary and possible for you to foresee at least some details of the new way of life that may be here sooner than you think.  }  
  \ParallelRText
  {\small 我们有信心说服您接受我们的论点,尽管有时候会经过某些思维混乱的地方。如果有些地方不太容易理解,那并不是因为我们想卖弄,或者使用掐头去尾的说法来预言未来。我们不是隐晦不明的人。如果我们的论点不清晰,那是因为我们在书写中没有成功地表达好我们的理念。与许多预言家不同的是,我们希望您能够理解并且复制我们的思维方式。我们的论点是根据老派、丑陋的逻辑推导而成,而非凭借心理幻觉或行星运转。由于一些十分合乎逻辑的原因,我们相信微处理将不可避免地颠覆和摧毁民族国家,在此过程中产生新的社会组织形式。至少可以预见到生活的某些新细节即将到来,这一点是必要的和可能的。
}
  \ParallelPar

\subsection{讽刺的未来预言}

  \ParallelLText
  {For centuries, the end of this millennium has been seen as a pregnant moment in history. More than 850 years ago, St. Malachy fixed 2000 as the date of the Last Judgment. American psychic Edgar Cayce said in 1934 that the earth would shift on its axis in the year 2000, causing California to split in two and inundating New York City and Japan. A Japanese rocket scientist, Hideo Itokawa, announced in 1980 that the alignment of the planets in a “Grand Cross” on August 18, 1999, would cause widespread environmental devastation, leading to the end of human life on earth.}  
  \ParallelRText
  {\small 数百年来,千禧年末日被视为历史上一个充满可能性的时刻。850多年前,圣马拉基预言2000年是世界末日的日期。1934年,美国通灵者埃德加·凯西称地球会在2000年转移其轴心,导致加利福尼亚州分裂成两半,淹没纽约市和日本。1980年,日本火箭科学家板川英雄宣布,1999年8月18日行星的“宏伟十字”将引起广泛的环境破坏,导致地球上的人类生命的终结。}
  \ParallelPar


  \ParallelLText
  {Such visions of apocalypse make a plump target for ridicule. After all, the year 2000, while an imposing round number, would appear to be only an arbitrary artifact of the Christian calendar as adopted in the West. Other calendars and dating systems calculate centuries and millennia from different starting points. By the reckoning of the Islamic calendar, for example, A.D. 2000 will be the year 1378. As ordinary-sounding as a year can be. According to the Chinese calendar, which repeats itself every sixty years, A.D. 2000 is just another year of the dragon. It is part of a continuous cycle that extends millennia into the past. Yet there is more than theological investment in the year 2000. Its importance is undergirded not only by Christian tradition, but by the limitations of mid-century information technology. The so-called Y2K or year 2000 computer problem, a potentially devastating logic flaw in billions of lines of computer code, could approximate apocalyptic conditions by closing down essential elements of industrial society on the millennial midnight. Many computers and microprocessors use software preserved and recycled from the earliest days of computers, when memory space, at \$600,000 per megabyte, was more valuable than gold. To save expensive space, the early programmers tracked dates with only the last two numbers of the year. This convention of employing two-digit date fields was carried over into most software employed in mainframe computers, and even found wide use in personal computers and so-called embedded chips, microprocessors that are used to control almost everything, from VCRs to car ignition systems, security systems, telephones, the switching systems that control the telephone network, process and control systems in factories, power plants, oil refineries, chemical plants, pipelines and much more. Thus, abbreviated into a two-digit field, the year 1999 would be “99.” The trouble is what happens when 00 comes up for the year 2000. Many computers will read this as 1900. This may make it impossible for many unremediated computers and other digital devices to recognize the year 2000 in date fields.}  
  \ParallelRText
  {\small 这样的末日预言成为嘲笑的目标。毕竟,2000年虽然是一个庞大的圆整数,但在西方采用的基督教日历中似乎只是牵强附会的结果。其他日历和日期计算系统从不同的起始点计算世纪和千禧年。例如,按照伊斯兰教历,公元2000年将是1378年。根据中国的农历,每60年重复一次的公元2000年只是龙年的又一年。它是一个延续了几千年的持续循环的一部分。然而,2000年的重要性不仅在于基督教传统,还在于20世纪信息技术的局限性。所谓的Y2K或2000年计算机问题是数十亿行电脑代码中可能严重的逻辑漏洞,可以通过关闭千禧时刻的主要元素来近似末日条件。许多计算机和微处理器使用了早期计算机保存和回收的软件,当时每兆字节的存储空间价格高达60万美元,比黄金更贵。为了节省昂贵的空间,早期的程序员只使用年份的最后两个数字跟踪日期。这个使用两位数日期字段的约定传承到大型计算机中使用的大多数软件中,甚至在个人计算机和嵌入式芯片中也得到广泛应用,微处理器用于控制几乎所有东西,从录像机到汽车点火系统、安全系统、电话、控制电话网络的交换系统,在工厂、发电厂、炼油厂、化工厂、管道和其它众多设施的处理和控制系统中使用。因此,缩写成两位数字的1999年将是“99”年。问题在于00年出现后会发生什么。许多计算机将把这个读作1900年。这可能使得许多未进行修复的计算机和其它数字设备无法识别日期字段中的2000年。}
  \ParallelPar


  \ParallelLText
  {The result will be a massive problem of data corruption that will provide an accidental illustration of a new potential for information warfare. In the Information Age, potential adversaries will be able to wreak havoc by detonating “logic bombs” that sabotage the functions of essential systems by corrupting the data upon which their functioning depends. As a military exercise, for example, you would not need to shoot down an airplane, if you could corrupt data crucial to its safe operation. Data corruption can do almost as much as physical weapons can to thwart the function of a modern society. That this has potentially far-reaching consequences should be obvious on reflection. For example, the \emph{Mail of London} reported on December 14, 1997, that airlines around the globe were planning to cancel hundreds of flights on January 1, 2000 out of fear that air traffic control systems could fail.19 Potential problems include not only the air traffic systems, but also date-sensitive functions built into the airplanes themselves. According to Boeing, many airplanes will require Y2K remediation. Many devices may have a problem if they try to log an event on an invalid date. The fly-by-wire computer-controlled systems that operate airplanes may malfunction if they are programmed to conclude that crucial maintenance was last performed in the year 1900. They many even go into an error loop and shut down. }  
  \ParallelRText
  {\small 结果将是大规模的数据损坏问题,将提供一个关于信息战新潜力的意外说明。在信息时代,潜在的对手将能够通过炸毁破坏其功能的必须数据的“逻辑炸弹”来制造混乱。例如,作为一个军事演习,如果你能破坏关键维护的数据,就不需要击落一架飞机了。数据损坏几乎可以像物理武器一样阻扰现代社会的功能。这个潜在的影响应该毋庸置疑。例如,《伦敦邮报》在1997年12月14日报道,全球航空公司计划取消2000年1月1日的数百个航班,因为他们担心空中交通管制系统可能会失效。潜在的问题不仅包括空中交通系统,还包括内置在飞机本身中的日期敏感功能。据波音公司称,许多飞机将需要进行Y2K修复。如果设备试图记录无效日期上的事件可能会出现问题。如果计算机程序被编程为认为关键维护是在1900年进行的,则控制飞机的电脑化飞控系统可能会出现故障。它们可能甚至会进入错误循环并关闭。}
  \ParallelPar


  \ParallelLText
  {The potentially lethal feedback effects of a logic time bomb that closes down noncompliant control systems could make the turn of the millennium a memorable date for unpleasant reasons. Remember, you can be affected by many devices that go into an error loop and shut down even if you are lucky enough not to find yourself in midair when the new millennium begins.  }  
  \ParallelRText
  {\small 潜在致命的反馈效应可能会关闭不符合规定的控制系统,使千年之交成为一个不愉快的纪念日。要记住,即使你在新千年开始时没有置身于空中,你也可能会受到许多设备进入错误循环并关闭的影响。}
  \ParallelPar


  \ParallelLText
  {You would be well advised to avoid an accident arising from non-Y2K-compliant pacemakers, or simply inebriated millennial revelers, because if the pacemakers shut down, the phone system might also, so the ambulance might never come. Unless you live in Brazil or Ukraine, you are used to picking up the telephone or turning on the car phone and automatically getting a dial tone. Happily, you seldom have to concern yourself with the technical details of how the telephone system operates. But it turns out that phone network switches and routers are highly date dependent. All connections are logged to a date and time, which is crucial to calculating call duration for billing. If you happen to make a one-minute call at 11:59:30 on December 31, 1999, and at 12:00:00 the system reads your call as having had a negative duration of more than 99 years, error loops and shutdown are possible. While long-distance companies are spending great sums to upgrade their switches to make them year 2000 compliant, and local service providers presumably are too, if even a few smaller companies fail to comply and go down, the whole network could be affected. You will be lucky to get a dial tone on January 1,2000.   }  
  \ParallelRText
  {\small 你最好避免因非Y2K兼容的起搏器或醉酒的千禧年狂欢者而发生事故,因为如果起搏器关闭,电话系统可能也会关闭,那么救护车可能永远不会来。除非你住在巴西或乌克兰,否则你习惯于拿起电话或打开车载电话自动获得拨号音。令人高兴的是,你很少需要关注电话系统的技术细节。但事实证明,电话网络交换机和路由器高度依赖日期。所有连接都记录在日期和时间上,这对于计算通话持续时间很关键。如果你在1999年12月31日的11:59:30打了一个一分钟电话,而在12:00:00系统读取你的通话时,认为它持续时间为负99年以上,那么错误循环和关闭是可能的。虽然长途公司正在花费大量资金升级其交换机以使其符合2000年标准,地方服务提供商也可能在这样做,但如果即使有一些较小的公司未能遵守规定而垮掉,整个网络都可能受到影响。你很幸运在2000年1月1日能听到拨号音。}
  \ParallelPar


  \ParallelLText
  {In the words of the Y2K expert Peter de Jager, “If we lose the ability to make a phone call, then we lose everything. We lose electronic fund transfers, we lose trading, we lose branch banking.” And the follow-on consequences of Y2K failures could come to more than that.}  
  \ParallelRText
  {\small 正如Y2K专家彼得·德贾格所说:“如果我们失去打电话的能力,那么我们就失去了一切。我们失去了电子资金转移,失去了交易,失去了分支银行。”}
  \ParallelPar

  \ParallelLText
  {Today, no one knows how pervasively crucial systems will crash because of the year 2000 problem. Embedded systems that cannot be reprogrammed but must be replaced if nonfunctional on a date-sensitive basis are found in cars, trucks, and buses built after 1976. (Perhaps you won't be in an accident with vehicles driven by persons with noncompliant pacemakers, because their vehicles might not start.) Embedded systems are also widespread in all types of power plants, water and sewage systems, medical devices, military equipment, aircraft, offshore oil platforms, oil tankers, alarm systems, and elevators. While many assemblies of microprocessors perform no date sensitive functions, they may nonetheless depend upon a clock, which may be Y2K sensitive, for their internal operations.   }  
  \ParallelRText
  {\small 由于Y2K问题的后续影响可能更多,因此目前没有人知道关键系统由于Y2K问题会崩溃到什么程度。不能重新编程但必须在日期敏感的基础上更换的嵌入式系统可以在1976年之后建造的汽车、卡车和公共汽车中找到。(也许你不会与搭载非符合性心脏起搏器的人开的车辆发生交通事故,因为他们的车可能无法启动。)嵌入式系统在各种发电厂、水和污水系统、医疗设备、军用设备、飞机、海上石油平台、油轮、报警系统和电梯中也很常见。虽然许多微处理器组件不执行日期敏感功能,但它们可能仍然依赖一个时钟,该时钟可能对其内部运行敏感于Y2K问题。}
  \ParallelPar

\section{主机和Y2K时间炸弹}

  \ParallelLText
  {The large-scale command and control systems of government and major corporations that involve high transaction volumes on mainframe computers were the original focus of Y2K concern. Because they operate on big machines for which most software is decades old and mostly noncompliant, the original alarms about Y2K, first sounded by Peter de Jager early in the 1990s, have focused mainly on the need to upgrade operating systems for big, multiprocessing mainframes. Mr. de Jager voiced concern that there might not be enough programmers conversant with COBOL, the old mainframe language, to complete the necessary patches and repairs to date sensitive code, even if every company and government agency with a vulnerable system had begun a crash program several years ago. Since this has not happened, and many operators of date-sensitive information systems have only just begun to assess their vulnerability, you can predict with a high degree of confidence that many mainframe systems will not be prepared to operate smoothly into the year 2000.}  
  \ParallelRText
  {\small 政府和大型企业的大规模指挥和控制系统涉及在主机计算机上进行高交易量。这些系统最初是Y2K关注的焦点,因为它们在大型设备上运行,而大多数软件已经数十年了,大部分都不兼容。最初在20世纪90年代由彼得德贾格提出的Y2K警报主要关注于需要升级大型多处理主机的操作系统。De Jager先生对于可能没有足够熟悉COBOL的程序员完成必要的补丁和修复日期敏感代码表示担忧,即使每个存在易受攻击系统的公司和政府机构几年前就开始了紧急计划。由于这种情况从未发生过,许多日期敏感信息系统的操作者直到最近才开始评估其易受攻击性,你可以有很大把握预测许多主机系统将无法平稳地运行到2000年。}
  \ParallelPar


  \ParallelLText
  {This is certainly a major concern because there is really no alternative to computer processing as the economy is now structured. Most businesses that are large enough to require a mainframe to handle their transactions are dependent upon transaction volumes that could not be managed with old fashioned nineteenth-century paperwork systems. If such businesses were forced to revert to shuffling paper they could expect to complete only a fraction of their normal transaction volume. The revenue shock from such a drop-off in business would endanger the survival of all but the most highly capitalized companies.}  
  \ParallelRText
  {\small 这肯定是一个重大问题,因为随着经济结构的变化,计算机处理确实没有替代品。大多数需要大型机来处理交易的企业都依赖于无法通过老式的19世纪的文书系统来管理的交易量。如果这些企业被迫回归手工处理文书,他们只能完成正常交易量的一小部分。这种业务量的下降将导致收入的严重损失,危及大多数只有高度资本化的公司才能存活。}
  \ParallelPar


  \ParallelLText
  {Almost everything related to money --invoicing, purchasing, and payroll systems, plus inventory controls and regulatory compliance-- would be fouled up. Huge quantities of data would be lost as computers crash or spew out false data in response to the Y2K problem. In some cases, it would actually prove a blessing if systems crash immediately rather than corrupting their data on a compounding basis until massive malfunction draws attention to the problem. What happens to files when a backup utility copies files originating on 07/04/99 to an update on 01/04/00? Who knows? Will the computer interpret a payment made on January 4, “1900,” for an insurance policy as a signal that the policy has been in default for a century, resulting in a canceled policy that is stricken from the file? Will banks and finance company computers seek to assess a hundred years of interest on loans that span the shift to the new millennium? Will your banks and brokerage firms retain accurate records of your account balances and give you timely access to your funds? These are just some of the interesting quandaries that you will confront because of the Y2K problem.}  
  \ParallelRText
  {\small 几乎与金钱有关的一切——包括发票、采购和支付系统,加上库存控制和监管合规性——都将混乱不堪。由于计算机崩溃或响应Y2K问题而产生了错误数据,大量数据将会丢失。有些情况下,如果系统立即崩溃而不是在复合的基础上破坏它们的数据,直到大规模故障引起关注,那么它实际上可能会被证明是一个好事。备份工具将07/04/99起源的文件复制到01/04/00更新时,文件会发生什么?谁知道?会不会让计算机将在1900年1月4日支付的保险费解释为一个信号,表明保单已经逾期一个世纪,导致保单被取消并从文件中删除?银行和金融公司的计算机会尝试在跨越新千年的贷款上评估一百年的利息吗?您的银行和经纪公司会保留准确的账户余额信息并及时为您提供资金访问吗?这些只是由于Y2K问题而面临的一些有趣的困境。}
  \ParallelPar


  \ParallelLText
  {Also high on your list of concerns should be what happens if the electricity goes off because of Y2K-related malfunctions. Without electricity, even most systems that are not Y2K-impaired will not function: your refrigerator, your freezer, perhaps even your source of heat. Y2K compliance issues could effect safety-related access and control functions at nuclear power plants. For example, personnel at nuclear facilities wear dosimetry devices that measure the amount of radiation exposure they receive while in the plant. These devices are analyzed regularly, with the data on exposure amounts maintained on a computer system that controls personnel access to the facility. Obviously, if the controlling computers fail, they will make a hash of all the elaborate controls designed to insure safe operation and guarantee proper maintenance. But, more importantly, a Nuclear Regulatory Commission memo notes that many “non-safety-related, but important computer-based systems, primarily databases and data collection necessary for plant operations,” are date sensitive.}  
  \ParallelRText
  {\small 此外,你最担心的事情之一应该是,如果由于Y2K相关的故障而导致电力中断会发生什么。没有电力,即使是没有Y2K问题的大多数系统也将无法运作:你的冰箱、冷冻机,甚至你的供暖系统。Y2K合规问题可能会影响核电站的与安全有关的访问和控制功能。例如,核设施的人员佩戴能够测量工作期间接受的辐射暴露量的剂量计。这些设备经常进行分析,暴露量的数据保存在控制人员进入该设施的计算机系统上。很明显,如果控制计算机出现故障,将破坏所有旨在确保安全操作和保证正确维护的复杂控制。但更重要的是,核能监管委员会备忘录指出,许多“不涉及安全但重要的基于计算机的系统,主要是必须用于电厂运营的数据库和数据收集”,对日期敏感。}
  \ParallelPar


  \ParallelLText
  {The conventional generating plants are not less vulnerable to Y2K disruption. For one thing, coal-powered plants are susceptible to disruptions in the surface transportation system that brings the coal to the boilers. In the 1997-1998 winter heating season, operators of coal-fired electricity generation found themselves forced to reduce output in some instances because of a slowdown in rail deliveries of Western coal arising from the merger of the Southern Pacific and Union Pacific railway systems. The problem arose because of incompatibilities between the computer control and dispatch systems employed by the two railroads. According to a Union Pacific spokesman, integrating the two systems became a “nightmare,” in spite of the fact that Union Pacific Technologies has been considered an industry leader in developing computerized transportation control systems. As a result of the programming difficulties, the railroad was unable to accurately track the movements of its freight cars. The failure of Union Pacific to master the assimilation of Southern Pacific is a bad omen about what could happen when Y2K logic time bombs disrupt transportation, power generation, and other aspects of the economy.}  
  \ParallelRText
  {\small 常规的发电厂同样容易受到Y2K的干扰。首先,燃煤发电厂容易受到表面运输系统中运煤的干扰。在1997-1998年的供暖季节中,一些燃煤发电运营商不得不减少产量,因为由于南太平洋和联合太平洋铁路系统合并而导致的西部煤炭铁路交货减缓。这个问题的出现是由于这两家铁路公司所采用的计算机控制和调度系统之间的不兼容。据联合太平洋一位发言人称,将两个系统整合起来变成了“噩梦”,尽管联合太平洋技术一直被认为是开发计算机化运输控制系统的行业领导者。由于编程困难,该铁路无法准确地跟踪其货车的运动。联合太平洋未能掌握统合南太平洋的失败是一个不祥之兆,表明逻辑时钟炸弹干扰了运输、发电和其他经济方面的后果。}
  \ParallelPar


  \ParallelLText
  {The biggest worry about the electric grid, however, arises from the fact that the whole system is subject to sensitive monitoring and computer control to transfer electricity from areas of surplus generation to those with a deficit. This process must be carefully monitored by computer to prevent power surges and system failures. All the transfers of electricity are logged to time and date for duration, much like a telephone connection. While heavy-duty mechanical relays are used to make the connections, they are controlled by computer systems. These computer controls, essential for load balancing, may fail for the same reasons as the phone networks. In fact, the power load distribution-control systems in North America are networked together through T-1 lines and telephone microwave links. So if the phone network fails, you can expect the electricity to go down as well. And remember, as the experience in Canada in January 1998 confirms, once the electricity shuts down over a wide area, getting the system running again is a challenge. A blackout may last for an inconveniently long time.  }  
  \ParallelRText
  {\small 然而,关于电网的最大担忧,是整个系统都受监控和计算机控制的敏感性影响,以将电力从盈余发电区域传输到不足区域。必须通过计算机仔细监控此过程,以防止电力过载和系统故障。所有电力转移都记录了时间和日期的持续时间,就像电话连接一样。虽然使用重型机械继电器进行连接,但它们由计算机系统控制。这些计算机控制对于负载平衡至关重要,可能因与电话网络相同的原因而失败。事实上,北美的电力负载分配控制系统通过T-1线和电话微波链接在一起。因此,如果电话网络出现故障,您可以预期电力也会停止供应。并且记住,正如加拿大在1998年1月的经验所证实的那样,一旦大范围停电,再使系统运行起来就是一个挑战。停电可能会持续很长时间。}
  \ParallelPar


\section{千年虫问题和核武库}

  \ParallelLText
  {For modern economies to have the electricity turn off in the dead of winter would be disruptive and potentially health threatening, especially for those who depend upon electric heat and medical equipment. Yet the worst case scenario is even worse. According to John Koskinen, who heads President Clinton's Y2K Conversion Council, U.S. military arsenals may cease to function on the stroke of midnight, December 31, 1999. While indicating that he does not wish to touch off undue alarm, Koskinen adds, “It needs to be worried about.” One concern about nuclear missiles “is if the data doesn't function and they actually go off.”}  
  \ParallelRText
  {\small 对于现代经济而言,在寒冬季节断电将会对依赖于电热和医疗设备的人们造成一定的破坏和健康威胁。然而最糟糕的情况更糟。根据约翰·科斯基嫩(John Koskinen)的说法,他是克林顿总统Y2K Conversion Council的负责人,美国军队的武器库可能会在1999年12月31日午夜时刻停止运作。虽然表示他不希望引起不必要的恐慌,但科斯基嫩补充说:“这是应该引起担忧的。” 关于核导弹的一个担忧是“如果数据不正常,它们可能发射。”}
  \ParallelPar


  \ParallelLText
  {Of course, this concern would apply with equal or greater force to Russian nuclear missiles. Russia's bankruptcy has made upgrades for Y2K compliance even more problematic than in the United States. And there is evidence that Russia is not yet taking Y2K conversion seriously. While one would pray that no accidental launches would occur, there should be little doubt that the turn of the year 2000 has a potential for aggravating global insecurity if for no other reason than that military communications systems in many countries may not function normally. As Koskinen puts it, “If you're sitting in a country and suddenly you can't quite figure out exactly what's happening, and your communications don't work as well, you get even more nervous.” So put that on your list of Y2K worries. The logic time bomb could precipitate the launch of genuinely explosive bombs -- a fact that highlights the danger from information warfare to centralized command and control systems.}  
  \ParallelRText
  {\small 当然,这个问题同样适用于俄罗斯的核导弹,甚至可能更加严峻。俄罗斯的破产使Y2K合规性的升级更加困难。此外,有证据表明,俄罗斯并未认真对待Y2K转换。虽然希望不会发生意外发射事件,但毫无疑问,千禧年交替可能导致全球不安全因素加剧,原因是许多国家的军事通讯系统可能无法正常工作。正如科斯基嫩所说:“如果你坐在一个国家里,突然你无法确定发生了什么事情,你的通讯也没有那么顺畅,你会变得更加紧张。” 因此,请将这列入您的Y2K担忧清单中。逻辑时限炸弹可能引发真正爆炸性的炸弹发射,这一事实凸显了信息战对于集中式指挥和控制系统的危险性。}
  \ParallelPar


  \ParallelLText
  {If terrorists wish to strike any centralized system, they may pick December 31, 1999, as the date for action because it will be a time of maximum vulnerability of many systems. Not only will communications be strained at best, with the possibility that electricity may fail, vehicles may not start, police, fire, and ambulance 911 service may not work, and so on, but many other functions you probably take for granted, such as air traffic control, may cease to function. No power means no water from the tap. Sewage systems would fail. Traffic lights could turn off. Within a few hours of a genuine breakdown in the transportation system, food in grocery stores would be shopped out. (Or looted.) On the basis of recent experience in American cities, you could suppose that no power, no water, no heat for many, no light, and fragmented communications with emergency services, including police and fire, all add up to no civilization. While no one can be sure what the impact of the Y2K problem may be, it could extend to looting and rioting in the streets, especially if it becomes known that there could be widespread failures to issue payroll, welfare and pension checks. }  
  \ParallelRText
  {\small 如果恐怖分子想要攻击任何集中式系统,他们可能会选择1999年12月31日作为行动时间,因为这将是许多系统最易受攻击的时候。不仅通讯最多只能保持紧绷状态,有停电的可能,车辆可能无法启动,警察、消防和救护911服务可能不起作用等,而且许多其他您可能认为理所当然的功能,如空中交通管制,可能会停止运作。没有电力就意味着自来水也没有了。污水系统将会失灵。交通信号灯可能会熄灭。在交通系统真正崩溃的几个小时内,超市里的食品就会被全部抢购(或掠夺)。根据美国城市最近的经验,如果没有电力、没有自来水、没有暖气、没有光线和紊乱的紧急服务通讯(包括警察和消防),所有这些因素加起来就意味着没有文明。虽然谁也不能确定Y2K问题的影响会是什么,但它可能会延伸到街头抢劫和暴动,特别是如果人们知道可能会有大量的工资、福利和养老金支票发放失败。}
  \ParallelPar


  \ParallelLText
  {Premonitions of doom about the new millennium do not necessarily rest upon theology tied to the Christian faith, but they do fit within the millennial tradition of Joachim de Fiore whose mediations convinced him that Christ was only “the second hinge of history” and that another was destined to unfold. “21 So argues the philosopher Michael Grosso, who suggests that the Information Revolution is piloting human history toward the realization of the prophetic vision of the Western world. He calls this “technocalypse.” Whether or not the development of technology is somehow informed by millennial visions, the Y2K phenomenon is an artifact of the predominant Western imagination of time. In a strange way, it could complement dreams, reveries and visions, or numerical interpretations of visions, like Newton's gloss on the prophecies of Daniel. These intuitive leaps begin with a perspective that takes the birth of Christ to be the central fact of history. They are compounded by the psychological power of large round numbers, which every trader will recognize as having an arresting quality. The two thousandth year of our epoch cannot help but become a focus for the imagination of intuitive people.}  
  \ParallelRText
  {\small 对于新千年的末日预言并不一定建立在基督教信仰上的神学基础,但它符合Joachim de Fiore的千年传统的思想。他的思考使他确信,基督仅是“历史的第二个转轴”,另一个转轴正注定要展开。所以,哲学家Michael Grosso认为,信息革命正在引领人类历史走向西方世界预言的实现,他称之为“technocalypse”。无论技术的发展是否与千禧年异象有所联系,Y2K现象都是主导西方时间想象的产物。以奇怪的方式,它都能够与梦想、沉思和幻想相辅相成,或像牛顿对但以理预言的诠释那样,对异象进行数字解读。这些直觉跳跃始于一个将基督的诞生视为历史中心事件的角度。它们被大而圆的数字的心理力量所弥漫,任何商人都会认识到这种数字具有引人注目的特性。我们时代的第二千年必须成为具有直觉力人们想象的焦点所在。}
  \ParallelPar


  \ParallelLText
  {A critic could easily make these premonitions seem silly, without even addressing the ambiguous and debatable theological notions of the Apocalypse and the Last Judgment that give these visions so much of their power. Interestingly, however, the Y2K computer glitch trumps the errors of arithmetic that otherwise might seem to devalue the importance of the year 2000 even within the Christian framework. The year 2000 has the potential to become an inflection point for the next stage of history simply because it brings forward the arrival of the new millennium. In strict logic, the next millennium will not begin until 2001. The year 2000 will be only the last year of the twentieth century, the two thousandth year since Christ's birth. Or it would be had Christ been born in the first year of the Christian era. He was not. In 533, when Christ's birth replaced the founding date of Rome as the basis for calculating years according to the Western calendar, the monks who introduced the new convention miscalculated Christ's birth. It is now accepted that he was born in 4 B.C. On that basis, a full two thousand years since his birth were completed sometime in 1997. Hence Carl Jung's apparently odd launch date for the start of a New Age. }  
  \ParallelRText
  {\small 批评家可能轻易地使这些预言看起来很傻,甚至不考虑启示录和末日审判的模糊和有争议的神学概念,这些幻象赋予了这些幻象如此强大的力量。然而有趣的是,Y2K电脑故障可以挑战算术错误,否则可能会在基督教框架内贬低2000年的重要性。2000年有可能成为历史下一阶段的拐点,仅仅因为它提前了新千年的到来。严格来说,下一个千年将不会开始直到2001年。2000年只是二十世纪的最后一年,也是自基督诞生以来的第二千年。如果基督在基督教纪元的第一年出生,那么它将是。但事实并非如此。533年,当基督的诞生取代罗马建立日期作为根据西方日历计算年份的基础时,引入新约定的修道士们计算出了基督的诞生年份。现在认为他在公元前4年出生。基于这一点,从他的诞生到现在已经过去了整整两千年,在1997年的某个时候完成了。因此,卡尔·荣格对新时代开始日期的明显奇怪的启动日期。}
  \ParallelPar


  \ParallelLText
  {Giggle if you will, but we do not despise or dismiss intuitive understandings of history. Although our argument is grounded in logic, not in reveries, we are awed by the prophetic power of human consciousness. Time after time, it redeems the visions of madmen, psychics, and saints. So it may be with the transformation of the year 2000. The date that has long been fixed in the imagination of the West looks to be the inflection point that at least half confirms that history has a destiny. We cannot explain why this should be, but nonetheless we are convinced that it is so.  }  
  \ParallelRText
  {\small 如果你愿意笑,但我们不蔑视或否认对历史的直觉理解。虽然我们的论点基于逻辑而不是幻想,但我们对人类意识的先知力量感到惊叹。一次又一次地,它弥合了疯子、通灵者和圣徒的幻象。关于2000年的转型可能也是如此。在西方想象中早已确定的日期似乎是转折点,至少半数证明了历史有一个命运。我们无法解释为什么会这样,但我们仍然相信它是这样的。}
  \ParallelPar


  \ParallelLText
  {Our intuition is that history has a destiny, and that free will and determinism are two versions of the same phenomenon. The human interactions that form history behave as though they were informed by a kind of destiny. Just as an electron plasma, a dense gas of electrons, behaves as a complex system, so do human beings. The freedom of individual movement by the electrons turns out to be compatible with highly organized collective behavior. As David Bohm said of an electron plasma, human history is “a highly organized system which behaves as a whole.”}  
  \ParallelRText
  {\small 我们的直觉是历史有一个命运,自由意志和决定论是同一个现象的两个版本。形成历史的人类互动表现得好像它们受到某种命运的启示。正如一个电子等离子,一个电子的密集气体的行为是一个复杂的系统,人类也是如此。电子个体的运动自由证明与高度组织的集体行为是兼容的。正如David Bohm所说的电子等离子体,人类历史是“一个高度组织的系统,表现为整体”。}
  \ParallelPar


  \ParallelLText
  {Understanding the way the world works means developing a realistic intuition of the way that human society obeys the mathematics of natural processes. Reality is nonlinear. But most people's expectations are not. To understand the dynamics of change, you have to recognize that human society, like other complex systems in nature, is characterized by cycles and discontinuities. That means certain features of history have a tendency to repeat themselves, and the most important changes, when they occur, may be abrupt rather than gradual. }  
  \ParallelRText
  {\small 了解世界的运作方式意味着发展对人类社会服从自然过程数学规律的现实直觉。现实是非线性的,但大多数人的期望并非如此。为了理解变化的动力学,您必须认识到人类社会像其他自然复杂系统一样具有周期性和不连续性。这意味着历史的某些特征有循环复发的倾向,并且最重要的变化可能是突然而非渐进式的。}
  \ParallelPar


  \ParallelLText
  {Among the cycles that permeate human life, a mysterious five-hundred-year cycle appears to mark major turning points in the history of Western civilization. As the year 2000 approaches, we are haunted by the strange fact that the final decade of each century divisible by five has marked a profound transition in Western civilization, a pattern of death and rebirth that marks new phases of social organization in much the way that death and birth delineate the cycle of human generations. This has been true since at least 500 B.C. when Greek democracy emerged with the constitutional reforms of Cleisthenes in 508 B.C. The following five centuries were a period of growth and intensification of the ancient economy, culminating in the birth of Christ in 4 B.C. This was also the time of the greatest prosperity of the ancient economy, when interest rates reached their lowest level prior to the modern period. }  
  \ParallelRText
  {\small 在渗透人类生活的周期中,一种神秘的五百年周期似乎标志着西方文明历史上的重大转折点。随着2000年的临近,我们受到一个奇怪的事实的困扰,即每个以五为倍数结尾的世纪的最后十年标志着西方文明的深刻转变,这种死亡和重生的模式标志着社会组织的新阶段,就像死亡和出生划分人类世代周期一样。至少从公元前500年希腊民主政治在克里斯泰尼斯的宪法改革中出现以来,这种情况就已经成立。接下来的五个世纪是古代经济增长和强化的时期,最终在公元前4年的基督诞生中达到高峰。这也是古代经济最繁荣的时期,利率在现代期前达到最低水平。}
  \ParallelPar

  \ParallelLText
  {The next five centuries saw a gradual winding down of prosperity, leading to the collapse of the Roman Empire late in the fifth century A.D. William Playfair's summary is worth repeating: “When Rome was at its highest pitch of greatness ... will be seen to be at the birth of Christ, that is, during the reign of Augustus, and by the same means it will be found declining gradually till the year 490.” It was then that the last legions dissolved, and the Western world sank into the Dark Ages. }  
  \ParallelRText
  {\small 接下来的五个世纪见证了繁荣逐渐走向尽头,导致罗马帝国在公元五世纪末崩溃。威廉·普莱费尔的总结值得重复:“当罗马处于最高峰时……就是在基督诞生时期,即在奥古斯都统治期间,随着同样的手段会逐渐衰落,直到公元490年。”就在那时,最后一批军队解散,西方世界沉入黑暗时代。}
  \ParallelPar

  \ParallelLText
  {During the following five centuries, the economy withered, long-distance trade ground to a halt, cities were depopulated, money vanished from circulation, and art and literacy almost disappeared. The disappearance of effective law with the collapse of the Roman Empire in the West led to the emergence of more primitive arrangements for settling disputes. The blood feud began to be significant at the end of the fifth century. The first recorded incident of trial by ordeal occurred precisely in the year 500.  }  
  \ParallelRText
  {\small 在接下来五个世纪里,经济萎缩,远程贸易停滞,城市人口减少,货币从流通中消失,艺术和文化也几乎消失。随着西罗马帝国的崩溃,有效法律的消失导致更为原始的纠纷解决安排的出现。血仇开始在公元五世纪末变得重要,首次记录的禁锢审判事件恰好发生在公元500年。}
  \ParallelPar

  \ParallelLText
  {Once again, a thousand years ago, the final decade of the tenth century witnessed another “tremendous upheaval in social and economic systems.” Perhaps the least known of these transitions, the feudal revolution, began at a time of utter economic and political turmoil. In \emph{The Transformation of the Year One Thousand}, Guy Bois, a professor of medieval history at the University of Paris, claims that this rupture at the end of the tenth century involved the complete collapse of the remnants of ancient institutions, and the emergence of something new out of the anarchy--feudalism. In the words of Raoul Glaber, “It was said that the whole world, with one accord, shook off the tatters of antiquity.” The new system that suddenly emerged accommodated the slow revival of economic growth. The five centuries now known as the Middle Ages saw a rebirth of money and international trade, along with the rediscovery of arithmetic, literacy, and time awareness. }  
  \ParallelRText
  {\small 再一次,一千年前,十世纪的最后十年见证了另一场“社会和经济体制的巨大动荡”。也许,最少为人所知的这些转变之一,封建革命,开始于完全的经济和政治动荡时期。在《一千年的转变》中,Guy Bois教授是巴黎大学的一名中世纪历史教授,他声称,在十世纪末这一断裂涉及古老制度的残留完全崩溃,新的一些东西由混乱之中诞生 -- 封建主义。用Raoul Glaber的话来说,“据说整个世界全都一齐摇掉了古代的破布。”新的系统突然出现,容纳了经济增长的缓慢复苏。现在被称为中世纪的五个世纪见证了货币和国际贸易的复兴,以及算术,读写能力和时间意识的重新发现。}
  \ParallelPar


  \ParallelLText
  {Then, in the final decade of the fifteenth century, there was yet another turning point. It was then that Europe emerged from the demographic deficit caused by the Black Death and almost immediately began to assert dominion over the rest of the globe. The “Gunpowder Revolution,” the “Renaissance,” and the “Reformation” are names given to different aspects of this transition that ushered in the Modern Age. It was announced with a bang when Charles VIII invaded Italy with new bronze cannon. It involved an opening to the world, epitomized by Columbus sailing to America in 1492. This opening to the New World launched a push toward the most dramatic economic growth in the experience of humanity. It involved a transformation of physics and astronomy that led to the creation of modem science. And its ideas were disseminated widely with the new technology of the printing press. }  
  \ParallelRText
  {\small 在15世纪的最后十年,又到了一个转折点。这时欧洲从黑死病造成的人口赤字中走出来,几乎立即开始对全球施加统治。“火药革命”、“文艺复兴”和“宗教改革”是这个转变的不同方面所命名的。当查理八世率军入侵意大利带着新铜炮时,它就以轰鸣声宣告了来临;它涉及了一个打开世界的过程,以哥伦布1492年航行到美洲为代表。这种向新世界的开放启动了人类经历中最为激动人心的经济增长。它涉及了对物理学和天文学的转变,导致了现代科学的诞生。它的思想以新的印刷技术广泛传播。}
  \ParallelPar


  \ParallelLText
  {Now we sit at the threshold of another millennial transformation. The large command and control systems inherited from the Industrial Era may break down like the one-horse shay on the stroke of the millennial midnight. Yet whether or not the Y2K logic bomb precipitates an immediate collapse of industrial society, its days are numbered. We expect the advent of the Information Society to utterly transform the world, in ways that this book is meant to explain. You would be perfectly within your rights to doubt this, since no cycle that repeats itself only twice in a millennium has demonstrated enough iterations to be statistically significant. Indeed, even much shorter cycles have been viewed skeptically by economists demanding more statistically satisfying proof. “Professor Dennis Robertson once wrote that we had better wait a few centuries before being sure” about the existence of four-year and eight- to ten-year trade cycles. By that standard, Professor Robertson would have to suspend judgment for about thirty thousand years to be sure that the five-hundred-year cycle is not a statistical fluke. We are less dogmatic, or more willing to take a hint. We recognize that the patterns of reality are more complex than the static- and linear-equilibrium models of most economists.  }  
  \ParallelRText
  {\small 现在我们正处于另一个千年转型的门槛上,从工业时代继承而来的大型指挥和控制系统可能会在千年之夜的一举手之劳间崩溃。然而,无论Y2K逻辑炸弹是否会导致工业社会的立即崩溃,其日子都已经被数了。我们期待信息社会的出现将彻底改变这个世界,而这本书就是为了解释这些变化。您有权怀疑这一点,因为在一个千年里只重复两次的周期还没有展示足够多的重复次数来证明其统计学意义。事实上,即使是更短的周期,也被经济学家持怀疑态度,要求更多的统计学证明。“丹尼斯·罗伯逊教授曾经写道,我们最好再等几个世纪才能确定”四年和八到十年的贸易周期是否存在。按照这一标准,罗伯逊教授将不得不暂缓判断大约三万年,以确定五百年周期不是一种统计学偶然。我们不那么教条,或者更愿意接受暗示。我们认识到现实模式比大多数经济学家的静态和线性平衡模型更为复杂}
  \ParallelPar

  \ParallelLText
  {We believe that the coming of the year 2000 marks more than another convenient division along an endless continuum of time. We believe it will be an inflection point between the Old World and a New World to come. The Industrial Age is rapidly passing, and its demise may, ironically, be accelerated by the fact that early computer memory was so expensive that it encouraged the widespread adoption of two-digit date fields. When Hallerith punch cards could accommodate only eighty characters each, abbreviating dates seemed a prudent thing to do. Contrary to the expectations of the early programmers, however, their abbreviation of the date field endured four decades until the end of the millennium as an accidental logic bomb that could destroy a large part of industrials society. The U.S. government's Office of Management and Budget described the problem in “Getting Federal Computers Ready for 2000,” a report dated February 7, 1997. The OMB concludes of computers: “Unless they are fixed or replaced, they will fail at the tum of the century in one of three ways: they will reject legitimate entries, or they will compute erroneous results, or they simply will not run.” These three outcomes in combination could cripple Industrial society. Its technology of mass production is destined to be eclipsed by a new technology of miniaturization in any event. A near-term crisis will merely accelerate the process. With the new information technology has come a new science of nonlinear dynamics, one whose startling conclusions are mere strands that have yet to be woven together into a comprehensive worldview. We live in the time of the computer, but our dreams are still spun on the loom. We continue to live by the metaphors and thoughts of industrialism. We don't yet imagine the world in terms of strange attractors. Our politics still straddles the industrial divide between right and left, as mapped by thinkers like Adam Smith and Karl Marx, who died before almost everyone now living was born. The industrial worldview, incorporating the operating principles of industrial science, is still the “commonsense” intuition of educated opinion. It is our thesis that the “common sense” of the Industrial Age will no longer apply to many areas as the world is transformed.}  
  \ParallelRText
  {\small 我们相信,2000年的到来标志着旧世界与即将到来的新世界之间的一种拐点。工业时代正在迅速过去,而它的终结可能会被加速,这是具有讽刺意味的,因为早期计算机内存非常昂贵,这促使广泛采用两位数字日期字段。当哈勃里斯打孔卡只能容纳80个字符时,缩写日期似乎是一件明智的事。然而,与早期程序员的预期相反,他们对日期字段的缩写持续了四十年,直到千年结束,作为一个意外的逻辑炸弹,可能摧毁工业社会的大部分。美国政府的管理和预算办公室在1997年2月7日的一份报告中描述了这个问题。OMB针对计算机的结论是:“除非修复或更换,否则它们将在世纪之交以以下三种方式之一失败: 它们将拒绝合法条目,或者它们将计算错误的结果,或者它们根本无法运行。” 这三个结果的组合可能会瘫痪工业社会。无论如何,其大规模生产的技术注定会被一种微型化的新技术所超越。近期危机只会加速这一过程。随着新信息技术的出现,一个新的非线性动力学科学应运而生,其惊人的结论只是尚未被织在一起形成全面的世界观。我们生活在计算机的时代,但我们的梦想仍然在纺织机上编织。我们继续按照工业主义的隐喻和思想进行生活。我们还没有想象以奇怪的吸引子而非工业主义的方式望去的世界。我们的政治仍然跨越着工业分化,就像亚当·斯密和卡尔·马克思这样的思想家绘制的那样,在右翼和左翼之间。这些思想家在几乎所有现在活着的人出生之前就去世了。工业世界观,融合了工业科学的操作原则,仍然是受过教育的观点的“常识”直觉。我们的论点是,工业时代的“常识”在许多领域将不再适用于随着世界的转变。}
  \ParallelPar

  \ParallelLText
  {More than eighty-five years after the day in 1911 when Oswald Spengler was seized with an intuition of a coming world war and “the decline of the West,” we, too, see “a historical change of phase occurring ... at the point preordained for it hundreds of years ago.” Like Spengler, we see the impending death of Western civilization, and with it the collapse of the world order that has predominated these past five centuries, ever since Columbus sailed west to open contact with the New World. Yet unlike Spengler we see the birth of a new stage in Western civilization in the coming millennium.   }  
  \ParallelRText
  {\small 在1911年,奥斯瓦尔德·施宾格被一种即将来临的世界大战和“西方文明的衰落”的直觉抓住,85年后的今天,我们也看到“历史性转变的阶段正在发生……在数百年前就预定的时刻”。像施宾格一样,我们看到西方文明的迫在眉睫的死亡,随之而来的是这过去五个世纪以来主导的世界秩序的崩溃,自哥伦布西航以来。然而,与施宾格不同,我们看到在新千年即将出现西方文明的新阶段。}
  \ParallelPar

\end{Parallel}